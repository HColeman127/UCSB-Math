\documentclass[12pt]{article}

% Packages
\usepackage[margin=1in]{geometry}
\usepackage{fancyhdr}
\usepackage{amsmath, amsthm, amssymb}

% Page Style
\fancypagestyle{plain}{
    \fancyhf{}
    \renewcommand{\headrulewidth}{0pt}
    \renewcommand{\footrulewidth}{0pt}
    \fancyfoot[R]{\thepage}
}
\pagestyle{plain}

% Problem Box
\setlength{\fboxsep}{4pt}
\newsavebox{\savefullbox}
\newenvironment{fullbox}{\begin{lrbox}{\savefullbox}\begin{minipage}{\dimexpr\textwidth-2\fboxsep\relax}}{\end{minipage}\end{lrbox}\begin{center}\framebox[\textwidth]{\usebox{\savefullbox}}\end{center}}
\newenvironment{pbox}[1][]{\begin{fullbox}\ifx#1\empty\else\paragraph{#1}\fi}{\end{fullbox}}

% Options
\renewcommand{\thesubsection}{\thesection(\alph{subsection})}
\allowdisplaybreaks
\addtolength{\jot}{4pt}
\theoremstyle{definition}

% Default Commands
\newtheorem{proposition}{Proposition}
\newtheorem{lemma}{Lemma}
\newcommand{\ds}{\displaystyle}
\newcommand{\isp}[1]{\quad\text{#1}\quad}
\newcommand{\N}{\mathbb{N}}
\newcommand{\Z}{\mathbb{Z}}
\newcommand{\Q}{\mathbb{Q}}
\newcommand{\R}{\mathbb{R}}
\newcommand{\C}{\mathbb{C}}
\newcommand{\eps}{\varepsilon}
\renewcommand{\phi}{\varphi}
\renewcommand{\emptyset}{\varnothing}
\newcommand{\pfrac}[2]{\left(\frac{#1}{#2}\right)}

% Extra Commands
\DeclareMathOperator{\id}{id}
\DeclareMathOperator{\Aut}{Aut}

\newcommand{\isom}{\cong}
\newcommand{\clo}{\overline}
\newcommand{\eqc}{\overline}

\newcommand{\F}{\mathbb{F}}

\newcommand{\divides}{\bigm|}
\newcommand{\ndivides}{%
  \mathrel{\mkern.5mu % small adjustment
    % superimpose \nmid to \big|
    \ooalign{\hidewidth$\big|$\hidewidth\cr$\nmid$\cr}%
  }%
}


% Document Info
\fancypagestyle{title}{
    \renewcommand{\headrulewidth}{0.4pt}
    \setlength{\headheight}{15pt}
    \fancyhead[R]{Harry Coleman}
    \fancyhead[L]{MATH 111C Midterm}
    \fancyhead[C]{April 29, 2021}
}

% Begin Document
\begin{document}
\thispagestyle{title}


\section*{1}

Since the roots of the polynomial are $\pm\sqrt{2}, \pm\sqrt{10}$, then the splitting field is given by $\Q(\sqrt{2}, \sqrt{10})$. Clearly, $\Q(\sqrt{2} + \sqrt{5})$ is a subset of the splitting field, we will show the opposite inclusion. Since the former contains $\sqrt{2} + \sqrt{5}$, it also contains
\[
    (\sqrt{2} + \sqrt{5})^2 = 7 + \sqrt{10}
\]
and $-7 \in \Q$, so $\sqrt{10} \in \Q(\sqrt{2} + \sqrt{5})$. Then
\[
    (\sqrt{2} + \sqrt{5})\sqrt{10} = 2\sqrt{5} + 5\sqrt{2} \in \Q(\sqrt{2} + \sqrt{5}),
\]
and subtracting $2(\sqrt{2} + \sqrt{5})$, we obtain $3\sqrt{2} \in \Q(\sqrt{2} + \sqrt{5})$. And since $1/3 \in \Q$, then in fact $\sqrt{2} \in \Q(\sqrt{2} + \sqrt{5})$. Hence, the splitting field $\Q(\sqrt{2}, \sqrt{10})$ is precisely $\Q(\sqrt{2} + \sqrt{5})$.


\newpage
\section*{2}


Given that $K$ is the splitting field for $f(x)$, then
\[
    f(x) = a(x - \alpha_1) \cdots (x - \alpha_n)
\]
for some $a \in F^\times$ and $\alpha_1, \dots, \alpha_n \in K$. Let $S = \{\alpha_1, \dots, \alpha_n\}$, then we know $K = F(S)$. Given any $\alpha \in S$, we know $\alpha$ is a root of $f(x)$, so
\[
    m_{\alpha, F}(x) \divides f(x).
\]
Since $\gcd(f(x), f'(x)) = 1$, then $f(x)$ is separable, so $m_{\alpha, F}(x)$ must also be separable (otherwise, $m_{\alpha, F}(x) \divides f(x)$ would imply $f(x)$ has multiple roots). That is, $\alpha$ is separable over $F$, so every element of $S$ is separable over $F$. And since $K = F(S)$, then $K/F$ is separable.


\newpage
\section*{3}

The proof is symmetric with respect to $\alpha$ and $\beta$; it suffices to show one direction. Suppose $m_{\alpha, F}(x)$ is irreducible over $(F(\beta))[x]$. We have 
\[
    [F(\alpha, \beta) : F] = [(F(\beta))(\alpha) : F(\beta)][F(\beta) : F],
\]
where $[F(\beta) : F] = \deg m_{\beta, F}(x)$. Since $m_{\alpha, F}(x) \in (F(\beta))[x]$ is monic, irreducible  and has $\alpha$ as a root, then in fact, it is the minimal polynomial of $\alpha$ in $(F(\beta))[x]$. Therefore, we have
\[
    [F(\alpha, \beta) : F] = (\deg m_{\alpha, F}(x))(\deg m_{\beta, F}(x)).
\]

On the other hand, we have
\[
    [F(\alpha, \beta) : F] = [(F(\alpha))(\beta) : F(\alpha)][F(\alpha) : F],
\]
where $[F(\alpha) : F] = \deg m_{\alpha, F}(x)$. Therefore, we have
\[
    [(F(\alpha))(\beta) : F(\alpha)] = \deg m_{\beta, F}(x).
\]
Now since $m_{\beta, F}(x) \in (F(\alpha))[x]$ has $\beta$ as a root and has the same degree as the extension $(F(\alpha))(\beta)/F(\alpha)$, then we must have $m_{\beta, F}(x)$ irreducible in $(F(\alpha))[x]$.


\newpage
\section*{4}

\subsection*{(a)}

Since $\theta_1 \in \clo{\F_p}$, then $\F_p(\theta_1)$ is a finite subfield of $\clo{\F_p}$. Then the minimal polynomial of $\theta_1$ over $\F_p$ divides $f(x)$. Since $f(x)$ is irreducible, then $m_{\theta_1, \F_p}(x) = a^{-1}f(x)$, where $a \in \F_p^\times$ is the leading coefficient of $f(x)$. In particular,
\[
    [\F_p(\theta_1) : \F_p] = \deg m_{\theta_1, \F_p}(x) = \deg f(x).
\]
By the same argument.
\[
    [\F_p(\theta_2) : \F_p] = \deg m_{\theta_2, \F_p}(x) = \deg f(x).
\]
This implies that $|\F_p(\theta_1)| = p^{\deg f(x)} = |\F_p(\theta_2)|$. Since both $F_p(\theta_1)$ and $\F_p(\theta_2)$ are subfields of $\clo{\F_p}$ with the same cardinality, then they must be precisely the same subfield, i.e., $\F_p(\theta_1) = \F_p(\theta_2)$.


\subsection*{(b)}

Since $K \subseteq \clo{\F_p}$ is the splitting field for $f(x)$ over $\F_p$, then $K = \F_p(S)$, where $S \subseteq \clo{\F_p}$ is the set of roots of $f(x)$ in $\clo{\F_p}$. For any pair $\theta_1, \theta_2 \in S$, from $(a)$, we know that $\F_p(\theta_1) = \F_p(\theta_2)$. Moreover, this means
\[
    \F_p(\theta_1, \theta_2) = \F_p(\theta_1).
\]
Continuing inductively, suppose $\F_p(\theta_1, \dots, \theta_{n - 1}) = \F_p(\theta_1)$, for roots $\theta_1, \dots, \theta_n \in S$. Then 
\[
    \F_p(\theta_1, \dots, \theta_n)
        = (\F_p(\theta_1, \dots, \theta_{n - 1}))(\theta_n)
        = \F_p(\theta_1, \theta_n)
        = \F_p(\theta_1).
\]
Since $S$ has only finitely many roots, this induction shows that $K = \F_p(S) = \F_p(\theta)$ for any root $\theta \in S$. Then $m_{\theta, \F_p}(x) = a^{-1}f(x)$, where $a \in \F_p^\times$ is the leading coefficient of $f(x)$ (since $a^{-1}f(x) \in \F_p[x]$ monic, irreducible, and has $\theta$ as root), so
\[
    [K : \F_p] = [\F_p(\theta) : F] = \deg m_{\theta, \F_p}(x) = \deg f(x).
\]



\newpage
\section*{5}


\subsection*{(a)}

By definition, $E \subseteq K$. Clearly, each $\alpha \in F$ is separable over $F$ since $m_{\alpha, F}(x) = x - \alpha \in F[x]$ has only $\alpha$ as a simple root. Therefore, as sets, $F \subseteq E \subseteq K$.

We now show $E$ is a field. Since $F \subseteq E$, then in particular, $0, 1 \in E$. If $\alpha, \beta \in E$, i.e., $\alpha, \beta$ are separable over $F$, then $F(\alpha, \beta)/F$ is a separable field extension of $F$. Since $F(\alpha, \beta)$ is a field containing $\alpha$ and $\beta$, then we know $\alpha - \beta, \alpha^{-1}\beta \in F(\alpha, \beta)$. Since $F(\alpha, \beta)$ is separable over $F$, then both $\alpha - \beta$ and $\alpha^{-1}\beta$ are separable over $F$. That is, both are contained in $E$, proving that $E$ is a field. 


\subsection*{(b)}

Since $F$ is characteristic $p$, if $m_{\alpha, F}(x)$ is inseparable, we have shown its derivative will be identically zero. This means that all the powers of $x$ are multiples of $p$, so we can write it as a polynomial in $x^p$, say $m_{\alpha, F}(x) = f(x^p)$. Then either $f(x)$ is separable or its inseparable, if it is inseparable, we repeat the process until we obtain a separable function $g(x)$ such that $g(x^{p^m}) = m_{\alpha, F}(x)$. Then $\alpha^{p^m}$ is separable over $F$.

After this, it would remain to show that $n \geq m$ implies $\alpha^{p^n}$ is separable


\subsection*{(c)}

\end{document}