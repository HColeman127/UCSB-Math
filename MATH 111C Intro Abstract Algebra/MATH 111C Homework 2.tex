\documentclass[12pt]{article}

% Packages
\usepackage[margin=1in]{geometry}
\usepackage{fancyhdr}
\usepackage{amsmath, amsthm, amssymb, physics}

% Page Style
\fancypagestyle{plain}{
    \fancyhf{}
    \renewcommand{\headrulewidth}{0pt}
    \renewcommand{\footrulewidth}{0pt}
    \fancyfoot[R]{\thepage}
}
\pagestyle{plain}

% Problem Box
\setlength{\fboxsep}{4pt}
\newsavebox{\savefullbox}
\newenvironment{fullbox}{\begin{lrbox}{\savefullbox}\begin{minipage}{\dimexpr\textwidth-2\fboxsep\relax}}{\end{minipage}\end{lrbox}\begin{center}\framebox[\textwidth]{\usebox{\savefullbox}}\end{center}}
\newenvironment{pbox}[1][]{\begin{fullbox}\ifx#1\empty\else\paragraph{#1}\fi}{\end{fullbox}}

% Options
\renewcommand{\thesubsection}{\thesection(\alph{subsection})}
\allowdisplaybreaks
\addtolength{\jot}{4pt}
\theoremstyle{definition}

% Default Commands
\newtheorem{proposition}{Proposition}
\newtheorem{lemma}{Lemma}
\newcommand{\ds}{\displaystyle}
\newcommand{\isp}[1]{\quad\text{#1}\quad}
\newcommand{\N}{\mathbb{N}}
\newcommand{\Z}{\mathbb{Z}}
\newcommand{\Q}{\mathbb{Q}}
\newcommand{\R}{\mathbb{R}}
\newcommand{\C}{\mathbb{C}}
\newcommand{\eps}{\varepsilon}
\renewcommand{\phi}{\varphi}
\renewcommand{\emptyset}{\varnothing}
\newcommand{\pfrac}[2]{\left(\frac{#1}{#2}\right)}

% Extra Commands


% Document Info
\fancypagestyle{title}{
    \renewcommand{\headrulewidth}{0.4pt}
    \setlength{\headheight}{15pt}
    \fancyhead[R]{Harry Coleman}
    \fancyhead[L]{MATH 111C Homework 2}
    \fancyhead[C]{April 16, 2021}
}

% Begin Document
\begin{document}
\thispagestyle{title}


\begin{pbox}[Q1]
    Let $K/F$ be a field extension. Show that if $\alpha \in K$ is a root of $f(x) \in F[x]$ and $\deg f(x) = [F(\alpha) : F]$, then $f(x)$ is irreducible in $F[x]$.
\end{pbox}

\begin{proof}
    By assumption $f(x) \in F[x]$ and $f(\alpha) = 0$, so $\alpha$ is algebraic over $F$. Then consider the minimal polynomial $m_{\alpha, F}(x) \in F[x]$. We know that $m_{\alpha, F}(x) \mid f(x)$ and
    \[
        \deg f(x) = [F(\alpha) : F] = \deg m_{\alpha, F}(x).
    \]
    Therefore, there is some $a \in F^\times$ such that $am_{\alpha, F}(x) = f(x)$. Since the minimal polynomial of $\alpha$ is irreducible and $f(x)$ is associate to that minimal polynomial, we conclude that $f(x)$ must be irreducible.

\end{proof}



\newpage
\begin{lemma}
    If $K/F$ is a field extension and $\alpha \in K, \alpha \notin F$ with $\alpha^2 \in F$, then $[F(\alpha) : F] = 2$.
\end{lemma}


\begin{proof}
    Note that $x^2 - \alpha^2 \in F[x]$ and has $\alpha$ as a root, so $\alpha$ is algebraic over $F$. Then there is some minimal polynomial $m_{\alpha, F}(x) \in F[x]$, with $[F(\alpha) : F] = \deg m_{\alpha, F}(x) \leq 2$. If the degree of the minimal polynomial were $1$, then we would have $m_{\alpha, F}(x) = x - \alpha$, but this is not a polynomial in $F[x]$. So in fact $[F(\alpha) : F] = \deg m_{\alpha, F}(x) = 2$.

\end{proof}

\begin{pbox}[Q2 Problem 13.2.13]
    Suppose $F = \Q(\alpha_1, \alpha_1, \dots, \alpha_n)$ where $\alpha_i^2 \in \Q$ for $i = 1, 2, \dots, n$. Prove that $\sqrt[3]{2} \notin F$.
\end{pbox}


\begin{proof}
    Let $F_n = \Q(\alpha_1, \dots, \alpha_n)$. If $F_n = \Q$, then clearly $\sqrt[3]{2} \notin F_n$. We will first prove, by induction on $n$, that $[F_n : \Q]$ is a power of $2$. Without loss of generality, assume $\alpha_1 \notin \Q$ (as, otherwise, $F_1 = \Q$), then Lemma 1 gives us the base case of $[\Q(\alpha_1) : \Q] = 2$. Now assume $[F_{n-1} : \Q] = 2^k$ with $k \geq 0$. If $\alpha_n \in F_{n-1}$, then $F_n = F_{n-1}$ and the result holds, trivially. On the other hand, if $\alpha_n \notin F_{n-1}$ then by Lemma 1, $[F_n : F_{n-1}] = 2$, so
    \[
        [F_n : \Q] = [F_n : F_{n-1}][F_{n-1} : \Q] = 2 \cdot 2^k = 2^{k+1}.
    \]
    Returning to the notation $F = F_n$, we conclude that $[F : \Q]$ is a power of $2$.

    We now consider the polynomial $x^3 - 2 \in \Q[x]$. By Eisenstein's criterion for $\Z[x]$, this polynomial is irreducible in $\Q[x]$. Since it is also monic and has $\sqrt[3]{2}$ as a root, then it is the minimal polynomial for $\sqrt[3]{2}$ over $\Q$. Therefore,
    \[
        [\Q(\sqrt[3]{2}) : \Q] = \deg m_{\sqrt[3]{2}, \Q}(x) = 3.
    \]

    Suppose for contradiction that $\sqrt[3]{2} \in F$. Then $\Q(\sqrt[3]{2})$ is a subfield of $F$, giving us
    \[
        [F : \Q] = [F : \Q(\sqrt[3]{2})][\Q(\sqrt[3]{2}) : \Q] = [F : \Q(\sqrt[3]{2})] \cdot 3.
    \]
    However, this contradicts the fact that $[F : \Q]$ is a power of $2$, therefore $\sqrt[3]{2} \notin F$.

\end{proof}



\newpage
\begin{pbox}[Q3 Problem 13.2.16]
    Let $K/F$ be an algebraic extension and let $R$ be a ring contained in $K$ and containing $F$. Show that $R$ is a subfield of $K$ containing $F$.
\end{pbox}

\begin{proof}
    Since $R$ is a subring of $K$, it suffices to show that $R$ is closed under taking inverses. Let $r \in R \subseteq K$ be nonzero, so $r$ is algebraic over $F$. Then $r$ has a minimal polynomial $m_{r, F}(x) \in F[x] \subseteq R[x]$. In particular, take
    \[
        m_{r, F}(x) = x^n + a_{n-1}x^{n-1} + \cdots + a_1x + a_0,
    \]
    where $a_0, \dots, a_{n-1} \in F \subseteq R$. Note that we must have $a_0 \ne 0$, otherwise
    \[
        m_{r, F}(r) = r(r^{n-1} + a_{n-1}r^{n-2} + \cdots + a_2r + a_1) = -a_0 = 0.
    \]
    But $r \ne 0$ and $R$ is an integral domain, implying that the polynomial
    \[
        x^{n-1} + a_{n-1}x^{n-2} + \cdots + a_2x + a_1 \in F[x]
    \]
    would have $r$ as a root, but not have $m_{r, F}(x)$ as a factor. Hence, $a_0 \ne 0$ so
    \[
        r(r^{n-1} + a_{n-1}r^{n-2} + \cdots + a_2r + a_1)(-a_0^{-1}) = 1.
    \]
    Since $a_0 \in F$, then $a_0^{-1} \in F \subseteq R$. Thus,
    \[
        r^{-1} = (r^{n-1} + a_{n-1}r^{n-2} + \cdots + a_2r + a_1)(-a_0^{-1}) \in R,
    \]
    so $R$ is a subfield of $K$.

\end{proof}


\newpage
\begin{pbox}[Q4 Problem 13.4.1]
    Determine the splitting field and its degree over $\Q$ for $x^4 - 2$.
\end{pbox}

The roots of $x^4 - 2$ in $\C$ are $\pm\sqrt[4]{2}, \pm i\sqrt[4]{2}$. Both $\sqrt[4]{2}$ and $i\sqrt[4]{2}$ must be in the splitting field, implying that
\[
    (\sqrt[4]{2})^{-1} \cdot i\sqrt[4]{2} = i
\]
is in the splitting field. Since $\sqrt[4]{2}$ and $i$ generate all four roots, the splitting field is $\Q(\sqrt[4]{2}, i)$. Since  $x^4 - 2$ is irreducible in $\Q[x]$ (by Eisenstein's criterion for $\Z[x]$) and has $\sqrt[4]{2}$ as a root, then it is the minimal polynomial for $\sqrt[4]{2}$ over $\Q$, so $[\Q(\sqrt[4]{2}) : \Q] = 4$.

The minimal polynomial for $i$ over $\Q$ is $x^2 + 1$, and we claim it to be the same over $\Q(\sqrt[4]{2})$. Note that $\Q(\sqrt[4]{2}) \subseteq \R$, so for all $a \in \Q(\sqrt[4]{2})$, we have $a^2 \geq 0$. Thus, $x^1 + 1$ has no roots in $\Q(\sqrt[4]{2})$, so it is irreducible in $(\Q(\sqrt[4]{2}))[x]$. Hence,
\[
    [\Q(\sqrt[4]{2}, i) : \Q] = [\Q(\sqrt[4]{2}, i) : \Q(\sqrt[4]{2})][\Q(\sqrt[4]{2}) : \Q] = 2 \cdot 4 = 8.
\]




\begin{pbox}[Q5]
    Show that $\Q(\sqrt{3}, \sqrt{-1})$ is a splitting field of $x^{12} - 1 \in \Q[x]$ and $[\Q(\sqrt{3}, \sqrt{-1}) : \Q] = 4$.
\end{pbox}

Factoring the polynomial in $\C$, we find
\[
    x^{12} - 1 = \prod_{k = 0}^{11} (x - e^{\pi i k/6}).
\]
The roots are evenly spaced on unit circle, separated by an angle of $\pi/6$. The real and imaginary components of the roots are $0, \pm1/2, \pm\sqrt{3}, \pm 1$. In particular, the splitting field contains $\sqrt{3}$ and $i$ and all the of roots. Moreover, all the roots can be obtained through field operations on $\Q$ with $\sqrt{3}$ and $i$. Hence, the splitting field is precisely $\Q(\sqrt{3}, i)$. 

The minimal polynomial for $\sqrt{3}$ over $\Q$ is $x^2 - 3$, and the minimal polynomial for $i$ over $\Q(\sqrt{3})$ is $x^2 + 1$ (by the same argument as in Q4). Therefore,
\[
    [\Q(\sqrt{3}, i) : \Q] = [\Q(\sqrt{3}, i) : \Q(\sqrt{3})][\Q(\sqrt{3}) : \Q] = 2 \cdot 2 = 4.
\]


\end{document}