\documentclass[12pt]{article}

% Packages
\usepackage[margin=1in]{geometry}
\usepackage{fancyhdr, parskip}
\usepackage{amsmath, amsthm, amssymb}
\usepackage{tikz, tikz-cd}

% Page Style
\makeatletter
\fancypagestyle{title}{
    \renewcommand{\headrulewidth}{0.4pt}
    \setlength{\headheight}{15pt}
    \fancyhead[R]{\@author}
    \fancyhead[L]{\@title}
    \fancyhead[C]{\@date}
}
\makeatother
\renewcommand{\maketitle}{\thispagestyle{title}}
\fancypagestyle{plain}{
    \fancyhf{}
    \renewcommand{\headrulewidth}{0pt}
    \renewcommand{\footrulewidth}{0pt}
    \fancyfoot[R]{\thepage}
}
\pagestyle{plain}

% Problem Box
\setlength{\fboxsep}{4pt}
\newlength{\myparskip}
\setlength{\myparskip}{\parskip}
\newsavebox{\savefullbox}
\newenvironment{fullbox}{\begin{lrbox}{\savefullbox}\begin{minipage}{\dimexpr\textwidth-2\fboxsep\relax}\setlength{\parskip}{\myparskip}}{\end{minipage}\end{lrbox}\framebox[\textwidth]{\usebox{\savefullbox}}}
\newenvironment{pbox}[1][]{\begin{fullbox}\ifx#1\empty\else\paragraph{#1}\phantom{}\fi}{\end{fullbox}}

% Theorem Environments
\theoremstyle{definition}
\newtheorem{lemma}{Lemma}
\newtheorem{corollary}{Corollary}

% Tikz Environments
\newenvironment{drawing}{\begin{center}\begin{tikzpicture}}{\end{tikzpicture}\end{center}}
\tikzcdset{row sep/normal=0pt}
\newenvironment{cd}{\begin{center}\begin{tikzcd}}{\end{tikzcd}\end{center}}

% Default Commands
\newcommand{\isp}[1]{\quad\text{#1}\quad}
\newcommand{\N}{\mathbb{N}} 
\newcommand{\Z}{\mathbb{Z}}
\newcommand{\Q}{\mathbb{Q}}
\newcommand{\R}{\mathbb{R}}
\newcommand{\C}{\mathbb{C}}
\newcommand{\A}{\mathbb{A}}
\renewcommand{\P}{\mathbb{P}}
\newcommand{\eps}{\varepsilon}
\renewcommand{\phi}{\varphi}
\renewcommand{\emptyset}{\varnothing}
\newcommand{\<}{\langle}
\renewcommand{\>}{\rangle}
\newcommand{\isom}{\cong}
\newcommand{\eqc}{\overline}
\newcommand{\clo}{\overline}
\newcommand{\teq}{\trianglelefteq}
\DeclareMathOperator{\id}{id}
\DeclareMathOperator{\im}{im}

% Extra Commands
\newcommand{\FF}{\mathcal{F}}
\DeclareMathOperator{\inter}{int}
\newcommand{\hf}{\widehat{f}}
\renewcommand{\SS}{\mathcal{S}}


% Document
\begin{document}
\title{MATH 221A Homework 8}
\author{Harry Coleman}
\date{December 3, 2021}
\maketitle

\begin{pbox}[1]
    A real number $x$ is \emph{badly approximable} if there is some $c>0$ such that for every rational $p/q$ we have
  \[\left\lvert x-\frac{p}{q} \right\rvert>\frac{c}{q^2}.\]
  A somewhat weaker condition (let's say $\alpha$-\emph{badly approximable}) is when $q^2$ is replaced by $q^\alpha$ for some $\alpha>2$.
\end{pbox}

For $c > 0$ we say $x \in \R$ is \textbf{$(\alpha, c)$-bad} if for every rational $p/q$ we have 
\[
    \left| x - \frac{p}{q} \right| > \frac{c}{q^\alpha}.
\]
Then $x$ is \textbf{$\alpha$-bad} ($\alpha$-badly approximable) if it is $(\alpha, c)$-bad for some $c > 0$.

For $c > 0$ we say $x \in \R$ is \textbf{$(\alpha, c)$-good} if there is a rational $p/q$ such that
\[
    \left| x - \frac{p}{q} \right| < \frac{c}{q^\alpha}.
\]
We say $x \in \R$ is \textbf{$\alpha$-good} if it is $(\alpha, c)$-good for all $c > 0$.

Clearly, if $x$ is $\alpha$-good then it must not be $\alpha$-bad, since $\alpha$-good uses a strict inequality, which is stronger than the negation of $\alpha$-bad.
On the other hand, if $x$ is not $\alpha$-good, then it must not be $(\alpha, c)$-good for some $c > 0$.
In which case, $c < c/2$ so $x$ is $(\alpha, c/2)$-bad---hence $\alpha$-bad.

We may therefore conclude that a real number is $\alpha$-good if and only if it is not $\alpha$-bad.

\begin{pbox}
    Show that there are lots of $\alpha$-badly approximable numbers and lots of not-$\alpha$-badly approximable numbers (for example, both sets are uncountable).
\end{pbox}

\begin{proof}
    For $c > 0$ define the set
    \[
        G_c
            = \bigcup_{p/q \in \Q} B_{c/q^\alpha}(p/q)
            = \{x \in \R : x \text{ is $(\alpha, c)$-good}\},
    \]
    (where representatives $p/q \in \Q$ are chosen such that $\gcd(p, q) = 1$) and its complement
    \[
        N_c = \R \setminus G_c = \{x \in \R : x \text{ is $(\alpha, c)$-bad}\}.
    \]
    Then define
    \[
        G = \bigcap_{n \in \N} G_{1/n}
        \isp{and}
        N = \R \setminus G = \bigcup_{n \in \N} N_{1/n}.
    \]
    We claim that $G$ is precisely the set of $\alpha$-good real numbers and, by extension, $N$ is the set of $\alpha$-bad real numbers.
    Note that if $x \in \R$ is $\alpha$-good, then it must be $(\alpha, 1/n)$-good for every $n \in \N$, i.e., $x \in G$.
    On the other hand, if $x \in G$ and $c > 0$ then there is some $n \in \N$ such that $1/n < c$.
    In which case, $x$ is $(\alpha, 1/n)$-good, implying it is $(\alpha, c)$-good---hence $\alpha$-good.

    Note that $G_c$ is an open ``neighborhood'' of $\Q$ since it is the union of open balls---one for each rational.
    Since $\Q$ is dense in $\R$, $G_c$ is also dense in $\R$.
    It follows that $N_c$ is nowhere dense as a closed set with dense complement, therefore $N$ is meager.
    Then $G$ is comeager, so it cannot be meager.
    Since countable sets are meager, we conclude that $G$ is uncountable.

    Let $\lambda$ denote the Lebesgue measure on $\R$.
    For any $c > 0$ we have $G \subseteq G_c$ so
    \[
        \lambda(G)
            \leq \lambda(G_c)
            \leq \sum_{p/q \in \Q} \lambda(B_{c/q^\alpha}(p/q))
            = \sum_{p/q \in \Q} \frac{2c}{q^\alpha}.
    \]
    Again, we are assuming $\gcd(p, q) = 1$ so
    \[
        \sum_{p/q \in \Q} \frac{2c}{q^\alpha}
            = \sum_{n \in \N} \sum_{\substack{k \in \Z \\ \gcd(n, k) = 1}} \frac{1}{n^\alpha}
            = \sum_{n \in \N} \frac{2 \phi(n)}{n^\alpha},
    \]
    where $\phi$ denote Euler's totient function, i.e., $\phi(n)$ is the number of positive integers less than or equal to $n$ that are relatively prime to $n$.
    Note that $\phi(n) \leq n$ for all $n \in \N$, (with equality only if $n$ is prime).
    Then
    \[
        \sum_{n \in \N} \frac{\phi(n)}{n^\alpha}
            \leq \sum_{n \in \N} \frac{n}{n^\alpha} 
            = \sum_{n \in \N} \frac{1}{n^{\alpha-1}}.
    \]
    This is precisely the sum of the $p$-series for $p = \alpha - 1$.
    With $\alpha > 2$ we have $p > 1$ so the sum is $\zeta(\alpha - 1)$, i.e., the evaluation of the Riemann zeta function at $\alpha - 1$.
    In particular, we conclude that $\lambda(G) \leq \zeta(\alpha - 1) < \infty$.
    Therefore $\lambda(N) = \infty$, implying $N$ is uncountable.
\end{proof}

\newpage
\begin{pbox}[2 \\ (a)]
    Prove the \textbf{Uniform boundedness principle}:
    \begin{quote}
      Let $X$ be a complete metric space and let $\mathcal F$ be a subset of
      $C(X)$ such that for each $x \in X$, the set
      \[\mathcal F_x=\{f(x) : f \in \mathcal F\}\]
      is bounded.  Then there is a nonempty open set $U$ of $X$ on which the
      functions in $\mathcal F$ are \emph{uniformly bounded}, i.e.~there is an
      $M$ such that $|f(x)| \leq M$ for all $x \in U$ and $f \in \mathcal F$.
    \end{quote}
    \textbf{Hint:} Choose $A_N=\{x \in X: \lvert f(x) \rvert \leq N\text{ for all }f \in \mathcal F\}$.
\end{pbox}

\begin{proof}
    Given $x \in X$ we know by assumption that $\FF_x \subseteq [-N, N]$ for some $N \in \N$.
    In other words, $|f(x)| \leq N$ for all $f \in \FF$, hence $X = \bigcup_{N \in \N} A_N$.
    Assuming $X = \inter X$ is nonempty, the Baire category theorem implies it cannot be meager.
    Therefore, some $A_N$ must not be nowhere dense, i.e., $\inter \clo{A_N} \ne \emptyset$.
    We can write
    \[
        A_N = \bigcap_{f \in \FF} f^{-1}([-N, N]),
    \]
    which means $A_N$ is closed.
    Then
    \[
        U = \inter A_N = \inter\clo{A_N} \ne \emptyset
    \]
    is an open set of $X$ such that $|f(x)| \leq N$ for all $x \in U$ and $f \in \FF$.
\end{proof}

\begin{pbox}[(b)]
    Suppose now that $X$ is a Banach space and the functions in $\mathcal F$
    are linear.  Show that there is an $M$ such that for every $x \in X$ and
    $f \in \mathcal F$, $|f(x)| \leq M\lVert x \rVert$.
\end{pbox}

\begin{proof}
    Let $U$ and $M$ be as in part (a).
    Choose $z \in U$ and $\eps > 0$ such that $B_\eps(z) \subseteq U$.
    Then for any $f \in \FF$ and $u \in B_\eps(0)$ we have $z + u \in U$ so
    \[
        |f(u)|
            = |f(-z + z + u)|
            = |-f(z) + f(z + u)|
            \leq |f(z)| + |f(z + u)|
            \leq 2M.
    \]
    In other words, $B_\eps(0) \subseteq A_{2M}$.
    Given $x \in X$ choose $a > 0$ such that $\eps/2 \leq a\|x\| < \eps$.
    Then
    \[
        |f(x)|
            = \frac{|f(ax)|}{a}
            \leq \frac{2\|x\|}{\eps} \cdot 2M
            = \frac{4M}{\eps}\|x\|.
    \]

\end{proof}


\newpage
\begin{pbox}[3 \\ (a)]
    Let $Z$ and $X$ be locally compact Hausdorff spaces.  Suppose
    $K \subseteq Z \times X$ is a compact set contained in an open set $U$.  Show
    that $K$ is covered by a finite number of compact boxes
    $A_i \times B_i \subseteq U$.
\end{pbox}

\begin{proof}
    Consider a point $(z, x) \in K \subseteq U$.
    Then there are open neighborhoods $V \subseteq Z$ and $W \subseteq X$ of $z$ and $x$, respectively, such that $V \times W \subseteq U$.
    Since $Z$ and $X$ are locally compact and Hausdorff, we can choose a compact neighborhood $A_z \subseteq V$ of $x$ and $B_x \subseteq W$ of $y$.
    The product of interiors $\inter A_z \times \inter B_x \subseteq A_z \times B_x$ is an open neighborhood of $(z, x)$, hence $A_z \times B_x$ is a compact box neighborhood of $(z, x)$. 
    Importantly, we have found a compact box neighborhood of $(z, x)$ contained in $U$.

    Then the collection of interior boxes $\{\inter A_z \times \inter B_x\}_{(z, x) \in K}$ is an open cover of $K$.
    If we choose $\{\inter A_i \times \inter B_i\}_{i=1}^{n}$ to be a finite subcover of $K$, then $\{A_i \times B_i\}_{i=1}^{n}$ is the desired cover by compact boxes contained in $U$.
\end{proof}


\begin{pbox}[(b)]
    In class, we showed that if $X$ is a locally compact Hausdorff space,
    then for any spaces $Z$ and $Y$,
    \[C(Z,C(X,Y))\text{ and }C(Z \times X,Y)\]
    are in bijection as sets, where $C(X,Y)$ is given the compact-open topology.
    Show that if $Z$ is also locally compact Hausdorff, then this bijection is a
    homeomorphism if both sets are considered as spaces with the compact-open
    topology.

    \textbf{Hint:} As always, a good strategy to show two spaces are homeomorphic
    is to show that a subbasis for the topology of one is open in the other and
    vice versa.
\end{pbox}

\begin{lemma}
    If $\SS$ is a subbasis of open sets for $Y$ then 
    \[
        \{V(K, U) : K \subseteq X \text{ compact}, U \in \SS\}
    \]
    is a subbasis of open sets for $C(X, Y)$ with the compact open topology.
\end{lemma}

\begin{proof}
    Let $\mathcal{B}$ be the collection of finite intersections of sets in $\SS$.
    Then $\mathcal{B}$ is a basis of open sets for $Y$.
    Consider an arbitrary standard subbasis set $V(K, U) \subseteq C(X, Y)$, i.e., $K \subseteq X$ is compact and $U \subseteq Y$ is open.
    Then $U = \bigcup_{\alpha \in I} U_\alpha$ for some $U_\alpha \in \mathcal{B}$ so
    \[\textstyle
        V(K, U) = \bigcup\limits_{\alpha \in I} V(K, U_\alpha).
    \]
    In other words,
    \[
        \{V(K, U) : K \subseteq X \text{ compact}, U \in \mathcal{B}\}
    \]
    generates the standard subbasis for $C(X, Y)$ under arbitrary unions and is therefore itself a subbasis.
    Given a basis set $U \in \mathcal{B}$, we can write $U = \bigcap_{i=1}^{n} U_i$ for some $U_i \in \SS$ so
    \[\textstyle
        V(K, U) = \bigcap_{i=1}^{n} V(K, U_i).
    \]
    That is, the collection subbasis sets in the statement of the Lemma generates a subbasis for $C(X, Y)$ under finite intersections and arbitrary unions, hence the collection is itself a subbasis.
\end{proof}


\begin{lemma}
    The collection
    \[
        \{V(A \times B, U) : A \subseteq Z \text{ and } B \subseteq X \text{ compact}, U \subseteq Y \text{ open}\}
    \]
    is a subbasis of open sets for $C(Z \times X, Y)$ with the compact open topology.
\end{lemma}


\begin{proof}
    Let $V(K, U)$ be a standard subbasis set and $f \in V(K, U)$.
    Then $K \subseteq Z \times X$ is a compact set contained in the open set $f^{-1}(U)$.
    As in part (a), let $\{A_i \times B_i\}_{i=1}^{n}$ be a cover of $K$ by compact boxes contained in $f^{-1}(U)$.
    Then $f \in K(A_i \times B_i, U)$ for $i = 1, \dots, n$ and
    \[
        \bigcap_{i=1}^{n} V(A_i \times B_i, U)
            = V({\textstyle\bigcup\limits_{i=1}^{n}} (A_i \times B_i), U)
            \subseteq V(K, U).
    \]
    This is an open neighborhood of $f$ contained in $V(K, U)$.
    We can then write $V(K, U)$ as the union of all such intersections for $f \in V(K, U)$.
    That is, the collection in question generates the standard subbasis under finite intersections and arbitrary unions and is therefore itself a subbasis.
\end{proof}


\begin{proof}[Proof (a)]
    Lemma 1 tells us that
    \[
        \{V(A, V(B, U)) : A \subseteq Z \text{ and } B \subseteq X \text{ compact}, U \subseteq Y \text{ open}\}
    \]
    is a subbasis of open sets for $C(Z, C(X, Y))$ with the compact open topology.
    Applying Lemma 2, the bijection induces a correspondence of subbasis sets:
    \begin{align*}
        C(Z, C(X, Y)) &\longleftrightarrow C(Z \times X, Y) \\
        V(A, V(B, U)) &\longleftrightarrow V(A \times B, U)
    \end{align*}
\end{proof}

\end{document}