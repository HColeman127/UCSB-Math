\documentclass[12pt]{article}

% Packages
\usepackage[margin=1in]{geometry}
\usepackage{fancyhdr, parskip}
\usepackage{amsmath, amsthm, amssymb}

% Page Style
\makeatletter
\fancypagestyle{title}{
    \renewcommand{\headrulewidth}{0.4pt}
    \setlength{\headheight}{15pt}
    \fancyhead[R]{\@author}
    \fancyhead[L]{\@title}
    \fancyhead[C]{\@date}
}
\makeatother
\renewcommand{\maketitle}{\thispagestyle{title}}
\fancypagestyle{plain}{
    \fancyhf{}
    \renewcommand{\headrulewidth}{0pt}
    \renewcommand{\footrulewidth}{0pt}
    \fancyfoot[R]{\thepage}
}
\pagestyle{plain}

% Problem Box
\setlength{\fboxsep}{4pt}
\newlength{\myparskip}
\setlength{\myparskip}{\parskip}
\newsavebox{\savefullbox}
\newenvironment{fullbox}{\begin{lrbox}{\savefullbox}\begin{minipage}{\dimexpr\textwidth-2\fboxsep\relax}\setlength{\parskip}{\myparskip}}{\end{minipage}\end{lrbox}\framebox[\textwidth]{\usebox{\savefullbox}}}
\newenvironment{pbox}[1][]{\begin{fullbox}\ifx#1\empty\else\paragraph{#1}\fi}{\end{fullbox}}

% Default Commands
\newcommand{\isp}[1]{\quad\text{#1}\quad}
\newcommand{\N}{\mathbb{N}} 
\newcommand{\Z}{\mathbb{Z}}
\newcommand{\Q}{\mathbb{Q}}
\newcommand{\R}{\mathbb{R}}
\newcommand{\C}{\mathbb{C}}
\newcommand{\eps}{\varepsilon}
\renewcommand{\phi}{\varphi}
\renewcommand{\emptyset}{\varnothing}
\newcommand{\<}{\langle}
\renewcommand{\>}{\rangle}
\newcommand{\isom}{\cong}
\newcommand{\eqc}{\overline}
\newcommand{\clo}{\overline}
\newcommand{\teq}{\trianglelefteq}
\DeclareMathOperator{\id}{id}

% Extra Commands
\DeclareMathOperator{\sgn}{sgn}

% Document
\begin{document}
\title{MATH 221A Homework 3}
\author{Harry Coleman}
\date{October 20, 2021}
\maketitle

\begin{pbox}[1]
    If $(X,d)$ is a metric space, a map $f:X \to X$ is a \emph{contraction} if there is a number $\alpha<1$ such that
  \[d(f(x),f(y)) \leq \alpha d(x,y).\]
  Show that if $f$ is a contraction of a complete metric space, then there is a unique point $x \in X$ such that $f(x)=x$.
\end{pbox}

\begin{proof}
    We construct a sequence of points in $X$ inductively: let $x_0 \in X$ be any arbitrary point and define $x_n = f(x_{n-1})$ for all $n \geq 1$. We will prove that this sequence is Cauchy, therefore convergent, and that the limit of the sequence in $X$ is the unique fixed point.

    For $n \in \N$, we use the fact that $f$ is a contraction to compute
    \begin{align*}
        d(x_0, x_n)
            &= d(x_0, f^{n-1}(x_1)) \\
            &\leq \sum_{k=0}^{n-1} d\big(f^k(x_0), f^k(x_1)\big) \\
            &\leq \sum_{k=0}^{n-1} \alpha^k d(x_0, x_1) \\
            &< \sum_{k=0}^{\infty} \alpha^k d(x_0, x_1) \\
            &= \frac{d(x_0, x_1)}{1 - \alpha}.
    \end{align*}
    Define $M = d(x_0, x_1)/(1 - \alpha) \in \R_{\geq0}$. For $n, m \in \N$, assuming $n \leq m$, we compute
    \[
        d(x_n, x_m)
            = d(f^n(x_0), f^n(x_{m-n}))
            \leq \alpha^n d(x_0, x_{m-n})
            < \alpha^nM.
    \]
    Given $\eps > 0$, we can choose $N \in \N$ such that $\alpha^NM < \eps$. Then, if $n, m \geq N$,
    \[
        d(x_n, x_m)
            < \alpha^{\min(n, m)}M
            \leq \alpha^NM
            < \eps.
    \]
    Hence, the sequence is Cauchy.
    
    Since $X$ is complete, the sequence converges to some $x \in X$; we will show that $x$ is a fixed point of $f$. Let $\eps > 0$ be given and choose $N \in \N$ such that $d(x, x_n) < \eps$ for all $n \geq N$. Then
    \begin{align*}
        d(x, f(x))
            &\leq d(x, x_{N+1}) + d(x_{N+1}, f(x)) \\
            &= d(x, x_{N+1}) + d(f(x_N), f(x)) \\
            &\leq d(x, x_{N+1}) + \alpha d(x_N, x) \\
            &< \eps + \alpha \eps \\
            &= (1 + \alpha)\eps.
    \end{align*}
    Since this holds for all $\eps > 0$, we must have $d(x, f(x)) = 0$, implying that $x = f(x)$.

    Lastly, we will prove that $x$ is the unique fixed point of $f$. Suppose $y \in X$ is a fixed point of $f$, i.e., $f(y) = y$. Since $d(x, y) = d(f(x), f(y)) \leq \alpha d(x, y)$, we must have $d(x, y) = 0$, which means that $x = y$.

\end{proof}




\newpage
\begin{pbox}[2]
    In this question, we use the following definition of completion: the
  metric space $Y$ is a completion of $X$ if it contains an isometrically
  embedded copy of $X$ whose closure is $Y$.
\end{pbox}

\begin{pbox}[(a)]
    A map $f:X \to Z$ between metric spaces is \emph{Lipschitz} if there's a constant $L$ such that
    \[d(f(x),f(y)) \leq Ld(x,y).\]
    Show that if $Z$ is a complete metric space, then any Lipschitz map
    $f:X \to Z$ extends uniquely to the completion.
\end{pbox}

\begin{proof}
    

    Let $X \hookrightarrow Y$ be the isometric inclusion of $X$ into a completion $Y$; we identify $X$ with its image in $Y$. We will construct a function $\tilde{f} : Y \to Z$ which is an extension of $f$.
    
    For all $x \in X$, we define $\tilde{f}(x) = f(x)$, i.e., we manually enforce that $\tilde{f}|_X = f$.
    
    Given $y \in Y \setminus X$, there is some sequence $(x_n)$ in $X$, converging to $y$ in $Y$. In particular, this is a Cauchy sequence in $X$. We check that $(f(x_n))$ is a Cauchy sequence in $Z$. Given $\eps > 0$, we can choose $N \in \N$ such that $d(x_n, x_m) < \eps/L$ for all $n, m \geq N$. Then, for any $n, m \geq N$,
    \[
        d(f(x_n), f(x_m))
            \leq Ld(x_n, x_m)
            < \eps.
    \]
    Hence, $(f(x_n))$ is a Cauchy sequence in the complete metric space $Z$ and, therefore, converges to some point $z \in Z$. We define $\tilde{f}(y) = z$ (one can check that this is well-defined in the sense that it does not depend on the original choice of sequence $(x_n)$). 

    Next, we check that $\tilde{f}$ is Lipschitz on $Y$. Suppose $x, y \in Y$, then there are sequences $(x_n)$, $(y_n)$ in $X$ such that $x_n \to x$ and $y_n \to y$. For any $n \in \N$,
    \[
        d(\tilde{f}(x), \tilde{f}(y))
            \leq d(\tilde{f}(x), f(x_n)) + d(f(x_n), f(y_n)) + f(f(y_n), \tilde{f}(y)).
    \]
    Let $\eps > 0$ be given. Since $\tilde{f}(x)$ is defined as the limit of $f(x_n)$, there is some $N \in \N$ such that $d(\tilde{f}(x), f(x_n)) < \eps$ for all $n \geq N$. (More precisely, this works when $x \in Y \setminus X$. In the case that $x \in X$, we may assume $x_n = x$ for all $n$. Then, trivially, $f(x_n) \to f(x)$.) For the same reason, we can assume $N$ is chosen large enough that $d(\tilde{f}(y), f(y_n)) < \eps$ for all $n \geq N$. Then, for $n \geq N$, we have
    \begin{align*}
        d(\tilde{f}(x), \tilde{f}(y))
            &< 2\eps + d(f(x_n), f(y_n)) \\
            &\leq 2\eps + Ld(x_n, y_n).
    \end{align*}
    Now, we examine
    \[
        d(x_n, y_n) \leq d(x_n, x) + d(x, y) + d(y, y_n).
    \]
    Since $x_n \to x$ and $y_n \to y$, we can also assume $N$ is chosen large enough that $d(x, x_n)$ and $d(y, y_n)$ are both less than $\eps$ for all $n \geq N$. In which case,
    \[
        d(x_n, y_n) < 2\eps + d(x, y).
    \]
    This now gives us
    \[
        d(\tilde{f}(x), \tilde{f}(y)) < 2\eps + L(2\eps + d(x, y)).
    \]
    Letting $\eps \to 0$, we obtain $d(\tilde{f}(x), \tilde{f}(y)) \leq Ld(x, y)$, i.e., $\tilde{f}$ is Lipschitz, with the same Lipschitz constant as $f$.

    It remains to prove that this Lipschitz extension of $f$ is unique. Suppose $g : Y \to Z$ is a Lipschitz function such that $g|_X = f$. For any $x \in Y$, there is some sequence $(x_n)$ in $X$ converging to $x$. Then, for all $n \in \N$, we have
    \begin{align*}
        d(\tilde{f}(x), g(x))
            &\leq d(\tilde{f}(x), \tilde{f}(x_n)) + d(\tilde{f}(x_n), g(x)) \\
            &= d(\tilde{f}(x), \tilde{f}(x_n)) + d(g(x_n), g(x)) \\
            &\leq Ld(x, x_n) + L'd(x_n, x) \\
            &= (L + L')d(x_n, x),
    \end{align*}
    where $L'$ is the Lipschitz constant for $g$. Letting $n \to \infty$, we obtain $d(\tilde{f}(x), g(x)) = 0$, so in fact $\tilde{f} = g$.

\end{proof}

\begin{pbox}[(b)]
    Show that the completion of a metric space is unique.  That is, if $Y$
    and $Z$ are two completions of $X$, show that the identity map $X \to X$
    extends to an isometry $Y \to Z$.
\end{pbox}

\begin{proof}
    The isometric embedding $\id_X : X \hookrightarrow Y$ is, in particular, a Lipschitz map from $X$ to a complete metric spaces. Therefore, by part (a), there exists a unique Lipschitz map $f : Y \to Y$ such that $f|_X = \id_X$. Since $\id_Y$ is a Lipschitz map on $Y$ with $\id_Y|_X = \id_X$, then in fact $f = \id_Y$. In other words, the identity on $Y$ is the only Lipschitz map on $Y$ which restricts to the identity on $X$. The same is also true of $Z$.

    The inclusion $X \hookrightarrow Z$ uniquely extends to a Lipschitz map $f : Y \to Z$, which restricts to the identity on $X$. We claim that $f$ is surjective. Let $z \in Z$, then there is a sequence $(x_n)$ in $X$ converging to $z$ (in $Z$). In particular, the sequence is Cauchy in $X$, so has some limit $y \in Y$. Since $f$ is continuous, $f(x_n) \to f(y)$. And since $f|_X = \id_X$, we also have $f(x_n) = x_n \to z$. Hence, $f(y) = z$, so $f$ is surjective.

    Similarly, the inclusion $X \hookrightarrow Y$ uniquely extends to a Lipschitz map $g : Z \to Y$, which restricts to the identity on $X$. Moreover, by the same argument, $g$ is surjective. Then $g \circ f : Y \to Y$ and $f \circ g : Z \to Z$ are Lipschitz maps which restrict to the identity on $X$. Since the identities on $Y$ and $Z$, respectively, are the unique such maps, then $f$ and $g$ are inverses.

    Lastly, we check that $f$ is an isometry. In part (a), we showed that the extension has the same Lipschitz constant as the original map, which is $1$ for $\id_X$, $\id_Y$, implying the same for $f$ and $g$. So, for all $x, y \in Y$,
    \[
        d(f(x), f(y))
            \leq d(x, y)
            = d(g(f(x)), g(f(y)))
            \leq d(f(x), f(y)).
    \]
    Hence $d(f(x), f(y)) = d(x, y)$, so $f$ is an isometry $Y \to Z$.

\end{proof}


\newpage
\begin{pbox}[3]
    Show that the closed unit ball in $C_B([0,1])$ with the norm topology is not compact.
\end{pbox}

\begin{proof}
    For $n \in \N$, define the function $f_n \in C_B([0, 1])$ by $f_n(x) = x^n$. Pointwise, this sequence converges to the function
    \[
        f(x) = \begin{cases}
            0 &\text{if } 0 \leq x < 0, \\
            1 &\text{if } x = 1.
        \end{cases}
    \]
    Clearly, $f$ is not continuous, so $f \notin C_B([0, 1])$. Therefore, the sequence $(f_n)$ does not converge in $C_B([0, 1])$, with the norm topology. This is because converging in the norm topology means uniform convergence in $[0, 1]$, and if a sequence of functions uniformly converges, it must also converges pointwise to the same limit. But $(f_n)$ converges pointwise to $f \notin C_B([0, 1])$, i.e., it does not converge uniformly to a function in $C_B([0, 1])$.

    In particular, $(f_n)$ is a sequence in $C_B([0, 1])$ with no convergent subsequence, implying that $C_B([0, 1])$ cannot be compact.

\end{proof}

\newpage
\begin{pbox}[4]
    Recall that we say that a sequence $(v_n)$ in a topological vector space $X$ is \emph{Cauchy} if for every neighborhood $U$ of $0$ there is an $N$ such that for $m,n \geq N$, $v_m-v_n \in U$.  The topological vector space $X$ is \emph{complete} if every Cauchy sequence converges.
\end{pbox}

\begin{pbox}[(a)]
    Show that if $X$ is a \emph{locally compact} space (every point has a compact neighborhood) then $C(X)$ with the compact-open topology is complete.
\end{pbox}

For $K \subseteq X$ compact and $U \subseteq \R$ open, denote
\[
    S(K, U) = \{f \in C(x) : f(K) \subseteq U\}.
\]
The collection of all such $S(K, U)$ is a subbasis for the compact-open topology on $C(X)$, meaning that the collection of all finite intersections of such sets is a basis.

\begin{proof}
    Suppose $(f_n)$ is Cauchy sequence in $C(X)$.

    For any $x \in X$, there is a sequence $(f_n(x))$ in $\R$, which we claim to be Cauchy. Let $\eps > 0$ be given. Let $K \subseteq X$ be a compact neighborhood of $x$ and define $U = S(K, B_\eps(0))$, which is an open neighborhood of $0 \in C(X)$. Since $(f_n)$ is Cauchy in $C(X)$, there is some $N \in \N$ such that, for all $n, m \geq N$, $f_n - f_m \in U$. Then, for $m, n \geq N$, we have
    \[
        |f_n(x) - f_m(x)|
            \leq \|f_n - f_m\|_K
            < \eps.
    \]
    That is, $(f_n(x))$ is Cauchy in $\R$. Since $\R$ is complete, we may define a function $f : X \to \R$ by
    \[
        f(x) = \lim_{n\to\infty} f_n(x).
    \]

    By definition, $f_n \to f$ pointwise in $X$. We claim that this convergence is uniform in every compact subset of $X$. Let $K \subseteq X$ be compact and $\eps > 0$ be given. As before, we choose $N \in \N$ such that $f_n - f_m \in S(K, B_\eps(0))$, for all $n, m \geq N$. So, for all $x \in K$ and $n, m \geq N$,
    \[
        |f_n(x) - f_m(x)| \leq \|f_n - f_m\|_K < \eps.
    \]
    Letting $m \to \infty$ in this inequality, we obtain $|f_n(x) - f(x)| \leq \eps$ for all $x \in K$. Hence, we have uniform convergence $f_n \to f$ in every compact subset of $X$.

    For any compact subset $K \subseteq X$, $(f_n|_K)$ is a sequence of continuous functions on $K$, converging uniformly to $f|_K$. This implies that $f|_K$ is a continuous function, i.e., $f$ is continuous on each compact subset of $X$.
    
    Each point of $X$ has a compact neighborhood, which contains an open neighborhood. Then $f$ is continuous on each of the compact neighborhoods, so it is also continuous on each of the open neighborhoods. These open neighborhoods form an open cover of $X$, so we conclude that $f$ is continuous on all of $X$, hence $f \in C(X)$.

    In remains to prove that $(f_n)$ converges to $f$ in $C(X)$ (with the compact-open topology). Let $\eps > 0$ be given. Suppose $S(K, U)$ is an open neighborhood of $f$ in the subbasis, i.e., $K \subseteq X$ compact, $U \subseteq \R$ open, and $f(K) \subseteq U$. Since $f(K)$ is a compact subset of the metric space $\R$, then by Homework 2 Problem 3(d), we can assume $\eps$ is small enough so that $U(f(K), \eps) \subseteq U$, where
    \[
        U(f(K), \eps) = \{a \in \R : d(a, f(K)) < \eps\}.
    \]
    Since $f_n \to f$ uniformly in $K$, there is some $N \in \N$ such that $\|f_n - f\|_K < \eps$, for $n \geq N$. In which case,
    \[
        f_n(K) \subseteq U(f(K), \eps) \subseteq U,
    \]
    implying $f_n \in S(K, U)$. Now, suppose $B = \bigcap_{i=1}^{r} S(K_i, U_i)$ is an arbitrary set in the basis for the compact-open topology on $C(X)$. For $i = 1, \dots, r$, we have just shown that there is some $N_i \in \N$ such that $f_n \in S(K_i, U_i)$, for all $n \geq N_i$. Define $N = \max\{N_1, \dots, N_r\}$. Then, for all $n \geq N$, $f_n \in S(K_i, U_i)$ for $i = 1, \dots, r$, implying that $f_n \in B$. Since every open neighborhood of $f$ contains a neighborhood in the basis, this proves that $f_n \to f$ in the compact-open topology on $C(X)$.

\end{proof}

\begin{pbox}[(b)]
    Show that $C_B(\mathbb R)$ is not complete when given the compact-open topology.
\end{pbox}

\begin{proof}
    We construct a sequence which is Cauchy in the compact-open topology on $C_B(\R)$, but does not converge. For $n \in \N$, define the function $f_n : \R \to \R$ by
    \[
        f_n(x) = \begin{cases}
            -n &\text{if } x < -n, \\
            x &\text{if } -n \leq x \leq n, \\
            n &\text{if } x > n.
        \end{cases}
    \]
    Then $f_n \in C_B(\R)$ and the sequence $(f_n)$ converges pointwise to $f = \id_\R \notin C_B(\R)$. This is a Cauchy sequence in the compact-open topology on $C_B(\R)$, since it converges uniformly on any compact subset of $\R$. However, the sequence does not converge in the compact-open topology on $C_B(\R)$, since the limit there would be the same as the pointwise limit, which is not in $C_B(\R)$.

\end{proof}

\begin{pbox}[(c)]
    (Optional, only if you're familiar with Lebesgue integration) Show that $C_B([0,1])$ is not complete when given the \emph{weak-* topology}, which is generated by inverse images of open sets under maps $\int_M:C_B([0,1]) \to \mathbb{R}$ which send $f$ to its integral over a measurable set $M$.  (The completion is called $L^\infty([0,1])$.)
\end{pbox}


\end{document}