\documentclass[12pt]{article}

% Packages
\usepackage[margin=1in]{geometry}
\usepackage{fancyhdr, parskip}
\usepackage{amsmath, amsthm, amssymb}
\usepackage{tikz, tikz-cd}

% Page Style
\makeatletter
\fancypagestyle{title}{
    \renewcommand{\headrulewidth}{0.4pt}
    \setlength{\headheight}{15pt}
    \fancyhead[R]{\@author}
    \fancyhead[L]{\@title}
    \fancyhead[C]{\@date}
}
\makeatother
\renewcommand{\maketitle}{\thispagestyle{title}}
\fancypagestyle{plain}{
    \fancyhf{}
    \renewcommand{\headrulewidth}{0pt}
    \renewcommand{\footrulewidth}{0pt}
    \fancyfoot[R]{\thepage}
}
\pagestyle{plain}

% Problem Box
\setlength{\fboxsep}{4pt}
\newlength{\myparskip}
\setlength{\myparskip}{\parskip}
\newsavebox{\savefullbox}
\newenvironment{fullbox}{\begin{lrbox}{\savefullbox}\begin{minipage}{\dimexpr\textwidth-2\fboxsep\relax}\setlength{\parskip}{\myparskip}}{\end{minipage}\end{lrbox}\framebox[\textwidth]{\usebox{\savefullbox}}}
\newenvironment{pbox}[1][]{\begin{fullbox}\ifx#1\empty\else\paragraph{#1}\phantom{}\fi}{\end{fullbox}}

% Theorem Environments
\theoremstyle{definition}
\newtheorem{lemma}{Lemma}

% Tikz Environments
\newenvironment{drawing}{\begin{center}\begin{tikzpicture}}{\end{tikzpicture}\end{center}}
% \tikzcdset{row sep/normal=0pt}
\newenvironment{cd}{\begin{center}\begin{tikzcd}}{\end{tikzcd}\end{center}}

% Default Commands
\newcommand{\isp}[1]{\quad\text{#1}\quad}
\newcommand{\N}{\mathbb{N}} 
\newcommand{\Z}{\mathbb{Z}}
\newcommand{\Q}{\mathbb{Q}}
\newcommand{\R}{\mathbb{R}}
\newcommand{\C}{\mathbb{C}}
\newcommand{\A}{\mathbb{A}}
\renewcommand{\P}{\mathbb{P}}
\newcommand{\eps}{\varepsilon}
\renewcommand{\phi}{\varphi}
\renewcommand{\emptyset}{\varnothing}
\newcommand{\<}{\langle}
\renewcommand{\>}{\rangle}
\newcommand{\isom}{\cong}
\newcommand{\eqc}{\overline}
\newcommand{\clo}{\overline}
\newcommand{\seq}{\subseteq}
\newcommand{\teq}{\trianglelefteq}
\DeclareMathOperator{\id}{id}
\DeclareMathOperator{\im}{im}
\newcommand{\inc}{\hookrightarrow}

% Extra Commands


% Document
\begin{document}
\title{MATH 237B Homework 7}
\author{Harry Coleman}
\date{Febraury 22, 2022}
\maketitle

\begin{pbox}[1 Vakil Exercise 21.7.K]
    Suppose $C$ is an irreducible smooth projective curve over an algebraically closed field $k = \clo{k}$ of characteristic $0$, of genus $g \geq 2$.
    Suppose that $G$ is a finite group of automorphisms of $C$.
\end{pbox}

\begin{pbox}[(a)]
    Let $C'$ be the smooth projective curve corresponding to the field extension $K(C)^G$ of $k$ (via Theorem 17.4.3).
    ($(K(C)^G$ means the $G$-invariants of $K(C)$.)
    Describe a morphism $\pi : C \to C'$ of degree $|G|$, as well as a faithful $G$-action on $C$ that commutes with $\pi$.
\end{pbox}



\begin{pbox}[(b)]
    Show that above each branch point of $\pi$, the preimages are all ramified to the same order (as $G$ acts transitively on them).
    Suppose there are $n$ branch points and the $i$th one has ramification $r_i$ (each $|G|/r_i$ times).
\end{pbox}

\begin{pbox}[(c)]
    Use the Riemann-Hurwitz formula to show that
    \[
        (2g - 2) = |G| \left(2g(C') - 2 \sum_{i=1}^{n} \frac{r_i - 1}{r_i}\right)
    \]
\end{pbox}

\begin{pbox}[2 Gathamnn Exercise 7.8.9]
    Show that any smooth projective curve of genus $2$...
\end{pbox}

\begin{pbox}[(i)]
    can be realized as a curve of degree $5$ in $\P^3$,
\end{pbox}

\begin{pbox}[(ii)]
    admits a two-to-one morphism to $\P^1$.
    How many ramification points does such a morphism have?
\end{pbox}

\begin{pbox}[3 Gathamnn Exercise 7.8.10]
    Let $X$ be a smooth projective curve, and let $P \in X$ be a point.
    Show there is a rational function on $X$ that is regular everywhere except at $P$.
\end{pbox}

\end{document}