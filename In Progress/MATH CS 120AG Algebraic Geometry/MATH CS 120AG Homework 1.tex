\documentclass[12pt]{article}

% Packages
\usepackage[margin=1in]{geometry}
\usepackage{fancyhdr}
\usepackage{amsmath, amsthm, amssymb, physics, comment}

% Page Style
\fancypagestyle{plain}{
    \fancyhf{}
    \renewcommand{\headrulewidth}{0pt}
    \renewcommand{\footrulewidth}{0pt}
    \fancyfoot[R]{\thepage}
}
\pagestyle{plain}

% Problem Box
\setlength{\fboxsep}{4pt}
\newsavebox{\savefullbox}
\newenvironment{fullbox}{\begin{lrbox}{\savefullbox}\begin{minipage}{\dimexpr\textwidth-2\fboxsep\relax}}{\end{minipage}\end{lrbox}\begin{center}\framebox[\textwidth]{\usebox{\savefullbox}}\end{center}}
\newenvironment{pbox}[1][]{\begin{fullbox}\ifx#1\empty\else\paragraph{#1}\fi}{\end{fullbox}}

% Options
\renewcommand{\thesubsection}{\thesection(\alph{subsection})}
\allowdisplaybreaks
\addtolength{\jot}{4pt}
\theoremstyle{definition}

% Default Commands
\newtheorem{proposition}{Proposition}
\newtheorem{lemma}{Lemma}
\newcommand{\ds}{\displaystyle}
\newcommand{\isp}[1]{\quad\text{#1}\quad}
\newcommand{\N}{\mathbb{N}}
\newcommand{\Z}{\mathbb{Z}}
\newcommand{\Q}{\mathbb{Q}}
\newcommand{\R}{\mathbb{R}}
\newcommand{\C}{\mathbb{C}}
\newcommand{\eps}{\varepsilon}
\renewcommand{\phi}{\varphi}
\renewcommand{\emptyset}{\varnothing}
\newcommand{\pfrac}[2]{\left(\frac{#1}{#2}\right)}

% Extra Commands
\newcommand{\<}{\langle}
\renewcommand{\>}{\rangle}
\newcommand{\teq}{\trianglelefteq}
\newcommand{\A}{\mathbb{A}}
\newcommand{\rad}{\sqrt}
\renewcommand{\forall}{\text{ for all }}
\newcommand{\eqc}{\overline}
\newcommand{\isom}{\cong}

% Document Info
\fancypagestyle{title}{
    \renewcommand{\headrulewidth}{0.4pt}
    \setlength{\headheight}{15pt}
    \fancyhead[R]{Harry Coleman}
    \fancyhead[L]{MATH CS 120AG Homework 1}
    \fancyhead[C]{April 5, 2021}
}

% Begin Document
\begin{document}
\thispagestyle{title}

\begin{comment}
\begin{pbox}[Exercise 1.19]
    Prove that every affine variety $X \subset \A^n$ consisting of only finitely many points can be written as the zero locus of $n$ polynomials.
\end{pbox}


\begin{proof}
    Given a finite subset $A = \{t_0, t_1, \dots, t_m\} \subset K$ of distinct points, we can define a polynomial
    \[
        f = \prod_{j = 1}^{m} \frac{x - t_j}{t_0 - t_j} \in K[x].
    \]
    Then $f$ is nonzero in $K \setminus A$, $f(t_0) = 1$, and  $f(t_j) = 0$ for $j = 1, \dots, m$. 

    
    

    Suppose $X = \{a_1, \dots, a_m\} \subset \A^n$ with $a_i = (a_{i1}, \dots, a_{in})$. For $j = 1, \dots, n$, we define polynomials
    \[
        f_1 = (x_1 - a_{11}) \cdots (x_1 - a_{m1}) \in K[x_j].
    \]
    So $f_1(x) = 0$ if and only if $x_1 = a_{i1}$ for some $a_i \in X$, i.e.,
    \[
        V(f_1) = \{x \in \A^n \mid x_1 = a_{i1} \text{ for some } a_i \in X\}.
    \]
    Then for $k = 1, \dots, m$, we define the polynomials
    \[
        \ell_k = \prod_{\substack{j = 1 \\ j \ne k}}^{m} \frac{x_1 - a_{j1}}{a_{k1} - a_{j1}} \in K[x_1],
    \]
    so $\ell_k(a_{j1}) = \delta_{kj}$. In other words, Then for $j = 2, \dots, n$, we define the polynomials
    \[
        f_j = \sum_{k=1}^{m} \ell_k(x_1) (x_j - a_{kj}) \in K[x_1, x_j].
    \]
    For a point $x \in V(f_1)$, we have $x_1 = a_{i1}$ for some $a_{i1} \in X$. In which case, we have the evaluation
    \[
        f_j(x)
            = \sum_{k=1}^{m} \ell_k(a_{i1}) (x_j - a_{kj})
            = x_j - a_{ij},
    \]
    which is zero if and only if $x_j = a_{ij}$. That is, if and only if $x$ shares its $j$th coordinate with a point in $X$, with which $x$ also shares its first coordinate. In particular, note that $f_1, \dots, f_n \in K[x_1, \dots, x_n]$, and we now consider the affine variety given by $V(f_1, \dots, f_n)$. First, we show that $X \subseteq V(f_1, \dots, f_n)$. For any point $a_i \in X$, we have
    \[
        f_1(a_i) = (a_{i1} - a_{11}) \cdots (a_{i1} - a_{i1}) \cdots (x_1 - a_{m1}) = 0,
    \]
    and for $j = 2, \dots, n$, we have
    \[
        f_j(a_i) 
            = \sum_{k=1}^{m} \ell_k(a_{i1}) (a_{ij} - a_{kj})
            = a_{ij} - a_{ij}
            = 0.
    \]
    Next, we show that $V(f_1, \dots, f_n) \subseteq X$. Suppose $x \in \A^n$ such that $f_1(x) = \cdots = f_n(x) = 0$. Since $f_1(x) = 0$, assume that we have $x_1 = a_{i1}$. Then for $j = 2, \dots, n$, we have $f_j(x) = x_j - a_{ij} = 0$. So in fact $x_j = a_{ij}$, implying that we have $x = a_i \in X$. Thus, $X = V(f_1, \dots, f_n)$.

\end{proof}




\begin{pbox}[Exercise 1.20]
    Let $X$ be an affine variety.  Show that the coordinate ring $A(X)$ is a field if and only if $X$ is a single point.
\end{pbox}

\begin{proof}
    Suppose $X = \{a\}$, then we have the maximal ideal $J = \<x_1 - a_1, \dots, x_n - a_n\> \teq K[x_1, \dots, x_n]$. And since each polynomial of $J$ is zero on $X$, then $J \subseteq I(X)$. Then any ideal $J' \teq K[x_1, \dots, x_n]$ containing $I(X)$ must also contain $J$. And since $J$ is maximal then either $J' = K[x_1, \dots, x_n]$ or $J' = J$, and in the latter case, $J \subseteq I(X) \subseteq J'$ implies that $J = (X) = J'$. Hence, $I(X)$ is a maximal ideal.
    
    Now suppose the coordinate ring
    \[
        A(X) = K[x_1, \dots, x_n]/I(X)
    \]
    is a field. Since $A(X)$ can be regarded as the ring of polynomial functions on $X$, then the zero of $A(X)$ is the polynomial function which is zero on $X$, or, in other words, the equivalence class of functions in $K[x_1, \dots, x_n]$ which are zero on $X$. So a nonzero element of $A(X)$ is a polynomial of $K[x_1, \dots, x_n]$ which is nonzero at some point of $X$ (strictly speaking it would be an equivalence class of such polynomials agreeing on $X$). 
    
    Suppose we have $a \in X$, and consider the polynomial functions on $X$ given by $g_i = x_i - a_i$ for $i = 1, \dots, n$. We can see that $g_i(a) = 0$. And for any function $f \in A(X)$, we have
    \[
        (g_if)(a) = g_i(a)f(a) = 0f(a) = 0 \ne 1.
    \]
    In particular, this means that $g_i$ is not a unit in $A(X)$, since we would need $g_i g_i^{-1} = 1$ on all of $X$. Therefore, since $A(X)$ is a field, we must have $g_i = 0$ on $X$, which implies that $x_i = a_i$ for all $x \in X$. Since this is true for $i = 1, \dots, n$, then we must have $x = a$ for all $x \in X$, so in fact $X = \{a\}$.

\end{proof}

\end{comment}


\newpage

\begin{pbox}[Exercise 1.21]
    Determine the radical of the ideal $\<x_1^3 - x_2^6,\; x_1x_2 - x_2^3\> \teq \C[x_1, x_2]$.
\end{pbox}

Let $J = \<x_1^3 - x_2^6,\; x_1x_2 - x_2^3\>$ and let $z = (z_1, z_2) \in V(J)$. Then we have
\[
    z_1^3 = z_2^6 \isp{and} z_1z_2 = z_2^3.
\]
If either $z_1$ or $z_2$ is zero, then the first condition implies that they are both zero. Then if $z_2 \ne 0$, the second condition is equivalent to $z_1 = z_2^2$. So $z = (w^2, w)$ where $w = z_2$.

Now consider the point $z = (w^2, w)$ for some $w \in \C$. We have
\[
    (w^2)^3 - w^6 = 0 \isp{and} w^2w - w^3 = 0,
\]
so in fact $z \in V(J)$. Thus, we have that
\[
    V(J) = \{(w^2, w) \mid w \in \C\}.
\]
This is precisely the zero locus of the polynomial $x_1 - x_2^2 \in \C[x_1, x_2]$, since $x_1 - x_2^2 = 0$ if and only if $x_1 = x_2^2$. Therefore, we have
\[
    \rad{J} = I(V(J)) = I(V(\<x_1 - x_2^2\>)) = \rad{\<x_1 - x_2^2\>}.
\]
If we can show that $x_1 - x_2^2$ is irreducible in the unique factorization domain $\C[x_1, x_2]$, then its ideal is prime and, therefore, radical. Suppose for contradiction that $x_1 - x_2^2$ is reducible into two nonzero, non-unit polynomials $p, q \in \C[x_1, x_2]$. Since $\C$ is a field, then the degrees of both $p$ and $q$ must be at least $1$. And since
\[
    2 = \deg(x_1 + x_2^2) = \deg pq = \deg p + \deg q,
\]
then we must have $\deg p = \deg q = 1$. Then $p$ and $q$ are of the form
\[
    p = a_0 + a_1 x_1 + a_2 x_2 \isp{and} q = b_0 + b_1 x_1 + b_2 x_2.
\]
Since $pq = x_1 - x_2^2$ has no constant term, then either $a_0 = 0$ or $b_0 = 0$. Without loss of generality, assume $b_0 = 0$, then
\begin{align*}
    x_1 - x_2^2
        &= pq \\
        &= (a_0 + a_1 x_1 + a_2 x_2)(b_1 x_1 + b_2 x_2) \\
        &= a_0b_1x_1 + a_0b_2x_2 + a_1b_1x_1^2 + (a_1b_2 + a_2b_1)x_1x_2 + b_1b_2x_2^2
\end{align*}
Comparing the coefficient of $x_1$ on both sides, we see that $1 = a_0b_1$, implying that $a_0 \ne 0$. Similarly, the coefficient of $x_2^2$ is $-1 = b_1b_2$, so $b_2 \ne 0$. Lastly, the coefficient of $x_2$ is given by $0 = a_0b_2$. This is a contradiction since $a_0$ and $b_2$ are nonzero. Hence, the ideal of $x_1 - x_2^2$ is radical, so
\[
    \rad{J} = \rad{\<x_1 - x_2^2\>} = \<x_1 - x_2^2\>.
\]




\begin{pbox}[Exercise 1.22]
    Let $X \subset \A^3$ be the union of the three coordinate axes. Compute the generators for the ideal $I(X)$, and show that $I(X)$ cannot be generated by fewer than three elements.
\end{pbox}

\begin{proposition}
    Let $S = \{x_1x_2,\, x_1x_3,\, x_2x_3\}$. Then $I(X) = \<S\>$ and cannot be generated by a smaller set.
\end{proposition}

\begin{proof}
    In general, the $j$th coordinate axis of $\A^n$ is given by
    \[
        X_j = \{a \in \A^n \mid a_i = 0 \forall i \ne j\{.
        \]
    In particular, we are considering $X = X_1 \cup X_2 \cup X_3 \subset \A^3$, which has the ideal
    \[
        I(X) = I(X_1) \cap I(X_2) \cap I(X_3).
    \]
    Given any $a \in X_1$, we know that $a_2 = a_3 = 0$, so
    \[
        a_1a_2 = a_1a_3 = a_2a_3 = 0.
    \]
    This means that $f(a) = 0$ for all $f \in S$, implying $\<S\> \subseteq I(X_1)$. The same can be said for $X_2$ and $X_3$, so $\<S\> \subseteq I(X)$.

    Consider a polynomial $f \in K[x_1, x_2, x_3]$. The restriction $f|_{X_1}$ is a polynomial function on the first coordinate axis, which can be found explicitly by `evaluating' $f$ at the point $(x_1, 0, 0)$. That is, $f|_{X_1}$ can be thought of as a polynomial in $K[x_1]$ given the terms of $f$ which contain neither $x_2$ nor $x_3$.

    If $f \in I(X_1)$, then we know that $f|_{X_1} = 0$, meaning that $f$ has no terms which contain neither $x_2$ nor $x_3$. In other words, any term of $f$ which has a factor of $x_1$, must also have a factor of either $x_2$ or $x_3$. Applying the same argument to the other axes, we conclude that any polynomial $f$ has no terms containing only a single indeterminate and neither of the other two. So each term of $f$ is a multiple of either $x_1x_2$, $x_1x_3$, or $x_2x_3$, i.e., $f \in \<S\>$. Thus, we have $I(X) = \<S\>$.

    We now show that no fewer than three elements can generate $I = \<S\>$. Let $M = I(0) = \<x_1, x_2, x_3\>$ be the maximal ideal whose zero locus is the origin. Then we have the product of ideals
    \[
        MI = \<x_1, x_2, x_3\>\<S\> = \<x_1x_2x_3,\, x_1^2x_2,\, x_1x_2^2,\, x_1^2x_3,\, x_1x_3^2,\, x_2^2x_3,\, x_2x_3^2\>,
    \]
    which is the set of polynomials $f \in K[x_1, x_2, x_3]$ such that each term of $f$ is of degree at least $3$ and has a factor in $S$. Since $I = \<S\>$ is precisely the polynomials in $K[x_1, x_2, x_3]$ with each term having a factor in $S$, then $MI$ is the polynomials in $I$ with each term of degree at least $3$. Therefore, the quotient $I/MI$ is the (equivalence classes of) polynomials in $I$ with degree $2$.
    
    Let $R = K[x_1, x_2, x_3]$, then $I/MI$ is an $R$-module with the action $r\eqc{f} = \eqc{rf}$ for $r \in R$ and $\eqc{f} \in I/MI$. Given $m \in M$ and $f \in I$, we know that $mf \in MI$, implying that $m\eqc{f} = \eqc{mf} = \eqc{0}$ in $I/MI$. In other words, the $M$-action on $I/MI$, as a subset of the $R$-action, is annihilated. Therefore, the $(R/M)$-action given by by $(r + M)\eqc{f} = r\eqc{f}$ is well-defined on $I/MI$, making it an $(R/M)$-module.
    
    We have
    \[
        R/M = K[x_1, x_2, x_3]/\<x_1, x_2, x_3\> \isom K,
    \]
    and we regard $I/MI$ as the set of polynomials
    \[
        V = \{k_1x_1x_2 + k_2x_1x_3 + k_3x_2x_3 \mid k_1, k_2, k_3 \in K\}.
    \]
    Then $V$ is a $K$-module with the action of left-multiplication, and is `the same as' the $(R/M)$-module $I/MI$. In fact, this makes $V$ a $K$-vector space with addition from the polynomial ring and scalar multiplication from the $(R/M)$-action. Note that $S$ is a generating set for the vector space $V$, and if
    \[
        k_1x_1x_2 + k_2x_1x_3 + k_3x_2x_3 = 0,
    \]
    then we must have $k_1 = k_2 = k_3 = 0$, so $S$ is a basis for $V$. Hence, $\dim_K V = 3$. 

    Now suppose for contradiction that $I = \<p, q\>$. We can map $p$ and $q$ into $V$ by restricting each to only its terms of degree $2$. Since they generate $I$, then in particular, they generate $V$ as a subset of $I$. Let $f \in V$ and suppose $f = ap + bq$ for some $a, b \in R$. Since $p, q \in I$, then their terms must be of degree at least $2$. So the value of $ap + bq$ depends only on the degree $2$ terms of $p$ and $q$ and the constant terms of $a$ and $b$. Therefore, $f$ must be some $K$-linear combination of $p$ and $q$ restricted to their terms of degree $2$. This means that the restrictions of $p$ and $q$ generate $V$ as a $K$-vector space, implying that $\dim_K V \leq 2$, which is a contradiction.

\end{proof}









\begin{comment}
    

\begin{pbox}[Exercise 1.23]
    Let $Y \subset \A^n$ be an affine variety, and denote by $\pi : K[x_1, \dots, x_n] \to K[x_1, \dots, x_n]/I(Y) = A(Y)$ the quotient map.
\end{pbox}

\begin{pbox}[(a)]
    Show that $V_Y(J) = V(\pi^{-1}(J))$ for every ideal $J$ in $A(Y)$.
\end{pbox}

\begin{pbox}[(b)]
    Show that $\pi^{-1}(I_Y(X)) = I(X)$ for every affine subvariety $X$ of $Y$.
\end{pbox}

\begin{pbox}[(c)]
    Use (a) and (b) to deduce the (interesting part of the) relative Nullstellensatz $I_Y(V_Y(J)) \subset \rad{J}$ for every ideal $J \teq A(Y)$ from the corresponding absolute statement $I(V(J)) \subset \rad{J}$ for every ideal $J \teq K[x_1, \dots, x_n]$ in Proposition 1.10. In particular, conclude that there is an inclusion-reversing bijection between affine subvarieties of $Y$ and radical ideal in $A(Y)$.
\end{pbox}

\end{comment}


\end{document}