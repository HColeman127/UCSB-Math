\documentclass[12pt]{article}

% Packages
\usepackage[margin=1in]{geometry}
\usepackage{fancyhdr, parskip}
\usepackage{amsmath, amsthm, amssymb}
\usepackage{tikz, tikz-cd}

% Page Style
\makeatletter
\fancypagestyle{title}{
    \renewcommand{\headrulewidth}{0.4pt}
    \setlength{\headheight}{15pt}
    \fancyhead[R]{\@author}
    \fancyhead[L]{\@title}
    \fancyhead[C]{\@date}
}
\makeatother
\renewcommand{\maketitle}{\thispagestyle{title}}
\fancypagestyle{plain}{
    \fancyhf{}
    \renewcommand{\headrulewidth}{0pt}
    \renewcommand{\footrulewidth}{0pt}
    \fancyfoot[R]{\thepage}
}
\pagestyle{plain}

% Problem Box
\setlength{\fboxsep}{4pt}
\newlength{\myparskip}
\setlength{\myparskip}{\parskip}
\newsavebox{\savefullbox}
\newenvironment{fullbox}{\begin{lrbox}{\savefullbox}\begin{minipage}{\dimexpr\textwidth-2\fboxsep\relax}\setlength{\parskip}{\myparskip}}{\end{minipage}\end{lrbox}\framebox[\textwidth]{\usebox{\savefullbox}}}
\newenvironment{pbox}[1][]{\begin{fullbox}\ifx#1\empty\else\paragraph{#1}\phantom{}\fi}{\end{fullbox}}

% Theorem Environments
\theoremstyle{definition}
\newtheorem{lemma}{Lemma}

% Tikz Environments
\newenvironment{drawing}{\begin{center}\begin{tikzpicture}}{\end{tikzpicture}\end{center}}
\tikzcdset{row sep/normal=0pt}
\newenvironment{cd}{\begin{center}\begin{tikzcd}}{\end{tikzcd}\end{center}}

% Default Commands
\newcommand{\isp}[1]{\quad\text{#1}\quad}
\newcommand{\N}{\mathbb{N}} 
\newcommand{\Z}{\mathbb{Z}}
\newcommand{\Q}{\mathbb{Q}}
\newcommand{\R}{\mathbb{R}}
\newcommand{\C}{\mathbb{C}}
\newcommand{\A}{\mathbb{A}}
\renewcommand{\P}{\mathbb{P}}
\newcommand{\eps}{\varepsilon}
\renewcommand{\phi}{\varphi}
\renewcommand{\emptyset}{\varnothing}
\newcommand{\<}{\langle}
\renewcommand{\>}{\rangle}
\newcommand{\isom}{\cong}
\newcommand{\eqc}{\overline}
\newcommand{\clo}{\overline}
\newcommand{\teq}{\trianglelefteq}
\DeclareMathOperator{\id}{id}
\DeclareMathOperator{\im}{im}

% Extra Commands
\DeclareMathOperator{\Aut}{Aut}


% Document
\begin{document}
\title{MATH 220 Group Project}
\author{Harry Coleman}
\date{, 2021}
\maketitle

Groups of order $104 = 2^3 \cdot 13$.

By the Sylow theorems, $n_{13} \equiv 1 \bmod 13$ and $n_{13} \mid 8$, which implies $n_{13} = 1$.

Let $N$ be the unique sylow $13$-subgroup of $G$. Then $|N| = 13$ and $N \teq G$.

If $H$ is a sylow $2$-subgroup of $G$ then $|H| = 8$.

We conclude that $|NH| = |N||H| = 13 \cdot 8 = |G|$, so $G \isom N \rtimes_\phi H$ where $\phi : H \to \Aut(N)$.

Since $|N| = 13$, then $N \isom Z_{13}$ so $\Aut(N) \isom Z_{12}$.

Note that $8 = |H| = |\ker\phi||\im\phi|$ and $\im\phi \leq \Aut(N)$ so $|\im\phi|$ divides $\gcd(8, 12) = 4$.

In particular, if $\Aut(N) = \<\sigma\> \isom Z_{12}$ then $\im\phi \leq \<\sigma^3\> \isom Z_4$.

\textbf{Case $H \isom Z_8$.}

Say $H = \<y \mid y^8 = 1\>$.

If $\ker\phi = H$ then (1) $G \isom Z_{13} \times Z_8$

If $\ker\phi = \<y^2\>$ then $\im\phi = \<\sigma^6\> \isom Z_2$ and (2) $G \isom Z_{13} \rtimes Z_8$ with $yxy^{-1} = \sigma^6(x) = x^{12}$.

If $\ker\phi = \<y^4\>$ then $\im\phi = \<\sigma^3\> \isom Z_4$ and (3) $G \isom Z_{13} \rtimes Z_8$ with $yxy^{-1} = \sigma^3(x) = x^8$.

\textbf{Case $H \isom Z_4 \times Z_2$.}

Say $H = \<y, z \mid y^4 = z^2 = [y, z] = 1\>$.

If $\ker\phi = H$ then (4) $G \isom Z_{13} \times Z_4 \times Z_2$.

If $\ker\phi = \<y\>$ then $\im\phi = \<\sigma^6\> \isom Z_2$ and (5) $G \isom (Z_{13} \rtimes Z_2) \times Z_4$ with $zxz^{-1} = x^{12}$, i.e., $D_{2(13)} \times Z_4$.

If $\ker\phi = \<y^2, z\>$ then $\im\phi = \<\sigma^6\> \isom Z_2$ and (6) $G \isom (Z_{13} \rtimes Z_4) \times Z_2$ with $yzy^{-1} = x^{12}$, i.e., $G \isom BD_{4(13)} \times Z_2$.

Where $BD_{4n} = \<a, b \mid a^{2n} = 1, a^n = b^2, bab^{-1} = a^{-1}\>$ take $a = xy^2$, $b = y$.


If $\ker\phi = \<z\>$ then $\im\phi = \<\sigma^3\> \isom Z_4$ and (7) $G \isom (Z_{13} \rtimes Z_4) \times Z_2$ with $yxy^{-1} = x^8$.


\textbf{Case $H \isom Z_2^3$.} 

Say $H \isom \<y, z, w \mid y^2 = z^2 = w^2 = [y, z] = [y, w] = [z, w] = 1\>$.

If $\ker\phi = H$ then (8) $G \isom Z_{13} \times Z_2^3$.

If $\ker\phi = \<z, w\>$ then $\im\phi = \<\sigma^6\> \isom Z_2$ and (9) $G \isom (Z_{13} \rtimes Z_2) \times Z_2^2$ with $yxy^{-1} = x^{12}$, i.e., $D_{2(13)} \times Z_2^2$.

\textbf{case $H \isom D_{2(4)}$.} 

Say $H = \<y, z \mid y^4 = z^2 = 1, zyz = y^{-1}\>$, 

If $\ker\phi = H$ then (10) $G \isom Z_{13} \times D_{2(4)}$.

If $\ker\phi = \<y\> \isom Z_4$ then $\im\phi = \<\sigma^6\> \isom Z_2$ and (11) $G \isom Z_{13} \rtimes D_{2(4)}$ with $yxy^{-1} = x$ and $zxz^{-1} = x^{12}$, i.e., $D_{2(52)}$ where $\tau = xy$ and $\sigma = z$

If $\ker\phi = \<y^2, z\> \isom \<y^2, yz\> \isom Z_2 \times Z_2$ then $\im\phi = \<\sigma^6\> \isom Z_2$ and (12) $G \isom Z_{13} \rtimes D_{2(4)}$ with $yxy^{-1} = x^{12}$ and $zxz^{-1} = x$.

\textbf{Case $H \isom Q_8$.}

Say $H = \<y, z \mid y^4 = 1, y^2 = z^2, zyz^{-1} = y^{-1}\>$.

If $\ker\phi = H$ then (13) $G \isom Z_{13} \times Q_8$.

If $\ker\phi = \<y\> \isom Z_4$ then $\im\phi = \<\sigma^6\> \isom Z_2$ and (14) $G \isom Z_{13} \rtimes Q_8$ with $yxy^{-1} = x^{12}$ and $zxz^{-1} = x$, i.e., $BD_{4(26)}$ where $a = xz$ and $b = y$.

\newpage

\begin{enumerate}
    \item $Z_{13} \times Z_8$
    \item $Z_{13} \rtimes Z_8$ by $Z_8/Z_4 = Z_2$
    \item $Z_{13} \rtimes Z_8$ by $Z_8/Z_2 = Z_4$
    \\
    \item $Z_{13} \times Z_4 \times Z_2$
    \item $D_{2(13)} \times Z_4$
    \item $BD_{4(13)} \times Z_2$
    \item $(Z_{13} \rtimes Z_4) \times Z_2$ by $Z_4/1 = Z_4$
    \\
    \item $Z_{13} \times Z_2^3$
    \item $D_{2(13)} \times Z_2^2$
    \\
    \item $Z_{13} \times D_{2(4)}$
    \item $D_{2(52)}$
    \item $Z_{13} \rtimes D_{2(4)}$ by $D_{2(4)}/Z_2^2 = Z_2$
    \\
    \item $Z_{13} \times Q_8$
    \item $BD_{4(26)}$
\end{enumerate}


\newpage

\begin{enumerate}
    \item $Z_{13} \rtimes Z_8$
    \begin{enumerate}
        \item $Z_{13} \times Z_8$ or $Z_{104}$
        \item $Z_8/Z_2 \to Z_4$
        \item $Z_8/Z_4 \to Z_2$
    \end{enumerate}
    \item $Z_{13} \rtimes (Z_4 \times Z_2)$
    \begin{enumerate}
        \item $Z_{13} \times Z_4 \times Z_2$ or $Z_{52} \times Z_2$
        \item $(Z_4 \times Z_2)/(Z_4 \times 1) \to Z_2$ or $D_{2(13)} \times Z_4$
        \item $(Z_4 \times Z_2)/(Z_2 \times Z_2) \to Z_2$ or $BD_{4(13)} \times Z_2$
        \item $(Z_4 \times Z_2)/(1 \times Z_2) \to Z_4$ or $(Z_{13} \rtimes Z_4) \times Z_2$
    \end{enumerate}
    \item $Z_{13} \rtimes (Z_2^3)$
    \begin{enumerate}
        \item $Z_{13} \times Z_2^3$
        \item $Z_2^3/(Z_2^2 \times 1) \to Z_2$ or $D_{2(13)} \times Z_2^2$
    \end{enumerate}
    \item $Z_{13} \rtimes D_{2(4)}$
    \begin{enumerate}
        \item $Z_{13} \times D_{2(4)}$
        \item $D_{2(4)}/\<\tau\> = D_{2(4)}/Z_4 \to Z_2$ or $D_{2(52)}$
        \item $D_{2(4)}/\<\sigma, \tau^2\> = D_{2(4)}/Z_2^2 \to Z_2$
    \end{enumerate}
    \item $Z_{13} \rtimes Q_8$
    \begin{enumerate}
        \item $Z_{13} \times Q_8$
        \item $Q_8/\<i\> = Q_8/Z_4 \to Z_2$ or $BD_{4(26)}$
    \end{enumerate}
\end{enumerate}



\end{document}