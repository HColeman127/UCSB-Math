\documentclass[12pt]{article}

% Packages
\usepackage[margin=1in]{geometry}
\usepackage{fancyhdr, parskip}
\usepackage{amsmath, amsthm, amssymb}

% commutative diagrams
\usepackage{tikz-cd}
\tikzcdset{row sep/normal=0pt}
\newenvironment{cd}{\begin{center}\begin{tikzcd}}{\end{tikzcd}\end{center}}

% Page Style
\makeatletter
\fancypagestyle{title}{
    \renewcommand{\headrulewidth}{0.4pt}
    \setlength{\headheight}{15pt}
    \fancyhead[R]{\@author}
    \fancyhead[L]{\@title}
    \fancyhead[C]{\@date}
}
\makeatother
\renewcommand{\maketitle}{\thispagestyle{title}}
\fancypagestyle{plain}{
    \fancyhf{}
    \renewcommand{\headrulewidth}{0pt}
    \renewcommand{\footrulewidth}{0pt}
    \fancyfoot[R]{\thepage}
}
\pagestyle{plain}

% Problem Box
\setlength{\fboxsep}{4pt}
\newlength{\myparskip}
\setlength{\myparskip}{\parskip}
\newsavebox{\savefullbox}
\newenvironment{fullbox}{\begin{lrbox}{\savefullbox}\begin{minipage}{\dimexpr\textwidth-2\fboxsep\relax}\setlength{\parskip}{\myparskip}}{\end{minipage}\end{lrbox}\framebox[\textwidth]{\usebox{\savefullbox}}}
\newenvironment{pbox}[1][]{\begin{fullbox}\ifx#1\empty\else\paragraph{#1}\fi}{\end{fullbox}}

% Default Commands
\newcommand{\isp}[1]{\quad\text{#1}\quad}
\newcommand{\N}{\mathbb{N}} 
\newcommand{\Z}{\mathbb{Z}}
\newcommand{\Q}{\mathbb{Q}}
\newcommand{\R}{\mathbb{R}}
\newcommand{\C}{\mathbb{C}}
\newcommand{\eps}{\varepsilon}
\renewcommand{\phi}{\varphi}
\renewcommand{\emptyset}{\varnothing}
\newcommand{\<}{\langle}
\renewcommand{\>}{\rangle}
\newcommand{\isom}{\cong}
\newcommand{\eqc}{\overline}
\newcommand{\clo}{\overline}
\newcommand{\teq}{\trianglelefteq}
\DeclareMathOperator{\id}{id}

% Extra Commands
\DeclareMathOperator{\im}{im}
\DeclareMathOperator{\GL}{GL}
\newcommand{\mat}[1]{\begin{bmatrix}#1\end{bmatrix}}

% Document
\begin{document}
\title{MATH 220A Homework 5}
\author{Harry Coleman\makebox[0pt][r]{\raisebox{-0.25in}[0pt][0pt]{(worked with Joseph Sullivan)}}}
\date{November 1, 2021}
\maketitle

\begin{pbox}[1 Exercise I.13 \\ (a)]
    Let $H$, $N$ be normal subgroups of a finite group $G$. Assume that the orders of $H$, $N$ are relatively prime. Prove that $xy = yx$ for all $x \in H$ and $y \in N$, and that $H \times N \isom HN$.
\end{pbox}

\begin{proof}
    Since $H \cap N$ is a subgroup of both $H$ and $N$, its order divides $\gcd(|H|, |N|) = 1$, so it must be the trivial subgroup. Hence, we have an isomorphism
    \begin{cd}
        H \rar["\sim"] & H/(H \cap N) \rar["\sim"] & HN/N, \\
        x \ar[rr, mapsto] & & xN.
    \end{cd}
    Given $x \in H$ and $y \in N$, we have $yxy^{-1} \in H$, since $H$ is normal. This map sends $yxy^{-1}$ to
    \[
        yxy^{-1}N = yxN = yNx = Nx = xN,
    \]
    which is also the image of $x$. Since this is an isomorphism, we conclude that $yxy^{-1} = x$, or, equivalently, that $xy = yx$.

    By Proposition 2.1 in Lang, the map
    \begin{cd}
        H \times N \rar & HN, \\
        (x, y) \rar[mapsto] & xy,
    \end{cd}
    is a group isomorphism.
\end{proof}

\begin{pbox}[(b)]
    Let $H_1, \dots, H_r$ be normal subgroups of $G$ such that the order of $H_i$ is relatively prime to the order of $H_j$ for $i \ne j$. Prove that
    \[
        H_1 \times \cdots \times H_r \isom H_1 \cdots H_r.
    \]
\end{pbox}

\begin{proof}
    We perform induction on $r$. The result is trivial for $r = 1$. 

    Assume that the result holds for any $r - 1$ normal subgroups with mutually relatively prime orders, so $H_1 \times \cdots \times H_{r-1} \isom H_1 \cdots H_{r-1}$. Then $H_1 \cdots H_{r-1}$ is a normal subgroup of $G$ with order $|H_1| \cdots |H_{r-1}|$, which is relatively prime to $|H_r|$. Applying the inductive hypothesis, followed by part (a), we deduce
    \[
        (H_1 \times \cdots \times H_{r-1}) \times H_r 
            \isom (H_1 \cdots H_{r-1}) \times H_r
            \isom H_1 \cdots H_{r-1}H_r.
    \]

\end{proof}


\newpage
\begin{pbox}[2 Exercise I.20]
    Let $P$ be a $p$-group. Let $A$ be a normal subgroup of order $p$. Prove that $A$ is contained in the center of $P$.
\end{pbox}

\begin{proof}
    Since the order of $A$ is prime, it is cyclic; say $A = \<x\> \isom \Z/p\Z$. The fact that $A$ is normal implies $yxy^{-1} \in A$, for all $y \in P$. In particular, $yxy^{-1} = x^k$ with $0 < k < p$. So, we can define a group homomorphism $\alpha : P \to (\Z/p\Z)^\times$ such that $yxy^{-1} = x^{\alpha(y)}$, for all $y \in P$. Then $\im\alpha$ is a subgroup of $(\Z/p\Z)^\times$, so its order divides $p - 1$. On the other hand, $|\im\alpha| = [P : \ker\alpha]$ divides $|P|$, so $|\im\alpha|$ must be a power of $p$. Therefore, $\im\alpha$ must be trivial, implying $yxy^{-1} = x^{\alpha(y)} = x^1$. Since the generator of $A$ commutes with every element of $P$, so does every other element of $A$, hence $A \subseteq Z(P)$.

\end{proof}



\newpage
\begin{pbox}[3 Exercise I.32]
    Let $S_n$ be the permutation group on $n$ elements. Determine the $p$-Sylow subgroups of $S_3$, $S_4$, $S_5$ for $p = 2$ and $p = 3$.
\end{pbox}

\paragraph{$\mathbf{S_3}$} We have $n_2 \equiv 1 \pmod{2}$ and $n_2 \mid 3$, implying that $n_2 \in \{1, 3\}$. There are at least three distinct Sylow $2$-subgroups:
\[
    \<(1 \; 2)\>, \quad \<(1 \; 3)\>, \quad \<(2 \; 3)\>.
\]
Therefore, $n_2 = 3$, and these are the only Sylow $2$-subgroups.

We have $n_3 \equiv 1 \pmod{3}$ and $n_3 \mid 2$, implying that $n_3 = 1$. The unique Sylow $3$-subgroup is
\[
    \<(1 \; 2 \; 3)\>.
\]

\paragraph{$\mathbf{S_4}$} We have $n_2 \equiv 1 \pmod{2}$ and $n_2 \mid 3$, implying that $n_2 \in \{1, 3\}$. We can embed the dihedral group
\[
    D_8 = \<\sigma, \tau \mid \sigma^4 = \tau^2 = 1, \tau\sigma\tau = \sigma^{-1}\>
\]
into $S_4$ in three distinct ways (arising from labeling the vertices of a square):
\[
    \sigma = (1 \; 2 \; 3 \; 4), \tau = (2 \; 4); \quad
    \sigma = (1 \; 2 \; 4 \; 3), \tau = (2 \; 3); \quad
    \sigma = (1 \; 3 \; 2 \; 4), \tau = (3 \; 4).
\]
Therefore, $n_2 = 3$, and these are the only Sylow $2$-subgroups.

We have $n_3 \equiv 1 \pmod{3}$ and $n_3 \mid 8$, implying that $n_3 \in \{1, 4\}$. We can embed the cyclic group
\[
    \Z/3\Z \isom \<\alpha \mid \alpha^3 = 1\>
\]
into $S_4$ in four distinct ways:
\[
    \alpha = (1 \; 2 \; 3), \quad \alpha = (1 \; 2 \; 4), \quad \alpha = (1 \; 3 \; 4), \quad \alpha = (2 \; 3 \; 4).
\]
Therefore, $n_3 = 4$, and these are the only Sylow $3$-subgroups.


\paragraph{$\mathbf{S_5}$} We have $n_2 \equiv 1 \pmod{2}$ and $n_2 \mid 15$, implying that $n_2 \in \{1, 3, 5, 15\}$. We can embed the symmetric group $S_4$ into $S_5$ in five distinct ways: one being the natural inclusion (i.e., treating every permutation of four elements as a permutation of five elements, always fixing the fifth), and the other four being obtained by replacing any occurrence of some chosen index ($1$, $2$, $3$, or $4$) with $5$ in each permutation. From our previous analysis, we know that $D_8$ can be embedded into $S_4$ in three ways, and composing with the five embeddings of $S_4$ into $S_5$, we obtain fifteen distinct embeddings of $D_8$ into $S_5$. (The fact that all fifteen are distinct can be seen from the fact that each embedding of $D_8$ into $S_4$ utilizes all four indices, and each embedding of $S_4$ differs by exactly one index.) Therefore, $n_2 = 15$, and these are the only Sylow $2$-subgroups.

We have $n_3 \equiv 1 \pmod{3}$ and $n_3 \mid 40$, implying that $n_3 \in \{1, 4, 10\}$. We can embed the cyclic group $\Z/3\Z$ into $S_5$ in $\binom{5}{3} = 10$ distinct ways:
\[
    \alpha = (i \; j \; k), \text{ for each choice of } \{i, j, k\} \subseteq \{1, 2, 3, 4, 5\}.
\]
(One can check that these can also be obtained by composing embeddings of the cyclic group into $S_4$, then $S_4$ into $S_5$, but that there will be overlap.) Therefore, $n_3 = 10$, and these are the only Sylow $3$-subgroups.



\newpage
\begin{pbox}[4 Exercise I.55]
    Let $M \in \GL_2(\C)$. We let
    \[
        M = \mat{a & b \\ c & d}, \text{ and for } z \in \C \text{ we let } M(z) = \frac{az + b}{cz + d}.
    \]
    If $z = -d/c$ ($c \ne 0$) then we put $M(z) = \infty$. Then you can verify that $\GL_2(\C)$ thus operates on $\C \cup \{\infty\}$. Let $\lambda$, $\lambda'$ be the eigenvalues of $M$ viewed as a linear map on $\C^2$. Let $W$, $W'$ be the corresponding eigenvectors,
    \[
        W = \mat{w_1 & w_2}^T \isp{and} W' = \mat{w_1' & w_2'}^T.
    \]
    Assume that $M$ has two distinct fixed points $\ne \infty$.
\end{pbox}

\begin{pbox}[(a)]
    Show that there cannot be more than two fixed points and that these fixed points are $w = w_1/w_2$ and $w' = w_1'/w_2'$. In fact one may take
    \[
        W = \mat{w & 1}^T, \quad W' = \mat{w' & 1}^T.
    \]
\end{pbox}

\begin{proof}
    If $w$ is a fixed point of $M$, $w = (aw + b)/(cw + d)$ implies $w$ is a root of the complex polynomial
    \[
        cz^2 + (d - a)z + b = 0.
    \]
    Since a complex quadratic has to most two distinct complex roots, $M$ has at most two distinct fixed points.

    We now check that $w_1/w_2$ is a fixed point of $M$:
    \[
        M(w_1/w_2)
            = \frac{aw_1/w_2 + b}{cw_1/w_2 + d}
            = \frac{aw_1 + bw_2}{cw_1 + dw_2}
            = \frac{(MW)_1}{(MW)_2}
            = \frac{(\lambda W)_1}{(\lambda W)_2}
            = \frac{\lambda w_1}{\lambda w_2}
            = \frac{w_1}{w_2}.
    \]
    The same is true of $w_1'/w_2'$.

    By linearity, any nonzero scalar of an eigenvector is, again, an eigenvector with the same eigenvalue. So $\frac{1}{w_2}W = \mat{w & 1}^T$ and $\frac{1}{w_2'}W' = \mat{w' & 1}^T$ are eigenvectors for $\lambda$ and $\lambda'$, respectively.

\end{proof}

\newpage
\begin{pbox}[(b)]
    Assume that $|\lambda| < |\lambda'|$. Given $z \ne w$, show that
    \[
        \lim_{k \to \infty} M^k(z) = w'.
    \]
    [\textit{Hint}: Let $S = \mat{W & W'}$ and consider $S^{-1}M^kS(z) = \alpha^kz$ where $\alpha = \lambda/\lambda'$.]
\end{pbox}

\begin{proof}
    We compute
    \begin{align*}
        S^{-1}MS
            &= S^{-1}M\mat{W & W'} \\
            &= S^{-1}\mat{\lambda W & \lambda' W'} \\
            &= S^{-1}\mat{\lambda w & \lambda' w' \\ \lambda & \lambda'} \\
            &= S^{-1}S\mat{\lambda & 0 \\ 0 & \lambda'} \\
            &= \mat{\lambda & 0 \\ 0 & \lambda'},
    \end{align*}
    so
    \[
        S^{-1}M^kS = (S^{-1}MS)^k = \mat{\lambda^k & 0 \\ 0 & (\lambda')^k}.
    \]
    Thus, as per the hint, $S^{-1}M^kS(z) = \alpha^kz$ where $\alpha = \lambda/\lambda'$. Since $0 \leq \alpha < 1$, we have the limit $a^kz \to 0$. Note that fractional linear transformations are bicontinuous, so
    \begin{align*}
        \lim_{k \to \infty} M^k(z)
            &= SS^{-1}\left(\lim_{k \to \infty} M^k(SS^{-1}(z))\right) \\
            &= S\left(\lim_{k \to \infty} S^{-1}M^kS(S^{-1}(z))\right) \\
            &= S\left(\lim_{k \to \infty} \alpha^kS^{-1}(z)\right) \\
            &= S(0) \\
            &= \frac{w \cdot 0 + w'}{1 \cdot 0 + 1} \\
            &= w'.
    \end{align*}
    (Note that $S^{-1}(z)$ contains $w - z$ in its denominator, so we require that $z \ne w$ for the third step to make sense.)

\end{proof}

\end{document}