\documentclass[12pt]{article}

% Packages
\usepackage[margin=1in]{geometry}
\usepackage{fancyhdr, parskip}
\usepackage{amsmath, amsthm, amssymb}

% Page Style
\fancypagestyle{plain}{
    \fancyhf{}
    \renewcommand{\headrulewidth}{0pt}
    \renewcommand{\footrulewidth}{0pt}
    \fancyfoot[R]{\thepage}
}
\pagestyle{plain}

% Problem Box
\setlength{\fboxsep}{4pt}
\newsavebox{\savefullbox}
\newenvironment{fullbox}{\begin{lrbox}{\savefullbox}\begin{minipage}{\dimexpr\textwidth-2\fboxsep\relax}}{\end{minipage}\end{lrbox}\begin{center}\framebox[\textwidth]{\usebox{\savefullbox}}\end{center}}
\newenvironment{pbox}[1][]{\begin{fullbox}\ifx#1\empty\else\paragraph{#1}\fi}{\end{fullbox}}

% Theorem Environments
\theoremstyle{definition}
%\newtheorem{proposition}{Proposition}
\newtheorem{lemma}{Lemma}

% Options
%\allowdisplaybreaks
%\addtolength{\jot}{4pt}

% Default Commands
\newcommand{\isp}[1]{\quad\text{#1}\quad}
\newcommand{\N}{\mathbb{N}} 
\newcommand{\Z}{\mathbb{Z}}
\newcommand{\Q}{\mathbb{Q}}
\newcommand{\R}{\mathbb{R}}
\newcommand{\C}{\mathbb{C}}
\newcommand{\eps}{\varepsilon}
\renewcommand{\phi}{\varphi}
\renewcommand{\emptyset}{\varnothing}
\newcommand{\<}{\langle}
\renewcommand{\>}{\rangle}
\newcommand{\isom}{\cong}
\newcommand{\eqc}{\overline}
\newcommand{\clo}{\overline}

% Extra Commands
\DeclareMathOperator{\id}{id}
\DeclareMathOperator{\Aut}{Aut}
\DeclareMathOperator{\Gal}{Gal}
\newcommand{\teq}{\trianglelefteq}
\newcommand{\inc}{\hookrightarrow}

% Document Info
\fancypagestyle{title}{
    \renewcommand{\headrulewidth}{0.4pt}
    \setlength{\headheight}{15pt}
    \fancyhead[R]{Harry Coleman}
    \fancyhead[L]{MATH 111C Homework 6}
    \fancyhead[C]{May 21, 2021}
}

% Begin Document
\begin{document}
\thispagestyle{title}


\begin{pbox}[Q1 Problem 14.2.15]
    (\textit{Biquadratic Extensions}) Let $F$ be a field of characteristic $\ne 2$.
\end{pbox}

In particular, $\operatorname{char}F \ne 2$ implies that $2 = 1_F + 1_F \in F^\times$; this fact will be used implicitly in the following proofs. 

\begin{pbox}[(a)]
    If $K = F(\sqrt{D_1}, \sqrt{D_2})$ where $D_1, D_2 \in F$ have the property that none of $D_1$, $D_2$, or $D_1D_2$ is a square in $F$, prove that $K/F$ is a Galois extension with $\Gal(K/F)$ isomorphic to the Klein $4$-group.
\end{pbox}

\begin{proof}
    Since $K$ is a splitting field for the separable polynomial $(x^2 - D_1)(x^2 - D_2) \in F[x]$, then $K/F$ is a Galois extension. An automorphism of $K = F(\sqrt{D_1}, \sqrt{D_2})$ fixing $F$ (i.e., an element of $\Gal(K/F)$) is completely determined by the images of $\sqrt{D_1}$ and $\sqrt{D_2}$. Moreover, each must map to a root of its minimal polynomial over $F$, i.e.,
    \[
        \sqrt{D_1} \mapsto \pm\sqrt{D_1} \isp{and} \sqrt{D_2} \mapsto \pm\sqrt{D_2}.
    \]
    There are four such maps, namely $\id_K, \sigma, \tau, \sigma\tau$, where $\sigma$ and $\tau$ are maps from $K$ to itself defined by
    \[
        \sigma : \begin{cases}
            \sqrt{D_1} \mapsto -\sqrt{D_1}, \\
            \sqrt{D_2} \mapsto \sqrt{D_2},
        \end{cases}
        \isp{and}
        \tau : \begin{cases}
            \sqrt{D_1} \mapsto \sqrt{D_1}, \\
            \sqrt{D_2} \mapsto -\sqrt{D_2}.
        \end{cases}
    \]
    We will show that there are four distinct elements of $\Gal(K/F)$, proving that the above four maps are in fact automorphisms of $K$ fixing $F$.

    Since $K/F$ is Galois,
    \[
        |\Gal(K/F)| = [K : F] = [K : F(\sqrt{D_1})][F(\sqrt{D_1}) : F] = [K : F(\sqrt{D_1})] \cdot 2.
    \]
    We claim that $[K : F(\sqrt{D_1})] = 2$, and will prove this by showing that the minimal polynomial of $\sqrt{D_2}$ over $F(\sqrt{D_1})$ is the same as the minimal polynomial over $F$, i.e.,
    \[
        m_{\sqrt{D_2}, F(\sqrt{D_1})}(x) = m_{\sqrt{D_2}, F}(x) = x^2 - D_2.
    \]
    Since the minimal polynomial over $F(\sqrt{D_1})$ divides $x^2 - D_2$, it suffices to show that $\sqrt{D_2} \notin F(\sqrt{D_1})$. Suppose to the contrary that $D_2$ is the square of some $a + b\sqrt{D_1} \in F(\sqrt{D_1})$ for $a, b \in F$, then
    \[
        D_2 = (a + b\sqrt{D_1})^2 = a^2 + 2ab\sqrt{D_1} + b^2D_1.
    \]
    It cannot be the case that $a = 0$; otherwise $D_2 = b^2D_1$, implying that
    \[
        D_1D_2 = D_1(b^2D_1) = (bD_1)^2
    \]
    is a square in $F$, which is false by assumption. Similarly, it cannot be the case that $b = 0$; otherwise $D_2 = a^2$ is a square in $F$. Therefore, with $a, b, D_2 \in F$ nonzero, we find that
    \[
        \sqrt{D_1} = \frac{D_2 - a^2 - b^2D_2}{2ab} \in F,
    \]
    which is a contradiction. Hence, $\sqrt{D_2} \notin F(\sqrt{D_1})$, so indeed
    \[
        |\Gal(K/F)| = [K : F(\sqrt{D_1})] \cdot 2 = 4.
    \]

    From this, we deduce that $\Gal(K/F) = \{\id_K, \sigma, \tau, \sigma\tau\}$, where $\sigma$ and $\tau$ are as above. There is now an obvious isomorphism 
    \begin{align*}
        \Gal(K/F) &\xrightarrow{\sim} \Z/2\Z \times \Z/2\Z \\
            \sigma &\mapsto (1, 0) \\
            \tau &\mapsto (0, 1).
    \end{align*}
    That is, $\Gal(K/F)$ is isomorphic to the Klein $4$-group.

\end{proof}

\begin{pbox}[(b)]
    Conversely, suppose that $K/F$ is a Galois extension with $\Gal(K/F)$ isomorphic to the Klein $4$-group. Prove that $K = F(\sqrt{D_1}, \sqrt{D_2})$ where $D_1, D_2 \in F$ have the property that none of $D_1$, $D_2$, or $D_1D_2$ is a square in $F$.
\end{pbox}

\begin{lemma}
    If $L/F$ is a field extension with $[L : F] = 2$, then $L = F(\sqrt{D})$ for some $D \in F$ such that $m_{\sqrt{D}, F} = x^2 - D \in F[x]$.
\end{lemma}

\begin{proof}
    For any element $\alpha \in L \setminus F$, the degree of the minimal polynomial of $\alpha$ over $F$ must divide $[L : F] = 2$, snd since $\alpha \notin F$, then the degree must be exactly $2$. Then
    \[
        m_{\alpha, F}(x) = x^2 + bx + c \in F[x],
    \]
    which we can rewrite to be
    \[
        \left(x + \tfrac{b}{2}\right)^2 - \left(\tfrac{b^2}{4} - c\right) \in F[x].
    \]
    Define $D = \tfrac{b^2}{4} - c \in F$, then
    \[
        D = \left(\alpha + \tfrac{b}{2}\right)^2.
    \]
    Naturally, we define $\sqrt{D} = \alpha + \tfrac{b}{2}$, which is an element of $L$, but not of $F$, with
    \[
        m_{\sqrt{D}, F} = x^2 - D \in F[x].
    \]
    The fact that $x^2 - D$ is irreducible in $F[x]$ can be seen by that fact that its roots are precisely $\pm\sqrt{D} \notin F$. Since $F(\sqrt{D})/F$ is a subextension of $L/F$ of degree $2$, and $[L : F] = 2$, then we must have $L = F(\sqrt{D})$. 

\end{proof}

\newpage
Denote by $Z_2$ the cyclic group of 2 elements: $\Z/2\Z$. We now prove the main result.

\begin{proof}
    Given that $\Gal(K/F)$ is isomorphic to the Klein $4$-group ($Z_2 \times Z_2$), then there is a normal subgroup $H \teq \Gal(K/F)$ isomorphic to $Z_2$ (explicitly, we could take $H \isom Z_2 \times \{0\}$, under the same isomorphism which gives us $\Gal(K/F) \isom Z_2 \times Z_2$). By the fundamental theorem of Galois theory, this corresponds to a Galois extension $K^H/F$, where $K^H$ is the subfield of $K$ fixed by $H$. Then
    \[
        \Gal(K^H/F) \isom \Gal(K/F)/H \isom (Z_2 \times Z_2)/(Z_2\times \{0\}) \isom Z_2,
    \]
    from which we deduce that
    \[
        [K^H : F] = |\Gal(K^H/F)| = |Z_2| = 2.
    \]
    By Lemma 1, $K^H = F(\sqrt{D_1})$ for some $D_1 \in F$. In particular, $D_1$ is not a square in $F$, since the only roots of $m_{\sqrt{D_1}, F}(x) = x^2 - D_1$ are $\pm\sqrt{D_1} \notin F$.

    Since we also have
    \[
        [K : K^H] = |H| = |Z_2 \times \{0\}| = 2,
    \]
    then, by Lemma 1, $K = K^H(\sqrt{\beta})$ for some $\beta \in K^H$ such that $m_{\sqrt{\beta}, K^H}(x) = x^2 - \beta \in K^H[x]$. Since $K^H = F(\sqrt{D_1})$, then $\beta = a + b\sqrt{D_1}$ for some $a, b \in F$. We now write
    \[
        m_{\sqrt{\beta}, K^H}(x) = \left(x - \tfrac{b}{2}\sqrt{D_1}\right)^2 - \left(\tfrac{b^2}{4}D_1 - a\right).
    \]    
    Define $D_2 = \tfrac{b^2}{4}D_1 - a \in F$, then
    \[
        D_2 = \left(\sqrt{\beta} - \tfrac{b}{2}\sqrt{D_1}\right)^2.
    \]
    Naturally, we define $\sqrt{D_2} = \sqrt{\beta} - \tfrac{b}{2}\sqrt{D_1}$, which is an element of $K$, but not $K^H \supseteq F$, with
    \[
        m_{\sqrt{D_2}, K^H}(x) = x^2 - D_2 \in F[x].
    \]
    The fact that $x^2 - D_2$ is irreducible in $K^H[x]$ follows from the fact that its only roots are $\pm\sqrt{D_2} \notin K^H$. In particular, $D_2$ is not a square in $F$. Since $K^H(\sqrt{D_2})/K^H$ is a degree $2$ subextension of $K/K^H$ and $[K : K^H] = 2$, then we must have
    \[
        K = K^H(\sqrt{D_2}) = F(\sqrt{D_1}, \sqrt{D_2}).
    \]

    It remains to show that $D_1D_2$ is not a square in $F$. The the only roots of $x^2 - D_1D_2 \in F[x]$ are $\pm\sqrt{D_1}\sqrt{D_2} \in K$. If it were the case that $\sqrt{D_1}\sqrt{D_2} = a$ for some $a \in F$, then we would have
    \[
        \sqrt{D_2} = (\sqrt{D_1})^{-1}a \in F(\sqrt{D_1}) = K^H,
    \]
    which is not the case. Hence, neither $D_1$, $D_2$, nor $D_1D_2$ is a square in $F$.

\end{proof}



\newpage
\begin{pbox}[Q2]
    Let $K/F$ be a separable finite extension. Show that $K$ has finitely many subfields containing $F$.
\end{pbox}

\begin{proof}
    Fix an algebraic closure $\clo{F} = \clo{K}$ of $F$ containing $K$. Since $K/F$ is a finite extension, then $K = F(\alpha_1, \dots, \alpha_n)$ for some algebraic elements $\alpha_1, \dots, \alpha_n \in K$. Define $S \subseteq \clo{F}$ to be the set of all roots in $\clo{F}$ of the minimal polynomials $m_{\alpha_j, F}(x)$ for $j = 1, \dots, n$. Since $K$ is separable, $F(S)$ is a Galois closure of $K$ over $F$. In particular, $S$ is a finite set, so
    \[
        [F(S) : F] = |\Gal(F(S)/F)|
    \]
    is finite. Therefore, $\Gal(F(S)/F)$ has finitely many subgroups.

    Every subfield of $K$ containing $F$ is a subextension of $F(S)/F$, so there are at least as many subextensions of $F(S)/F$ as there are subfields of $K$ containing $F$. Since the subextensions of $F(S)/F$ correspond bijectively (by the fundamental theorem of Galois theory) to the subgroups of $\Gal(F(S)/F)$, then there are finitely many subextensions of $F(S)/F$. Hence, there are finitely many subfields of $K$ containing $F$.

\end{proof}



\newpage
\begin{pbox}[Q3]
    Let $K$ be the Galois closure of $\Q(\sqrt{1 + \sqrt{3}})$.
\end{pbox}

\begin{pbox}[(a)]
    Show that $[K : \Q] = 8$.
\end{pbox}

\begin{proof}
    The polynomial $f(x) = x^4 - 2x^2 - 2$ is irreducible in $\Q[x]$, by Eisenstein's criterion, and has $\sqrt{1 + \sqrt{3}}$ as a root; so $f(x)$ is the minimal polynomial of $\sqrt{1 + \sqrt{3}}$ over $\Q$. Note that $f(x)$ is indeed separable, with the four distinct roots $\pm\sqrt{1 \pm \sqrt{3}} \in \clo{\Q}$, so the definition of $K$ as a Galois closure over $\Q$ makes sense. 
    
    We now consider the field $F = \Q(\sqrt{3})$ and define $D_1 = 1 + \sqrt{3}$ and $D_2 = 1 - \sqrt{3}$, which are elements of $F$. It can be seen that $F$ is a subfield of $K$ containing $\Q$, so $K/F$ is a Galois extension with
    \[
        K = \Q(\pm\sqrt{D_1}, \pm\sqrt{D_2}) = F(\sqrt{D_1}, \sqrt{D_2}).
    \]
    In order to apply Q1(a), we must check that none of $D_1$, $D_2$, or $D_1D_2$ are squares in $F$. For any $a + b\sqrt{3} \in F$, we have
    \[
        (a + b\sqrt{3})^2 = a^2 + 2ab\sqrt{3} + b^23 = (a^2 + 3b^2) + 2ab\sqrt{3}.
    \]
    One can check that no values of $a, b \in \Q$ will yield $D_1$, $D_2$, or $D_1D_2$.

    Applying Q1(a), we find $\Gal(K/F) \isom \Z/2\Z \times \Z/2\Z$, so
    \[
        [K : F] = |\Gal(K/F)| = |\Z/2\Z \times \Z/2\Z| = 4.
    \]
    Recall that $F = \Q(\sqrt{3})$, and we have seen that $[\Q(\sqrt{3}) : \Q] = 2$. Then
    \[
        [K : Q] = [K : F][F : \Q] = 4 \cdot 2 = 8.
    \]

    
\end{proof}

\begin{pbox}[(b)]
    Show that $\Gal(K/\Q)$ is not commutative.
\end{pbox}

This proof uses the result of (d), the proof of which does not rely on this result.

\begin{proof}
    From (d), $\Gal(K/\Q) \isom D_8$, and we know that the dihedral group $D_{2n}$ is non-abelian for $n \geq 3$. In particular, $sr = r^{n-1}s \ne rs$ when $n \geq 3$. Where $\sigma, \tau \in \Gal(K/\Q)$ as in (d), we can say more specifically that $\sigma\tau = \tau^3\sigma \ne \tau\sigma$. Hence, $\Gal(K/\Q)$ is non-abelian.

\end{proof}

\begin{pbox}[(c)]
    Show that $\Gal(K/\Q)$ has a normal subgroup isomorphic to $\Z/2\Z \times \Z/2\Z$.
\end{pbox}

\begin{proof}
    As in part (a), we have $F = \Q(\sqrt{3})$, so $F/\Q$ is a Galois extension (as the splitting field of the separable polynomial $x^2 - 3$). In particular, $F/\Q$ is a Galois subextension of the Galois extension $K/\Q$; the fundamental theorem of Galois theory implies $\Gal(K/F) \teq \Gal(K/\Q)$. And, from part (a), we know that $\Gal(K/F)$ is isomorphic to the Klein $4$-group.

\end{proof}


\newpage
\begin{pbox}[(d)]
    Show that $\Gal(K/\Q) \isom D_8$.
\end{pbox}

\begin{proof}
    We will use the fact that $|\Gal(K/\Q)| = [K : \Q] = 8$ to characterize the elements of $\Gal(K/\Q)$. Any automorphism of $K$ fixing $\Q$ is completely determined by the images of
    \[
        \sqrt{D_1} = \textstyle\sqrt{1 + \sqrt{3}} \isp{and} \sqrt{D_2} = \sqrt{1 - \sqrt{3}}.
    \]
    Moreover, such an automorphism must permute the roots of irreducible polynomials in $\Q[x]$; in particular $x^4 - 2x^2 - 2$, which has the four roots $\pm\sqrt{D_1}, \pm\sqrt{D_2}$. 
    
    Given $\sigma \in \Gal(K/\Q)$, we know that $\sigma(-\alpha) = - \sigma(\alpha)$ for all $\alpha \in K$. From the fact that $\sqrt{D_2} \ne \pm\sqrt{D_1}$, it follows that $\sigma(\sqrt{D_2}) \ne \pm\sigma(\sqrt{D_1})$. In other words,
    \begin{align*}
        \sigma : \sqrt{D_1} \mapsto \pm\sqrt{D_1} \quad&\implies\quad \sigma: \sqrt{D_2} \mapsto \pm\sqrt{D_2}, \\
        \sigma : \sqrt{D_1} \mapsto \pm\sqrt{D_2} \quad&\implies\quad \sigma: \sqrt{D_2} \mapsto \pm\sqrt{D_1}.
    \end{align*}
    This means that $\sigma$ can map $\sqrt{D_1}$ to any of the four options $\pm\sqrt{D_1}, \pm\sqrt{D_2}$, which determines the image of $-\sqrt{D_1}$. Once the images of $\pm\sqrt{D_1}$ is determined, $\sigma$ must map $\sqrt{D_2}$ to one of the two remaining options, which then determines the whole of sigma.
    
    From this, we deduce that there are eight possible automorphisms of $K$ fixing $\Q$ which permute the roots of $x^4 - 2x - 2$. Since $|\Gal(K/\Q)| = 8$, then $\Gal(K/\Q)$ is precisely the set of all those eight possible maps described in the previous paragraph. Define the following automorphism of $K$ fixing $\Q$:
    \[
        \sigma : \begin{cases}
            \sqrt{D_1} \mapsto \sqrt{D_2}, \\
            \sqrt{D_2} \mapsto \sqrt{D_1}. 
        \end{cases}
        \isp{and}
        \tau : \begin{cases}
            \sqrt{D_1} \mapsto \sqrt{D_2}, \\
            \sqrt{D_2} \mapsto -\sqrt{D_1}. 
        \end{cases}
    \]
    One can check that $\Gal(K/\Q) = \<\sigma, \tau\>$, i.e., all possible automorphisms of $K$ fixing $\Q$ can be written as a composition of finitely many copies of $\sigma$ and $\tau$. Moreover, it can be seen that $\tau^4 = \sigma^2 = (\sigma\tau)^2 = \id_K$, which suggests a natural choice of isomorphism
    \begin{align*}
        \Gal(K/\Q) &\xrightarrow{\sim} D_8 \\
            \sigma &\mapsto s \\
            \tau &\mapsto r.
    \end{align*}


\end{proof}



\newpage
\begin{pbox}[Q4]
    Show that if $K/\Q$ is a finite Galois extension with $\Gal(K/\Q) \isom S_3$, then $K$ is the splitting field for some irreducible cubic polynomial in $\Q[x]$.
\end{pbox}

\begin{proof}
    Representing $S_3$ as the set of permutations on three elements, we have
    \[
        S_3 = \{\id, (1\,2), (1\,3), (2\,3), (1\,2\,3), (1\,3\,2)\}.
    \]
    Let $\sigma, \tau \in \Gal(K/\Q)$ correspond to $(1\,2\,3), (1\,2) \in S_3$, respectively, under some fixed isomorphism of $\Gal(K/\Q) \isom S_3$. Then $\Gal(K/\Q) = \<\sigma, \tau\>$, and $\<\sigma\>$ is the only normal subgroup. We consider the non-normal subgroup $\<\tau\> = \{\id_K, \tau\}$, and the corresponding fixed field $K^{\<\tau\>}$. Since $K^{\<\tau\>}$ is a subfield of $K$ containing $\Q$, then $K^{\<\tau\>}/\Q$ is a non-Galois subextension of $K/\Q$; in particular, $K^{\<\tau\>}/\Q$ is separable but not normal.

    We see that $K^{\<\tau\>}/\Q$ is a finite extension and compute its degree to be 
    \[
        [K^{\<\tau\>} : \Q]
            = \frac{[K : \Q]}{[K : K^{\<\tau\>}]}
            = \frac{|\Gal(K/\Q)|}{|\<\tau\>|}
            = \frac{|S_3|}{2}
            = \frac{6}{2}
            = 3.
    \]
    As a finite separable extension, the primitive element theorem implies that $K^{\<\tau\>} = \Q(\alpha)$ for some $\alpha \in K^{\<\tau\>}$. Then
    \[
        \deg m_{\alpha, \Q}(x) = [\Q(\alpha) : \Q] = [K^{\<\tau\>} : \Q] = 3,
    \]
    which means that $m_{\alpha, \Q}(x)$ is an irreducible cubic polynomial in $\Q[x]$. Since $\Q(\alpha)/\Q$ is a separable non-normal extension, then $m_{\alpha, \Q}(x)$ is separable but does not split over $\Q(\alpha)$. However,  $m_{\alpha, \Q}(x)$ does split in $K[x]$ (because $K/\Q$ is Galois), so there is a splitting field of $m_{\alpha, \Q}(x)$ which is a subfield of $K$ strictly containing $\Q(\alpha)$.
    
    If $E$ is a subfield of $K$ containing $\Q(\alpha)$, then $[\Q(\alpha) : \Q] = 3$ must divide $[E : \Q]$ which, in turn, must divide $[K : \Q] = 6$. Therefore, $[E : \Q]$ must be either $3$, in which case $E = \Q(\alpha)$, or $6$, in which case $E = K$.

    Since the splitting field of $m_{\alpha, \Q}(x)$ is a subfield of $K$ strictly containing $\Q(\alpha)$ (i.e., $\ne \Q(\alpha)$), we conclude that $K$ is the splitting field of the irreducible cubic polynomial $m_{\alpha, \Q}(x) \in \Q[x]$.

\end{proof}

\end{document}