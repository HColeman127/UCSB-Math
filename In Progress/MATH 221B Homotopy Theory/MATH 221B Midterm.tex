\documentclass[12pt]{article}

% Packages
\usepackage[margin=1in]{geometry}
\usepackage{fancyhdr, parskip}
\usepackage{amsmath, amsthm, amssymb}
\usepackage{tikz, tikz-cd}
\usepackage[shortlabels]{enumitem}

% Page Style
\makeatletter
\fancypagestyle{title}{
    \renewcommand{\headrulewidth}{0.4pt}
    \setlength{\headheight}{15pt}
    \fancyhead[R]{\@author}
    \fancyhead[L]{\@title}
    \fancyhead[C]{\@date}
}
\makeatother
\renewcommand{\maketitle}{\thispagestyle{title}}
\fancypagestyle{plain}{
    \fancyhf{}
    \renewcommand{\headrulewidth}{0pt}
    \renewcommand{\footrulewidth}{0pt}
    \fancyfoot[R]{\thepage}
}
\pagestyle{plain}

% Problem Box
\setlength{\fboxsep}{4pt}
\newlength{\myparskip}
\setlength{\myparskip}{\parskip}
\newsavebox{\savefullbox}
\newenvironment{fullbox}{\begin{lrbox}{\savefullbox}\begin{minipage}{\dimexpr\textwidth-2\fboxsep\relax}\setlength{\parskip}{\myparskip}}{\end{minipage}\end{lrbox}\framebox[\textwidth]{\usebox{\savefullbox}}}
\newenvironment{pbox}[1][]{\begin{fullbox}\ifx#1\empty\else\paragraph{#1}\phantom{}\fi}{\end{fullbox}}

% Theorem Environments
\theoremstyle{definition}
\newtheorem{lemma}{Lemma}

% Tikz Environments
\newenvironment{drawing}{\begin{center}\begin{tikzpicture}}{\end{tikzpicture}\end{center}}
% \tikzcdset{row sep/normal=0pt}
\newenvironment{cd}{\begin{center}\begin{tikzcd}}{\end{tikzcd}\end{center}}

% Default Commands
\newcommand{\isp}[1]{\quad\text{#1}\quad}
\newcommand{\N}{\mathbb{N}} 
\newcommand{\Z}{\mathbb{Z}}
\newcommand{\Q}{\mathbb{Q}}
\newcommand{\R}{\mathbb{R}}
\newcommand{\C}{\mathbb{C}}
\newcommand{\A}{\mathbb{A}}
\renewcommand{\P}{\mathbb{P}}
\newcommand{\eps}{\varepsilon}
\renewcommand{\phi}{\varphi}
\renewcommand{\emptyset}{\varnothing}
\newcommand{\<}{\langle}
\renewcommand{\>}{\rangle}
\newcommand{\isom}{\cong}
\newcommand{\eqc}{\overline}
\newcommand{\clo}{\overline}
\newcommand{\seq}{\subseteq}
\newcommand{\teq}{\trianglelefteq}
\DeclareMathOperator{\id}{id}
\DeclareMathOperator{\im}{im}
\newcommand{\inc}{\hookrightarrow}

% Extra Commands
\newcommand{\ovl}{\overline}
\newcommand{\htpy}{\simeq}
\newcommand{\bd}{\partial}

\newcommand{\const}{\mathrm{const}}


% Document
\begin{document}
\title{MATH 221B Midterm}
\author{Harry Coleman}
\date{February 28, 2022}
\maketitle

\phantom{.}

\begin{pbox}[1 Exercise 1.2.1]
    Show that the free product $G * H$ of nontrivial groups $G$ and $H$ has trivial center,
\end{pbox}

\begin{proof}
    Each element $x \in G * H$ can be uniquely written as word $x = x_1 \cdots x_n$ such that
    \begin{enumerate}[(i)]
        \item $x_i \in (G \sqcup H) \setminus \{1_G, 1_H\}$,
        \item $x_i \in G$ implies $x_{i+1} \in H$, and $x_i \in H$ implies $x_{i+1} \in G$.
    \end{enumerate}
    (As usual, the identity corresponds to the empty word.)
    We will call a word $x_1 \cdots x_n$ satisfying (i) and (ii) an \textit{alternating reduced word} (in $G$ and $H$).
    To see this, one could put a suitable group structure on the set of alternating reduced words and check that this group satisfies the necessary universal property of the free product for $G * H$.
    It follows that the map $x \mapsto x_1 \cdots x_n$ is necessarily an isomorphism.


    Suppose $x$ is in the center of $G * H$ and has alternating reduced word $x = x_1 \cdots x_n$.
    Then
    \[
        x = x_1^{-1}xx_1 = x_2 \cdots x_n x_1.
    \]
    We now show that $x_1 \cdots x_n$ must be the empty word, corresponding to the identity, by considering a number of cases, with regards to the value of $n$ and whether or not certain letters in the alternating reduced word of $x$ are contained in the same group.

    If $n \geq 2$ and $x_n$ and $x_1$ are not elements of the same group (i.e., one in $G$ and the other in $H$), then $x_2 \cdots x_n x_1$ is an alternating reduced word for $x$.
    However, property (ii) requires that $x_1$ and $x_2$ are from different groups, so $x_2 \cdots x_nx_1$ and $x_1 \cdots x_n$ must be different words.
    But the alternating reduced word is unique, so this is a contradiction.

    If $n \geq 2$ and $x_n$ and $x_1$ are elements of the same group, we can define $y = x_nx_1 \in G \sqcup H$.
    If $y$ is not the identity, then $x_2 \cdots x_{n-1}y$ is an alternating reduced word with $n-1$ letters.
    Otherwise, $x_2 \cdots x_{n-1}$ is an alternating reduced word with $n-2$ letters.
    In either case, we find an alternating reduced word for $x$ different from $x_1 \cdots x_n$ (each has strictly fewer letters), which is a contradiction.

    If $n = 1$ then, without loss of generality, assume $x = x_1 \in G \setminus \{1_G\}$.
    Since $H$ is nontrivial, we can pick some $h \in H \setminus\{1_H\}$.
    Then
    \[
        x = hxh^{-1} = hx_1h^{-1}
    \]
    is an alternating reduced word for $x$ different from $x_1$, which is a contradiction.

    We conclude that $n = 0$, i.e., $x$ is the empty word---the identity in $G * H$.
\end{proof}

\begin{pbox}
    and that the only elements of $G * H$ of finite order are the conjugates of finite-order elements of $G$ and $H$.
\end{pbox}

\begin{proof}
    An element of $G$ or $H$ has the same order when included into $G * H$, and all the conjugates of an element have the same order.
    In particular, conjugates of finite-order elements of $G$ and $H$ are finite order in $G * H$.
    It remains to show that all elements of finite order in $G * H$ have the described form.

    Suppose $x \in G * H$ has $x^k = 1$ and reduced word $x = x_1 \cdots x_n$.
    If $n = 1$, then the result is immediate as $x = x_1$ is a finite order element of $G$ or $H$.
    Assuming $n \geq 2$, we write
    \[
        1 = x^k = (x_1 \cdots x_n)(x_1 \cdots x_n) \cdots (x_1 \cdots x_n).
    \]
    If $x_n$ and $x_1$ are not elements of the same group, then the word obtained by concatenating $x_1 \cdots x_n$ with itself $k$ times is an alternating reduced word.
    However, the identity has the empty word, so this cannot be the case.
    Similarly, if $x_n$ and $x_1$ are elements of the same group (implying $n \geq 3$) but their product $y = x_n x_1$ is not the identity of that group, we again obtain a nonempty alternating reduced word
    \[
        1 = (x_1 \cdots x_{n-1})y(x_2 \cdots x_{n-1}) \cdots y(x_2 \cdots x_n).
    \]
    Again, this is a contradiction, and it follows that $y$ is the identity in its respective group, so $x_n = x_1^{-1}$.
    Repeating the same argument, $x_{n-1} = x_2^{-1}$ and so on.
    If $n$ is even, this implies $x = 1$.
    If $n$ is odd, we deduce that $x = yx_iy^{-1}$ where $i = (n+1)/2$.
    Then
    \[
        1 = x^k = yx_i^ky^{-1},
    \]
    which implies $x_i^k = 1$, i.e., $x_i$ is an element of finite order in its group ($G$ or $H$).
\end{proof}



\begin{pbox}[2 Exercise 1.2.2]
    Let $X \seq \R^m$ be the union of convex open sets $X_1, \dots, X_n$ such that $X_i \cap X_j \cap X_k \ne \emptyset$ for all $i, j , k$.
    Show that $X$ is simply-connected.
\end{pbox}

\begin{proof}
    We first show $X$ is path-connected.
    Let $x, y \in X$---say $x \in X_i$ and $y \in X_j$.
    Then for any $k$, there exists some point
    \[
        z \in X_i \cap X_j \cap X_k \ne \emptyset.
    \]
    Take $\ovl{xz} : I \to \R^m$ to be the straight line path from $x$ to $z$, then $\ovl{xz}(I) \seq X_i$ since $X_i$ is convex.
    Similarly, $\ovl{zy}(I) \seq X_j$, so $\ovl{xz} \cdot \ovl{zy}$ is a path in $X$ from $x$ to $y$.
    Hence, $X$ is path connected.
    
    We now prove that $\pi_1(X) = 0$ by induction on $n$.
    For $n = 1$, the convex set $X = X_1$ is simply-connected, so $\pi_1(X) = 0$.
    For the inductive hypothesis, assume that for any $i$, we have $\pi_1(\bigcup_{j \ne i} X_j) = 0$, i.e., the union of any $n-1$ of the sets is simply-connected.

    Let $f$ be a loop in $X$ based at $x_0 \in X$.
    We will show that $f$ is homotopic to a loop entirely contained in the union of $n - 1$ of the $X_i$'s.
    If this is the case, then the inductive hypothesis will imply that $f$ is in fact homotopic to the constant loop at $x_0$, proving that $\pi_1(X) = 0$.

    Each $f^{-1}(X_i)$ is an open set in $I = [0, 1] \seq \R$ and therefore the union of disjoint open intervals $\bigcup_j (a_{i, j}, b_{i, j})$ (intersected with $I$).
    Then the collection $\{(a_{i, j}, b_{i, j})\}_{i, j}$ is an open cover of $I \seq \R$.
    By the Lebesgue number lemma, there is some $\eps > 0$ such that each interval $(x, x + \eps) \seq I$ is entirely contained in some interval $(a_{i, j}, b_{i, j})$.
    Choose $N \in \N$ such that $1/N < \eps$, and define $t_0 = 0$.
    For each $\ell \geq 1$, define
    \[
        t_\ell = \max \{\tfrac{a}{N} \mid a \in \Z \text{ and } [t_{\ell-1}, \tfrac{a}{N}] \seq f^{-1}(X_i) \text{ for some } i\}.
    \]
    The choice of $\eps$ and $N$ ensures that we could at least take $t_\ell \geq t_{\ell-1} + 1/N$, so this procedure will eventually finish, giving us a partition of the interval $I$.
    Take $r$ to be the first index with $t_r = 1$; this is the last point in the partition.
    For $\ell = 1, \dots, r$, let $f_\ell$ be the path obtained by restricting $f$ to the interval $[t_{\ell-1}, t_\ell]$.
    Then $f$ factors into
    \[
        f = f_1 \cdots f_r
    \]
    and each $f_\ell$ is contained in some $X_i$, but $f_\ell$ and $f_{\ell+1}$ are not contained in the same $X_i$.
    Without loss of generality, we may assume that $f_1$ is contained in $X_1$.

    Suppose $\ell \geq 2$ is an index such that $f_\ell$ is contained in $X_n$ and no other $X_i$ for $i \ne n$.
    By construction, $f_{\ell-1}$ is contained in $X_i$ for some $i \ne n$ and $f_{\ell+1}$ is contained in $X_j$ for some $j \ne n$.
    In particular,
    \[
        f_\ell(0) = f_{\ell-1}(1) \in X_i \cap X_n \isp{and} f_\ell(1) = f_{\ell+1}(0) \in X_j \cap X_n
    \]
    Choose a point $x \in X_i \cap X_j \cap X_n \ne \emptyset$.
    We can homotope $f_\ell$ inside of $X_n$ to the piecewise linear path $\ovl{f_\ell(0) x} \cdot \ovl{x f_\ell(1)}$, with the first segment contained in $X_i$ and the second segment contained in $X_j$.
    That is, $f_\ell$ is homotopic to a path contained in $X_i \cup X_j$ for $i \ne n$ and $j \ne n$.
    Doing this for each such $\ell$, we deduce that $f$ is homotopic to a loop in $\bigcup_{i=1}^{n-1} X_i$ based at $x_0$, completing the induction.
\end{proof}


\begin{pbox}[3 Exercise 1.2.3]
    Show that the complement of a finite set of points in $\R^n$ is simply-connected if $n \geq 3$.
\end{pbox}

\begin{proof}
    By Problem 6 (Hatcher Exercise 1.2.6), it suffices to show that a finite set of points in $\R^n$ is closed and discrete. 
    Given a finite set of points $C \seq \R^n$, and any point $x \in \R^n$, we can define the radius
    \[
        r = \min\{\|x - y\| : y \in C\} > 0.
    \]
    If $x \notin C$, then $B_r(x) \cap C = \emptyset$, so $C$ is closed.
    If $x \in C$, then $B_r(x) \cap C = \{x\}$, so $C$ is discrete.
\end{proof}

\begin{pbox}[4 Exercise 1.2.4]
    Let $X \seq \R^3$ be the union of $n$ lines through the origin.
    Compute $\pi_1(\R^3 \setminus X)$.
\end{pbox}

Denote $Y = \R^3 \setminus X$.

For $n = 0$, we have $\pi_1(Y) = \pi_1(\R^3) = 0$.

Assume $n \geq 1$.
In this case, $Y$ does not contain the origin, so we can construct a deformation retraction of $Y$ onto its intersection with the unit sphere $S^2 \cap Y = S^2 \setminus X$.
Explicitly, we can take the straight line homotopy $h_t : Y \to Y$ with $h_0 = \id_Y$ and $h_1$ a retraction of $Y$ onto $S^2 \setminus X$, where each point of $Y$ is homotoped to its normalization:
\[
    h_t(y) = (1- t)y + t \frac{y}{\|y\|}.
\]
We deduce the homotopy equivalence $Y \htpy S^2 \setminus X$, where $S^2 \setminus X$ is the $2$-dimensional sphere with $n$ pairs of antipodal points removed.
Unfolding the sphere at one of the removed points, we find that space is in fact homeomorphic to an open disc (a $2$-cell) with $2n-1$ points removed.
Slightly deforming the boundary, this is homotopy equivalent to a closed disc with the same number of very small discs removed:
\begin{drawing}
    \begin{scope}[xshift=-2.5cm]
        \fill[gray!20] (0, 0) circle (1.5);
        \foreach \p in {(0, 0), (0.2, 1), (1, 0), (0.5, -0.7), (-0.5,-0.2), (-0.4, -0.9), (-0.8, 0.6), (-1, -0.5), (0.4, 0.3)} {
            \fill[white] \p circle (2pt);
        }
    \end{scope}
    \node at (0, 0) {$\htpy$};
    \begin{scope}[xshift=2.5cm]
        \draw[fill=gray!20] (0, 0) circle (1.5);
        \foreach \p in {(0, 0), (0.2, 1), (1, 0), (0.5, -0.7), (-0.5,-0.2), (-0.4, -0.9), (-0.8, 0.6), (-1, -0.5), (0.4, 0.3)} {
            \draw[fill=white] \p circle (4pt);
        }
    \end{scope}
\end{drawing}

We can freely move these holes around the disc and maintain a homeomorphic space; for instance, we might distribute the holes around the disc near the boundary.
There is a deformation retraction of this space onto a wheel of circles, each connected by a spoke to the center point:
\begin{drawing}
    \begin{scope}[xshift=-2.5cm]
        \draw[fill=gray!20] (0, 0) circle (1.5);
        \foreach \a in {0, 1, 2, 3, 4, 5, 6, 7, 8} {
            \draw[fill=white] (360 * \a / 9 : 1.1) circle (0.3);
        }
    \end{scope}
    \node at (0, 0) {$\htpy$};
    \begin{scope}[xshift=2.5cm]
        \foreach \a in {0, 1, 2, 3, 4, 5, 6, 7, 8} {
            \draw (360 * \a / 9 : 1.1) circle (0.3);
            \draw (0, 0) -- (360 * \a / 9 : 0.8);
        }
    \end{scope}
\end{drawing}
Collapsing each spoke, we obtain the wedge sum of $2n-1$ circles, i.e., $Y \htpy \bigvee^{2n-1} S^1$, so
\[\textstyle
    \pi_1(Y)
        \isom \pi_1(\bigvee^{2n-1} S^1)
        \isom *^{2n-1} \pi_1(S^1)
        \isom *^{2n-1} \Z.
\]
In other words, $\pi_1(Y)$ is the free group on $2n - 1$ generators.

\begin{pbox}[5 Exercise 1.2.5]
    Let $X \seq \R^2$ be a connected graph that is the union of a finite number of straight line segments.
    Show that $\pi_1(X)$ is free with a basis consisting of loops formed by the boundaries of the bounded complementary regions of $X$, joined by suitably chosen paths in $X$.
\end{pbox}

\begin{proof}
    In the style of Hatcher Section 1.A, we take a graph $X$ to be a $1$-dimensional CW complex with vertex set $X^0$ and edges $\{I_\alpha\}_\alpha$.

    The choice of $X \seq \R^2$ as the union of finitely many straight line segments means, in particular, that $X$ is a planar graph.
    We will allow $X$ to be any connected planar graph with finitely many edges.

    If $X$ has only one vertex $v_0 \in X^0$, it is the wedge sum of (finitely many) circles $\bigvee_\alpha S^1_\alpha$, one for each edge $e_\alpha$.
    Then $\pi_1(X)$ is a free group, with basis of loops $S^1_\alpha \to X$, which are precisely the boundaries of the bounded complementary regions of $X$.

    Suppose now that $X$ has more than one vertex, but still only finitely many. Denote the bounded complementary regions $R_i \seq \R^2 \setminus X$ for $i = 1, \dots, m$; these are the connected components of $\R^2 \setminus X$. 
    Let $f_i$ be a loop in its boundary $\bd R_i \seq X$ based at some vertex $v_i \in X^0$ also in its boundary.
    Let $R_0$ be the exterior complementary region of $X$ and choose any vertex $v_0 \in X^0$ to be the basepoint.
    
    Pick a pair of distinct adjacent vertices $u, v \in X$ and an edge $I_\alpha = \clo{e}_\alpha$ between them.
    This edge lies on the boundary of each of two regions, $R_i$ and $R_j$, for some $i$ and $j$ not necessarily distinct.
    By Proposition 0.17, the quotient $X \to X/I_\alpha$ is a homotopy equivalence.
    Moreover, $X/I_\alpha$ is still a connected planar graph with exactly one less edge and one less vertex than $X$.
    Allowing curved edges, we can embed $X/I_\alpha$ into $\R^2$ such that the graph structure is preserved, giving us something like the following picture:
    \begin{drawing}
        \begin{scope}
            \draw[thick] (0, 0) -- (1, 0);
            \fill (0, 0) circle (2pt) node[left] {$u$};
            \fill (1, 0) circle (2pt) node[right] {$v$};
            
            \foreach \a in {112.5, 135, 225, 247.5} {
                \draw (0, 0) -- (\a : 1);
            }
            \node at (-0.75, 0.1) {$\vdots$};
    
            \foreach \a in {67.5, 45, -45, -67.5} {
                \draw (1, 0) -- +(\a : 1);
            }
            \node at (1.75, 0.1) {$\vdots$};
    
            \node at (0.5, 0.5) {$R_i$};
            \node at (0.5, -0.5) {$R_j$};
        \end{scope}
        \begin{scope}[xshift=3cm]
            \node at (0, 0) {$\htpy$};
        \end{scope}
        \begin{scope}[xshift=5cm]
            \fill (0.5, 0) circle (2pt);
            
            \foreach \a in {112.5, 135, 225, 247.5} {
                \draw (0.5, 0) -- (\a : 1);
            }
            \node at (-0.5, 0.1) {$\vdots$};
    
            \foreach \a in {67.5, 45, -45, -67.5} {
                \draw (0.5, 0) -- ([xshift=1cm]\a : 1);
            }
            \node at (1.5, 0.1) {$\vdots$};
    
            \node at (0.5, 0.75) {$R_i'$};
            \node at (0.5, -0.75) {$R_j'$};
        \end{scope}
    \end{drawing}
    The regions $R_i$ and $R_j$ of $X$ are distinct if and only if the regions $R_i'$ and $R_j'$ of $X/I_\alpha$ are distinct.
    Moreover, each loop $f_k$ in $\bd R_k$ is sent to a loop $f_k'$ in $\bd R_k'$ based at the corresponding vertex $v_k'$ of $X/I_\alpha$.

    We repeatedly perform this procedure until we obtain a graph with only one vertex, by collapsing some choice of edges $\{e_\alpha\}_{\alpha \in A}$ of $X$.
    Then $T = \bigcup_{\alpha \in A} I_\alpha$ is a maximal tree in $X$ and the quotient $X \to X/T$ is a homotopy equivalence.
    And, as discussed above, $\pi_1(X/T)$ is a free group with basis consisting of a loop for each edge in $X/T$.
    The edges in $X/T$ are precisely the boundaries of the bounded complementary regions, which correspond to those of $X$ under the quotient, so the loops $f_i'$ in $X/T$ provide a basis for $\pi_1(X/T)$.

    For each $i = 1, \dots, m$, let $\gamma_i$ be a path in $T$ from $v_0$ to $v_i$; this corresponds to a constant loop in $X/T$ based at $v_0'$.
    Then $\gamma_i \cdot f_i \cdot \ovl{\gamma_i}$ is a loop in $X$ based at $v_0$ and whose homotopy class in $\pi_1(X/T)$ is
    \[
        [\gamma_i' \cdot f_i' \cdot \ovl{\gamma_i}']
            = [\gamma_i'][f_i'][\gamma_i']^{-1}
            = [f_i'].
    \]
    It follows that the loops $\gamma_i \cdot f_i \cdot \ovl{\gamma_i}$ provide a basis for the free group $\pi_1(X) \isom \pi_1(X/T)$.
\end{proof}



\newpage
\begin{pbox}[6 Exercise 1.2.6]
    Suppose a space $Y$ is obtained from a path-connected subspace $X$ by attaching $n$-cells for a fixed $n \geq 3$.
    Show that the inclusion $X \inc Y$ induces an isomorphism on $\pi_1$.
    [See proof of Proposition 1.26.]
\end{pbox}

\textbf{The print version of the textbook includes this part of the problem, but the PDF version does not.
It looks like the PDF version simply proves this generalization of  the print version's proposition.
I'm leaving it here because I typed it up.} 

\begin{proof}
    Suppose we attach $n$-cells $e^n_\alpha$ with attaching maps $\phi_\alpha : S^{n-1} \to X$.
    Choose basepoints $s_0 \in S^{n-1}$ and $x_0 \in X$, and choose paths $\gamma_\alpha$ in $X$ from $x_0$ to $\phi_\alpha(s_0)$.
    Then for any loop $f$ in $S^{n-1}$ based at $s_0$, there is a loop
    \[
        f_\alpha = \gamma_\alpha \cdot (\phi_\alpha \circ f) \cdot \ovl{\gamma_\alpha}
    \]
    in $X$ based at $x_0$.
    Since $S^{n-1}$ is simply-connected, this loop is nullhomotopic in $X$ (relative to $x_0$).

    We perform the following---slightly modified---construction from the proof of Proposition 1.26:
    Let us expand $Y$ to a slightly larger space $Z$ that deformation retracts onto $Y$ and is more convenient for applying van Kampen's theorem.
    The space $Z$ is obtained from $Y$ by attaching rectangular strips $S_\alpha = I \times I$, with the lower edge $I \times \{0\}$ attached along $\gamma_\alpha$, the right edge $\{1\} \times I$ attached along an arc $e^n_\alpha$, and all the left edges $\{0\} \times I$ of the different strips identified together.
    The top edges of the strips are not attached to anything, and this allows us to deformation retract $Z$ onto $Y$.

    In each cell $e^n_\alpha$ choose a point $y_\alpha$ not in the arc along which $S_\alpha$ is attached.
    Let $A = Z \setminus \bigcup_\alpha\{y_\alpha\}$ and let $B = Z \setminus X$.
    Then $A$ deformation retracts onto $X$, and $B$ is contractible.
    Since $\pi_1(B) = 0$, van Kampen's theorem applied to the cover $\{A, B\}$ of $Z$ says that
    \[
        \pi_1(Z) \isom \pi_1(A) / \<\!\<\im(\pi_1(A \cap B) \to \pi_1(A))\>\!\>.
    \]
    However, $\pi_1(A \cap B)$ is trivial.
    This can be shown by another application of van Kampen's theorem, this time to the cover of $A \cap B$ by the open sets $A_\alpha = A \cap B \setminus \bigcup_{\beta \ne \alpha} e^n_\beta$.
    Since $A_\alpha$ deformation retracts onto a sphere $S^{n-1}$ in $e^n_\alpha \setminus \{y_\alpha\}$, we have $\pi_1(A_\alpha) \isom \pi_1(S^{n-1}) = 0$.
    It follows that
    \[
        \pi_1(X) \isom \pi_1(A) \isom \pi_1(Z) \isom \pi_1(Y),
    \]
    with the isomorphism being induced by the inclusion $X \inc Y$.
\end{proof}

\begin{pbox}
    Use Proposition 1.26 to show that to show that the complement of a closed discrete subspace of $\R^n$ is simply-connected if $n \geq 3$.
\end{pbox}

\begin{proof}
    Let $\bigcup_\alpha\{x_\alpha\} \seq \R^n$ be the closed discrete subspace.
    For each $\alpha$, there is a radius $r_\alpha > 0$ such that the open ball $B_{r_\alpha}(x_\alpha) \seq \R^n$ contains no other points $x_\beta$ for $\beta \ne \alpha$.
    Let $B_\alpha = B_{r_\alpha/2}(x_\alpha)$ be the open ball of half the radius, also centered at $x_\alpha$.
    Then $B_\alpha \cap B_\beta = \emptyset$ for $\alpha \ne \beta$.
    Each punctured closed ball $\clo{B_\alpha} \setminus \{x_\alpha\}$ deformation retracts onto its unpunctured boundary $\bd B_\alpha \isom S^{n-1}$.
    Gluing these deformation retractions together and taking the identity elsewhere, we obtain a deformation retraction of $\R^n \setminus \bigcup_\alpha \{x_\alpha\}$ onto $\R^n \setminus \bigcup_\alpha B_\alpha$.
    We can now reconstruct $\R^n$ by attaching $n$-cells $e^n_\alpha$, via attaching maps $S^{n-1} \to \bd B_\alpha$, filling in each of the holes formed by removing $B_\alpha$.
    By the previous result (part of Proposition 1.26 in ), we conclude
    \[\textstyle
        \pi_1(\R^n \setminus \bigcup_\alpha \{x_\alpha\})
            \isom \pi_1(\R^n \setminus \bigcup_\alpha B_\alpha)
            \isom \pi_1(\R^n)
            = 0.
    \]
\end{proof}


\begin{pbox}[7 Exercise 1.2.8]
    Compute the fundamental group of the space obtained from two tori $S^1 \times S^1$ by identifying a circle $S^1 \times \{x_0\}$ in one torus with the corresponding circle $S^1 \times \{x_0\}$ in the other torus.
\end{pbox}

This space is simply $S^1 \times (S^1 \vee S^1)$, so we can compute
\[
    \pi_1(S^1 \times (S^1 \vee S^1))
        \isom \pi_1(S^1) \times \pi_1(S^1 \vee S^1)
        \isom Z \times (\pi_1(S^1) * \pi_1(S^1))
        \isom \Z \times (\Z * \Z).
\]
This group has three copies of $\Z$, the first of which commutes with each of the other two, but the latter two do not commute with each other.
We might give the following presentation of this group:
\[
    \<a, b, c \mid [a, b] = [a, c] = 1\>.
\]


\end{document}