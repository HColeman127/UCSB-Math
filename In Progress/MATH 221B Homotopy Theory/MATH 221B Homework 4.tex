\documentclass[12pt]{article}

% Packages
\usepackage[margin=1in]{geometry}
\usepackage{fancyhdr, parskip}
\usepackage{amsmath, amsthm, amssymb}
\usepackage{tikz, tikz-cd}

% Page Style
\makeatletter
\fancypagestyle{title}{
    \renewcommand{\headrulewidth}{0.4pt}
    \setlength{\headheight}{15pt}
    \fancyhead[R]{\@author}
    \fancyhead[L]{\@title}
    \fancyhead[C]{\@date}
}
\makeatother
\renewcommand{\maketitle}{\thispagestyle{title}}
\fancypagestyle{plain}{
    \fancyhf{}
    \renewcommand{\headrulewidth}{0pt}
    \renewcommand{\footrulewidth}{0pt}
    \fancyfoot[R]{\thepage}
}
\pagestyle{plain}

% Problem Box
\setlength{\fboxsep}{4pt}
\newlength{\myparskip}
\setlength{\myparskip}{\parskip}
\newsavebox{\savefullbox}
\newenvironment{fullbox}{\begin{lrbox}{\savefullbox}\begin{minipage}{\dimexpr\textwidth-2\fboxsep\relax}\setlength{\parskip}{\myparskip}}{\end{minipage}\end{lrbox}\framebox[\textwidth]{\usebox{\savefullbox}}}
\newenvironment{pbox}[1][]{\begin{fullbox}\ifx#1\empty\else\paragraph{#1}\phantom{}\fi}{\end{fullbox}}

% Theorem Environments
\theoremstyle{definition}
\newtheorem{lemma}{Lemma}

% Tikz Environments
\newenvironment{drawing}{\begin{center}\begin{tikzpicture}}{\end{tikzpicture}\end{center}}
% \tikzcdset{row sep/normal=0pt}
\newenvironment{cd}{\begin{center}\begin{tikzcd}}{\end{tikzcd}\end{center}}

% Default Commands
\newcommand{\isp}[1]{\quad\text{#1}\quad}
\newcommand{\N}{\mathbb{N}} 
\newcommand{\Z}{\mathbb{Z}}
\newcommand{\Q}{\mathbb{Q}}
\newcommand{\R}{\mathbb{R}}
\newcommand{\C}{\mathbb{C}}
\newcommand{\A}{\mathbb{A}}
\renewcommand{\P}{\mathbb{P}}
\newcommand{\eps}{\varepsilon}
\renewcommand{\phi}{\varphi}
\renewcommand{\emptyset}{\varnothing}
\newcommand{\<}{\langle}
\renewcommand{\>}{\rangle}
\newcommand{\isom}{\cong}
\newcommand{\eqc}{\overline}
\newcommand{\clo}{\overline}
\newcommand{\seq}{\subseteq}
\newcommand{\teq}{\trianglelefteq}
\DeclareMathOperator{\id}{id}
\DeclareMathOperator{\im}{im}

% Extra Commands
\newcommand{\inc}{\hookrightarrow}
\newcommand{\htpy}{\simeq}
\newcommand{\bd}{\partial}

\newcommand{\const}{\mathrm{const}}

% Document
\begin{document}
\title{MATH 221B Homework 4}
\author{Harry Coleman\makebox[0pt][r]{\raisebox{-0.25in}[0pt][0pt]{(I worked with Joseph Sullivan and Gahl Shemy)}}}
\date{February 14, 2022}
\maketitle


\begin{pbox}[1 Exercise 1.1]
    Show that the composition of paths satisfies the following cancellation property: if $f_0 \cdot g_0 \htpy f_1 \cdot g_1$ and $g_0 \htpy g_1$ then $f_0 \htpy f_1$.
\end{pbox}

\begin{proof}
    Since $g_0 \htpy g_1$, we have inverse paths $\overline{g_0} \htpy \overline{g_1}$, such that $g_i \cdot \overline{g_i} \htpy c$, for $c$ a constant path with $c(s) = g_i(0) = f_i(1)$.
    Using the known properties of path composition, we compute
    \begin{align*}
        f_0 &\htpy f_0 \cdot c \\
            &\htpy f_0 \cdot (g_0 \cdot \overline{g_0}) \\
            &\htpy (f_0 \cdot g_0) \cdot \overline{g_0} \\
            &\htpy (f_1 \cdot g_1) \cdot \overline{g_1} \\
            &\htpy f_1 \cdot (g_1 \cdot \overline{g_1}) \\
            &\htpy f_1 \cdot c \\
            &\htpy f_1.
    \end{align*}
\end{proof}

\begin{pbox}[2 Exercise 1.2]
    Show that the change-of-basepoint homomorphism $\beta_h$ depends only on the homotopy class of $h$.
\end{pbox}

\begin{proof}
    If $h \htpy k$ are paths from $x_0$ to $x_1$ in a space $X$, then we have the induced homomorphisms  $\beta_h, \beta_k : \pi_1(X, x_1) \to \pi_1(X, x_0)$.
    Note that we have inverse paths $\overline{h} \htpy \overline{k}$, so for all loops $f : (I, \bd I) \to (X, x_0)$,
    \[
        h \cdot f \cdot \overline{h} \htpy k \cdot f \cdot \overline{k}.
    \]
    It follows that
    \[
        \beta_h[f]
            = [h \cdot f \cdot \overline{h}]
            = [k \cdot f \cdot \overline{k}]
            = \beta_{h_1}[f].
    \]
    Hence, $[h] = [k]$ implies $\beta_h = \beta_k$.
\end{proof}

\begin{pbox}[3 Exercise 1.3]
    For a path-connected space $X$, show that $\pi_1(X)$ is abelian if and only if all basepoint-change homomorphisms $\beta_h$ depend only on the endpoints of the path $h$.
\end{pbox}

\begin{proof}
    Suppose $\pi_1(X)$ is abelian and $h, k : I \to X$ are paths from $x_0$ to $x_1$.
    Then for a given homotopy class $[f] \in \pi_1(X, x_1)$, we want to show that $\beta_h[f] = [h \cdot f \cdot \overline{h}]$ is equal to $\beta_k[f] = [k \cdot f \cdot \overline{k}]$.
    Equivalently, we want $[f] = [\overline{k} \cdot h \cdot f \cdot \overline{h} \cdot k]$.
    But with $\pi_1(X)$ abelian, we compute
    \[
        [\overline{k} \cdot h \cdot f \cdot \overline{h} \cdot k]
            = [\overline{k} \cdot h] [f] [\overline{h} \cdot k]
            = [\overline{k} \cdot h] [f] [\overline{k} \cdot h]^{-1}
            = [f].
    \]
    Hence $\beta_h = \beta_k$.

    In the case that all the base-change homomorphisms depend only on the endpoints of the path, for every $[f], [g] \in \pi_1(x, x_0)$ we have
    \[
        [g][f][g]^{-1} = \beta_g[f] = \beta_f[f] = [f][f][f]^{-1} = [f].
    \]
    Hence, $[f][g] = [g][f]$ for all $[f], [g] \in \pi_1(x, x_0)$, so indeed $\pi_1(X)$ is abelian.
\end{proof}

\newpage
\begin{pbox}[4 Exercise 1.4]
    A subspace $X \seq \R^n$ is said to be \textit{star-shaped} if there is a point $x_0 \in X$ such that, for each $x \in X$, the line segment from $x_0$ to $x$ lies in $X$. 
    Show that if a subspace $X \seq \R^n$ is locally star-shaped, in the sense that every point of $X$ has a star-shaped neighborhood in $X$, then every path in $X$ is homotopic in $X$ to a piecewise linear path, that is, a path consisting of a finite number of straight line segments traversed at constant speed.
\end{pbox}

\begin{proof}
    Let $f : I \to X$ be a path and for every $x \in f(I)$, let $S_x \seq X$ be a star-shaped neighborhood of $x$.
    In case the $S_x$'s are not themselves open, we also consider open neighborhoods of each point: $x \seq U_x \seq S_x$. 
    Then $\{f^{-1}(U_x)\}_{x \in f(I)}$ is an open cover of the compact (metric) space $I$.
    By the Lebesgue number lemma, there is some $\delta > 0$ such that for every $t \in I$, the open ball $B_\delta(t) \seq I$ is entirely contained in some $f^{-1}(U_x)$ of the open cover.

    Choose $k \in \N$ such that $1/k \leq \delta$. Denote the point $t_j = j/k \in I$ for $j = 0, \dots, k$, and the interval $I_j = [t_{j-1}, t_j] \seq I$ for $j = 1, \dots, k$.
    Then a homeomorphism $\phi_j : I \xrightarrow{\sim} I_j$ (say by shifting and linear scaling) gives us a path $f_j = f \circ \phi_j : I \to X$ from $f(t_{j-1})$ to $f(t_j)$.
    With this construction, we obtain a homotopy
    \[
        f \htpy f_1 \cdot f_2 \cdots f_k,
    \]
    which is simply a reparameterization, fixing the points $t_0, \dots, t_j$.
    
    Fix an index $j \in \{1, \dots, k\}$.
    By construction, the interval $I_j$ is contained in an open ball of radius $\delta$ and, therefore, in some $f^{-1}(U_x)$; so
    \[
        f_j(I) = f(I_j) \seq U_x \seq S_x.
    \]
    In other words, $f_j$ is a path in $S_x$ from $f(t_{j-1})$ to $f(t_j)$.
    Say $v \in S_x$ is a vantage point of $S_x$, in the sense that for every other point $y \in S_x$ the line segment $\overline{vy}$ is contained in $S_x$.
    We can then perform the straight line homotopy $f_j \htpy \const_{v}$ inside of $S_x$.
    This homotopy is represented by a map $H : I \times I \to S_x$ such that $H(s, 0) = f_j(s)$ and $H(s, 1) = \const_v$.
    The parameter space of $H$ can be drawn as follows:
    \begin{drawing}
        \filldraw (0, 0) circle (2pt) node[anchor=north east] {$v$};
        \filldraw (2, 0) circle (2pt) node[anchor=north west] {$v$};
        \filldraw (0, 2) circle (2pt) node[anchor=south east] {$f_j(0)$};
        \filldraw (2, 2) circle (2pt) node[anchor=south west] {$f_j(1)$};

        \draw (0, 0) rectangle (2, 2);

        \draw[->] (-0.5, 1.5) -- (-0.5, 0.5) node[midway, anchor=east] {$t$};
        \draw[->] (0.5, -0.5) -- (1.5, -0.5) node[midway, anchor=north] {$s$};
    \end{drawing}
    The top edge is $f_j$, the bottom edge is $\const_v$, and every vertical line corresponds under $H$ to a line segment in $S_x$.
    In this parameter space, we construct a homotopy $\gamma_t : I \to I \times I$ (of paths) from the path following the top edge to the path following the other three edges.
    In particular, we split the top edge into thirds, sending the middle third to the bottom edge and the left/right thirds to their respect side of the square.
    Explicitly, we perform the straight line homotopy in the parameter space, giving us the following picture:
    \begin{drawing}
        \foreach \x in {0, 2} {
            \foreach \y in {0, 2} {
                \filldraw (\x, \y) circle (2pt);
            }
        }
        \draw (0, 0) rectangle (2, 2);

        \draw[dashed] (0, 0) -- (0.666, 2);
        \draw[dashed] (2, 0) -- (1.333, 2);
        \draw[thick] (0, 2) -- (0.333, 1) -- (1.666, 1) node[midway, anchor=south] {$\gamma_t$} -- (2, 2);

        \node[anchor=south] at (1, 2) {$\gamma_0$};
        \node[anchor=north] at (1, 0) {$\gamma_1$};
    \end{drawing}
    Then $H \circ \gamma_t$ gives us a homotopy of paths in $S_x$ from $\gamma_0 = f_j$ to $\gamma_1 = \ell_j$, where $\ell_j : I \to S_x$ is the piecewise linear path from $f_j(0)$ to $v$, pausing at $v$, then from $v$ to $f_j(1)$.

    Constructing the homotopy of paths $f_j \htpy \ell_j$ for all $j = 1, \dots, k$, we obtain
    \[
        f \htpy f_1 \cdot f_2 \cdots f_k \htpy \ell_1 \cdot \ell_2 \cdots \ell_k.
    \]
    If necessary, we can reparameterize the resulting path so that it is traversed at a constant speed, not pausing at the vantage points.
\end{proof}

\begin{pbox}
    Show that this applies in particular when $X$ is open or when $X$ is a union of finitely many closed convex sets.
\end{pbox}

\begin{proof}
    If $X \seq \R^n$ is open, then for every point $x \in X$ there is a radius $r > 0$ such that the open ball $B_r(x) \seq \R^n$ is completely contained in $X$.
    As $B_r(x)$ is convex, it is trivially a star-shaped neighborhood of $x$.
    Hence, open subsets of $\R^n$ are locally star-shaped.

    Suppose $X = \bigcup_{i=1}^{k} C_i$ for closed convex sets $C_i \seq \R^n$, and consider a point $x_0 \in X$.
    Since the set $E = \bigcup\{C_i : x_0 \notin C_i\}$ is closed and does not contain $x_0$, its complement is an open neighborhood of $x_0$.
    So we can find a radius $r > 0$ such that $B_r(x_0) \cap E = \emptyset$.
    This means $U = B_r(x_0) \cap X$ is an open neighborhood of $x_0$ (that is, open in the subspace $X \seq \R^n$), and we can write
    \[
        U = (B_r(x_0) \cap (X \setminus E)) \cup (B_r(x_0) \cap E)= \bigcup\{B_r(x_0) \cap C_i : x_0 \in C_i\}.
    \]
    Then $U$ is star-shaped as the union of convex sets with nonempty intersection.
    That is, for every point $x \in U$, we have $x \in B_r(x_0) \cap C_i$ for some $C_i$ containing $x_0$.
    Since $B_r(x_0)$ and $C_i$ are convex, so is their intersection, which must therefore contain the line segment $\overline{x_0x}$.
    Hence, $U$ is star-shaped (with $x_0$ as a vantage point), and we conclude that $X$ is locally star-shaped.
\end{proof}

\newpage
\begin{pbox}[5 Exercise 1.5]
    Show that for a space $X$, the following three conditions are equivalent:
    \begin{enumerate}
        \item[(a)] Every map $S^1 \to X$ is homotopic to a constant map, with image a point.
        \item[(b)] Every map $S^1 \to X$ extends to a map $D^2 \to X$.
        \item[(c)] $\pi_1(X, x_0) = 0$ for all $x_0 \in X$.
    \end{enumerate}
\end{pbox}

\begin{proof}
    Assuming (a) holds, we aim to prove (b).
    Let $f : S^1 \to X$ be any map.
    By (a), there is a homotopy $F : S^1 \times I \to X$ with $F_0$ a constant map and $F_1 = f$.
    We interpret $S^1 \times I$ as the annulus and notice that $(S^1 \times I)/(S^2 \times \{0\}) \isom D^2$, with the quotient map:
    \begin{drawing}
        \draw[fill=gray!20] (0, 0) circle (1.5cm) node[above, yshift=1.5cm] {$S^1 \times I$};
        \draw[thick, fill=white] (0, 0) circle (0.5cm) node[above, yshift=0.5cm] {$\scriptstyle S^1 \times \{0\}$};

        \draw[->>] (2, 0) -- (3, 0) node[midway, above] {$q$};

        \draw[fill=gray!20] (5, 0) circle (1.5cm) node[above, yshift=1.5cm] {$D^2$};
        \filldraw (5, 0) circle (2pt) node[above] {$\scriptstyle \{S^1 \times \{0\}\}$};
    \end{drawing}
    The fact that $F_0$ is a constant map means that $F$ is constant on the subset $S^2 \times \{0\}$.
    So by the universal property of quotients, there is a unique map $\tilde{f} : D^2 \to X$ such that the following diagram commutes:
    \begin{cd}
        S^1 \times I \rar["F"] \dar["q"'] & X \\
        D^2 \urar[dashed, "\exists!\tilde{f}"']
    \end{cd}
    Where $S^1 = S^1 \times \{1\}$ is the boundary of $D^2$, we have
    \[
        \tilde{f}|_{S^1}
            = (\tilde{f} \circ q)|_{S^1 \times \{1\}}
            = F|_{S^1 \times \{1\}}
            = F_1
            = f.
    \]
    Hence, $\tilde{f}$ is an extension of $f$.

    Assuming (b) holds, we aim to prove (c).
    Fix a basepoint $x_0 \in X$ and consider a homotopy class $[f] \in \pi_1(X, x_0)$ of a loop $f : (S^1, s_0) \to (X, x_0)$.
    Applying (b) and taking $S^1 = \bd D^2$, we have the following commutative diagram:
    \begin{cd}
        D^2 \drar[dashed, "\tilde{f}"] \\
        S^1 \uar[hookrightarrow] \rar["f"'] & X
    \end{cd}
    Let $R : D^2 \times I \to D^2$ be a deformation retraction to the point $s_0 \in S^1 \inc D^2$.
    In other words, $R$ is a homotopy from $\id_{D^2}$ to $\const_{s_0}$ relative to $s_0$, which we depict using the following diagram:
    \begin{cd}
        D^2
            \rar[bend left=40, "\id_{D^2}"{name=A}]
            \rar[bend right=40, "\const_{s_0}"'{name=B}]
            \ar[from=A, to=B, Rightarrow, shorten=2mm, "R"]
        & D^2
    \end{cd}
    We extend this diagram as follows:
    \begin{cd}
        S^1 \rar[hookrightarrow]
        & D^2
            \rar[bend left=40, "\id_{D^2}"{name=A}]
            \rar[bend right=40, "\const_{s_0}"'{name=B}]
            \ar[from=A, to=B, Rightarrow, shorten=2mm, "R"]
        & D^2 \rar["\tilde{f}"]
        & X
    \end{cd}
    This diagram suggests a homotopy from the top edge, $f$, and the bottom edge, $\const_{x_0}$.
    Moreover, we should expect this new homotopy, like $R$, to be relative to $s_0$.
    Explicitly, we can construct the homotopy $f \htpy \const_{x_0}$ as the composition
    \begin{cd}
        S^1 \times I \rar[hookrightarrow] & D^2 \times I \rar["R"] & D^2 \rar["\tilde{f}"] & X.
    \end{cd}
    This is indeed a homotopy from $f$ to $\const_{x_0}$ relative to $s_0$, which means $[f] = [\const_{x_0}]$ in the fundamental group $\pi_1(X, x_0)$, so the fundamental group must be trivial.

    Assuming (c) holds, we aim to prove (a).
    We can consider any given map $f : S^1 \to X$ as a map $(S^1, s_0) \to (X, x_0)$, where $s_0 \in S^1$ is fixed and $x_0 = f(s_0)$.
    By (c),
    \[
        [f] = [\const_{x_0}] \in \pi_1(X, x_0) = 0,
    \]
    which means $f \htpy \const_{x_0}$.
    (Moreover, the homotopy is relative to $s_0$, but (a) is satisfied regardless of this fact.)
\end{proof}

\begin{pbox}
    Deduce that a space $X$ is simply-connected iff all maps $S^1 \to X$ are homotopic.
\end{pbox}

\begin{proof}
    Assume $X$ is simply connected and let $f, g : S^1 \to X$ be any two maps.
    As (c) is immediately satisfied, (a) tells us that $f \htpy \const_x$ and $g \htpy \const_y$ for some points $x, y \in X$.
    With $X$ path-connected, there is a path $I \to X$ from $x$ to $y$, which can be used to describe a homotopy from $\const_x$ to $\const_y$.
    Hence,
    \[
        f \htpy \const_x \htpy \const_y \htpy g.
    \]

    Assume all maps $S^1 \to X$ are homotopic.
    For any two points $x, y \in X$, we have (homotopic) maps $\const_x, \const_y : S^1 \to X$.
    If $\gamma : S^1 \times I \to X$ is a homotopy from $\const_x$ to $\const_y$, then we have a path $I \to X$ from $x$ to $y$, given by $t \mapsto \gamma_t(s_0)$, for any fixed $s_0 \in S^1$.
    This shows $X$ is path-connected.
    Every map $S^1 \to X$ is, in particular, homotopic to a constant map, i.e., (a) is satisfied.
    By the proven equivalence, (c) is also satisfied, so we conclude that $X$ is simply connected.
\end{proof}

\begin{pbox}[6 Exercise 1.6]
    We can regard $\pi_1(X, x_0)$ as the set of base-point preserving homotopy classes of maps $(S^1, s_0) \to (X, x_0)$.
    Let $[S^1, X]$ be the set of homotopy classes of maps $S^1 \to X$, with no conditions on basepoints.
    Thus there is a natural map $\Phi : \pi_1(X, x_0) \to [S^1, X]$ obtained by ignoring basepoints.

    Show that $\Phi$ is surjective if $X$ is path-connected,
\end{pbox}

\begin{proof}
    Suppose $X$ is path-connected and let $f : S^1 \to X$ be any map.
    Let $h : I \to X$ be a path from $x_0$ to $f(s_0)$, giving us the homotopy class $[h \cdot f \cdot \overline{h}] \in \pi_1(X, x_0)$.
    Possibly after reparameterizing, we can consider $h \cdot f \cdot \overline{h}$ to be a path $I \to X$, where on the interval $[0, \frac{1}{3}]$ we follow $h$, on $[\frac{1}{3}, \frac{2}{3}]$ we follow $f$, and on $[\frac{2}{3}, 1]$ we follow $\overline{h}$.
    We define a homotopy of reparameterizations $\phi_t : I \to I$ by
    \[
        \phi_t(s) = (1 - t)s + t\tfrac{1}{3}(1 + s).
    \]
    Then the composition $(h \cdot f \cdot \overline{h}) \circ \phi_t$ describes a homotopy $h \cdot f \cdot \overline{h} \htpy f$.
    (Note that this is simply a homotopy of maps, not a homotopy of paths, i.e., it need not fix basepoints.)
    This gives us $\Phi([h \cdot f \cdot \overline{h}]) = [f]$, where $[f] \in [S^1, X]$ was chosen arbitrarily.
\end{proof}

\begin{pbox}
    and that $\Phi([f]) = \Phi([g])$ iff $[f]$ and $[g]$ are conjugate in $\pi_1(X, x_0)$.
    Hence $\Phi$ induces a one-to-one correspondence between $[S^1, X]$ and the set of conjugacy classes in $\pi_1(X)$, when $X$ is path-connected.
\end{pbox}

\begin{proof}
    Suppose $\Phi([f]) = \Phi([g])$, which means $f, g : S^1 \to X$ are homotopic maps, and it happens that $f(s_0) = g(s_0) = x_0$. If the homotopy is described by a map $H : S^1 \times I \to X$, with $H(s, 0) = f(s)$ and $H(s, 1) = g(s)$, then $H(s_0, t)$ need not be constant.
    The parameter space of $H$ can be drawn as an annulus:
    \begin{drawing}
        \draw (0, 0) circle (1.5cm);
        \draw (0, 0) circle (0.5cm);

        \filldraw (0, -0.5) circle (2pt) node[above] {$x_0$};
        \filldraw (0, -1.5) circle (2pt) node[below] {$x_0$};
        \draw (0, -0.5) -- (0, -1.5);

        \node[above] at (0, 1.5) {$f$};
        \node[above] at (0, 0.5) {$g$};
    \end{drawing}
    The outer edge is $f$ (where $t = 0$) and the inner edge is $g$ (where $t = 1$).
    The line connecting the two edges is the path traced by $H(s_0, t)$.
    In this parameter space, we construct a homotopy $\gamma_t : S^1 \to S^1 \times I$ (of loops) from the outer edge to the path up through the connecting line, around the inner edge, and back down the connecting line.
    The homotopy could look something like the following:
    \begin{drawing}
        \draw (0, 0) circle (1.5cm);
        \draw (0, 0) circle (0.5cm);

        \filldraw (0, -0.5) circle (2pt);
        \filldraw (0, -1.5) circle (2pt);
        \draw (0, -0.5) -- (0, -1.5);

        \draw[dashed] (-115 : 1.5) -- (0, -0.5) -- (-65 : 1.5);
        \draw[thick] (0, -1.5) -- (-110 : 1) arc (250 : -70 : 1) -- cycle;

        \node[above] at (0, 1.5) {$\gamma_0$};
        \node[below] at (0, 0.5) {$\gamma_1$};
        \node[above] at (0, 1) {$\gamma_t$};
    \end{drawing}
    Then $H \circ \gamma_t$ gives us a homotopy of loops from $\gamma_0 = f$ to $\gamma_1 = h \cdot g \cdot \overline{h}$, where $h$ is the loop $(S^1, s_0) \to (X, x_0)$ following $H(s_0, t)$.
    Since $\gamma_t(s_0) = (s_0, 0)$ for all $t$, then $H \circ \gamma_t$ is a homotopy relative to the basepoints.
    So in $\pi_1(X, x_0)$ we have
    \[
        [f]
            = [h \cdot g \cdot \overline{h}]
            = [h] [g] [h]^{-1},
    \]
    which is to say that $[f]$ and $[g]$ are conjugate.

    Let $f, h : (S^1, s_0) \to (X, x_0)$ be loops---then $[f]$ and $[h \cdot f \cdot \overline{h}]$ are conjugate in $\pi_1(X, x_0)$.
    There is a homotopy of parameters $\phi_t : S^1 \to S^1$ given by rotating the circle up to a third of a full revolution.
    In other words, $\phi_t$ moves all the points along the circle $S^1$ at a constant rate of $1/3$ revolutions per unit time.
    Assuming $h \cdot f \cdot \overline{h}$ is nicely parameterized into thirds, the composition $(h \cdot f \cdot \overline{h}) \circ \phi_t$ describes a homotopy of maps between $h \cdot f \cdot \overline{h}$ and $f \cdot \overline{h} \cdot h$.
    Hence, $h \cdot f \cdot \overline{h} \htpy f \cdot \overline{h} \cdot h \htpy f$, where both homotopies are as maps $S^1 \to X$ (and the second is also as loops based at $x_0$), so in fact $\Phi([f]) = \Phi([g])$.
\end{proof}

\begin{pbox}[7 Exercise 1.9]
    Let $A_1, A_2, A_3$ be compact sets in $\R^3$.
    Use the Borsuk-Ulam theorem to show that there is one plane $P \seq \R^3$ that simultaneously divides each $A_i$ into two pieces of equal measure.
\end{pbox}

\begin{proof}
    We consider $S^2$ as the unit sphere in $\R^3$.
    A plane in $\R^3$ is determined by a normal vector $v \in S^2$ and a distance $r \in \R$ from the origin, in the direction of the normal vector.
    That is, each point $(v, r) \in S^2 \times \R$ defines a plane
    \[
        P(v, r) = \{x \in \R^3 \mid v \cdot x = r\}.
    \]
    Notice that each plane can be written in exactly two ways: $P(v, r) = P(-v, -r)$.
    This relates to the fact that $P(v, r)$ divides $\R^3$ into two (closed) half-spaces, defined by
    \[
        H(v, r) = \{x \in \R^3 \mid v \cdot x \geq r\}
        \isp{and}
        H(-v, -r) = \{x \in \R^3 \mid v \cdot x \leq r\},
    \]
    and whose intersection is the plane $P(v, r)$.
    Let $\lambda$ be the Lebesgue measure on $\R^3$.
    For $i = 1, 2, 3$ define the function $h_i : S^2 \times \R \to \R$ by
    \[
        h_i(v, r) = \lambda(A_i \cap H(v, r)).
    \]
    Since the $A_i$'s are compact, they have finite measure.
    We know that $\lambda(P(v, r)) = 0$ and half-spaces are Lebesgue-measurable, so
    \[
        \lambda(A_i)
            = \lambda(A_i \cap H(v, r)) + \lambda(A_i \cap H(-v, -r))
            = h_i(v, r) + h_i(-v, -r) < \infty.
    \]
    In other words, the plane $P(v, r)$ divides $A_i$ into two pieces, one with measure $h_i(v, r)$ and the other with measure $h_i(-v, -r)$.
    Define the function $f_i : S^2 \times \R \to \R$ by
    \[
        f_i(v, r) = h_i(v, r) - h_i(-v, -r).
    \]
    Notice that $f_i(v, r) = -f_i(-v, -r)$ and $f_i(v, r) = 0$ if and only if $P(v, r)$ divides $A_i$ into two pieces of equal measure.

    For a fixed $v \in S^2$, there is a real function $\R \to \R$ where $r \mapsto f_i(v, r)$.
    As $A_i \seq \R^3$ is compact, it is contained in a ball $B_R(0)$ for some $R > 0$.
    On one hand,
    \[
        A_i \seq B_R(0) \seq H(v, -R),
    \]
    which means $h_i(v, -R) = \lambda(A_i)$, so $f_i(v, -R) = \lambda(A_i)$.
    On the other hand,
    \[
        A_i \seq B_R(0) \seq (\R^3 \setminus H(v, R)),
    \]
    which means $h_i(v, R) = 0$, so $f_i(v, R) = -\lambda(A_i)$.
    One can check that $f_i$ is continuous, so the intermediate value theorem tells us there must be some $r \in (-R, R)$ such that $f_i(v, r) = 0$.
    We can therefore make a choice of function $r_i : S^2 \to \R$ such that $f_i(v, r_i(v)) = 0$ for all $v \in S^2$.
    
    \textit{Note.} We will assume that $r_i$ may be chosen to be continuous.
    I believe this can be shown using the hypothesis that $A_i$ is compact, since that should allow for small changes in $v$ to necessitate (at most) small changes in $r_i(v)$.
    Moreover, we would like for $r_i(-v) = -r_i(v)$.

    For the next construction, we will use $r_3$ to ensure that $A_3$ is always divided into parts of equal measure.
    Define the function $f : S^2 \to \R^2$ by
    \[
        f(v) = \big(f_1(v, r_3(v)),\; f_2(v, r_3(v))\big).
    \]
    By the Borsuk-Ulam theorem, there is a point $v \in S^2$ such that $f(v) = f(-v)$.
    But since
    \[
        f_i(-v, r_3(-v)) = f_i(-v, -r_3(v)) = -f_i(v, r_3(v)),
    \]
    we must have $f_i(v, r_3(v)) = 0$ for $i = 1, 2$.
    Since we have $f_3(v, r_3(v)) = 0$ by choice of $r_3$, we conclude that the plane $P(v, r_3(v))$ divides each $A_i$ into two pieces of equal measure.
\end{proof}

\begin{pbox}[8 Exercise 1.11]
    If $X_0$ is the path-component of a space $X$ containing the basepoint $x_0$, show that the inclusion $X_0 \inc X$ induces an isomorphism $\pi_1(X_0, x_0) \to \pi_1(X, x_0)$.
\end{pbox}

\begin{proof}
    Suppose $[f] \in \pi_1(X_0, x_0)$ is is the kernel of the induced homomorphism, which means there is a homotopy of loops $f \htpy \const_{x_0}$ in $X$.
    Since $f$ is a loop in $X_0$, and the homotopy to $\const_{x_0}$ takes every point on $f(I)$ to $x_0$ through a path in $X$, that path must be contained in the path-component of $x_0$.
    That is, the homotopy from $f$ to $\const_{x_0}$ must be entirely contained in $X_0$, which means $f \htpy \const_{x_0}$ in $X_0$.
    We conclude that $[f] = [\const_{x_0}] \in \pi_1(X_0, x_0)$, hence the induced homomorphism is injective.

    Given any homotopy class $[f] \in \pi_1(X, x_0)$, the loop $f$ is a path in $X$ with both endpoints at $x_0$.
    So every point along the path is path-connected to $x_0$, and therefore contained in the path-component of $x_0$.
    This means that $f$ is a loop in $X_0$, so $[f] \in \pi_1(X_0, x_0)$.
    Hence, the induced homomorphism is surjective.
\end{proof}

\begin{pbox}[9 Exercise 1.13]
    Given a space $X$ and a path-connected subspace $A$ containing the basepoint $x_0$, show that the map $\pi_1(A, x_0) \to \pi_1(X, x_0)$ induced by the inclusion $A \inc X$ is surjective iff every path in $X$ with endpoints in $A$ is homotopic to a path in $A$.
\end{pbox}

\begin{proof}
    Suppose the map is surjective and let $f : I \to X$ be a path with endpoints in $A$.
    Since $A$ is path-connected, there exist paths $\alpha, \beta : I \to A$ such that
    \[
        \alpha(0) = x_0,
            \quad \alpha(1) = f(0),
            \qquad \beta(0) = f(1),
            \quad \beta(1) = x_0.
    \]
    Then $\alpha \cdot f \cdot \beta$ is a loop in $X$ based at $x_0$.
    By assumption, there is a loop $g : (I, \bd I) \to (A, x_0)$ such that $[\alpha \cdot f \cdot \beta] = [g] \in \pi_1(X, x_0)$.
    That is, there is a homotopy of paths $\alpha \cdot f \cdot \beta \htpy g$ in $X$, described by a map $H : I \times I \to X$ with $H(s, 0) = \alpha \cdot f \cdot \beta$ and $H(s, 1) = g$.
    The parameter space of $H$ can be drawn as follows:
    \begin{drawing}
        \filldraw (0, 0) circle (2pt) node[anchor=north east] {$x_0$};
        \filldraw (3, 0) circle (2pt) node[anchor=north west] {$x_0$};
        \filldraw (0, 2) circle (2pt) node[anchor=south east] {$x_0$};
        \filldraw (3, 2) circle (2pt) node[anchor=south west] {$x_0$};

        \filldraw (1, 2) circle (2pt) node[anchor=south] {$f(0)$};
        \filldraw (2, 2) circle (2pt) node[anchor=south] {$f(1)$};


        \draw (0, 0) rectangle (3, 2);

        \draw[->] (-0.5, 1.5) -- (-0.5, 0.5) node[midway, anchor=east] {$t$};
        \draw[->] (1, -0.5) -- (2, -0.5) node[midway, anchor=north] {$s$};
    \end{drawing}
    The three segments along the top edge are $\alpha$, $f$, and $\beta$, respectively, and the bottom edge is $g$.
    In this parameter space, we construct a homotopy $\gamma_t : I \to I \times I$ of paths from the middle segment on the top edge to the path around the edge using the other segments.
    In particular, we split the top middle segment into fifths, each being sent to the corresponding segment on the edge of the square.
    The homotopy could look something like the following:
    \begin{drawing}
        \filldraw (0, 0) circle (2pt);
        \filldraw (3, 0) circle (2pt);
        \filldraw (0, 2) circle (2pt);
        \filldraw (3, 2) circle (2pt);
        \filldraw (1, 2) circle (2pt);
        \filldraw (2, 2) circle (2pt);

        \draw (0, 0) rectangle (3, 2);

        \draw[dashed] (1.4, 2) -- (0, 0);
        \draw[dashed] (1.6, 2) -- (3, 0);
        \draw[dashed] (0, 2) arc (-135 : -45 : 0.849);
        \draw[dashed] (3, 2) arc (-45 : -135 : 0.849);

        \draw[thick] (1, 2) -- (0.6, 1.75) -- (0.7, 1) -- (2.3, 1) node[midway, above] {$\gamma_t$} -- (2.4, 1.75) -- (2, 2);

        \node[anchor=south] at (1.5, 2) {$\gamma_0$};
        \node[anchor=north] at (1.5, 0) {$\gamma_1$};
    \end{drawing}
    With a reparameterization, this gives us the homotopy of paths
    \[
        f
            = \gamma_0
            \htpy \gamma_1
            \htpy \overline{\alpha} \cdot \const_{x_0} \cdot g \cdot \const_{x_0} \cdot \overline{\beta}
            \htpy \overline{\alpha} \cdot g \cdot \overline{\beta}.
    \]
    Since, $\overline{\alpha}$, $g$, and $\overline{\beta}$ are all paths in $A$, so is their composition.
    Hence, this is the desired homotopy of $f$.

    Suppose every path in $X$ with endpoints in $A$ is homotopic to a path in $A$.
    Choose any homotopy class $[f] \in \pi_1(X, x_0)$, for a loop $f : (I, \bd I) \to (X, x_0)$.
    Since $x_0 \in A$, then by assumption, $f$ is homotopic to a path $g : (I, \bd I) \to (A, x_0)$.
    Then $[g]$ is a well-defined homotopy class in $\pi_1(A, x_0)$, and we have $[f] = [g] \in \pi_1(X, x_0)$.
    Hence, the induced map is surjective.
\end{proof}


\end{document}