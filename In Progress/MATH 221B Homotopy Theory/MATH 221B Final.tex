\documentclass[12pt]{article}

% Packages
\usepackage[margin=1in]{geometry}
\usepackage{fancyhdr, parskip}
\usepackage{amsmath, amsthm, amssymb}
\usepackage{tikz, tikz-cd}
\usetikzlibrary{decorations.markings}

% Page Style
\makeatletter
\fancypagestyle{title}{
    \renewcommand{\headrulewidth}{0.4pt}
    \setlength{\headheight}{15pt}
    \fancyhead[R]{\@author}
    \fancyhead[L]{\@title}
    \fancyhead[C]{\@date}
}
\makeatother
\renewcommand{\maketitle}{\thispagestyle{title}}
\fancypagestyle{plain}{
    \fancyhf{}
    \renewcommand{\headrulewidth}{0pt}
    \renewcommand{\footrulewidth}{0pt}
    \fancyfoot[R]{\thepage}
}
\pagestyle{plain}

% Problem Box
\setlength{\fboxsep}{4pt}
\newlength{\myparskip}
\setlength{\myparskip}{\parskip}
\newsavebox{\savefullbox}
\newenvironment{fullbox}{\begin{lrbox}{\savefullbox}\begin{minipage}{\dimexpr\textwidth-2\fboxsep\relax}\setlength{\parskip}{\myparskip}}{\end{minipage}\end{lrbox}\framebox[\textwidth]{\usebox{\savefullbox}}}
\newenvironment{pbox}[1][]{\begin{fullbox}\ifx#1\empty\else\paragraph{#1}\phantom{}\fi}{\end{fullbox}}

% Theorem Environments
\theoremstyle{definition}
\newtheorem{lemma}{Lemma}

% Tikz Environments
\newenvironment{drawing}{\begin{center}\begin{tikzpicture}}{\end{tikzpicture}\end{center}}
% \tikzcdset{row sep/normal=0pt}
\newenvironment{cd}{\begin{center}\begin{tikzcd}}{\end{tikzcd}\end{center}}

% Default Commands
\newcommand{\isp}[1]{\quad\text{#1}\quad}
\newcommand{\N}{\mathbb{N}} 
\newcommand{\Z}{\mathbb{Z}}
\newcommand{\Q}{\mathbb{Q}}
\newcommand{\R}{\mathbb{R}}
\newcommand{\C}{\mathbb{C}}
\newcommand{\A}{\mathbb{A}}
\renewcommand{\P}{\mathbb{P}}
\newcommand{\eps}{\varepsilon}
\renewcommand{\phi}{\varphi}
\renewcommand{\emptyset}{\varnothing}
\newcommand{\<}{\langle}
\renewcommand{\>}{\rangle}
\newcommand{\isom}{\cong}
\newcommand{\eqc}{\overline}
\newcommand{\clo}{\overline}
\newcommand{\seq}{\subseteq}
\newcommand{\teq}{\trianglelefteq}
\DeclareMathOperator{\id}{id}
\DeclareMathOperator{\im}{im}
\newcommand{\inc}{\hookrightarrow}

% Extra Commands
\newcommand{\bd}{\partial}
\newcommand{\htpy}{\simeq}
\newcommand{\To}{\longrightarrow}
\newcommand{\Mapsto}{\longmapsto}

\DeclareMathOperator{\inter}{int}

% Document
\begin{document}
\title{MATH 221B Final}
\author{Harry Coleman}
\date{March 18, 2022}
\maketitle

\vspace*{0pt}

\begin{pbox}[1]
    
\end{pbox}

We first prove the generalization of Hatcher Lemma 1.19.

\begin{lemma}
    If $\phi_t : X \to Y$ is a homotopy and $h$ is the path $\phi_t(x_0)$ formed by the images of a basepoint $x_0 \in X$, then the three maps in the following diagram satisfy $\phi_{0*} = \beta_h \circ \phi_{1*}$.
    \begin{cd}[row sep=tiny, column sep=large]
        & \pi_n(Y, \phi_1(x_0)) \ar[dd, "\beta_h"]\\
        \pi_n(X, x_0) \urar["\phi_{1*}"] \drar["\phi_{0*}"'] \\
        & \pi_n(Y, \phi_0(x_0))
    \end{cd}
\end{lemma}

\begin{proof}
    Let $h_t$ be the restriction of $h$ to the interval $[0, t]$, with a reparameterization so that the domain of $h_t$ is still $[0, 1]$.
    Explicitly, we can take $h_t(s) = h(ts)$.
    Given a map $f : (I^n, \bd I^n) \to (X, x_0)$, the product $h_t \cdot (\phi_t \circ f)$  gives a homotopy of maps $(I^n, \bd I^n) \to (Y, \phi_0(x_0))$.
    Restricting this homotopy to $t = 0$ and $t = 1$, we see that $\phi_{0*}[f] = \beta_h (\phi_{1*}[f])$.
\end{proof}

\begin{pbox}
    Show that if $f : X \to Y$ is a homotopy equivalence and $y_0 = f(x_0)$ then the induced homomorphism $f_* : \pi_n(X, x_0) \to \pi_n(Y, y_0)$ is an isomorphism. [$n = 1$ was done in class.]
\end{pbox}

\begin{proof}
    Let $g : Y \to X$ be a homotopy inverse to $f$, so that $g \circ f \htpy \id_X$ and $f \circ g \htpy \id_Y$.
    Consider the sequence of group homomorphisms
    \begin{cd}
        \pi_n(X, x_0) \rar["f_*"] & \pi_n(Y, y_0) \rar["g_*"] & \pi_n(X, g(y_0)) \rar["f_*"] & \pi_n(Y, (f \circ g)(y_0)).
    \end{cd}
    By the lemma, $g_* \circ f_* = \beta_h$ for some $h$.
    In particular, since $\beta_h$ is an isomorphism,
    \[
        0 = \ker \beta_h = \ker (g_* \circ f_*) = f_*^{-1}(\ker g_*) \supseteq f_*^{-1}(0) = \ker f_*.
    \]
    This implies $\ker f_* = 0$ so $f_*$ is injective.
    Again, the lemma tells us $f_* \circ g_* = \beta_h$ for some $h$.
    In particular, since $\beta_h$ is an isomorphism,
    \[
        \pi_n(Y, (f \circ g)(y_0))
            = \im \beta_h
            = \im (f_* \circ g_*)
            = \im f_*|_{\im g_*}
            \seq \im f_*.
    \]
    This implies $\im f_* = \pi_n(Y, (f \circ g)(y_0))$ so $f_*$ is surjective.
\end{proof}



\newpage
\begin{pbox}[2]
    If $X$ is simply connected carefully show that $\bd : \pi_2(X, A, x_0) \to \pi_1(A, a_0)$ is surjective.
    You may not assume any results about exact sequences.
\end{pbox}

\begin{proof}
    Let $f : (I, \bd I) \to (A, a_0)$ be a loop in $A$ based at $a_0$, representing an arbitrary element $[f] \in \pi_1(A, a_0)$.
    Since $X$ is simply connected, we have
    \[
        [f] = [c] \in \pi_n(X, a_0),
    \]
    where $c \equiv a_0$ is the constant loop $(I, \bd I) \to (X, a_0)$.
    In other words, $f \htpy c$ in $X$ relative to basepoints.
    Suppose $H_t : (I, \bd I) \to (X, a_0)$ is a homotopy with $H_0 = f$ and $H_1 = c$.
    Then the parameter space of $H : I \times I \to X$ can be drawn as follows:
    \begin{drawing}
        \draw[fill=gray!20] (0, 0) rectangle (2, 2);
        \foreach \x in {0, 2} {
            \foreach \y in {0, 2} {
                \fill (\x, \y) circle (2pt);
            }
        }
        

        \draw[->] (-0.75, 0.5) -- (-0.75, 1.5) node[midway, left] {$t$};
        \draw[->] (0.5, -0.75) -- (1.5, -0.75) node[midway, below] {$s$};

        \node[below] at (1, 0) {$f$};
        \node[left] at (0, 1) {$c$};
        \node[right] at (2, 1) {$c$};
        \node[above] at (1, 2) {$c$};
        
        \node[below left] at (0, 0) {$a_0$};
        \node[below right] at (2, 0) {$a_0$};
        \node[above left] at (0, 2) {$a_0$};
        \node[above right] at (2, 2) {$a_0$};
    \end{drawing}
    We can interpret this as being the square ($2$-cube) $I^2$.
    Under the map $H : I^2 \to X$, the front face $F^1$ is mapped via $f$, in particular $H(F^1) \seq A$, and the remaining faces $J^1$ are mapped constantly to $a_0$.
    Therefore, $H$ can be viewed as a map $H : (I^2, F^1, J^1) \to (X, A, a_0)$.
    We have thus found an element $[H] \in \pi_2(X, A, a_0)$ such that
    \[
        \bd[H] = [H|_{F^1}] = [f] \in \pi_1(A, a_0).
    \]
\end{proof}

\newpage
\begin{pbox}[3]
    In each case either prove a covering exists or that no such covering exists.
    You may state what the fundamental groups of spaces are without proof.
\end{pbox}

\begin{pbox}[(a)]
    $p : S^2 \to T^2$
\end{pbox}

No such covering exists.
By Hatcher Proposition 4.1, such a covering would induce an isomorphism
\begin{cd}
    \Z \isom \pi_2(S^2) \rar["p_*"] & \pi_2(T^2) = 0,
\end{cd}
but this is not possible.

\begin{pbox}[(b)]
    $p : T^2 \to S^2$
\end{pbox}

No such covering exists.
By Hatcher Proposition 1.31, such a covering would induce an injection
\begin{cd}
    Z^2 \isom \pi_1(T^2) \rar["p_*"] & \pi_1(S^2) = 0,
\end{cd}
but this is not possible.


\begin{pbox}[(c)]
    $p : \R^2 \to T^2$
\end{pbox}

Let $q : \R \to S^1$ be the usual covering and define the product map
\[
    p = q \times q : \R^2 \To S^2.
\]
We claim that this map as the even covering property.

\begin{proof}
    For a given point $(x, y) \in S^2$, we choose neighborhoods $x \in U \seq S^1$ and $y \in V \seq S^1$ such that each component $\tilde{U}_i \seq \R$ of $q^{-1}(U)$ is mapped homeomorphically to $U$ and each component $\tilde{V}_j \seq \R$ of $q^{-1}(V)$ is mapped homeomorphically to $V$.
    Since the finite product topology is simply the box topology, $U \times V \seq S^2$ is an open neighborhood of $(x, y)$.
    Moreover, the components of
    \[
        p^{-1}(U \times V) = (q \times q)^{-1}(U \times V) = q^{-1}(U) \times q^{-1}(V) \seq \R^2
    \]
    are precisely the sets $\tilde{U}_i \times \tilde{V}_j \seq \R^2$, each of which is mapped homeomorphically to $U \times V$ via our map $p = q \times q$.
\end{proof}

\begin{pbox}[(d)]
    $p : \R^2 \to S^2$
\end{pbox}

No such covering exists.
By Hatcher Proposition 4.1, such a covering would induce an isomorphism
\begin{cd}
    0 = \pi_2(\R^2) \rar["p_*"] & \pi_2(S^2) \isom \Z,
\end{cd}
but this is not possible.

\newpage
\begin{pbox}[4]
    The cell complex $X$ is a wedge of two circles with one $0$-cell $x_0$, and two $1$-cells and $a$ and $b$.
    Let $[a], [b] \in \pi_1(X, x_0)$ denote the elements represented by loops that go once along $a$ and $b$ respectively.
    For parts (i) and (ii) of the following carefully draw a covering space $\tilde{X}$ of $X$, label each $1$-cell of $\tilde{X}$ by $a$ or $b$ according to which $1$-cell it projects to, and put an arrow on it to indicate the covering (as is done in the text on page 58).
    More a basepoint $\tilde{x}_0 \in \tilde{X}$ that maps to $x_0$, and let $H$ be the subgroup of $\pi_1(X, x_0)$ determined by the covering and $\tilde{x}_0$.
    In each part \textbf{prove your answer is correct}
\end{pbox}



\begin{drawing}
    \fill (0, 0) circle (2pt) node[below, yshift=-3pt] {$x_0$};
    \draw[->] (0, 0) to[out=-45, in=-90] (2, 0) node[right] {$a$};
    \draw (2, 0) to[out=90, in=45] (0, 0);

    \draw[->] (0, 0) to[out=135, in=90] (-2, 0) node[left] {$b$};
    \draw (-2, 0) to[out=-90, in=-135] (0, 0);
\end{drawing}

\begin{pbox}[(i)]
    Draw the covering $p_1 : \tilde{X}_1 \to X$ with $H = \<a^3, b, aba^2, a^2ba\>$.
\end{pbox}

We start with the required paths for each of the generators of $H$:
\begin{drawing}
    \begin{scope}[yshift=1cm]
        \fill (-90 : 1) circle (2pt) node[below] {$\tilde{x}_0$};
        % \fill (30 : 1) circle (2pt) node[right] {$\tilde{x}_1$};
        % \fill (150 : 1) circle (2pt) node[left] {$\tilde{x}_2$};

        \begin{scope}[decoration={
            markings,
            mark=at position 0.5 with {\arrow{>}}}
            ] 
            \draw[postaction={decorate}] (-90 : 1) -- (30 : 1) node[midway, below right] {$a$};
            \draw[postaction={decorate}] (30 : 1) -- (150 : 1) node[midway, above] {$a$};
            \draw[postaction={decorate}] (150 : 1) -- (-90 : 1) node[midway, below left] {$a$};
        \end{scope}
    \end{scope}

    \begin{scope}[xshift=3cm]
        \fill (0, 0) circle (2pt) node[below] {$\tilde{x}_0$};
        \draw[->] (0, 0) to[out=45, in=0] (0, 1) node[above] {$b$};
        \draw(0, 1) to[out=180, in=135] (0, 0);
    \end{scope}

    \begin{scope}[xshift=6cm, yshift=1cm]
        \fill (-90 : 1) circle (2pt) node[below] {$\tilde{x}_0$};
        % \fill (0 : 1) circle (2pt) node[right] {$\tilde{x}_2$};

        \begin{scope}[decoration={
            markings,
            mark=at position 0.5 with {\arrow{>}}}
            ] 
            \draw[postaction={decorate}] (-90 : 1) -- (0 : 1) node[midway, below right] {$a$};
            \draw[postaction={decorate}] (0 : 1) -- (90 : 1) node[midway, above right] {$b$};
            \draw[postaction={decorate}] (90 : 1) -- (180 : 1) node[midway, above left] {$a$};
            \draw[postaction={decorate}] (180 : 1) -- (-90 : 1) node[midway, below left] {$a$};
        \end{scope}
    \end{scope}

    \begin{scope}[xshift=10cm, yshift=1cm]
        \fill (-90 : 1) circle (2pt) node[below] {$\tilde{x}_0$};
        % \fill (0 : 1) circle (2pt) node[right] {$\tilde{x}_1$};
        % \fill (90 : 1) circle (2pt) node[above] {$\tilde{x}_2$};
        % \fill (180 : 1) circle (2pt) node[left] {$\tilde{x}_2$};

        \begin{scope}[decoration={
            markings,
            mark=at position 0.5 with {\arrow{>}}}
            ] 
            \draw[postaction={decorate}] (-90 : 1) -- (0 : 1) node[midway, below right] {$a$};
            \draw[postaction={decorate}] (0 : 1) -- (90 : 1) node[midway, above right] {$a$};
            \draw[postaction={decorate}] (90 : 1) -- (180 : 1) node[midway, above left] {$b$};
            \draw[postaction={decorate}] (180 : 1) -- (-90 : 1) node[midway, below left] {$a$};
        \end{scope}
    \end{scope}
\end{drawing}

Each vertex of $\tilde{X}$ must have each of $a$ and $b$ entering and exiting exactly once.
Taking $\tilde{x}_0$ as the basepoint, we can deduce what identifications are necessary so that $\tilde{X}$ is a covering of $X$.
Without loss of generality, we may label the two other vertices of the $a^3$ loop as $\tilde{x_1}$ and $\tilde{x_2}$.
Then, in each of the remaining loops, we must end end up at the same vertex after following any path.
In particular, following the path of either $a$ or $a^{-2}$ from $\tilde{x}_0$ must land at $\tilde{x}_1$ and following the path of either $a^2$ or $a^{-1}$ from $\tilde{x}_0$ must land at $\tilde{x}_2$.
Hence, the vertices must be labeled as follows:
\begin{drawing}
    \begin{scope}[yshift=1cm]
        \fill (-90 : 1) circle (2pt) node[below] {$\tilde{x}_0$};
        \fill (30 : 1) circle (2pt) node[right] {$\tilde{x}_1$};
        \fill (150 : 1) circle (2pt) node[left] {$\tilde{x}_2$};

        \begin{scope}[decoration={
            markings,
            mark=at position 0.5 with {\arrow{>}}}
            ] 
            \draw[postaction={decorate}] (-90 : 1) -- (30 : 1) node[midway, below right] {$a$};
            \draw[postaction={decorate}] (30 : 1) -- (150 : 1) node[midway, above] {$a$};
            \draw[postaction={decorate}] (150 : 1) -- (-90 : 1) node[midway, below left] {$a$};
        \end{scope}
    \end{scope}

    \begin{scope}[xshift=3cm]
        \fill (0, 0) circle (2pt) node[below] {$\tilde{x}_0$};
        \draw[->] (0, 0) to[out=45, in=0] (0, 1) node[above] {$b$};
        \draw(0, 1) to[out=180, in=135] (0, 0);
    \end{scope}

    \begin{scope}[xshift=6cm, yshift=1cm]
        \fill (-90 : 1) circle (2pt) node[below] {$\tilde{x}_0$};
        \fill (0 : 1) circle (2pt) node[right] {$\tilde{x}_1$};
        \fill (90 : 1) circle (2pt) node[above] {$\tilde{x}_1$};
        \fill (180 : 1) circle (2pt) node[left] {$\tilde{x}_2$};

        \begin{scope}[decoration={
            markings,
            mark=at position 0.5 with {\arrow{>}}}
            ] 
            \draw[postaction={decorate}] (-90 : 1) -- (0 : 1) node[midway, below right] {$a$};
            \draw[postaction={decorate}] (0 : 1) -- (90 : 1) node[midway, above right] {$b$};
            \draw[postaction={decorate}] (90 : 1) -- (180 : 1) node[midway, above left] {$a$};
            \draw[postaction={decorate}] (180 : 1) -- (-90 : 1) node[midway, below left] {$a$};
        \end{scope}
    \end{scope}

    \begin{scope}[xshift=10cm, yshift=1cm]
        \fill (-90 : 1) circle (2pt) node[below] {$\tilde{x}_0$};
        \fill (0 : 1) circle (2pt) node[right] {$\tilde{x}_1$};
        \fill (90 : 1) circle (2pt) node[above] {$\tilde{x}_2$};
        \fill (180 : 1) circle (2pt) node[left] {$\tilde{x}_2$};

        \begin{scope}[decoration={
            markings,
            mark=at position 0.5 with {\arrow{>}}}
            ] 
            \draw[postaction={decorate}] (-90 : 1) -- (0 : 1) node[midway, below right] {$a$};
            \draw[postaction={decorate}] (0 : 1) -- (90 : 1) node[midway, above right] {$a$};
            \draw[postaction={decorate}] (90 : 1) -- (180 : 1) node[midway, above left] {$b$};
            \draw[postaction={decorate}] (180 : 1) -- (-90 : 1) node[midway, below left] {$a$};
        \end{scope}
    \end{scope}
\end{drawing}

Making these identifications, we obtain $\tilde{X}_1$ as the following space:
\begin{drawing}
    \fill (-90 : 1) circle (2pt) node[below] {$\tilde{x}_0$};
    \fill (30 : 1) circle (2pt) node[right, yshift=5pt] {$\tilde{x}_1$};
    \fill (150 : 1) circle (2pt) node[left, yshift=5pt] {$\tilde{x}_2$};

    \begin{scope}[decoration={
        markings,
        mark=at position 0.5 with {\arrow{>}}}
        ] 
        \draw[postaction={decorate}] (-90 : 1) -- (30 : 1) node[midway, below right] {$a$};
        \draw[postaction={decorate}] (30 : 1) -- (150 : 1) node[midway, above] {$a$};
        \draw[postaction={decorate}] (150 : 1) -- (-90 : 1) node[midway, below left] {$a$};

        \draw[postaction={decorate}] (-90 : 1) arc (-270 : 90 : 0.5) node[midway, below] {$b$};
        \draw[postaction={decorate}] (30 : 1) arc (-150 : 210 : 0.5) node[midway, right] {$b$};
        \draw[postaction={decorate}] (150 : 1) arc (-30 : 330 : 0.5) node[midway, left] {$b$};
    \end{scope}
\end{drawing}

We apply Hatcher Proposition 1A.2 to check the fundamental group.
Take $T$ to be the maximal tree of the edges corresponding to the path $a^2$ from the basepoint.
Then we obtain a generator for each of the edges in $X \setminus T$:
\[
    H = \pi_1(\tilde{X}_1, \tilde{x_0}) = \<a^3, b, aba^{-1}, a^2ba^{-2}\>.
\]
Multiplying the last two generators on the right by $a^3$ gives us the original generating set.

\begin{pbox}[(ii)]
    Draw a cover $p_2 : \tilde{X}_2 \to X$ of degree $3$ that is not normal and not shown on page 58 and write down $H$ for this cover.
    You need to prove the cover is not normal.
\end{pbox}

There are four undirected graphs on $3$ vertices of degree $4$ (that is degree $4$ as vertices in the graph):
\begin{drawing}
    \begin{scope}[xshift=-0.5cm]
        \draw (0, 0) circle (1);
        \draw (-90 : 1) -- (30 : 1) -- (150 : 1) -- cycle;
    \end{scope}
    \begin{scope}[xshift=3cm, yshift=0.2cm]
        \draw (-90 : 0.75) -- (30 : 0.75) -- (150 : 0.75) -- cycle;
        \draw (-90 : 0.75) arc (-270 : 90 : 0.3);
        \draw (30 : 0.75) arc (-150 : 210 : 0.3);
        \draw (150 : 0.75) arc (-30 : 330 : 0.3);
    \end{scope}
    \begin{scope}[xshift=5.5cm]
        \draw (0, 0) circle (0.5);
        \draw (1, 0) circle (0.5);
        \draw (2, 0) circle (0.5);
        \draw (3, 0) circle (0.5);
    \end{scope}
    \begin{scope}[xshift=12cm]
        \draw (-1, 0) arc (180 : 270 : 1);
        \draw (-90 : 1) arc (-90 : 90 : 1);
        \draw (90 : 1) arc (90 : 180 : 1);
        
        \draw (90 : 1) arc (135 : 225 : 1.414);
        \draw (-90 : 1) arc (-45 : 45 : 1.414);
        \draw (180 : 1) arc (0 : 360 : 0.5);
    \end{scope}
\end{drawing}

The first two of these will certainly be normal and the third is present on page 58.
And, indeed, the fourth is $2$-orientable giving us $\tilde{X}_2$ as follows:
\begin{drawing}
    \fill (-1, 0) circle (2pt);
    \begin{scope}[decoration={
        markings,
        mark=at position 0.5 with {\arrow{>}}}
        ] 
        \draw[postaction={decorate}] (-1, 0) arc (180 : 270 : 1) node[midway, below left] {$a$};
        \draw[postaction={decorate}] (-90 : 1) arc (-90 : 90 : 1) node[midway, right] {$a$};
        \draw[postaction={decorate}] (90 : 1) arc (90 : 180 : 1) node[midway, above left] {$a$};
        
        \draw[postaction={decorate}] (-90 : 1) arc (225 : 135 : 1.414) node[midway, below right] {$b$};
        \draw[postaction={decorate}] (90 : 1) arc (45 : -45 : 1.414) node[midway, above left] {$b$};
        \draw[postaction={decorate}] (180 : 1) arc (0 : 360 : 0.5) node[midway, left] {$b$};
    \end{scope}
\end{drawing}
This covering is not normal as the marked vertex has a loop of $b$, while neither of the other two vertices do.
This means that there can be no graph automorphism taking the vertex to either of the other vertices.
As the graph automorphisms of $\tilde{X}_2$ are precisely the automorphisms as a covering space $p_2 : \tilde{X}_2 \to X$, we conclude that there are no automorphisms of $\tilde{X}_2$ sending the lifted basepoint $\tilde{x}_0$ to any other lifted basepoint, i.e., $\tilde{X}_2$ is not normal.

We apply Hatcher Proposition 1A.2 to compute the fundamental group.
Take $T$ to be the maximal tree of the edges corresponding to the path $a^2$ from the basepoint.
Then we obtain a generator for each of the edges in $X \setminus T$:
\[
    H = \pi_1(\tilde{X}_2, \tilde{x_0}) = \<a^3, b, aba, a^2ba^{-1}\>.
\]
We can check again that this is not normal; consider $a^2b = a(aba)a^{-1} \in aHa^{-1}$.
Following the corresponding path in $\tilde{X}_2$ does not return to the basepoint, so this element could not be in $H$, hence $H$ is not a normal subgroup.

\newpage
\begin{pbox}[(iii)]
    Does the map $p_2$ lift to a map $\tilde{p}_2 : \tilde{X}_2 \to \tilde{X}_1$ so that $p_1 \circ \tilde{p}_2 = p_2$?
    (it is not required that $\tilde{p}_2$ sends basepoint to basepoint.)
\end{pbox}

No. If $\tilde{p}_2 : \tilde{X}_2 \to \tilde{X}_1$ were such a lift, then consider the loop $\gamma : (I, \bd I) \to X$ corresponding to $a^2b$.
This uniquely lifts to a path $\tilde{\gamma}_2 : I \to \tilde{X}_2$ such that $p_2 \circ \tilde{\gamma}_2 = \gamma$.
Notably, $\tilde{\gamma}_2$ is a loop in $\tilde{X}_2$.
Therefore, the composition $\tilde{p}_2 \circ \tilde{\gamma}_2 : I \to \tilde{X}_1$ is a loop in $\tilde{X}_1$ such that
\[
    p_1 \circ (\tilde{p_2} \circ \tilde{\gamma}_2) = p_2 \circ \tilde{\gamma}_2 = \gamma.
\]
But $\gamma$ uniquely lifts to a path $\tilde{\gamma}_1 : I \to \tilde{X}_1$ such that $p_1 \circ \tilde{\gamma}_1 = \gamma$, so $\tilde{\gamma}_1 = \tilde{p_2} \circ \tilde{\gamma}_2$.
However, $\tilde{\gamma}_1$ is not a loop in $\tilde{X}_1$, which is a contradiction.
(In order for the composition to work out, $\tilde{p}_2(\tilde{x}_0)$ must be one of the vertices of $\tilde{X}_1$, and $a^2b$ is not a loop based at any of these vertices.)




\newpage
\begin{pbox}[5]
    Given a sequence $f : \N \to (0, \infty)$ define $p_k = (0, 0, f(k))$ and $C_k$ is the boundary of the triangle in the plane $\{(x, y, 0) \in \R^3\}$ with vertices $0, v_{2k}, v_{2k+1}$ where $v_n = (1/n, 1/n^2, 0)$.
    For $k \in \N$ the set
    \[
        A_k = \{t \cdot q + (1 - t) \cdot p_k : q \in C_k \quad 0 \leq t \leq 1\}
    \]
    is the cone from $p_k$ to $C_k$.
    The subspace $X \seq \R^3$ is $X = \bigcup A_k$.
\end{pbox}

\begin{pbox}[(a)]
    If $f(k) = 1$ for all $k$, prove that $X$ is simply connected.
\end{pbox}

\begin{proof}
    In this case, $p_k = (0, 0, 1) = e_3$ for all $k \in \N$ and $X$ contains the line segment between $e_3$ and any point in $X$.
    In particular, $X$ is path-connected, where a path between any two points can be found through their line segments to $e_3$.
    We construct a deformation retraction of $X$ to the point $e_3$ as the straight line homotopy
    \begin{align*}
        H : X \times I &\To X, \\
            (q, t) &\Mapsto (1 - t) \cdot q + t \cdot e_3.
    \end{align*}
    In particular, Hatcher Proposition 1.17 gives us an isomorphism
    \[
        \pi_1(X, e_3) \isom \pi_1(\{e_3\}, e_3) = 0.
    \]
    Hence, $X$ is simply connected.
\end{proof}

\begin{pbox}[(b)]
    If $f(k) = 1/k$ prove $X$ is simply connected.
\end{pbox}

\begin{proof}
    Similar to part (a), $X$ contains the line segment between $p_k$ and any point in $A_k$. Moreover, $X$ contains the line segment between any two focal points $p_k$ and $p_\ell$.
    Thus, a path between any two points of $X$ can be found by first taking the line segments to their respective focal points, then connecting the focal points via a path in the unit interval on the $z$-axis $\{0\} \times \{0\} \times [0, 1] \seq X$.


    Each $A_k$ is homeomorphic to a cell complex of the following form:
    \begin{drawing}
        \filldraw[fill=gray!20] (-90 : 1.5) -- (30 : 1.5) -- (150 : 1.5) -- cycle;
        \fill (-90 : 1.5) circle (1.5pt) node[below] {$0$};
        \fill (30 : 1.5) circle (1.5pt) node[right] {$v_{2k}$};
        \fill (150 : 1.5) circle (1.5pt) node[left] {$v_{2k+1}$};

        \draw (30 : 1.5) -- (0, 0) -- (150 : 1.5);
        \draw[very thick] (-90 : 1.5) -- (0, 0);

        \fill (0, 0) circle (1.5pt) node[below right] {$p_k$};
    \end{drawing}
    There is a deformation retraction of this space to the bold $1$-cell.
    Let $h_k : A_k \times I \to A_k$ be the corresponding deformation retraction of $A_k$ onto the line segment $\{0\} \times \{0\} \times [0, p_k]$.
    The $A_k$'s only overlap inside of $\{0\} \times \{0\} \times [0, 1]$, on which each $h_k$ is the identity, so there is a well-defined function $H : X \times I \to X$ such that $H|_{A_k \times I} = h^k$.
    However, it is not immediate that that $H$ is continuous.
    To use the pasting lemma, we would require a cover of open sets or a finite cover of closed sets---we have an infinite cover of closed sets $A_k \times I$.
    
    For $n \in \N$, define the subspace $X_n = \bigcup_{k=1}^{n} A_k$, so then $X_n \times I = \bigcup_{k=1}^{n} (A_k \times I)$.
    Since we have a cover of $X_n \times I$ by finitely many closed sets $A_k \times I$, on each of which $H$ is continuous, the pasting lemma tells us that $H$ is continuous on all of $X_n \times I$.
    The the continuity of $H$ at every point in $X_n \times I \setminus \{0\} \times I$ follows from the continuity in each $X_n \times I$, since every point of $X$ has a neighborhood contained in some $X_n$.
    
    The continuity on $\{0\} \times I$ is shown similarly.
    For an open neighborhood $U \seq X$ of $0$, its complement $X \setminus U$ is closed and is contained in some $X_n$.
    Then $H^{-1}(X \setminus U) = X \times I \setminus H^{-1}(U)$ is a closed subset of $X_n \times I$, which is a closed subspace of $X \times I$.
    We conclude that $H^{-1}(U)$ is open in $X$ since its complement is closed.

\end{proof}

\begin{pbox}[(c)]
    If $f(k) = k$ prove $X$ is not simply connected [hint: compactness]
\end{pbox}

\begin{proof}
    Define the path $\gamma : (I, \bd I) \to (X, 0)$ which makes a single loop around $C_k$ on the interval $[1/k, 1/(k+1)]$ for each $k \in \N$.
    If $X$ were simply connected, then there would be a homotopy $H_t : I \to X$ such that $H_0 = \gamma$, $H_1 \equiv 0$, and $H_t(0) = 0$.
    In particular, restricting $H$ to $[1/k, 1/(k+1)] \times I$ gives a homotopy of the loop around $C_k$ to the origin.
    The only possible way for this to happen is if the loop passes through the focal point $p_k \in A_k$ at some time during the homotopy.
    In particular, $p_k$ is in the image of $H$ for all $k \in \N$.
    However, $I \times I$ is a compact space, which implies its image $H(I \times I) \seq X \seq \R^3$ is also compact.
    But compactness in $\R^3$ requires boundedness and the set of focal points $\{p_k\} \seq \im H$ is unbounded, so this is a contradiction.
\end{proof}


\end{document}