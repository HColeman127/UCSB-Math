\documentclass[12pt]{article}

% Packages
\usepackage[margin=1in]{geometry}
\usepackage{fancyhdr, parskip}
\usepackage{amsmath, amsthm, amssymb}
\usepackage{tikz, tikz-cd}
\usepackage[shortlabels]{enumitem}

% Page Style
\makeatletter
\fancypagestyle{title}{
    \renewcommand{\headrulewidth}{0.4pt}
    \setlength{\headheight}{15pt}
    \fancyhead[R]{\@author}
    \fancyhead[L]{\@title}
    \fancyhead[C]{\@date}
}
\makeatother
\renewcommand{\maketitle}{\thispagestyle{title}}
\fancypagestyle{plain}{
    \fancyhf{}
    \renewcommand{\headrulewidth}{0pt}
    \renewcommand{\footrulewidth}{0pt}
    \fancyfoot[R]{\thepage}
}
\pagestyle{plain}

% Problem Box
\setlength{\fboxsep}{4pt}
\newlength{\myparskip}
\setlength{\myparskip}{\parskip}
\newsavebox{\savefullbox}
\newenvironment{fullbox}{\begin{lrbox}{\savefullbox}\begin{minipage}{\dimexpr\textwidth-2\fboxsep\relax}\setlength{\parskip}{\myparskip}}{\end{minipage}\end{lrbox}\framebox[\textwidth]{\usebox{\savefullbox}}}
\newenvironment{pbox}[1][]{\begin{fullbox}\ifx#1\empty\else\paragraph{#1}\phantom{}\fi}{\end{fullbox}}

% Theorem Environments
\theoremstyle{definition}
\newtheorem{lemma}{Lemma}

% Tikz Environments
\newenvironment{drawing}{\begin{center}\begin{tikzpicture}}{\end{tikzpicture}\end{center}}
% \tikzcdset{row sep/normal=0pt}
\newenvironment{cd}{\begin{center}\begin{tikzcd}}{\end{tikzcd}\end{center}}

% Default Commands
\newcommand{\isp}[1]{\quad\text{#1}\quad}
\newcommand{\N}{\mathbb{N}} 
\newcommand{\Z}{\mathbb{Z}}
\newcommand{\Q}{\mathbb{Q}}
\newcommand{\R}{\mathbb{R}}
\newcommand{\C}{\mathbb{C}}
\newcommand{\A}{\mathbb{A}}
\renewcommand{\P}{\mathbb{P}}
\newcommand{\eps}{\varepsilon}
\renewcommand{\phi}{\varphi}
\renewcommand{\emptyset}{\varnothing}
\newcommand{\<}{\langle}
\renewcommand{\>}{\rangle}
\newcommand{\isom}{\cong}
\newcommand{\eqc}{\overline}
\newcommand{\clo}{\overline}
\newcommand{\seq}{\subseteq}
\newcommand{\teq}{\trianglelefteq}
\DeclareMathOperator{\id}{id}
\DeclareMathOperator{\im}{im}

% Extra Commands
\newcommand{\FF}{\mathcal{F}}
\newcommand{\GG}{\mathcal{G}}

\DeclareMathOperator{\diam}{diam}
\newcommand{\dd}{\,\mathrm{d}}


% Document
\begin{document}
\title{MATH 201B Homework 1}
\author{Harry Coleman}
\date{February 2, 2022}
\maketitle

\begin{pbox}[1]
    Let $\FF$ be a collection of non-degenerate (not a point) closed balled in $\R^n$.
    A family $\GG \seq \FF$ of disjoint balls in $\FF$ is called maximal if $\GG$ consists of disjoint balls in $\FF$ such that, for any ball $B \in \FF$, there exists another ball $B' \in \GG$ such that $B \cap B' \ne \emptyset$.

    Prove that any such collection $\FF$ contains a maximal family $\GG$ of disjoint balls, if
    \begin{enumerate}[(i)]
        \item There exists two positive constants $d, D > 0$ such that any ball $B \in \FF$ satisfies the lower and upper bounds
        \[
            d \leq \diam(B) \leq D.
        \]
        \item $\FF$ is any collection of closed non-degenerate balls in $\R^n$.
    \end{enumerate}

    
\end{pbox}

\begin{proof}
    Let $\FF$ be any collection of closed non-degenerate balls in $\R^n$.
    Denote the collection of families of disjoint balls in $\FF$ by
    \[
        P = \{\GG \seq \FF : B \cap B' = \emptyset \text{ for all } B, B' \in \GG\}.
    \]
    We take $P$ with inclusion ``$\seq$'' to be a partially ordered set, and we will apply Zorn's lemma to find a maximal family.
    Suppose $C \seq P$ is a chain, i.e., a totally ordered subset.
    Define the family
    \[
        \mathcal{U} = \bigcup_{\GG \in C} \GG.
    \]
    By construction, we have $\GG \seq \mathcal{U}$ for all $\GG \in \mathcal{U}$, i.e., $\mathcal{U}$ is an upper bound for $C$ with respect to inclusion; we must check that $\mathcal{U} \in P$.
    Since $\GG \seq \FF$ for all $\GG \in C$, then we have $\mathcal{U} \seq \FF$.
    And for any balls $B, B' \in \mathcal{U}$, we must have $B \in \GG$ and $B' \in \GG'$ for some $\GG, \GG' \in C$.
    Since $C$ is totally ordered, we can assume $\GG \seq \GG'$, so then $B \in \GG'$.
    With $\GG' \in P$ and $B, B' \in \GG'$, we must have $B \cap B' = \emptyset$.
    Hence, $\mathcal{U}$ is a family of disjoint balls in $\FF$, so indeed $\mathcal{U} \in P$.

    As $P$ is a partially ordered set such that every chain has an upper bound in $P$, Zorn's lemma tells us that there exists some $\GG \in P$ which is maximal with respect to inclusion.
    It remains to prove that $\GG$ is maximal in sense of the problem.
    Suppose $\GG$ is not maximal in this sense, i.e., that there is some ball $B \in \FF$ which is disjoint from all balls in $\GG$.
    However, we could then take $\GG' = \GG \cup \{B\}$ to be another family of disjoint balls in $\FF$ which strictly contains $\GG$.
    That is, $\GG' \in P$ but $\GG \subsetneq \GG'$, which contradicts the maximality of $\GG$ in $P$ with respect to inclusion.
    Hence, $\GG$ is the desired family.
\end{proof}

\begin{pbox}
    Is the family $\GG$ unique in general?
\end{pbox}

No.

Consider the collection of closed balls
\[
    \FF = \{\clo{B}_{1/2}(n) \seq \R : n \in \Z\}.
\]
Then we have maximal families
\[
    \GG = \{\clo{B}_{1/2}(2n) \seq \R : n \in \Z\}
    \isp{and}
    \GG' = \{\clo{B}_{1/2}(2n+1) \seq \R : n \in \Z\}.
\]
It can be seen that these are maximal since each ball centered at an odd integer touches the two adjacent balls centered at even integers---and vice versa.

\newpage
\begin{pbox}[2]
    Let $\lambda$ be Lebesgue measure on $\R$.
\end{pbox}

\begin{pbox}[(a)]
    Construct a Lebesgue-integrable function $f : \R \to [-\infty, \infty]$ for which there exists a Lebesgue-measurable set $A \seq \R$ such that $\lambda(A) > 0$ and for any $x \in A$ the limit
    \[
        \lim_{r \to 0} \frac{1}{\lambda(B_r(x))} \int_{B_r(x)} f(y) \dd{y} 
    \]
    exists but is different from $f(x)$.
\end{pbox}

\begin{pbox}[(b)]
    Prove that in fact for any $\eps > 0$ one can reach $\lambda(\R \setminus A) < \eps$ in the first part.
\end{pbox}

Let $\eps > 0$ be given.

Let $\{q_k\}_{k=1}^{\infty}$ be an enumeration of the rationals.
Consider the set
\[
    U = \bigcup_{k=1}^{\infty} B_{\eps/2^{k+1}}(q_k),
\]
which is an open neighborhood of $\Q$ with measure
\[
    \lambda(U) \leq \sum_{k=1}^{\infty} \frac{\eps}{2^k} = \eps.
\]
Define the function
\[
    f = \infty \cdot \chi_U.
\]
As $U$ is open, it is Lebesgue measurable, so $f$ is Lebesgue-measurable.
Moreover, $f$ is nonnegative, so in fact $f$ is Lebesgue-integrable.

Then the set $A = \R \setminus U$ is Lebesgue-measurable with
\[
    \lambda(A) = \lambda(\R) - \lambda(U) \geq \infty - \eps = \infty.
\]
For all $x \in A$, we have $f(x) = 0$.
However, for every $r > 0$ there is some rational $q_k \in B_r(x)$.
Then $B_r(x) \cap U$ is an open neighborhood of $q_k$ contained in $B_r(x)$.
In particular, we know $\lambda(B_r(x) \cap U) > 0$, so
\begin{align*}
    \int_{B_r(x)} f(y) \dd{y}
        &= \int_{B_r(x) \cap U} f(y) \dd{y} + \int_{B_r(x) \setminus U} f(y) \dd{y} \\
        &= \int_{B_r(x) \cap U} \infty \dd{y} + \int_{B_r(x) \setminus U} 0 \dd{y} \\
        &= \infty \lambda(B_r(x) \cap U) \\
        &= \infty.
\end{align*}
Hence, for all $x \in A$ we have
\[
    \lim_{r \to 0} \frac{1}{\lambda(B_r(x))} \int_{B_r(x)} f(y) \dd{y}
        = \infty
        \ne 0 
        = f(x).
\]







\newpage
\begin{pbox}[3]
    Let $\alpha \in (0, 1)$ and let $\lambda$ be the Lebesgue measure on $\R$.
    Construct a Borel set $E \seq [-1, 1]$ such that
    \[
        \lim_{r \to 0} \frac{\lambda(E \cap [-r, r])}{2r} = \alpha.
    \]
\end{pbox}

For each $k \in \N$,  will construct $I_k$ to be an interval contained in $[\frac{1}{k+1}, \frac{1}{k})$, such that
\[\textstyle
    \lambda(I_k)
        = \alpha \lambda\left[\frac{1}{k+1}, \frac{1}{k}\right)
        = \alpha \left(\frac{1}{k} - \frac{1}{k+1}\right)
        = \frac{\alpha}{k(k+1)}.
\]
Explicitly, we write
\[\textstyle
    I_k = \left[\frac{1}{k+1}, \frac{1 + \alpha/k}{k+1}\right] \seq \left[\frac{1}{k+1}, \frac{1}{k}\right),
\]
which has the desired measure.

Now define
\[
    E' = \bigcup_{k=1}^{\infty} I_k \seq (0, 1).
\]
For $r \in (0, 1)$, consider the intersection
\[
    E' \cap [0, r]
        = \bigcup_{k=1}^{\infty} I_k \cap [0, r]
        = (I_n \cap [\tfrac{1}{n+1}, r]) \cup \bigcup_{k=n+1}^{\infty} I_k,
\]
where $n \in \N$ is such that $r \in [\frac{1}{n+1}, \frac{1}{n})$.
We compute the measure of this set using the fact that the $I_k$'s are disjoint and Lebesgue-measurable:
\[
    \lambda(E' \cap [0, r])
        = \lambda(I_n \cap [\tfrac{1}{n+1}, r]) + \sum_{k=n+1}^{\infty} \frac{\alpha}{k(k+1)}
        = \lambda(I_n \cap [\tfrac{1}{n+1}, r]) + \frac{\alpha}{n+1}.
\]
We bound the measure
\[
    0
        \leq \lambda(I_n \cap [\tfrac{1}{n+1}, r])
        \leq \lambda(I_n)
        = \frac{\alpha}{n(n + 1)}.
\]
Then 
\[
    \frac{\alpha}{r(n + 1)}
        \leq \frac{\lambda(E' \cap [0, r])}{r}
        \leq \frac{\alpha}{r(n + 1)} \left(1 + \tfrac{1}{n}\right).
\]
Since $\frac{1}{n+1} \leq r < \frac{1}{n}$, we have
\[
    \frac{n}{n+1} < \frac{1}{r(n+1)} \leq \frac{n+1}{n+1} = 1.
\]
Note that $n \to \infty$ as $r \to 0$, so this gives us
\[
    \lim_{r \to 0} \frac{1}{r(n+1)} = 1,
\]
and therefore
\[
    \lim_{r \to 0} \frac{\lambda(E' \cap [0, r])}{r} = \alpha.
\]
Then the desired subset of $[-1, 1]$ is given by
\[
    E = \pm E' = \{\pm x : x \in E\}.
\]


\newpage
\begin{pbox}[4]
    Let the function $f : \R \to \R$ be increasing and right-continuous.
    Define the function $\nu : 2^\R \to [0, \infty]$ as follows:
    \begin{enumerate}
        \item For any open set $U = \bigcup_{i=1}^{\infty} (a_i, b_i)$ where $(a_i, b_i)$ are disjoint, set
        \[
            \nu(U) = \sum_{i=1}^{\infty} (f(b_i) - f(a_i)).
        \]
        If $U$ is a finite union of disjoint open intervals, then $\nu$ is defined as a finite sum accordingly.
        \item For any $A \seq \R$ define
        \[
            \nu(A) = \inf\{\nu(U) : A \seq U, U \text{ open}\}.
        \]
        Prove that $\nu$ is a Borel measure on $\R$.
    \end{enumerate}
\end{pbox}

\end{document}