\documentclass[12pt]{article}

% Packages
\usepackage[margin=1in]{geometry}
\usepackage{fancyhdr, parskip}
\usepackage{amsmath, amsthm, amssymb}
\usepackage{tikz, tikz-cd}


% Page Style
\makeatletter
\fancypagestyle{title}{
    \renewcommand{\headrulewidth}{0.4pt}
    \setlength{\headheight}{15pt}
    \fancyhead[R]{\@author}
    \fancyhead[L]{\@title}
    \fancyhead[C]{\@date}
}
\makeatother
\renewcommand{\maketitle}{\thispagestyle{title}}
\fancypagestyle{plain}{
    \fancyhf{}
    \renewcommand{\headrulewidth}{0pt}
    \renewcommand{\footrulewidth}{0pt}
    \fancyfoot[R]{\thepage}
}
\pagestyle{plain}

% Problem Box
\setlength{\fboxsep}{4pt}
\newlength{\myparskip}
\setlength{\myparskip}{\parskip}
\newsavebox{\savefullbox}
\newenvironment{fullbox}{\begin{lrbox}{\savefullbox}\begin{minipage}{\dimexpr\textwidth-2\fboxsep\relax}\setlength{\parskip}{\myparskip}}{\end{minipage}\end{lrbox}\framebox[\textwidth]{\usebox{\savefullbox}}}
\newenvironment{pbox}[1][]{\begin{fullbox}\ifx#1\empty\else\paragraph{#1}\phantom{}\fi}{\end{fullbox}}

% Theorem Environments
\theoremstyle{definition}
\newtheorem{lemma}{Lemma}

% Tikz Environments
\newenvironment{drawing}{\begin{center}\begin{tikzpicture}}{\end{tikzpicture}\end{center}}
% \tikzcdset{row sep/normal=0pt}
\newenvironment{cd}{\begin{center}\begin{tikzcd}}{\end{tikzcd}\end{center}}

% Default Commands
\newcommand{\isp}[1]{\quad\text{#1}\quad}
\newcommand{\N}{\mathbb{N}} 
\newcommand{\Z}{\mathbb{Z}}
\newcommand{\Q}{\mathbb{Q}}
\newcommand{\R}{\mathbb{R}}
\newcommand{\C}{\mathbb{C}}
\newcommand{\A}{\mathbb{A}}
\renewcommand{\P}{\mathbb{P}}
\newcommand{\eps}{\varepsilon}
\renewcommand{\phi}{\varphi}
\renewcommand{\emptyset}{\varnothing}
\newcommand{\<}{\langle}
\renewcommand{\>}{\rangle}
\newcommand{\isom}{\cong}
\newcommand{\eqc}{\overline}
\newcommand{\clo}{\overline}
\newcommand{\seq}{\subseteq}
\newcommand{\teq}{\trianglelefteq}
\DeclareMathOperator{\id}{id}
\DeclareMathOperator{\im}{im}
\newcommand{\inc}{\hookrightarrow}

% Extra Commands


% Document
\begin{document}
\title{MATH 201B Homework 4}
\author{Harry Coleman}
\date{March 12, 2022}
\maketitle

\begin{pbox}[1]
    Let $\{B_n\}$ be a nested sequence of closed balls in a complete normed space $X$, where
    \[
        B_n = \bar{B}_{r_n}(x_n), \isp{with} r_n \geq r > 0 \isp{for all} n \in \N.
    \]
\end{pbox}

\begin{pbox}[(a)]
    Is it true that $\bigcap_{n=1}^{\infty} B_n \ne \emptyset$?
\end{pbox}

Yes.

\begin{proof}    
    Define the radius $R = \inf r_n \geq r > 0$.
    Since the balls are nested, $\{r_n\}$ is a decreasing sequence converging to $R$.

    We will prove that the sequence of center points $\{x_n\}$ is Cauchy.

    Let $\eps > 0$ be given and choose $N \in \N$ such that $r_N < R + \eps$.
    Suppose $m \geq n \geq N$.
    If $x_n = x_m$, then trivially $\|x_n - x_m\| = 0 < \eps$.
    Otherwise, if $x_n \ne x_m$, consider the ray in $X$ starting from $x_n$ in the direction of $x_m$.
    Let $y \in X$ be the point where this ray meets the boundary of $B_m$:
    \[
        y = x_n + \frac{r_m}{\|x_m - x_n\|}(x_m - x_n).
    \]
    Then $x_n, x_m, y$ are all collinear and contained in $B_n$.
    \begin{drawing}
        \draw (0, 0) circle (2);
        \fill (0, 0) circle (2pt) node[below right] {$x_n$};
        \draw (0, 0) -- (-2, 0) node[midway, above] {$r_n$};

        \draw (30 : 0.75) circle (1.1);
        \fill (30 : 0.75) circle (2pt) node[below right] {$x_m$};
        \draw (30 : 0.75) -- +(-1.1, 0) node[midway, above] {$r_m$};

        \draw[->, thick] (0, 0) -- (30 : 3);
        \fill (30 : 1.85) circle (2pt) node[below, xshift=-2pt, yshift=-3pt] {$y$};
    \end{drawing}
    It follows that
    \[
        \|x_n - x_m\| = \|x_n - y\| - \|y - x_m\| = \|x_n - y\| - r_m \leq r_n - r_m < \eps.
    \]
    As $\{x_n\}$ is a Cauchy sequence in the complete space $X$, we know that it converges to some point $x \in X$.
    For each $n \in \N$, all the points $x_k$ for $k \geq n$ are contained in the closed ball $B_n$.
    In particular, $B_n$ is sequentially closed so $x_k \to x$ implies $x \in B_n$.
    Hence, $x \in \bigcap_{n=1}^{\infty} B_n$.
\end{proof}


\newpage
\begin{pbox}[(b)]
    Is it true $B \seq \bigcap_{n=1}^{\infty} B_n$ for some closed ball $B$ with radius $r$?
\end{pbox}

Yes.

\begin{proof}
    Let $y \in \bar{B}_r(x)$, where $x = \lim_{n \to \infty} x_n \in X$ as in the previous proof.
    For $n \in \N$ define the point $y_n = x_n + (y - x)$.
    Then $x_n \to x$ implies $y_n \to y$, since
    \[
        \|y_n - y\| = \|x_n + (y - x) - y\| = \|x_n - x\| \xrightarrow{n \to \infty} 0.
    \]
    On the other hand,
    \[
        \|y_n - x_n\| = \|y - x\| \leq r \leq r_n,
    \]
    which means $y_n \in B_n$, so $y \in B_k$ for all $k \leq n$.
    In other words, $y_k \in B_n$ for all $k \geq n$.
    Therefore, $y_k \to y$ in the closed ball $B_n$, so in fact $y \in B_n$.
    Hence, $\bar{B}_r(x) \seq \bigcap_{n=1}^{\infty} B_n$.
\end{proof}

\newpage
\begin{pbox}[2]
    Construct a Lebesgue-measurable set $A \seq [0, 1]$ such that $m(A) = 1$ and $A$ is of Baire first category in $[0, 1]$.
\end{pbox}  

For any $0 < \alpha \leq 1$, we can construct a ``fat Cantor set'' by removing the middle $\alpha/3$ of each interval (instead of the usual middle $1/3$).
The resulting set $C_\alpha \seq [0, 1]$ is homeomorphic to the usual cantor set by appropriate stretching and shrinking of the interval.
In particular, the set is nowhere dense and has measure
\[
    m(C_\alpha)
        = 1 - \sum_{n=0}^{\infty} 2^n \left(\frac{\alpha}{3}\right)^{n+1}
        = 1 - \frac{\alpha}{3} \sum_{n=0}^{\infty} \left(\frac{2\alpha}{3}\right)^{n}
        = 1 - \frac{\alpha}{3} \cdot \frac{1}{1 - \frac{2\alpha}{3}}
        = \frac{3 - 3\alpha}{3 - 2\alpha}.
\]
We now define the Baire first category set $A = \bigcup_{k=1}^{\infty} C_{1/k}$.
Since $[0, 1] \supseteq A \supseteq C_{1/k}$, we have
\[
    1 \geq m(A) \geq m(C_{1/k}) = \frac{3k - 3}{3k - 2}.
\]
Taking the limit as $k \to \infty$, we obtain $m(A) = 1$.



\newpage
\begin{pbox}[3]
    Let $f : (0, 1) \to \R$ be continuous.
    Prove that if $\lim_{n \to \infty} f(\frac{x}{n}) = 0$ for all $x \in (0, 1)$, then $\lim_{x \to 0} f(x) = 0$.
\end{pbox}


\newpage
\begin{pbox}[4]
    Let $f \in C^\infty(\R, \R)$ be such that for every $x \in \R$, there exists $N_x \in \N$ such that $f^{(n)}(x) = 0$ for all $n \geq N_x$.
    Prove that $f$ is a polynomial on $\R$.
\end{pbox}


\end{document}