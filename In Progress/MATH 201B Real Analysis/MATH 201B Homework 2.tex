\documentclass[12pt]{article}

% Packages
\usepackage[margin=1in]{geometry}
\usepackage{fancyhdr, parskip}
\usepackage{amsmath, amsthm, amssymb}
\usepackage{tikz, tikz-cd}

% Page Style
\makeatletter
\fancypagestyle{title}{
    \renewcommand{\headrulewidth}{0.4pt}
    \setlength{\headheight}{15pt}
    \fancyhead[R]{\@author}
    \fancyhead[L]{\@title}
    \fancyhead[C]{\@date}
}
\makeatother
\renewcommand{\maketitle}{\thispagestyle{title}}
\fancypagestyle{plain}{
    \fancyhf{}
    \renewcommand{\headrulewidth}{0pt}
    \renewcommand{\footrulewidth}{0pt}
    \fancyfoot[R]{\thepage}
}
\pagestyle{plain}

% Problem Box
\setlength{\fboxsep}{4pt}
\newlength{\myparskip}
\setlength{\myparskip}{\parskip}
\newsavebox{\savefullbox}
\newenvironment{fullbox}{\begin{lrbox}{\savefullbox}\begin{minipage}{\dimexpr\textwidth-2\fboxsep\relax}\setlength{\parskip}{\myparskip}}{\end{minipage}\end{lrbox}\framebox[\textwidth]{\usebox{\savefullbox}}}
\newenvironment{pbox}[1][]{\begin{fullbox}\ifx#1\empty\else\paragraph{#1}\phantom{}\fi}{\end{fullbox}}

% Theorem Environments
\theoremstyle{definition}
\newtheorem{lemma}{Lemma}

% Tikz Environments
\newenvironment{drawing}{\begin{center}\begin{tikzpicture}}{\end{tikzpicture}\end{center}}
% \tikzcdset{row sep/normal=0pt}
\newenvironment{cd}{\begin{center}\begin{tikzcd}}{\end{tikzcd}\end{center}}

% Default Commands
\newcommand{\isp}[1]{\quad\text{#1}\quad}
\newcommand{\N}{\mathbb{N}} 
\newcommand{\Z}{\mathbb{Z}}
\newcommand{\Q}{\mathbb{Q}}
\newcommand{\R}{\mathbb{R}}
\newcommand{\C}{\mathbb{C}}
\newcommand{\A}{\mathbb{A}}
\renewcommand{\P}{\mathbb{P}}
\newcommand{\eps}{\varepsilon}
\renewcommand{\phi}{\varphi}
\renewcommand{\emptyset}{\varnothing}
\newcommand{\<}{\langle}
\renewcommand{\>}{\rangle}
\newcommand{\isom}{\cong}
\newcommand{\eqc}{\overline}
\newcommand{\clo}{\overline}
\newcommand{\seq}{\subseteq}
\newcommand{\teq}{\trianglelefteq}
\DeclareMathOperator{\id}{id}
\DeclareMathOperator{\im}{im}

% Extra Commands
\newcommand{\dd}{\,\mathrm{d}}


% Document
\begin{document}
\title{MATH 201B Homework 2}
\author{Harry Coleman}
\date{February 14, 2022}
\maketitle

\begin{pbox}[1]
    For a function $f : [a, b] \to \R$ define for every $x \in [a, b)$
    \[
        D^+ f(x) = \limsup_{h \to 0^+} \frac{f(x + h) - f(x)}{h}.
    \]
    Prove that if $f : [a, b] \to \R$ is continuous and $D^+ f(x) \geq 0$ for all $x \in [a, b)$, then $f(b) \geq f(a)$.
\end{pbox}

\begin{proof}
    First, suppose $D^+f(x) > 0$ for all $x \in [a, b)$.
    Since $f$ is continuous on the compact set $[a, b]$, it attains its supremum $M = \sup_{[a, b]} f$ at some point.
    Now define $c = \sup f^{-1}(M)$, which is on the interval $[a, b]$.
    Moreover, we can choose a sequence of points in $[a, b]$ approaching $c$ from the left, and on which $f$ equals $M$.
    With $f$ continuous, it follows that $f(c) = M$.

    If $c = b$, then $f(b) = M \geq f(a)$ and we are done.
    Assume for contradiction that $c < b$.
    With $D^+f(c) > 0$, there must be some $h > 0$ such that
    \[
        \frac{f(c + h) - f(c)}{h} > 0.
    \]
    But this means $f(c) < f(c + h)$, which contradicts the maximality of $f(c) = M$.

    Now suppose $D^+f(x) \geq 0$ for all $x \in [a, b)$ and assume for contradiction that $f(b) < f(a)$.
    Then there is some $\eps > 0$ such that $f(b) + (b - a)\eps < f(a)$.
    Define the function
    \[
        g(x) = f(x) + (x - a)\eps,
    \]
    which has
    \[
        g(b) = f(b) + (b - a)\eps < f(a) = g(a).
    \]
    However,
    \[
        \frac{g(x + h) - g(x)}{h}
            = \frac{f(x + h) - f(x)}{h} + \eps
            \geq 0 + \eps
            = \eps,
    \]
    so letting $h \to 0^+$, we obtain $D^+g(x) \geq \eps > 0$ for all $x \in [a, b)$.
    This contradicts the first result, which implies $g(b) \geq g(a)$.
\end{proof}

\newpage
\begin{pbox}[2]
    Suppose $f_n : [0, 1] \to [0, \infty)$ is a sequence of increasing and right-continuous function.
    Let
    \[
        f(x) = \sum_{n=1}^{\infty} f_n(x), \quad x \in [0, 1],
    \]
    and assume that $f(1)$ is finite.
    Prove that
    \[
        f'(x) = \sum_{n=1}^{\infty} f_n'(x)
    \]
    for almost every $x \in [0, 1]$ (in the sense of the Lebesgue measure).
\end{pbox}

\begin{proof}
    For $x, y \in [0, 1]$ with $x \leq y$, we have $f_n(x) \leq f_n(y)$ for all $n \in \mathbb{N}$.
    This gives us
    \[
        f(x) = \sum_{n=1}^{\infty} f_n(x) \leq \sum_{n=1}^{\infty} f_n(y) = f(y),
    \]
    which means $f$ is increasing, and therefore differentiable $\lambda$-a.e.\ in $[0, 1]$.

    For $k \in \N$, denote the partial sum $s_k = \sum_{n=1}^{k} f_n$ and remainder $r_k = \sum_{n=k+1}^{\infty} f_n$, which are both increasing and, therefore, differentiable $\lambda$-a.e.\ in $[0, 1]$.
    In particular, if $x \in [0, 1]$ is a point where the derivatives exist, we have
    \[
        f'(x) = s_k'(x) + r_k'(x) = \sum_{n=1}^{k} f_n'(x) + r_k'(x).
    \]
    Moreover, the derivatives of increasing functions are nonnegative, so
    \[
        f'(x) \geq s_k'(x) = \sum_{n=1}^{k} f_n'(x).
    \]
    Letting $k \to \infty$, we obtain
    \[
        f'(x) \geq \sum_{n=1}^{\infty} f_n'(x) \qquad \text{for $\lambda$-a.e.\ $x \in [0, 1]$.}
    \]
    We estimate
    \[
        \int_{0}^{1} f' \dd{x}
            = \int_{0}^{1} s_k' \dd{x} + \int_{0}^{1} r_k' \dd{x}
            \leq \int_{0}^{1} \sum_{n=1}^{\infty} f_n'  \dd{x} + r_k(1) - r_k(0).
    \]
    Since $\sum_{n=1}^{\infty} f_n'(x) \leq f'(x)$ for $\lambda$-a.e.\ $x \in [0, 1]$, we have
    \[
        \int_{0}^{1} \sum_{n=1}^{\infty} f_n' \dd{x} \leq \int_{0}^{1} f' \dd{x}.
    \]
    These integrals are finite, so
    \[
        0 
            \leq \int_{0}^{1} f' \dd{x} - \int_{0}^{1} \sum_{n=1}^{\infty} f_n' \dd{x}
            = \int_{0}^{1} \left(f' - \sum_{n=1}^{\infty} f_n' \right) \dd{x}
            \leq r_k(1) - r_k(0).
    \]
    On one hand, we have
    \[
        f(1) - f(0) = \lim_{k \to \infty} (s_k(1) - s_k(0)).
    \]
    On the other hand, $f(0) \leq f(1) < \infty$, so we can write
    \[
        r_k(1) - r_k(0) = (f(1) - f(0)) - (s_k(1) - s_k(0)).
    \]
    Taking the limit as $k \to \infty$, we obtain
    \[
        \lim_{k \to \infty} (r_k(1) - s_k(0)) = 0.
    \]
    Letting $k \to \infty$ in the above inequality, we deduce
    \[
        \int_{0}^{1} \left(f' - \sum_{n=1}^{\infty} f_n' \right) \dd{x} = 0.
    \]
    Since the integrand is nonnegative, we in fact must have
    \[
        f'(x) = \sum_{n=1}^{\infty} f_n'(x) \qquad \text{for $\lambda$-a.e.\ $x \in [0, 1]$.}
    \]
\end{proof}

\newpage
\begin{pbox}[3]
    Find an increasing function $f : [0, 1] \to \R$ such that $f'(x) = 0$ a.e.\ in $[0, 1]$, but $f$ is not constant on any open subinterval of $[0, 1]$.
\end{pbox}

For $n \in \N$, define a piecewise constant function $f_n : [0, 1] \to \R$ by
\[
    f_n(x) = \frac{\lfloor 2^nx \rfloor}{2^{2n}}.
\]
On each interval $[\frac{m}{2^n}, \frac{m+1}{2^n})$, this map is constantly $\frac{m}{2^{2n}}$.
In particular, each $f_n$ is increasing and right-continuous.
For $x \in [0, 1]$ define
\[
    f(x) = \sum_{n=1}^{\infty} f_n(x).
\]
Then $f$ is an increasing function and we we will check that it is finite:
\[
    f(1)
        = \sum_{n=1}^{\infty} f_n(1)
        = \sum_{n=1}^{\infty} \frac{\lfloor 2^n \rfloor}{2^{2n}}
        = \sum_{n=1}^{\infty} \frac{1}{2^n}
        = 1.
\]
Then applying Problem 2, we find
\[
    f'(x)
        = \sum_{n=1}^{\infty} f_n'(x)
        = \sum_{n=1}^{\infty} 0
        = 0
\]
for $\lambda$-a.e.\ $x \in [0, 1]$.
Moreover, $f$ is not constant on any open subinterval of $[0, 1]$, since $f_n$ ensures that points separated by at least a distance of $1/2^n$ will have distinct values (and $1/2^n$ is eventually small enough to affect any given open subinterval).



\end{document}