\documentclass[12pt]{article}

% Packages
\usepackage[margin=1in]{geometry}
\usepackage{fancyhdr, parskip}
\usepackage{amsmath, amsthm, amssymb}
\usepackage{tikz, tikz-cd}

\usepackage{suffix}

% Page Style
\makeatletter
\fancypagestyle{title}{
    \renewcommand{\headrulewidth}{0.4pt}
    \setlength{\headheight}{15pt}
    \fancyhead[R]{\@author}
    \fancyhead[L]{\@title}
    \fancyhead[C]{\@date}
}
\makeatother
\renewcommand{\maketitle}{\thispagestyle{title}}
\fancypagestyle{plain}{
    \fancyhf{}
    \renewcommand{\headrulewidth}{0pt}
    \renewcommand{\footrulewidth}{0pt}
    \fancyfoot[R]{\thepage}
}
\pagestyle{plain}

% Problem Box
\setlength{\fboxsep}{4pt}
\newlength{\myparskip}
\setlength{\myparskip}{\parskip}
\newsavebox{\savefullbox}
\newenvironment{fullbox}{\begin{lrbox}{\savefullbox}\begin{minipage}{\dimexpr\textwidth-2\fboxsep\relax}\setlength{\parskip}{\myparskip}}{\end{minipage}\end{lrbox}\framebox[\textwidth]{\usebox{\savefullbox}}}
\newenvironment{pbox}[1][]{\begin{fullbox}\def\temp{#1}\ifx\temp\empty\else\paragraph{#1}\phantom{}\fi}{\end{fullbox}}

% Theorem Environments
\theoremstyle{definition}
\newtheorem{lemma}{Lemma}

% Tikz Environments
\newenvironment{drawing}{\begin{center}\begin{tikzpicture}}{\end{tikzpicture}\end{center}}
% \tikzcdset{row sep/normal=0pt}
\newenvironment{cd}{\begin{center}\begin{tikzcd}}{\end{tikzcd}\end{center}}

% Default Commands
\newcommand{\isp}[1]{\quad\text{#1}\quad}
\newcommand{\N}{\mathbb{N}} 
\newcommand{\Z}{\mathbb{Z}}
\newcommand{\Q}{\mathbb{Q}}
\newcommand{\R}{\mathbb{R}}
\newcommand{\C}{\mathbb{C}}
\newcommand{\A}{\mathbb{A}}
\renewcommand{\P}{\mathbb{P}}
\newcommand{\eps}{\varepsilon}
\renewcommand{\phi}{\varphi}
\renewcommand{\emptyset}{\varnothing}
\newcommand{\<}{\langle}
\renewcommand{\>}{\rangle}
\newcommand{\iso}{\cong}
\newcommand{\eqc}{\overline}
\newcommand{\clo}{\overline}
\newcommand{\seq}{\subseteq}
\newcommand{\teq}{\trianglelefteq}
\DeclareMathOperator{\id}{id}
\DeclareMathOperator{\im}{im}
\newcommand{\inc}{\hookrightarrow}
\newcommand{\dd}{\mathrm{d}}
\newcommand{\mat}[1]{\begin{bmatrix}#1\end{bmatrix}}

% Extra Commands

\renewcommand{\mod}{\text{-}\mathsf{mod}}
\newcommand{\Rep}{\mathrm{Rep}}

\newcommand{\GL}{\mathrm{GL}}

\newcommand{\udl}{\underline}
\DeclareMathOperator{\Dim}{\udl{dim}}

\renewcommand{\dd}{\mathbf{d}}



% Document
\begin{document}
\title{MATH 237C Note}
\author{Harry Coleman}
\date{2022}
\maketitle

\begin{lemma}
    Let $M \in \Lambda\mod$ with $\Dim M = \dd$, and let $x \in \Rep_\dd\Lambda$ such that $M$ corresponds to the $G$-orbit $G.x$.
    If $U \seq M$ is a submodule, then the direct sum $U \oplus M/U$ corresponds to a $G$-orbit contained in $\clo{G.x}$.
\end{lemma}

\begin{proof}
    Let $\dd' = \Dim U$.
    Then $\dd'' = \dd - \dd' = \Dim M/U$, since $e_i(M / U) \iso e_i M / e_i U$ canonically.

    For each $i \leq n$, choose an ordered $K$-basis for $e_i U$ and supplement it to an ordered basis for $e_i M$.

    Suppose $x = (x_\alpha)_{\alpha \in Q_1}$.
    Since $\alpha U \seq U$ by hypothesis, the $x_\alpha$ have the following block format.
    If $\alpha : e_i \to e_j$ then
    \[
        x_\alpha = \mat{A_\alpha & C_\alpha \\ 0 & B_\alpha}
    \]
    where $A_\alpha$ is a $d_i' \times d_i'$ matrix and $B_\alpha$ is a $d_i'' \times d_i''$ matrix.

    For each $c \in K$, define an element $g(c) \in \prod_{1 \leq i \leq n} M_{d_i}(K)$ as follows:
    \[
        g(c) = (g(c)_1, \dots, g(c)_n)
    \]
    where
    \[
        g(c)_i = \mat{cI_{d_i'} & 0 \\ 0 & I_{d_i''}}
    \]
    where $I_m$ is the $m \times m$ identity matrix.

    Clearly, $g(c) \in G = \prod_{1 \leq i \leq n} \GL_{d_i}(K)$ whenever $c \in K^\times$.

    Now consider the morphism of varieties
    \begin{align*}
        \psi : K &\longrightarrow \Rep_\dd(\Lambda), \\
            c &\longmapsto \left(\mat{A_\alpha & cC_\alpha \\ 0 & B_\alpha}\right)_{\alpha \in Q_1}.
    \end{align*}
    Observe: for $c \in K^\times$, we have
    \[
        \psi(c)
            = \left(g(c)_{\mathrm{end}(\alpha)} x_\alpha g(c)_{\mathrm{start}(\alpha)}\right)_{\alpha \in Q_1}
            = g(c).x
            \in G.x
    \]

    Since $\psi$ is Zariski-continuous, $\psi^{-1}(\clo{G.x})$ is closed in $K$, and thus $\psi^{-1}(\clo{G.x}) = K$.

    In other words,
    \[
        \psi(0)
            = \left(\mat{A_\alpha & 0 \\ 0 & B_\alpha}\right)_{\alpha \in Q_1} 
            \in \clo{G.x},
    \]
    but clearly the orbit of $\psi(0)$ in $\clo{G.x}$ represents the direct sum $U \oplus M/U$.

\end{proof}

\end{document}