\documentclass[12pt]{article}

% Packages
\usepackage[margin=1in]{geometry}
\usepackage{fancyhdr, parskip}
\usepackage{amsmath, amsthm, amssymb}
\usepackage{tikz, tikz-cd}
\usepackage[shortlabels]{enumitem}

% Page Style
\makeatletter
\fancypagestyle{title}{
    \renewcommand{\headrulewidth}{0.4pt}
    \setlength{\headheight}{15pt}
    \fancyhead[R]{\@author}
    \fancyhead[L]{\@title}
    \fancyhead[C]{\@date}
}
\makeatother
\renewcommand{\maketitle}{\thispagestyle{title}}
\fancypagestyle{plain}{
    \fancyhf{}
    \renewcommand{\headrulewidth}{0pt}
    \renewcommand{\footrulewidth}{0pt}
    \fancyfoot[R]{\thepage}
}
\pagestyle{plain}

% Problem Box
\setlength{\fboxsep}{4pt}
\newlength{\myparskip}
\setlength{\myparskip}{\parskip}
\newsavebox{\savefullbox}
\newenvironment{fullbox}{\begin{lrbox}{\savefullbox}\begin{minipage}{\dimexpr\textwidth-2\fboxsep\relax}\setlength{\parskip}{\myparskip}}{\end{minipage}\end{lrbox}\framebox[\textwidth]{\usebox{\savefullbox}}}
\newenvironment{pbox}[1][]{\begin{fullbox}\def\temp{#1}\ifx\temp\empty\else\paragraph{#1}\phantom{}\fi}{\end{fullbox}}

% Theorem Environments
\theoremstyle{definition}
\newtheorem{lemma}{Lemma}

% Tikz Environments
\newenvironment{drawing}{\begin{center}\begin{tikzpicture}}{\end{tikzpicture}\end{center}}
% \tikzcdset{row sep/normal=0pt}
\newenvironment{cd}{\begin{center}\begin{tikzcd}}{\end{tikzcd}\end{center}}

% Default Commands
\newcommand{\isp}[1]{\quad\text{#1}\quad}
\newcommand{\N}{\mathbb{N}} 
\newcommand{\Z}{\mathbb{Z}}
\newcommand{\Q}{\mathbb{Q}}
\newcommand{\R}{\mathbb{R}}
\newcommand{\C}{\mathbb{C}}
\newcommand{\A}{\mathbb{A}}
\renewcommand{\P}{\mathbb{P}}
\newcommand{\eps}{\varepsilon}
\renewcommand{\phi}{\varphi}
\renewcommand{\emptyset}{\varnothing}
\newcommand{\<}{\langle}
\renewcommand{\>}{\rangle}
\newcommand{\iso}{\cong}
\newcommand{\eqc}{\overline}
\newcommand{\clo}{\overline}
\renewcommand{\tilde}{\widetilde}
\renewcommand{\hat}{\widehat}
\newcommand{\seq}{\subseteq}
\newcommand{\teq}{\trianglelefteq}
\DeclareMathOperator{\id}{id}
\DeclareMathOperator{\im}{im}
\newcommand{\inc}{\hookrightarrow}
\newcommand{\dd}{\mathrm{d}}
\newcommand{\mat}[1]{\begin{bmatrix}#1\end{bmatrix}}

% Extra Commands
\renewcommand{\AA}{\mathcal{A}}

% Document
\begin{document}
\title{MATH 236B Homework 1}
\author{Harry Coleman}
\date{April 7, 2023}
\maketitle

\begin{pbox}
    By an \textbf{element} $y$ of an object $Y$ (notation $y \in Y$) of an abelian category $\AA$ we mean an equivalence class of pairs $(X, h)$, $X \in \AA$, $h : X \to Y$, by the following equivalence relation: $(X, h) \sim (X', h')$ if and only if there exists $Z \in \AA$ and epimorphisms $u : Z \to X$ and $u' : Z \to X'$ such that $hu = h'u'$.
\end{pbox}

We will use the following properties from the previous exercise.

\begin{pbox}[(a)]
    $f : Y_1 \to Y_2$ is a monomorphism if and only if $f(y) = 0$ for $y \in Y_1$ implies $y = 0$.
\end{pbox}

\begin{pbox}[(b)]
    $f : Y_1 \to Y_2$ is a monomorphism if and only if $f(y) = f(y')$ for $y, y' \in Y_1$ implies $y = y'$.
\end{pbox}

\begin{pbox}[(c)]
    $f : Y_1 \to Y_2$ is an epimorphism if and only if for any $y \in Y_2$ there exists $y' \in Y_1$ such that $f(y') = y$.
\end{pbox}

\begin{pbox}[(d)]
    $f : Y_1 \to Y_2$ is the zero morphism if and only if $f(y) = 0$ for all $y \in Y_1$.
\end{pbox}

\begin{pbox}[(e)]
    A sequence \begin{tikzcd} Y_1 \rar["f"] & Y \rar["g"] & Y_2 \end{tikzcd} is exact at $Y$ if and only if $gf = 0$ and for any $y \in Y$ with $g(y) = 0$ there exists $y' \in Y_1$ with $f(y') = y$.
\end{pbox}

\begin{pbox}[(f)]
    Assume we are given a morphism $g : Y_1 \to Y_2$ and elements $y, y' \in Y_1$ such that $g(y) = g(y')$.
    Then there exists $z \in Y_1$ such that $g(z) = 0$ and, moreover, for any $f : Y_1 \to Y$ with $f(y) = 0$ we have $f(z) = -f(y')$ and for any $f' : Y_1 \to Y$ with $f'(y') = 0$ we have $f'(z) = f'(y)$. (The element $z$ is an analogue of the difference $y - y'$.)
\end{pbox}

\begin{pbox}[1 Exercise II.5.6 (Five Lemma)]
    Assume we are given a commutative diagram
    \begin{cd}
        X_1 \rar["a_1"] \dar["f_1"]
        & X_2 \rar["a_2"] \dar["f_2"]
        & X_3 \rar["a_3"] \dar["f_3"]
        & X_4 \rar["a_4"] \dar["f_4"]
        & X_5 \dar["f_5"]
        \\
        Y_1 \rar["b_1"']
        & Y_2 \rar["b_2"']
        & Y_3 \rar["b_3"']
        & Y_4 \rar["b_4"']
        & Y_5
    \end{cd}
    with exact rows.
    Assume also that $f_1$ is an epimorphism, $f_5$ is a monomorphism, and $f_2$ and $f_4$ are isomorphisms.
    Then $f_3$ is also an isomorphism.
\end{pbox}

\begin{proof}
    We prove that $f_3$ is an isomorphism by showing that it is both a monomorphism and and epimorphism.

    Suppose $x \in \ker f_3$, i.e., $x \in X_3$ such that $f_3(x) = 0$.
    Then commutativity of the third square gives us
    \[
        0 = b_3f_3(x) = f_4a_3(x).
    \]
    Since $f_4$ is an monomorphism and $f_4(0) = 0$, we deduce that $a_3(x) = 0$.
    By exactness at $X_3$, there exists $x' \in X_2$ such that $a_2(x') = x$.
    Then commutativity of the second square gives us
    \[
        0 = f_3(x) = f_3a_2(x') = b_2f_2(x').
    \]
    So, by exactness at $Y_2$, there exists $y \in Y_1$ such that $b_1(y) = f_2(x')$.
    Since $f_1$ is an epimorphism, there exists $x'' \in X_1$ such that $f_1(x'') = y$.
    Then commutativity of the first square gives us
    \[
        f_2a_1(x'') = b_1f_1(x'') = b_1(y) = f_2(x').
    \]
    And since $f_2$ is a monomorphism, we deduce that $a_1(x'') = x'$.
    But exactness at $X_2$ gives us $a_2a_1 = 0$, so in fact
    \[
        x = a_2(x') = a_2a_1(x'') = 0.
    \]
    This proves that $f_2$ is a monomorphism.

    Suppose $y \in Y_3$.
    Exactness at $Y_4$ gives us $b_4b_3(y) = 0$.
    Since $f_4$ is an epimorphism, there exists $x \in X_4$ such that $f_4(x) = b_3(y)$.
    By commutativity of the fourth square, we have
    \[
        0 = b_4b_3(y) = b_4f_4(x) = f_5a_4(x).
    \]
    This $f_5$ is a monomorphism and $f_5(0) = 0$, we deduce that $a_4(x) = 0$.
    By exactness at $X_4$, there exists $x' \in X_3$ such that $a_3(x') = x$.
    Then commutativity of the third square gives us
    \[
        b_3f_3(x') = f_4a_3(x') = f_4(x) = b_3(y).
    \]
    In other words, we do not know if $f_3(x')$ and $y$ are equal, but they are sent to the same element by $b_3$.
    Let $z \in Y_3$ be as in (f), i.e., $z$ is an analogue of the difference $f_3(x') - y$.
    In particular, $b_3(z) = 0$, so exactness at $Y_3$ tells us there exists $y' \in Y_2$ such that $b_2(y') = z$.
    Since $f_2$ is an epimorphism, there exists $x'' \in X_2$ such that $f_2(x'') = y'$.
    Commutativity of the second square gives us
    \[
        z = b_2(y') = b_2f_2(x'') = f_3a_2(x'').
    \]
    We now apply (f) to obtain $\tilde{x} \in X_3$ as an analogue of the difference $x' - a_2(x'')$.

    Then we should get something like the following:
    \[
        f_3(\tilde{x})
            = f_3(x') - f_3a_2(x'')
            = f_3(x') - z
            \approx f_3(x') - (f_3(x') - y)
            = y.
    \]
    However, I am not sure how to make this rigorous given the current setup with (f) handling the differences.
    Assuming this works out, we conclude that $f_3$ is an epimorphism.
\end{proof}

\end{document}