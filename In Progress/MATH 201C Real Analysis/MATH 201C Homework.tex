\documentclass[12pt]{article}

% Packages
\usepackage[margin=1in]{geometry}
\usepackage{fancyhdr, parskip}
\usepackage{amsmath, amsthm, amssymb}
\usepackage{tikz, tikz-cd}

% Page Style
\makeatletter
\fancypagestyle{title}{
    \renewcommand{\headrulewidth}{0.4pt}
    \setlength{\headheight}{15pt}
    \fancyhead[R]{\@author}
    \fancyhead[L]{\@title}
    \fancyhead[C]{\@date}
}
\makeatother
\renewcommand{\maketitle}{\thispagestyle{title}}
\fancypagestyle{plain}{
    \fancyhf{}
    \renewcommand{\headrulewidth}{0pt}
    \renewcommand{\footrulewidth}{0pt}
    \fancyfoot[R]{\thepage}
}
\pagestyle{plain}

% Problem Box
\setlength{\fboxsep}{4pt}
\newlength{\myparskip}
\setlength{\myparskip}{\parskip}
\newsavebox{\savefullbox}
\newenvironment{fullbox}{\begin{lrbox}{\savefullbox}\begin{minipage}{\dimexpr\textwidth-2\fboxsep\relax}\setlength{\parskip}{\myparskip}}{\end{minipage}\end{lrbox}\framebox[\textwidth]{\usebox{\savefullbox}}}
\newenvironment{pbox}[1][]{\begin{fullbox}\ifx#1\empty\else\paragraph{#1}\phantom{}\fi}{\end{fullbox}}

% Theorem Environments
\theoremstyle{definition}
\newtheorem{lemma}{Lemma}

% Tikz Environments
\newenvironment{drawing}{\begin{center}\begin{tikzpicture}}{\end{tikzpicture}\end{center}}
% \tikzcdset{row sep/normal=0pt}
\newenvironment{cd}{\begin{center}\begin{tikzcd}}{\end{tikzcd}\end{center}}

% Default Commands
\newcommand{\isp}[1]{\quad\text{#1}\quad}
\newcommand{\N}{\mathbb{N}} 
\newcommand{\Z}{\mathbb{Z}}
\newcommand{\Q}{\mathbb{Q}}
\newcommand{\R}{\mathbb{R}}
\newcommand{\C}{\mathbb{C}}
\newcommand{\A}{\mathbb{A}}
\renewcommand{\P}{\mathbb{P}}
\newcommand{\eps}{\varepsilon}
\renewcommand{\phi}{\varphi}
\renewcommand{\emptyset}{\varnothing}
\newcommand{\<}{\langle}
\renewcommand{\>}{\rangle}
\newcommand{\iso}{\cong}
\newcommand{\eqc}{\overline}
\newcommand{\clo}{\overline}
\newcommand{\seq}{\subseteq}
\newcommand{\teq}{\trianglelefteq}
\DeclareMathOperator{\id}{id}
\DeclareMathOperator{\im}{im}
\newcommand{\inc}{\hookrightarrow}

% Extra Commands
\newcommand{\dd}{\mathrm{d}}
\newcommand{\esssup}{\operatorname{ess\,sup}}

% Document
\begin{document}
\title{MATH 201C Homework 1}
\author{Harry Coleman}
\date{April 12, 2022}
\maketitle


\begin{pbox}[1]
    Let $\{x_n\}$ be a Cauchy sequence of a metric space $(X, d)$.
    Suppose that for some subsequence $\{x_{n_k}\}$, $x_{n_k}$ converges to $x_0$ as $k \to \infty$, show that $x_n$ converges to $x_0$ too.
\end{pbox}

\begin{proof}
    Let $\eps > 0$ be given.
    Choose $N \in \N$ for the definitions of both the Cauchyness of $\{x_n\}$ and the convergence of the subsequence $x_{n_k} \to x_0$, with respect to $\eps/2$.
    For any $n \geq N$, choose any $n_k \geq N$, then
    \[
        \|x_n - x_0\|
            \leq \|x_n - x_{n_k}\| + \|x_{n_k} - x_0\|
            < \frac{\eps}{2} + \frac{\eps}{2}
            = \eps.
    \]
    Hence, we indeed have convergence of the whole sequence $x_n \to x_0$.
\end{proof}


\newpage
\begin{pbox}[2]
    Construct a sequence $f_n(x)$ of bounded functions on $[0, 1]$ converging to zero in $L^1$, but $f_n$ converges at no point in $[0, 1]$.
\end{pbox}

For each $k \in \N$ and $i = 1, \dots, k$, define $f_{k, i} = \chi_{[\frac{i-1}{k}, \frac{i}{k}]}$, i.e., the characteristic function on the interval $[\frac{i-1}{k}, \frac{i}{k}] \seq [0, 1]$.
Choose an indexing $k_n, i_n \in \N$ such that the sequence $\{f_{k_n, i_n}\}$ progresses as follows: $f_{k,1}, f_{k,2}, \dots, f_{k,k}, f_{k+1, 1}$.
In particular, $k_n \to \infty$ as $n \to \infty$.
It follows that the integral
\[
    \int_{0}^{1} |f_{k,i}(x)| \dd{x}
        = \int_{0}^{1} \chi_{[\frac{i-1}{k}, \frac{i}{k}]}(x) \dd{x}
        = \frac{1}{k}
\]
converges to zero as $n \to \infty$ (with $k = k_n$).
However, for each $x \in [0, 1]$ and $k \in \N$, there is some $i \in \{1, \dots, k\}$ such that $x \in [\frac{i-1}{k}, \frac{i}{k}]$, which gives $f_{k,i}(x) = 1$.
Since there are infinitely many such pairs of $k, i$ satisfying this, all of which appear in the sequence, the sequence does not converges pointwise at $x$.


\newpage
\begin{pbox}[3]
    Let $(M, \mu)$ be a measure space, show that $L^\infty(M, \mu)$ is a Banach space.
\end{pbox}

\begin{proof}
    We will take for granted that this is a normed vector space, as proving this primarily involves expanding definitions.
    It remains to prove that the space is complete.

    Recall that the set $L^\infty(M, \mu)$ is a quotient---we denote the equivalence class of $f$ by $[f]$.
    Then $[f] = [g]$ whenever $\|[f] - [g]\|_\infty = \esssup_M|f - g| = 0$.

    Suppose $\{[f_n]\}$ is a Cauchy sequence in $L^\infty(M, \mu)$.
    By Problem 1, it suffices to show any subsequence converges.
    For each $k \in \N$, choose $n_k \in \N$ for the definition of Cauchyness with respect to $\eps = 1/2^k$, and such that $n_k \geq n_{k-1}$.
    Replacing the original sequence with this subsequence, we have the property that
    \[
        n \leq m \implies \|[f_n - f_m]\|_\infty = \|[f_n] - [f_m]\|_\infty < \frac{1}{2^n}.
    \]
    For $n \leq m$, choose a null set $E_{n, m} \seq M$ (i.e., $\mu(E_{n,m}) = 0$) such that
    \[
        \|[f_n - f_m]\|_\infty
            = \esssup_M|f_n - f_m|
            \leq \sup_{M \setminus E_{n,m}} |f_n - f_m|
            < \frac{1}{2^n}.
    \]
    Define the set
    \[
        E = \bigcup_{n=1}^{\infty}\bigcup_{m=n}^{\infty} E_{n, m}.
    \]
    By subadditivity, we have
    \[
        \mu(E)
            \leq \sum_{n=1}^{\infty} \sum_{m=n}^{\infty} \mu(E_{n, n})
            = \sum_{n=1}^{\infty} \sum_{m=n}^{\infty} 0
            = 0.
    \]
    Moreover, $E_{n,m} \seq E$ implies $M \setminus E \seq M \setminus E_{n, m}$, so
    \[
        \sup_{M \setminus E} |f_n - f_m|
            \leq \sup_{M \setminus E_{n,m}} |f_n - f_m|
            < \frac{1}{2^n}.
    \]
    For each $x \in M \setminus E$ and $m \geq n$, we have
    \[
        |f_n(x) - f_m(x)| \leq \sup_{M \setminus E} |f_n - f_m| < \frac{1}{2^n}.
    \]
    This means that the sequence $\{f_n(x)\}$ of complex values is Cauchy: given $\eps > 0$, one can choose $N \in \N$ such that $1/2^N < \eps$.
    Since $\C$ is complete, the sequence converges to some value in $\C$, which we will denote by $f(x)$.
    This gives us a function $f : M \setminus E \to \C$, to which the sequence $\{f_n|_{M \setminus E}\}$ converges pointwise.
    Letting $m \to \infty$ in the previous inequality yields
    \[
        |f_n(x) - f(x)| \leq \frac{1}{2^n}.
    \]
    Taking the supremum over all $x \in M \setminus E$ gives us
    \[
        \sup_{M \setminus E} |f_n - f| \leq \frac{1}{2^n}.
    \]
    In particular, this value is finite.
    Since $[f_1] \in L^\infty(M, \mu)$, we know $\|[f_1]\|_\infty < \infty$.
    Choose a null set $E' \seq M$ such that $\sup_{M \setminus E'}|f| < \infty$.
    Then $E'' = E \cap E'$ is also a null set and
    \[
        \sup_{M \setminus E''} |f|
            = \sup_{M \setminus E''} |f - f_1 + f_1|
            \leq \sup_{M \setminus E'} |f_1 - f| + \sup_{M \setminus E'} |f_1|
            < \infty.
    \]
    We extend $f$ to $M$ by defining $f|_E = 0$.
    Then
    \[
        \|[f]\|_\infty = \esssup_M |f| \leq \sup_{M \setminus E''} |f| < \infty,
    \]
    which means $[f] \in L^1(M, \mu)$.
    Moreover,
    \[
        \|[f_n] - [f]\|_\infty
            = \|[f_n - f]\|_\infty
            \leq \sup_{M \setminus E} |f_n - f| \leq \frac{1}{2^n},
    \]
    which converges to zero as $n \to \infty$, hence $[f_n] \to [f]$ in $L^1(M, \mu)$.
\end{proof}


\newpage
\begin{pbox}[4]
    Let $X$ be Banach space and $Y$ be a closed subspace of $X$.
    Prove that the quotient space $X/Y$ is a Banach space.
    Note that if $[x] = [y]$ in $X/Y$, then $x - y \in Y$, $\|[x]\| = \inf_{y \in [x]} \|y\|$.
\end{pbox}

\begin{proof}
    The quotient of a vector space by a linear subspace is always a vector space, but we must check that the proposed norm is in fact a norm.

    Nonnegativity follows immediately from the nonnegativity of the norm in $X$.

    Since $0 \in [0]$, then $\|[0]\| \leq \|0\| = 0$, so indeed $\|[0]\| = 0$.

    Suppose $\|[x]\| = 0$.
    For each $n \in \N$, choose $y_n \in [x]$ such that $\|y_n\| \leq 1/2^n$.
    In other words, $\{y_n\}$ is a sequence in $[x]$ converging to zero.
    Define $z_n = x - y_n \in Y$, then
    \[
        \|x - z_n\| = \|-y_n\| \leq \frac{1}{2^n}.
    \]
    This means that $\{z_n\}$ is a sequence in $Y$ converging to $x$.
    Since $Y$ is closed, this implies $x \in Y$, so $[x] = 0$.

    For all nonzero scalars $a$, we have $y \in [x]$ if and only if $ay \in [ax]$.
    And in which case, $\|ay\| = |a|\|y\|$, hence we have homogeneity
    \[
        \|[ax]\| = \inf_{ay \in [ax]}\|ay\| = |a|\inf_{y \in [x]}\|y\| = |a|\|[x]\|.
    \]

    Lastly, the triangle inequality:
    \begin{align*}
        \|[x] + [y]\|
            &= \inf_{z \in [x + y]}\|z\| \\
            &= \inf_{u \in [x], v \in [y]}\|u + v\| \\
            &\leq \inf_{u \in [x], v \in [y]}(\|u\| + \|v\|) \\
            &\leq \inf_{u \in [x]}\|u\| + \inf_{v \in [y]}\|v\| \\
            &= \|[x]\| + \|[y]\|.
    \end{align*}
    We conclude that $X/Y$ is a normed space with the stated norm.

    Finally, we show that the space is complete.
    Let $\{[x_n]\}$ be a Cauchy sequence in $X/Y$.
    Our goal is to choose a representative sequence in $X$ which is also Cauchy.
    By Problem 1, and similar to the proof of Problem 2, we may assume without loss of generality that for all $n \in \N$ and $m \geq n$,
    \[
        \|[x_n] - [x_m]\| < \frac{1}{2^n}.
    \]
    (If this is not the case, use Cauchyness to choose a subsequence such that it is true, and re-index.)
    For each $n \in \N$, choose $y_n \in [x_n - x_{n+1}]$ such that 
    \[
        \|[x_n] - [x_{n+1}]\| \leq \|y_n\| < \frac{1}{2^n}.
    \]
    Set $x'_1 = x_1$ and $x'_{n+1} = x'_n - y_n$ for all $n \in \N$.
    This inductive definition gives us a new representative sequence $\{x'_n\}$ in $X$ with $[x'_n] = [x_n]$.
    Moreover,
    \[
        \|x'_n - x'_{n+1}\| = \|x'_n - (x'_n - y_n)\| = \|y_n\| < \frac{1}{2^n},
    \]
    so $\{x'_n\}$ is a Cauchy sequence in $X$ and, therefore, converges to some $x \in X$.
    Then
    \[
        \|[x_n] - [x]\|
            = \|[x'_n] - [x]\|
            = \|[x'_n - x]\|
            \leq \|x'_n - x\|,
    \]
    which converges to zero as $n \to \infty$.
    So $[x_n] \to [x]$ in $X/Y$, which is therefore is complete.
\end{proof}


\newpage
\begin{pbox}[5 \\ (a)]
    Prove that in every inner product space $X$, the following identity (parallelogram law) holds:
    \[
        \|x + y\|^2 + \|x - y\|^2 = 2\|x\|^2 + 2\|y\|^2, \quad \forall x, y \in X.
    \] 
\end{pbox}

\begin{proof}
    Applying the definition of the induced norm and the bilinearity (with respect to real scalars), we calculate
    \begin{align*}
        \|x + y\|^2 + \|x - y\|^2
            &= \<x + y, x + y\> + \<x - y, x - y\> \\
            &= \<x, x\> + \<x, y\> + \<y, x\> + \<y, y\> \\
            &\quad + \<x, x\> - \<x, y\> - \<y, x\> + \<y, y\> \\
            &= 2\|x\|^2 + 2\|y\|^2.
    \end{align*}
\end{proof}



\newpage
\begin{pbox}[(b)]
    Prove that the parallelogram law characterizes inner product spaces.
    Namely, suppose that the parallelogram law holds on a normed linear space $X$, then one can define an inner product $\<-, -\>$ in $X$ in such a way that $\|x\| = \<x, x\>^{1/2}$ for all $x \in X$.
    (Hint: consider the polarization identity
    \[
        \<x, y\> = \frac{1}{4}\left((\|x + y\|^2 - \|x - y\|^2) - i(\|x + iy\|^2 - \|x - iy\|^2)\right).
    \]
\end{pbox}

\begin{proof}
    We first check positive definite:
    \begin{align*}
        \<x, x\>
            &= \frac{1}{4}\left((\|x + x\|^2 - \|x - x\|^2) - i(\|x + ix\|^2 - \|x - ix\|^2)\right) \\
            &= \frac{1}{4}\left((\|2x\|^2 - 0) - i(\|x + ix\|^2 - (|-i|\|ix + x\|)^2)\right) \\
            &= \frac{1}{4}\left(4\|x\|^2 - i(\|x + ix\|^2 - \|x + ix\|^2)\right) \\
            &= \|x\|^2 \\
            &\geq 0,
    \end{align*}
    with equality if and only if $x = 0$, by definition of the norm.

    We now check conjugate symmetry:
    \begin{align*}
        \overline{\<y, x\>}
            &= \frac{1}{4}\left((\|y + x\|^2 - \|y - x\|^2) + i(\|y + ix\|^2 - \|y - ix\|^2)\right) \\
            &= \frac{1}{4}\left((\|x + y\|^2 - \|x - y\|^2) + i((|i|\|x - iy\|)^2 - (|-i|\|x + iy\|)^2)\right) \\
            &= \frac{1}{4}\left((\|x + y\|^2 - \|x - y\|^2) - i(\|x + iy\|^2 - \|x - iy\|^2)\right).
    \end{align*}
    
    From the parallelogram law, we deduce
    \[
        \|x + x' + y\|^2 = 2\|x + y\|^2 + 2\|x'\|^2 - \|x - x' + y\|^2
    \]
    and
    \[
        \|x + x' + y\|^2 = 2\|x' + y\|^2 + 2\|x\|^2 - \|x' - x + y\|^2.
    \]
    Then
    \[
        \|x + x' + y\|^2 = \|x\|^2 + \|x'\|^2 + \|x + y\|^2 + \|x' + y\|^2 - \|x - x' + y\|^2 - \|x' - x + y\|^2
    \]
    and, similarly,
    \[
        \|x + x' - y\|^2 = \|x\|^2 + \|x'\|^2 + \|x - y\|^2 + \|x' - y\|^2 - \|x - x' - y\|^2 - \|x' - x - y\|^2.
    \]
    Therefore,
    \[
        \|x + x' + y\|^2 - \|x + x' - y\|^2
            = \|x + y\|^2 - \|x - y\| + \|x' + y\| - \|x' - y\|
    \]
    and, similarly,
    \[
        \|x + x' + iy\|^2 - \|x + x' - iy\|^2
            = \|x + iy\|^2 - \|x - iy\| + \|x' + iy\| - \|x' - iy\|.
    \]
    Substituting into the definition of the inner product gives us 
    \[
        \<x + x', y\> = \<x, y\> + \<x', y\>.
    \]
    It follows that the inner product is $\Z$-linear and therefore $\Q$-linear in the first argument.
    Then
    \[
        i\<x, y\> = 
    \]
    \begin{align*}
        i\<x, y\>
            &= i\frac{1}{4}\left((\|x + y\|^2 - \|x - y\|^2) - i(\|x + iy\|^2 - \|x - iy\|^2)\right) \\
            &= \frac{1}{4}\left(i((|-i|\|ix + iy\|)^2 - (|-i|\|ix - iy\|)^2) + ((|-i|\|ix - y\|)^2 - (|-i|\|ix + y\|)^2)\right) \\
            &= \frac{1}{4}\left((\|ix + y\|^2 - \|ix - y\|^2) - i(\|ix + iy\|^2 - \|ix - iy\|^2)\right) \\
            &= \<ix, y\>.
    \end{align*}
    Hence, the inner product is $\Q(i)$-linear in the first argument.
    Since $\Q(i)$ is dense in $\C$ and the inner product is continuous, we conclude that it must also be $\C$-linear.
\end{proof}

\end{document}