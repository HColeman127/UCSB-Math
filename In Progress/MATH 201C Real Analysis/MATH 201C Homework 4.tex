\documentclass[12pt]{article}

% Packages
\usepackage[margin=1in]{geometry}
\usepackage{fancyhdr, parskip}
\usepackage{amsmath, amsthm, amssymb}
\usepackage{tikz, tikz-cd}

% Page Style
\makeatletter
\fancypagestyle{title}{
    \renewcommand{\headrulewidth}{0.4pt}
    \setlength{\headheight}{15pt}
    \fancyhead[R]{\@author}
    \fancyhead[L]{\@title}
    \fancyhead[C]{\@date}
}
\makeatother
\renewcommand{\maketitle}{\thispagestyle{title}}
\fancypagestyle{plain}{
    \fancyhf{}
    \renewcommand{\headrulewidth}{0pt}
    \renewcommand{\footrulewidth}{0pt}
    \fancyfoot[R]{\thepage}
}
\pagestyle{plain}

% Problem Box
\setlength{\fboxsep}{4pt}
\newlength{\myparskip}
\setlength{\myparskip}{\parskip}
\newsavebox{\savefullbox}
\newenvironment{fullbox}{\begin{lrbox}{\savefullbox}\begin{minipage}{\dimexpr\textwidth-2\fboxsep\relax}\setlength{\parskip}{\myparskip}}{\end{minipage}\end{lrbox}\framebox[\textwidth]{\usebox{\savefullbox}}}
\newenvironment{pbox}[1][]{\begin{fullbox}\def\temp{#1}\ifx\temp\empty\else\paragraph{#1}\phantom{}\fi}{\end{fullbox}}

\newcommand{\pnum}{\paragraph}


% Theorem Environments
\theoremstyle{definition}
\newtheorem{lemma}{Lemma}

% Tikz Environments
\newenvironment{drawing}{\begin{center}\begin{tikzpicture}}{\end{tikzpicture}\end{center}}
% \tikzcdset{row sep/normal=0pt}
\newenvironment{cd}{\begin{center}\begin{tikzcd}}{\end{tikzcd}\end{center}}

% Default Commands
\newcommand{\isp}[1]{\quad\text{#1}\quad}
\newcommand{\N}{\mathbb{N}} 
\newcommand{\Z}{\mathbb{Z}}
\newcommand{\Q}{\mathbb{Q}}
\newcommand{\R}{\mathbb{R}}
\newcommand{\C}{\mathbb{C}}
\newcommand{\A}{\mathbb{A}}
\renewcommand{\P}{\mathbb{P}}
\newcommand{\eps}{\varepsilon}
\renewcommand{\phi}{\varphi}
\renewcommand{\emptyset}{\varnothing}
\newcommand{\<}{\langle}
\renewcommand{\>}{\rangle}
\newcommand{\iso}{\cong}
\newcommand{\eqc}{\overline}
\newcommand{\clo}{\overline}
\newcommand{\seq}{\subseteq}
\newcommand{\teq}{\trianglelefteq}
\DeclareMathOperator{\id}{id}
\DeclareMathOperator{\im}{im}
\newcommand{\inc}{\hookrightarrow}
\newcommand{\dd}{\mathrm{d}}
\newcommand{\mat}[1]{\begin{bmatrix}#1\end{bmatrix}}

% Extra Commands
\renewcommand{\SS}{\mathcal{S}}
\newcommand{\DD}{\mathrm{D}}
\newcommand{\LL}{\mathcal{L}}
\newcommand{\HH}{\mathcal{H}}

\newcommand{\conj}{\overline}

% Document
\begin{document}
\title{MATH 201C Homework 4}
\author{Harry Coleman}
\date{May 28, 2022}
\maketitle

\begin{pbox}[1]
    Let $\SS$ be the Schwartz space (functions of rapid decay), and $\delta$ is the Dirac delta function.
    Prove directly that the derivative of $\delta$, denoted by $\delta'$, is in $S'$ (the dual space of $\SS$).
\end{pbox}

\begin{proof}
    Note that $\delta \in \SS'(\R)$.
    By definition, $\delta' : \SS(\R) \to \C$ is a linear map given by
    \[
        \delta'(f)
            = \DD^1\delta(f)
            = (-1)^1\delta(\DD^1\!f)
            = -\delta(f')
            = - f'(0).
    \]
    Recall that $\SS$ is a locally compact space with the seminorms $\|{-}\|_{\alpha,\beta}$.
    Trivially, $\C$ is also a locally compact space with its usual norm $|{-}|$.
    Therefore, boundedness follows from
    \[
        |\delta'(f)|
            = |f'(0)|
            \leq \|f'\|_\infty
            = \sup_{x \in \R} |x^0\DD^1\!f(x)|
            = \|f\|_{0,1}.
    \]
\end{proof}

\begin{pbox}
    Prove that $\delta'$ does not come from a measure.
\end{pbox}

\begin{proof}
    Suppose there is a measure $\nu$ on $\R$ such that for all $f \in \SS$
    \[
        \int_\R f \,\dd{\nu}
            = \delta'(f)
            = -f'(0).
    \]
    Consider the function $f(x) = \int_{0}^{x} e^{-t^2} \dd{t}$.
    This is an increasing smooth function whose derivatives near $\pm$ infinity go to zero faster than any polynomial grows, i.e., $x^\alpha\DD^{\beta}\!f(x) \to 0$ as $x \to \pm\infty$ for all $\alpha, \beta \in \N$.
    Since each derivative is continuous, this implies $\|f\|_{\alpha,\beta} < \infty$, so in fact $f \in \SS$.
    Additionally, $f'(0) = 1$.

    Define the functions $f_n(x) = f(nx)/n$ for all $n \in \N$.
    Then $f_n \in \SS$ and 
    \[
        \int_\R f_n \,\dd{\nu}
            = \delta'(f_n)
            = -f_n'(0)
            = -1.
    \]
    Note that $\|f\|_\infty \leq 1$ so $\|f_n\|_\infty \to 0$ as $n \to \infty$.
    Each $f_n$ is integrable and $|f_n| \geq |f_{n+1}|$, so monotone convergence gives us
    \[
        -1
            = \lim_{n \to \infty} \int_\R f_n \,\dd{\nu}
            = \int_{\R} \lim_{n\to\infty} f_n \,\dd{\nu} 
            = \int_\R 0 \,\dd{\nu}
            = 0.
    \]
    This is a contradiction, so no such measure $\nu$ exists.
\end{proof}

\newpage

\begin{pbox}[2]
    Let $X$ and $Y$ be Banach spaces.
    Prove that if $T_n \in \LL(X, Y)$ and $\{T_n x\}$ is a Cauchy sequence for each $x \in X$, then there exists $T \in \LL(X, Y)$ so that $T_n \to T$ in the strong operator.
\end{pbox}

\begin{proof}
    Since $Y$ is Banach, $T_nx$ converges to some point $Tx \in Y$.
    Then $T : X \to Y$ is a linear map:
    \[
        T(ax + y)
            = \lim_{n \to \infty} T_n(ax + y)
            = a\lim_{n\to \infty} T_nx + \lim_{n \to \infty} T_ny
            = aTx + Ty.
    \]
    By construction, $T_nx \to Tx$ for all $x \in X$.
    In particular,
    \[
        \sup_{n \in \N}\|T_nx\| < \infty
    \]
    for all $x \in X$.
    So the uniform boundedness principle gives us a bound on the operator norms
    \[
        M = \sup_{n\in\N}\|T_n\| < \infty.
    \]
    Then for all $x \in X$ we have
    \[
        \|Tx\|
            = \lim_{n \to \infty} \|T_nx\|
            \leq \lim_{n \to \infty} \|T_n\|\|x\|
            \leq \lim_{n \to \infty} M\|x\|
            = M\|x\|,
    \]
    Hence, $T$ is bounded, i.e., $T \in \LL(X, Y)$.
\end{proof}


\newpage

\begin{pbox}[3]
    Let $T_t$ be an operator on $L^2(\R)$ with $T_t\phi(x) := \phi(x + t)$.
    What is the norm of $T_t$?
\end{pbox}

We have
\[
    \|T_t\phi\|_2
        = \int_\R |\phi(x+t)|^2 \,\dd{x}
        = \int_\R |\phi(x)|^2 \,\dd{x}
        = \|\phi\|_2,
\]
so $\|T_t\| = 1$.


\begin{pbox}
    To what operator does $T_t$ converge as $t \to \infty$ and in what topology?
\end{pbox}

Claim that $T_t \to 0$ weakly.

\begin{proof}
    Suppose $\phi \in L^2(\R)$ and $\ell \in L^2(\R)^*$.
    Choose $\psi \in L^2(\R)$ by the Riesz lemma for $\ell$.
    Then
    \[
        |\ell(T_t\phi)| = \left|\int_\R \conj{\psi(x)}\phi(x + t) \,\dd{x}\right|.
    \]
    Given $\eps > 0$ choose $M > 0$ such that
    \[
        \int_{|x|>M} |\psi(x)|^2 \dd{x} < \eps
        \isp{and}
        \int_{|x|>M} |\phi(x)|^2 \dd{x} < \eps.
    \]
    Then Cauchy-Schwarz gives us
    \begin{align*}
        \left|\int_\R \conj{\psi(x)}\phi(x - t) \,\dd{x}\right|
            &\leq \left|\int_{|x|>M} \conj{\psi(x)}\phi(x + t) \,\dd{x}\right| + \left|\int_{|x|\leq M} \conj{\psi(x)}\phi(x + t) \,\dd{x}\right| \\
            &\leq \left(\int_{|x|>M} |\psi(x)|^2\dd{x}\right)^{1/2}
                \left(\int_{|x|>M} |\phi(x + t)|^2\dd{x}\right)^{1/2} \\
            &\qquad + \left(\int_{|x|\leq M} |\psi(x)|^2\dd{x}\right)^{1/2}
                \left(\int_{|x|\leq M} |\phi(x + t)|^2\dd{x}\right)^{1/2} \\
            &\leq \eps \|\phi\|_\infty + \|\psi\|_\infty \left(\int_{|x|\leq M} |\phi(x + t)|^2\dd{x}\right)^{1/2} \\
            &\leq \eps \|\phi\|_\infty + \|\psi\|_\infty \left(\int_{|x-t|\leq M} |\phi(x)|^2\dd{x}\right)^{1/2}.
    \end{align*}
    If we choose $t$ large enough, $|x - t| \leq M$ will imply $|x| > M$, giving us
    \[
        |\ell(T_t\phi)|
            = \left|\int_\R \conj{\psi(x)}\phi(x - t) \,\dd{x}\right|
            \leq \eps \|\phi\|_\infty + \|\psi\|_\infty \eps.
    \]
    Hence, we have convergence $T_t \to 0$ in the weak topology.
\end{proof}



\newpage

\begin{pbox}[4 (a)]
    Let $A$ be a self-adjoint operator on a Hilbert space $\HH$.
    Prove that 
    \[
        \|A\| = \sup_{\|x\| = 1} |\<Ax, x\>|.
    \]
\end{pbox}


Following the hint, assume $\|x\| = \|y\| = 1$, then we find
\begin{align*}
    4|\operatorname{Re}\<x, Ay\>|
        &\leq |\<x + y, A(x + y)\>| + |\<x - y, A(x - y)\>| \\
        &\leq \|x + y\|^2 \sup_{\|u\| = 1} \<u, Au\> + \|x - y\|^2 \sup_{\|u\| = 1} |\<u, Au\>| \\
        &= (2\|x\|^2 + 2\|y\|^2) \sup_{\|u\| = 1} |\<u, Au\>| \\
        &= 4 \sup_{\|u\| = 1} |\<u, Au\>|.
\end{align*}
Consider the unit vector $w = \conj{\<x, Ay\>}/|\<x, Ay\>|$, then $w\<x, Ay\> = \<wx, Ay\>$ is real, so applying the above inequality we get
\[
    |\<x, Ay\>|
        = |w\<x, Ay\>|
        = |\operatorname{Re}\<wx, Ay\>|
        \leq \sup_{\|u\| = 1} |\<u, Au\>|.
\]
We now write
\[
    \|A^2\|
        = \|A\|^2
        = \sup_{\|x\|=1} \|Ax\|^2
        = \sup_{\|x\|=1} |\<Ax, Ax\>|
        = \sup_{\|x\|=1} |\<x, A^2x\>|.
\]

I did not complete the proof.

\begin{pbox}[b]
    Find an example which shows that the conclusion of (a) need not be true if $A$ is not self-adjoint.
\end{pbox}

Let $\HH = \R^2$ and consider the rotation $A(x_1, x_2) = (-x_2, x_1)$.

Then $\|A\| = 1$ but $\<Ax, x\> = 0$ for all $x \in \R^2$.

\newpage

\begin{pbox}[5]
    Show that the spectral radius of the Volterra integral operator
    \[
        (Tf)(x) = \int_{0}^{x} f(y) \dd{y} 
    \]
    as a map on $C[0, 1]$, with the supremum norm, is equal to zero.
\end{pbox}

\begin{proof}
    Note that $C[0, 1]$ with the supremum norm is a Banach space, so the spectral radius is given by $\lim_{n \to \infty} \|T^n\|^{1/n}$.
    We claim that for $f \in C[0, 1]$ with $\|f\|_\infty = 1$ and $x \in [0, 1]$
    \[
        |T^nf(x)| \leq \frac{x^n}{n!}.
    \]
    We perform induction on $n \geq 0$.
    The base case of is trivial with $|f(x)| \leq \|f\|_\infty = 1$.
    The inductive step is shown by
    \[
        |T^nf(x)|
            = \left|\int_{0}^{x} T^{n-1}f(y) \,\dd{y} \right|
            \leq \int_{0}^{x} |T^{n-1}f(y)| \,\dd{y}
            \leq \int_{0}^{x} \frac{y^{n-1}}{(n-1)!} \,\dd{y}
            = \frac{x^n}{n!}.
    \]
    We now compute
    \[
        \|T^n\|
            = \sup_{\|f\|_\infty = 1} \|T^nf\|_\infty
            \leq \sup_{\|f\|_\infty = 1} \sup_{x \in [0, 1]} \frac{x^n}{n!}
            = \frac{1}{n!},
    \]
    so then
    \[
        \lim_{n\to\infty} \|T^n\|^{1/n}
            \leq \lim_{n\to\infty} \frac{1}{\sqrt[n]{n!}}
            = 0.
    \]
    Hence, the spectral radius must be zero.
\end{proof}

\begin{pbox}
    What is the norm of $T$?
\end{pbox}

In the first part, we found the upper bound $\|T\| \leq 1$.
To see that we in fact have equality, consider the constant function $f(x) = 1$:
\[
    \|Tf\|_\infty
        = \sup_{x \in [0, 1]} \int_{0}^{x} f(y) \dd{y} 
        = \sup_{x \in [0, 1]} \int_{0}^{x} 1 \dd{y} 
        =  \sup_{x \in [0, 1]} x
        = 1.
\]
Hence, $\|T\| = 1$.

\end{document}