\documentclass[12pt]{article}

% Packages
\usepackage[margin=1in]{geometry}
\usepackage{fancyhdr, parskip}
\usepackage{amsmath, amsthm, amssymb}

% Page Style
\fancypagestyle{plain}{
    \fancyhf{}
    \renewcommand{\headrulewidth}{0pt}
    \renewcommand{\footrulewidth}{0pt}
    \fancyfoot[R]{\thepage}
}
\pagestyle{plain}

% Problem Box
\setlength{\fboxsep}{4pt}
\newsavebox{\savefullbox}
\newenvironment{fullbox}{\begin{lrbox}{\savefullbox}\begin{minipage}{\dimexpr\textwidth-2\fboxsep\relax}}{\end{minipage}\end{lrbox}\begin{center}\framebox[\textwidth]{\usebox{\savefullbox}}\end{center}}
\newenvironment{pbox}[1][]{\begin{fullbox}\ifx#1\empty\else\paragraph{#1}\fi}{\end{fullbox}}

% Theorem Environments
%\theoremstyle{definition}
%\newtheorem{proposition}{Proposition}
%\newtheorem{lemma}{Lemma}

% Options
%\allowdisplaybreaks
%\addtolength{\jot}{4pt}

% Default Commands
\newcommand{\isp}[1]{\quad\text{#1}\quad}
\newcommand{\N}{\mathbb{N}} 
\newcommand{\Z}{\mathbb{Z}}
\newcommand{\Q}{\mathbb{Q}}
\newcommand{\R}{\mathbb{R}}
\newcommand{\C}{\mathbb{C}}
\newcommand{\eps}{\varepsilon}
\renewcommand{\phi}{\varphi}
\renewcommand{\emptyset}{\varnothing}
\newcommand{\<}{\langle}
\renewcommand{\>}{\rangle}
\newcommand{\isom}{\cong}
\newcommand{\eqc}{\overline}
\newcommand{\clo}{\overline}

% Extra Commands
\newcommand{\mat}[1]{\begin{bmatrix}#1\end{bmatrix}}
\newcommand{\mdet}[1]{\begin{vmatrix}#1\end{vmatrix}}

% Document Info
\fancypagestyle{title}{
    \renewcommand{\headrulewidth}{0.4pt}
    \setlength{\headheight}{15pt}
    \fancyhead[R]{Harry Coleman}
    \fancyhead[L]{MATH 118C Midterm}
    \fancyhead[C]{May 6, 2021}
}

% Begin Document
\begin{document}
\thispagestyle{title}

\section*{1}

\begin{proof}
    In particular, as $f$ is continuous on $\R$, it is continuous on the compact interval $[-\pi, \pi]$ and, therefore, Lipschitz continuous on $[-\pi, \pi]$. Moreover, since $f$ has a period of $2\pi$ and is continuous, then it is Lipschitz on all of $\R$. So there exists some $M > 0$ such that
    \[
        |f(x) - f(y)| \leq M|x - y|
    \]
    for all $x, y \in \R$. Then for any $x_0 \in \R$, we have
    \[
        |f(x_0 + t) - f(x_0)| \leq M|t|
    \]
    for all $t \in \R$. This condition implies that $s_n(f; x_0) \to f(x_0)$ as $n \to \infty$, i.e.,
    \[
        f(x_0)
            = \sum_{n = -\infty}^{\infty} c_n e^{inx_0}
            = \sum_{n = -\infty}^{\infty} 0 e^{inx_0}
            = 0.
    \]

\end{proof}


\newpage
\section*{2}

\section*{2(a)}

\begin{proof}
    We first assume that both $x$ and $y$ are nonzero. The derivative of $F$ at $(x, y)$ is given by
    \[
        [F'(x, y)]
            = \mat{(D_1F_1)(x, y) & (D_2F_1)(x, y) \\ (D_1F_2)(x, y) & (D_2F_2)(x, y)}
            = \mat{2x & -2y \\ 2x & 2y},
    \]
    so the Jacobian is
    \[
        J_F(x, y)
            = \det F'(x, y) 
            = (2x)(2y) - (2x)(-2y)
            = 4xy + 4xy
            = 8xy
    \]
    Since, $x \ne 0$ and $Y \ne 0$, then $J_f(x, y) \ne 0$. In which case, $F'(x, y)$ is invertible, so the implicit function theorem implies the existence of a neighborhood of $(x, y)$ on which $F$ is injective.

\end{proof}

\section*{2(b)}

No.

Consider $F$ along the $x$-axis, i.e., where $y = 0$. We have
\[
    F_1(x, 0) = x^2 \isp{and} F_2(x, 0) = 0.
\]
Then any neighborhood of $(0, 0)$ contains the points $(\pm\delta, 0)$ for some $\delta > 0$. Then
\[
    F(\delta, 0) = (\delta^2, 0) = F(-\delta, 0).
\]
Therefore, $F$ is not injective on any neighborhood of $(0, 0)$.


\newpage
\section*{3}

Since $F$ is continuously differentiable the map $B^n \to \R$ which maps $x \mapsto \|F'(x)\|$ is a continuous function. Since it is a continuous function on the compact set $B^n$, then it attains a maximum at some point $a \in B^n$. Define $\delta = \|F'(a)\| < 1$. Then $\|F'(x)\| \leq \delta$ for all $x \in B^n$. Since $B^n$ is a convex set on which $F$ is continuously differentiable, then we deduce that
\[
    |F(x) - F(y)| \leq \delta|x - y|
\]
for all $x, y \in B^n$. Since $\delta < 1$, this means that $F$ is a contraction mapping on $B^n$. Therefore, there exists a fixed point $x_0 \in B^n$ of $F$, i.e., $F(x_0) = x_0$.

\newpage
\section*{4}

\section*{4(a)}

Let $U \subseteq \R^n$ be an open subset and $F : U \to \R^n$ be a continuously differentiable function. If $x_0 \in U$ with $J_F(x_0) \ne 0$, then there exists an open neighborhood $V$ of $x_0$ on which $F$ is injective. Moreover, if we define $W = F(V)$ and $G = F^{-1}|_{W}$, then $G : W \to V$ is a continuously differentiable function.

\section*{4(b)}

Let $U \subseteq \R^{n+m}$ be an open subset and $F : U \to \R^m$ be a continuously differentiable.

Assume $F =(F_1, \dots, F_m)$, define
\[
    A_x = \mat{D_1F_1 & \cdots & D_n F_1 \\ \vdots & & \vdots \\ D_1F_m & \cdots & D_nF_m}
    \isp{and}
    A_y = \mat{D_{n+1}F_1 & \cdots & D_m F_1 \\ \vdots & & \vdots \\ D_{n+1}F_m & \cdots & D_mF_m},
\]
then $[F'] = \mat{A_x & A_y}$.


If $(x_0, y_0) \in U$ such that $F(x_0, y_0) = 0$ and $\det A_x \ne 0$, then there exists open neighborhoods $V$ of $(x_0, y_0$ and $W$ of $y_0$ such that for every $y \in W$, there exists a unique $x \in \R^n$ with $(x, y) \in V$ and $F(x, y) = 0$.

Moreover, we can define a map $G : W \to \R^n$ such that $(G(y), y) \in V$ and $F(G(y), y) = 0$ for all $y \in W$. Then $G$ is continuously differentiable with $G'(y_0) = -(A_x)^{-1}A_y$.


\section*{4(c)}

\begin{proof}
    Given a continuously differentiable function $F : U \to \R^n$ with $U \subseteq \R^n$ open, we extend $F$ to a function $(F, I) : \R^{2n} \to \R$ by $(x, y) \mapsto (F(x), y)$. Then for each point $x_0 \in U$ with $J_F(x_0) \ne 0$, this is condition $\det A_x \ne 0$ for the implicit function theorem. Then the local function $G$ from the implicit function theorem is precisely a local inverse of $F$.

\end{proof}




\newpage
\section*{5}

\begin{proof}
    If $n \leq m$, then we can project $F$ onto the first $n$ coordinates, giving us a continuously differentiable map $U \to V \cap \R^n$. Then using the inverse function theorem, we can construct a local inverse for each point in $U$ of this map. However, since we already have a global inverse $F^{-1}$, then we could not have $n < m$, since the projection of $F$ is locally bijective $U \to V \cap \R^n$. If $m \leq n$, we make the same argument for $F^{-1}$, implying that we must have $n = m$.

\end{proof}

\end{document}