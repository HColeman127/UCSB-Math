\documentclass[12pt]{article}

% Packages
\usepackage[margin=1in]{geometry}
\usepackage{fancyhdr}
\usepackage{amsmath, amsthm, amssymb, physics}

% Page Style
\fancypagestyle{plain}{
    \fancyhf{}
    \renewcommand{\headrulewidth}{0pt}
    \renewcommand{\footrulewidth}{0pt}
    \fancyfoot[R]{\thepage}
}
\pagestyle{plain}

% Problem Box
\setlength{\fboxsep}{4pt}
\newsavebox{\savefullbox}
\newenvironment{fullbox}{\begin{lrbox}{\savefullbox}\begin{minipage}{\dimexpr\textwidth-2\fboxsep\relax}}{\end{minipage}\end{lrbox}\begin{center}\framebox[\textwidth]{\usebox{\savefullbox}}\end{center}}
\newenvironment{pbox}[1][]{\begin{fullbox}\ifx#1\empty\else\paragraph{#1}\fi}{\end{fullbox}}

% Options
\renewcommand{\thesubsection}{\thesection(\alph{subsection})}
\allowdisplaybreaks
\addtolength{\jot}{4pt}
\theoremstyle{definition}

% Default Commands
\newtheorem{proposition}{Proposition}
\newtheorem{lemma}{Lemma}
\newcommand{\ds}{\displaystyle}
\newcommand{\isp}[1]{\quad\text{#1}\quad}
\newcommand{\N}{\mathbb{N}}
\newcommand{\Z}{\mathbb{Z}}
\newcommand{\Q}{\mathbb{Q}}
\newcommand{\R}{\mathbb{R}}
\newcommand{\C}{\mathbb{C}}
\newcommand{\eps}{\varepsilon}
\renewcommand{\phi}{\varphi}
\renewcommand{\emptyset}{\varnothing}
\newcommand{\pfrac}[2]{\left(\frac{#1}{#2}\right)}

% Extra Commands
\newcommand{\bm}{\mathbf}
\newcommand{\x}{\mathbf{x}}
\newcommand{\y}{\mathbf{y}}
\newcommand{\e}{\mathbf{e}}

% Document Info
\fancypagestyle{title}{
    \renewcommand{\headrulewidth}{0.4pt}
    \setlength{\headheight}{15pt}
    \fancyhead[R]{Harry Coleman}
    \fancyhead[L]{MATH 118C Homework 2}
    \fancyhead[C]{Aprile 13, 2021}
}

% Begin Document
\begin{document}
\thispagestyle{title}


\begin{pbox}[Exercise 9.2]
    Prove that $BA$ is linear if $A$ and $B$ are linear transformations.
\end{pbox}

\begin{proof}
    Assume the codomain of $A$ is the domain of $B$. Then for any $\x_1, \x_2$ in the domain of $X$ and scalar $c$, the linearity of $A$ and $B$ give us
    \[
        BA(c\x_1 + \x_2) = B(cA\x_1 + A\x_2) = cBA\x_1 + BA\x_2.
    \]
    Hence, $BA$ is linear.
\end{proof}

\begin{pbox}[]
    Prove also that $A^{-1}$ is linear and invertible.
\end{pbox}

\begin{proof}
    Assume $A$ is an invertible (bijective) linear transformation with inverse $A^{-1}$. Then for any $\y_1, \y_2$ is the domain of $A^{-1}$ (same as the codomain of $A$) and scalar $c$, we find
    \begin{align*}
        A^{-1}(c\y_1 + \y_2)
            &= A^{-1}(cAA^{-1}\y_1 + AA^{-1}\y_2) \\
            &= A^{-1}A(cA^{-1}\y_1 + A^{-1}\y_2) \\
            &= cA^{-1}\y_1 + A^{-1}\y_2.
    \end{align*}
    Hence, $A^{-1}$ is linear. For any $\x$ in the codomain of $A^{-1}$ (same as the domain of $A$), we know that $A^{-1}A\x = \x$, so $A^{-1}$ is surjective. If $A^{-1}\y_1 = A^{-1}\y_2$, then applying $A$, we find
    \[
        y_1 = AA^{-1}\y_1 = AA^{-1}\y_2 = \y_2.
    \]
    Hence, $A^{-1}$ is also injective, therefore invertible. 

\end{proof}




\begin{pbox}[Exercise 9.3]
    Assume $A \in L(X, Y)$ and $A\x = \bm{0}$ only when $\x = \bm{0}$. Prove that $A$ is then 1-1.
\end{pbox}

\begin{proof}
    Suppose $\x_1, \x_2 \in X$ such that $A\x_1 = A\x_2$. Since $A$ is linear,
    \[
        \bm{0} = A\x_1 - A\x_2 = A(\x_1 - \x_2).
    \]
    Therefore, $\x_1 - \x_2 = \bm{0}$, so in fact $\x_1 = \x_2$. Hence, $A$ is injective.

\end{proof}




\newpage
\begin{pbox}[Exercise 9.5]
    Prove that to every $A \in L(\R^n, \R^1)$ corresponds a unique $\y \in \R^n$ such that $A\x = \x \cdot \y$.
\end{pbox}

\begin{proof}
    Let $\{\e_1, \dots, \e_n\}$ be the standard basis for $\R^n$. Given $A \in L(\R^n, \R^1)$, define the vector
    \[
        \y = A\e_1 + \cdots + A\e_n = \mqty[A\e_1 & \cdots & A\e_n]^T.
    \]
    Each $\x \in \R^n$ has a unique representation as
    \[
        \x = a_1\e_1 + \cdots + a_n\e_n = \mqty[a_1 & \cdots & a_n]^T,
    \]
    for some $a_1, \dots, a_n \in \R$. Then we find
    \begin{align*}
        A\x &= A(a_1\e_1 + \cdots + a_n\e_n) \\
            &= a_1A\e_1 + \cdots + a_nA\e_n \\
            &= \mqty[a_1 & \cdots & a_n]^T \cdot \mqty[A\e_1 & \cdots & A\e_n]^T \\
            &= \x \cdot \y.
    \end{align*}
    This proves existence, we now show uniqueness. Suppose there are two vectors $\y_1, \y_2 \in \R^n$ such that $\x \cdot \y_1 = A\x = \x \cdot \y_2$ for all $\x \in \R$. In particular, take
    \[
        \y_1 = \mqty[a_1 & \cdots & a_n]^T
        \isp{and}
        \y_2 = \mqty[b_1 & \cdots & b_n]^T.
    \]
    Then for $j = 1, \dots, n$, we find
    \[
        a_j = \e_j \cdot \y_1 = \e_j \cdot \y_2 = b_j.
    \]
    Hence, $y_1 = y_2$.

\end{proof}


\begin{pbox}[]
    Prove also that $\|A\| = |\y|$.
\end{pbox}

\begin{proof}
    For any $\x \in \R^n$ with $|x| = 1$, the Cauchy-Schwarz inequality gives us
    \[
        |A\x| = |\x \cdot \y| \leq |\x||\y| = |\y|.
    \]
    Since this inequality holds for all unit vectors, then
    \[
        \|A\| = \sup_{\substack{x \in \R^n \\ |x| = 1}}|Ax| \leq |\y|.
    \]
    In particular, consider the the unit vector $\y/|\y|$. We find
    \[
        \|A\| 
            \geq |A\y| 
            = |\y \cdot (\y/|\y|)| 
            = \frac{|\y \cdot \y|}{|\y|}
            = \frac{|\y|^2}{|\y|}
            = |\y|
    \]
    Thus, $\|A\| = |\y|$.

\end{proof}




\begin{pbox}[Exercise 9.6]
    If $f(0, 0) = 0$ and
    \[
        f(x, y) = \frac{xy}{x^2 + y^2} \quad \text{if } (x, y) \ne (0, 0),
    \]
    prove that $(D_1f)(x, y)$ and $(D_2f)(x, y)$ exist at every point of $\R^2$, although $f$ is not continuous at $(0, 0)$.
\end{pbox}

\begin{proof}
    Note that $f(x, y)$ is symmetric in the arguments $x$ and $y$, so it is sufficient to show that $(D_1f)(x, y)$ exists at every point of $\R^2$. For all $x \in \R$, we have $f(x, 0) = 0$, so $(D_1f)(x, 0) = 0$. If we fix $y \ne 0$, then $f(x, y)$ is simply a differentiable function on $\R$ in the variable $x$, with derivative $(D_1f)(x, y)$. Hence, all the partial derivatives of $f$ exist.

    As previously noted, $f(x, 0) = 0$ for all $x \in \R$, which means that $f(x, 0) \to 0$ as $x \to 0$. On the other hand, for $x \ne 0$, we have
    \[
        f(x, x) = \frac{x^2}{2x^2} = \frac{1}{2.}
    \]
    So $f(x, x) \to 1/2$ as $x \to 0$. Thus, the limit of $f(x, y)$ as $(x, y) \to (0, 0)$ does not exist, implying that $f$ is discontinuous at the origin.

\end{proof}




\begin{pbox}[Exercise 9.8]
    Suppose that $f$ is a differentiable real function in an open set $E \subseteq \R^n$, and that $f$ has a local maximum at a point $\x \in E$. Prove that $f'(\x) = 0$.
\end{pbox}

\begin{proof}
    Let $\x = (x_1, \dots, x_n)$. For each $j = 1, \dots, n$, define the function
    \[
        f_j(x) = f(x_1, \dots, x_{j-1}, x, x_{j+1}, \dots, x_n),
    \]
    Define $E_j \subseteq $ to be the projection of $E$ onto the $j$th coordinate. Then $f_j : E_j \to \R$ is a differentiable function with a local maximum at $x_j \in E_j$. Then as an instance of the $1$-dimensional case, we know that 
    \[
        (D_jf)(\x) = f_j'(x_j) = 0.
    \]
    Therefore,
    \[
        [f'(\x)] = \mqty[(D_1f)(\x) & \cdots & (D_nf)(\x)]^T = \mqty[0 & \cdots & 0]^T,
    \]
    so in fact $f'(\x) = 0$.

\end{proof}




\newpage
\begin{pbox}[Exercise 9.9]
    If $\bm{f}$ is a differentiable mapping of a connected open set $E \subseteq \R^n$ into $\R^m$, and if $\bm{f}'(\x) = \bm{0}$ for every $\x \in E$, prove that $\bm{f}$ is constant in $E$.
\end{pbox}

\begin{proof}
    Choose some point $\x_0 \in E$, and define the set
    \[
        U = \{\x \in E : \bm{f}(\x) = \bm{f}(\x_0)\}.
    \]
    We claim that $U$ is open. For any $\x \in U \subseteq E$, there is some radius $\eps > 0$ such that $B_{\eps}(\x) \subseteq E$. Since $\bm{f}'$ is zero on the convex set $B_{\eps}(\x)$ and $\bm{f}(\x) = \bm{f}(\x_0)$, then $\bm{f}$ is constantly $\bm{f}(\x_0)$ on the open ball. Hence, $B_{\eps}{(\x)} \subseteq U$, and we conclude that $U$ is open.

    Now define the set
    \[
        V = E \setminus U = \{\x \in E : \bm{f}(\x) \ne \bm{f}(\x_0)\}.
    \]
    Again, we claim that $V$ is open, and give a similar argument. Any point $\x \in V$ must have some open ball centered at $\x$ and contained in $E$, on which $\bm{f}$ is constantly $\bm{f}(\x)$. In particular, $\bm{f}(\x) \ne \bm{f}(\x_0)$, implying that the open ball is contained in $V$, which tells us that $V$ is open.

    Since we now have disjoint open sets $U$ and $V$ with $U$ nonempty such that $E = U \cup V$ (and $U$ is nonempty), then in fact $V$ must be empty. Hence, $\bm{f}$ is constant on $E$, in particular equal to $\bm{f}(\x_0)$.

\end{proof}



\newpage
\begin{pbox}[Exercise 9.10]
    If $f$ is a real function defined in a convex open set $E \subseteq \R^n$, such that $(D_1f)(\x) = 0$ for every $\x \in E$, prove that $f(\x)$ depends only on $x_2, \dots, x_n$.
\end{pbox}

\begin{proof}
    Fix some $x_2, \dots, x_n \in \R$ and define the set
    \[
        D = \{x \in \R : (x, x_2, \dots, x_n) \in E\}.
    \]
    Then there is an injective mapping $D \to E$ given by $x \mapsto (x, x_2, \dots, x_n)$. We see that $D$ is convex (i.e., an interval in $\R$) since any pair of points $x, y \in D$ have a corresponding pair of points $(x, x_2, \dots, x_n), (y, x_2, \dots, x_n) \in E$. The segment between the pair in $E$ is contained in $E$, and the points on that segment vary only in the first coordinate. Hence, there is a corresponding segment between $x$ and $y$ in $D$, implying $D$ is convex.

    Then function the function $D \to \R$ defined by $x \mapsto f(x, x_2, \dots, x_n)$ is differentiable with derivative $(D_1f)(x, x_2, \dots, x_n) = 0$ for all $x \in D$. Then since the derivative of this function is zero on the real interval $D$, it must be constant on $D$. That is, $f$ is constant on all the points in $E$ of the form $(x, x_2, \dots, x_n)$. Hence the value of $f(\x)$ is independent of the first coordinate of $\x$.


\end{proof}

\begin{pbox}[]
    Show that the convexity of $E$ can be replaced by a weaker condition, but that some condition is required.
\end{pbox}

All that is required for the above proof is that $E$ contain every interval between pairs of points which differ only in the first coordinate. This is precisely the condition which ensures $D$ to be an interval in $\R$.


\end{document}