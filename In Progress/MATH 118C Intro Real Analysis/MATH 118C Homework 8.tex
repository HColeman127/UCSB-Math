\documentclass[12pt]{article}

% Packages
\usepackage[margin=1in]{geometry}
\usepackage{fancyhdr, parskip}
\usepackage{amsmath, amsthm, amssymb}

% Page Style
\fancypagestyle{plain}{
    \fancyhf{}
    \renewcommand{\headrulewidth}{0pt}
    \renewcommand{\footrulewidth}{0pt}
    \fancyfoot[R]{\thepage}
}
\pagestyle{plain}

% Problem Box
\setlength{\fboxsep}{4pt}
\newsavebox{\savefullbox}
\newenvironment{fullbox}{\begin{lrbox}{\savefullbox}\begin{minipage}{\dimexpr\textwidth-2\fboxsep\relax}}{\end{minipage}\end{lrbox}\begin{center}\framebox[\textwidth]{\usebox{\savefullbox}}\end{center}}
\newenvironment{pbox}[1][]{\begin{fullbox}\ifx#1\empty\else\paragraph{#1}\fi}{\end{fullbox}}

% Theorem Environments
%\theoremstyle{definition}
%\newtheorem{proposition}{Proposition}
%\newtheorem{lemma}{Lemma}

% Options
%\allowdisplaybreaks
%\addtolength{\jot}{4pt}


% Default Commands
\newcommand{\isp}[1]{\quad\text{#1}\quad}
\newcommand{\N}{\mathbb{N}} 
\newcommand{\Z}{\mathbb{Z}}
\newcommand{\Q}{\mathbb{Q}}
\newcommand{\R}{\mathbb{R}}
\newcommand{\C}{\mathbb{C}}
\newcommand{\eps}{\varepsilon}
\renewcommand{\phi}{\varphi}
\renewcommand{\emptyset}{\varnothing}
\newcommand{\<}{\langle}
\renewcommand{\>}{\rangle}
\newcommand{\isom}{\cong}
\newcommand{\eqc}{\overline}
\newcommand{\clo}{\overline}

% Extra Commands
\newcommand{\dd}[1]{\,\mathrm{d}#1}

\newcommand{\of}[1]{\!\left(#1\right)}

\newcommand{\longlimit}[2]{
    {
        \def\arraystretch{0}
        \begin{array}[t]{@{}l@{}}
            \displaystyle{#2} \\[8.5pt] \scriptstyle{#1}
        \end{array}
    }
}

\newcommand{\intw}[1]{
    \int_{\rlap{$\scriptstyle{#1}$}\hspace{1ex}}
}

% Document Info
\fancypagestyle{title}{
    \renewcommand{\headrulewidth}{0.4pt}
    \setlength{\headheight}{15pt}
    \fancyhead[R]{Harry Coleman}
    \fancyhead[L]{MATH 118C Homework 8}
    \fancyhead[C]{June 6, 2021}
}

% Begin Document
\begin{document}
\thispagestyle{title}

\begin{pbox}[1 Exercise 11.1]
    If $f \geq 0$ and $\int_E f \dd{\mu} = 0$, prove that $f(x)=0$ almost everywhere on $E$. \textit{Hint:} Let $E_n$ be the subset of $E$ on which $f(x) > 1/n$. Write $A = \bigcup E_n$. Then $\mu(A) = 0$ if and only if $\mu(E_n) = 0$ for every $n$.
\end{pbox}

\begin{proof}
    Since $f$ is a measurable function, $\{x : f(x) \geq \tfrac{1}{n}\}$ is a measurable set, for all $n \in \N$. Then the intersection of measurable sets
    \[
        E_n 
            = \{x : f(x) \geq \tfrac{1}{n}\} \cap E
            = \{x \in E : f(x) \geq \tfrac{1}{n}\}
    \]
    is measurable, and the countable union
    \[
        A
            = \bigcup_{n=1}^{\infty} E_n
            = \{x \in E : f(x) > 0\}
    \]
    is the measurable subset of $E$ on which $f$ is nonzero. For each $n \in \N$, Theorem 11.24 (i.e., additivity of integration over measurable sets) and Remark 11.23(b) give us
    \[
        \int_E f \dd{\mu}
            = \int_{E_n} f \dd{\mu} + \int_{E \setminus E_n} f \dd{\mu}
            \geq \frac{1}{n} \mu(E_n) + 0,
    \]

    
    so
    \[
        \mu(E_n) \leq n\int_E f \dd{\mu} = n \cdot 0 = 0.
    \]
    Therefore, $\mu(E_n) = 0$, and we obtain
    \[
        \mu(A) 
            \leq \sum_{n=1}^{\infty} \mu(E_n)
            = \sum_{n=1}^{\infty} 0
            = 0.
    \]
    Hence $\mu(A) = 0$, which is to say that $f$ is nonzero on a measure zero subset of $E$, i.e., $f$ is zero almost everywhere on $E$.

\end{proof}



\newpage
\begin{pbox}[2 Exercise 11.2]
    If $\int_A f \dd{\mu} = 0$ for every measurable subset $A$ of a measurable set $E$, then $f(x) = 0$ almost everywhere on $E$.
\end{pbox}

\begin{proof}
    Since $f$ is measurable, $\{x : f(x) \geq 0\}$ is a measurable. Therefore, the intersection
    \[
        A = \{x : f(x) \geq 0\} \cap E = \{x \in E : f(x) \geq 0\}
    \]
    is a measurable subset of $E$, implying $\int_A f \dd{\mu} = 0$. Moreover, $A$ is precisely subset of $E$ on which $f$ is nonnegative, so we may apply Exercise 1 to deduce that $f$ is zero almost everywhere on $A$.

    Similarly,
    \[
        B = \{x : f(x) \leq 0\} \cap E = \{x \in E : f(x) \leq 0\}
    \]
    is the measurable subset of $E$ on which $f$ is nonpositve. Then $-f$ is a measurable function, and is nonnegative on $B$. Remark 11.23(d) gives us
    \[
        \int_B -f \dd{\mu} = - \int_B f \dd{\mu} = -1 \cdot 0 = 0.
    \]
    From Exercise 1, $-f$ is zero almost everywhere on $B$ and, therefore, so is $f$. 
    
    By construction, $E = A \cup B$, so the subset of $E$ on which $f$ is nonzero is given by
    \[
        E \setminus f^{-1}(0) = (A \setminus f^{-1}(0)) \cup (B \setminus f^{-1}(0)).
    \]
    (We may consider a singleton as a closed interval, i.e., $\{0\} = [0, 0]$, so the preimage under $f$ is a measurable set.) Taking the measure, we find
    \[
        \mu\of{E \setminus f^{-1}(0)}
            = \mu\of{A \setminus f^{-1}(0)} + \mu\of{B \setminus f^{-1}(0)}
            = 0 + 0
            = 0.
    \]
    In other words, $f$ is zero almost everywhere on $E$.

\end{proof}


\newpage
\begin{pbox}[3 Exercise 11.3]
    If $\{f_n\}$ is a sequence of measurable functions, prove that the set of points $x$ at which $\{f_n(x)\}$ converges is measurable.
\end{pbox}

\begin{proof}
    By Theorem 11.17, the functions
    \[
        \underline{f}(x) = \liminf_{n \to \infty} f_n(x)
        \isp{and}    
        \overline{f}(x) = \limsup_{n \to \infty} f_n(x) 
    \]
    are measurable, so the intersection 
    \[
        E = \{x : -\infty < \underline{f}(x)\} \cap \{x : \overline{f}(x) < +\infty\}
    \]
    is a measurable set. Explicitly, $E$ is the set of points $x$ for which both the pointwise inferior and superior limits of $\{f_n(x)\}$ are finite. It can then be seen that $\{f_n(x)\}$ converges if and only if $x \in E$ and  $\underline{f}(x) = \overline{f}(x)$; in which case, $f_n(x) \to \underline{f}(x) = \overline{f}(x)$.

    Define the function $g = \underline{f} - \overline{f}$ on $E$, which is measurable on $E$, by Theorem 11.18. Then the preimage
    \[
        g^{-1}([0, 0]) = \{x \in E : g(x) = 0\} = \{x \in E : \underline{f} = \overline{f}\}
    \] 
    is measurable and is precisely the set of points $x$ at which $\{f_n(x)\}$ converges.

\end{proof}




\newpage
\begin{pbox}[4 Exercise 11.4]
    If $f \in L(\mu)$ on $E$ and $g$ is bounded and measurable on $E$, then $fg \in L(\mu)$ on $E$.
\end{pbox}

\begin{proof}
    By Theorem 11.18, $fg$ is measurable on $E$, so we must check that its integral over $E$ is finite. Define $M = \sup_E |g|$, so $-M \leq g \leq M$ on $E$. Then $\pm Mf$ are integrable functions such that, on $E$, we have
    \[
        -Mf \leq fg \leq Mf.
    \]
    Applying Remarks 11.23(b) and (d), we obtain
    \[
        -M\int_M f \dd{\mu} 
            = \int_E -Mf \dd{\mu} 
            \leq \int_E fg \dd{\mu}
            \leq \int_E Mf \dd{\mu}
            = M\int_E f \dd{\mu}.
    \]
    Since $M$ and $\int_E f \dd{\mu}$ is finite, then so is $\int_E fg \dd{\mu}$, hence $fg \in L(\mu)$ on $E$.

\end{proof}


\newpage
\begin{pbox}[5 Exercise 11.5]
    Put
    \[
        \begin{array}{r@{\,}lr}
            g(x)        &= \left\{\begin{array}{@{}l} 0 \\ 1 \end{array}\right. 
                            & \begin{array}{r@{}} 
                                (0\leq x \leq \frac{1}{2}),\\
                                (\frac{1}{2} < x \leq 1),
                            \end{array} \\
            f_{2k}(x)   &= g(x) &  (0 \leq x \leq 1),\\
            f_{2k+1}(x) &= g(1-x) & \hfill (0 \leq x \leq 1).
        \end{array}
    \]
    
    Show that
    \[\liminf_{n \to \infty} f_n(x) = 0 \qquad (0\leq x\leq 1),\]
    but
    \[\int_0^1 f_n(x) \dd{x} = \frac{1}{2}.\]
    [Compare with (77).]
\end{pbox}

\begin{proof}
    For all $n \in \N$, $f_n \geq 0$, i.e., $\liminf f_n(x) \geq 0$ for all $x \in [0, 1]$. So for a given $x$, we have equality if there exist infinitely many $n \in \N$ such that $f_n(x) = 0$. If $x \in [0, \frac{1}{2}]$, there are infinitely many even $n \in \N$, so that $f_n(x) = 0$. If $x \in (\frac{1}{2}, 1]$, there are infinitely many odd $n \in \N$, so that $f_n(x) = 0$. Hence, $\liminf f_n(x) = 0$ for all $x \in [0, 1]$.

    If $n \in \N$ is even, then $f_n = \chi_{[0, \frac{1}{2}]}$ (indicator function), so
    \[
        \int_0^1 f_n(x) \dd{x}
            = \int_{[0, 1]} \chi_{[0, \frac{1}{2}]} \dd{m}
            = m\of{[0, \tfrac{1}{2})}
            = \frac{1}{2} - 0
            = \frac{1}{2}.
    \]
    If $n \in \N$ is odd, then $f_n = \chi_{(\frac{1}{2}, 1]}$, so
    \[
        \int_0^1 f_n(x) \dd{x}
            = \int_{[0, 1]} \chi_{(\frac{1}{2}, 1]} \dd{m}
            = m\of{(\tfrac{1}{2}, 1]}
            = 1 - \frac{1}{2}
            = \frac{1}{2}.
    \]

\end{proof}

\end{document}