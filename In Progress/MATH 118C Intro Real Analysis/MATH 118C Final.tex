\documentclass[12pt]{article}

% Packages
\usepackage[margin=1in]{geometry}
\usepackage{fancyhdr, parskip}
\usepackage{amsmath, amsthm, amssymb}

% Page Style
\fancypagestyle{plain}{
    \fancyhf{}
    \renewcommand{\headrulewidth}{0pt}
    \renewcommand{\footrulewidth}{0pt}
    \fancyfoot[R]{\thepage}
}
\pagestyle{plain}

% Problem Box
\setlength{\fboxsep}{4pt}
\newsavebox{\savefullbox}
\newenvironment{fullbox}{\begin{lrbox}{\savefullbox}\begin{minipage}{\dimexpr\textwidth-2\fboxsep\relax}}{\end{minipage}\end{lrbox}\begin{center}\framebox[\textwidth]{\usebox{\savefullbox}}\end{center}}
\newenvironment{pbox}[1][]{\begin{fullbox}\ifx#1\empty\else\paragraph{#1}\fi}{\end{fullbox}}

% Theorem Environments
%\theoremstyle{definition}
%\newtheorem{proposition}{Proposition}
%\newtheorem{lemma}{Lemma}

% Options
%\allowdisplaybreaks
%\addtolength{\jot}{4pt}

% Default Commands
\newcommand{\isp}[1]{\quad\text{#1}\quad}
\newcommand{\N}{\mathbb{N}} 
\newcommand{\Z}{\mathbb{Z}}
\newcommand{\Q}{\mathbb{Q}}
\newcommand{\R}{\mathbb{R}}
\newcommand{\C}{\mathbb{C}}
\newcommand{\eps}{\varepsilon}
\renewcommand{\phi}{\varphi}
\renewcommand{\emptyset}{\varnothing}
\newcommand{\<}{\langle}
\renewcommand{\>}{\rangle}
\newcommand{\isom}{\cong}
\newcommand{\eqc}{\overline}
\newcommand{\clo}{\overline}

% Extra Commands
\newcommand{\dd}[1]{\,\mathrm{d}#1}
\newcommand{\DD}[1]{\,\mathrm{D}#1}
\newcommand{\pdv}[1]{\frac{\partial}{\partial #1}}


\newcommand{\mat}[1]{\begin{bmatrix}#1\end{bmatrix}}
\newcommand{\mdet}[1]{\begin{vmatrix}#1\end{vmatrix}}
\newcommand{\jac}[2]{\frac{\partial(#1)}{\partial(#2)}}

\newcommand{\of}[1]{\!\left(#1\right)}

\newcommand{\RR}{\mathcal{R}}
\newcommand{\EE}{\mathcal{E}}
\newcommand{\MM}{\mathfrak{M}}

\newcommand{\bd}{\partial}

% Document Info
\fancypagestyle{title}{
    \renewcommand{\headrulewidth}{0.4pt}
    \setlength{\headheight}{15pt}
    \fancyhead[R]{Harry Coleman}
    \fancyhead[L]{MATH 118C Final}
    \fancyhead[C]{June 8, 2021}
}

% Begin Document
\begin{document}
\thispagestyle{title}

This one is bad. Don't look please.

\newpage
\begin{pbox}[1]
    Define $r := \sqrt{x_1^2+x_2^2+x_3^2}$. Let $\omega := (\frac{1}{r})^3 (x_3 dx_1 \wedge dx_2 - x_2 dx_1 \wedge dx_3 + x_1 dx_2 \wedge dx_3)$ be a 2-form on $\mathbb R^3\setminus (0,0,0).$
\end{pbox}

\begin{pbox}[(a)]
    Show that $d\omega = 0.$
\end{pbox}

Note that
\[
    (\dd{x_i} \wedge \dd{x_j}) \wedge \dd{x_i}
        = -(\dd{x_i} \wedge \dd{x_i}) \wedge \dd{x_j}
        = 0,
\]
so if $\{i, j, k\} = \{1, 2, 3\}$, then
\begin{align*}
    \dd{\of{\frac{x_i}{r^3}}} \wedge \dd{x_j} \wedge \dd{x_k}
        &= \DD_i\frac{x_i}{r^3} \dd{x_i} \wedge \dd{x_j} \wedge \dd{x_k}
            + \DD_j\frac{x_i}{r^3} \dd{x_j} \wedge \dd{x_j} \wedge \dd{x_k}
            + \DD_k\frac{x_i}{r^3} \dd{x_k} \wedge \dd{x_j} \wedge \dd{x_k} \\
        &= \DD_i\frac{x_i}{r^3} \dd{x_i} \wedge \dd{x_j} \wedge \dd{x_k} + 0 + 0.
\end{align*}
Then
\begin{align*}
    \dd{\omega}
        &= \dd{\of{\frac{x_3}{r^3}}} \wedge \dd{x_1} \wedge \dd{x_2}
            + \dd{\of{\frac{x_2}{r^3}}} \wedge \dd{x_1} \wedge \dd{x_3}
            + \dd{\of{\frac{x_1}{r^3}}} \wedge \dd{x_2} \wedge \dd{x_3} \\
        &= \DD_3 \frac{x_3}{r^3} \dd{x_3} \wedge \dd{x_1} \wedge \dd{x_2}
            - \DD_2 \frac{x_2}{r^3} \dd{x_2} \wedge \dd{x_1} \wedge \dd{x_3}
            + \DD_1 \frac{x_1}{r^3} \dd{x_1} \wedge \dd{x_2} \wedge \dd{x_3} \\
        &= \left(\DD_1 \frac{x_1}{r^3} 
            + \DD_2 \frac{x_2}{r^3} 
            + \DD_3 \frac{x_3}{r^3}\right) \dd{x_1} \wedge \dd{x_2} \wedge \dd{x_3} \\
        &= \left(\frac{-2x_1^2 + x_2^2 + x_3^2}{r^5} 
            + \frac{x_1^2 - 2x_2^2 + x_3^2}{r^5}
            + \frac{x_1^2 + x_2^2 - 2x_3^2}{r^5}\right) \dd{x_1} \wedge \dd{x_2} \wedge \dd{x_3} \\
        &= 0 \dd{x_1} \wedge \dd{x_2} \wedge \dd{x_3} \\
        &= 0.
\end{align*}


\begin{pbox}[(b)]
    Let $B:=\{(x_1, x_2, x_3): (x_1-2)^2+x_2^2+x_3^2=3\}$ be a sphere in $\mathbb R^3,$ find the integral $\int_B \omega.$
\end{pbox}





\newpage
\begin{pbox}[2]
    
\end{pbox}

Note that $y = Ix = x$ and $\det(I)$, so
\[
    \dd{y} = \dd{x} = \det(I)\dd{x}.
\]



The wedge product satisfies the multilinearity and alternating of the determinant, which uniquely characterizes it


\newpage
\begin{pbox}[3]
    Let $D$ be the closed unit disk in $\mathbb R^2$ and $f$ be a continuous function on $D.$ Show that for any $\epsilon >0,$ there exists a number $n$ and functions $f_1, f_2, \ldots, f_n$ such that $f=f_1+\ldots+f_n$ on $D$ and the support of $f_i$ has Lebesgue measure less than $\epsilon,$ for any $i =1, \ldots, n.$ State any theorem you use.
\end{pbox}

Given $\eps > 0$, choose $n \in \N$ with $n > 4\pi/\eps$. For $k = 0, \dots, n - 1$, define the open annulus
\[
    A_k = \{x \in \R^2 : \tfrac{k - 1}{n} < |x| <  \tfrac{k + 1}{n} \},
\]
Area in $\R^2$ coincides with the Lebesgue measure. By construction, for all $k$,
\[
    m(A_k)
        \leq m(A_{n-1}))
        = \pi\left((\tfrac{n}{n})^2 - (\tfrac{n-2}{n})^2\right)
        = \pi\left(\tfrac{4}{n} - \tfrac{4}{n^2}\right)
        = \tfrac{4\pi}{n}(1 - \tfrac{1}{n})
        \leq \tfrac{4\pi}{n}
        < \eps.
\]

The collection $\{A_k\}_{k = 0}^{n}$ forms an open cover of the unit disc in $\R^2$. Then there exists a partition of unity $\{\psi_j\}_{j=1}^{m}$ with each $\psi_j$ having its support contained in some $A_k$. Define $f_j = \psi_j f$, then  $f = f_1 + \dots + f_m$ and each $f_j$ has its support in some $A_k$, in particular, the support of $f_j$ has Lebesgue measure at most $m(A_k) < \eps$.


\newpage
\begin{pbox}[5]
    Prove that a subset $E$ of $\mathbb R^n$ is Lebesgue measurable if and only if for any $\epsilon >0,$ there exists an open set $U\subset \mathbb R^n$ such that $E\subset U$ and $m(U\setminus E)<\epsilon.$
\end{pbox}

We know that $m$ is regular on Lebesgue measurable sets, i.e., there exists an open set $U \subseteq \R^n$ containing $E$ such that $m(U \setminus E) < \eps$.

If such a $U$ exists, then $U$ and $U \setminus E$ are measurable, so $U \setminus (U \setminus E) = E$ is measurable.




\newpage
\begin{pbox}[5]
    Let $\{f_n\}$ be a sequence of measurable functions and define $f:=\liminf_n f_n.$ Is $f$ measurable? If yes, justify your answer. If no, give a counterexample.
\end{pbox}

Yes.

Define $g_n = -f_n$ measurable for all $n \in \N$, then we have that $g = \limsup_n g_n$ is measurable function. Therefore, so is $f = \liminf_n f_n = -\limsup_n g_n = -g$.

\newpage
\begin{pbox}[6]
    Let $\{f_n\}$ be a uniformly convergent and uniformly bounded sequence of Lebesgue integrable functions on $\mathbb R^1$ and let $f:=\lim_n f_n$ be the limit. Is it true that $$\lim_n \int_{\mathbb R^1} f_n dm = \int_{\mathbb R^1} f dm ?$$ If yes, justify your answer. If no, give a counterexample. All integrals are Lebesgue integrals.
\end{pbox}

No
\[
    f_n(x) = \begin{cases}
        1/n & 0 \leq x \leq n, \\
        0 &\text{otherwise}.
    \end{cases}
\]
Then $f_n \to 0$ uniformly on $\R^1$ and uniformly bounded by $1$. But $\int_{\R^1} f_n = 1$ and $\int_{\R^1} 0 = 0$.





\end{document}