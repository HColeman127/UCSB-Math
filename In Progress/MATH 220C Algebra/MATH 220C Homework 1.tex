\documentclass[12pt]{article}

% Packages
\usepackage[margin=1in]{geometry}
\usepackage{fancyhdr, parskip}
\usepackage{amsmath, amsthm, amssymb}
\usepackage{tikz, tikz-cd}

% Page Style
\makeatletter
\fancypagestyle{title}{
    \renewcommand{\headrulewidth}{0.4pt}
    \setlength{\headheight}{15pt}
    \fancyhead[R]{\@author}
    \fancyhead[L]{\@title}
    \fancyhead[C]{\@date}
}
\makeatother
\renewcommand{\maketitle}{\thispagestyle{title}}
\fancypagestyle{plain}{
    \fancyhf{}
    \renewcommand{\headrulewidth}{0pt}
    \renewcommand{\footrulewidth}{0pt}
    \fancyfoot[R]{\thepage}
}
\pagestyle{plain}

% Problem Box
\setlength{\fboxsep}{4pt}
\newlength{\myparskip}
\setlength{\myparskip}{\parskip}
\newsavebox{\savefullbox}
\newenvironment{fullbox}{\begin{lrbox}{\savefullbox}\begin{minipage}{\dimexpr\textwidth-2\fboxsep\relax}\setlength{\parskip}{\myparskip}}{\end{minipage}\end{lrbox}\framebox[\textwidth]{\usebox{\savefullbox}}}
\newenvironment{pbox}[1][]{\begin{fullbox}\def\temp{#1}\ifx\temp\empty\else\paragraph{#1}\phantom{}\fi}{\end{fullbox}}

% Theorem Environments
\theoremstyle{definition}
\newtheorem{lemma}{Lemma}

% Tikz Environments
\newenvironment{drawing}{\begin{center}\begin{tikzpicture}}{\end{tikzpicture}\end{center}}
% \tikzcdset{row sep/normal=0pt}
\newenvironment{cd}{\begin{center}\begin{tikzcd}}{\end{tikzcd}\end{center}}

% Default Commands
\newcommand{\isp}[1]{\quad\text{#1}\quad}
\newcommand{\N}{\mathbb{N}} 
\newcommand{\Z}{\mathbb{Z}}
\newcommand{\Q}{\mathbb{Q}}
\newcommand{\R}{\mathbb{R}}
\newcommand{\C}{\mathbb{C}}
\newcommand{\A}{\mathbb{A}}
\renewcommand{\P}{\mathbb{P}}
\newcommand{\eps}{\varepsilon}
\renewcommand{\phi}{\varphi}
\renewcommand{\emptyset}{\varnothing}
\newcommand{\<}{\langle}
\renewcommand{\>}{\rangle}
\newcommand{\iso}{\cong}
\newcommand{\eqc}{\overline}
\newcommand{\clo}{\overline}
\newcommand{\seq}{\subseteq}
\newcommand{\teq}{\trianglelefteq}
\DeclareMathOperator{\id}{id}
\DeclareMathOperator{\im}{im}
\newcommand{\inc}{\hookrightarrow}

% Extra Commands


% Document
\begin{document}
\title{MATH 220C Homework 1}
\author{Harry Coleman}
\date{April 20, 2022}
\maketitle

\begin{pbox}[1]
    Let $F \seq K$ be a field extensions, $F[X] \seq K[X]$ the corresponding polynomial rings, and $f, g \in F[X]$ nonzero polynomials.
    Show that $g \mid f$ in $F[X]$ if and only if $g \mid f$ in $K[X]$.
\end{pbox}

\begin{proof}
    Since $F[X]$ is a subring of $K[X]$, divisibility in $F[X]$ is simply a special case of divisibility in $K[X]$.

    Suppose $g \mid f$ in $K[X]$, i.e., there exists $h \in K[X]$ such that $f = gh$ in $K[X]$.
    The result is trivial if any of the three polynomials is zero, so we assume all are nonzero.
    Suppose $a, b, c$ are the leading coefficients of $f, g, h$, respectively.
    We know that $a, b \in F$ and $c \in K$.
    Since the leading coefficient of the product of polynomials is the product of the leading coefficients, we have $a = bc$, which implies $c = b^{-1}a \in F$.
    That is, the leading coefficient of $h$ is in $F$.

    We prove, by induction on the degree of $h$, that all the coefficients of $h$ must be in $F$.
    For the base case, when $\deg h = 0$, we have $h = c \in F$.
    For the inductive step, assume that the result holds whenever the degree of $h$ is less than some $n > 0$, i.e., the following holds:
    \[
        f, g \in F[X],\; h \in K[X],\; f = gh \text{ in } K[X],\; \deg h < n \quad\implies\quad h \in F[X].
    \]
    Assume $f, g, h$ are as in the hypothesis, but $\deg h = n$.
    Write $h = cX^n + h_0$, where $cX^n$ is the leading term of $h$ and $h_0$ is the sum of the remaining terms.
    In particular, $h_0 \in K[X]$ with $\deg h_0 < \deg h = n$.
    We now have
    \[
        f = gh = g(cX^n + h_0) = cX^ng + gh_0,
    \]
    then define
    \[
        \tilde{f} = f - cX^ng = gh_0.
    \]
    We have shown above that $c \in F$, which implies $\tilde{f} \in F[X]$.
    Applying the inductive hypothesis to $\tilde{f}, g, h_0$, we conclude that $h_0 \in F[X]$.
    Since the coefficients of $h$ consist of its leading coefficient, $c$, and the coefficients of $h_0$, we conclude all the coefficients of $h$ are in $F$, hence $h \in F[X]$.
\end{proof}


\newpage
\begin{pbox}[2]
    Let $a := \sqrt[4]{3} \in \R_{>0}$.
    Show that
\end{pbox}

\begin{pbox}[(a)]
    Neither $a$ nor $i$ lies in $\Q(ai)$.
\end{pbox}

\begin{proof}
    We first note that it suffices to prove that $a \notin \Q(ai)$, since $i \in \Q(ai)$ implies $a = ai \cdot i^{-1} \in \Q(ai)$. 

    Denote the polynomial $f = X^4 - 3 \in \Q[X]$, which is irreducible by Eisenstein's criterion.
    Since $f$ is monic and irreducible with $ai$ as a root, it must be the minimal polynomial of $ai$ over $\Q$.
    Therefore,
    \[
        [\Q(ai) : \Q] = \deg p_{ai, \Q} = \deg f = 4.
    \]
    By the same argument, $[\Q(a) : \Q] = 4$.

    Assume for contradiction that $a \in \Q(ai)$.
    Then we have a tower $\Q \seq \Q(a) \seq \Q(ai)$, so the tower rule gives us
    \[
        [\Q(ai) : \Q(a)]
            = \frac{[\Q(ai) : \Q]}{[\Q(a) : \Q]}
            = \frac{4}{4}
            = 1.
    \]
    It follows that $\Q(a) = \Q(ai)$, but this is a contradiction since the former consists entirely of real elements, while the latter contains the imaginary element $ai$.
    We conclude that $a$ does not lie in $\Q(ai)$.

    Note that $i \in \Q(ai)$ implies $a = ai \cdot i^{-1} \in \Q(ai)$, so the contrapositive tells us $i \notin \Q(ai)$.
\end{proof}

\begin{pbox}[(b)]
    The minimal polynomial of $a$ over $\Q(i)$ is $X^4 - 3$.
\end{pbox}

\begin{proof}
    Since $a$ is a root of $X^4 - 3$, the minimal polynomial of $a$ over $\Q(i)$ divides $X^4 - 3$, so
    \[
        \deg p_{a, \Q(i)} \leq \deg(X^4 - 3) = 4.
    \]
    It remains to prove that $\deg p_{a, \Q(i)} = 4$.

    The complex roots of $X^2 + 1$ are $\pm i$; neither is real, so the polynomial is irreducible over $\R$.
    In particular, $X^2 + 1$ is irreducible over $\Q(a)$, since $\Q(a)$ is contained in $\R$.
    Therefore, $X^2 + 1$ is the minimal polynomial of $i$ over $\Q(a)$ and we calculate
    \[
        [\Q(a, i) : \Q(a)]
            = \deg p_{i, \Q(a)}
            = \deg(X^2 + 1)
            = 2.
    \]
    By the same argument, $[\Q(i) : \Q] = 2$.

    Applying the tower rule to $\Q \seq \Q(a) \seq \Q(a, i)$, we obtain
    \[
        [\Q(a, i) : \Q]
            = [\Q(a, i) : \Q(a)][\Q(a) : \Q]
            = 2 \cdot 4
            = 8.
    \]
    Lastly, applying the tower rule to $\Q \seq \Q(i) \seq \Q(a, i)$ gives
    \[
        \deg p_{a, \Q(i)}
            = [\Q(a, i) : \Q(i)]
            = \frac{[\Q(a, i) : \Q]}{[\Q(i) : \Q]}
            = \frac{8}{2}
            = 4.
    \]
    We conclude that $X^4 - 3$ is a monic polynomial divisible by and having the same degree as the minimal polynomial $p_{a, \Q(i)}$, so in fact $p_{a, \Q(i)} = X^4 - 3$.
\end{proof}



\newpage
\begin{pbox}[3]
    Let $F \seq K = F(a)$ be a simple algebraic field extension such that the minimal polynomial of $a$ over $F$ has degree $p^m$ for some odd prime $p$ and some $m > 0$.
    Show that $F(a^i) = K$ for all $i = 2, \dots, p - 1$.
\end{pbox}

\begin{proof}
    Fix some $i \in \{2, \dots, p-1\}$.
    Since $a^i \in K$, we have a tower $F \seq F(a^i) \seq K$.
    Then
    \[
        p^m
            = \deg p_{a, F}
            = [K : F]
            = [K : F(a^i)][F(a^i) : F].
    \]
    Since $p$ is prime, it follows that
    \[
        \deg p_{a^i, F} = [F(a^i) : F] = p^n,
    \]
    for some $0 < n \leq m$.
    Define the polynomial $f = p_{a^i, F}(X^i) \in F[X]$, which has degree $ip^n$.
    Then $a$ is a root of $f$, so $p_{a, F}$ divides $f$.
    In particular,
    \[
        p^m = \deg p_{a, F} \leq \deg f = ip^n < p^{n + 1},
    \]
    so $m \leq n$, hence $n = m$.
    We then calculate
    \[
        [K : F(a^i)]
            = \frac{[K : F]}{[F(a^i) : F]}
            = \frac{p^m}{p^n}
            = 1,
    \]
    which implies $F(a^i) = K$.

\end{proof}


\newpage

\begin{pbox}[4]
    
\end{pbox}

\begin{lemma}
    For nonnegative integers $M, N, P$ with $M \leq N$, we have
    \[
        \frac{N!}{M!} \leq \frac{(N + P)!}{(M + P)!}.
    \]
\end{lemma}

\begin{proof}
    We perform induction on $P$. For the base case, note that $(M + 1)/(N + 1) \leq 1$, so
    \[
        \frac{N!}{M!}
            = \frac{(N + 1)!/(N + 1)}{(M + 1)!/(M + 1)}
            = \frac{(N + 1)!}{(M + 1)!} \cdot \frac{M + 1}{N + 1}
            \leq \frac{(N + 1)!}{(M + 1)!}.
    \]
    Assuming the result holds for $P - 1$, we find
    \[
        \frac{N!}{M!}
            \leq \frac{(N + P - 1)!}{(M + P - 1)!}
            \leq \frac{(N + P)!}{(M + P)!},
    \]
    where the second inequality is an application of the base case.
\end{proof}

\begin{pbox}[]
    Let $F \seq K$ be a field extension, $f \in F[X]$ a polynomial with degree $n > 0$, and $r$ the number of distinct roots of $f$ in $K$.
    Assume that $K$ is a splitting field for $f$ over $F$.
    Show that $[K : F] \leq n!/(n - r)!$.
\end{pbox}

\begin{proof}
    Write $K = F(a_1, \dots, a_r)$, where the $a_i$'s are the distinct roots of $f$ in $K$.
    Let $m_i \in \Z_{>0}$ be the multiplicity of $a_i$ in $f$, then in $K[X]$ we can write
    \[
        f = \prod_{i=1}^{r} (X - a_i)^{m_i}.
    \]

    To prove the inequality, we perform induction on $r$.

    When $r = 1$, we have $f = (X - a)^{n}$ with $a = a_1$.
    Then $a$ is a root of $f$, so the minimal polynomial of $a$ over $F$ divides $f$, thus
    \[
        [K : F]
            = [F(a) : F]
            = \deg p_{a, F}
            \leq \deg f
            = n
            = \frac{n!}{(n - 1)!}.
    \]
    For the inductive step, assume the result hold in any case that the number of distinct roots is at most than $r - 1$.
    Define the polynomial
    \[
        g = \prod_{i=1}^{r-1} (X - a_i)^{m_i} \in K[X],
    \]
    then $f = (X - a_r)^{m_r} g$ in $K[X]$.
    Both $f$ and $(X - a_r)^{m_r}$ are in $F(a_r)[X]$, so Problem 1 tells us that $g$ is also in $F(a_r)[X]$.
    Moreover, $g$ is of degree $n - m_r$ and has $r - 1$ distinct roots in $K$, so the inductive hypothesis gives us
    \[
        [K : F(a_r)]
            \leq \frac{(n - m_r)!}{(n - m_r - (r - 1))!}
            \leq \frac{(n - 1)!}{(n - r)!},
    \]
    where the second inequality is an application of Lemma 1 with $P = m_r - 1$.
    Similar to the base case, the minimal polynomial of $a_r$ over $F$ divides $f$, so
    \[
        [F(a_r) : F] = \deg p_{a_r, F} \leq \deg f = n.
    \]
    Combining these inequalities with the tower rule, we obtain
    \[
        [K : F]
            = [K : F(a_r)][F(a_r) : F]
            \leq \frac{(n - 1)!}{(n - r)!} \cdot n
            = \frac{n!}{(n - r)!}.
    \]
\end{proof}



\end{document}