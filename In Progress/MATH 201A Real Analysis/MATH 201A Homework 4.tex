\documentclass[12pt]{article}

% Packages
\usepackage[margin=1in]{geometry}
\usepackage{fancyhdr, parskip}
\usepackage{amsmath, amsthm, amssymb}
\usepackage{tikz, tikz-cd}

% Page Style
\makeatletter
\fancypagestyle{title}{
    \renewcommand{\headrulewidth}{0.4pt}
    \setlength{\headheight}{15pt}
    \fancyhead[R]{\@author}
    \fancyhead[L]{\@title}
    \fancyhead[C]{\@date}
}
\makeatother
\renewcommand{\maketitle}{\thispagestyle{title}}
\fancypagestyle{plain}{
    \fancyhf{}
    \renewcommand{\headrulewidth}{0pt}
    \renewcommand{\footrulewidth}{0pt}
    \fancyfoot[R]{\thepage}
}
\pagestyle{plain}

% Problem Box
\setlength{\fboxsep}{4pt}
\newlength{\myparskip}
\setlength{\myparskip}{\parskip}
\newsavebox{\savefullbox}
\newenvironment{fullbox}{\begin{lrbox}{\savefullbox}\begin{minipage}{\dimexpr\textwidth-2\fboxsep\relax}\setlength{\parskip}{\myparskip}}{\end{minipage}\end{lrbox}\framebox[\textwidth]{\usebox{\savefullbox}}}
\newenvironment{pbox}[1][]{\begin{fullbox}\ifx#1\empty\else\paragraph{#1}\fi}{\end{fullbox}}

% Tikz Environments
\newenvironment{drawing}{\begin{center}\begin{tikzpicture}}{\end{tikzpicture}\end{center}}
\tikzcdset{row sep/normal=0pt}
\newenvironment{cd}{\begin{center}\begin{tikzcd}}{\end{tikzcd}\end{center}}

% Default Commands
\newcommand{\isp}[1]{\quad\text{#1}\quad}
\newcommand{\N}{\mathbb{N}} 
\newcommand{\Z}{\mathbb{Z}}
\newcommand{\Q}{\mathbb{Q}}
\newcommand{\R}{\mathbb{R}}
\newcommand{\C}{\mathbb{C}}
\newcommand{\A}{\mathbb{A}}
\renewcommand{\P}{\mathbb{P}}
\newcommand{\eps}{\varepsilon}
\renewcommand{\phi}{\varphi}
\renewcommand{\emptyset}{\varnothing}
\newcommand{\<}{\langle}
\renewcommand{\>}{\rangle}
\newcommand{\isom}{\cong}
\newcommand{\eqc}{\overline}
\newcommand{\clo}{\overline}
\newcommand{\teq}{\trianglelefteq}
\DeclareMathOperator{\id}{id}
\DeclareMathOperator{\im}{im}

% Extra Commands


% Document
\begin{document}
\title{MATH 201A Homework 4}
\author{Harry Coleman}
\date{November 18, 2021}
\maketitle

\begin{pbox}[1]
    Let $X$ be a nonempty topological space and let $\mu$ be a measure on $X$. Prove that if the functions $f_n : X \to [-\infty, +\infty]$ are $\mu$-measurable for $n = 1, 2, \dots$, then the set
    \[
        A = \{x \in X : \lim_{n \to \infty} f_n(x) \text{ exists}\}
    \]
    is $\mu$-measurable.
\end{pbox}


\begin{proof}
    Consider the function know that the functions $F: X \to [-\infty, +\infty]$ defined by
    \[
        F(x) = \liminf_{n \to \infty} f_n(x) - \limsup_{n \to \infty} f_n(x)
    \]
    is $\mu$-measurable. Note that the limit of $f_n(x)$ exists if and only if the limit infimum and limit supremum are equal, i.e., $A = F^{-1}(0)$. Since the singleton $\{0\} \in [-\infty, +\infty]$ is a closed---therefore Borel---set, its preimage is $\mu$-measurable. 
\end{proof}


\newpage
\begin{pbox}[2]
    Prove that any Lebesgue-measurable function $f : \R \to \R$ that satisfies the relation
    \[
        f(x + y) = f(x) + f(y) \isp{for all} x, y \in \R,
    \]
    must be linear.
\end{pbox}

\begin{proof}
    
\end{proof}


\newpage
\begin{pbox}[3]
    Let $f : (0, 1) \to \R$ be such that for every $x \in (0, 1)$ there exists $\delta > 0$ and a Borel-measurable function $g : \R \to \R$ (both dependent on $x$), such that $f(y) = g(y)$ for all $y \in (x - \delta, x + \delta) \cap (0, 1)$. Prove that $f$ is Borel-measurable. (You can assume that $f(x) = 0$ outside the interval $(0, 1)$).
\end{pbox}

\begin{proof}
    We claim that for any closed interval $[a, b] \subseteq (0, 1)$, we can find a Borel-measurable function $g : \R \to \R$ such that $g(x) = f(x)$ for all $x \in [a, b]$.
    For each $x \in [a, b]$ we can choose a value $\delta_x > 0$ and a Borel-measurable function $g_x : \R \to \R$ be  such that $B_{\delta_x}(x) \subseteq (0, 1)$ and $g_x(y) = f(y)$ for all $y \in B_{\delta_x}(x)$.
    The collection $\{B_{\delta_x}(x)\}_{x \in [a, b]}$ forms an open cover of the compact interval $[a, b]$, so there is a finite subcover denoted by $B_{\delta_{x_i}}(x_i)$ for $i = 1, \dots, m$.

    Define the initial set $A_1 = B_{\delta_{x_1}}(x_1)$ and for $k = 2, \dots, m$, define the sets
    \[
        A_i = B_{\delta_{x_i}}(x_i) \setminus \bigcup_{j=1}^{i-1} A_j.
    \]
    Then the $A_k$'s are mutually disjoint Borel-measurable subsets of $(0, 1)$ such that
    \[
        [a, b]
            \subseteq \bigcup_{i=1}^{m} B_{\delta_{x_i}}(x_i) 
            = \bigcup_{i=1}^{m} A_i.
    \]
    Additionally, $g_{x_i}(x) = f(x)$ for all $x \in A_i$.
    We now define the function
    \[
        g = \sum_{i=1}^{m} \chi_{A_i}g_{x_i}.
    \]
    As the sum of products of Borel-measurable functions, $g$ is also Borel-measurable.
    Every point $x \in [a, b]$ is contained in exactly one $A_i$. If $x \in A_k$, then $A_k \subseteq B_{\delta_{x_k}}(x_k)$, so
    \[
        g(x)
            = \sum_{i=1}^{m} \chi_{A_i}(x)g_{x_i}(x)
            = g_{x_k}(x)
            = f(x).
    \]
    Hence, for every closed interval $[a, b] \subseteq (0, 1)$, there is a Borel-measurable function $g : \R \to \R$ that agrees with $f$ on $[a, b]$ and is zero outside $(0, 1)$.

    For each $n \in \N$ (for $n \geq 3$), we consider the closed interval $I_n = [\tfrac{1}{n}, 1 - \tfrac{1}{n}] \subseteq (0, 1)$. 
    By the above result, there is a Borel-measurable function $f_n : \R \to \R$ that agrees with $f$ on $I_n$ and is zero outside $(0, 1)$. 
    Then $f$ can be written as limit of Borel-measurable functions
    \[
        f(x) = \lim_{n \to \infty} f_n(x).
    \]
    Hence, $f$ is Borel-measurable.
\end{proof}

\newpage
\begin{pbox}[4]
    Give an example of a collection of Lebesgue-measurable nonnegative functions $\{f_\alpha\}_{\alpha \in A}$ ($f_\alpha : \R \to \R$) such that the function
    \[
        g(x) = \sup_{\alpha \in A} f_\alpha(x), \quad x \in \R
    \]
    is finite for all $x \in \R$ but $g$ is not Lebesgue-measurable. Here $A$ is a nonempty indexing set.
\end{pbox}

Let $V \subseteq \R$ be a Vitali set.
For each $v \in V$, the characteristic function $\chi_{\{v\}}$ is Lebesgue-measurable and nonnegative.
Then for all $x \in \R$,
\[
    \sup_{v \in V} \chi_{\{v\}}(x) = \chi_V(x)
\]
is clearly finite. However, $\{1\} \subseteq \R$ is a Borel set with preimage
\[
    \chi_V^{-1}(\{1\}) = V,
\]
which is not Lebesgue-measurable.


\newpage
\begin{pbox}[5]
    A function $f : \R^n \to \R$ is called lower semi-continuous at the point $x \in \R^n$ if, for any sequence $x_k \in \R^n$ with $x_k \to x$, one has
    \[
        \liminf_{k \to \infty} f(x_k) \geq f(x).
    \]
    Prove that any lower semi-continuous function on $\R^n$ is Borel-measurable.
\end{pbox}

\begin{proof}
    Let $a \in \R$ and consider the set $A = f^{-1}((a, +\infty)) \subseteq \R^n$.
    To show $f$ is Borel-measurable, it suffices to check that $A$ is Borel-measurable.
    Fix a point $x \in A$ and choose $0 < \eps < f(x) - a$.
    Then the lower semi-continuity of $f$ tells us that there is some $\delta > 0$ such that $B_{\delta}(x) \subseteq A$, hence $A$ is open---therefore Borel-measurable. 
\end{proof}

\end{document}