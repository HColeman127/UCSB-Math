\documentclass[12pt]{article}

% Packages
\usepackage[margin=1in]{geometry}
\usepackage{fancyhdr, parskip}
\usepackage{amsmath, amsthm, amssymb}

% Page Style
\fancypagestyle{plain}{
    \fancyhf{}
    \renewcommand{\headrulewidth}{0pt}
    \renewcommand{\footrulewidth}{0pt}
    \fancyfoot[R]{\thepage}
}
\pagestyle{plain}

% Problem Box
\setlength{\fboxsep}{4pt}
\newlength{\myparskip}
\setlength{\myparskip}{\parskip}
\newsavebox{\savefullbox}
\newenvironment{fullbox}{\begin{lrbox}{\savefullbox}\begin{minipage}{\dimexpr\textwidth-2\fboxsep\relax}\setlength{\parskip}{\myparskip}}{\end{minipage}\end{lrbox}\framebox[\textwidth]{\usebox{\savefullbox}}}
\newenvironment{pbox}[1][]{\begin{fullbox}\ifx#1\empty\else\paragraph{#1}\fi}{\end{fullbox}}

% Default Commands
\newcommand{\isp}[1]{\quad\text{#1}\quad}
\newcommand{\N}{\mathbb{N}} 
\newcommand{\Z}{\mathbb{Z}}
\newcommand{\Q}{\mathbb{Q}}
\newcommand{\R}{\mathbb{R}}
\newcommand{\C}{\mathbb{C}}
\newcommand{\eps}{\varepsilon}
\renewcommand{\phi}{\varphi}
\renewcommand{\emptyset}{\varnothing}
\newcommand{\<}{\langle}
\renewcommand{\>}{\rangle}
\newcommand{\isom}{\cong}
\newcommand{\eqc}{\overline}
\newcommand{\clo}{\overline}

% Extra Commands
\theoremstyle{definition}
\newtheorem{proposition}{Proposition}
\newtheorem{definition}{Definition}

\DeclareMathOperator{\dist}{dist}
\newcommand{\CC}{\mathcal{C}}
\renewcommand{\AA}{\mathcal{A}}
\newcommand{\NN}{\mathcal{N}}

\newcommand{\tbigcap}{{\textstyle\bigcap}}

\newcommand{\qty}[2][0]{
    \ifnum #1=0 (#2) \fi
    \ifnum #1=1 \big(#2\big) \fi
    \ifnum #1=2 \Big(#2\Big) \fi
    \ifnum #1=3 \bigg(#2\bigg) \fi
    \ifnum #1=4 \Bigg(#2\Bigg) \fi
}


% Document Info
\fancypagestyle{title}{
    \renewcommand{\headrulewidth}{0.4pt}
    \setlength{\headheight}{15pt}
    \fancyhead[R]{Harry Coleman}
    \fancyhead[L]{MATH 201A Homework 1}
    \fancyhead[C]{October 6, 2021}
}

% Begin Document
\begin{document}
\thispagestyle{title}

\begin{pbox}[1]
    Let $X$ be a nonempty set and let $\mu$ be a measure on $X$. We have a theorem on sequences of decreasing measurable sets that states the following: Assume $X \supseteq A_1 \supseteq A_2 \supseteq A_3 \supseteq \cdots$ are $\mu$-measurable, such that $\mu(A_1) < \infty$. Then one has
    \[
        \lim_{n\to\infty} \mu(A_n) = \mu(\tbigcap_{n=1}^{\infty} A_n).
    \]
    Prove that in this theorem the condition $\mu(A_1) < \infty$ is necessary.
\end{pbox}

\begin{proof}
    Consider $X = \Z$ with the cardinality measure: $\mu(E) = |E|$, for all $E \subseteq X$. One can check that this satisfies the necessary properties of a measure. Moreover, every subset of $\Z$ is $\mu$-measurable.
    
    For $n \in \N$, define the subset of integers $A_n = \{k \in \Z : k \geq n\}$. Then $X \supseteq A_1 \supseteq A_2 \supseteq A_3 \supseteq \cdots$ is a decreasing chain of $\mu$-measurable subsets. Additionally, we have $\mu(A_n) = \infty$, for all $n \in \N$, and $\bigcap_{n=1}^{\infty} A_n = \emptyset$. Then
    \[
        \lim_{n\to\infty} \mu(A_n) = \lim_{n\to\infty} \infty = \infty,
    \]
    but
    \[
        \mu\qty[1]{\tbigcap_{n=1}^{\infty} A_n} = \mu(\emptyset) = 0.
    \]

\end{proof}


\newpage
\begin{pbox}[2]
    Does there exist an infinite $\sigma$-algebra that has countably many elements?
\end{pbox}

No.

\begin{proof}
    Assume, for contradiction, that $\AA$ is a countably infinite $\sigma$-algebra on a set $X$.
    
    In order to derive a contradiction, we will construct an infinite collection $\NN \subseteq \AA$ of disjoint sets in $\AA$. Then an injection $\N \to \NN$ gives us a bijection $2^\N \to 2^\NN$; in particular, we deduce that $2^\NN$ is uncountable. For any $S \in 2^\NN$, there is a set $\bigcup_{E \in S} E \in \AA$, since $S \subseteq \AA$ is at most countable. Because all the sets in $\NN$ are disjoint, different choices of $S \in 2^\NN$ produce different unions, giving us an injection $2^\NN \to \AA$. However, this is a contradiction, as we assumed $\AA$ to be countable. It remains to construct the desired $\NN$.
    
    For each $x \in X$, we define the set $N_x = \bigcap_{x \in E \in \AA} E$, i.e., the intersection of all sets in $\AA$ containing the point $x$. We claim that the collection $\NN = \{N_x\}_{x \in X}$ is as desired.
    
    First, since $\AA$ contains only countably many sets, each $N_x$ is the intersection of at most countably many sets. Then, because $\AA$ is a $\sigma$-algebra, $N_x \in \AA$, so in fact $\NN \subseteq \AA$.
    
    Next, we check that the sets in $\NN$ are mutually disjoint. More specifically, we will prove that $N_x$ and $N_y$ are either equal or disjoint, for all $x, y \in X$, which would mean that $\NN$ describes a partition of $X$ into the components $N_x$.
    
    Suppose $x, y \in X$ such that $y \in N_x$. By construction, $N_x$ is the intersection of all $E \in \AA$ containing $x$. So $y \in N_x$ means that the collection of sets in $\AA$ containing $x$ is a subset of the collection of those containing $y$, therefore $N_y \subseteq N_x$. Assume, for contradiction, $N_x \nsubseteq N_y$. By the contrapositive of the first inclusion, we must have $x \notin N_y$. This means that there is some set $E \in \AA$ containing $y$ but not $x$, so $E^c \in \AA$ is a set containing $x$ but not $y$. However, this means that $N_x \subseteq E^c$ but $y \notin E^c$, implying $y \notin N_x$, This is a contradiction, so $N_x \subseteq N_y$.
    
    Lastly, we prove that $\NN$ is infinite. First, note that each set $E \in \AA$ can be written as the union $E = \bigcup_{x \in E} N_x$. This suggests $\NN$ be interpreted as the ``irreducible'' sets in $\AA$, and every other set in $\AA$ has an ``irreducible decomposition'' into sets in $\NN$. Since $\AA$ is infinite, there must be infinitely many different combinations of sets in $\NN$, which is only possible if $\NN$ itself is infinite.

\end{proof}

\newpage
\begin{pbox}[3]
    Is it true that if $\mu$ is a Borel measure on a nonempty set $X$, then for any sets $A, B \subset X$ with $\dist(A, B) > 0$, one has
    \[
        \mu(A \cup B) = \mu(A) + \mu(B)?
    \]
\end{pbox}

Yes.

\begin{proof}
    Let $r = \dist(A, B) > 0$, and define the set $E = \bigcup_{x \in A} B_r(x) \subseteq X$. This construction gives us $A \subseteq E$ and $B \subseteq E^c$. As the union of open balls, $E$ is an open subset of $X$ and, therefore, $\mu$-measurable. Hence,
    \[
        \mu(A \cup B)
            = \mu((A \cup B) \cap E) + \mu((A \cup B) \setminus E)
            = \mu(A) + \mu(B).
    \]

\end{proof}


\newpage
\begin{pbox}[4]
    Let $X$ be an uncountable set and let $\CC$ be the collection of all subsets $A$ of $X$ such that either $A$ or $A^c$ is at most countable. Prove that $\CC$ is a $\sigma$-algebra.
\end{pbox}

\begin{proof}
    Since $\emptyset = X^c$ is at most countable, $\emptyset, X \in \CC$.
    
    If $A \in \CC$, then either $A^c$ or $A = (A^c)^c$ is at most countable, so $A^c \in \CC$.

    Suppose $A_1, A_2, \ldots \in \CC$ and let $A = \bigcup_{n=1}^{\infty} A_n$. It is either the case that all $A_n$ are at most countable or some $A_N^c$ is at most countable. In the first case, $A$ is countable, as a countable union of countable subsets. In the second case, $A^c = \bigcap_{n=1}^{\infty} A_n^c$ is contained in some countable set $A_N^c$, implying $A^c$ is countable. In both cases, either $A$ is at most countable or $A^c$ is at most countable, so in fact $A \in \CC$.

\end{proof}

\end{document}