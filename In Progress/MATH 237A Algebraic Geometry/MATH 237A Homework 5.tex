\documentclass[12pt]{article}

% Packages
\usepackage[margin=1in]{geometry}
\usepackage{fancyhdr, parskip}
\usepackage{amsmath, amsthm, amssymb}

% commutative diagrams
\usepackage{tikz-cd}
\tikzcdset{row sep/normal=0pt}
\newenvironment{cd}{\begin{center}\begin{tikzcd}}{\end{tikzcd}\end{center}}

% Page Style
\makeatletter
\fancypagestyle{title}{
    \renewcommand{\headrulewidth}{0.4pt}
    \setlength{\headheight}{15pt}
    \fancyhead[R]{\@author}
    \fancyhead[L]{\@title}
    \fancyhead[C]{\@date}
}
\makeatother
\renewcommand{\maketitle}{\thispagestyle{title}}
\fancypagestyle{plain}{
    \fancyhf{}
    \renewcommand{\headrulewidth}{0pt}
    \renewcommand{\footrulewidth}{0pt}
    \fancyfoot[R]{\thepage}
}
\pagestyle{plain}

% Problem Box
\setlength{\fboxsep}{4pt}
\newlength{\myparskip}
\setlength{\myparskip}{\parskip}
\newsavebox{\savefullbox}
\newenvironment{fullbox}{\begin{lrbox}{\savefullbox}\begin{minipage}{\dimexpr\textwidth-2\fboxsep\relax}\setlength{\parskip}{\myparskip}}{\end{minipage}\end{lrbox}\framebox[\textwidth]{\usebox{\savefullbox}}}
\newenvironment{pbox}[1][]{\begin{fullbox}\ifx#1\empty\else\paragraph{#1}\fi}{\end{fullbox}}

% Default Commands
\newcommand{\isp}[1]{\quad\text{#1}\quad}
\newcommand{\N}{\mathbb{N}} 
\newcommand{\Z}{\mathbb{Z}}
\newcommand{\Q}{\mathbb{Q}}
\newcommand{\R}{\mathbb{R}}
\newcommand{\C}{\mathbb{C}}
\newcommand{\eps}{\varepsilon}
\renewcommand{\phi}{\varphi}
\renewcommand{\emptyset}{\varnothing}
\newcommand{\<}{\langle}
\renewcommand{\>}{\rangle}
\newcommand{\isom}{\cong}
\newcommand{\eqc}{\overline}
\newcommand{\clo}{\overline}
\newcommand{\teq}{\trianglelefteq}
\DeclareMathOperator{\id}{id}

% Extra Commands
\newcommand{\mm}{\mathfrak{m}}
\newcommand{\FF}{\mathcal{F}}
\newcommand{\GG}{\mathcal{G}}
\DeclareMathOperator{\Supp}{Supp}
\DeclareMathOperator{\im}{im}
\renewcommand{\Im}{\operatorname{Im}}
\DeclareMathOperator{\Ker}{Ker}

% Document
\begin{document}
\title{MATH 237A Homework 5}
\author{Harry Coleman\makebox[0pt][r]{\raisebox{-0.25in}[0pt][0pt]{(worked with Joseph Sullivan)}}}
\date{November 1, 2021}
\maketitle

\begin{pbox}[1 Eisenbud Exercise 2.2]
    Let $R$ be a ring and let $U$ be any subset of $R$. Show that $R[U^{-1}]$ is the result of adjoining inverses of elements of $U$ to $R$ in the freest possible way, in the sense that
    \[
        R[U^{-1}] \isom R[\{x_u\}_{u \in U}]/\<\{ux_u - 1\}_{u \in U}\>.
    \]
\end{pbox}

\begin{proof}
    Denote $X_U = \{x_u\}_{u \in U}$ and $I_U = \<\{ux_u - 1\}_{u \in U}\> \teq R[X_U]$. 
    
    Denote $S_U = R[X_U]/I_U$ and let $y_u \in S_U$ be the image of $x_u$ under the projection $R[X_U] \to S_U$. Then each $y_u$ is a unit in $S_U$ with inverse $u$.

    Every element of $S_U$ has a representation as $ry_{u_1} \cdots y_{u_n}$, for some $r \in R$ and $u_i \in U$. As an $R$-algebra, $S_U$ is generated by all the $y_u$'s, so this fact is clear for products of coefficients and generators. For $a, b \in R$ and $u, v \in U$, we find
    \[
        ay_u + by_v
            = ay_u(vy_v) + by_v(uy_u)
            = (av + bu)y_uy_v.
    \]
    One can check that a similar argument holds for sums of arbitrary terms, and that this generalizes to all finite sums of terms, covering all of $S_U$. Moreover, one can also check that this representation is unique up to units and reordering of the indeterminates.

    Let $\clo{U}$ be the multiplicative closure of $U$. In $S_{\clo{U}}$, the following holds:
    \[
        y_uy_v
         = y_uy_v(uvy_{uv})
         = (uy_u)(vy_v)y_{uv}
         = y_{uv}.
    \]
    This means that every element of $S_{\clo{U}}$ has a representation as $ry_u$ for some $r \in R$ and $u \in \clo{U}$. Since every element of $\clo{U}$ is a finite product of elements in $U$, the obvious $R$-algebra homomorphism $S_U \to S_{\clo{U}}$, defined by $y_u \mapsto y_u$, is surjective. We will check that this map is also injective.

    If $ry_{u_1} \cdots y_{u_n} \in S_U$ is in the kernel, $ry_{u_1} \cdots y_{n} = 0$ in $S_{\clo{U}} = R[X_{\clo{U}}]/I_{\clo{U}}$. On one hand, this means $rx_{u_1} \cdots x_{u_n} \in I_{\clo{U}}$. On the other hand, $u_1, \dots, u_n \in U$, so $rx_{u_1} \cdots x_{u_n} \in R[X_U]$. And since $I_U = I_{\clo{U}} \cap R[X_U]$, we in fact have $rx_{u_1} \cdots x_{u_n} \in I_U$, implying $ry_{u_1} \cdots y_{u_n} = 0$ in $S_U$.

    Hence, $S_U \isom S_{\clo{U}}$ in the natural way. And since $R[U^{-1}] = R[\clo{U}^{-1}]$, it will suffice to prove the isomorphism in the case that $U$ is multiplicatively closed.

    For the rest of the proof, we assume $U$ is multiplicatively closed.
    
    There is a natural $R$-algebra homomorphism $R[X_U] \to R[U^{-1}]$ defined by $x_u \mapsto 1/u$. This map sends the generators of $ux_u - 1$ of $I_U$ map to zero, so it factors through the natural projection $R[X_U] \to S_U$. In other words, there is a well-defined $R$-algebra homomorphism $S_U \to R[U^{-1}]$, with $y_u \mapsto 1/u$. Since $ry_u \mapsto r/u$, it is clear that this map is surjective. If $ry_u$ is in the kernel, meaning $r/u = 0$, there is some $v \in U$ such that $vr = 0$ in $R$. In $S_U$,
    \[
        ry_u = r(vy_v)y_u = (vr)y_uy_v = 0y_{uv} = 0,
    \]
    so the map is injective. Therefore, this is the desired isomorphism.

\end{proof}

\newpage
\begin{pbox}[2 Eisenbud Exercise 2.6]
    Let $R$ be a ring, and let $Q_1, \dots, Q_n$ be ideals of $R$ such that $Q_i + Q_j = R$ for all $i \ne j$. Show that $R/(\bigcap_i Q_i) \isom \prod_i R/Q_i$ as follows:
\end{pbox}

\begin{pbox}[(a)]
    Consider the map of rings $\phi : R \to \prod_i R/Q_i$ obtained from the $n$ projection maps $R \to R/Q_i$. Show that $\ker\phi = \bigcap_i Q_i$.
\end{pbox}

\begin{proof}
    If $x \in \ker\phi$, then $\pi_i(x) = 0$ for all $i$. Therefore, $x \in Q_i$ for all $i$, implying $x \in \bigcap_i Q_i$.

    If $x \in \bigcap_i Q_i$, then $\pi_i(x) = 0$ for all $i$, implying $x \in \ker\phi$.

\end{proof}

\begin{pbox}[(b)]
    Let $\mm$ be a maximal ideal of $R$. Show that the hypothesis that $Q_i + Q_j = R$ for all $i \ne j$ means that at most one of the $Q_i$ is contained in $\mm$. Now use Corollary 2.9 to show that $\phi$ is surjective.
\end{pbox}

\begin{proof}
    Denote $S = \prod_i R/Q_i$.

    By definition, $Q_i + Q_j$ is the smallest ideal of $R$ containing both $Q_i$ and $Q_j$. If $\mm$ were to contain both $Q_i$ and $Q_j$ for $i \ne j$, then it would also need to contain $Q_i + Q_j = R$. But $\mm$ is a proper ideal, so this is not possible.

    To apply Corollary 2.9, we need to check that $\phi_\mm : R_\mm \to S_\mm$ is surjective. Note that
    \[\textstyle
        S_\mm \isom \prod_i R_\mm/(Q_i)_\mm.
    \]
    For a given $i$, if $Q_i \nsubseteq \mm$, there is some $x \in Q_i \setminus \mm$. Under the natural map $R \to R_\mm$, the image of $x$ is a unit and is in $(Q_i)_\mm$. Therefore, $(Q_i)_\mm$ is the whole ring, so $R_\mm/(Q_i)_\mm = 0$.

    If $\mm$ contains no $Q_i$, then $R_\mm/(Q_i)_\mm = 0$ for all $i$, implying $S_\mm = 0$. In which case, $\phi_\mm$ is trivially surjective.

    If $\mm$ contains $Q_i$ (and, therefore, no $Q_j$ for $j \ne i$), $S_\mm$ is the product of $R_\mm/(Q_i)_\mm$ and a copy of the zero ring for each $j \ne i$. In this case, the surjectivity of $\phi_\mm$ follows from the surjectivity of the natural projection $R_\mm \to R_\mm/(Q_i)_\mm$.

    In either case, $\phi_\mm$ is surjective. Since this is true for all maximal ideals $\mm$ of $R$, we may apply Corollary 2.9 to conclude that $\phi$ itself is surjective.
\end{proof}



\newpage
\begin{pbox}[3 Hartshorne Exercise II.1.6 \\ (a)]
    Let $\FF'$ be a subsheaf of a sheaf $\FF$. Show that the natural map of $\FF$ to the quotient sheaf $\FF/\FF'$ is surjective, and has kernel $\FF'$. Thus there is an exact sequence
    \begin{cd}
        0 \rar & \FF' \rar & \FF \rar & \FF/\FF' \rar & 0
    \end{cd}
\end{pbox}

\begin{proof}
    Let $\GG$ be the quotient presheaf defined by $\GG(U) = \FF(U)/\FF'(U)$. Then, by definition of the quotient sheaf, its sheafification is $\GG^+ = \FF/\FF'$.

    There is a morphism of presheaves $\pi : \FF \to \GG$, where each
    \begin{cd}
        \pi_U : \FF(U) \rar & \GG(U) = \FF(U)/\FF'(U)
    \end{cd}
    is the natural projection of abelian groups. Identifying $\GG$ with a subpresheaf of $\FF/\FF'$, we may consider $\pi$ to be a morphism of sheaves $\FF \to \FF/\FF'$ (which we take to be the natural map).

    Denote by $\Im'\pi$ the image presheaf of $\pi$, then
    \[
        (\Im'\pi)(U) = \im\pi_U = \GG(U).
    \]
    So $\Im'\pi = \GG$, and the image sheaf of $\pi$ is given by the sheafification
    \[
        \Im\pi = (\Im'\pi)^+ = \GG^+ = \FF/\FF',
    \]
    hence $\pi$ is surjective.

    Denote by $\Ker\pi$ the kernel sheaf of $\pi$, then
    \[
        (\Ker\pi)(U) = \ker\pi_U = \FF'(U).
    \]
    So, in fact, $\Ker\pi = \FF'$.

    Thus, we have the exact sequence
    \begin{cd}
        0 \rar & \FF' \rar[hookrightarrow] & \FF \rar["\pi"] & \FF/\FF' \rar & 0
    \end{cd}

\end{proof}


\newpage
\begin{pbox}[(b)]
    Conversely, if $0 \to \FF' \to \FF \to \FF'' \to 0$ is an exact sequence, show that $\FF'$ is isomorphic to a subsheaf of $\FF$, and that $\FF''$ isomorphic to quotient $\FF$ by this subsheaf.
\end{pbox}

\begin{proof}
    Label the exact sequence by
    \begin{cd}
        0 \rar & \FF' \rar["\alpha"] & \FF \rar["\beta"] & \FF'' \rar & 0.
    \end{cd}
    Since $\alpha$ is injective, we have an exact sequence
    \begin{cd}
        0 = \Ker\alpha \rar[hookrightarrow] & \FF' \rar & \Im\alpha \rar & 0.
    \end{cd}
    One can check that this implies $\FF' \isom \Im\alpha$. (This is the result of Hartshorne Exercise II.1.5, which can be proven by taking the inverse morphism of sheaves to be the inverse morphism of abelian groups on each open set.) By exactness, $\Im\alpha = \Ker\beta$, so $\FF' \isom \Ker\beta$, where $\Ker\beta$ is a subsheaf of $\FF$, by definition.

    
    Let $\GG$ be the quotient presheaf defined by $\GG(U) = \FF(U)/(\Ker\beta)(U)$, so $\GG^+ = \FF/\Ker\beta$. There is a morphism $\GG \to \Im\beta$, defined at each $U$ by
    \begin{cd}
        \GG(U) = \FF(U)/\ker\beta_U \rar["\sim"] & \im\beta_U = (\Im'\beta)(U) \rar[hookrightarrow] & (\Im\beta)(U),
    \end{cd}
    where $\Im'\beta$ denotes the image presheaf of $\beta$. The first map comes from the first isomorphism theorem on abelian groups, and the second from the inclusion of a presheaf into its sheafification. By the universal property of sheafification, this factors through a morphism of sheaves $\phi : \FF/\Ker\beta \to \Im\beta$. Since each $\phi_U$ is injective, we know that $\phi$ is injective. By construction, $\Im'\phi = \Im'\beta$, implying $\Im\phi = \Im\beta$. And since $\Im\beta = \FF''$, we conclude that $\phi$ is surjective. Applying Hartshorne Exercise II.1.5 to $\phi$, we obtain $\FF/\Ker\beta \isom \FF''$.
\end{proof}



\newpage
\begin{pbox}[4 Hartshorne Exercise II.1.14]
    Let $\FF$ be a sheaf on $X$, and let $s \in \FF(U)$ be a section over an open set $U$. The \textit{support} of $s$, denoted $\Supp s$ is defined to be $\{P \in X \mid s_P \ne 0\}$, where $s_P$ denotes the germ of $s$ in the stalk $\FF_P$. Show that $\Supp s$ is a closed subset of $U$. We define the \textit{support} of $\FF$, $\Supp \FF$, to be $\{P \in X \mid \FF_P \ne 0\}$. It need not be a closed subset.
\end{pbox}

\begin{proof}
    Let $V = U \setminus \Supp s = \{P \in U \mid s_P = 0\}$; we will show that $V$ is open in $U$. For a given $P \in V$, the fact that $s_P = 0$ means there is an open neighborhood $U_P$ of $P$ contained in $U$, such that $s|_{U_P} = 0 \in \FF(U_P)$. Then, for any $Q \in U_P$, it follows that $s_Q = 0$, since $U_P$ is also an open neighborhood of $Q$. Hence $U_P \subseteq V$, implying $V$ is open in $U$.


\end{proof}

\end{document}