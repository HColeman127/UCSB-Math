\documentclass[12pt]{article}

% Packages
\usepackage[margin=1in]{geometry}
\usepackage{fancyhdr, parskip}
\usepackage{amsmath, amsthm, amssymb}

% Page Style
\makeatletter
\fancypagestyle{title}{
    \renewcommand{\headrulewidth}{0.4pt}
    \setlength{\headheight}{15pt}
    \fancyhead[R]{\@author}
    \fancyhead[L]{\@title}
    \fancyhead[C]{\@date}
}
\makeatother
\renewcommand{\maketitle}{\thispagestyle{title}}
\fancypagestyle{plain}{
    \fancyhf{}
    \renewcommand{\headrulewidth}{0pt}
    \renewcommand{\footrulewidth}{0pt}
    \fancyfoot[R]{\thepage}
}
\pagestyle{plain}


% Problem Box
\setlength{\fboxsep}{4pt}
\newlength{\myparskip}
\setlength{\myparskip}{\parskip}
\newsavebox{\savefullbox}
\newenvironment{fullbox}{\begin{lrbox}{\savefullbox}\begin{minipage}{\dimexpr\textwidth-2\fboxsep\relax}\setlength{\parskip}{\myparskip}}{\end{minipage}\end{lrbox}\framebox[\textwidth]{\usebox{\savefullbox}}}
\newenvironment{pbox}[1][]{\begin{fullbox}\ifx#1\empty\else\paragraph{#1}\fi}{\end{fullbox}}

% Default Commands
\newcommand{\isp}[1]{\quad\text{#1}\quad}
\newcommand{\N}{\mathbb{N}} 
\newcommand{\Z}{\mathbb{Z}}
\newcommand{\Q}{\mathbb{Q}}
\newcommand{\R}{\mathbb{R}}
\newcommand{\C}{\mathbb{C}}
\newcommand{\eps}{\varepsilon}
\renewcommand{\phi}{\varphi}
\renewcommand{\emptyset}{\varnothing}
\newcommand{\<}{\langle}
\renewcommand{\>}{\rangle}
\newcommand{\isom}{\cong}
\newcommand{\eqc}{\overline}
\newcommand{\clo}{\overline}
\newcommand{\teq}{\trianglelefteq}
\DeclareMathOperator{\id}{id}

% Extra Commands
\newcommand{\A}{\mathbb{A}}
\renewcommand{\P}{\mathbb{P}}
\DeclareMathOperator{\im}{im}
\newcommand{\rad}{\sqrt}

% Document
\begin{document}
\title{MATH 237A Homework 3}
\author{Harry Coleman\makebox[0pt][r]{\raisebox{-0.25in}[0pt][0pt]{(worked with Joseph Sullivan)}}}
\date{October 16, 2021}
\maketitle


\begin{pbox}[1 Eisenbud Exercise 1.19]
    Let $k$ be a field. Let $I \subseteq k[x, y, z, w]$ be the ideal generated by the $2 \times 2$ minors of the matrix
    \[
        \begin{bmatrix}
            x & y & z \\
            y & z & w
        \end{bmatrix},
    \]
    that is, $I = \<yw - z^2, xw - yz, xz - y^2\>$.

    Show that $R = k[x, y, z, w]/I$ is a finitely generated free module over $S = k[x, w]$. Exhibit a basis for $R$ as an $S$-module.
\end{pbox}

\begin{proof}
    We claim that each monomial in $R = S[y, z]/I$ has a representative $sy^az^b$, where $s \in S$ and $a, b \in \{0, 1\}$. Let $y^az^b$ be an arbitrary monomial in $S[y, z]/I$, so $a, b \in \Z_{\geq 0}$. We define a recursive procedure on the monomial to find a new representative.
    
    If $a \geq 2$, proceed as follows. Write $a = a' + 2c$, where $c \geq 1$ and $a' \in \{0, 1\}$. Then
    \[
        y^a
            = y^{a' + 2c}
            = y^{a'}(y^2)^c
            = y^{a'}(xz)^c
            = x^cy^{a'}z^c,
    \]
    so $y^az^b = x^cy^{a'}z^{c + b}$, where $x^c \in S$ and $a' + c + b < a + b$.

    If $b \geq 2$, proceed as follows. Write $b = b' + 2c$, where $c \geq 1$ and $b' \in \{0, 1\}$. Then
    \[
        z^b
            = z^{b' + 2c}
            = z^{b'}(z^2)^c
            = y^{a'}(yw)^c
            = w^cy^cz^{b'},
    \]
    so $y^az^b = w^cy^{a + c}z^{b'}$, where $w^c \in S$ and $a + c + b' < a + b$.

    In either case, a representative in $S[y, z]/I$ is produced with a strictly smaller total degree. Since the total degree of the original monomial is finite, the procedure must terminate. When it does, we obtain a representative of the form $sy^az^b$, with $a, b \in \{0, 1\}$.
    
    Moreover, the monomial $yz$ has the representative $wx$. It follows, then, that every monomial in $S[y, z]/I$ has a representative of either $s$, $sy$, or $sz$., for some $s \in S$. The leading terms (under a suitable choice of monomial order) of the generators of $I$ are $y^2$, $zy$, and $z^2$, which are linearly independent in the $S$-module $S[y, z]$. This means that we cannot use $I$ to further reduce the total degree, in the variables $y$ and $z$. In other words, $1$, $y$, and $z$ are linearly independent in the $S$-module $S[y, z]/I$.

    (Equivalently, we can notice that $I$ is given with a Gr\"obner basis, so every polynomial in $S[y, z]$ has a unique representative in $S[y, z]/I$ such that no monomial terms are divisible by the leading terms of the generators. The leading terms of the generators are $y^2$, $yz$, $z^2$, which means that each monomial term has degree at most $1$.)

    Hence, $R = S[y, z]/I$ has a basis $\{1, y, z\}$ as an $S$-module, i.e., $R = S \oplus Sy \oplus Sz$.

\end{proof}


\newpage
\begin{pbox}
    Show that there is a ring homomorphism $R \to k[s, t]$ such that $x \mapsto s^3$, $y \mapsto s^2t$, $z \mapsto st^2$, $w \mapsto t^3$.
\end{pbox}

\begin{proof}
    Consider the $k$-algebra homomorphism $\phi : k[x, y, z, w] \to k[s, t]$ defined by $x \mapsto s^3$, $y \mapsto s^2t$, $z \mapsto st^2$, $w \mapsto t^3$. We apply this map to the generators of $I$:
    \begin{alignat*}{3}
        & yw - z^2  & &\longmapsto (s^2t)t^3 - (st^2)^2     & &= 0, \\
        & xw - yz   & &\longmapsto s^3t^3 - (s^2t)(st^2)    & &= 0, \\
        & xz - y^2  & &\longmapsto s^3(st^2) - (s^2t)^2     & &= 0.
    \end{alignat*}
    This implies $I \subseteq \ker\phi$, so $\phi$ factors through the natural projection
    \[
        \pi : k[x, y, z, w] \to k[x, y, z, w]/I = R.
    \]
    That is, there is a unique $k$-algebra homomorphism $\psi : R \to k[s, t]$ such that $\psi \circ \pi = \phi$. Thus, $\psi$ is the desired ring homomorphism.

\end{proof}

\begin{pbox}
    Use the basis you constructed to show that it is a monomorphism.
\end{pbox}

\begin{proof}
    Suppose $f, g \in R$ such that $\psi(f) = \psi(g)$, i.e.,
    \[
        f(s^3, s^2t, st^2, t^3) = g(s^3, s^2t, st^2, t^3) \in k[s, t].
    \]
    Since $R = S \oplus Sy \oplus Sz$, we have
    \[
        f = a_0 + a_1y + a_2z
        \isp{and}
        g = b_0 + b_1y + b_2z,
    \]
    for some $a_i, b_i \in S$.
    
    For every $c \in S = k[x, w]$, we have $\psi(c) = c(s^3, t^3) \in k[s^3, t^3]$. In particular, both the $s$- and $t$-degree of every monomial term of $\psi(c)$ is a nonnegative multiple of $3$, i.e., equivalent to $0$ mod $3$. Along similar lines, every monomial of $\psi(cy) = c(s^3, t^3)s^2t$ has $s$-degree equivalent to $2$ mod $3$ and $t$-degree equivalent to $1$ mod $3$. And every monomial of $\psi(cz) = c(s^3, t^3)st^2$ has $s$-degree equivalent to $1$ mod $3$ and $t$-degree equivalent to $2$ mod $3$.

    We deduce that $\psi(a_0)$, $\psi(a_1y)$, and $\psi(a_2z)$ share no monomial terms with each other, and the same is true for $\psi(b_0)$, $\psi(b_1y)$, and $\psi(b_2z)$. So $\psi(f) = \psi(g)$ implies that $\psi(a_0) = \psi(b_0)$, $\psi(a_1y) = \psi(b_1y)$, and $\psi(a_2z) = \psi(b_2z)$. Since
    \[
        \psi(a_1)s^2t = \psi(a_1y) = \psi(b_1y) = \psi(b_1)s^2t
    \]
    and
    \[
        \psi(a_2)st^2 = \psi(a_2z) = \psi(b_2z) = \psi(b_1)st^2,
    \]
    then in fact $\psi(a_i) = \psi(b_i)$ for $i = 0, 1, 2$.
    
    Note that $\psi|_S : x \mapsto s^3, w \mapsto t^3$ describes a $k$-algebra isomorphism from $S = k[x, w]$ to $k[s^3, t^3]$. In particular, it is an injection $S \to k[s, t]$. We conclude that $a_i = b_i$ for $i = 0, 1, 2$, so indeed $f = g$. Hence, $\psi$ is an injective homomorphism (monomorphism).

\end{proof}

\begin{pbox}
    Conclude that $I$ is prime.
\end{pbox}

\begin{proof}
    Since $\psi$ is an injective homomorphism, $R \isom \im\psi \subseteq k[s, t]$. Recall that $\psi \circ \pi = \phi$ and $\pi$ is surjective, so $\im\psi = \im\phi$. Since $\phi$ is a $k$-algebra homomorphism, we have
    \begin{align*}
        \im\phi
            &= \phi(k[x, y, z, w]) \\
            &= k[\phi(x), \phi(y), \phi(z), \phi(w)] \\
            &= k[s^3, s^2t, st^2, t^3].
    \end{align*}
    Hence, $R \isom k[s^3, s^2t, st^2, t^3]$. In particular, $R = k[x, y, z, w]/I$ is an integral domain, proving that $I$ is a prime ideal.
    
\end{proof}

\begin{pbox}
    From the rank of $R$ as a free $S$-module, and the degrees of the generators, deduce the Hilbert function of $R$.
\end{pbox}

\begin{proof}
    The monomials in $S= k[x, w]$ of degree $s \geq 0$ are $x^aw^{s-a}$ for $a = 0, \dots, s$, so
    \[
        H_S(s) = \begin{cases}
            s + 1 &\text{if } s \geq 0, \\
            0 &\text{otherwise}.
        \end{cases}
    \]
    Recall that $R$ as an $S$-module, is $S \oplus Sy \oplus Sz$. Note that $Sy$ and $Sz$ can be treated as copies of $S$ with its degree shifted up by $1$, i.e., $H_{Sy}(s) = H_{Sz}(s) = H_S(s - 1)$. Hence,
    \[
        H_R(s) = H_S(s) + 2H_S(s - 1) = \begin{cases}
            3s + 1 &\text{if } s \geq 0, \\
            0 &\text{otherwise}.
        \end{cases}
    \]

\end{proof}

\begin{pbox}
    Show that $R$ is not finitely generated as a module over $k[x, y]$.
\end{pbox}


\newpage
\begin{pbox}[2 Hartshorne Exercise I.1.2]
    Let $Y \subseteq \A^3$ be the set $Y = \{(t, t^2, t^3) \mid t \in k\}$. Show that $Y$ is an affine variety of dimension $1$. Find generators for the ideal $I(Y)$. Show that $A(Y)$ is isomorphic to a polynomial ring in one variable over $k$. We say that $Y$ is given by the parametric representation $x = t$, $y = t^2$, $z = t^3$.
\end{pbox}

Let $f = x^2 - y, g = x^3 - z \in k[x, y, z]$. We claim $Y = Z(f, g)$. For $P = (t, t^2, t^3) \in Y$, we have $f(P) = g(P) = 0$, so $P \in Z(f, g)$. On the other hand, for $P = (a, b, c) \in Z(f, g)$, we have $a^2 - b = a^3 - c = 0$, so $P = (a, a^2, a^3) \in Y$. Hence, $Y = Z(f, g)$.

Let $J = \<f, g\> \teq k[x, y, z]$. By the Nullstellensatz,
\[
    I(Y) = I(Z(J)) = \rad{J}.
\]
We claim that $J$ is a radical ideal, i.e., that $\rad{J} = J$.

Notice that $J = \<y - x^2, z - x^3\>$ is simply the kernel of the evaluation $k[x]$-algebra homomorphism $(k[x])[y, z] \to k[x]$ defined by $y \mapsto x^2$, and $z \mapsto x^3$. This is a surjective map, so  we obtain $k[x, y, z]/J \isom k[x]$. In particular, $k[x, y, z]/J$ is a reduced ring (has no nonzero nilpotent elements), so $J$ is a radical ideal.

So, $A(Y) = k[x, y, z]/I(Y) \isom k[x]$, i.e., $A(Y) \isom A(\A^1)$ as $k$-algebras, so $Y \isom \A^1$ as varieties.


\newpage
\begin{pbox}[3 Hartshorne Exercise I.1.4]
    If we identify $\A^2$ with $\A^1 \times \A^1$ in the natural way, show that the Zariski topology on $\A^2$ is not the product topology of the Zariski topologies of the two copies of $\A^1$.
\end{pbox}

\begin{proof}
    The closed subsets of $\A^1$ under the Zariski topology are, in addition to $\A^1$ itself, precisely the finite subsets. (In other words, the Zariski topology on $\A^1$ is the cofinite topology.) This means that the closed subsets of $\A^1 \times \A^1$ (under the product topology) are finite unions of subsets of the form $X_1 \times X_1$, where $X_1, X_2 \subseteq \A^1$ are closed in the Zariski topology, i.e., either finite or all of $\A^1$.

    By this characterization, we see that the set $X = \{(x, x) \mid x \in k\} \subseteq \A^1 \times \A^1$ is not closed in the product topology. This is because $X$ contains no copies of $\A^1 \times \{x\}$ or $\{x\} \times \A^1$. This means that the only way $X$ can be written as the union of closed sets in the product topology is as the infinite union $X = \bigcup_{x \in k} \{(x, x)\}$.

    However, $X = Z(x - y) \subseteq \A^2$ is closed in the Zariski topology.

\end{proof}


\newpage
\begin{pbox}[4 Hartshorne Exercise I.2.9]
    If $Y \subseteq \A^n$ is an affine variety, we identify $\A^n$ with an open set $U_0 \subseteq \P^n$ by the homeomorphism $\phi_0$. Then we can speak of $\clo{Y}$, the closure of $Y$ in $\P^n$, which is called the \textit{projective closure} of $Y$.
\end{pbox}

\begin{pbox}[(a)]
    Show that $I(\clo{Y})$ is the ideal generated by $\beta(I(Y))$, using the notation of the proof of (2.2).
\end{pbox}

\begin{proof}
    By definition, $I(\clo{Y})$ is the ideal generated by the homogeneous polynomials which are zero on $\clo{Y}$. Let $f \in I(\clo{Y})$ be homogeneous, so $f(P) = 0$ for all $P \in \clo{Y}$. In particular, $f(P) = 0$ for all points $P \in \phi_0(Y)$, i.e.,
    \[
        \alpha(f)(a_1, \dots, a_n) = f(1, a_1, \dots, a_n) = 0
    \]
    for all $(a_1, \dots, a_n) \in Y$. Therefore, $\alpha(f) \in I(Y)$, so in fact $f = \beta(\alpha(f)) \in \beta(I(Y))$. Since this holds for all the generators, we conclude that $I(\clo{Y}) \subseteq \<\beta(I(Y))\>$.

    If $f \in \beta(I(Y))$, then $\alpha(f) \in I(Y)$. So for all $(a_1, \dots, a_n) \in Y$, we have
    \[
        f(1, a_1, \dots, a_n) = \alpha(f)(a_1, \dots, a_n) = 0.
    \]
    This means that $f(P) = 0$ for all $P \in \phi_0(Y)$. In other words, $\phi_0(Y) \subseteq Z(f) \subseteq \P^n$. Since $Z(f)$ is a closed subset, this implies that $\clo{Y} = \clo{\phi_0(Y)} \subseteq Z(f)$. By the Nullstellensatz,
    \[
        f \in \rad{\<f\>} = I(Z(f)) \subseteq I(\clo{Y}).
    \]
    Since all generators are contained in $I(\clo{Y})$, we conclude that $\<\beta(I(Y))\> \subseteq I(\clo{Y})$.
    
\end{proof}

\begin{pbox}[(b)]
    Let $Y \subseteq \A^3$ be the twisted cubic of (Ex. 1.2). Its projective closure $\clo{Y} \subseteq \P^3$ is called the \textit{twisted cubic curve} in $\P^3$. Find generators for $I(Y)$ and $I(\clo{Y})$, and use this example to show that if $f_1, \dots, f_r$ generate $I(Y)$, then $\beta(f_1), \dots, \beta(f_r)$ do not necessarily generate $I(\clo{Y})$.
\end{pbox}

\begin{proof}
    In the proof of Problem 2, we showed that $I(Y) = \<y - x^2, z - x^3\> \teq k[x, y, z]$.

    Let $t$ be the fourth projective coordinate in $\P^3$, then
    \[
        \beta(y - x^2) = yt - x^2
        \isp{and}
        \beta(z - x^3) = zt^2 - x^3.
    \]
    Define the homogeneous ideal $I = \<yt - x^2, zt^2 - x^3\> \teq k[x, y, z, t]$, then $I \subseteq I(\clo{Y})$.

    We have
    \[
        xy - z = x(y - x^2) - (z - x^3) \in I(Y),
    \]
    so $xy - zt \in I(\clo{Y})$.

    However, $xy - zt \notin I$, since the only homogeneous degree $2$ generator of $I$ is $yt - x^2$, which is also the homogeneous generator of least degree.
    
\end{proof}

\end{document}