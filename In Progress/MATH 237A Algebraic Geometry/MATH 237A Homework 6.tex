\documentclass[12pt]{article}

% Packages
\usepackage[margin=1in]{geometry}
\usepackage{fancyhdr, parskip}
\usepackage{amsmath, amsthm, amssymb}

% commutative diagrams
\usepackage{tikz-cd}
\tikzcdset{row sep/normal=0pt}
\newenvironment{cd}{\begin{center}\begin{tikzcd}}{\end{tikzcd}\end{center}}

% Page Style
\makeatletter
\fancypagestyle{title}{
    \renewcommand{\headrulewidth}{0.4pt}
    \setlength{\headheight}{15pt}
    \fancyhead[R]{\@author}
    \fancyhead[L]{\@title}
    \fancyhead[C]{\@date}
}
\makeatother
\renewcommand{\maketitle}{\thispagestyle{title}}
\fancypagestyle{plain}{
    \fancyhf{}
    \renewcommand{\headrulewidth}{0pt}
    \renewcommand{\footrulewidth}{0pt}
    \fancyfoot[R]{\thepage}
}
\pagestyle{plain}

% Problem Box
\setlength{\fboxsep}{4pt}
\newlength{\myparskip}
\setlength{\myparskip}{\parskip}
\newsavebox{\savefullbox}
\newenvironment{fullbox}{\begin{lrbox}{\savefullbox}\begin{minipage}{\dimexpr\textwidth-2\fboxsep\relax}\setlength{\parskip}{\myparskip}}{\end{minipage}\end{lrbox}\framebox[\textwidth]{\usebox{\savefullbox}}}
\newenvironment{pbox}[1][]{\begin{fullbox}\ifx#1\empty\else\paragraph{#1}\fi}{\end{fullbox}}

% Default Commands
\newcommand{\isp}[1]{\quad\text{#1}\quad}
\newcommand{\N}{\mathbb{N}} 
\newcommand{\Z}{\mathbb{Z}}
\newcommand{\Q}{\mathbb{Q}}
\newcommand{\R}{\mathbb{R}}
\newcommand{\C}{\mathbb{C}}
\newcommand{\eps}{\varepsilon}
\renewcommand{\phi}{\varphi}
\renewcommand{\emptyset}{\varnothing}
\newcommand{\<}{\langle}
\renewcommand{\>}{\rangle}
\newcommand{\isom}{\cong}
\newcommand{\eqc}{\overline}
\newcommand{\clo}{\overline}
\newcommand{\teq}{\trianglelefteq}
\DeclareMathOperator{\id}{id}

% Extra Commands
\newcommand{\A}{\mathbb{A}}
\newcommand{\FF}{\mathcal{F}}
\newcommand{\GG}{\mathcal{G}}
\DeclareMathOperator{\Hom}{Hom}
\DeclareMathOperator{\shHom}{\mathcal{H}\!\textit{om}}
\newcommand{\mat}[1]{\begin{bmatrix}#1\end{bmatrix}}
\newcommand{\pdv}[2]{\frac{\partial #1}{\partial #2}}

% Document
\begin{document}
\title{MATH 237A Homework 6}
\author{Harry Coleman}
\date{November 9, 2021}
\maketitle

\begin{pbox}[1 Exercise I.5.1]
    Locate the singular points and sketch the following curves in $\A^2$ (assume $\operatorname{char} k \ne 2$). Which is in figure 4?
\end{pbox}

\begin{pbox}[(a)]
    $x^2 = x^4 + y^4$
\end{pbox}

Let $f = x^2 - x^4 - y^4$ and consider the rank of
\[
    \mat{\pdv{f}{x} & \pdv{f}{y}} = \mat{2x - 4x^3 & -4y^3}.
\]
The singularities occur when the rank is less than $1$, i.e., then the rank is zero. The rank is zero when
\[
    2x - 4x^3 = 0 \isp{and} -4y^3 = 0.
\]
This occurs when $x = 0, \pm1/\sqrt{2}$ and $y = 0$. Note however that $(\pm1/\sqrt{2}, 0) \notin Z(f)$, so the only singularity is at $(0, 0)$.

This is the Tacnode in Figure 4.

\begin{pbox}[(b)]
    $xy = x^6 + y^6$
\end{pbox}

Let $f = xy - x^6 - y^6$ and consider the rank of
\[
    \mat{\pdv{f}{x} & \pdv{f}{y}} = \mat{y - 6x^5 & x - 6y^5}.
\]
The singularities occur when
\[
    y = 6x^5 \isp{and} x = 6y^5.
\]
This occurs when $x = y = 0$, and we check that no other solution lies in $Z(f)$. If $x$ is nonzero, $y$ must also be nonzero and
\[
    x = 6y^5 = 6(6x^5)^5 = 6^6x^{25},
\]
which implies $x^{24} = 1/6^6$. However, in order to satisfy $f(x, y) = 0$, we must have
\[
    0 = xy - x^6 - y^6 = x(6x^5) - x^6 - (6x^5)^6 = 5x^6 - 6^6 x^{30},
\]
which implies $x^{24} = 5/6^6$. Hence, the only singularity is at $(0, 0)$.

This is the Node in Figure 4.


\newpage
\begin{pbox}[(c)]
    $x^3 = y^2 + x^4 + y^4$
\end{pbox}

Let $f = x^3 - x^4 - y^2 - y^4$ and consider the rank of
\[
    \mat{\pdv{f}{x} & \pdv{f}{y}} = \mat{3x^2 - 4x^3 & -2y - 4y^3}.
\]
The singularities occur when
\[
    3x^2 - 4x^3 = 0 \isp{and} -2y - 4y^3 = 0.
\]
This occurs when $x = 0, 3/4$ and $y = 0, \pm i/\sqrt{2}$. One can check that the only combination of components satisfying $f(x, y) = 0$ is $(0, 0)$.


This is the Cusp in Figure 4.

\begin{pbox}[(d)]
    $x^2y + xy^2 = x^4 + y^4$
\end{pbox}

Let $f = x^2y + xy^2 - x^4 - y^4$ and consider the rank of
\[
    \mat{\pdv{f}{x} & \pdv{f}{y}} = \mat{2xy + y^2 - 4x^3 & x^2 + 2xy - 4y^3}.
\]
Let $g_1 = 2xy + y^2 - 4x^3$ and $g_2 = x^2 + 2xy - 4y^3$. The singularities occur when
\[
    g_1(x, y) = 0 \isp{and} g_2(x, y) = 0.
\]
In other words, the set of singularities can be written as
\[
    Z(f, g_1, g_2) = Z(\<f, g_1, g_2\>).
\]
Using Buchberger's algorithm to compute a reduced Gr\"obner basis with respect to the lexicographic monomial order, we find
\[
    \<f, g_1, g_2\> = \<y^3,\, x^2 + y^2,\, xy + \tfrac{1}{2}y^2\>.
\]
So we must have $y^3 = 0$, which implies $x = y = 0$. Hence, the only singularity is at $(0, 0)$.

This is the Triple Point in Figure 4.



\newpage
\begin{pbox}[2 Exercise I.5.2]
    Locate the singular points and describe the singularities of the following surfaces in $\A^3$ (assume $\operatorname{char} k \ne 2$). Which is in figure 5?
\end{pbox}

\begin{pbox}[(a)]
    $xy^2 = z^2$
\end{pbox}

Let $f = xy^2 - z^2$ and consider the rank of
\[
    \mat{\pdv{f}{x} & \pdv{f}{y} & \pdv{f}{z}} = \mat{y^2 & -2xy & -2z}.
\]
The singularities occur when
\[
    y^2 = 0, \quad -2xy = 0, \quad -2z = 0.
\]
That is, the surface is singular when $y = z = 0$, i.e., alone the line $(x, 0, 0)$ for $x \in \C$.

This is the Pinch Point in Figure 5.

\begin{pbox}[(b)]
    $x^2 + y^2 = z^2$
\end{pbox}

Let $f = x^2 + y^2 - z^2$ and consider the rank of
\[
    \mat{\pdv{f}{x} & \pdv{f}{y} & \pdv{f}{z}} = \mat{2x & 2y & -2z}.
\]
The singularities occur when
\[
    2x = 0, \quad 2y = 0, \quad -2z = 0.
\]
That is, the only singularity is at $(0, 0, 0)$.

This is the Conical Double Point in Figure 5.

\newpage
\begin{pbox}[(c)]
    $xy + x^3 + y^3 = 0$
\end{pbox}

Let $f = xy + x^3 + y^3$ and consider the rank of
\[
    \mat{\pdv{f}{x} & \pdv{f}{y} & \pdv{f}{z}} = \mat{y + 3x^2 & x + 3y^2}.
\]
The singularities occur when
\[
    y + 3x^2 = 0, \quad x + 3y^2 = 0.
\]
In other words, the set of singularities can be written as
\[
    Z(f, y+3x^2, x+3y^2) = Z(\<f, y+3x^2, x+3y^2\>).
\]
Using Buchberger's algorithm to compute a reduced Gr\"obner basis with respect to the lexicographic monomial order, we find
\[
    \<f, y+3x^2, x+3y^2\> = \<x, y\>.
\]
So we must have $x = y = 0$. Hence, the only singularity is at $(0, 0, 0)$.

This is the Double Line in Figure 5.



\newpage
\begin{pbox}[3 Exercise I.5.3]
    Let $Y \subseteq \A^2$ be a curve defined by the equation $f(x, y) = 0$. Let $P = (a, b)$ be a point of $\A^2$. Make a linear change of coordinates so that $P$ becomes the point $(0, 0)$. Then write $f$ as a sum $f = f_0 + f_1 + \cdots f_d$, where $f_i$ is the homogeneous polynomial of degree $i$ in $x$ and $y$. Then we define the \textit{multiplicity} of $P$ on $Y$, denoted $\mu_P(Y)$, to be the least $r$ such that $f_r \ne 0$. (Note that $P \in Y \iff \mu_P(Y) > 0$.) The linear factors of $f_r$ are called the \textit{tangent directions} at $P$.
\end{pbox}

\begin{pbox}[(a)]
    Show that $\mu_P(Y) = 1$ if and only if $P$ is a nonsingular point of $Y$.
\end{pbox}

\begin{proof}
    After a linear change of coordinates, we
    \[
        f = ax + by + \sum_{i=2}^{d} f_i,
    \]
    where $a, b \in \C$ and $f_i \in \C[x, y]$ is homogeneous of degree $i$. Then to check if $Y$ is singular at $P$, we consider the rank of
    \[
        \mat{\pdv{f}{x}\big|_{(0,0)} & \pdv{f}{y}\big|_{(0,0)}} = \mat{a & b}.
    \]
    We know that $Y$ is nonsingular at $P$ if and only if the rank of this matrix is $1$. The rank of this matrix is $1$ if and only if $a$, $b$ are not both zero. And $a$, $b$ are not both zero if and only if the linear part $ax + by$ of $f$ at $P$ is nonzero. Lastly, the linear part $ax + by$ of $f$ at $P$ is nonzero if and only if $\mu_Y(P) = 1$.
\end{proof}

\begin{pbox}[(b)]
    Find the multiplicity of each of the singular points in (Ex. 5.1) above.
\end{pbox}

\begin{enumerate}
    \item[(a)] 2
    \item[(b)] 2
    \item[(c)] 2
    \item[(d)] 3  
\end{enumerate}



\newpage
\begin{pbox}[4 Exercise II.1.15]
    Let $\FF$, $\GG$ be sheaves of abelian groups on $X$. For any open set $U \subseteq X$, show that the set $\Hom(\FF|_U, \GG|_U)$ of morphisms of the restricted sheaves has a natural structure of abelian group. 
\end{pbox}

\begin{proof}
    Note that $\FF|_U(V) = \FF(V)$ for all open sets $V \subseteq U$.

    Given $\phi, \psi \in \Hom(\FF|_U, \GG|_U)$ and $V \subseteq U$ open, define
    \begin{align*}
        (\phi + \psi)(V) : \FF|_U(V) &\longrightarrow \GG|_U(V) \\
            f &\longmapsto \phi(V)(f) + \psi(V)(f).
    \end{align*}
    One can check that this describes an abelian group structure on $\Hom(\FF|_U, \GG|_U)$.
\end{proof}

\begin{pbox}
    Show that the presheaf $U \mapsto \Hom(\FF|_U, \GG|_U)$ is a sheaf. It is called the \textit{sheaf of local morphisms} of $\FF$ into $\GG$, ``sheaf hom'' for short, and is denoted $\shHom(\FF, \GG)$.
\end{pbox}

\begin{proof}
    Let $\{V_i\}$ be an open cover of an open set $U \subseteq X$ and $\phi_i \in \Hom(\FF|_{V_i}, \GG|_{V_i})$ such that $\phi_i|_{V_i \cap V_j} = \phi_j|_{V_i \cap V_j}$. We want to construct $\phi \in \Hom(\FF|_U, \GG|_U)$ such that $\phi|_{V_i} = \phi_i$. For an open subset $V \subseteq U$ we have an open cover $\{W_i = V \cap V_i\}$ of $V$. Then given $f \in \FF|_U(V)$, consider the images $\phi_i(W_i)(f|_{W_i}) \in \GG|_U(W_i) = \GG(W_i)$. We have
    \begin{align*}
        \phi_i(W_i)(f|_{W_i})|_{W_i \cap W_j}
            &= \phi_i(V \cap V_i)(f|_{V \cap V_i})|_{V \cap V_i \cap V_j} \\
            &= \phi_i(V \cap V_i \cap V_j)(f|_{V \cap V_i \cap V_j}) \\
            &= \phi_i|_{V_i \cap V_j}(V \cap V_i \cap V_j)(f|_{V \cap V_i \cap V_j}) \\
            &= \phi_j|_{V_i \cap V_j}(V \cap V_i \cap V_j)(f|_{V \cap V_i \cap V_j}) \\
            &= \phi_j(V \cap V_i \cap V_j)(f|_{V \cap V_i \cap V_j}) \\
            &= \phi_j(V \cap V_j)(f|_{V \cap V_j})|_{V \cap V_i \cap V_j} \\
            &= \phi_j(W_j)(f|_{W_j})|_{W_i \cap W_j}.
    \end{align*}
    Since $\GG|_U$ is a sheaf, there is a unique section in $\GG|_U(V)$ which restricts to $\phi_i(W_i)(f|_{W_i})$ on each $W_i$; denote this section by $\phi(f)$. One can check that that $\phi : \FF|_U \to \GG|_U$ defines a morphism of sheaves.

    Let $\{V_i\}$ be an open cover of an open set $U \subseteq X$ and $\phi \in \Hom(\FF|_U, \GG|_U)$ such that $\phi|_{V_i} = 0$. Given an open set $V \subseteq U$, consider $\phi(V) : \FF(V) \to \GG(V)$. For $f \in \FF(V)$, we have $\phi(V)(f) \in \GG(V)$. Then
    \[
        \phi(V)(f)|_{V \cap V_i}
            = \phi(V \cap V_i)(f|_{V \cap V_i})
            = \phi|_{V_i}(V \cap V_i)(f|_{V \cap V_i})
            = 0.
    \]
    Since $\GG|_U$ is a sheaf, this implies $\phi(V)(f) = 0$. So in fact $\phi(V) = 0$ for all $V \subseteq U$ open, and we conclude that $\phi = 0$.

\end{proof}

\end{document}