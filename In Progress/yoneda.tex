\documentclass[12pt]{article}

% Packages
\usepackage[margin=1in]{geometry}
\usepackage{fancyhdr, parskip}
\usepackage{amsmath, amsthm, amssymb}
\usepackage{tikz, tikz-cd}

\usepackage{suffix} % for \WithSuffix (category stuff)

% Page Style
\makeatletter
\fancypagestyle{title}{
    \renewcommand{\headrulewidth}{0.4pt}
    \setlength{\headheight}{15pt}
    \fancyhead[R]{\@author}
    \fancyhead[L]{\@title}
    \fancyhead[C]{\@date}
}
\makeatother
\renewcommand{\maketitle}{\thispagestyle{title}}
\fancypagestyle{plain}{
    \fancyhf{}
    \renewcommand{\headrulewidth}{0pt}
    \renewcommand{\footrulewidth}{0pt}
    \fancyfoot[R]{\thepage}
}
\pagestyle{plain}

% Problem Box
\setlength{\fboxsep}{4pt}
\newlength{\myparskip}
\setlength{\myparskip}{\parskip}
\newsavebox{\savefullbox}
\newenvironment{fullbox}{\begin{lrbox}{\savefullbox}\begin{minipage}{\dimexpr\textwidth-2\fboxsep\relax}\setlength{\parskip}{\myparskip}}{\end{minipage}\end{lrbox}\framebox[\textwidth]{\usebox{\savefullbox}}}
\newenvironment{pbox}[1][]{\begin{fullbox}\def\temp{#1}\ifx\temp\empty\else\paragraph{#1}\phantom{}\fi}{\end{fullbox}}

% Theorem Environments
\theoremstyle{definition}
\newtheorem{lemma}{Lemma}
\newtheorem{theorem}{Theorem}

% Tikz Environments
\newenvironment{drawing}{\begin{center}\begin{tikzpicture}}{\end{tikzpicture}\end{center}}
% \tikzcdset{row sep/normal=0pt}
\newenvironment{cd}{\begin{center}\begin{tikzcd}}{\end{tikzcd}\end{center}}

% Default Commands
\newcommand{\isp}[1]{\quad\text{#1}\quad}
\newcommand{\N}{\mathbb{N}} 
\newcommand{\Z}{\mathbb{Z}}
\newcommand{\Q}{\mathbb{Q}}
\newcommand{\R}{\mathbb{R}}
\newcommand{\C}{\mathbb{C}}
\newcommand{\A}{\mathbb{A}}
\renewcommand{\P}{\mathbb{P}}
\newcommand{\eps}{\varepsilon}
\renewcommand{\phi}{\varphi}
\renewcommand{\emptyset}{\varnothing}
\newcommand{\<}{\langle}
\renewcommand{\>}{\rangle}
\newcommand{\iso}{\cong}
\newcommand{\eqc}{\overline}
\newcommand{\clo}{\overline}
\newcommand{\seq}{\subseteq}
\newcommand{\teq}{\trianglelefteq}
\DeclareMathOperator{\id}{id}
\DeclareMathOperator{\im}{im}
\newcommand{\inc}{\hookrightarrow}
\newcommand{\dd}{\mathrm{d}}
\newcommand{\mat}[1]{\begin{bmatrix}#1\end{bmatrix}}

% Category Names
\newcommand{\mathcat}{\mathsf}
\newcommand{\newcat}[2]{\newcommand{#1}{\mathcat{#2}}}
\WithSuffix\newcommand\newcat*[2]{\WithSuffix\newcommand#1*{\mathcat{#2}}}

\newcat{\Set}{Set}

% Extra Commands
\newcommand{\CC}{\mathcal{C}}
\DeclareMathOperator{\Mor}{Mor}
\DeclareMathOperator{\Nat}{Nat}
\newcommand{\nato}{\Rightarrow}

% Document
\begin{document}
\title{Yoneda Lemma}
\author{Harry Coleman}
\date{, 2024}
\maketitle

   
\begin{pbox}[Yoneda Lemma]
    For any functor $F : \CC \to \Set$, whose domain $\CC$ is locally small and any object $c \in \CC$, there is a bijection
    \[
        \Nat(\Mor_\CC(c, -), F) \iso Fc
    \]
    that associates a natural transformation $\alpha : \Mor_\CC(c, -) \to F$ with the element $\alpha_c(1_c) \in Fc$.
    Moreover, this correspondence is natural in both $c$ and $F$.
\end{pbox}

\begin{proof}
    The map $\Phi : \Nat(\Mor_\CC(c, -), F) \to Fc$ is easy to construct.
    Given a natural transformation $\alpha : \Mor_\CC(C, -) \nato F$, we simply define
    \[
        \Phi(\alpha) := \alpha_c(1_c),
    \]
    where $\alpha_c : \Mor_\CC(c, c) \to Fc$ is the component of $\alpha$ at $c$.

    We now wish to construct $\Phi$'s inverse, $\Psi : Fc \to \Nat(\Mor_\CC(c, -), F)$.
    That is, given an element $x \in Fc$, we must construct a natural transformation $\Psi(x) : \Mor_\CC(c, -) \nato F$.
    To do this, we will construct its components $\Psi(x)_a : \Mor_\CC(c, a) \to Fd$ for each object $a \in \CC$.
    Moreover, this construction must adhere to the naturality condition, i.e., for all morphisms $f : a \to b$ in $\CC$, the following diagram must commute:
    \begin{cd}
        \Mor_\CC(c, a) \dar["f_*"'] \rar["\Psi(x)_a"] & Fa \dar["Ff"] \\
        \Mor_\CC(c, b) \rar["\Psi(x)_b"'] & Fb
    \end{cd}
    Here, $f_* = \Mor_\CC(c, f)$ is the function which takes a morphism $g : c \to a$ and sends it to the composition $f_*g = f \circ g : c \to b$.

    Let us look at what this diagram is saying in the particular case of $a = c$ and $f : c \to b$.
    The diagram looks like this:
    \begin{cd}
        \Mor_\CC(c, c) \dar["f_*"'] \rar["\Psi(x)_c"] & Fc \dar["Ff"] \\
        \Mor_\CC(c, b) \rar["\Psi(x)_b"'] & Fb
    \end{cd}
    Consider the identity $1_c$ in the upper left corner.
    Following the left side of the square, we obtain
    \[
        \Psi(x)_b(f_*1_c) = \Psi(x)_b(f \circ 1_c) = \Psi(x)_b(f).
    \]
    Since we eventually want to this square to commute, this must be equal to the result of following the right side, i.e., we must have
    \[
        \Psi(x)_b(f) = Ff(\Psi(x)_c(1_c)).
    \]

    In other words, it would suffice to define $\Psi(x)_c(1_c)$.
    
    Recall our definition of $\Phi$, for which we want $\Psi$ to be an inverse.
    Plugging in $\Psi(x)$ for $\alpha$, we have
    \[
        \Psi(x)_c(1_c) = \Phi(\Psi(x)) = x.
    \]
    Our hand is now forces to define
    \[
        \Psi(x)_b(f) := Ff(x).
    \]

    

\end{proof}



\end{document}