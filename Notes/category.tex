\documentclass[12pt]{article}

% Packages
\usepackage[margin=1in]{geometry}
\usepackage{parskip}
\usepackage{amsmath, amsthm, amssymb}
\usepackage{mathrsfs}
\usepackage{tikz, tikz-cd}
\usepackage[shortlabels]{enumitem}

\usepackage{suffix}
\usetikzlibrary{decorations.pathmorphing}

% Problem Box
\setlength{\fboxsep}{4pt}
\newlength{\myparskip}
\setlength{\myparskip}{\parskip}
\newsavebox{\savefullbox}
\newenvironment{fullbox}{\begin{lrbox}{\savefullbox}\begin{minipage}{\dimexpr\textwidth-2\fboxsep\relax}\setlength{\parskip}{\myparskip}}{\end{minipage}\end{lrbox}\framebox[\textwidth]{\usebox{\savefullbox}}}

% Environments
\setlist[enumerate]{nosep}
\newcommand{\keyword}[1]{\textbf{#1}}
\newcommand{\sepline}{\rule{\textwidth}{0.4pt}}

% Tikz Environments
\newenvironment{drawing}{\begin{center}\begin{tikzpicture}}{\end{tikzpicture}\end{center}}
% \tikzcdset{row sep/normal=0pt}
\newenvironment{cd}{\begin{center}\begin{tikzcd}}{\end{tikzcd}\end{center}}


% Document Formatting
\newtheoremstyle{mythmstyle}% name of the style to be used
  { }% measure of space to leave above the theorem. E.g.: 3pt
  { }% measure of space to leave below the theorem. E.g.: 3pt
  { }% name of font to use in the body of the theorem
  { }% measure of space to indent
  {\scshape}% name of head font
  {.}% punctuation between head and body
  { }% space after theorem head; " " = normal interword space
  {\thmname{#1}\thmnumber{ #2}\thmnote{ (#3)}}% Manually specify head

\theoremstyle{definition}
\newtheorem{theorem}{Theorem}
\newtheorem{corollary}{Corollary}
\newtheorem{lemma}{Lemma}
\newtheorem{proposition}{Proposition}


% Math Formatting
\newcommand{\isp}[1]{\quad\text{#1}\quad}

% mathbb
\newcommand{\N}{\mathbb{N}}
\newcommand{\Z}{\mathbb{Z}}
\newcommand{\Q}{\mathbb{Q}}
\newcommand{\R}{\mathbb{R}}
\newcommand{\C}{\mathbb{C}}

% mathcal
\renewcommand{\AA}{\mathcal{A}}
\newcommand{\BB}{\mathcal{B}}
\newcommand{\CC}{\mathcal{C}}
\newcommand{\DD}{\mathcal{D}}
\newcommand{\EE}{\mathcal{E}}
\newcommand{\OO}{\mathcal{O}}

% Symbols

\newcommand{\eps}{\varepsilon}
\renewcommand{\phi}{\varphi}
\renewcommand{\emptyset}{\varnothing}

% Delimiters
\newcommand{\<}{\left\langle}
\renewcommand{\>}{\right\rangle}

% Relations
\newcommand{\iso}{\cong}
\newcommand{\seq}{\subseteq}
\newcommand{\teq}{\trianglelefteq}
\newcommand{\tensor}{\otimes}

\newcommand{\inc}{\hookrightarrow}
\newcommand{\surj}{\twoheadrightarrow}
\newcommand{\To}{\longrightarrow}
\newcommand{\tto}{\rightrightarrows}
\newcommand{\Mapsto}{\longmapsto}

\newcommand{\nato}{\Rightarrow}
\newcommand{\Nato}{\Longrightarrow}


\newcommand{\eqby}[1]{\overset{\mathrm{(#1)}}{=}}

% Math Operators
\DeclareMathOperator{\Ob}{Ob}
\DeclareMathOperator{\Mor}{Mor}
\DeclareMathOperator{\Hom}{Hom}
\DeclareMathOperator{\Iso}{Iso}
\DeclareMathOperator{\End}{End}
\DeclareMathOperator{\Aut}{Aut}



\DeclareMathOperator{\dom}{dom}
\DeclareMathOperator{\cod}{cod}
\newcommand{\op}{\mathrm{op}}


\DeclareMathOperator{\id}{id}
\DeclareMathOperator{\im}{im}
\DeclareMathOperator{\Tor}{Tor}
\DeclareMathOperator{\Ann}{Ann}

% Other
\newcommand{\eqc}{\overline}
\newcommand{\udl}{\underline}

% Category Names
\newcommand{\mathcat}{\mathsf}
\newcommand{\newcat}[2]{\newcommand{#1}{\mathcat{#2}}}
\WithSuffix\newcommand\newcat*[2]{\WithSuffix\newcommand#1*{\mathcat{#2}}}


\newcat{\Set}{Set}

\newcat{\Top}{Top}
\newcat{\Htpy}{Htpy}

\newcat{\Mon}{Mon}
\newcat{\CMon}{CMon}
\newcat{\Grp}{Grp}
\newcat{\Ab}{Ab}

\newcat{\Ring}{Ring}
\newcat{\CRing}{CRing}

\newcat{\Mod}{\text{-}Mod}
\newcat*{\Mod}{Mod}

\newcat{\lMod}{\text{-}Mod}
\newcat{\rMod}{Mod\text{-}}

\newcat{\lmod}{\text{-}mod}
\newcat{\rmod}{mod\text{-}}


\newcat{\Vect}{\text{-}Vect}
\newcat*{\Vect}{Vect}
\newcat{\Comp}{\text{-}Comp}
\newcat*{\Comp}{Comp}

\newcat{\Cat}{Cat}
\newcat{\CAT}{CAT}

\renewcommand{\_}[1]{{_{#1}}}



\title{Category \\
    \large 
}
\author{}
\date{}


\begin{document}

We make reference to set theory concepts.

Set theory is not a strictly conceptually necessary prerequisite, but is helpful for understanding.

The word ``collection'' is used in a generally nonmathematical way, i.e., its typical usage in natural language (here English).
However, we will mostly refer to collections of abstract things, as opposed to physical things.

By convention, our language will make no distinction between the various ways in which certain things (arguably) exist---e.g., physically, abstractly, hypothetically, etc.---all modes of existence are considered to be the same.

There will be no discussion of subatomic concepts.
The first definition we give will not rely on any prior mathematical foundation, and it will be assumed that the reader is able to reconcile any personal concerns over the nature of sense and reference.

\sepline

A note on notation.

In a moment I am going to say the following common mathematical sentence.[see footnote]:
\begin{enumerate}[(0)]
    \item Let $S$ be a set.
\end{enumerate}
I will clarify that I have not yet said the sentence (0) with the intent of expressing its meaning.
So far, I have only presented it as linguistic thing to be discussed.
This is very subtly contrasted with the next paragraph in which I will say (0) with the intent of expressing its meaning.
It is intended that (0) be read and understood in the usual way.
The sentence itself it not meant to be confusing or deceptive, though what follows may be rather pedantic.

Let $S$ be a set.

I really must emphasize here that the previous paragraph is fundamentally different from the first occurrence of the sentence (0) in this text.
At its first occurrence, there were no mathematical objects present in the discussion.
Only as it occurs in the previous paragraph does (0) have any mathematical meaning.
I will repeat that the meaning of (0) is the (hopefully) obvious one, so the previous paragraph has the same meaning within the context of this text as it would within any other.

I would now like to consider a number of related sentences, which will be enumerated.
Unlike (0), these will be explicitly declarative statements.
The truth of these statements is unclear and for that reason I will be saying none of them with the intent of expressing their meaning.

\begin{enumerate}[(1)]
    \item $S$ is a set.
    \item $S$ denotes a set.
    \item ``$S$'' denotes a set.
    \item $S$ is empty.
    \item $S$ has a cardinality.
    \item The cardinality of $S$ is $3$.
\end{enumerate}



[footnote] \textit{This is of course metaphorical.
By ``say'' I mean ``write,'' but more exactly I mean that the reader is experiencing a sort of metaphorical dialogue with the text.
For the most part, the text can be seen as a lecturer whose script is the text itself.
When the reader is engaged, they will listen (read) relatively swiftly and with focus.
When the text becomes boring, the reader may choose to end the conversation.
When the reader is confused, they may ask the lecturer to repeat something they previously said.
And so on and so forth.
By ``moment'' I mean a moment in the context of this metaphorical conversation, in which real time is passing to the reader.
In literal terms, I mean a short distance down the page, or possibly on the next page.
Though not likely further than that.}


\newpage
\sepline

A \keyword{category} $\CC$ is given by the following data:
\begin{itemize}
    \item a collection of \keyword{objects}, denoted $\Ob(\CC)$;
    \item for each $x, y \in \Ob(\CC)$, a collection of \keyword{morphisms/arrows/maps} from $x$ to $y$, denoted $\Mor(x, y)$;
    \item for each $x, y, z \in \Ob(\CC)$, a function
    \begin{align*}
        \Mor(y, z) \times \Mor(x, y) &\To \Mor(x, z) \\
            (g, f) &\Mapsto g \circ f
    \end{align*}
    called \keyword{composition};
    \item for each $x \in \Ob(\CC)$, an \keyword{identity morphism} $1_x \in \Mor(x, x)$;
\end{itemize}
such that the following axioms hold:
\begin{itemize}
    \item (ass) $(h \circ g) \circ f = h \circ (g \circ f)$ for all $f \in \Mor(x, y)$, $g \in \Mor(y, z)$, $h \in \Mor(z, w)$;
    \item(id) $f \circ 1_x = f = 1_y \circ f$ for all $f \in \Mor(x, y)$.
\end{itemize}

\sepline

Remarks.

Notation we often use:
\begin{itemize}
    \item $x \in \CC$ to mean $x \in \Ob(\CC)$;
    \item $\Mor_\CC(x, y)$ to mean $\Mor(x, y)$ to distinguish the category $\CC$;
    \item $\Mor(\CC)$ to mean the collection of all morphisms in $\CC$, i.e., $\bigsqcup_{x, y \in \CC} \Mor(x, y)$;
    \item $f : x \to y$ to mean $f \in \Mor(x, y)$.
\end{itemize}

Notation we rarely use (if at all, but is prevalent elsewhere):
\begin{itemize}
    \item $\CC(x, y)$ to mean $\Mor(x, y)$ to distinguish the category $\CC$;
    \item  $x \xrightarrow{f} y$ to mean $f \in \Mor(x, y)$;
    \item $gf$ to mean $g \circ f$.
\end{itemize}

For $f : x \to y$, call $x$ the \keyword{domain} (or \keyword{source}) of $f$ and $y$ the \keyword{codomain} (or \keyword{target}) of $f$.


\sepline

Some categories.
\begin{center}
    \begin{tabular}{c|c|c}
        Category & Objects & Morphisms \\
        \hline
        $\Set$ & sets & functions \\
        $(P, \leq)$ & elements of $P$ & relations in the partial order $\leq$ \\
        $\Top$ & topological spaces & continuous maps \\
        $\Htpy$ & topological spaces & homotopy classes of continuous maps \\
        $\OO(X)$ & open subsets of $X$ & inclusions of subsets \\
        $\Grp$ & groups & group homomorphisms \\
        $\Ab$ & abelian groups & group homomorphisms \\
        $\Ring$ & rings & ring homomorphisms \\
        $\CRing$ & commutative rings & ring homomorphisms \\
        $R\Mod$ & left $R$-modules & left $R$-module homomorphisms \\
        $\Mod*\text{-}R$ & right $R$-modules & right $R$-module homomorphisms \\
        $k\Vect$ & $k$-vector spaces & $k$-linear transformations
    \end{tabular}
\end{center}

Note $\Ab = \Z\Mod$ and $k\Vect = k\Mod$.

\sepline

We often discuss \keyword{diagrams} in a category.
This is a drawing of some objects in the category and some morphisms between them.
The objects are typically drawn as their names and the morphisms as labeled arrows from their domain to their codomain.
For instance, given objects $x, y, z, w \in \CC$ and morphisms $f : x \to y$, $g : x \to z$, $h : y \to w$, $k : z \to w$, we say that we have the the following diagram in $\CC$:
\begin{cd}
    x \rar["f"] \dar["g"'] & y \dar["h"] \\
    z \rar["k"'] & w
\end{cd}

One can think of such a diagram $\CC$ as a drawing of a directed graph whose vertices correspond to objects of $\CC$ and whose (directed) edges correspond to morphisms in $\CC$.
A finite path in this graph corresponds to an ordered sequence of morphisms in $\CC$ such that the codomain of each morphism is the domain of the following morphism.
We can therefore compose these morphisms in the specified order to obtain another morphism in the $\CC$.
We say that a diagram \keyword{commutes} if any two paths in the graph with the same source and target correspond to the same morphism of objects in $\CC$.

In the above diagram, there are two paths from $x$ to $w$, corresponding to the composite morphisms $h \circ f$ and $k \circ g$.
We would say that this diagram commutes if $h \circ f = k \circ g$.

\sepline

Fix a category $\CC$.

A morphism $f : x \to y$ is called an \keyword{isomorphism} if there exists a morphism $g : y \to x$ such that $g \circ f = 1_x$ and $f \circ g = 1_y$.
Equivalently, $f$ is an isomorphism if there exists $g : y \to x$ such that the following diagram commutes:
\begin{cd}
    x \ar[loop left, "1_x"] \rar[shift left=1, "f"] & y \ar[loop right, "1_y"] \lar[shift left=1, "g"]
\end{cd}
In which case, $g$ is called the[footnote] \keyword{inverse} of $f$, denoted $f^{-1} = g$.

Additionally, we say $x$ and $y$ are \keyword{isomorphic}, written $x \iso y$.

An \keyword{endomorphism} is a morphism whose domain and codomain are the same.

An \keyword{automorphism} is an endomorphism which is also an isomorphism.

Define the collections
\begin{itemize}
    \item $\Iso(\CC) \seq \Mor(\CC)$ of all isomorphisms in $\CC$;
    \item $\Iso(x, y) = \Mor(x, y) \cap \Iso(\CC)$ of all isomorphisms $x \to y$;
    \item $\End(\CC) \seq \Mor(\CC)$ of all endomorphisms in $\CC$;
    \item $\End(x) = \Mor(x, x)$ of all endomorphisms of $x$;
    \item $\Aut(\CC) = \End(\CC) \cap \Iso(\CC)$ of all automorphisms in $\CC$;
    \item $\Aut(x) = \End(x) \cap \Iso(\CC) = \Iso(x, x)$ of all automorphisms $x$;
\end{itemize}

Examples
\begin{center}
    \begin{tabular}{c|c}
        Category & Isomorphisms \\\hline
        $\Set$ & bijections \\
        $\Top$ & homeomorphisms \\
        $\Htpy$ & homotopy equivalences \\
        $\Grp$, $\Ring$, $R\Mod$ & bijective homomorphisms
    \end{tabular}
\end{center}

\sepline

A \keyword{groupoid} is a category in which every morphism is an isomorphism.

As both a definition and example: a \keyword{group} is a groupoid with only one object.

A \keyword{subcategory} $\DD$ of a category $\CC$ is a category such that $\Ob(\DD) \seq \Ob(\CC)$ and $\Mor_\DD(x, y) \seq \Mor_\CC(x, y)$ for all $x, y \in \DD$, and we write $\DD \seq \CC$.
(e.g., $\Ab \seq \Grp$ and $\CRing \seq \Ring$.)
It is sometimes required that $\Mor_\DD(x, y) = \Mor_\CC(x, y)$ for all $x, y \in \DD$.

\sepline

mono and epic



\sepline

A category $\CC$ is called
\begin{itemize}
    \item \keyword{small} if $\Mor(\CC)$ is a set;
    \item \keyword{locally small} if $\Mor(x, y)$ is a set for all $x, y \in \CC$ (in which case, we usually write $\Hom$ instead of $\Mor$); 
\end{itemize}

\sepline

For any category $\CC$, we construct its \keyword{opposite} (or \keyword{dual}) category $\CC^\op$ as follows:
\begin{itemize}
    \item $\Ob(\CC^\op) = \Ob(\CC)$;
    \item for each $f \in \Mor_\CC(x, y)$, a morphism $f^\op \in \Mor_{\CC^\op}(y, x)$;
    \item composition $f^\op \circ g^\op = (g \circ f)^\op$ for all $f \in \Mor_\CC(x, y)$ and $g \in \Mor_\CC(y, z)$;
    \item for each $x \in \CC^\op$, an identity $1_x^\op$.
\end{itemize}

\sepline

Duality stuff.
If you do a category thing, reverse all the arrows to get a dual thing.

\sepline

Fix an object in a category $c \in \CC$.

The \keyword{slice category} of $\CC$ \keyword{over} $c$ is a category $\CC/c$ with
\begin{itemize}
    \item objects $\Ob(\CC/c) = \bigcup_{x \in \CC} \Mor_\CC(x, c) = \{f : x \to c \mid x \in \CC\}$;
    \item morphisms: $\Mor_{\CC/c}(f : x \to c, g : y \to c) = \{h : x \to y \mid f = g \circ h\}$, i.e., $\Mor_{\CC/c}(f, g)$ is the collection of morphisms $h : x \to y$ such that the following diagram commutes:
    \begin{cd}[row sep=small, column sep=tiny]
        x \drar["f"'] \ar[rr, "h"] && y \dlar["g"] \\
        & c
    \end{cd}
\end{itemize}

The slice category of $\CC$ \keyword{under} $c$ is a category $c/\CC$ with
\begin{itemize}
    \item objects $\Ob(c/\CC) = \bigcup_{x \in \CC} \Mor_\CC(c, x) = \{f : c \to x \mid x \in \CC\}$;
    \item morphisms: $\Mor_{c/\CC}(f : c \to x, g : c \to y) = \{h : x \to y \mid g = h \circ f\}$, i.e., $\Mor_{c/\CC}(f, g)$ is the collection of morphisms $h : x \to y$ such that the following diagram commutes:
    \begin{cd}[row sep=small, column sep=tiny]
        & c \dlar["f"'] \drar["g"] \\
        x \ar[rr, "h"'] && y
    \end{cd}
\end{itemize}

These are dual notions in the sense that $c/\CC = (\CC^\op/c)^\op$ and $\CC/c = (c/\CC^\op)^\op$.

\sepline

A (\keyword{covariant}) \keyword{functor} $F : \CC \to \DD$ between categories is given by the following data:
\begin{itemize}
    \item for each object $x \in \CC$, an object $F(x) = Fx \in \DD$;

    equiv, a function $F_0 : \Ob(\CC) \to \Ob(\DD)$
    \item for each morphism $f \in \Mor_\CC(x, y)$, a morphism $F(f) = Ff \in \Mor_\DD(x, y)$;

    equiv, a function $F_{x,y} : \CC(x, y) \to \DD(Fx, Fy)$
\end{itemize}
such that the following \keyword{functoriality axioms} hold:
\begin{itemize}
    \item $F(g \circ f) = Fg \circ Ff$ for all $f : x \to y$ and $g : y \to z$ in $\CC$;
    \item $F(1_x) = 1_{Fx}$ for all $x \in \CC$.
\end{itemize}

Remark: A functor $\CC \to \DD$ sends isomorphisms in $\CC$ to isomorphisms in $\DD$.

A \keyword{contravariant functor} $\CC \to \DD$ is given by the data of a covariant functor $\CC^\op \to \DD$.

\sepline

Given functors $F : \CC \to \DD$ and $G : \DD \to \EE$ there is a functor $G \circ F : \CC \to \EE$ defined in the obvious way.


\sepline

There is a category $\Cat$ whose objects are small categories and whose morphisms are functors.

There is a category $\CAT$ whose objects are locally small categories and whose morphisms are functors.


\sepline

A \keyword{presheaf of sets} on a category $\CC$ is a functor $\CC^\op \to \Set$ (i.e., a contravariant functor $\CC \to \Set$).

For a topological space $X$, a presheaf of sets on $X$ is a presheaf of sets on the category $\OO(X)$.

\sepline

Fix an object in a locally small category $c \in \CC$.

We construct the following pair of covariant and contravariant \keyword{functors represented} by $c$:
\begin{center}
    \begin{tikzcd}[row sep=tiny, column sep=tiny]
        \CC \ar[rr, "{\Mor(c, -)}"] && \DD \\
        x \ar[dd, "f"'] & \Mapsto & \CC(c, x) \ar[dd, "f_*"] \\
        & \Mapsto \\
        y & \Mapsto & \CC(c, y)
    \end{tikzcd}
    \hspace{2cm}
    \begin{tikzcd}[row sep=tiny, column sep=tiny]
        \CC \ar[rr, "{\Mor(-, c)}"] && \DD \\
        x \ar[dd, "f"'] & \Mapsto & \CC(x, c)\\
        & \Mapsto \\
        y & \Mapsto & \CC(y, c) \ar[uu, "f^*"'] 
    \end{tikzcd}
\end{center}



\sepline

For any categories $\CC$ and $\DD$ there is a category $\CC \times \DD$ called their \keyword{product} with
\begin{itemize}
    \item objects are ordered pairs $(c, d)$ with $c \in \CC$ and $d \in \DD$;
    
    i.e., $\Ob(\CC \times \DD) = \Ob(\CC) \times \Ob(\DD)$
    \item morphisms are ordered pairs $(f, g) : (c, d) \to (c', d')$ where $f : c \to c' \in \CC$ and $g : d \to d' \in \DD$.
    
    i.e., $\Mor_{\CC \times \DD}((c, d), (c', d')) = \Mor_{\CC}(c, c') \times \Mor_{\DD}(d, d')$

    i.e., $\Mor(\CC \times \DD) = \Mor(\CC) \times \Mor(\DD)$
\end{itemize}

\sepline

A \keyword{bifunctor} is a functor whose domain is a product category, e.g., $F : \AA \times \BB \to \CC$.
For each $a \in \AA$ and $b \in \BB$ there are functors
\begin{center}
    \begin{tikzcd}[row sep=tiny, column sep=tiny]
        \BB \ar[rr, "{F(a, -)}"] && \CC \\
        b \ar[dd, "g"'] & \Mapsto & F(a, b) \ar[dd, "{F(1_a, g)}"] \\
        & \Mapsto \\
        b' & \Mapsto & F(a, b')
    \end{tikzcd}
    \hspace{2cm}
    \begin{tikzcd}[row sep=tiny, column sep=tiny]
        \AA \ar[rr, "{F(-, b)}"] && \CC \\
        a \ar[dd, "f"'] & \Mapsto & F(a, b) \ar[dd, "{F(f, 1_b)}"] \\
        & \Mapsto \\
        a' & \Mapsto & F(a', b)
    \end{tikzcd}
\end{center}


\sepline

If $\CC$ is locally small, there is a \keyword{two-sided represented functor}
\[
    \Mor_\CC(-, -) : \CC^\op \times \CC \to \Set.
\]

In other words, $\Mor_\CC(-, -)$ is a bifunctor such that each $\Mor_\CC(c, -)$ and $\Mor_\CC(-, c)$ is a (obviously) a represented functor.

\sepline

Given functors $F : \AA \to \CC$ and $G : \BB \to \CC$ there is a \keyword{comma category} $F \downarrow G$ with
\begin{itemize}
    \item objects are triples $(a \in \AA, b \in \BB, f: Fa \to Gb \in \CC)$;
    \item morphisms $(a, b, f) \to (a', b', f')$ are pairs $(h : a \to a', k : b \to b')$ such that the following diagram commutes:
    \begin{cd}
        Fa \dar["Fh"'] \rar["f"] & Gb \dar["Gk"] \\
        Fa' \rar["f'"'] & Gb'
    \end{cd}
\end{itemize}

There are projection functors $\dom : F \downarrow G \to \AA$ and $\cod : F \downarrow G \to \BB$.

\begin{center}
    \begin{tikzcd}[row sep=tiny, column sep=tiny]
        F \downarrow G \ar[rr, "\dom"] && \AA \\
        (a, b, f) \ar[dd, "{(h, k)}"'] & \Mapsto & a \ar[dd, "h"] \\
        & \Mapsto \\
        (a', b', f') & \Mapsto & a'
    \end{tikzcd}
    \hspace{2cm}
    \begin{tikzcd}[row sep=tiny, column sep=tiny]
        F \downarrow G \ar[rr, "\cod"] && \BB \\
        (a, b, f) \ar[dd, "{(h, k)}"'] & \Mapsto & b \ar[dd, "k"] \\
        & \Mapsto \\
        (a', b', f') & \Mapsto & b'
    \end{tikzcd}
\end{center}

\sepline

Can construct slice categories $\CC/c$ and $c/\CC$ as special cases of comma categories.

\sepline

Given functors $F, G : \CC \tto \DD$, a \keyword{natural transformation} $\alpha : F \nato G$ is given by the data of a morphism $\alpha_x : Fx \to Gx \in \DD$ for each $x \in \CC$, the collection of which are called the \keyword{components} of $\alpha$, such that for all $f : x \to y \in \CC$ the following diagram in $\DD$ commutes:
\begin{cd}
    Fx \dar["Ff"'] \rar["\alpha_x"] & Gx \dar["Gf"] \\
    Fy \rar["\alpha_y"'] & Gy
\end{cd}

In the case that $\alpha$ is a natural transformation from $F$ to $G$. we express this fact with the following diagram:
\begin{cd}
    \CC
        \rar[bend left=50, "F" name=F]
        \rar[bend right=40, "G"' name=G]
        \ar[from=F, to=G, shorten=1ex, Rightarrow, "\alpha"]
    & \DD
\end{cd}

We say that $\alpha$ is a \keyword{natural isomorphism} if each component $\alpha_x : Fx \to Gx$ is an isomorphism in $\DD$; sometimes write this as $\alpha : F \iso G$.

\sepline

Consider the opposite map $(-)^\op : \Grp \to \Grp$; this is a covariant endofunctor.
A homomorphism $\phi : G \to H$ induces a homomorphism $\phi^\op : G^\op \to H^\op$, which behaves the same as $\phi$ on elements.
Moreover, this functor is naturally isomorphic to the identity.
For each group $G \in \Grp$ define the map $\eta_G : G \to G^\op$ sending $g \in G$ to its inverse  $g^{-1} \in G^\op$.
This is not an automorphism of $G$ because taking the inverse does not in general commute with multiplication---in fact, it reverses multiplication.
In other words, $\eta_G$ defines an isomorphism $G \to G^\op$.
And given any homomorphism $\phi : G \to H$, thew following diagram commutes:
\begin{cd}
    G \dar["\phi"'] \rar["\eta_G"] & G^\op \dar["\phi^\op"] \\
    H \rar["\eta_H"'] & H^\op
\end{cd}
Hence, we have a natural isomorphism
\begin{cd}
    \Grp
        \rar[bend left=40, "1" name=U]
        \rar[bend right=30, "(-)^\op"' name=L]
        \ar[from=U, to=L, shorten=1ex, Rightarrow, "\eta"]
    & \Grp
\end{cd}

\sepline

Fix a locally small category $\CC$.

Let $f : x \to y$ and $h : z \to w$ be morphisms.
Post-composition by $h$ and precomposition by $f$ make the following diagram commute:
\begin{cd}
    \CC(y, z) \dar["- \circ f"'] \rar["h \circ -"] & \CC(y, w) \dar["- \circ f"] \\
    \CC(x, z) \rar["h \circ -"'] & \CC(x, w)
\end{cd}
The commutativity of this diagram follows from the associativity of composition, which implies the commutativity of the following diagram for any $g : y \to z$:
\begin{cd}
    x \rar["f" description]
        \ar[rr, bend left, "g \circ f" description]
        \ar[rrr, bend left=40, "h \circ (g \circ f)" description]
        \ar[rrr, bend right=40, "(h \circ g) \circ f" description]
    & y \rar["g" description] 
        \ar[rr, bend right, "h \circ g" description]
    & z \rar["h" description]
    & w
\end{cd}

We therefore obtain natural transformations
\[
    h_* : \CC(-, z) \Nato \CC(-, w) \isp{and} f^* : \CC(y, -) \Nato \CC(x, -).
\]

\sepline

Consider the category of sets $\Set$.

For $A, B \in \Set$, recall that $A \times B$ is their cartesian product, $A \sqcup B$ their disjoint union, and $A^B$ the set of functions from $B$ to $A$.
These define functors
\begin{align*}
    \Set \times \Set &\To \Set & \Set \times \Set &\To \Set & \Set^\op \times \Set &\To \Set \\
    (A, B) &\Mapsto A \times B & (A, B) &\Mapsto A \sqcup B & (A, B) &\Mapsto B^A
\end{align*}
(The third is the two-sided represented functor $\Set(-, -)$ previously discussed.)
Then we have the following natural isomorphisms:
\begin{align*}
    A \times (B \sqcup C) &\iso (A \times B) \sqcup (A \times C)   & (A \times B)^C &\iso A^C \times B^C \\
    A^{B \sqcup C} &\iso A^B \times A^C 
        & (A^B)^C &\iso A^{B \times C}
\end{align*}

\sepline

Let $F : \AA \to \CC$ and $G : \BB \to \CC$ be functors.
Consider the comma category $F \downarrow G$.
There is a natural transformation $\alpha : F\dom \nato G\cod$ with components
\[
    \alpha_{(a, b, f)} = f : Fa \To Gb.
\]
The naturally is immediate from the construction of the comma category, hence the following diagram commutes:
\begin{cd}[row sep=small, column sep=tiny]
    & F \downarrow G \dlar["\dom"'] \drar["\cod"]\\
    \AA \drar["F"'] \ar[rr, shorten=1.5em, Rightarrow, "\alpha"] & & \BB \dlar["G"] \\
    & \CC
\end{cd}

\sepline

An \keyword{equivalence of categories} is given by
\begin{itemize}
    \item functors $F : \CC \to \DD$ and $G : \DD \to \CC$;
    \item natural isomorphisms $\eta : 1_\CC \iso GF$ and $\eps : FG \iso 1_\DD$.
\end{itemize}
Say $\CC$ and $\DD$ are \keyword{equivalent}, written $\CC \simeq \DD$.

The notion of category equivalence is an equivalence relation.

\sepline

A functor $F : \CC \to \DD$ is called
\begin{itemize}
    \item \keyword{full} if each map $F_{x,y} : \CC(x, y) \to \DD(Fx, Fy)$ is surjective (i.e., ``surjective on morphisms'');
    \item \keyword{faithful} if each map $F_{x,y} : \CC(x, y) \to \DD(Fx, Fy)$ is injective (i.e., ``injective on morphisms'');
    \item \keyword{essentially surjective on objects} if each $y \in \DD$ is isomorphic to some $Fx$.
\end{itemize}

An \keyword{embedding} is a faithful functor which is injective on objects.

A \keyword{fully faithful} functor is both full and faithful.

A \keyword{full embedding} is a fully faithful functor which is injective on objects, defining a \keyword{full subcategory} of the codomain.

\sepline



\begin{theorem}[equiv]
    A functor defining an equivalence of categories is full, faithful, and essentially surjective on objects.
    With axiom of choice, any functor with these properties defines an equivalence of categories.
\end{theorem}

\begin{lemma}
    Any morphism $f : x \to y$ and fixed isomorphisms $x \iso x'$ and $y \iso y'$ determine a unique morphism $f' : x' \to y'$ so that any of---or, equivalently, all of---the following four diagrams commute:
    \begin{center}
        \begin{tikzcd}
            x \dar["f"'] \ar[from=r, tail, two heads] & x' \dar["f'"] \\
            y \rar[tail, two heads] & y'
        \end{tikzcd}
        \hspace{1em}
        \begin{tikzcd}
            x \dar["f"'] \rar["\iso"] & x' \dar["f'"] \\
            y \rar["\iso"'] & y'
        \end{tikzcd}
        \hspace{1em}
        \begin{tikzcd}
            x \dar["f"'] \ar[from=r, "\sim"'] & x' \dar["f'"] \\
            y \ar[from=r, "\sim"] & y'
        \end{tikzcd}
        \hspace{1em}
        \begin{tikzcd}
            x \dar["f"'] \rar["\iso"'] & x' \dar["f'"] \\
            y \ar[from=r, "\iso"] & y'
        \end{tikzcd}
    \end{center}
\end{lemma}

\begin{proof}[Proof of Theorem]
    Let $F : \CC \to \DD$ be a functor.

    Suppose $F$ defines an equivalence of categories; let $G : \DD \to \CC$ be an inverse functor with natural isomorphisms $\eta : 1_\CC \iso GF$ and $\eps : GF \iso 1_\DD$.

    (ess. surj.) 
    Given $y \in \DD$, the natural isomorphism $\eps$ has component $\eps_y : FGy \iso y$.
    
    (faithful)
    Suppose $f, g \in \CC(x, y)$ such that $Ff = Fg$.
    Then $GFf = GFg$, so both $f$ and $g$ make the following diagram commute:
    \begin{cd}
        x \dar[dashed] \rar["\eta_x"] & GFx \dar["GFf = GFg"] \\
        y \rar["\eta_y"'] & GFy
    \end{cd}
    By the lemma, we must have $f = g$.
        
    (full) Let $g \in \DD(Fx, Fy)$ be given.
    By the lemma, there is a unique $f \in \CC(x, y)$ such that the following diagram commutes:
    \begin{cd}
        x \dar["f"'] \rar["\eta_x"] & GFx \dar["Gg"] \\
        y \rar["\eta_y"'] & GFy
    \end{cd}
    But then the right arrow can be filled by $GFf$, i.e.,
    \begin{cd}
        x \dar["f"'] \rar["\eta_x"] & GFx \dar["GFf"] \\
        y \rar["\eta_y"'] & GFy
    \end{cd}
    also commutes.
    By the lemma, we must have $Gg = GFf$.
    Applying the faithfulness result to $G$, we must therefore have $g = Ff$.

    Hence, $F$ is fully faithful and essentially surjective.

    Now suppose $F$ is fully faithful and essentially surjective.
    We will construct an functor $G : \DD \to \CC$ inverse to $F$.
    For each object $y \in \DD$, the fact that $F$ is essentially surjective means we can choose an object $Gx \in \CC$ and isomorphism $\eps_x : FGx \iso x$.
    Then we have bijections
    \begin{cd}[column sep=huge]
        \CC(Gx, Gy) \rar["F_{Gx, Gy}"] & \DD(FGx, FGy) \rar["\eps_y \circ - \circ \eps_x^{-1}"] & \DD(x, y)
    \end{cd}
    Define $G_{x, y}$ as the inverse of this map.
    We now check the functorality axioms of $G$.
    First,
    \[
        \eps_x \circ F1_{Gx} \circ \eps_x^{-1}
            = \eps_x \circ 1_{FGx} \circ \eps_x^{-1}
            = \eps_x \circ \eps_x^{-1}
            = 1_x,
    \]
    which implies $G1_x = 1_{Gx}$.
    Given $f : x \to y$ and $g : y \to z$ are morphisms in $\DD$, $G(g \circ f)$ is the unique morphism $Gx \to Gz$ in $\CC$ such that
    \[
        \eps_z \circ FG(g \circ f) \circ \eps_x^{-1} = g \circ f.
    \]
    But $Gf$ and $Gg$ are the unique morphisms such that
    \[
        \eps_y \circ FGf \circ \eps_x^{-1} = f
        \isp{and} 
        \eps_z \circ FGg \circ \eps_y^{-1} = g.
    \]
    Then
    \begin{align*}
        g \circ f
            &= (\eps_z \circ FGg \circ \eps_y^{-1}) \circ (\eps_y \circ FGf \circ \eps_x^{-1}) \\
            &= \eps_z \circ FGg \circ FGf \circ \eps_x^{-1} \\
            &= \eps_z \circ F(Gg \circ Gf) \circ \eps_x^{-1},
    \end{align*}
    hence $G(g \circ f) = Gg \circ Gf$.
    This shows $G$ is indeed a functor.

    By the construction of $G$, we immediately see that the $\eps_x$'s define the components of a natural isomorphism $\eps : FG \nato 1_\DD$; in particular, the following diagram commutes:
    \begin{cd}
        FGx \dar["FGf"'] \rar["\eps_x"] & x \dar["f"] \\
        FGy \rar["\eps_y"'] & y
    \end{cd}
    We now need a natural isomorphism $\eta : 1_\CC \nato GF$.
    There is a bijection
    \begin{cd}[column sep=huge]
        \CC(x, GFx) \rar["F_{x, GFx}"] & \DD(Fx, FGFx),
    \end{cd}
    so we can define $\eta_x$ to be the inverse image of $\eps_{Fx}^{-1}$ under this map.
    We first check that $\eta_x$ is an isomorphism.
    Let $\alpha$ be the inverse image of $\eps_{Fx}$ under $F_{GFx, x}$.
    Then
    \[
        F(\alpha \circ \eta_x)
            = F\alpha \circ F\eta_x
            = \eps_{Fx} \circ \eps_{Fx}^{-1}
            = 1_{Fx}
            = F1_x
    \]
    and
    \[
        F(\eta_x \circ \alpha)
            = F\eta_x \circ F\alpha
            = \eps_{Fx}^{-1} \circ \eps_{Fx}
            = 1_{FGFx}
            = F1_{GFx}.
    \]
    But as $F$ is injective on morphisms, we deduce that
    \[
        \alpha \circ \eta_x = 1_x \isp{and} \eta_x \circ \alpha = 1_{GFx},
    \]
    hence $\eta_x$ is an isomorphism with $\eta_x^{-1} = \alpha$.
    Lastly, we check the naturality of $\eta$, which means checking the commutativity of the following diagram in $\CC$:
    \begin{cd}
        x \dar["f"'] \rar["\eta_x"] & GFx \dar["GFf"] \\
        y \rar["\eta_y"'] & GFy
    \end{cd}
    Taking this diagram under $F$, we obtain the following diagram in $\DD$:
    \begin{cd}[column sep=huge]
        Fx \dar["Ff"'] \rar["F\eta_x = \eps_{Fx}^{-1}"] & FGFx \dar["FGFf"] \\
        Fy \rar["F\eta_y = \eps_{Fy}^{-1}"'] & FGFy
    \end{cd}
    This is the naturality diagram for $\eps$ at $Ff$ and therefore commutes, i.e.,
    \[
        F(\eta_y \circ f)
            = F\eta_y \circ Ff
            = FGFf \circ F\eta_x
            = F(GFf \circ \eta_x).
    \]
    Since $F$ faithful, it follows that $\eta_y \circ f = GFf \circ \eta_x$.
    In other words, the naturality diagram for $\eta$ at $f$ commutes.
    Hence, $\eta$ is a natural isomorphism and we conclude that $G$ is an inverse functor of $F$, which in turn defines an equivalence of categories.
\end{proof}

\sepline

The following things are invariant under categorical equivalence:
\begin{itemize}
    \item If a category is locally small, any equivalent category is also locally small.
    \item If a category is a groupoid, any equivalent category is also a groupoid.
    \item If $\CC \simeq \DD$ then $\CC^\op \simeq \DD^\op$.
    \item If $\CC \simeq \CC'$ and $\DD \simeq \DD'$ then $\CC \times \DD \simeq \CC' \times \DD'$.
    \item A morphism in a category is an isomorphism if and only if its image under an equivalence is an isomorphism.
\end{itemize}


\sepline

A \keyword{monoidal category} is a category $\CC$ equipped with
\begin{itemize}
    \item A (bi-)functor $\tensor : \CC \times \CC \to \CC$ called the \keyword{monoidal product} (or \keyword{tensor product});
    \item A distinguished object $1 \in \CC$ called the \keyword{unit object} (or \keyword{tensor unit});
    \item A natural isomorphism
    \[
        \alpha : ((- \tensor -) \tensor -) \Nato (- \tensor (- \tensor -))
    \]
    with components $\alpha_{x,y,z} : (x \tensor y) \tensor z \to x \tensor (y \tensor z)$ called the \keyword{associator};
    \item A natural isomorphism
    \[
        \lambda : (1 \tensor -) \Nato 1_\CC
    \]
    with components $\lambda_x : 1 \tensor x \to x$ called the \keyword{left-unitor};
    \item A natural isomorphism
    \[
        \rho : (- \tensor 1) \Nato 1_\CC
    \]
    with components $\rho_x : x \tensor 1 \to x$ called the \keyword{right-unitor};
\end{itemize}
such that the following diagrams commute:

For each pair $(x, y) \in \CC \times \CC$ the \keyword{triangle indetity}:
\begin{cd}[column sep=tiny]
    (x \tensor 1) \tensor y \drar["\rho_x \tensor 1_y"']
        \ar[rr, "\alpha_{x,1,y}"]
    && x \tensor (1 \tensor y) \dlar["1_x \tensor \lambda_y"]
    \\
    & x \tensor y
\end{cd}


For each quadruple $(x, y, z, w) \in \CC \times \CC \times \CC \times \CC$ the \keyword{pentagon identity}:
\begin{cd}[column sep=-2em]
    &[-3em]&
        (x \tensor y) \tensor (z \tensor w)
        \ar[drr, "\alpha_{x,y,z\tensor w}"]
    \\
        ((x \tensor y) \tensor z) \tensor w
            \ar[urr, "\alpha_{x\tensor y,z,w}"]
            \ar[dr, "\alpha_{x,y,z} \tensor 1_w"']
    &&&&[-3em] 
        x \tensor (y \tensor (z \tensor w))
    \\[2ex]
    &
        (x \tensor (y \tensor z)) \tensor w
            \ar[rr, "\alpha_{x,y\tensor z,w}"']
    &&
        x \tensor ((y \tensor z) \tensor w)
            \ar[ur, "1_x \tensor \alpha_{y,z,w}"']
\end{cd}

A monoidal category is called \keyword{strict} if the associator and unitors are identity morphisms (in which case the identities hold automatically).

\sepline

In a monoidal category, the following additional triangle identities hold:
\begin{center}
    \begin{tikzcd}[column sep=small]
        (1 \tensor x) \tensor y 
            \ar[rr, "\alpha_{1, x, y}"]
            \drar["\lambda_x \tensor 1_y"']
        && 1 \tensor (x \tensor y)
            \dlar["\lambda_{x \tensor y}"]
        \\
        & x \tensor y
    \end{tikzcd}
    \hspace{2em}
    \begin{tikzcd}[column sep=small]
        (x \tensor y) \tensor 1 
            \ar[rr, "\alpha_{x, y, 1}"]
            \drar["\rho_{x \tensor y}"']
        && x \tensor (y \tensor 1)
            \dlar["1_x \tensor \rho_y"]
        \\
        & x \tensor y
    \end{tikzcd}
\end{center}



\begin{proof}
    (largely taken from nlab, which references MacLane and Kelly)

    Consider the following diagram in $\CC$:
    \begin{cd}[row sep=large, cramped]
        ((1 \tensor 1) \tensor x) \tensor y
            \rar["\alpha_{1, 1, x} \tensor 1_y"]
            \drar["(\rho_1 \tensor 1_x) \tensor 1_y"']
        & (1 \tensor (1 \tensor x)) \tensor y
            \rar["\alpha_{1, 1 \tensor x, y}"]
            \dar["(1_1 \tensor \lambda_x) \tensor 1_y"]
        & 1 \tensor ((1 \tensor x) \tensor y)
            \rar["1_1 \tensor \alpha_{1, x, y}"]
            \dar["1_1 \tensor (\lambda_x \tensor 1_y)"']
        & 1 \tensor (1 \tensor (x \tensor y))
            \dlar[dashed]
        \\
        & (1 \tensor x) \tensor y 
            \rar["\alpha_{1, x, y}"']
        & 1 \tensor (x \tensor y)
    \end{cd}
    The left triangle commutes as the image of the triangle identity for $(1, x)$ under the functor $- \tensor y$.
    The center square commutes by the naturality diagram of $\alpha_{1,-,y}$ at the morphism $\lambda_x$.
    The dashed arrow is obtained by simply following any suitable path in the diagram, as all the arrows are isomorphisms.
    We claim that the dashed arrow is in fact $1_1 \tensor \lambda_{x \tensor y}$.

    Since all the arrows are isomorphisms, it suffices to check that the morphism commutes with any other path in the diagram.
    In particular, we will check that it makes the perimeter commute.
    We will use the following diagram:
    \begin{cd}[row sep=large, cramped]
        & (1 \tensor (1 \tensor x)) \tensor y
            \rar["\alpha_{1, 1 \tensor x, y}"]
        & 1 \tensor ((1 \tensor x) \tensor y)
            \drar["1_1 \tensor \alpha_{1, x, y}"]
        \\
        ((1 \tensor 1) \tensor x) \tensor y
            \urar["\alpha_{1, 1, x} \tensor 1_y"]
            \ar[rr, "\alpha_{1 \tensor 1, x, y}"]
            \drar["(\rho_1 \tensor 1_x) \tensor 1_y"']
        && (1 \tensor 1) \tensor (x \tensor y)
            \rar["\alpha_{1, 1, x \tensor y}"]
            \dar["\rho_1 \tensor 1_{x \tensor y}"']
        & 1 \tensor (1 \tensor (x \tensor y))
            \dlar["1_1 \tensor \lambda_{x \tensor y}"]
        \\
        & (1 \tensor x) \tensor y 
            \rar["\alpha_{1, x, y}"']
        & 1 \tensor (x \tensor y)
    \end{cd}
    The top pentagon is the pentagon identity for $(1, 1, x, y)$, the bottom left square is the naturality of $\alpha_{-, x, y}$ at the morphism $\rho_1$ (tensor functorality gives $1_x \tensor 1_y = 1_{x \tensor y}$), and the bottom right triangle is the triangle identity for $(1, x \tensor y)$.
    Hence, the entire diagram commutes, which implies that $1_1 \tensor \lambda_{x \tensor y}$ makes the original diagram commute at the dashed arrow.
    In particular, the right triangle commutes:
    \begin{cd}[column sep=tiny]
        1 \tensor ((1 \tensor x) \tensor y)
            \ar[rr, "1_1 \tensor \alpha_{1, x, y}"]
            \drar["1_1 \tensor (\lambda_x \tensor 1_y)"']
        && 1 \tensor (1 \tensor (x \tensor y))
            \dlar["1_1 \tensor \lambda_{x \tensor y}"]
        \\
        &1 \tensor (x \tensor y)
    \end{cd}
    But this is simply the desired diagram mapped under $1 \tensor -$, which is an equivalence under the natural isomorphism $\lambda$.
    Either via the equivalence or by mapping the diagram under $\lambda$, we deduce that the second triangle identity holds.
    
    It is similar to prove the third triangle identity holds.
\end{proof}

\sepline

In a monoidal category, $\lambda_1 = \rho_1 : 1 \tensor 1 \to 1$.

\begin{proof}
    Since $- \tensor 1$ is an equivalence, it suffices to show $\lambda_1 \tensor 1_1 = \rho_1 \tensor 1_1$.
    The naturality of $\lambda$ at $\lambda_1$ some morphism gives the following commutative diagram:
    \begin{cd}
        1 \tensor(1 \tensor 1)
            \rar["\lambda_{1 \tensor 1}"]
            \dar["1_1 \tensor \lambda_1"']
        & 1 \tensor 1
            \dar["\lambda_1"] \\
        1 \tensor 1
            \rar["\lambda_1"']
            \urar[dashed, no head]
        & 1
    \end{cd}
    The bottom right triangle is simply $\lambda_1^{-1} \circ \lambda_1 = 1_{1 \tensor 1}$, so in fact we have $1_1 \tensor \lambda_1 = \lambda_{1 \tensor 1}$.
    Then the second and first triangle identity on $(1, 1)$ give
    \[
        \lambda_1 \tensor 1_1
            = \lambda_{1 \tensor 1} \circ \alpha_{1,1,1}
            = \rho_1 \tensor 1_1.
    \]
\end{proof}

\sepline

A \keyword{braided monoidal category} is a monoidal category, which is also equipped with 
\begin{itemize}
    \item A natural isomorphism $\gamma$ with components $\gamma_{x,y} : x \tensor y \to y \tensor x$ called the \keyword{braiding};
\end{itemize}
such that the \keyword{hexagon identities} holds, i.e., the following diagrams commute:
\begin{cd}[column sep=large]
    (x \tensor y) \tensor z
        \dar["\gamma_{x,y} \tensor 1_z"']
        \rar["\alpha_{x,y,z}"]
    & x \tensor (y \tensor z)
        \rar["\gamma_{x,y\tensor z}"]
    & (y \tensor z) \tensor x
        \dar["\alpha_{y,z,x}"] \\
    (y \tensor x) \tensor z
        \rar["\alpha_{y,x,z}"']
    & y \tensor (x \tensor z)
        \rar["1_y \tensor \gamma_{x,z}"]
    & y \tensor (z \tensor x)
\end{cd}
\begin{cd}[column sep=large]
    (x \tensor y) \tensor z
        \dar["\gamma_{x,y} \tensor 1_z"']
        \rar["\alpha_{x,y,z}"]
    & x \tensor (y \tensor z)
        \rar["\gamma_{x,y\tensor z}"]
    & (y \tensor z) \tensor x
        \dar["\alpha_{y,z,x}"] \\
    (y \tensor x) \tensor z
        \rar["\alpha_{y,x,z}"']
    & y \tensor (x \tensor z)
        \rar["1_y \tensor \gamma_{x,z}"]
    & y \tensor (z \tensor x)
\end{cd}

\sepline

Can maybe prove without symmetric?
\begin{cd}[column sep=small]
    x \tensor 1 \drar["\rho_x"'] \ar[rr, "\gamma_{x,1}"] && 1 \tensor x \dlar["\lambda_x"] \\
    & x
\end{cd}

\sepline

A \keyword{symmetric monoidal category} is a braided monoidal category in which the braiding satisfies the following additional triangle identity:
\begin{cd}[column sep=small]
    x \tensor y \drar["\gamma_{x,y}"'] \ar[rr, "1_{x\tensor y}"] && x \tensor y  \\
    & y \tensor x \urar["\gamma_{y,x}"']
\end{cd}

\sepline

A \keyword{strict monoidal category} is a monoidal category in which the left and right unitors are all identity morphisms (up to ambient structure morphisms. see nlab.)

A \keyword{cartesian monoidal category} is a monoidal category in which the monoidal product is given by the usual category theoretic product.

\sepline

Let $\CC$ be a monoidal category with product $\tensor$ and identity $I$.

A \keyword{monoid} (or \keyword{monoid object}) in $\CC$ (or \keyword{monoid internal to} $\CC$) consists of data:
\begin{itemize}
    \item an object $M \in \CC$;
    \item a morphism $\mu : M \tensor M \to M$ called \emph{multiplication};
    \item a morphism $\eta : I \to M$ called the \emph{unit};
\end{itemize}
such that
\begin{itemize}
    \item the \keyword{associative law} holds:
    \begin{cd}[column sep=0pt, row sep=small]
        (M \tensor M) \tensor M
            \drar["\mu \tensor 1_M"']
            \ar[rr, "\alpha_{M, M, M}"]
        && M \tensor (M \tensor M)
            \ar[rr, "1_M \tensor \mu"]
        && M \tensor M
            \dlar["\mu"]
        \\
        & M \tensor M
            \ar[rr, "\mu"']
        && M
    \end{cd}
    \item the \keyword{left and right unit laws} hold:
    \begin{cd}
        I \tensor M
            \drar["\lambda_M"']
            \rar["\eta \tensor 1_M"]
        & M \tensor M
            \dar["\mu"]
        & M \tensor 1
            \dlar["\rho_M"]
            \lar["1_M \tensor \eta"']
        \\
        & M
    \end{cd}
\end{itemize}

\sepline

Let $\CC$ be a symmetric monoidal category with braiding $\gamma$.

A \keyword{commutative monoid} in $\CC$ is a monoid $M \in \CC$ such that
\begin{itemize}
    \item \keyword{commutativity} holds:
    \begin{cd}[column sep=tiny]
        M \tensor M
            \drar["\mu"']
            \ar[rr, "\gamma_{M, M}"]
        && M \tensor M
            \dlar["\mu"]
        \\
        & M
    \end{cd}
\end{itemize}

\sepline

Let $M_1$ and $M_2$ be monoids in a monoidal category $\CC$.

A \keyword{monoid homomorphism} $f : M_1 \to M_2$ consists of 
\begin{itemize}
    \item a morphism $f : M_1 \to M_2 \in \CC$
\end{itemize}
such that
\begin{itemize}
    \item $f$ commutes with multiplication:
    \begin{cd}
        M_1 \tensor M_1
            \dar["\mu_1"']
            \rar["f \tensor f"]
        & M_2 \tensor M_2
            \dar["\mu_2"]
        \\
        M_1
            \rar["f"']
        & M_2
    \end{cd}
    \item $f$ commutes with the identity:
    \begin{cd}[column sep=small]
        & I
            \dlar["\eta_1"']
            \drar["\eta_2"]
        \\
        M_1
            \ar[rr, "f"']
        && M_2
    \end{cd}
\end{itemize}

\sepline

If $\CC$ is a monoidal category, the \keyword{category of monoids} in $\CC$, denoted $\Mon(\CC)$, is the category whose objects are monoids in $\CC$ and morphisms are monoid homomorphisms.

If $\CC$ is a symmetric monoidal category, the \keyword{category of commutative monoids} in $\CC$, denoted $\CMon(\CC)$, is the category whose objects are commutative monoids in $\CC$ and morphisms are monoid homomorphisms.

\sepline

Let $\CC$ be a category.

A \keyword{diagram} in $\CC$ is a functor $J \to \CC$, where $J$ is some convenient sort of category.
More generally, we may let $J$ be any small category.
In most cases, $J$ is a free category (path category) on a quiver.

An object $i \in \CC$ is \keyword{initial} if for every $x \in \CC$ there is a unique morphism $i \to x$.

An object $t \in \CC$ is \keyword{terminal} if for every $x \in \CC$ there is a unique morphism $x \to t$.

A \keyword{zero object} is an object which is is both initial and terminal.

Say $\CC$ is a \keyword{concrete category} when it comes equipped with a faithful functor $U : \CC \to \Set$.
Consider $U$ to take objects of $\CC$ to their \keyword{underlying set}.

\sepline

Let $\CC$ and $\DD$ be categories.

The \keyword{functor category} (or \keyword{category of functors} from $\CC$ to $\DD$), denoted $\DD^\CC$ or $[\CC, \DD]$, is the category whose objects are functors $\CC \to \DD$ and whose morphisms are natural transformations.

If $\CC$ and $\DD$ are small categories, then so is $[\CC, \DD]$.

If $\CC$ is small and $\DD$ is locally small, then $[\CC, \DD]$ is locally small.

If only know $\CC$ and $\DD$ to be locally small, know nothing of $[\CC, \DD]$.

\sepline

Let $J$ and $\CC$ be any categories.
 
For any object $c \in \CC$, there is a \keyword{constant functor} $\Delta c : J \to \DD$ sending every object of $J$ to $c$ and every morphism to $1_c$.


\sepline

abelian categories innit

sections and retractions

A short exact sequence $0 \to K \to A \to Q \to 0$ \keyword{splits} iff $K \to A$ section iff $A \to Q$ retraction:
\begin{cd}[row sep=small]
    & & & Q \dar[equal] \dlar[dashed] \\
    0 \rar & K \rar & A \rar \dlar[dashed] & Q \rar & 0 \\
    & K \uar[equal] 
\end{cd}
in which case,
\begin{cd}[row sep=small]
    0 \rar 
    & K \rar \dar[phantom, "="{rotate=-90, anchor=center}]
    & A \rar \dar[phantom, "\iso"{rotate=-90, anchor=center}]
    & Q \rar \dar[phantom, "="{rotate=-90, anchor=center}]
    & 0 \\
    0 \rar & K \rar & K \oplus Q \rar & Q \rar & 0 \\
\end{cd}

projective and injective

``any epimorphism to a projective object has a section'' (epi to proj has sec)

``any monomorphism from an injective object has a retraction'' (mono from inj has retr)


\end{document}