\documentclass[12pt]{article}

% Packages
\usepackage[margin=1in]{geometry}
\usepackage{parskip}
\usepackage{amsmath, amsthm, amssymb}
\usepackage{tikz, tikz-cd}
\usepackage[shortlabels]{enumitem}

\usepackage{bbm}

\usepackage{suffix}
\usetikzlibrary{decorations.pathmorphing}

% Problem Box
\setlength{\fboxsep}{4pt}
\newlength{\myparskip} 
\setlength{\myparskip}{\parskip}
\newsavebox{\savefullbox}
\newenvironment{fullbox}{\begin{lrbox}{\savefullbox}\begin{minipage}{\dimexpr\textwidth-2\fboxsep\relax}\setlength{\parskip}{\myparskip}}{\end{minipage}\end{lrbox}\framebox[\textwidth]{\usebox{\savefullbox}}}

% Environments
\setlist[enumerate]{nosep}
\newcommand{\keyword}[1]{\textbf{#1}}
\newcommand{\sepline}{\rule{\textwidth}{0.4pt}}

% Tikz Environments
\newenvironment{drawing}{\begin{center}\begin{tikzpicture}}{\end{tikzpicture}\end{center}}
% \tikzcdset{row sep/normal=0pt}
\newenvironment{cd}{\begin{center}\begin{tikzcd}}{\end{tikzcd}\end{center}}


% Document Formatting
\newtheoremstyle{mythmstyle}% name of the style to be used
{ }% measure of space to leave above the theorem. E.g.: 3pt
{ }% measure of space to leave below the theorem. E.g.: 3pt
{ }% name of font to use in the body of the theorem
{ }% measure of space to indent
{\scshape}% name of head font
{.}% punctuation between head and body
{ }% space after theorem head; " " = normal interword space
{\thmname{#1}\thmnumber{ #2}\thmnote{ (#3)}}% Manually specify head

\theoremstyle{definition}
\newtheorem{theorem}{Theorem}
\newtheorem{corollary}{Corollary}
\newtheorem{lemma}{Lemma}
\newtheorem{proposition}{Proposition}
\newtheorem{claim}{Claim}


% Math Formatting
\newcommand{\isp}[1]{\quad\text{#1}\quad}

% mathbb
\newcommand{\N}{\mathbb{N}}
\newcommand{\Z}{\mathbb{Z}}
\newcommand{\Q}{\mathbb{Q}}
\newcommand{\R}{\mathbb{R}}
\newcommand{\C}{\mathbb{C}}
\renewcommand{\k}{\mathbbm{k}}

% mathcal
\newcommand{\LL}{\mathcal{L}}
\newcommand{\FF}{\mathcal{F}}

% Symbols
\newcommand{\eps}{\varepsilon}
\renewcommand{\phi}{\varphi}
\renewcommand{\emptyset}{\varnothing}

% Delimiters
\newcommand{\<}{\left\langle}
\renewcommand{\>}{\right\rangle}

% Relations
\newcommand{\isom}{\cong}
\newcommand{\seq}{\subseteq}

\newcommand{\inc}{\hookrightarrow}
\newcommand{\To}{\longrightarrow}
\newcommand{\Mapsto}{\longmapsto}



% Math Roman
\newcommand{\dd}{\mathrm{d}}
\newcommand{\DD}{\mathrm{D}}

\renewcommand{\Re}{\operatorname{Re}}

\DeclareMathOperator{\im}{im}



% Other
\newcommand{\pdv}[2]{\frac{\partial #1}{\partial #2}}
\newcommand{\mat}[1]{\begin{bmatrix}#1\end{bmatrix}}
\newcommand{\clo}{\overline}
\newcommand{\conj}{\overline}



\title{Functional Analysis \\
    \large 
}
\author{}
\date{}


\begin{document}


requires like linear algebra and some topology

\sepline

Two Hilbert space $H_1$ and $H_1$ are said to be \keyword{isomorphic} (maybe isometric) if there is a linear operator $U : H_1 \to H_2$ such that
\[
    \<Ux, Uy\>_{H_2} = \<x, y\>_{H_1}
\]
for all $x, y \in H_1$.
Such an operator $U$ is called \keyword{unitary}.

\sepline

\begin{lemma}
    Let $H$ be a Hilbert space, $M \leq H$ a closed subspace, and $x \in H$.
    Then there exists a unique element $z \in M$ with minimum distance to $x$.
\end{lemma}

\begin{theorem}[projection theorem]
    Let $H$ be a Hilbert space and $M \leq H$ closed.
    Then $H = M \oplus M^\perp$.
\end{theorem}

\sepline

Let $H_1$ and $H_2$ be Hilbert spaces.
Denote the set
\[
    \LL(H_1, H_2) := \{\text{bounded linear transformations } H_1 \to H_2 \}.
\]
This is a Banach space with the norm
\[
    \|T\| = \sup_{\|x\|_{H_1} = 1} \|Tx\|_{H_2}
\]


The \keyword{dual space} of a Hilbert space $H$ is
\[
    H^* := \LL(H, \C).
\]
It's elements are called \keyword{continuous linear functionals}.


\sepline


\begin{theorem}[Riesz lemma]
    For each $T \in H^*$ there is a unique $y_T \in H$ such that $Tx = \<y_T, x\>$ for all $x \in H$ and $\|y_T\|_H = \|T\|_{H^*}$.
\end{theorem}

\sepline

An \keyword{orthonormal basis} of a Hilbert space is a maximal orthonormal subset.

\begin{theorem}
    Every Hilbert space $H$ has an orthonormal basis.
\end{theorem}

\sepline

A metric space is \keyword{separable} if it has a countable dense subset.

\begin{theorem}
    A Hilbert space $H$ is separable if and only if it has a countable orthonormal basis $S$.
    If there are $N < \infty$ elements in $S$, then $H$ is isometric to $\C^N$.
    If there are countably infinitely many elements in $S$, then $H$ is isometric to $\ell_2$.
\end{theorem}

\sepline

Let $(M, \mu)$ be a measure space and $p \geq 1$ a real number.
Define the set
\[
    L^p(M, \dd{\mu}) = 
\]

\begin{theorem}
    Let $1 \leq p < \infty$.
    \begin{enumerate}[(a)]
        \item (Minkowski inequality) If $f, g \in L^p(M, \dd\mu)$ then $\|f + g\|_p \leq \|f\|_p + \|g\|_p$.
        \item (Riesz-Fisher) $L^p(M, \dd\mu)$ is complete.
        \item (H\"older inequality) Let $p, q, r \geq 1$ satisfy $\frac{1}{p} + \frac{1}{q} = \frac{1}{r}$.
        Suppose $f \in L^p(M, \dd\mu)$, $g \in L^q(M, \dd\mu)$.
        Then $fg \in L^r(M, \dd\mu)$ and $\|fg\|_r \leq \|f\|_p \|g\|_q$.
    \end{enumerate}
\end{theorem}

\sepline

We can define the space $\LL(X, Y)$ of bounded linear operators for any normed spaces $X$ and $Y$.

Dual space $X^* = \LL(X, \C)$.

\begin{theorem}
    If $Y$ is complete, $\LL(X, Y)$ is a Banach space.
\end{theorem}

\begin{theorem}
    A normed linear space is complete if and only if every absolutely summable sequence is summable.
\end{theorem}

\begin{theorem}
    Let $X$ be a Banach space.
    For each $x \in X$, let $\tilde{x} : X^* \to \C$ be the linear functional $\tilde{x}(\lambda) = \lambda(x)$ for all $\lambda \in X^*$.
    Then the map $J : X \to X^{**}$, $x \mapsto \tilde{x}$, is an isometric embedding of $X$ as a subspace of $X^{**}$.
\end{theorem}

If $J$ is surjective, say $X$ is \keyword{reflexive}.

\sepline

\begin{theorem}[Hahn-Banach --- real convex]
    Let $X$ be a real vector space and $p : X \to \R$ convex.
    Let $Y \leq X$ and $\lambda \in Y^*$ linear with $\lambda(y) \leq p(y)$ for all $y \in Y$.
    Then there exists $\Lambda \in X^*$ such that $\Lambda(x) \leq p(x)$ for all $x \in X$ and $\Lambda|_Y = \lambda$.
\end{theorem}

\begin{theorem}[Hahn-Banach --- complex convex]
    Let $X$ be a complex vector space and $p : X \to \R$ absolutely convex.
    Let $Y \leq X$ and $\lambda \in Y^*$ with $|\lambda(y)| \leq p(y)$ for all $y \in Y$.
    Then there exists $\Lambda \in X^*$ such that $|\Lambda(x)| \leq p(x)$ for all $x \in X$ and $\Lambda|_Y = \lambda$.
\end{theorem}

\begin{theorem}[Hahn-Banach --- real quasi-seminorm]
    Let $X$ be a real vector space and $q : X \to \R$ a quasi-seminorm.
    Let $Y \leq X$ and $\lambda \in Y^*$ with $\lambda(y) \leq q(y)$ for all $y \in Y$.
    Then there exists $\Lambda \in X^*$ such that $\Lambda(x) \leq q(x)$ for all $x \in X$ and $\Lambda|_Y = \lambda$.
\end{theorem}

\begin{theorem}[Hahn-Banach --- complex quasi-seminorm]
    Let $X$ be a complex vector space and $q : X \to \R$ a quasi-seminorm.
    Let $Y \leq X$ and $\lambda \in Y^*$ with $\Re\lambda(y) \leq q(y)$ for all $y \in Y$.
    Then there exists $\Lambda \in X^*$ such that $\Re\Lambda(x) \leq q(x)$ for all $x \in X$ and $\Lambda|_Y = \lambda$.
\end{theorem}

\begin{theorem}[Hahn-Banach --- real seminorm]
    Let $X$ be a real vector space and $p : X \to \R$ a seminorm.
    Let $Y \leq X$ and $\lambda \in Y^*$ with $\lambda(y) \leq p(y)$ for all $y \in Y$.
    Then there exists $\Lambda \in X^*$ such that $\Lambda(x) \leq p(x)$ for all $x \in X$ and $\Lambda|_Y = \lambda$.
\end{theorem}

\begin{theorem}[Hahn-Banach --- complex seminorm]
    Let $X$ be a complex vector space and $p : X \to \R$ a seminorm.
    Let $Y \leq X$ and $\lambda \in Y^*$ with $|\lambda(y)| \leq p(y)$ for all $y \in Y$.
    Then there exists $\Lambda \in X^*$ such that $|\Lambda(x)| \leq p(x)$ for all $x \in X$ and $\Lambda|_Y = \lambda$.
\end{theorem}

\begin{corollary}
    Let $X$ be a normed linear space, $Y \leq X$, and $\lambda \in Y^*$.
    Then there exists $\Lambda \in X^*$ with $\Lambda|_Y = \lambda$ and $\|\Lambda\|_{X^*} = \|\lambda\|_{Y^*}$.
\end{corollary}

\begin{corollary}
    Let $X$ be a normed linear space and $y \in X$.
    Then there exists a nonzero $\Lambda \in X^*$ such that $\Lambda(y) = \|\Lambda\|_{X^*}\|y\|$.
\end{corollary}

\begin{corollary}
    Let $X$ be a normed linear space, $Z \leq X$, and $y \in X$ with $d = d(y, Z) > 0$.
    Then there exists $\Lambda \in X^*$ such that $\|\Lambda\| \leq 1$, $\Lambda(y) = d$, and $\Lambda|_Z = 0$.
\end{corollary}

\sepline

\begin{theorem}
    Let $B$ be a Banach space.
    If $B^*$ is separable, then $B$ is separable.
\end{theorem}

\sepline

Baire category

\sepline

\begin{theorem}[uniform boundedness principle]
    Let $B$ be a Banach space and $V$ a normed space.
    Let $\FF \seq \LL(B, V)$ such that for all $x \in X$
    \[
        \sup_{T \in \FF} \|Tx\|_V < \infty
    \]
    (i.e., the set $\{\|Tx\|_V : T \in \FF\}$ is bounded).
    Then
    \[
        \sup_{T \in \FF} \|T\| < \infty
    \]
    (i.e., the set $\{\|T\| : T \in \FF\}$ is bounded).
\end{theorem}

\begin{theorem}[open mapping]
    Let $B_1$ and $B_2$ be Banach spaces and $T \in \LL(B_1, B_2)$.
    If $T$ is surjective then $T$ is open.
\end{theorem}

\begin{theorem}[inverse mapping]
    Let $B_1$ and $B_2$ be Banach spaces and $T \in \LL(B_1, B_2)$.
    If $T$ is bijective then $T^{-1}$ is bounded/continuous.
\end{theorem}

\begin{theorem}[closed graph]
    Let $B_1$ and $B_2$ be Banach spaces and $T : B_1 \to B_2$ linear.
    Then $T$ is bounded if and only if $\Gamma(T)$ is closed in $B_1 \oplus B_2$
\end{theorem}

\begin{corollary}[Hellinger-Toeplitz theorem]
    Let $H$ be a Hilbert space and $A \in \LL(H)$ with $\<x, Ay\> = \<Ax, y\>$ for all $x, y \in H$.
    Then $A$ is bounded.
\end{corollary}

\sepline

Let $X$ be a topological space and $S$ any set.
Let 


\newpage

\sepline

5/19/22

Let $H$ be a Hilbert space.

An operator $B \in \LL(H)$ is \keyword{positive}, written $B \geq 0$, if $\<Bx, x\> \geq 0$ for all $x \in H$.

\begin{lemma}
    If $B \in \LL(H)$ is positive, then $B = B^*$.
\end{lemma}

\begin{proof}
    Using polarization identity we deduce
    \begin{align*}
        \<Bx, y\> + \<By, x\>
            &= \frac{1}{2}\big(\<B(x + y), x + y\> - \<B(x - y), x - y\>\big) \\
            &= \frac{1}{2}\big(\<x+ y, B(x + y)\> - \<x - y, B(x - y)\>\big) \\
            &= \<x, By\> + \<y, Bx\>.
    \end{align*}
    Rearranging, we find
    \[
        \<Bx, y\> - \conj{\<Bx, y\>}
            = \<x, By\> - \conj{\<x, By\>}.
    \]
    Replace $y$ bu $iy$, get
    \[
        \<Bx, y\> + \conj{\<Bx, y\>}
            = \<x, By\> + \conj{\<x, By\>}.
    \]
    Adding these and dividing by $2$, conclude $\<Bx, y\> = \<x, By\>$, so indeed $B = B^*$.
\end{proof}

\begin{theorem}[Square Root Lemma]
    If $A \in \LL(H)$ is positive, then there exists a unique $B \in \LL(H)$ such that $B^2 = A$ and $B \geq 0$.
    Moreover, for any $C \in \LL(H)$, if $AC = CA$ then $BC = CB$.
\end{theorem}

\begin{proof}
    Want to construct $B$.
    Without loss of generality, may assume $\|A\| \leq 1$; otherwise rescale.
    Recall that $A = A^*$ implies $\|A\| = \sup_{\|x\| = 1}|\<Ax, x\>|$.
    Now consider
    \begin{align*}
        \|I - A\|
            &= \sup_{\|x\| = 1} |\<(I - A)x, x\>| \\
            &= \sup_{\|x\| = 1} \left|\|x\|^2 - \<Ax, x\>\right| \\
            &\leq 1,
    \end{align*}
    since $\|A\| \leq 1$, know $0 \leq \<Ax, x\> \leq 1$.

    \begin{lemma}
        $\sqrt{1 - z}$ has power series absolutely convergent for $|z| \leq 1$.
        ($\sqrt{1 - z} = \sum_{n=0}^{\infty} c_n z^n$ with $c_n < 0$ if $n \geq 1$.)
    \end{lemma}
    
    By Lemma, $B := \sum_{n=0}^{\infty} c_n (I - A)^n$ converges in the operator norm, so is well-defined.
    This construction gives $B^2 = I - (I - A) = A$.

    Since $B$ is defined in terms of $A$ and $I$, then if $A$ commutes with $C$, it is easy to check that $B$ also commutes with $C$.

    We now check $B$ is positive; suffices to check for $\|x\| = 1$.
    Consider
    \[
        \<Bx, x\> = 1 + \sum_{n=1}^{\infty} c_n\<(I - A)^nx, x\>.
    \]

    \begin{claim}
        $0 \leq \<(I - A)^nx, x\> \leq 1$ for all $n \in \N$.
    \end{claim}

    \begin{proof}
        If $n = 2k$ is even, then $\<(I - A)^nx, x\> = \|(I - A)^kx\|^2$.

        If $n = 2k + 1$ is odd,
        \[
            \<(I - A)^nx, x\>
                = \<(I - A)((I - A)^kx), (I - A)^kx\>
                = \<(I - A)y, y\>,
        \]
        where $y = (I - A)^kx$.
        Then by an above argument, this is between $0$ and $1$.
    \end{proof}

    By the claim,
    \[
        \<Bx, x\>
            \geq 1 + \sum_{n=1}^{\infty} c_n 
            = \sqrt{1 - 1}
            = 0,
    \]
    hence $B \geq 0$.

    It remains to prove uniqueness.
    Suppose $C^2 = A$ and $C \geq 0$.
    Have
    \[
        CA = C^3 = AC
        \quad\implies\quad
        CB = BC.
    \]
    Also have
    \[
        C^2 = A = B^2.
    \]
    These facts imply
    \[
        (C - B)(B + C)(C - B) = 0.
    \]
    which implies
    \[
        (C - B)B(C - B) = (C - B)C(C - B) = 0.
    \]
    Can verify this by applying to an element and using commutativity.
    In particular, if sum of two positive operators is zero then both must be zero.
    It follows that $(C - B)^3 = 0$, so of course $(C - B)^n = 0$ for $n \geq 3$.
    Since $B, C$ self-adjoint, deduce
    \[
        \|(C - B)\|^4 = \|(C - B)^4\| = 0,
    \]
    so indeed $B = C$.
\end{proof}

From the Square Root Lemma, denote $\sqrt{A} = B$ for positive $A \in \LL(H)$

Let $A \in \LL(H)$ be any, define \keyword{absolute value} as $|A| = \sqrt{A^*A}$.
(Note $A^*A$ is positive so this makes sense: $\<A^*Ax, x\> \geq 0$).
Theorem gives $|A|^2 = A^*A$.

\paragraph{Remark.}
It is not true in general that $|AB| = |A||B|$; this is only notation.

\paragraph{Example.}
Let $T : \ell_2 \to \ell_2$ be the right shift operator $T(x_1, x_2, \dots) = (0, x_1, x_2, \dots)$.
Adjoint $T^*$ is left shift, then $T^*T = I$ implies $|T| = I \ne T$.

An operator $U \in \LL(H)$ is called a \keyword{partial isometry} (or \keyword{partial unitary}) if $U$ is an isometry on $(\ker U)^\perp$, i.e., $\|Ux\| = \|x\|$ for all $x \in (\ker U)^\perp$.

\paragraph{Fact.}
$U$ is a partial isometry if $U^*U = P$ and $UU^* = Q$ where $P : H \to (\ker U)^\perp$ and $Q : H \to \im U$ are orthogonal projections


\begin{theorem}[Polar Decomposition]
    Let $A \in \LL(H)$ arbitrary, then there exists a unique partial isometry $U$ such that $\ker U = \ker A$ and $A = U|A|$.
    Moreover, $\im U = \clo{\im A}$.
\end{theorem}

\paragraph{Example.}
Let $T : \ell_2 \to \ell_2$ be right shift.
Since $|T| = I$, then this theorem would imply $U = T$.
Additionally, theorem says that $\im T$ should be closed.


\begin{proof}
    Will construct $U$, prove properties, then prove uniqueness.

    First, want to define $U : \im|A| \to \im A$ by $U(|A|x) = Ax$ for all $x \in H$, but must check well-defined.
    Notice
    \[
        \||A|x\|^2 = \<|A|^2x, x\> = \<A^*Ax, x\> = \|Ax\|^2,
    \]
    which implies $\||A|x\| = \|Ax\|$ and $\ker|A| = \ker A$.
    So $|A|x_1 = |A|x_2$ implies $Ax_1 = Ax_2$, hence $U$ is indeed well-defined on $\im|A|$.

    Naturally extend to $U : \clo{\im|A|} \to \clo{\im A}$, and notice that this is an isometry.

    Then we can simply define $U|_{\clo{\im|A|}^\perp} = 0$, i.e.,
    \[
        \ker U = (\im|A|)^\perp = \ker|A| = \ker A
    \]
    One can check second equality by self-adjointness of $|A|$.

    We have so far constructed a partial isometry $U$ with the desired properties.
    To show uniqueness, can suppose there is another and show equality...
\end{proof}

\sepline

5/19/22

Let $B_1, B_2$ be Banach spaces.

A bounded operator $T \in \LL(B_1, B_2)$ is called \keyword{compact} (or completely continuous) if for every bounded sequence $\{x_n\}$ in $B_1$ the image sequence $\{Tx_n\}$ in $B_2$ has a convergent subsequence.

\paragraph{Recall.}
If $X$ is a metric space, a subset $K \seq X$ is compact (every open cover of $K$ has a finite subcover) if and only if every sequence in $K$ has a convergence subsequence.
In other words, for metric spaces compactness is equivalent to sequential compactness.

A subset of a topological space is \keyword{precompact} if its closure in the larger space is compact.

Equivalently, bounded operator $T \in \LL(B_1, B_2)$ is compact if and only if it maps any bounded set in $B_1$ to a precompact (relatively compact) set in $B_2$.

\paragraph{Remark.}
One can verify that if $T$ is compact, it is bounded.
To see this, consider the closed unit ball $D \seq B_1$, which is in particular a bounded set.
Then by the equivalent definition, $\clo{T(D)} \seq B_2$ is compact, which is in particular bounded.
So then
\[
    \|T\| = \sup_{\|x\|=1} \|Tx\| \leq \sup_{x \in D} \|Tx\| < \infty.
\]


\paragraph{Example.}
\begin{enumerate}
    \item Finite rank operators: $\dim\im T < \infty$.
    In this case, can write $\im T = \C y_1 \oplus \cdots \oplus \C y_n \leq B_2$.
    For a bounded sequence $\{x_k\}_k$ in $B_1$, we can write
    $Tx_k = \sum_{i=1}^{n} a_{i,k} y_i$.
    Then for each $i$, $\{a_{i,k}\}_k$ is a bounded sequence in $\C$, so there exists a convergent subsequence $\{a_{i,k_j}\}_j$.
    Then $\{Tx_{k_j}\}_j$ is a convergent subsequence in $B_2$.
    Hence, $T$ is compact.
    \item Classical integral operators.
    Given $k \in C([0, 1] \times [0, 1])$, consider $B_1 = B_2 = C[0, 1]$ with $\|\cdot\|_\infty$.
    Then have operator $\hat{k} \in \LL(C[0, 1])$ defined by
    \[
        (\hat{k}f)(x) = \int_{0}^{1} k(x, y) f(y) \dd{y}.
    \]
    It follows from this definition that $\|\hat{k}\| \leq \|k\|_\infty$.
    For $\eps > 0$ choose $\delta > 0$ for uniform continuity of $k$.
    Then $|x - x'| < \delta$ gives
    \[
        \left|\hat{k}f(x) - \hat{k}f(x')\right|
            \leq \sup_{y \in [0, 1]} \left|k(x, y) - k(x', y)\right|\|f\|_\infty
            < \eps\|f\|_\infty.
    \]
    Can verify that if $\{f_n\}$ is bounded sequence in $C[0, 1]$ then $\{\hat{k}f_n\}$ is uniformly bounded.
    Moreover, $\{\hat{k}f_n\}$ is equicontinuous.
    By the Arzel\`a-Ascoli theorem $\hat{k}$ has a convergence subsequence.
\end{enumerate}

Given sequence $\{x_n\}$ in $B$ and $x \in B$:
\begin{itemize}[nosep]
    \item say $x_n \to x$ strongly if $\|x_n - x\| \xrightarrow{n \to \infty} 0$;
    \item say $x_n \to x$ weakly if $\ell(x_n) \to \ell(x)$ for all $\ell \in B^*$.
\end{itemize}

\begin{theorem}
    Compact operators map weakly convergent sequences to strongly convergent.
\end{theorem}

\begin{proof}
    Suppose $x_n \to x$ weakly in $B_1$, i.e., $\ell(x_n) \to \ell(x)$ for all $\ell \in B_1^*$.
    Identifying $B_1 \leq B_1^{**}$, we can view this as a sequence $\tilde{x}_n(\ell) \to \tilde{x}(\ell)$ for all $\ell \in B_1^*$.
    By the uniform boundedness principle, $\{\tilde{x}_n\}$ is bounded in $B_1^{**}$---equivalently $\{x_n\}$ is bounded in $B_1$ since $\|\tilde{y}\| = \|y\|$ for all $y \in B_1$.

    Let $T \in \LL(B_1, B_2)$ be a compact operator.
    Then $\{y_n = Tx_n\}$ in $B_2$ has a convergent subsequence $\{y_{n_k}\}$.
    Let $y = T_x$, then $y_n \to y$ weakly in $B_2$.
    For $\ell \in B_2^*$, we have
    \[
        \ell(y_n) - \ell(y)
            = \ell(T(x_n)) - \ell(T(x))
            = T'\ell(x_n - x)
    \]
    with $T'\ell \in B_1^*$.
    This converges to zero by weak $x_n \to x$, hence $y_n \to y$ weakly in $B_2$.

    Assuming $y_n \not\to y$ strongly, there exists a subsequence $\{y_{n_j}$ and $\eps > 0$ such that $\|y_{n_j} - y\| \geq \eps$.
    However, $y_{n_j} = Tx_{n_j}$ has a convergent subsequence, i.e., $y_{n_j} \to y' \ne y$.
    But then also have $y_{n_j} \to y'$ weakly, which contradicts weak convergence $y_{n_j} \to y$.
    Hence, $y_n \to y$ strongly.
\end{proof}

\begin{theorem}
    Let $T \in \LL(B_1, B_2)$.
    \begin{enumerate}[(a)]
        \item If $\{T_k\}$ is a sequence of compact operators in $\LL(B_1, B_2)$ such that $T_k \to T$ with respect to the operator norm, i.e., $\|T_k - T\| \to 0$, then $T$ is also compact.
        \item $T$ is compact if and only if $T'$ is compact. (special case $T^*$?)
        \item If $S \in \LL(B_0, B_1)$ arbitrary and $T$ is compact, then $TS \in \LL(B_0, B_2)$ is compact.
    \end{enumerate}
\end{theorem}

\begin{proof}
    \begin{enumerate}[(a)]
        \item Use diagonal argument.
        Let $\{x_n\}$ be a bounded sequence in $B_1$.
        Each $T_k$ is bounded, so $\{T_kx_n\}_n$ has a convergent subsequence $\{T_kx_{n_{k,j}}\}_j$.
        Then get subsequences
        \[
            \begin{array}{c|cccc}
                & x_1 & x_2 & x_3 & \cdots \\
                \hline
                T_1 & T_1(x_{n_{1,1}}) & T_1(x_{n_{1,2}}) & T_1(x_{n_{1,3}}) \\
                T_2 & T_2(x_{n_{2,1}}) & T_2(x_{n_{2,2}}) & T_2(x_{n_{2,3}}) \\
                T_3 & T_3(x_{n_{3,1}}) & T_3(x_{n_{3,2}}) & T_3(x_{n_{3,3}}) \\
                \vdots &&&& \ddots \\
            \end{array}
        \]
        Then diagonal $\{x_{n_{k,k}}\}_k$ is a subsequence such that $\{T_jx_{n_{k,k}}\}_k$ is convergent for all $j$.
        Then
        \begin{align*}
            \|Tx_{n_{k,k}} - Tx_{n_{\ell,\ell}}\|
                &\leq \|Tx_{n_{k,k}} - T_jx_{n_{k,k}}\|
                    + \|T_jx_{n_{k,k}} = T_jx_{n_{\ell,\ell}}\|
                    + \|T_jx_{n_{\ell,\ell}} = Tx_{n_{\ell,\ell}}\| \\
                &< \frac{\eps}{3} + \frac{\eps}{3} + \frac{\eps}{3}
                = \eps.
        \end{align*}
        First two for large $j$ and middle for large $k, \ell$.
        Therefore, $Tx_{n,k}$ is convergent, so $T$ is compact.
    \end{enumerate}
\end{proof}

\sepline

5/24/22

\begin{proof}
     \begin{enumerate}[(b)]
        \item For $B_1 = B_2 = H$, $T: H \to H$, $T^* : H \to H$.
        
        \begin{lemma}
            For $T \in \LL(H)$ with $T^*T$ compact, then $T$ is compact.
        \end{lemma}

        \begin{proof}
            Given $\{x_n\} \seq H$ bounded, get
            \[
                \|T(x_n - x_m)\|^2
                    = \<T^*T(x_n - x_m), x_n - x_m\>
                    \leq C\|T^*T(x_n - x_m)\|.
            \]
            Then some subsequence $\{T^*Tx_{n_k}\}$ is Cauchy.
            Plugging in to the above inequality implies $\{Tx_{n_k}\}$ is Cauchy, therefore convergent in $H$.
            Hence, $T$ is compact.
        \end{proof}
        Then if $T$ is compact, then (c) implies $TT^* = (T^*)^*T^*$ is compact, then Lemma implies $T^*$ is compact.

        Remains to show for general Banach spaces and converse.
    \end{enumerate}
\end{proof}

\begin{lemma}
    If $T \in \LL(H)$  is compact, then $\im T$ is separable.
\end{lemma}

\begin{proof}
    Idea: $T(B_n(0))$ is precompact so $\clo{T(B_n(0))}$ is compact.
    Compact sets in metric space are separable (Look at center points of each finite cover by balls of radius $1/n$).
    Then $\clo{T(B_n(0))}$ is separable, and so is $T(B_n(0))$.
    Then can write $H = \bigcup_{n=1}^{\infty} B_n(0)$, and $T(H) \seq \bigcup_{n=1}^{\infty}$ so $\im T$ is separable.

    To complete this proof, must give stronger argument for each step.
\end{proof}

\begin{lemma}
    If $T \in \LL(H)$ is compact, then $(\ker T)^\perp$ is separable.
\end{lemma}

\begin{proof}
    Verify that $(\ker T)^\perp = \clo{\im T^*}$.
    Then $T$ compact implies $T^*$, so previous lemma implies $\im T^*$ is separable, and therefore so is its closure.
\end{proof}

\begin{theorem}
    Let $T \in \LL(H)$ compact.
    Then there exists a sequence $\{T_n\}$ of finite rank operators such that $T_n \to T$ in the operator norm.
\end{theorem}

\begin{proof}
    By Lemma, $(\ker T)^\perp$ has a countable orthonormal basis $\{x_n\}$.
    Since $T$ acting on the kernel is zero, only care about this orthogonal complement.
    Define space
    \[
        H_n = \ker T \oplus \C x_1 \oplus \cdots \oplus \C x_n.
    \]
    Then $\{H_n\}$ is increasing sequence of subspaces, so $\{H_n^\perp\}$ is a shrinking sequence of subspaces.
    Let
    \[
        \lambda_n = \sup_{\substack{x \in H_n^\perp \\ \|x\|=1}} \|Tx\|
    \]
    This $\{\lambda_n\}$ is a decreasing nonnegative sequence, therefore convergent $\lambda = \lim_{n\to\infty} \lambda_n \geq 0$.

    We claim that $\lambda = 0 $.

    Suppose $y_n \in H_n^\perp$ satisfying $\|y_n\| = 1$ and $\|Ty_n\| \geq \lambda/2$.

    Then claim $y_n \to 0$ weakly, i.e., $\ell(y_n) \to 0$ for all $\ell \in H^*$.

    By Riesz lemma, can write $\ell(y_n) = \<y_n, x\>$ for some $x \in H$.
    Write $y_n = \sum_{i=n+1}^{\infty} a_i x_i$ then
    \[
        \<y_n, x\> = \sum_{i=n+1}^{\infty} a_i\<x_i, x\>.
        \quad
        (a_{n,i}?)
    \]
    The tail goes to zero as $n \to \infty$, so indeed $\ell(y_n) \to 0$.
    
    Then $T$ compact implies $Ty_n \to 0$ strongly, so must have $\lambda = 0$.

    Now construct finite rank operators.
    Let $T_nx = \sum_{i=1}^{n} \<x, x_i\> Tx_i$ for $x \in (\ker T)^\perp$ and set $T_n|_{\ker T} = 0$.
    Then $T_n$ is finite rank.

    Should be clear from construction that $T_n \to T$.
    To verify, look at
    \[
        \|(T - T_n)x\|
            \leq \lambda_n\|x\|
    \]
    for $x \in (\ker T)^\perp$.
    Can write $(T - T_n)x = Tz_n$ for some $z_n \in H_n^\perp$.
    Then $\lambda_n \to 0$ implies $\|T - T_n\| \to 0$.
\end{proof}

\begin{theorem}[Analytic Fredholm]
    Let $D \seq \C$ open and $f : D \to \LL(H)$ be analytic (i.e., $\lim_{h \to 0} \frac{f(z + h) - f(z)}{h}$ exists for all $x \in D$).
    If $f(z)$ is compact for all $z \in D$, then either 
    \begin{enumerate}[(a)]
        \item $(I - f(z))^{-1}$ does not exits for all $z \in D$ or
        \item there exists $S \seq D$ discrete such that $(I - f(z))^{-1}$ exists for all $z \in D \setminus S$.
        Moreover, $(I - f(z))^{-1}$ is meromorphic on $D$ (analytic on $D \setminus S$), and $f(z)x = x$ has a nontrivial solution in $H$ for all $z \in S$.
    \end{enumerate}
\end{theorem}

\begin{proof}
    Pick any $z_0 \in D$ and choose radius $r > 0$ such that $B_r(z_0) \seq D$ and $\|f(z) - f(z_0)\| < 1/2$ for all $z \in B_r(z_0)$, by analyticity of $f$.
    By previous theorem applied to $f(z_0)$, choose a finite rank operator $F$ such that $\|f(z_0) - F\| < 1/2$.
    Then $\|f(z) - F\| < 1$ for all $z \in B_r(z_0)$.
    Then $(I - f(z) + F)^{-1}$ exists for $z \in B_r(0)$ since can write as a power series.
    Moreover, this inverse is analytic for reasons...

    Denote $g(z) = F(I - f(z) + F)^{-1}$, then
    \[
        I - f(z) = (I - g(z))(I - f(z) + F).
    \]
    Then $I - f(z)$ is invertible if and only if $I - g(z)$ is invertible.
    Since $F$ is finite rank, so is $g(z)$.

    \begin{center}
        --- 5/26/22 ---
    \end{center}

    $F$ is finite rank, so there exist $y_1, \dots y_N \in H$ such that
    \[
        F(x)
            = \sum_{i=1}^{N} \ell_i(x)y_i
            = \sum_{i=1}^{N} \<x, x_i\>y_i.
    \]
    (Second equality comes from Riesz Lemma.)
    Let $\phi_n(z) = ((I - f(z) + F)^{-1})^*(x_n)$, then
    \[
        g(z)
            = F(I - f(z) + F)^{-1}(z)
            = \sum_{n=1}^{N} \<-, \phi_n(z)\>y_n.
    \]
    To see if $I - g(z)$ is invertible, we are asking if it has any nontrivial zeros, i.e., whether there are any nontrivial solutions for $g(z)x = x$.
    If $x$ is such a solution, then in particular $x = g(z)x \in \im F$, so we can write $x = \sum_{k=1}^{N} c_k y_k$ giving
    \[
        g(z)x
            = \sum_{n=1}^{N} \sum_{k=1}^{N} \<c_k y_k, \phi_n(z)\>y_n.
    \]
    Can check that $g(z)x = x$ has a nontrivial solution if and only if
    \[
        \det Q = 0 \isp{where} Q = [\delta_{kn} - \<y_k, \phi_n(z)\>].
    \]
    This determinant function is analytic with respect to $z \in B_r(z_0)$.
    By complex analysis, we know that either $\det Q \equiv 0$ ($=\mathrm{const_0}$ constantly zero) or $\det Q = 0$ only on a discrete set.
    (In other words, $(\det Q)^{-1}(0)$ is either $B_r(z_0)$ or discrete.)
    The result now follows.
\end{proof}

\begin{corollary}[Fredholm alternative]
    For compact $A \in \LL(H)$, either $(I - A)^{-1}$ exists or $A\psi = \psi$ has a nonzero solution.
\end{corollary}

\begin{proof}
    To apply Fredholm theorem, we must construct an analytic function $f : D \to \LL(H)$.
    
    Choosing constant $f \equiv A$, then it might simply happen that we are in case (a), where $I - A$ is simply not invertible.

    We choose $f(z) = z \cdot A$ and $D \seq \C$ a neighborhood of zero, because $I - f(0) = I$ is of course invertible.
    Hence, we are in case (b), so $(I - f(z))^{-1}$ is invertible for all $z \in D \setminus S$ for a discrete set $S \seq D$.

    If $1 \notin S$, then $I - f(1) = I - A$ is invertible.
    
    If $1 \in S$, then $x = f(1)x = Ax$ has nontrivial solutions.
\end{proof}

\begin{theorem}[Fredholm 1st]
    Let $A \in \LL(H)$ compact and $\lambda \ne 0$.
    Then $\lambda$ is an eigenvalue for $A$ if and only if $\conj{\lambda} \ne 0$ is an eigenvalue for $A^*$.
\end{theorem}

\begin{proof}
    In previous proofs, reduce the problem to the invertibility of a finite rank square matrix.
    Something about relationship between invertibility and existence of nontrivial zeros.

    Let $f(z) = z \cdot A$.
    Consider $z = 1/\lambda$ for Analytic Fredholm theorem; have
    \[
        I - f(z)
            = I - zA
            = I - \tfrac{1}{\lambda}A
            = \tfrac{1}{\lambda}(\lambda I - A).
    \]
    Then $\lambda$ is an eigenvalue for $A$ if and only if $I - g(z)$ is not invertible, if and only if $\det Q = 0$, if and only if $\det \conj{Q} = 0$, if and only if $I - g(z)^*$ is not invertible, if and only if $\conj{\lambda}$ is an eigenvalue of $A^*$.
\end{proof}

\begin{theorem}[Fredholm 2nd]
    Let $A \in \LL(H)$ compact.
    \begin{enumerate}[(a)]
        \item For $\lambda \ne 0$, $A - \lambda I$ is invertible if and only if $\ker(A - \lambda I) = 0$;
        \item For $y \in H$, $(A - \lambda I)x = y$ is solvable if and only $y \perp \ker(A^* - \conj{\lambda}I)$.
    \end{enumerate}
\end{theorem}

\begin{proof}
    (b)

    Write
    \[
        \ker(A^* - \conj{\lambda}I)^\perp
            = \clo{\im(A^* - \conj{\lambda}I)^*}
            = \clo{\im(A - \lambda I)}.
    \]
    Claim that this equals $\im(A - \lambda I)$, i.e., image is closed.
    Using previous techniques, there is finite rank $G$ (like $g(z)$) and invertible $P$ (like $(I - f(z) + F)^{-1}$) such that
    \[
        A - \lambda I = (I - G) P.
    \]
    Then $\im(A - \lambda I)$ is closed if and only if $\im(I - G)$ is closed.
    Can write
    \[
        H = \ker G \oplus H_1
    \]
    for finite dimensional $H_1$ (verify by linear algebra since $G$ finite rank).
    Then
    \[
        \im(I - G) = \ker G \oplus F
    \]
    for some finite dimensional $F$.
    Finite dimensional implies closed for subspaces, so $F$ closed.
    Kernels probably closed (something to check), so $\ker G$ is closed.
    Then $\ker G$ and $F$ both closed implies $\im(I - G)$ is closed.
\end{proof}

\paragraph{Remark.} $\dim\ker(A - \lambda I) < \infty$.


\begin{theorem}[Fredholm 3rd]
    Let $A \in \LL(H)$ compact and $\lambda \ne 0$.
    Then $\dim\ker(A - \lambda I) = \dim\ker(A^* - \conj{\lambda}I)$.
\end{theorem}

\begin{proof}[Idea]
    Put $f(z) = zA$ and apply previous stuff to $z = 1/\lambda$.
    Reduce things to finite dimensional linear algebra.
\end{proof}

\end{document}