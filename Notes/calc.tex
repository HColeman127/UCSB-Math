\documentclass[12pt]{article}

% Packages
\usepackage[margin=1in]{geometry}
\usepackage{parskip}
\usepackage{amsmath, amsthm, amssymb}
\usepackage{tikz, tikz-cd}
\usepackage[shortlabels]{enumitem}

\usepackage{bbm}

\usepackage{suffix}
\usetikzlibrary{decorations.pathmorphing}

% Problem Box
\setlength{\fboxsep}{4pt}
\newlength{\myparskip}
\setlength{\myparskip}{\parskip}
\newsavebox{\savefullbox}
\newenvironment{fullbox}{\begin{lrbox}{\savefullbox}\begin{minipage}{\dimexpr\textwidth-2\fboxsep\relax}\setlength{\parskip}{\myparskip}}{\end{minipage}\end{lrbox}\framebox[\textwidth]{\usebox{\savefullbox}}}

% Environments
\setlist[enumerate]{nosep}
\newcommand{\keyword}[1]{\textbf{#1}}
\newcommand{\sepline}{\rule{\textwidth}{0.4pt}}

% Tikz Environments
\newenvironment{drawing}{\begin{center}\begin{tikzpicture}}{\end{tikzpicture}\end{center}}
% \tikzcdset{row sep/normal=0pt}
\newenvironment{cd}{\begin{center}\begin{tikzcd}}{\end{tikzcd}\end{center}}


% Document Formatting
\newtheoremstyle{mythmstyle}% name of the style to be used
{ }% measure of space to leave above the theorem. E.g.: 3pt
{ }% measure of space to leave below the theorem. E.g.: 3pt
{ }% name of font to use in the body of the theorem
{ }% measure of space to indent
{\scshape}% name of head font
{.}% punctuation between head and body
{ }% space after theorem head; " " = normal interword space
{\thmname{#1}\thmnumber{ #2}\thmnote{ (#3)}}% Manually specify head

\theoremstyle{definition}
\newtheorem{theorem}{Theorem}
\newtheorem{corollary}{Corollary}
\newtheorem{lemma}{Lemma}
\newtheorem{proposition}{Proposition}


% Math Formatting
\newcommand{\isp}[1]{\quad\text{#1}\quad}

% mathbb
\newcommand{\N}{\mathbb{N}}
\newcommand{\Z}{\mathbb{Z}}
\newcommand{\Q}{\mathbb{Q}}
\newcommand{\R}{\mathbb{R}}
\newcommand{\C}{\mathbb{C}}
\renewcommand{\k}{\mathbbm{k}}

% mathcal
\newcommand{\LL}{\mathcal{L}}

% Symbols
\newcommand{\eps}{\varepsilon}
\renewcommand{\phi}{\varphi}
\renewcommand{\emptyset}{\varnothing}

% Delimiters
\newcommand{\<}{\left\langle}
\renewcommand{\>}{\right\rangle}

% Relations
\newcommand{\isom}{\cong}
\newcommand{\seq}{\subseteq}

\newcommand{\inc}{\hookrightarrow}
\newcommand{\To}{\longrightarrow}
\newcommand{\Mapsto}{\longmapsto}



% Math Roman
\newcommand{\dd}{\mathrm{d}}
\newcommand{\DD}{\mathrm{D}}





% Other
\newcommand{\pdv}[2]{\frac{\partial #1}{\partial #2}}
\newcommand{\mat}[1]{\begin{bmatrix}#1\end{bmatrix}}





\title{Calculus \\
    \large 
}
\author{}
\date{}


\begin{document}

idk what is wanted, but imma fix a base field $\k$, which is likely $\R$ or $\C$.
I think any subfield of $\C$ is fine.

A \keyword{vector space} (over $\k$) is a $\k$-module.

A \keyword{norm} on a vector space $X$ is a function $\|{-}\| : X \to \R$ such that
\begin{itemize}
    \item (positive definite) $\|x\| = 0$ implies $x = 0$, for all $x \in X$;
    \item (absolute homogeneity) $\|cx\| = |c| \|x\|$ for all $c \in \k$ and $x \in X$;
    \item (triangle inequality) $\|x + y\| \leq \|x\| + \|y\|$ for all $x, y \in X$.
\end{itemize}

One can check that these conditions also imply
\begin{itemize}
    \item $\|x\| = 0$ if and only if $x = 0$, for all $x \in X$;
    \item $\|x\| \geq 0$ for all $x \in X$.
\end{itemize}
These slightly stronger conditions are sometimes added into the definition of a norm.

A \keyword{normed vector space} is a vector space $X$ with a norm $\|{-}\|_X$.

\sepline

Let $X$ and $Y$ be normed vector spaces.

A linear map $T : X \to Y$ is called \keyword{bounded} if there exists $C \in \R$ such that
\[
    \|Tx\|_Y \leq C\|x\|_X, \qquad \text{for all } x \in X.
\]
(Necessarily, such a $C$ would be positive.)
We also call $T$ a \keyword{bounded linear operator}.

If $T : X \to Y$ is a bounded linear operator, define
\[
    \|T\| := \inf\{C \in \R : \|Tx\|_Y \leq C\|x\|_X \text{ for all } x \in X\}.
\]
One can check that
\[
    \|T\|
        = \sup_{\substack{x \in X \\ x \ne 0}} \frac{\|Tx\|_Y}{\|x\|_X}
        = \sup_{\substack{x \in X \\ 0 < \|x\|_X \leq 1}} \frac{\|Tx\|_Y}{\|x\|_X}
        = \sup_{\substack{x \in X \\ 0 < \|x\|_X < 1}} \frac{\|Tx\|_Y}{\|x\|_X}
        = \sup_{\substack{x \in X \\ \|x\|_X = 1}} \|Tx\|_Y
\]

\sepline

Let $X$ and $Y$ be normed vector spaces and $U \seq X$ be an open subset.

A function $f : U \to Y$ is called \keyword{differentiable at} $x \in U$ if there exists a bounded linear operator $T : X \to Y$ satisfying
\[
    \lim_{\substack{\|h\|_X \to 0 \\ h \in X \setminus\{0\}}} \frac{\|f(x + h) - f(x) - T(h)\|_Y}{\|h\|_X} = 0.
\]
\[
    Q(f, x, h, T) = \frac{\|f(x + h) - f(x) - T(h)\|_Y}{\|h\|_X}
\]
In which case, we say $T$ is the* \keyword{derivative of $f$ at $x$}, written $\dd{f}_x = \dd(f)_x = T$.



For any $V \seq U$, $f$ is called \keyword{differentiable on $V$} if it is differentiable at every point of $V$.

$f$ is called \keyword{differentiable} if it is differentiable on $U$.

\sepline

Let $U \seq X$ be open and $f : U \to Y$ be differentiable.

$\LL(X, Y)$.

\sepline

$f$ is called \keyword{continuously differentiable at $x$} 


\end{document}