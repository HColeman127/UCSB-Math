\documentclass[12pt]{article}

% Packages
\usepackage[margin=1in]{geometry}
\usepackage{parskip}
\usepackage{amsmath, amsthm, amssymb}
\usepackage{tikz, tikz-cd}

\usepackage[shortlabels]{enumitem}

% Formatting
\newcommand{\keyword}[1]{\textbf{#1}}
\newcommand{\sepline}{\rule{\textwidth}{0.4pt}}

% Theorem Environments
\theoremstyle{definition}
\newtheorem{lemma}{Lemma}
\newtheorem{theorem}{Theorem}

% Tikz Environments
\newenvironment{drawing}{\begin{center}\begin{tikzpicture}}{\end{tikzpicture}\end{center}}
% \tikzcdset{row sep/normal=0pt}
\newenvironment{cd}{\begin{center}\begin{tikzcd}}{\end{tikzcd}\end{center}}

% Default Commands
\newcommand{\isp}[1]{\quad\text{#1}\quad}
\newcommand{\N}{\mathbb{N}} 
\newcommand{\Z}{\mathbb{Z}}
\newcommand{\Q}{\mathbb{Q}}
\newcommand{\R}{\mathbb{R}}
\newcommand{\C}{\mathbb{C}}
\newcommand{\A}{\mathbb{A}}
\renewcommand{\P}{\mathbb{P}}
\newcommand{\eps}{\varepsilon}
\renewcommand{\phi}{\varphi}
\renewcommand{\emptyset}{\varnothing}
\newcommand{\<}{\langle}
\renewcommand{\>}{\rangle}
\newcommand{\iso}{\cong}
\newcommand{\eqc}{\overline}
\newcommand{\clo}{\overline}
\newcommand{\seq}{\subseteq}
\newcommand{\teq}{\trianglelefteq}
\DeclareMathOperator{\id}{id}
\DeclareMathOperator{\im}{im}
\newcommand{\inc}{\hookrightarrow}
\newcommand{\dd}{\mathrm{d}}

% Extra Commands
\DeclareMathOperator{\Tran}{Tran}
\DeclareMathOperator{\Stab}{Stab}
\DeclareMathOperator{\Fix}{Fix}


% Document
\begin{document}
\title{Linear Algebraic Groups}
\author{}
\date{}
% \maketitle

Fix a base field $K$.

\sepline

An \keyword{algebraic group} $G$ is a group object in the category of algebraic varieties.

In other words, $G$ is an algebraic variety endowed with the the following structural data:
\begin{itemize}
    \item an \keyword{identity} element $1 \in G$;
    \item a \keyword{multiplication} morphism $\mu : G \times G \to G$ of varieties, denoted $\mu(x, y) = xy$;
    \item an \keyword{inversion} morphism $i : G \to G$ of varieties, denoted $i(x) = x^{-1}$;
\end{itemize}
such that $(G, 1, \mu)$ specifies a group which is coherent with the inversion $i$.

A \keyword{morphism of algebraic groups} is a morphism of the underlying varieties which is also a group homomorphism on the groups.

\sepline

Basic Properties:

Let $G$ be an algebraic group.

The inversion is an automorphism of $G$ as an algebraic group with $i^2 = \id_G$.

The left and right multiplication maps (also called left/right translation maps) are isomorphisms of algebraic groups:
\begin{align*}
    \lambda_x = (x \cdot -) : G &\to G & \rho_y = (- \cdot y) : G &\to G & \\
    y & \mapsto xy & x &\mapsto xy
\end{align*}

In particular, if $G$ has any `local' geometric properties at a point $x \in G$ then such properties hold at any other point $y \in G$, since the translation $\lambda_{yx^{-1}} : G \to G$ is an isomorphism of varieties which sends $x$ to $y$.
In other words, `local properties of algebraic groups are global.'

\sepline

\begin{lemma}
    Let $G$ be an algebraic group.
    \begin{enumerate}[(i)]
        \item $G$ has precisely one irreducible component $G^*$, containing $1$.
        \item $G^*$ is a closed normal subgroup of finite index in $G$.
        \item The irreducible components of $G$ are precisely the cosets of $G^*$.
    \end{enumerate}
\end{lemma}

It follows that the irreducible and connected components of $G$ coincide.

Denote by $G^0 = G^\circ = G^*$ the \keyword{identity component} of $G$, which is the unique connected component of $G$ containing the identity $1$.

\begin{theorem}
    Let $G$ be an algebraic group.
    \begin{enumerate}[(1)]
        \item $G^0$ is a closed normal subgroup of $G$ with finite index.
        \item The irreducible components of $G$ are precisely the cosets of $G^0$.
        \item If $H \leq G$ is a closed subgroup of finite index, then $G^0 \seq H$.
        \item $G^0$ is smooth.
    \end{enumerate}
\end{theorem}

\sepline

Let $T$ be a topological space.

A subset $D \seq T$ is \keyword{locally closed} if $D = U \cap E$ for some $U$ open and $E$ closed in $T$.
(Equivalently if $D$ is open in $\clo{D}$.)

A subset of $T$ is \keyword{constructible} if it is the union of finitely many locally closed subsets.

The set of constructible subsets of $T$ is the boolean algebra generated by all the open and closed sets in $T$.

\sepline

\begin{theorem}[Chevalley]
    If $\phi : X \to Y$ is a morphism of (quasi-projective?) varieties, then $\im \phi = \phi(X)$ is a constructible subset of $Y$.

    Moreover, if $X$ and $Y$ are irreducible and $\phi$ is \keyword{dominant} ($\clo{\phi(X)} = Y$), then there exists a dense open subset $U \seq Y$ such that for all $u \in U \cap \phi(X)$, we have
    \[
        \dim\phi^{-1}(u) = \dim X - \dim Y.
    \]
\end{theorem}

\begin{lemma}
    Let $T$ be any Noetherian topological space and $C \seq T$ constructible.
    Then there exists $U \seq C$ open with $\clo{U} = \clo{C}$.
\end{lemma}

\sepline

\begin{lemma}
    Let $G$ be an algebraic group, $U, V \seq G$ open dense, $H \leq G$ not necessarily closed.
    \begin{enumerate}[(i)]
        \item $U \cdot V = G$.
        \item $\clo{H} \leq G$.
        \item If $H$ is constructible, then $H$ is closed.
    \end{enumerate}
\end{lemma}

\begin{theorem}
    Let $\phi : G \to G'$ be a morphism of algebraic groups.
    \begin{enumerate}[(1)]
        \item $\ker\phi \leq G$ and $\im\phi \leq G'$ are both closed.
        \item $\phi(G^0) = (\im\phi)^0$.
        \item $\dim G = \dim\im\phi + \dim\ker\phi$.
    \end{enumerate}
\end{theorem}


\sepline

Morphic actions

For $V, W \seq X$, define the \keyword{transporter set}
\[
    \Tran_G(V, W) = \{g \in G \mid g \cdot V \seq W\}.
\]

For $g \in G$, define the fixed something
\[
    \Fix_X(g) = \{x \in X \mid g \cdot x = x\}
\]

\sepline

\begin{lemma}
    Let $G$ be an algebraic group acting morphically on a (quasi-projective?) variety $X$.
    \begin{enumerate}[(1)]
        \item If $W \seq X$ is closed in $X$, then $\Tran_G(V, W)$ is closed in $G$.
        \item $\Fix_X(g)$ is closed in $X$ for all $g \in G$.
        \item For any closed $H \leq G$, the normalizer $N_G(H)$ and the centralizer $C_G(H)$ are both closed in $X$.
    \end{enumerate}
\end{lemma}

\begin{theorem}
    Let $G$ be an algebraic group acting morphically on a(quasi-projective?) variety $X$.
    For $x \in X$,
    \begin{enumerate}[(1)]
        \item $G \cdot x$ is locally closed in $X$, so it is a quasi-projective variety.
        \item $G \cdot x$ is smooth.
        \item $\clo{G \cdot x} \setminus G \cdot x$ is a union of orbits, all of which have dimension less than $G \cdot x$.
        \item $\dim G \cdot x = \dim G - \dim\Stab_G(x)$.
    \end{enumerate}
\end{theorem}

\end{document}