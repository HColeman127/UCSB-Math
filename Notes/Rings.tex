\documentclass[12pt]{article}

% Packages
\usepackage[margin=1in]{geometry}
\usepackage{amsmath, amsthm, amssymb}
\usepackage[shortlabels]{enumitem}
\usepackage{comment}

% Problem Box
\setlength{\fboxsep}{4pt}
\newsavebox{\savefullbox}
\newenvironment{fullbox}{\begin{lrbox}{\savefullbox}\begin{minipage}{\dimexpr\textwidth-2\fboxsep\relax}\setlength{\parskip}{8pt}}{\end{minipage}\end{lrbox}\begin{center}\framebox[\textwidth]{\usebox{\savefullbox}}\end{center}}

% Environments
\newenvironment{definition}{\begin{fullbox}}{\end{fullbox}}
\newcommand{\keyword}[1]{\textbf{#1}}
\setlength{\parindent}{0pt}
\setlength{\parskip}{6pt}
\setlist[enumerate]{nosep}

\newcommand{\sepline}{\rule{\textwidth}{0.4pt}}


% Document Formatting
\theoremstyle{definition}
\newtheorem{theorem}{Theorem}
\newtheorem{corollary}{Corollary}
\newtheorem{lemma}{Lemma}
\newtheorem{proposition}{Proposition}

% Math Formatting
\newcommand{\ds}{\displaystyle}
\newcommand{\isp}[1]{\quad\text{#1}\quad}

% Symbols
\newcommand{\N}{\mathbb{N}}
\newcommand{\Z}{\mathbb{Z}}
\newcommand{\Zpos}{\mathbb{Z}_{\geq0}}
\newcommand{\Q}{\mathbb{Q}}
\newcommand{\R}{\mathbb{R}}
\newcommand{\C}{\mathbb{C}}
\newcommand{\eps}{\varepsilon}
\renewcommand{\phi}{\varphi}
\renewcommand{\emptyset}{\varnothing}

% Delimiters
\newcommand{\<}{\left\langle}
\renewcommand{\>}{\right\rangle}

% Relations
\newcommand{\isom}{\cong}
\newcommand{\teq}{\trianglelefteq}

% Math Roman
\newcommand{\Syl}{\operatorname{Syl}}
\newcommand{\Aut}{\operatorname{Aut}}
\newcommand{\id}{\operatorname{id}}
\newcommand{\eqc}{\overline}
\newcommand{\divides}{\mid}
\newcommand{\ndivides}{\nmid}


\title{Abstract Algebra\\
    \large UCSB MATH 111
}
\author{}
\date{2020\textendash 2021}

\begin{document}

Let $R$ and $S$ be rings.

A \keyword{ring homomorphism} is a map $\phi : R \to S$ such that for all $a, b \in R$
    \begin{enumerate}[(i)]
        \item $\phi(a + b) = \phi(a) + \phi(b)$,
        \item $\phi(ab) = \phi(a)\phi(b)$.
    \end{enumerate}
    
Let $\phi : R \to S$ be a ring homomorphism. The \keyword{kernel} of $\phi$ is
\[
    \ker \phi = \{r \in R \mid \phi(r) = 0\}.
\]
The \keyword{image} of $\phi$ is
\[
    \phi(R) = \{\phi(r) \mid r \in R\}.
    \]
    
A \keyword{ring isomorphism} is a bijective ring homomorphism. If there exists an isomorphism between rings $R$ and $S$, then $R$ and $S$ are said to be \keyword{isomorphic}, written $R \isom S$.

\sepline

Let $R$ be a ring, $I \subseteq R$, and $r \in R$.

We say $I$ is an \keyword{ideal} of $R$ if
\begin{enumerate}[(i)]
    \item $I$ is a subring of $R$,
    \item $rI \subseteq I$ and $Ir \subseteq I$ for all $r \in R$.
\end{enumerate}

We say $I$ is a \keyword{proper ideal} if $I \ne R$.

The ideal $\{0\}$ is called the \keyword{trivial ideal} of $R$, and sometimes denoted by $0$.

\sepline

Let $I$ be an ideal of $R$. The \keyword{quotient ring} of $R$ by $I$ is the set
\[
    R/I = \{r + I \mid r \in I\}
\]
with operations
\[
    (r+I) + (s+I) = (r+s) + I \isp{and} (r+I) \cdot (s+I) = (rs) + I.
\]

We often write $\eqc{r} = r + I$, and the operations become
\[
    \eqc{r} + \eqc{s} = \eqc{r + s} \isp{and} \eqc{r} \cdot \eqc{s} = \eqc{rs}.
\]

\sepline

Let $I, J$ be ideal of $R$.

Their \keyword{sum} is $I+J = \{a+b \mid a \in I, b \in J\}$.

Their \keyword{product} is $IJ = \{\sum a_kb_k \mid a_k \in I, b_k \in J\}$ with finite support, i.e., only finite sums.

\sepline

Let $R$ be a ring and $A \subseteq R$.

Denote by $(A)$ the smallest ideal of $R$ containing $A$, called the \keyword{ideal generated by} $A$.

\begin{enumerate}
    \item If $A, B \subseteq R$, then $(A) + (B) = (A \cup B)$.
    
    \item If $a_1, \dots, a_n \in R$, then $(a_1) + \cdots + (a_n) = (a_1, \dots, a_n)$.
    
    \item If $r \in R$, then $(x - r) = \{p(x) \in R[x] \mid p(r) = 0\} = I_r\}$.
    
    \item In $\Z[x]$, $(2, x) = \{2a(x) + xb(x) \mid a(x), b(x) \in \Z[x]\}$ is polynomials on $\Z[x]$ with constants in $2\Z$.
    
    \item In $\Q[x]$, we have $(2, x) = \Q[x]$.
\end{enumerate}

An ideal generated by a single element is called a \keyword{principal ideal}, i.e., $(a)$ for $a \in R$.

An ideal generated by a finite set is called a \keyword{finitely generated ideal}.

\begin{enumerate}
    \item Every principal ideal is finitely generated.
    
    \item Every ideal of $\Z$ is principal: ideals are $n\Z = (n)$ for some $n \in \Z$.
    
    \item $(2, x) \subseteq \Z[x]$ is not principal.
    
    \item In $C^0([0,1])$, the ideal $\{f \mid f(1/2) = 0\}$ is not finitely generated.
\end{enumerate}

\sepline

A proper ideal $M$ is called a \keyword{maximal ideal} if the only ideals containing $M$ are $M$ and $R$.

Two ideals $I$ and $J$ of the ring $R$ are said to be \keyword{comaximal} if $I + J = R$.

\begin{enumerate}
    \item $n\Z, m\Z \subseteq \Z$ are comaximal if and only if $n$ and $m$ are coprime.
\end{enumerate}

\sepline

A proper ideal $P$ is called a \keyword{prime ideal} if $ab \in P$ implies that either $a \in P$ or $b \in P$.

\begin{enumerate}
    \item If $n \in \Zpos$, then $(n) = n\Z$ is a prime ideal in $\Z$ if and only if $n$ is a prime number.
\end{enumerate}

\sepline

A subset $S \subseteq R$ called a \keyword{multiplicative subset} if $1 \in S$ and $ab \in S$ for all $a, b \in S$. 

\begin{enumerate}
    \item $R^\times$ is a multiplicative subset of $R$.
    
    \item If $R$ is an integral domain, then $R - \{0\}$ is a multiplicative subset of $R$.
    
    \item If $P$ is a prime ideal of $R$, then $R - P$ is a multiplicative subset of $R$.
\end{enumerate}

\sepline

Let $S$ be a multiplicative subset of the ring $R$.

Define the equivalence relation $\sim$ on $R \times S$ by
\[
    (r_1, s_1) \sim (r_2, s_2) \iff u(r_1s_2 - r_2s_1) = 0 \text{ for some } u \in S.
\]

Denote the equivalence class $\eqc{(r, s)} \in S^{-1}R$ by $\frac{r}{s}$. Then
\[
    \frac{r_1}{s_1} = \frac{r_2}{s_2} \iff u(r_1s_2 - r_2s_1) = 0 \text{ for some } u \in S.
\]

The \keyword{localization of $R$ at $S$} is the set
\[
    S^{-1}R = \{\tfrac{r}{s} \mid r \in R, s \in S\}
\]
with operations
\[
    \frac{r_1}{s_1} + \frac{r_2}{s_2} = \frac{r_1s_2 + r_2s_1}{s_1s_2}
    \isp{and}
    \frac{r_1}{s_1} \cdot \frac{r_2}{s_2} = \frac{r_1r_2}{s_1s_2}.
\]

If $R$ is an integral domain and $S^{-1} = R - 0$, then $S^{-1}R$ is the \keyword{fraction field} of $R$

Given $a \in R$ non-nilpotent, take $S = \{a^n \mid n \in \Zpos\}$. Then $S^{-1}R$ is called the \keyword{localization of $R$ at the element $a$} and denoted by $R_a$.

For a $P$ is a prime ideal of $R$, denote by $R_P = (R - P)^{-1}R$ the \keyword{localization of $R$ at the prime ideal $P$}.

\begin{enumerate}
    \item The fraction field of $\Z$ is isomorphic to $\Q$.
    
    \item $\{1\}^{-1}R \isom R$.
    
    \item If $0 \in S$, then $S^{-1}R = 0$.
    
    \item Fix $N \in \Zpos$, $S = \{N^n \mid n \in \Zpos\}$, then $S^{-1}\Z = \{m/N^n \mid m \in \Z, n \in n \in \Zpos\}$.
    
    \item If $p$ is a prime number and $S = \Z - (p)$, then $S^{-1}\Z = \{m/n \mid m \in \Z, \gcd(n, p) = 1\}$
\end{enumerate}
\sepline

Let $R$ be an integral domain.

Any function $N : R \to \Zpos$ with $N(0) = 0$ is called a \keyword{norm}. If $N(a) > 0$ for $a \ne 0$, then $N$ is called a \keyword{positive norm}.

We say $R$ is a \keyword{Euclidean domain} if there is a norm $N$ on $R$ such that for all $a, b \in R$ with $b \ne 0$ there exist $q, r \in R$ such that
\[
    a = qb + r, \quad r=0 \text{ or } N(r) < N(b).
\]
The element $q$ is called the \keyword{quotient} and $r$ the \keyword{remainder} of the division of $a$ by $b$.

\begin{enumerate}
    \item $\Z$ is a Euclidean domain with $N(a) = |a|$.
    
    \item A field is a Euclidean domain with the zero norm.
    
    \item If $F$ is a field, $F[x]$ is a Euclidean domain with $N(p(x)) = \deg p(x)$.
\end{enumerate}

\sepline

Let $R$ be a commutative ring and $a, b \in R$ with $b \ne 0$.

$a$ is said to be a \keyword{multiple} of $b$ if there exists an element $x \in R$ with $a = bx$. Then $b$ is said to \keyword{divide} $a$ or be a \keyword{divisor} of $a$, written $b \divides a$.

A \keyword{greatest common divisor} (gcd) of $a$ and $b$ is a nonzero element $d$ such that
\begin{enumerate}[(i)]
    \item $d \divides a$ and $d \divides b$,
    \item if $d' \divides a$ and $d' \divides b$ then $d' \divides d$.
\end{enumerate}
In which case, we denote $d = \gcd(a, b)$.

\begin{enumerate}
    \item If $R$ is a PID, $a, b \in R$ with $b \ne 0$, then $(a, b) = (d)$ for some $d \in R$. Moreover, $d$ is a gcd of $a$ and $b$.
\end{enumerate}

\sepline

A \keyword{principal ideal domain} (PID) is an integral domain in which every ideal is principal.

\begin{enumerate}
    \item $\Z$ is a PID, but $\Z[x]$ is not.
\end{enumerate}

\sepline

Let $R$ be an integral domain.

A nonzero, non-unit element $r \in R$ is called \keyword{irreducible} in $R$ if
\[
    r = ab \implies a \in R^\times \text{ or } b \in R^\times,
\]
and \keyword{reducible}, otherwise.

A nonzero element $p \in R$ is called \keyword{prime} in $R$ if $(p)$ is a prime ideal of $R$. Equivalently, a nonzero, non-unit element $p \in R$ is prime  if
\[
    p \divides ab \implies p \divides a \text{ or } p \divides b.
\]

Two elements $a, b \in R$ are said to be \keyword{associate} in $R$ if $a = ub$ for some $u \in R^\times$.

\sepline

A \keyword{unique factorization domain} (UFD) is an integral domain $R$ in which every nonzero, non-unit element $r \in R$ has the following:
\begin{enumerate}[(i)]
    \item $r = p_1 \cdots p_n$ where each $p_i$ is irreducible in $R$,
    \item this decomposition is unique up to associates, i.e., if $r = q_1 \cdots q_m$ is another factorization into irreducibles, then $m = n$ and there is a renumbering such that $p_i$ is associate to $q_i$ for $i = 1, \dots, n$.
\end{enumerate}

\sepline

A ring $R$ is called \keyword{Noetherian} if every ideal is finitely generated.

\sepline

An integer $a$ is called a \keyword{primitive root} mod $n$ if $\eqc{a}$ is a generator of $(\Z/n\Z)^\times$.

\newpage


\begin{theorem}(First Isomorphism Theorem)
    Let $\phi : R \to S$ be a ring homomorphism.
    \begin{enumerate}
        \item $\ker \phi$ is an ideal of $R$,
        \item $\phi(R)$ is a subring of $S$,
        \item $R/\ker\phi \isom \phi(R)$.
    \end{enumerate}
    
    
    If $I$ is an ideal of $R$, then the natural projection
    \begin{align*}
        \pi : R &\to R/I \\
            r &\mapsto r + I
    \end{align*}
    is a surjective ring homomorphism with $\ker\pi = I$.
\end{theorem}

\begin{theorem}(Second Isomorphism Theorem)
    Let $A$ be a subring and $I$ be an ideal of $R$.
    \begin{enumerate}
        \item $A + I$ is a subring of $R$,
        \item $A \cap I$ is an ideal of $A$ and $I$ is an ideal of $A + I$,
        \item $(A + I)/I \isom A/(A \cap I)$.
    \end{enumerate}
\end{theorem}

\begin{theorem}(Third Isomorphism Theorem)
    Let $I$ and $J$ be ideals of $R$ with $I \subseteq J$.
    \begin{enumerate}
        \item $J/I$ is an ideal of $R/I$,
        \item $(R/I)/(J/I) \isom R/J$.
    \end{enumerate}
\end{theorem}

\begin{theorem}(Fourth Isomorphism Theorem)
    Let $I$ be an ideal of $R$. The map
    \begin{align*}
        \{\text{ideals of $R$ containing $I$}\} &\to \{\text{ideals of $R/I$}\} \\
        J &\mapsto J/I
    \end{align*}
    is an inclusion preserving bijection.
\end{theorem}

\sepline

\begin{theorem}(Chinese Remainder Theorem)
    Let $I_1, \dots, I_n$ be ideals of $R$. The map
    \begin{align*}
        \phi : R &\to R/I_1 \times \cdots \times R/I_n \\
            r &\mapsto (r+I_1, \dots, r+I_n)
    \end{align*}
    is a ring homomorphism with $\ker \phi = I_1 \cap \cdots \cap I_n$.
    
    If $I_i$ and $J_j$ are comaximal for $i \ne j$, then this map is surjective and $I_1 \cap \cdots \cap I_n = I_1 \cdots I_n$, so
    \[
        R/(I_1 \cdots I_n) \isom R/I_1 \times \cdots \times R/I_n.
    \]
\end{theorem}

\begin{corollary}
    Let $n$ be a positive integer and let $p_1^{\alpha_1} \cdots p_k^{\alpha_k}$ be its factorization into powers of distinct primes. Then
    \[
        \Z/n\Z \isom (\Z/p_1^{\alpha_1}\Z) \times \cdots \times (\Z/p_k^{\alpha_k}\Z)
    \]
\end{corollary}

\begin{corollary}
    Given $a_1, \dots, a_n, c_1, \dots, c_n \in \Q$ with $a_i \ne a_j$ for $i \ne j$. There exists a polynomial $p(x) \in \Q[x]$ such that $p(a_j) = c_j$ for $j = 1, \dots, n$.
\end{corollary}

\sepline

If $I$ is an ideal of $R$, then $I = R$ if and only if $I$ contains a unit.

$R$ is a field if and only if it has no nontrivial proper ideals, i.e., its only ideals are $0$ and $R$.

If $R$ is a field, then any nonzero ring homomorphism with domain $R$ is an injection.

(id) Every proper ideal is contained in a maximal ideal.

(comm) An ideal $M$ is maximal if and only if $R/M$ is a field.

(comm) An ideal $P$ is prime if and only if $R/P$ is an integral domain.

(comm) Every maximal ideal is a prime ideal.

\sepline

Every ideal in a Euclidean domain is principal.

Every nonzero prime ideal in a PID is maximal.

$R[x]$ is a PID if and only if $R$ is a field.

Let $R$ be an integral domain, $r \in R$. If $r$ is prime in $R$, then $r$ is irreducible in $R$.

A PID is a UFD.

In a UFD, an element is prime if an only if it is irreducible.

In a UFD, every nonzero non-unit has a prime factorization, unique up to associates.


\sepline

\begin{lemma}(Gauss' Lemma)
    Let $R$ be a UFD with fraction field $F$ and let $p(x) \in R[x]$. If $p(x)$ is reducible in $F[x]$ then $p(x)$ is reducible in $R[x]$. More precisely, if $p(x) = A(x)B(x)$ for some nonconstant polynomials $A(x), B(x) \in F[x]$, then there are nonzero elements $r, s \in F$ such that $rA(x) = a(x)$ and $sB(x) = b(x)$ both lie in $R[x]$ and $p(x) = a(x)b(x)$ is a factorization in $R[x]$.
\end{lemma}

$R[x]$ is a UFD if and only if $R$ is a UFD.

If $R$ is an integral domain and $r \in R$, then $r$ is irreducible/prime in $R$ if and only if it is irreducible/prime in $R[x]$.

\begin{corollary}
    Let $R$ be a UFD with fraction field $F$. If $p(x) \in R[x]$, then $p(x)$ is irreducible in $R[x]$ if and only if $p(x)$ is irreducible in $F[x]$ and the gcd of its coefficients is $1$. In particular, if $p(x)$ is a monic polynomial that is irreducible in $R[x]$, then $p(x)$ is irreducible in $F[x]$.
\end{corollary}

If $R$ is a UFD and $p(x) \in R[x]$, then $(p(x))$ is a prime ideal of $R[x]$ if and only if $p(x)$ is irreducible in $R[x]$.

If $F$ is a field and $p(x) \in G[x]$, then $(p(x))$ is a maximal ideal of $F[x]$ if and only if $p(x)$ is irreducible in $F[x]$.

Let $F$ be a field and $p(x) \in F[x]$. Then $p(x)$ has a degree one factor if and only if $p(x)$ has a root in $F$.

Let $F$ be a field. Then a polynomial of $F[x]$ of degree two or three is reducible if and only if it has a root in $F$.

Let $R$ be an integral domain, $I$ be a proper ideal of $R$, and $p(x) \in R[x]$ be a monic polynomial. If $\eqc{p(x)} \in (R/I)[x]$ cannot be factored into two polynomials of smaller degree, then $p(x)$ is irreducible in $R[x]$.

\begin{proposition}(Eisenstein's Criterion)
    Let $R$ be an integral domain, $P$ be a prime ideal of $R$, and $f(x) = x^n + a_{n-1}x^{n-1} + \cdots + a_1 x + a_0 \in R[x]$ with $n \geq 1$. If $a_{n-1}, \dots, a_1, a_0 \in P$ and $a_0 \notin P^2$, then $f(x)$ is irreducible in $R[x]$.
\end{proposition}

\begin{corollary}(Eisenstein's Criterion for $\Z[x]$)
    Let $p$ be a prime in $\Z$ and $f(x) = x^n + a_{n-1}x^{n-1} + \cdots + a_1 x + a_0 \in \Z[x]$ with $n \geq 1$. If $p \divides a_j$ for $j = 0, 1, \dots, n-1$ but $p \ndivides a_0$, then $f(x)$ is irreducible in both $\Z[x]$ and $\Q[x]$.
\end{corollary}

\sepline

A ring is Noetherian if and only if every ascending chain of $R$ eventually stabilizes, i.e, for all sequences $\{I_j\}_{j\in\N}$ of ideals of $R$ with $I_j \subseteq I_{j+1}$, there exists $N \in \N$ such that $I_n = I_N$ for all $n \geq N$.

Let $R$ be a Noetherian ring. If $I$ is an ideal of $R$, then $R/I$ is Noetherian. If $S$ is a multiplicative subset of $R$, then $S^{-1}R$ is Noetherian.

\begin{theorem}(Hilbert's Basis Theorem)
    If $R$ is a Noetherian ring, then so is $R[x]$.
\end{theorem}

\begin{theorem}(Primitive Root Theorem)
    Let $F$ be a field. Then any finite subgroup of $F^\times$ is cyclic. In particular if $p$ is a prime number, then $(\Z/p\Z)^\times$ is cyclic.
\end{theorem}

Let $n \geq 2$ be an integer. Then $(\Z/n\Z)^\times$ is cyclic if and only if $n = 2, 4, p^m, 2p^m$ where $p$ is an odd prime and $m$ is a positive integer.

\end{document}