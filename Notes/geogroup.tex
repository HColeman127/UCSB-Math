\documentclass[12pt]{article}

% Packages
\usepackage[margin=1in]{geometry}
\usepackage{parskip}
\usepackage{amsmath, amsthm, amssymb}
\usepackage{tikz, tikz-cd}
\usepackage[shortlabels]{enumitem}

\usepackage{bbm}

\usepackage{suffix}
\usetikzlibrary{decorations.pathmorphing}
\usetikzlibrary{decorations.markings, arrows.meta}

\tikzset{
    marrow/.style={decoration={
        markings,
        mark=at position 0.5 with {\arrow{>}}}, postaction=decorate,
    }
}


\makeatletter
\tikzset{nomorepostaction/.code=\let\tikz@postactions\pgfutil@empty}
\makeatother

% Problem Box
\setlength{\fboxsep}{4pt}
\newlength{\myparskip}
\setlength{\myparskip}{\parskip}
\newsavebox{\savefullbox}
\newenvironment{fullbox}{\begin{lrbox}{\savefullbox}\begin{minipage}{\dimexpr\textwidth-2\fboxsep\relax}\setlength{\parskip}{\myparskip}}{\end{minipage}\end{lrbox}\framebox[\textwidth]{\usebox{\savefullbox}}}

% Environments
% \setlist[enumerate]{nosep}
\newcommand{\keyword}[1]{\textbf{#1}}
\newcommand{\sepline}{\rule{\textwidth}{0.4pt}}

% Tikz Environments
\newenvironment{drawing}{\begin{center}\begin{tikzpicture}}{\end{tikzpicture}\end{center}}
% \tikzcdset{row sep/normal=0pt}
\newenvironment{cd}{\begin{center}\begin{tikzcd}}{\end{tikzcd}\end{center}}


% Document Formatting
\newtheoremstyle{mythmstyle}% name of the style to be used
{ }% measure of space to leave above the theorem. E.g.: 3pt
{ }% measure of space to leave below the theorem. E.g.: 3pt
{ }% name of font to use in the body of the theorem
{ }% measure of space to indent
{\scshape}% name of head font
{.}% punctuation between head and body
{ }% space after theorem head; " " = normal interword space
{\thmname{#1}\thmnumber{ #2}\thmnote{ (#3)}}% Manually specify head

\theoremstyle{definition}
\newtheorem{theorem}{Theorem}
\newtheorem{corollary}{Corollary}
\newtheorem{lemma}{Lemma}
\newtheorem{proposition}{Proposition}


% Math Formatting
\newcommand{\isp}[1]{\quad\text{#1}\quad}

% mathbb
\newcommand{\N}{\mathbb{N}}
\newcommand{\Z}{\mathbb{Z}}
\newcommand{\Q}{\mathbb{Q}}
\newcommand{\R}{\mathbb{R}}
\newcommand{\C}{\mathbb{C}}

% mathcal
\renewcommand{\AA}{\mathcal{A}}
\newcommand{\BB}{\mathcal{B}}
\newcommand{\CC}{\mathcal{C}}
\newcommand{\DD}{\mathcal{D}}
\newcommand{\OO}{\mathcal{O}}

% mathfrak
\renewcommand{\gg}{\mathfrak{g}}


% Symbols

\newcommand{\eps}{\varepsilon}
\renewcommand{\phi}{\varphi}
\renewcommand{\emptyset}{\varnothing}

% Delimiters
\newcommand{\<}{\left\langle}
\renewcommand{\>}{\right\rangle}

% Relations
\newcommand{\seq}{\subseteq}
\newcommand{\iso}{\cong}
% \newcommand{\qisom}{\underset{\mathrm{QI}}{\sim}}
\newcommand{\qisom}{\sim_\mathrm{QI}}

\newcommand{\teq}{\trianglelefteq}

\newcommand{\inc}{\hookrightarrow}
\newcommand{\To}{\longrightarrow}
\newcommand{\Mapsto}{\longmapsto}



% Math Roman
\newcommand{\id}{\mathrm{id}}
\newcommand{\source}{\mathrm{source}}
\newcommand{\target}{\mathrm{target}}

\newcommand{\QMet}{\mathsf{QMet}}
\newcommand{\QI}{\mathrm{QI}}



% Math Operators
\DeclareMathOperator{\Aut}{Aut}
\DeclareMathOperator{\Cay}{Cay}
\DeclareMathOperator{\Isom}{Isom}
\DeclareMathOperator{\rk}{rk}


% Other
\newcommand{\clo}{\overline}





\title{Geometric Group Theory \\
    \large 
}
\author{}
\date{}


\begin{document}

Geometry cares about distance, so we study metric spaces.

\sepline

Let $X$ be a set.

A \keyword{metrix} on $X$ is a function $d : X \times X \to \R$ such that
\begin{itemize}
    \item $d(x, y) = 0$ if and only if $x = y$ (identity of indiscernibles / Leibniz's law);
    \item $d(x, y) = d(y, x)$ (symmetry);
    \item $d(x, z) \leq d(x, y) + d(y, z)$ (triangle inequality).
\end{itemize}

Some authors require the following addition property (which follows from the firs three):
\begin{itemize}
    \item $d(x, y) \geq 0$ (nonnegativity).
\end{itemize}

A \keyword{metric space} is a set $X$ with a metric $d_X$.

\sepline

Examples

\begin{itemize}
    \item A normed vector space $X$ (e.g., $\R^n$) has a metric $d(x, y) = \|x - y\|_X$.
    \item Any set has the \keyword{discrete metric} $d(x, x) = 0$ and $d(x, y) = 1$ for $x \ne y$.
    \item $X$ a set and $Y$ a metric space, the set of bounded functions $X \to Y$ is a metric space with $d(f, g) = \sup_{x \in X} d_Y(f(x), g(x))$.
    (This might be called the uniform metric or supremum (sup) metric.)
    \item $X$ a metric space and $f : [0, \infty) \to [0, \infty)$ a nondecreasing concave function with $f^{-1}(0) = \{0\}$, then $f \circ d$ is a metric on $X$.
    \item $X$ a set, $Y$ a metric space, and $f : X \to Y$ an injection, then $d(x, y) = d_Y(f(x), f(y))$ is a metric on $Y$.
\end{itemize}

\sepline

Important Example.

Let $\Gamma$ be a connected graph, then set of vertices $V$ has a metric given by the shortest path between two vertices.
\[
    d(u, v) = \min_{P_\Gamma(u, v)} |{-}| = \min\{|p| : p \in \mathrm \Gamma[u, v]\}.
\]
This is called the \keyword{path metric} of $V$ with respect to $\Gamma$.

Given a group $G$ and generating set $S$, the \keyword{word metric} $d_S$ on $G$ with respect to $S$ is the path metric on $G$ with respect to the Cayley graph $\Cay(G, S)$, i.e.,
\[
    d_S(g, h) = \min\{n \in \N \mid gs_1 \cdots s_n = h \text{ for some } s_i \in S\}.
\]
The distance $d_S(1_G, g)$ is called the $\keyword{word length}$ of $g$ with respect to $S$.

Note
\[
    gs_1 \cdots s_n = h \iff s_1 \cdots s_n = g^{-1}h,
\]
so $d_S(g, h) = d_S(1_G, g^{-1}h)$.

\sepline


Let $X$ be a metric space.

Given $r > 0$ and $x \in X$, we construct the \keyword{open ball} in $X$ of radius $r$ centered at $x$ as follows:
\[
    B_r(x) = B_r(x; X) = B_r(x; d_X) = \{y \in X \mid d_X(x, y) < r\}.
\]
The second and third notation may be used to specify the specific space/metric with respect to which the ball is constructed.
Various notations are used for this set, including but not limited to:
\[
    B(x; r), \quad B_X(x; r), \quad B^X_r(x), \quad B(x; r; X), \quad B(x; r; d_X).
\]
Hopefully the sort of pattern is clear---it is usually fairly easy to intuit any given authors convention.
Usually any mix of a ``$B$'' and the necessary data denotes an open ball.

Sometimes this notation is also used for the following slightly modified construction.
We construct the \keyword{closed ball} in $X$ of radius $r$ centered at $x$ as follows:
\[
    \bar{B}_r(x) = \clo{B}_r(x) = \{y \in X \mid d_X(x, y) \leq r\}.
\]
There are also all the corresponding notational variants featuring``$\bar{B}$'' or ``$\clo{B}$.'' 




\sepline

Every metric space $X$ is `naturally' a topological space, with a basis of open sets given by the collection of all open balls:
\[
    \{B_r(x) \seq X : r > 0, x \in X\}.
\]
The topology generated by this basis is called the \keyword{metric topology} on $X$.

It it worth remarking that $\bar{B}_r(x) \ne \clo{B_r(x)}$ in general, where the former is the closed ball and the latter is the closure of the open ball in the metric topology.
On the other hand, it is always true that $\clo{\bar{B}_r(x)} = \bar{B}_r(x)$, i.e, $\bar{B}_r(x)$ is closed in the metric topology.

Something continuity something $\eps$-$\delta$.

\sepline

Examples

\begin{itemize}
    \item $\R^n$ with any usual metric induces Euclidean topology.
    \item discrete metric induces discrete topology.
\end{itemize}

\sepline

Let $X$ and $Y$ be metric spaces.

An \keyword{isometric embedding} is a function $f : X \to Y$ such that $d_Y(f(x), f(y)) = d_X(x, y)$.

All isometric embeddings are injective.)

Every isometric embedding is a continuous with respect to the metric topologies.

An \keyword{isometry} is a surjective isometric embedding.

All isometries are bijective.
Their inverses are isometric embeddings and isometries.

Every isometry is a homeomorphism with respect to the metric topologies.

Isometry is very strong---too strong...

\sepline

Let $f : X \to Y$ be a function between metric spaces.

Say $f$ is \keyword{Lipschitz} (or Lipschitz continuous) if there is a constant $C \in \R$ such that
\[
    d_Y(f(x), f(y)) \leq C d_X(x, y).
\]
In this case, also say $f$ is \keyword{$C$-Lipschitz}.

Since the metric is nonnegative, nontrivial $C$-Lipschitz functions only exist for $C > 0$, so we typically assume that $K$ is positive (or at least nonnegative).
Some authors require this in the definition, but it is a minor point in any case.

All lipschitz functions are continuous with respect to the metric topologies.

Say $f$ is \keyword{bilipschitz} (or called a \keyword{bilipschitz embedding}) if there exists a (positive) constant $K \in \R$ such that
\[
    \frac{1}{C} d_X(x, y) \leq d_Y(f(x), f(y)) \leq C d_X(x, y).
\]
In this case, also say $f$ is \keyword{$C$-bilipschitz}.

A $C$-bilipschitz function is clearly $C$-Lipschitz.

All bilipschitz functions are injections.

A \keyword{bilipschitz equivalence} is a bilipschitz function with a bilipschitz inverse. (equiv, a surjective bilipschitz function).

Say two spaces are \keyword{bilipschitz equivalent} if there exists a bilipschitz equivalence between them.

\sepline

Let $f : X \to Y$ be a function between metric spaces.

Say $f$ is a \keyword{quasi-isometric embedding} if there exist constants $C, K \in \R$ such that 
\[
    \frac{1}{C} d_X(x, y) - K \leq d_Y(f(x), f(y)) \leq C d_X(x, y) + K.
\]
In this case, also say $f$ is a \keyword{$(C, K)$-quasi-isometric embedding}.
(slogan ``almost injective'')

Say $f$ is a \keyword{quasi-isometry} if it is a $(C, K)$-quasi-isometric embedding and there exists a constant $D \in \R$ such that every point in $Y$ is a distance at most $D$ from a point in the image of $f$, i.e., for all $y \in Y$ there exists $x \in X$ such that $d_Y(y, f(x)) \leq D$.
(slogan ``almost surjective'')

Say $X$ and $Y$ are \keyword{quasi-isometric}, written $X \qisom Y$, if there exists a quasi-isometry between them.


It is equivalent to take a single constant $R = \max\{C, K\}$ in the definition of quasi-isometric embedding, and likewise $R = \max\{C, K, D\}$ in the definition of quasi-isometry.

``$\qisom$'' is an equivalence relation on metric spaces.

Some equivalent definition of quasi-isometry in terms of quasi-inverse and...

Say two functions $f, g : X \to Y$ between metric spaces are a \keyword{finite distance} from each other if there is a constant $D \in \R$ such that $d_Y(f(x), g(x)) \leq D$.

If $f : X \to Y$ and $g : Y \to X$ are maps such that $f \circ g$ is a finite distance to $\id_Y$ and $g \circ f$ is a finite distance to $\id_X$, say $f$ and $g$ are \keyword{quasi-inverse} to each other.

\sepline

Categories

Let $\mathsf{Met_{isom}}$ and $\mathsf{Met_{bilip}}$ be the categories whose objects are metric spaces and whose morphisms are isometric embeddings and bilipschitz embeddings, respectively.
Then the isomorphisms in these categories are the respective equivalences between metric spaces.

Let $\QMet'$ be the category whose objects are metric spaces and whose morphisms are quasi-isometric embeddings.
The isomorphisms in this category are bijective quasi-isometric embeddings whose inverse is also a quasi-isometric embedding, not the quasi-isometries.
The relation of ``finite distance from'' describes an equivalence relation on $\QMet'(X, Y)$.
We define the quotient category
\[
    \QMet = \QMet'/\text{finite distance},
\]
whose objects are metric spaces and whose morphisms are equivalence classes of quasi-isometric embeddings modulo finite distance.
The isomorphisms in $\QMet$ are precisely the quasi-isometries, as desired.

Natural inclusions (essentially subcategories with less morphisms):
\begin{cd}
    \mathsf{Met_{isom}} \rar[hook] & \mathsf{Met_{bilip}} \rar[hook] & \QMet.
\end{cd}


For $X \in \QMet$ (any metric space), the \keyword{quasi-isometry group} of $X$ is
\[
    \QI(X) := \Aut_{\QMet}(X).
\]
In other words, the group of quasi-isometries $X \to X$ modulo finite distance.

\sepline

Let $X$ be a metric space.

A \keyword{geodesic segment} in $X$ is an isometric embedding $[a, b] \to X$ of an interval $[a, b] \seq \R$.

A \keyword{geodesic} (line) in $X$ is an isometric embedding $\R \to X$.

We often identify geodesics (and geodesic segments) with their images in $X$.

We say $X$ is a \keyword{geodesic metric space} if for any $x, y \in X$, there is a geodesic segment $\gamma : [a, b] \to X$ such that $\gamma(a) = x$ and $\gamma(b) = y$.
In other words, any two points in $X$ are connected by a geodesic segment that realizes their distance.

\sepline

Geodesic metric spaces.

\begin{itemize}
    \item $\R^n$ with Euclidean metric; geodesics are straight lines.
    \item Any convex set in $\R^n$.
    \item $\R^n \setminus\{0\}$ is not geodesic as $v$ and $-v$ have no stright line between them for $v \ne 0$.
    \item For $G$ finitely generated group, $\Cay(G, S)$ with the path metric is geodesic.
    When $G$ is a free group over $S$, $\Cay(G, S)$ is a tree so there is a unique path between any pair of points.
\end{itemize}

\sepline

A metric space $X$ is \keyword{proper} if for all $x \in X$ and $r > 0$ the closed ball $\bar{B}_r(x)$ is a compact subset of $X$.

\sepline

Let $X$ be a metric space.

A \keyword{quasi-geodesic segment} in $X$ is a quasi-isometric embedding $[a, b] \to X$.

A \keyword{quasi-geodesic} (line) in $X$ is a quasi-isometric embedding $\R \to X$.

\sepline

Let $G$ be a group and $X$ a metric space.

An \keyword{action by isometries} of $G$ is a group homomorphism $G \to \Isom(X)$.



\newpage
\sepline

Let $G$ be a group and $X \seq G$.

Say $X$ \keyword{generates} $G$, written $G = \<X\>$ if every element $g \in G$ can be expressed as a product of elements in $X \cup X^{-1}$, where $X^{-1} = \{x^{-1} | x \in X\}$.
That is, there exists $x_i \in X$ and $\eps_i \in \{\pm 1\}$ such that
\[
    g = x_1^{\eps_1} x_2^{\eps_2} \cdots x_n^{\eps_n}.
\]
Say $X$ is \keyword{symmetric} if $X = X^{-1}$, i.e., $x^{-1} \in X$ for all $x \in X$.
If this is the case, we can simply express $g \in G$ as $g = x_1x_2 \cdots x_n$ for some $x_i \in X$.

If $G$ is finitely generated, we define the \keyword{rank} of $G$, denoted $\rk(G)$, to be the minimal cardinality of a finite generating subset, i.e.,
\[
    \rk(G) = \min\{|X| : G = \<X\>\}.
\]

If $H \leq G$ is finite index, then $G$ is finitely generated if and only if $H$ is finitely generated.
In particular,
\[
    \rk(G) \leq [G : H] + \rk(G) - 1
\]
and
\[
    \rk(H) \leq [G : H] \cdot \rk(G).
\]



\begin{theorem}[Ping-Pong Lemma]
    Let $G$ be a group generated by $X \seq G$ with $|X| \geq 2$.
    Suppose $G$ acts on a set $E$ and there is a family $\{A_x\}_{x \in X}$ of nonempty subsets of $E$ such that for all distinct $x, y \in X$ we have $A_y \nsubseteq A_x$ but $x^kA_y \seq A_x$ for all nonzero $k \in \Z$.
    Then $G$ is a free group with basis $X$.
\end{theorem}


\sepline

Let $G$ be a group.

A \keyword{series/chain} in $G$ is a collection of subgroups $G_i \leq G$ for $i \in \Z$ such that $G_i \leq G_{i+1}$.

We usually care about \keyword{ascending} chains, i.e.,
\[
    1 = G_0 \leq G_1 \leq G_2 \leq \cdots \leq G,
\]
and \keyword{descending} chains, i.e.,
\[
    G = G_0 \geq G_{-1} \geq G_{-2} \geq \cdots \geq 1,
\]


A (normal) series $\{G_i \teq G\}_{i \in \Z}$ is \keyword{central} if and $G_{i+1}/G_i$ is contained in the center of $G/G_i$.

\sepline

Let $G$ be a group.

The \keyword{commutator} of two elements $a, b \in G$ is the element
\[
    [a, b] := a^{-1}b^{-1}ab \in G.
\]
Also define $a^b := b^{-1}ab$.

For $a_i \in G$ recursively define
\[
    [a_1, \dots, a_n] := [[a_1, \dots, a_{n-1}], a_n],
\]
called a \keyword{simple commutator} of \keyword{weight} $n$.

Some identities:
\begin{enumerate}[(i)]
    \item $[a, b]^{-1} = [b, a]$
    \item $[ab, c] = [a, c]^b[b, c] = [a, c][a, c, b][b, c]$
    \item $[a, bc] = [a, c][a, b]^c = [a, c][b, a, c][a, b]$
    \item $[a, b^{-1}] = ([a, b]^{b^{-1}})^{-1}$
    \item $[a^{-1}, b] = [b, a]^{a^{-1}}$
    \item $[a, b^{-1}, c]^b[b, c^{-1}, a]^c[c, a^{-1}, b]^a = 1$ (Hall-Witt)
\end{enumerate}


In this notation, a subgroup $H \leq G$ is normal if and only if $h^g \in H$ for all $h \in H$ and $g \in G$.

Say $H$ is \keyword{characteristic} in $G$ if $\phi(H) \seq H$ for all $\phi \in \Aut(G)$.

Given subgroups $H, K \leq G$, define \keyword{commutator subgroup}
\[
    [H, K] := \<[h, k] : h \in H, k \in K\> \leq G.
\]
Notice $[H, K] = [K, H]$.

$[H, K] \teq H$ whenever $H \teq G$, and $[H, K] \teq G$ whenever $H, K \teq G$.

\begin{lemma}[three subgroup]
    Let $N \teq G$ and $H, K, L \leq G$.
    If $[[H, K], L]$ and $[[K, L], H]$ are contained in $N$, then so is $[[L, H], K]$.
\end{lemma}

\sepline

Let $G$ be a group.

The \keyword{lower central series} of $G$ is the descending central series 
$\{\gamma_k(G)\}_{k \geq 1}$ defined recursively by
\[
    \gamma_1(G) := G
    \isp{and}
    \gamma_{k+1}(G) := [\gamma_k(G), G].
\]
(Check that this is indeed a central series).


Say $G$ is \keyword{nilpotent} if its lower central series is finite.
In which case, its length is referred to as the \keyword{nilpotency class} of $G$.

\begin{lemma}
    $[\gamma_i(G), \gamma_j(G)] \seq \gamma_{i + j}(G)$ for all $i, j \geq 1$.
\end{lemma}

\begin{lemma}
    If $G \geq G_1 \geq G_2 \geq \cdots $ is a descending central series, then $G_i \supseteq \gamma_i(G)$ for all $i$.
\end{lemma}

\begin{proposition}
    A group is nilpotent if and only if it admits a finite central series.
\end{proposition}

\begin{proposition}
    If nilpotent $G$ is has nilpotency class $c$, then all subgroups and quotients of $G$ have nilpotency class at most $c$.
\end{proposition}


\sepline

The \keyword{center} of a group $G$ is the set
\[
    Z(G) := \{z \in G \mid zg = gz \text{ for all } g \in G\}.
\]


The \keyword{upper central series} of $G$ is the ascending central series defined recursively by
\[
    Z_0(G) := 1
    \isp{and}
    Z_{i+1}(G) := \pi_i^{-1}(Z(G/Z_i(G))),
\]
where $\pi_i : G \to G/Z_i(G)$ is the quotient map.
Then we have
\[
    Z(G/Z_i(G)) = Z_{i+1}(G)/Z_i(G).
\] 
In particular, $Z_1(G) = Z(G)$.

\begin{lemma}
    If $G = G_1 \geq G_2 \geq \cdots \geq G_r = 1$ is a finite central series, then $Z_i(G) \supseteq G_{r-i}$ for all $i$.
    In particular, $Z_{r-1}(G) = G$.
\end{lemma}

\begin{proposition}
    A group is nilpotent if and only if its upper central series is finite.
\end{proposition}

\sepline

A group is \keyword{torsion-free} if it has no nontrivial elements of finite order.

\begin{lemma}
    Let $G$ be a torsion-free nilpotent group with upper central series $\{Z_k(G)\}_{k \geq 0}$.
    Then all factors $Z_{i+1}(G)/Z_i(G)$ are torsion-free abelian groups.
\end{lemma}

\begin{theorem}[Malcev]
    Every finitely generated torsion-free nilpotent group embeds into a group of upper triangular integer matrices, i.e., $G \inc \mathrm{UT}_n(\Z)$ for some $n$.
\end{theorem}

\begin{lemma}
    It suffices to embed $G \in \mathrm{UT}_n(\Q)$.
\end{lemma}

\sepline

Let $G$ be a group with a finite symmetric generating set $S$.

The \keyword{growth function} of $G$ relative to $S$ is the function $\beta_{G,S} : \N \to \N$ defined by
\[
    \beta_{G, S}(n)
        := |B_n^{G,S}(e)|
        = \#\{g \in G \mid d_S(g, e) \leq n\}.
\]

Fixing $G$ and $S$, we briefly omit them from notation.

Note that $B_0(e) = \{e\}$, so $\beta(0) = 1$ and $\beta(n) \leq \beta(n + 1)$ for all $n \in \N$.

We also have $B_1(e) = S \cup \{e\}$, so $\beta(1) = |S \cup \{e\}|$.
(Either $|S|$ or $|S| + 1$ depending on whether $e \in S$ or $e \notin S$, respectively.)

The product map gives a surjection
\begin{align*}
    (S \cup \{1\})^{\times n} &\longrightarrow B_n(e) \\
    (s_1, \dots, s_n) &\longmapsto s_1 \cdots s_n
\end{align*}
so $\beta(n) \leq |S \cup \{1\}|^n$.


Some properties:
\begin{enumerate}[(1)]
    \item (Sub-multiplicativity) $\beta(m + n) \leq \beta(m) \cdot \beta(n)$.
    \item If $G$ is infinite, then $\beta(n) < \beta(n + 1)$ for all $n \in \N$.
    In particular, $\beta(n) \geq n$.
    \item $\beta_{G,S}(n) \leq \beta_{F, S}(n)$, where $F$ is the free group on $S$.
\end{enumerate}

\sepline

A \keyword{generalized growth function} is an increasing function $[0, \infty) \to [0, \infty)$ of nonnegative real numbers.

Let $f, g : [0, \infty) \to [0, \infty)$ be generalized growth functions.

Say $g$ \keyword{quasi-dominates} $f$, written $f \preceq g$, if there exist constants $C, K$ such that
\[
    f(r) \leq Cg(Cr + K) + K
\]
for all $r \in [0, \infty)$.

Say $f$ and $g$ are \keyword{quasi-equivalent}, written $f \sim g$, if $f \preceq g$ and $g \preceq f$.

For $G$ generated by $S$, have induced generalized growth function $r \mapsto \beta_{G,S}(\lceil r \rceil)$.

\sepline

Examples

Use $x$ as shorthand for identity in function expressions.

For $a\geq 0$, the power function $x^a : r \mapsto r^a$ is a generalized growth function.

For $a, b \geq 0$, $x^a \preceq x^b$ if and only if $a \leq b$, so $x^a \sim x^b$ if and only if $a = b$.

If $f : [0, \infty) \to [0, \infty)$ is a polynomial function of degree $d \geq 0$, then $f \sim x^d$.

For $a > 1$, the exponential function $a^x : r \mapsto a^r$ is a generalized growth function.

For $a, b > 1$, $a^x \sim b^x$.
In particular, $a^x \sim \exp$.

For $a \geq 0$, $x^a \prec \exp$ (strict, i.e., $x^a \preceq \exp$ but $x^a \not\sim \exp$).

\sepline



\begin{proposition}
    Let $G$ and $H$ be groups with finite generating sets $S$ and $T$, respectively.
    \begin{enumerate}[(1)]
        \item If there exists a quasi-isometric embedding $(G, d_S) \to (H, d_T)$, then $\beta_{G,S} \preceq \beta_{H,T}$.
        \item In particular, if $G$ and $H$ are quasi-isometric then $\beta_{G,S} \sim \beta_{H,T}$.
    \end{enumerate}
\end{proposition}


\begin{corollary}
    Let $G$ be a group with finite generating sets $S$ and $T$.
    Then $\beta_{G,S} \sim \beta_{G,T}$.
    Moreover, $\beta_{G,S} \preceq \exp$.
\end{corollary}


The \keyword{growth type} of finitely generated group $G$ is the equivalence of growth functions associated with its the finite symmetric generating sets; denote it by $\beta_G$.
This is either an equivalence class or representative as necessary.

Say $G$ is of \keyword{polynomial growth} if $\beta_G \sim x^a$ for some $a \geq 0$.

Say $G$ is of \keyword{exponential growth} if $\beta_G \sim \exp$.

Say $G$ is of \keyword{intermediate growth} if it is neither polynomial nor exponential.

\begin{proposition}
    ...
    \begin{enumerate}[(1)]
        \item Let $f : [0, \infty) \to [0, \infty)$ be a generalized growth function with $f(0) > 0$.
        Then $f \sim 1$ (constant) if and only if $f$ is bounded.
        \item Let $G$ be a finitely generated group.
        Then $\beta_G \sim 1$ if and only if $G$ is finite.
        In particular, all finite groups have the same growth type---namely constant $1$.
        \item Let $G$ be a finitely generated group.
        Then $G$ is infinite if and only if $x \succeq \beta_G$.
        \item Every finitely generated group of sub-polynomial growth (in particular of polynomial growth) has sub-exponential growth.
    \end{enumerate}
\end{proposition}

\sepline

Let $P$ be a predicate of groups.

Say a group $G$ is \keyword{virtually} $P$ if there is a finite index subgroup $H \leq G$ which is $P$.

e.g., a group is virtually nilpotent if it contains a nilpotent subgroup of finite index.

e.g., a group is virtually solvable if it contains a solvable subgroup of finite index.

\sepline

\begin{theorem}
    Let $G$ be a finitely generated nilpotent group with nilpotency class $n \in \N$ (length of lower central series).
    Then $G$ has polynomial growth of degree
    \[
        \sum_{k=1}^{n-1} (k + 1) \cdot \rk_\Z(\gamma_k(G)/\gamma_{k+1}(G)).
    \]
    where $\rk_\Z A$ is torsion-free rank of abelian group $A$.
\end{theorem}

\begin{theorem}[Gromov?]
    A finitely generated group has polynomial growth if and only if it is virtually nilpotent.
\end{theorem}

\begin{corollary}
    Virtual nilpotency is QI invariant (geometric property?) for finitely generated groups.
\end{corollary}


\newpage
\sepline

Let $M$ be a smooth manifold of dimension $n$.
Denote its tangent bundle by $TM$.
Given a local parameterization $\phi : U \to M$, $U \seq \R^n$ open, induces vector fields $\partial_{x_1}, \dots, \partial_{x_n}$ and the map $U \times \R^n \to TM$, $(x, v) \mapsto v_1 \partial_{x_1} + \cdots + v_n \partial_{x_n}$, is a local parameterization of $TM$.
(local frame? whatever that is.)

A \keyword{field of distributions} on $M$ is a subset $\Delta$


\sepline

A \keyword{Lie algebra} $\gg$ is a vector space equipped with a bilinear operation $[-, -] : \gg \times \gg \to \gg$ satisfying
\begin{itemize}
    \item (anti-commutativity) $[X, Y] = -[Y, X]$ for all $X, Y \in \gg$
    \item (Jacobi identity) $[[X, Y], Z] + [[Y, Z], X] + [[Z, X], Y] = 0$
\end{itemize}


\end{document}