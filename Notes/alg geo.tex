\documentclass[12pt]{article}

% Packages
\usepackage[margin=1in]{geometry}
\usepackage{fancyhdr, parskip}
\usepackage{amsmath, amsthm, amssymb, mathrsfs, tikz-cd}

% Page Style
\fancypagestyle{plain}{
    \fancyhf{}
    \renewcommand{\headrulewidth}{0pt}
    \renewcommand{\footrulewidth}{0pt}
    \fancyfoot[R]{\thepage}
}
\pagestyle{plain}

% Problem Box
\setlength{\fboxsep}{4pt}
\newlength{\myparskip}
\setlength{\myparskip}{\parskip}
\newsavebox{\savefullbox}
\newenvironment{fullbox}{\begin{lrbox}{\savefullbox}\begin{minipage}{\dimexpr\textwidth-2\fboxsep\relax}\setlength{\parskip}{\myparskip}}{\end{minipage}\end{lrbox}\framebox[\textwidth]{\usebox{\savefullbox}}}
\newenvironment{pbox}[1][]{\begin{fullbox}\ifx#1\empty\else\paragraph{#1}\fi}{\end{fullbox}}

% Theorem Environments
\theoremstyle{definition}
\newtheorem{thm}{Theorem}
\newtheorem{lem}{Lemma}
\newtheorem{cor}{Corollary}
\newtheorem{prop}{Proposition}

\newtheorem{defn}{Definition}
\newtheorem{construction}{Construction}
\newtheorem{notation}{Notation}

\newenvironment{bthm}{\begin{fullbox}\begin{thm}}{\end{thm}\end{fullbox}}
\newenvironment{blem}{\begin{fullbox}\begin{lem}}{\end{lem}\end{fullbox}}
\newenvironment{bprop}{\begin{fullbox}\begin{lem}}{\end{lem}\end{fullbox}}
\newenvironment{bdefn}{\begin{fullbox}\begin{defn}}{\end{defn}\end{fullbox}}

% Options
%\allowdisplaybreaks
%\addtolength{\jot}{4pt}

% Default Commands
\newcommand{\isp}[1]{\qquad\text{#1}\qquad}
\newcommand{\N}{\mathbb{N}} 
\newcommand{\Z}{\mathbb{Z}}
\newcommand{\Q}{\mathbb{Q}}
\newcommand{\R}{\mathbb{R}}
\newcommand{\C}{\mathbb{C}}
\newcommand{\eps}{\varepsilon}
\renewcommand{\phi}{\varphi}
\renewcommand{\emptyset}{\varnothing}
\newcommand{\<}{\langle}
\renewcommand{\>}{\rangle}
\newcommand{\isom}{\cong}
\newcommand{\eqc}{\overline}
\newcommand{\clo}{\overline}
\newcommand{\sseq}{\subseteq}

% Extra Commands
\newcommand{\A}{\mathbb{A}}
\renewcommand{\P}{\mathbb{P}}
\renewcommand{\O}{\mathscr{O}}

\newcommand{\teq}{\trianglelefteq}
\newcommand{\inc}{\hookrightarrow}
\newcommand{\blow}{\widetilde}
\newcommand*{\rad}{\sqrt}

\newcommand{\Vp}{V_{\mathrm{p}}}
\newcommand{\Va}{V_{\mathrm{a}}}

\DeclareMathOperator{\codim}{codim}
\DeclareMathOperator{\Hom}{Hom}
\renewcommand{\Im}{\operatorname{Im}}
\DeclareMathOperator{\dom}{dom}
\DeclareMathOperator{\im}{im}
\DeclareMathOperator{\Spec}{Spec}

\newcommand*{\longlimit}[2]{
    {
        \def\arraystretch{0}
        \begin{array}[t]{@{}l@{}}
            \displaystyle{#2} \\[8.5pt] \scriptstyle{#1}
        \end{array}
    }
}

\newcommand*{\longprod}[2]{\longlimit{#1}{\prod #2}}
\newcommand*{\longsum}[2]{\longlimit{#1}{\sum #2}}
\newcommand*{\longcap}[2]{\longlimit{#1}{\bigcap #2}}

\DeclareMathOperator{\LP}{LP}

\newcommand{\KK}{\mathbf{K}}



\newcommand{\dto}[1][]{\overset{#1}{\longrightarrow}}
\newcommand{\mto}{\longmapsto}

%\newcommand{\defeq}{:=}

\newcommand*{\defeq}{\mathrel{\vcenter{\baselineskip0.7ex \lineskiplimit0pt \hbox{.}\hbox{.}}}=}


% Document Info
\fancypagestyle{title}{
    \renewcommand{\headrulewidth}{0.4pt}
    \setlength{\headheight}{15pt}
    \fancyhead[R]{Harry Coleman}
    \fancyhead[L]{MATH CS 120AG Notes}
    \fancyhead[C]{date}
}

% Begin Document
\begin{document}
\thispagestyle{title}

\section*{Notation}

Let $K$ denote an algebraically closed ground field.

Let $K[x_1, \dots, x_n]$ to be the $K$-alegbra of polynomials, graded by degree. We ill mostly focus on $K[x, y]$.

For $n \in \N$, we call $\A^n = \A_K^n$ the \textbf{affine $n$-space} over $K$.

For $S \subseteq K[x_1, \dots, x_n]$, call
\[
    V(S) = \{x \in \A^n : f(x) = 0 \text{ for all } f \in S\}
\]
the (affine) \textbf{zero locus} of $S$. Subsets of $\A^n$ of this form are called \textbf{affine varieties}.




\begin{defn}(Affine Curves)
    \begin{itemize}
        \item[(a)] An \textbf{(affine plane algebraic) curve} is a nonconstant polynomial $F \in K[x, y]$ modulo units, i.e., modulo the equivalence relation $F \sim G$ if $F = \lambda G$ for some $\lambda \in K^\times$. 

        Call $V(F) = \{P \in \A^2 : F(P) = 0\}$ the \textbf{set of points} of $F$.
        
        \item[(b)] The \textbf{degree} of a curve is its degree as a polynomial, denoted $\deg F$.
        
        \item[(c)] A curve $F$ is called \textbf{irreducible} if it is as a polynomial, and \textbf{reducible} otherwise. Similarly, if $F = F_1^{d_1} \cdots F_k^{d_k}$ is the irreducible decomposition of $F$ as a polynomial, we will also call this the \textbf{irreducible decomposition} of the curve $F$. The curves $F_1, \dots, F_k$ are called the \textbf{(irreducible) components} of $F$ and $d_1, \dots, d_k$ their multiplicities.
    \end{itemize}
    
\end{defn}


\begin{lem}
    Let $F$ be an affine curve.
    \begin{itemize}
        \item[(a)] If $K$ is algebraically closed then $V(F)$ is infinite.
        \item[(b)] If $K$ is infinite then $\A_K^2 \setminus V(F)$ is infinite.
    \end{itemize}
\end{lem}


\begin{prop}
    If two curves $F$ and $G$ have no common component then their intersection $V(F, G)$ is finite.
\end{prop}

\begin{cor}
    Let $F$ be a curve over an algebraically closed field. THe for any irreducible curve $G$ we have $G \mid F$ if and only if $V(G) \subseteq V(F)$.

    In particular, the irreducible components of $F$ (but not their multiplicities) can be recovered from $V(F)$.
\end{cor}


\begin{notation}
    Due to the above correspondence between a curve $F$ and its set of points $V(F)$, we will sometimes write
    \begin{itemize}
        \item[(a)] $P \in F$ instead of $P \in V(F)$, i.e., $F(P) = 0$;
        \item[(b)] $F \cap G$ instead of $V(F, G)$ for the points that lie on both $F$ and $G$;
        \item[(c)] $F \cup G$ for the curve $FG$;
        \item[(d)] $G \subseteq F$ instead of $G \mid F$.
    \end{itemize}
\end{notation}


\begin{defn}
    Let $a \in \A^2$ be a point.
    \begin{itemize}
        \item[(a)] The \textbf{local ring} of $\A^2$ at $P$ is defined as
        \[
            \O_a 
                = \O_{\A^2, a} 
                = \left\{\frac{g}{f} : f, g \in K[x, y] \text{ with } f(a) \ne 0 \right\}
                \subseteq K(x, y)
        \]
        \item[(b)] It admits a well-defined ring homomorphism
        \[
            \O_a \to K, \frac{g}{f} \mapsto \frac{g(a)}{f(a)}
        \] 
        which we call the \textbf{evaluation map}. Its kernel will be denoted by
        \[
            I_a = I_{\A^2, a} = \{\phi \in \O_a \mid \phi(a) = 0\}
        \]
        which is the unique maximal ideal in $\O_a$.
    \end{itemize}
\end{defn}


\begin{defn}
    For a point $a \in \A^2$ and two curves $F$ and $G$ we define the \textbf{intersection multiplicity} of $F$ and $G$ at $a$ to be
    \[
        \mu_a(F, G) = \dim \O_a / \<F, G\> \in \N \cup \{\infty\},
    \]
    where $\dim$ denotes the dimension as a vector space over $K$.
\end{defn}

\begin{lem}
    Let $a \in \A^2$ and let $F$ and $G$ be two curves. We have
    \begin{itemize}
        \item[(a)] $\mu_a(F, G) \geq 1$ if and only if $a \in F \cap G$;
        \item[(b)] $\mu_a(F, G) = 1$ if and only if $\<F, G\> = I_a$ in $\O_a$.
    \end{itemize}
\end{lem}

\begin{notation}
    For a polynomial $F \in K[x, y]$ of degree $d$ and $i = 0, \dots, d$, we define the \textbf{degree-$i$ part} of $F$ to be the sum of all terms of $F$ of degree $i$. Hence all $F_i$ are homogeneous, and we have $F = F_0 + \cdots + F_d$. We call $F_0$ the \textbf{constant} part, $F_1$ the \textbf{linear} part, and $F_d$ the \textbf{leading} part of $F$.
\end{notation}


\begin{prop}
    Let $F$ and $G$ be two curves through the origin. Then $\mu_0(F, G) = 1$ if and only if the linear parts of $F$ and $G$ are linearly independent.
\end{prop}


\section{projective}

\begin{defn}
    For $n \in \N$, we define the \textbf{projective $n$-space} over $K$ as the set of all $1$-dimensional linear subspaces of $K^{n+1}$. It is denoted by $\P_K^n$ or simply $\P^n$.

    In other words, we have
    \[
        \P^n = (K^{n+1} \setminus \{0\}) / \sim
    \]
    with the equivalence relation $x \sim y$ if and only if $x = \lambda y$ for some $\lambda \in K^\times$. We denote the equivalence class of $(x_0, \dots, x_n)$ by $[x_0 : \cdots : x_n] \in \P^n$. Call $x_0, \dots, x_n$ the \textbf{homogeneous} or \textbf{projective coordinate} of the point $[x_0 : \cdots : x_n]$.

    For a subset $S \subseteq K[x_0, \dots, x_n]$ of homogeneous polynomials we call
    \[
        V(S) = \{P \in \P^n : f(P) = 0 \text{ for all } f \in S\} \subseteq \P^n
    \]
    the projective \textbf{zero locus} of $S$. Subsets of $\P^n$ of this form are called \textbf{projective varieties}.
\end{defn}


\begin{defn}(Projective curves)
    A \textbf{(projective plane algebraic) curve} is a nonconstant homogeneous polynomial $F \in K[x, y, z]$ modulo units. We call $V(F) = \{P \in \P^2 : F(P) = 0\}$ is \textbf{set of points}.

    The \textbf{degree} of a projective curve is its degree as a polynomial.

    The notions of irreducible/reducible/reduced curves, as well as irreducible components and their multiplicities, are defined in the same way as for affine curves. 
\end{defn}


\begin{construction}
    For $P \in \P^2$ we define the \textbf{local ring} of $\P^2$ at $P$ as
    \[
        \O_P = \O_{\P^2, P} = \{\frac{g}{f} : f, g \in K[x, y, z] \text{ homogeneous of the same degree with } f(P) \ne 0\}
    \]
    and the unique maximal ideal
    \[
        I_P = I_{\P^2, P} = \{\phi \in \O_P : f(P) = 0\}.
    \]
    There is an isomorphism $\O_{\P^2, [x_0 : y_0 : 1]} \xrightarrow{\sim} \O_{\A^2, (x_0, y_0)}$ given by $\phi \mapsto \phi^{\mathrm{i}}$, which then restricts to $I_{\P^2, [x_0 : y_0 : 1]} \xrightarrow{\sim} I_{\A^2, (x_0, y_0)}$.
\end{construction}




\begin{construction}
    Note that the local ring $\O_{\P^2, P}$ does not contain $K[x, y, z]$ as a subring. But for $F_1, \dots, F_k$ homogeneous there is still a generated ideal
    \[
        \<F_1, \dots, F_k\> = \left\{\frac{a_1}{b_1}F_1 + \cdots + \frac{a_k}{b_k}F_k : a_i = 0 \text{ or $a_ib_i$ homogeneous with $\deg(a_iF_i) = \deg b_i$ for all $i$}\right\}
    \]
    in $\O_P$. As in the affine case we can therefore define \textbf{intersection multiplicity} of two curves $F$ and $G$ at a point $P \in \P^2$ as
    \[
        \mu_P(F, G) = \dim \O_P/\<F, G\>.
    \]
\end{construction}


\newpage



\begin{bdefn}
    Let $R$ be a ring. The set of all prime ideals of $R$ is called the \textbf{spectrum} of $R$ or the \textbf{affine scheme} associated to $R$. We denote it by $\Spec R$.
\end{bdefn}

\begin{bdefn}
    Let $R$ be a ring, and let $P \in \Spec R$ be a point in the corresponding affine scheme, i.e., a prime ideal $P \teq R$.

    We denote by $K(P)$ the quotient field (fraction field) of the integral domain $R/P$. It is called the \textbf{residue field} of $\Spec R$ at $P$.

    For any $f \in R$ we define the \textbf{value} of $f$ at $P$, written as $f(P)$, to be the image of $f$ under the composite ring homomorphism $R \to R/P \to K(P)$.

    In particular, we have $f(P) = 0$ if and only if $f \in P$.
\end{bdefn}

\begin{bdefn}
    Let $R$ be a ring.

    For a subset $S \sseq R$, we define the \textbf{zero locus} of $S$ to be the set
    \[
        V(S) 
            \defeq \{P \in \Spec R : f(P) = 0 \text{ for all } f \in S\}
            = \{P \in \Spec R : S \sseq P\}
            \quad \sseq \Spec R.
    \]

    For a subset $X \sseq \Spec R$, we define the \textbf{idea;} of $X$ to be
    \[
        I(X) 
            \defeq \{f \in R : f(P) = 0 \text{ for all } P \in X\}
            = \longcap{P \in X}{P}
            \quad\teq R.
    \]
\end{bdefn}

\begin{bdefn}
    We define the \textbf{Zariski topology} on an affine scheme $\Spec R$ to be the topology whose closed sets are exactly the sets of the form $V(S) = \{P \in \Spec R : S \subseteq P\}$ for some $S \subseteq R$.
\end{bdefn}

\begin{bprop}
    (Scheme Nullstellensatz) Let $R$ be a ring.

    For any closed subset $X \subseteq \Spec R$, we have $V(I(X)) = X$.

    For any ideal $J \teq R$, we have $I(V(J)) = \rad{J}$.

    In particular, $V(-)$ and $I(-)$ induce and inclusion-reversing bijection
    \[
        \{\text{closed subsets of $\Spec R$}\} \longleftrightarrow \{\text{radical ideals in $R$}\}.
    \]
\end{bprop}

Might be true that a closed $Y \subseteq \Spec R$ is irreducible iff $I(Y) \teq J$ is prime. So in particular,
\[
    \{\text{irreducible closed subsets of $\Spec R$}\} 
    \longleftrightarrow \{\text{prime ideals in $R$}\}
    = \Spec R,
\]
and, in which case, the topological dimension $\dim \Spec R$  equals the Krull dimension $\dim R$.

\begin{bdefn}
    For a ring $R$ and an element $f \in R$, we call
    \[
        D(f) 
            \defeq \Spec R \setminus V(f)
            = \{P \in \Spec R : f \notin P\}.
    \]
    the \textbf{distinguished open subset} of $f$ in $\Spec R$. $A \unlhd B$ $A \teq B$
\end{bdefn}

\begin{bdefn}
    Let $R$ be a ring, and let $U$ and an open subset of the affine scheme $\Spec R$. A \textbf{regular function} on $U$ is a family $\phi = (\phi_P)_{P \in U}$ with $\phi_P \in R_P$ for all $P \in U$, such that, for every $P \in U$, there is an open neighborhood $P \in U_P \subseteq U$ and $f_P, g_P \in R$, such that for all $Q \in U_P$, we have
    \[
        \phi_Q = \frac{g_P}{f_P} \quad\in R_Q.
    \]
    
    In particular, we require $f_P(Q) \ne 0 \iff f_P \notin Q$ for all $Q \in U_P$, i.e., $U_P \subseteq D(f_P)$.

    Stated obtusely:
    \[
        \forall P \in U,\;
        \exists U_P \in \mathcal{N}_{\Spec R}(P),\;
        \exists f, g \in R :\quad 
        \forall Q \in U_P,\;
        \phi_Q = \frac{g}{f} \;\in R_Q.
    \]

    The set of all such regular function on $U$ is clearly a ring, and we denote it by $\O_{\Spec R}(U)$. Moreover, the condition imposed on $\phi$ is local and it is obvious that $\O_{\Spec R}$ is a sheaf; it is called the \textbf{structure sheaf} of $\Spec R$.
\end{bdefn}

\begin{blem}
    Let $R$ be a ring. Then for any point $P \in \Spec R$ the stalk $\O_{\Spec R, P}$ of the structure sheaf $\O_{\Spec R}$ at $P$ is isomorphic to the localization $R_P$.
\end{blem}

\begin{bprop}
    Let $R$ be a ring and $f \in R$, then $\O_{\Spec R}(D(f)) \isom R_f$ as rings.

    In particular, $\O_{\Spec R}(\Spec R) \isom R$.
\end{bprop}

\begin{bdefn}
    A \textbf{locally ringed space} is a ringed space $(X, \O_X)$ such that each stalk $\O_{X, P}$ for $P \in X$ is a local ring.

    A \textbf{morphism} of locally ringed spaces from $(X, \O_X)$ to $(Y, \O_Y)$ is given by the following data:
    \begin{itemize}
        \item a continuous map $f : X \to Y$,
        \item for every open subset $U \subseteq Y$ a ring homomorphism $f_{U}^* : \O_Y(U) \to \O_{X}(f^{-1}(U))$ called the pullback on $U$,
    \end{itemize}
    such that the following two conditions hold:
    \begin{itemize}
        \item The pullback maps are compatible with restrictions, i.e., $f_U^*(\phi_U) = (f_V^* \phi)|_{f^{-1}(U)}$ for all $U \subseteq V \subseteq Y$ and $\phi \in \O_Y(V)$. In particular, this implies that there are induced ring homomorphisms $f_P^* : \O_{Y, f(P)} \to \O_{X, P}$ on the stalks for all $P \in X$.
        \item For all $P \in X$, we have $(f_P^*)^{-1}(I_P) = I_{f(P)}$, where $I_P$ and $I_{f(P)}$ denote the maximal ideals in the local rings $\O_{X, P}$ and $\O_{Y, f(P)}$, respectively.
    \end{itemize}
\end{bdefn}

\begin{bprop}
    For any two rings $R$ and $S$ there is a bijection
    \[
        \{\text{morphism } \Spec R \to \Spec S\} \longleftrightarrow \{\text{ring homomorphism } S \to R\}
    \]
    In particular, this means that there is a natural bijection
    \[
        \{\text{affine schemes}\}/\text{isomorphisms} \longleftrightarrow \{\text{rings}\}/\text{isomorphisms}
    \]
\end{bprop}


\end{document}