\documentclass[12pt]{article}

% Packages
\usepackage[margin=1in]{geometry}
\usepackage{parskip}
\usepackage{amsmath, amsthm, amssymb}
\usepackage{mathrsfs}
\usepackage{tikz, tikz-cd}
\usepackage[shortlabels]{enumitem}

\usepackage{suffix}
\usetikzlibrary{decorations.pathmorphing}

% Problem Box
\setlength{\fboxsep}{4pt}
\newlength{\myparskip}
\setlength{\myparskip}{\parskip}
\newsavebox{\savefullbox}
\newenvironment{fullbox}{\begin{lrbox}{\savefullbox}\begin{minipage}{\dimexpr\textwidth-2\fboxsep\relax}\setlength{\parskip}{\myparskip}}{\end{minipage}\end{lrbox}\framebox[\textwidth]{\usebox{\savefullbox}}}

% Environments
\setlist[enumerate]{nosep}
\newcommand{\keyword}[1]{\textbf{#1}}
\newcommand{\sepline}{\rule{\textwidth}{0.4pt}}

% Tikz Environments
\newenvironment{drawing}{\begin{center}\begin{tikzpicture}}{\end{tikzpicture}\end{center}}
% \tikzcdset{row sep/normal=0pt}
\newenvironment{cd}{\begin{center}\begin{tikzcd}}{\end{tikzcd}\end{center}}


% Document Formatting
\newtheoremstyle{mythmstyle}% name of the style to be used
  { }% measure of space to leave above the theorem. E.g.: 3pt
  { }% measure of space to leave below the theorem. E.g.: 3pt
  { }% name of font to use in the body of the theorem
  { }% measure of space to indent
  {\scshape}% name of head font
  {.}% punctuation between head and body
  { }% space after theorem head; " " = normal interword space
  {\thmname{#1}\thmnumber{ #2}\thmnote{ (#3)}}% Manually specify head

\theoremstyle{definition}
\newtheorem{theorem}{Theorem}
\newtheorem{corollary}{Corollary}
\newtheorem{lemma}{Lemma}
\newtheorem{proposition}{Proposition}


% Math Formatting
\newcommand{\isp}[1]{\quad\text{#1}\quad}

% mathbb
\newcommand{\N}{\mathbb{N}}
\newcommand{\Z}{\mathbb{Z}}
\newcommand{\Q}{\mathbb{Q}}
\newcommand{\R}{\mathbb{R}}
\newcommand{\C}{\mathbb{C}}

\renewcommand{\P}{\mathbb{P}}
\newcommand{\E}{\mathbb{E}}
\newcommand{\D}{\mathbb{D}}


% mathcal
\renewcommand{\AA}{\mathcal{A}}
\newcommand{\BB}{\mathcal{B}}
\newcommand{\CC}{\mathcal{C}}
\newcommand{\DD}{\mathcal{D}}
\newcommand{\EE}{\mathcal{E}}
\newcommand{\OO}{\mathcal{O}}

\newcommand{\FF}{\mathcal{F}}
\newcommand{\GG}{\mathcal{G}}
\newcommand{\HH}{\mathcal{H}}

\renewcommand{\SS}{\mathcal{S}}

% Symbols

\newcommand{\eps}{\varepsilon}
\renewcommand{\phi}{\varphi}
\renewcommand{\emptyset}{\varnothing}

% Delimiters
\newcommand{\<}{\left\langle}
\renewcommand{\>}{\right\rangle}

% Relations
\newcommand{\iso}{\cong}
\newcommand{\htpy}{\simeq}
\newcommand{\seq}{\subseteq}
\newcommand{\teq}{\trianglelefteq}
\newcommand{\tensor}{\otimes}

\newcommand{\inc}{\hookrightarrow}
\newcommand{\surj}{\twoheadrightarrow}
\newcommand{\To}{\longrightarrow}
\newcommand{\tto}{\rightrightarrows}
\newcommand{\Mapsto}{\longmapsto}

\newcommand{\nato}{\Rightarrow}
\newcommand{\Nato}{\Longrightarrow}


\newcommand{\eqby}[1]{\overset{\mathrm{(#1)}}{=}}

% Math Operators
\DeclareMathOperator{\Ob}{Ob}
\DeclareMathOperator{\Mor}{Mor}
\DeclareMathOperator{\Hom}{Hom}
\DeclareMathOperator{\Iso}{Iso}
\DeclareMathOperator{\End}{End}
\DeclareMathOperator{\Aut}{Aut}
\DeclareMathOperator{\supp}{supp}

\DeclareMathOperator{\RHom}{RHom}

\DeclareMathOperator{\dom}{dom}
\DeclareMathOperator{\cod}{cod}
\newcommand{\op}{\mathrm{op}}


\DeclareMathOperator{\id}{id}
\DeclareMathOperator{\im}{im}
\DeclareMathOperator{\Tor}{Tor}
\DeclareMathOperator{\Ext}{Ext}
\DeclareMathOperator{\Ann}{Ann}
\DeclareMathOperator{\coker}{coker}
\DeclareMathOperator{\Cyl}{Cyl}




\newcommand{\dd}{\mathrm{d}}

% Other
\newcommand{\eqc}{\overline}
\newcommand{\clo}{\overline}
\newcommand{\udl}{\underline}

\renewcommand{\hat}{\widehat}
\renewcommand{\bar}{\overline}

\newcommand{\mat}[1]{\begin{bmatrix}#1\end{bmatrix}}

% Category Names
\newcommand{\mathcat}{\mathsf}
\newcommand{\newcat}[2]{\newcommand{#1}{\mathcat{#2}}}
\WithSuffix\newcommand\newcat*[2]{\WithSuffix\newcommand#1*{\mathcat{#2}}}


\newcat{\Set}{Set}

\newcat{\Top}{Top}
\newcat{\Htpy}{Htpy}

\newcat{\Mon}{Mon}
\newcat{\CMon}{CMon}
\newcat{\Grp}{Grp}
\newcat{\Ab}{Ab}

\newcat{\Ring}{Ring}
\newcat{\CRing}{CRing}

\newcat{\Mod}{\text{-}Mod}
\newcat*{\Mod}{Mod}

\newcat{\lMod}{\text{-}Mod}
\newcat{\rMod}{Mod\text{-}}

\newcat{\lmod}{\text{-}mod}
\newcat{\rmod}{mod\text{-}}


\newcat{\Vect}{\text{-}Vect}
\newcat*{\Vect}{Vect}
\newcat{\Comp}{\text{-}Comp}
\newcat*{\Comp}{Comp}

\newcat{\Cat}{Cat}
\newcat{\CAT}{CAT}

\newcat{\SAb}{SAb}
\newcat{\PAb}{PAb}

\newcat{\Kom}{Kom}

\renewcommand{\_}[1]{{_{#1}}}



\title{Homological Algebra \\
    \large 
}
\author{}
\date{}


\begin{document}

Stuff for MATH 236A Homological Algebra 

\sepline

Example: Homology functors

\begin{cd}[row sep=tiny, column sep=tiny]
    R\Comp \ar[rr, "{H_n}"] && \Z\Mod \\
    A_\bullet \ar[dd, "f"'] & \Mapsto & \ker d_n / \im d_{n+1} \ar[dd, "an + \im d_{n+1} \mapsto f_n(a_n) + \im d_{n+1}'"] \\
    & \Mapsto \\
    A_\bullet' & \Mapsto & \ker d_n' / \im d_{n+1}'
\end{cd}

\sepline

A \keyword{pre-additive category} is a category $\CC$ such that for all $A, B \in \Ob\CC$ the hom-set $\Hom_\CC(A, B)$ is an (additive) abelian group such that
\[
    h \circ (f + g) = hf + hg
    \isp{and}
    (f + g) \circ k = fk + gk
\]
for all suitable $f, g, h, k \in \Hom\CC$.

\sepline

Let $\CC$, $\DD$ be pre-additive categories.

A functor $F : \CC \to \DD$ is called \keyword{additive} if for all $A, B \in \Ob\CC$ the map
\[
    F : \Hom_\CC(A, B) \To \Hom_\DD(FA, FB)
\]
is a homomorphism of abelian groups.

\sepline

Let $F: R\Mod \to S\Mod$ be an additive functor.
Then $F$ induces a an additive functor
\begin{align*}
    \tilde{F} : R\Comp &\To S\Comp, \\
        A_\bullet &\Mapsto FA_\bullet.
\end{align*}

For $F : R\Mod \to S\Mod$, we care about e.g. $H_n \circ \tilde{F} : A_\bullet \mapsto H_n(FA_\bullet)$.

\sepline

Important Example I: hom-functor

\begin{align*}
    \Hom_R(\_RM_S, -) &: R\Mod \To S\Mod, \\
    \Hom_R(-, \_RN_T) &: R\Mod \To \Mod*\text{-}T.
\end{align*}

In general, $\Hom_R(\_RM_S, \_RN_T)$ is an $S$-$T$-bimodule.

\sepline

Important Example II: tensor-functor

Let $M \in \Mod*{-}R$, $N \in R\Mod$.

A \keyword{tensor product} of $M$ and $N$ consists of 
\begin{itemize}
    \item an abelian group $T$,
    \item a $\Z$-bilinear and $R$-balanced map $\tau : M \times N \to T$,
\end{itemize}
satisfying the following universal property:

Whenever $\sigma : M \times N \to A$ is $\Z$-bilinear and $R$-balanced ($A$ abelian group), there exists a unique $\Z$-linear map $\sigma' : T \to A$ such that the following diagram commutes:
\begin{cd}
    M \times N
        \rar["\sigma"]
        \dar["\tau"']
    & A \\
    T 
        \urar[dashed, "\exists!\sigma'"']
\end{cd}

Write $M \tensor_R N := T$ for a (the) tensor product.

\sepline

\begin{theorem}
    Let $M \in \Mod*{-}R$ and $N \in R\Mod$.
    There exists a unique up to isomorphism tensor product $M \tensor_R N$.
\end{theorem} 

\begin{proof}
    Let $F$ be the free abelian group on basis $M \times N$, i.e., $F = \bigoplus_{m,n} \Z_{m,n}$.

    Define $M \tensor_R N = F/U$ where $U$ is the submodule generated by all elements of the form
    \begin{align*}
        & (m_1 + m_2, n) - (m_1, n) - (m_2, n), \\
        & (m, n_1 + n_2) - (m, n_1) - (m, n_2), \\
        & (mr, n) - (m, rn).
    \end{align*}

    Define $\tau : M \times N \to M \tensor_R N$ by $(m, n) \mapsto m \tensor n := (m, n) + U$.
    Then $\tau$ is $\Z$-bilinear and $R$-balanced.

    Claim $M \tensor_R N$ with $\tau$ satisfied the universal property.

    Given $\sigma : M \times N \to A$ an $\Z$-bilinear, $R$-balanced map to abelian group $A$.
    Define $\tilde{\sigma} : F \to A$ by $(m, n) \mapsto \sigma(m, n)$ and extend bilinearly.

    By definition of $U$, $\tilde{\sigma}(U) = 0$.

    Hence, there exists a $\Z$-linear map $\sigma' : F/U \to A$ with the property that
    \[
        \sigma'(m, n) 
            = \tilde{\sigma}((m, n) + U)
            = \tilde{\sigma}(m \tensor n).
    \]

    Exercise: show $\sigma'$ is unique.
\end{proof}


\sepline

1/23/23 (missed)

define natural transformation, equivalence of categories, morita equivalence of rings, adjoint functors.

\sepline

1/25/23

category theory is ``the mathematician's filing cabinet.''
good tool for transporting information from one area of math to another area, when one has equivalence of categories or sufficiently good functors.


\sepline

Let $F : \CC \to \DD$ and $G : \DD \to \CC$ be functors.
We say that $F$ and $G$ are \keyword{adjoint}, written $F \dashv G$ if the bifunctors $\CC \times \DD \to \Set$ are naturally isomorphic:
\[
    \Mor_\DD(F(-), -) \iso \Mor_\CC(-, G(-)).
\]
That is, the following diagram commutes:
\begin{cd}[column sep=huge]
    \CC \times \DD
        \rar[bend left=20, "{\Mor_\DD(F(-), -)}" name=U]
        \rar[bend right=20, "{\Mor_\CC(-, G(-))}"' name=L]
        \ar[from=U, to=L, shorten=1ex, Rightarrow]
    & \Set
\end{cd}

\sepline

Examples

1.\begin{enumerate}[(a)]
    \item $R \tensor_R - \iso \id_{R\Mod}$.

    $R \tensor_R - : R\Mod \to R\Mod$ well-defined since $\_RR_R$ isa bimodule.
    \item $\Hom_{R\lMod}(\_RR_R, -) \iso \id_{R\lMod}$
\end{enumerate}


2. $R$ morita equivalent to $M_n(R)$ for $n \in \N$.

Why? Let $F_R = R_R^n$ and $S = \End_R(F) \iso M_n(R)$ (ring iso).

Let $\_SF_R$ is a bimodule.

Consider $F^* = \_R\left(\Hom_R(\_SF_R, R)\right)_S$.

Claim: The functors $\Hom_R(F, -) : \rMod{R} \to \Mod*{-}S$ and $\Hom_S(F^*, -) : \Mod*{-}S \to \Mod*{-}R$ are mutually inverse functors.

Want to show that for $M \in \rMod{R}$, have iso $\id_{\Mod*{-}R} \iso \Hom_S(F^*, \Hom_R(F, -))$.
For $M \in \Mod*{-}R$, have
\begin{align*}
    \Phi(M) : M &\To \Hom_S(F^*, \Hom_R(F, M)) \\
        m &\Mapsto \left(f \mapsto (x \mapsto mf(x))\right)
\end{align*}

Check $\Phi(M)$ is an $R$-module homomorphism and in fact is an isomorphism of $R$-modules.

3. Let $R = K$ bea field.
Then we have a duality $(-)^* : K\lmod \to K\lmod$, $V \mapsto V^*$ (dual vector space).
Then there is a natural isomorphism $\Phi : \id_{K\lmod} \Rightarrow (-)^{**}$
\begin{align*}
    \Phi(V) : V &\To V^{**} = \Hom_K(\Hom_K(V, K), K), \\
        x &\Mapsto \operatorname{eval}_x = (f \mapsto f(x)).
\end{align*}

Then $(\Phi(V))_{V \in K\lmod}$ is a duality $K\lmod \to K\lmod$.

We may extend $\Phi$ to a functor $K\lMod \to K\lMod$ contravariant.

$\Phi(V) : V \to V^{**}$ is not surjective if $\dim V = \infty$ (hw problem).

\sepline

Example of adjoint pair.

slogan: ``tensor functors are left-adjoint to hom-functors''

Let $\_SB_R$ be an $S$-$R$-bimodule.

Then the functors 
\[
    B\tensor_R - : R \To R\lMod
\]
is left-adjoint to
\[
    \Hom_S(B, -) : S\lMod \To R\lMod.
 \]

 \sepline

1/27/2023

An \keyword{additive category} is a pre-additive category $\CC$ together with the following additional data  
\begin{itemize}
    \item a \keyword{zero object} $0 \in \CC$
\end{itemize}
such that
\begin{itemize}
    \item for every $A \in \CC$ the hom-sets $\Hom(0, A)$ and $\Hom(A, 0)$ are singletons;
    \item $\CC$ has all finite direct sums and finite direct products;
\end{itemize}

Additionally, say $\CC$ is \keyword{abelian} if 
\begin{itemize}
    \item every morphism in $\CC$ has a kernel and cokernel;
    \item every monomorphism is a kernel and every epimorphism is a cokernel.
\end{itemize}

\sepline

Example

In $\Ring$, there are categorical epimorphisms which fail to be surjective, e.g., $f : \Z \inc \Q$.

Let $g, h \in \Hom_\Ring(\Q, R)$ be such that $gf = hf$, i.e., $g|_\Z = h|_\Z$.
It follows that $g = h$.
For $b \in \Z \setminus \{0\}$, 
\[
    1_R
        = g(1_\Q)
        = g(b \cdot 1/b)
        = g(b) g(1/b)
        = g(1/b)g(b).
\]
Hence, $g(b)$ is a unit in $R$ and $g(1/b) = g(b)^{-1}$.

\sepline

In $R\lMod$, $R\Comp$, monomorphisms/epimorphisms coincide with injective/surjective homomorphisms.

\sepline

Example

abelian: $R\lMod$, in particular $\Z\lMod = \Ab$.

$R\Comp$, let $\operatorname{\mathscr{T}-\Ab}$ be the full subcategory of $\Ab$ consisting of the torsion abelian groups ($A \in \Ab$ such that for all $a \in A$ there exists $n \in \Z \setminus \{0\}$ such that $na = 0$).

What about the full subcategory $\operatorname{\mathscr{F}-\Ab}$ having objects as torsion-free abelian groups? No.

Further non-abelian: $R\lmod$ is not abelian if $R$ is left-noetherian

e.g., $R = K^\N$ with $K$ a field.
$I = K^{(\N)}$ is not finitely generated as a left ideal, and canonical quotient map $\pi : R \surj R/I$ in $R\lmod$ does not have a kernel.

Note: $K^\N = \prod_{i\in\N} K_i$ with $K_i = K$ and $K^{(\N)} = \{(k_i) \in K^\N \text{ with finite support}\}$.

\sepline

Remark.

\begin{cd}[row sep=0]
    R\lMod \rar[rightsquigarrow] 
        & R\Comp \rar[rightsquigarrow] 
        & H(R\Comp) \rar[rightsquigarrow]
        & D(R\lMod) \\
    \text{ab} 
        & \text{ab}
        & \text{not ab}
        & \text{not ab} \\
    & & \text{in general}
        & \text{derived category}
\end{cd}

\sepline

On the road to derived functors.

Note: we will develop the theory for the abelian category $R\lMod$, but it easily adapts to arbitrary categories.

\sepline

Let $R$ and $S$ be rings, $F : R\lMod \to S\lMod$ an additive functor (covariant or contravariant).

Say covariant $F$ is \keyword{exact} if for all short exact sequences
\begin{cd}
    0 \rar & A \rar["f"] & B \rar["g"] & C \rar & 0
\end{cd}
in $R\lMod$, the image sequence
\begin{cd}
    0 \rar & FA \rar["Ff"] & FB \rar["Fg"] & FC \rar & 0
\end{cd}
is exact in $S\lMod$.

(In the case that $F$ is contravariant, the definition is the same with respect to the with appropriate image sequence.)

Say covariant $F$ is left-exact if for every exact sequence 
\begin{cd}
    0 \rar & A \rar["f"] & B \rar["g"] & C
\end{cd}
in $R\lMod$, (equivalent if add $C \to 0$) the image sequence
\begin{cd}
    0 \rar & FA \rar["Ff"] & FB \rar["Fg"] & FC
\end{cd}
is exact in $S\lMod$.


\sepline

2/1/23

\sepline

2/3/23


Recall: properties of groups
\begin{itemize}
    \item torsion and torsion-free,
    \item divisible and reduced (no nonzero divisible subgroup).
\end{itemize}
Then
\begin{itemize}
    \item $\Hom_\Z(\text{torsion}, \text{torsion-free}) = 0$,
    \item $\Hom_\Z(\text{divisible}, \text{reduced}) = 0$.
\end{itemize}

Examples showing that hom-functors fail to be right exact and tensor-functors fail to be left exact, in general.

1. $\Hom_\Z(\Z/n\Z, -)$ is not right exact, $n \geq 2$.
Apply the canonical map $\Z \to \Z/n\Z$, get
\begin{cd}
    \Hom_\Z(\Z/n\Z, \Z) 
        \rar 
    & \Hom_\Z(\Z/n\Z, \Z/n\Z)
\end{cd}

2. $F = \Hom_\Z(\Q, -)$ is not right exact.

Take $\_RM \in \R\lmod$, then
\begin{align*}
    \_RR^{(M)} &\To \_RM \\
    1_R &\Mapsto m
\end{align*}

Consider an epimorphism $g : \Z^{(\N)} \to \Q$.
Apply $F$  to get
\begin{cd}
    \Hom_\Z(\Q, \Z^{(\N)}) 
        \rar
    & \Hom_\Z(\Q, \Q).
\end{cd}
$\Q$ is divisible, $\Z^{\N}$ is reduced so left is zero, but right is nonzero.

3. $R = \Z$, $F = \Z/n\Z \tensor_\Z -$, $n \geq 2$.
Apply $F$ to embedding $\Z \inc \Q$ to get
\begin{cd}
    \Z/n\Z \tensor_\Z \Z
        \rar
    & \Z/n\Z \tensor_\Z \Q.
\end{cd}
Left is $\Z/n\Z$ but right is zero.

mnemonic $\text{torsion} \tensor_\Z \text{divisible} = 0$.

\sepline

Short-term program:
\begin{enumerate}
    \item find exact functors
    \item characterize exact sequences $A_\bullet \in R\lMod$ such that $F(A_\bullet) \in S\lMod$ is exact for any additive functor $F : R\lMod \to S\lMod$.
\end{enumerate}

\sepline

\begin{theorem}[4]
    If $F : R\lMod \to S\lMod$ is an exact functor (sends SES's to SES's), then $F(A_\bullet)$ is exact in $S\lMod$ whenever $A_\bullet$ is exact in $R\lMod$.
\end{theorem}

\begin{proof}
    Let $F : R\lMod \to S\lMod$ be exact and $A_\bullet$ an exact sequence in $R\lMod$.

    Then $F$ sends $A_\bullet$ to complex $F(A_\bullet)$ in $S\lMod$.

    Factor each $f_n : A_n \to A_{n-1}$ as 
    \begin{cd}
        A_n \rar["\tilde{f}_n"]
        & \im f_n \rar[hook, "\iota_n"]
        & A_{n-1}
    \end{cd}

    Consider the short exact sequences for $n \in \Z$
    \begin{cd}
        0 \rar
        & \im\iota_{n+1} \rar[hook, "\iota_{n+1}"]
        & A_n \rar["\tilde{f}_n"]
        & \im\tilde{f}_n = \im f_n \rar
        & 0.
    \end{cd}
    If $F$ is exact, we obtain a short exact sequence
    \begin{cd}
        0 \rar
        & F(\im\iota_{n+1}) \rar[hook, "F(\iota_{n+1})"]
        & F(A_n) \rar["F(\tilde{f}_n)"]
        & F(\im\tilde{f}_n) \rar
        & 0.
    \end{cd}

    In particular,
    \[
        \im F(\iota_{n+1}) = \ker F(\tilde{f}_n)
    \]
    for $n \in \N$.

    
    Claim.
    \begin{enumerate}[(a)]
        \item $\ker F(f_n) = \ker F(\tilde{f}_n)$,
        \item $\im F(f_{n+1}) = \im F(\iota_{n+1})$.
    \end{enumerate}

    Proof.
    \begin{enumerate}[(a)]
        \item $f_n = \iota_n \circ \tilde{f}_n$, so $F(f_n) = F(\iota_n) \circ F(f_n)$.

        Since $F$ is exact, $F(\iota_n)$ is a monomorphism.

        Hence, $\ker F(f_n) = \ker F(\tilde{f}_n)$.

        \item $F(f_{n+1}) = F(\iota_{n+1}) \circ F(\tilde{f}_{n+1})$.
        
        Since $F$ is exact, $F(\tilde{f}_{n+1})$ is an epimorphism.

        Hence, $\im F(f_{n+1}) = \im F(\iota_{n+1})$
    \end{enumerate}

    Combining all the stuff completes the proof.
\end{proof}

\sepline

First installment of program part 1.

\begin{proposition}[5]
    Let $I$ be any indexing set and consider the functors
    \begin{align*}
        (R\lMod)^I &\To R\lMod 
            & (R\lMod)^{(I)} &\To R\lMod \\
        (M_i)_{i \in I} &\Mapsto \prod_{i \in I} M_i 
            & (M_i)_{i \in I} &\Mapsto \bigoplus_{i \in I} M_i \\
        (f_i)_{i \in I} &\Mapsto \prod_{i \in I} f_i
            & (f_i)_{i \in I} &\Mapsto \bigoplus_{i \in I} f_i.
    \end{align*}
    Both functors are exact.
\end{proposition}

\begin{proof}
    obvious
\end{proof}

\sepline

First installment of program part 2.

Let $A, B \in R\lMod$ and $f \in \Hom_R(A, B)$.

Say $f$ is \keyword{split} if $\ker f$ is a direct summand of $A$ and $\im f$ is a direct summand of $B$, i.e., there exist $A',B' \in R\lMod$ with $A = \ker f \oplus A'$ and $B = \im f \oplus B'$.

In Birge notation, $f$ split if $\ker f \seq^\oplus A$ and $\im f \seq^\oplus B$.

\sepline

2/6/23

\sepline

2/8/23

Q: Is the following true or false?
$P \in R\lMod$ finitely generated projective, then there exists $n \in \N$ with $P$ isomorphic to a direct summand of $\_RR^n$. 

True.

Up to isomorphism, $P \seq^\oplus \_RR^{(I)} = \bigoplus_{i \in I} R_i$ with $R_i = R$.

With $P = Rx_1 + \cdots + Rx_m$.

Pick finite $I' \seq I$ with $x_k \in \bigoplus_{i \in I} R_i$ for all $k \leq m$.

Then $P \seq \bigoplus_{i \in I'} R_i \iso R^n$ with $n = |I'|$ and hence $P \seq^\oplus R^n$.

In general, $\_RV \seq \_U \seq \_RM$, if $V \seq^\oplus M$ then $V \seq^\oplus U$.
Look up the ``modular law.''

\sepline

\begin{theorem}[9]
    For $P \in R\lMod$, the following are equivalent
    \begin{enumerate}[(1)]
        \item $\_RP$ is projective (direct summand of free left $R$-module);
        \item $\Hom_R(P, -) : R\lMod \to \Z\lMod$ is exact;
        \item whenever $f \in \Hom_R(B, C)$ is an epimorphism and $g \in \Hom_R(P, C)$, there exists $\phi \in \Hom_R(P, B)$ with $f \circ \phi = g$, i.e., each diagram of the following format
        \begin{cd}
            & P \dar["g"] \dlar[dashed, "\phi"'] \\
            B \rar["f"'] & C \rar & 0
        \end{cd}
        can be supplemented to a commutative triangle;
        \item every epimorphism onto $M$ is split.
    \end{enumerate}
\end{theorem}

Remark. Will write $[M, -] := \Hom_R(M, -)$.

\begin{proof}
    Follow sequence (1) $\implies$ (3) $\implies$ (2) $\implies$ (4) $\implies$ (1).

    (1) $\implies$ (3).

    have $M \leq^\oplus \_RF$ with $F$ free on basis $\{x_i\}_{i \in I}$, $F = M \oplus N$.
    Let $f$ and $g$ be as under (3)
    \begin{cd}
        B \rar["f"] & C \rar & 0 \\
        & M \uar["g"'] \\
        & F \uar["\pi"'] \ar[uul, dashed, "\psi"]
    \end{cd}
    where $\phi : F \to M$ is the projection along $N$ and $\iota : M \inc F$ to be the embedding.
    Define $\psi \in \Hom_R(F, B)$ via $\psi(x_i) = b_i$ if $f(b_i) = g\pi(x_i)$.
    (This is how it has to be.)
    This is well-defined because $F$ is free, so we can choose any images of the basis elements $x_i$; simply choosing any $b_i \in f^{-1}g\pi(x_i)$, which exist because $f$ is an epimorphism.

    Then $f \circ \psi = g \circ \pi$ and thus $g \circ (\psi \circ \iota) = g \circ \pi \circ \iota = g$.
    
    Hence, define $\phi = \psi \circ \iota$ is as required.
    \begin{cd}
        B \rar["f"] & C \rar & 0 \\
        & M \uar["g"'] \ular[dashed, "\phi"'] \\
        & F \uar["\pi"'] \ar[uul, dashed, "\psi"]
    \end{cd}

    (3) $\implies$ (2).

    Since $[M, -]$ is left exact, it suffices to show that $[M, -]$ takes epis to epis.

    Let $B \xrightarrow{f} C \to 0$ be exact.
    
    To see that $[M, f] : [M, B] \to [M, C]$ is an epi, let $g \in [M, C]$.

    By (3), there exists $\phi \in [M, B]$ with $[M, f](\phi) = f \circ \phi = g$.

    (2) $\implies$ (4).

    Let $f : N \to M$ be an epimorphism.
    
    Then by (2), the map $[M, f] : [M, N] \to [M, M]$ is an epimorphism.

    In particular, $\id_M \in \im [M, f]$---there exists $\phi \in [M, N]$ with $f \circ \phi = \id_M$, i.e.,
    \begin{cd}
        M \rar["\phi"] & N \dlar["f"] \\
        M \uar[equal, "\iso"]
    \end{cd}
    Hence, $N = \im \phi \oplus \ker f$, so $\ker f \leq^\oplus N$ and indeed $f$ splits.

    (4) $\implies$ (1).

    If $\{m_i\}_{i \in I}$ is a generating set for $M$, then $F = R^{(I)} \to M$ sending $(r_i) \mapsto \sum_{\substack{i \in I \\ \text{finite}}} r_i m_i$ is an epimorphism.
    By (4), $f$ splits, i.e., $F = \ker f \oplus N$.
    Thus $N \iso F / \ker f \iso M$, i.e., (1) holds.
\end{proof}

\sepline

Examples of projective modules and structure results.

1. $R$ a ring.
All $M \in \R\lMod$ are projective if and only if every submodule $U$ of any left $R$ module $N$ is a direct summand $N = U \oplus V$. 

(Every $R\lMod$ is projective iff every $R\lMod$ is semisimple.)

\sepline

2/10/23

\sepline

2/13/23

$M \in R\lMod$.

Recall. A projective resolution of $M$ is any exact sequence
\begin{cd}
    \cdots \rar & P_2 \rar["f_2"] & P_1 \rar["f_1"] & P_0 \rar["f_0"] & M \rar & 0
\end{cd}
with all $P_i$ projective.

\sepline

Example

$R = \Z/p^n\Z$, $p$ prime, $n \geq$.

Write $\eqc{x} = x + p^n\Z$ and consider $M = R \eqc{p}^{n-1}$.

A projective resolution of $M$ is
\begin{cd}[column sep=small]
    \cdots \rar
    & R \ar[rr, "f_2"] \drar["r \mapsto r\eqc{p}^{n-1}" description]
    && R \ar[rr, "f_1"] \drar["r \mapsto r\eqc{p}^n" description]
    && R \ar[rr, "f_0", "r \mapsto r\eqc{p}^{n-1}"']
    && M = R\eqc{p}^{n-1} \rar
    & 0 \\
    && \ker f_1 = R\eqc{p}^{n-1} \urar
    && \ker f_0 = R\eqc{p} \urar
\end{cd}

All kernels fail to be free.

$R$ is a local ring: $R \geq R\eqc{p} \geq R\eqc{p}^2 \geq \cdots \geq R\eqc{p}^n = 0$.

These are all ideals, so $R\eqc{p}$ is the only maximal ideal.
Since projectives over a local ring are free, the modules $R\eqc{p}$ and $R\eqc{p}^{n-1}$ are not projective.

\sepline

Let $M \in R\lMod$ and
\begin{cd}
    \cdots \rar & P_2 \rar["f_2"] & P_1 \rar["f_1"] & P_0 \rar["f_0"] & M \rar & 0
\end{cd}
a projective resolution.
Then $\im f_n = \ker f_{n-1}$ is called an \keyword{$n$th syzygy} of $M$.

(``syzygy'' from astronomy term ``syn-zygon'' (yoked together).)

Remark. syzygies are not unique up to isomorphism.

e.g., $R\eqc{p}$ is a first syzygy of $M = R\eqc{p}^{n-1}$.

But consider
\begin{cd}[row sep=0]
    R \oplus R \rar["g_0"] & M \rar & 0 \\
    (x, y) \rar[mapsto] & x\eqc{p}^{n-1}
\end{cd}
Then $\ker g_0 = R\eqc{p} \oplus R$, also a first syzygy of $M$.

\sepline

\begin{lemma}[15 Schanuel]
    Let $f : P \to M$, $g : Q \to M$ be epimorphisms in $R\lMod$ with $P$ and $Q$ projective.
    Then $\ker f \oplus Q \iso \ker g \oplus P$.

    ``First syzygies of a module are unique up to projective summands.''
\end{lemma}

\begin{proof}
    Consider the following commutative diagram:
    \begin{cd}
        & & & 0 \dar \\
        & & & \ker g \dar[hook] \\
        & & & Q \dar["g"] \\
        0 \rar 
            & \ker f \rar[hook] 
            & P \rar["f"'] 
            & M \rar \dar
            & 0 \\
        & & & 0
    \end{cd}
    Let $L = P \times_M Q$ be the pullback of $f : P \to M$ and $g : Q \to M$ with projections $\pi_P : L \to P$ and $\pi_Q : L \to Q$:
    \begin{cd}
        & & & 0 \dar \\
        & & & \ker g \dar[hook] \\
        & & L \dar["\pi_P"'] \rar["\pi_Q"]
            & Q \dar["g"] \\
        0 \rar 
            & \ker f \rar[hook] 
            & P \rar["f"'] 
            & M \rar \dar
            & 0 \\
        & & & 0
    \end{cd}
    By the universal property of the pullback, we have unique maps $i$ and $j$ satisfying
    \begin{center}
        \begin{tikzcd}
            \ker f 
                \ar[ddr, bend right=20, hook] 
                \ar[drr, bend left=20, "0"]
                \drar[dashed, "i"] \\
            & L \dar["\pi_P"'] \rar["\pi_Q"]
                & Q \dar["g"] \\
            & P \rar["f"'] & M 
        \end{tikzcd}
        \begin{tikzcd}
            \ker g 
                \ar[ddr, bend right=20, "0"'] 
                \ar[drr, bend left=20, hook]
                \drar[dashed, "j"] \\
            & L \dar["\pi_P"'] \rar["\pi_Q"]
                & Q \dar["g"] \\
            & P \rar["f"'] & M 
        \end{tikzcd}
    \end{center}
    Note that the perimeter of these diagrams commute because the composition of a map and the inclusion of its kernel is always zero.
    Combining these into our original diagram, we get
    \begin{cd}
        & & & 0 \dar \\
        & & \ker g \rar[equals] \dar["j"']
            & \ker g \dar[hook] \\
        & \ker f \dar[equals] \rar["i"]
            & L \dar["\pi_P"'] \rar["\pi_Q"]
            & Q \dar["g"] \\
        0 \rar 
            & \ker f \rar[hook] 
            & P \rar["f"'] 
            & M \rar \dar
            & 0 \\
        & & & 0
    \end{cd}
    Add zeros in places to get following diagram:
    \begin{cd}
        & & 0 \dar & 0 \dar \\
        & & \ker g \rar[equals] \dar["j"']
            & \ker g \dar[hook] \\
        0 \rar
            & \ker f \dar[equals] \rar["i"]
            & L \dar["\pi_P"'] \rar["\pi_Q"]
            & Q \dar["g"] \rar
            & 0\\
        0 \rar 
            & \ker f \rar[hook] 
            & P \rar["f"'] \dar
            & M \rar \dar
            & 0 \\
        & & 0 & 0
    \end{cd}
    Claim that center row and column are exact.

    Easy to check that $i$ and $j$ are monomorphisms.

    By construction, $\pi_Q \circ i = 0$ and $\pi_P \circ j = 0$.

    Say $\ell \in \ker \pi_Q$, then
    \[
        f(\pi_P(\ell)) = g(\pi_Q(\ell)) = g(0) = 0.
    \]
    Therefore, $\pi_P(\ell) \in \ker f$, but then $\ell = i(\pi_P(\ell)) \in \im i$.
    Hence, $\im i = \ker \pi_Q$.

    Similarly, $\im j = \ker \pi_P$.

    Given $y \in Q$, since $f$ is an epimorphism there exists $x \in P$ such that $f(x) = g(y)$.
    With a concrete construction of the pullback, it is clear that $(x, y)$ should be an element of $L$.
    In which case, $\pi_Q(x, y) = y$, so $\pi_Q$ is an epimorphism.

    Since $P$ and $Q$ are projective, both $\pi_P$ and $\pi_Q$ split, hence
    \[
        P \oplus \ker g \iso L \iso Q \oplus \ker f.
    \]
\end{proof}

\sepline

\begin{lemma}[16]
    Suppose $U_n$ and $V_n$ are $n$th syzygies of $M$ in $R\lMod$.
    Then there exists projective modules $X_n$, $Y_n$ with $U_n \oplus X_n \iso V_n \oplus Y_n$.

    ``All the $n$th syzygies are unique up to projective summands.''
\end{lemma}

\begin{proof}
    Schanuel + induction [GauchoSpace]
\end{proof}

\sepline

\begin{theorem}[17]
    \
    \begin{enumerate}[(1)]
        \item For $M \in R\lMod$, the following are equivalent:
        \begin{enumerate}[(a)]
            \item The projective dimension $\operatorname{pdim} M \leq n < \infty$;
            \item There exists a projective resolution
            \begin{cd} 
                0 \rar & P_n \rar["f_n"] & P_{n-1} \rar["f_{n-1}"] & \cdots \rar["f_0"] & M \rar & 0 
            \end{cd}
            such that $\im f_n = \ker f_{n-1}$ (an $n$th syzygy) is projective;
            \item In all projective resolutions of $M$, the $n$th syzygy is projective.
        \end{enumerate}
        \item For $M \in R\lMod$, the following are equivalent:
        \begin{enumerate}[(a)]
            \item $\operatorname{pdim} M = \infty$;
            \item There exists a non-projective resolution of $M$ with a projective $n$th syzygy;
            \item In all projective resolutions of $M$, the $n$th syzygy is non-projective.
        \end{enumerate}
    \end{enumerate}
\end{theorem}

\begin{proof}(1) 

    (a $\implies$ b)

    Suppose $\operatorname{pdim} M \leq n$, i.e., there is a projective resolution
    \begin{cd} 
        0 \rar & P_n \rar["f_n"] & P_{n-1} \rar["f_{n-1}"] & \cdots \rar["f_0"] & M \rar & 0 
    \end{cd}
    Then $\im f_n \iso P_n$ is projective.

    (b $\implies$ c)

    Suppose (b) holds, and let
    \begin{cd} 
        \cdots \rar & Q_2 \rar["g_2"] & Q_1 \rar["g_1"] & Q_0 \rar["g_0"] & M \rar & 0 
    \end{cd}
    be another projective resolution of $M$.
    By Lemma 16, $\im g_n = \ker g_{n-1}$ differs only by projective summands from $\im f_n = \ker f_{n-1}$.
    Therefore, since $\ker f_{n-1}$ is projective, so is $\ker g_{n-1}$.

    (c $\implies$ a)

    Assume (c) and let
    \begin{cd} 
        \cdots \rar & P_{n+1} \rar["f_{n+1}"] & P_n \rar["f_n"] & P_{n-1} \rar["f_{n-1}"] & \cdots \rar & P_0 \rar["f_0"] & M \rar & 0 
    \end{cd}
    be a projective resolution of $M$.
    Then $\im f_n = \ker f_{n-1}$ is projective by hypothesis.
    So consider the sequence
    \begin{cd}
        0 \rar & \ker f_{n-1} \rar[hookrightarrow] & P_{n-1} \rar["f_{n-1}"] & P_{n-2} \rar["f_{n-2}"] & \cdots \rar["f_0"] & M \rar & 0
    \end{cd}
    This is a projective resolution of length $\leq n$.
\end{proof}

\sepline

2/15/23

$M \in R\lMod$ is \keyword{injective} if and only if $\Hom_R(-, M) : R\lMod \to \Z\lMod$ is exact.

\begin{theorem}[18]
    For $E \in R\lMod$, the following are equivalent:
    \begin{enumerate}[(1)]
        \item $\_RE$ is injective;
        \item whenever $f : X \to Y$ is an monomorphism and $g : X \to E$ is any, there exists $\psi$ with $\psi \circ f = g$, i.e., each diagram of the format 
        \begin{cd}
            0 \rar &X \rar["f"] \dar["g"'] & Y \dlar[dashed, "\psi"] \\
            & E
        \end{cd}
        can be supplement to a commutative triangle;
        \item every monomorphism with domain $M$ is split.
    \end{enumerate}
\end{theorem}

Remarks.
\begin{enumerate}[(a)]
    \item $M_i \in R\lMod$ for $i \in I$, then $\prod M_i$ injective iff all $M_i$ injective.
    \item All left $R$-modules injective iff $R$ semisimple iff all right $R$-modules injective.
    \item $\Z$, $\Z/m\Z$ ($m \geq 2$) not injective
\end{enumerate}

\begin{lemma}[19 Baer's Criterion]
    Let $M \in R\lMod$.
    Then the following are equivalent:
    \begin{enumerate}[(1)]
        \item $M$ is injective;
        \item For every diagram with $I \teq R$ left ideal
        \begin{cd}
            I \rar[hookrightarrow] \dar["\phi"'] & R \dlar[dashed, "\Phi"] \\
            M
        \end{cd}
        there exists $\Phi$ with $\Phi|_I = \phi$
    \end{enumerate}
\end{lemma}

\sepline

2/17/23

\begin{proof}
    (2 $\implies$ 1)

    Without loss of generality, $X \leq Y$ is a submodule and $f$ the embedding.
    To find $\psi \in \Hom(Y, M)$ with $g = \psi \circ f$, we picked a maximal element $(Y_0, g_0)$ in the poset
    \[
        \mathcal{P} = \{(Y', g') \mid X \leq Y' \leq Y, g'|_X = g\}.
    \]

    Assume $Y_0 \subset Y$ and pick $\hat{y} \in Y \setminus Y_0$.
    Define
    \[
        I = \{r \in R \mid r\hat{y} \in Y_0\} \seq \_RR
    \]
    and $\phi \in \Hom_R(I, M)$ with $\phi(r) = g_0(r\hat{y})$.

    By (2), there exists $\Phi \in \Hom_R(R, M)$ with $\Phi|_I = \phi$.

    Set $Y_1 = Y_0 + R\hat{y}$ and define $g_1 : Y_1 \to M$ with $y + r\hat{y} \mapsto g_0(y) + \Phi(r)$ if $y \in Y_0$ and $r \in R$.

    Well-definedness of $g_1$.
    Suppose $y_1 + r_1\hat{y} = y_2 + r_2\hat{y}$ with $y_i \in Y_0$ and $r_i \in R$.
    Then
    \begin{align*}
        0 
            &= g_0(0) \\
            &= g_0((y_1 + r_1\hat{y}) - (y_2 + r_2\hat{y})) \\
            &= g_0((y_1 - y_2) + (r_1 - r_2)\hat{y}) \\
            &= g_0(y_1 - y_2) + g_0((r_1 - r_2)\hat{y}) \\
            &= g_0(y_1) - g_0(y_2) + \phi(r_1 - r_2) \\
            &= g_0(y_1) - g_0(y_2) +\Phi(r_1 - r_2) \\
            &= (g_0(y_1) + \Phi(r_1)) - (g_0(y_2) + \Phi(r_2)) \\
            &= g_1(y_1 + r_1\hat{y}) - g_1(y_2 + r_2\hat{y}).
    \end{align*}
\end{proof}

\sepline

\begin{theorem}[20]
    Let $R$ be a PID and $M \in R\lMod$, then FTAE:
    \begin{enumerate}[(1)]
        \item $M$ is injective;
        \item $M$ is divisible, i.e., $aM = M$ for all $a \in R \setminus \{0\}$;
    \end{enumerate}
\end{theorem}

\begin{proof}
    (1 $\implies$ 2)

    Assume (1), and let $a \in R \setminus \{0\}$.
    Fix $x \in M$ and consider
    \begin{cd}
        ra \dar[mapsto] & I = Ra \rar[hook] \dar["g"'] & R \dlar[dashed, "\exists\psi"] \\
        rx & M
    \end{cd}
    So $g(a) = g(a \cdot 1) = \psi(a \cdot 1) = a\psi(1) \in aM$.

    (2 $\implies$ 1)

    Assume (2).
    We will apply Baer, so let $\_RI \inc \_RR$ and $\phi \in \Hom_R(I, M)$.

    Know $I = Ra$ wlog $a \ne 0$.
    By hypothesis, there exists $x \in M$ with $\phi(a) = a \cdot x$.
    Define $\Phi \in \Hom(R, M)$ via $r \mapsto rx$.
    Check that $\Phi|_I = \phi$.
\end{proof}

\sepline

Supplement: The divisible abelian groups.

Examples.

1. $\Q$

2. $\Z(p^\infty)$ ``Pr\"ufer groups'' for prime $p$, where
\[\textstyle
    \Z(p^\infty) := \bigoplus_{i \in \N} \Z x_i / U(p)
\]
where $U(p)$ is the subgroup of $\bigoplus_{i \in \N} \Z x_i$, generated by $px_1, px_{i+1} - x_i$ for all $i \in \N$.

Find that $\Z\eqc{x}_i \iso \Z/p^i\Z$.
Get embeddings
\begin{cd}[row sep=tiny]
    \Z\eqc{x}_i \iso \Z/p^i\Z \rar[hook] & \Z/p^{i+1}\Z \\
    z + p^i\Z \rar[mapsto] & pz + p^{i+1}\Z
\end{cd}

With $\Z\eqc{x}_i \iso \Z/p^i\Z$, can visualize
\begin{cd}
    \Z\eqc{x}_1 \rar[hook, "\cdot p"]
    & \Z\eqc{x}_2 \rar[hook, "\cdot p"]
    & \Z\eqc{x}_3 \rar[hook, "\cdot p"]
    & \Z\eqc{x}_4 \rar[hook, "\cdot p"]
    & \cdots \rar[hook]
    & \Z(p^\infty)
\end{cd}

Alternate descriptions:
\[\textstyle
    \Z(p^\infty)
        = T_p(\Q/\Z)
        = \{x + \Z \mid p^nx \in \Z \text{ for some } n \in \N\}.
\]

Reminder: $T$ a torsion abelian group.
Then
\[
    T = \bigoplus_{p\text{ prime}} T_p 
    \isp{where}
    T_p = \{x \in T \mid p^n x = 0 \text{ for some } n \in \N\}.
\]

Also
\[
    \Z(p^\infty) \iso \{x \in \C \mid x^{p^n} = 1 \text{ for some } n \in \N\}
    \quad\text{(multiplicative group)}.
\]

\sepline

\begin{theorem}[21]
    $A \in \Z\Mod$ is divisible iff
    \[
        A \iso \Q^{(I)} \oplus \bigoplus_{p\text{ prime}} (\Z(p^\infty))^{(I_p)}
    \]
\end{theorem}

\begin{proof}
    Exercise.
\end{proof}

\sepline

Injective Resolutions.

\begin{theorem}[22 Echmann]
    Every left $R$-module is a submodule of (or embeds into) an injective module.
\end{theorem}

\begin{lemma}[23]
    If $M \in \Z\lMod$, then there exists a divisible $D \in \Z\lMod$ with $M \overset{\iso}{\inc} D$.

    Every abelian group can be embedded as a subgroup of some divisible (injective) abelian group.
\end{lemma}

\begin{proof}
    We know $M \iso \Z^{(I)}/K \seq \Q^{(I)}/K$ (divisible) for some subgroup $K \leq \Z^{(I)}$.
\end{proof}


\sepline

2/22/23

A right $R$-module $M$ is \keyword{flat} in case the tensor functor $M \tensor_R - : R\lMod \to \Z\lMod$ is exact.
Flatness of a left $R$-module is defined symmetrically.

Remarks.
\begin{enumerate}[(1)]
    \item Any tensor functor $M \tensor_R -$ is right exact, hence $M_R$ is flat precisely when $M \tensor_R -$ takes monomorphisms to monomorphisms.
    
    \item Since tensor products commute with arbitrary direct sums, a direct sum $\bigoplus_{i \in I} M_i$ is flat if and only if each $M_i$ is flat.
    
    \item The regular $R$-module $R_R$ is flat, because $R \tensor_R - \iso \id_{R\lMod}$.
    In light of the preceding remark, we find that every free $R$-module is flat, and hence so is every projective $R$-module.
    In particular, if $R$ is semisimple, then all $R$-modules are flat.

    Caution: Here the converse does not hold.
    In fact, all right $R$-modules are flat if and only if $R$ is von Neumann regular, meaning that for any $r \in R$ there exists $s \in R$ with $rsr = r$.
    For instance: the endomorphism rings or infinite dimensional vector spaces are von Neumann regular but not semisimple.

    \item Using the adjointness of tensor and hom functors, one obtains: If $\_SM_R$ is a bimodule such that $M_R$ is flat and $N \in S\lMod$ is injective, then $\_R\Hom_S(M, N)$ is injective in $R\lMod$.
    The proof is a carbon copy of that for injectivity of $\_R\Hom_\Z(R, D)$ in case $D$ is injective in $\Z\lMod$.
\end{enumerate}

\sepline

\begin{lemma}
    If $D$ is a divisible abelian group, then $\Hom_\Z(R, D)$ is an injective left $R$-module.

    Generalize: $\_SM_R$ bimodule with $M_R$ flat and $\_SN$ injective, then $R(\Hom_S(M, N))$ is injective $R$-module.
\end{lemma}

\begin{proof}
    Consider a short exact sequence of left $R$-modules:
    \begin{cd}
        0 \rar & U \rar & V \rar & W \rar & 0
    \end{cd}
    Recall that $R \tensor_R - \iso \id_{R\lMod}$ is an exact functor, so we get an exact sequence of left $R$-modules:
    \begin{cd}
        0 \rar
        & \_R(\_RR_R \tensor_R \_RU) \rar 
        & \_R(\_RR_R \tensor_R \_RV) \rar 
        & \_R(\_RR_R \tensor_R \_RW) \rar 
        & 0
    \end{cd}
    The forgetful functor $R\lMod \to \Z\lMod = \Ab$ is exact so we also have an exact sequence of abelian groups:
    \begin{cd}
        0 \rar
        & \_\Z(\_{\Z}R_R \tensor_R \_RU) \rar 
        & \_\Z(\_{\Z}R_R \tensor_R \_RV) \rar 
        & \_\Z(\_{\Z}R_R \tensor_R \_RW) \rar 
        & 0
    \end{cd}
    Then if $\_{\Z}D \in \Z\lMod$ is divisible, it must also be injective by some result.
    Then by definition the contravariant hom functor $\Hom_\Z(-, D) : \Z\lMod \to \Z\lMod$ is exact, so we get an exact sequence of abelian groups:
    \begin{cd}
        0 \rar
        & \Hom_\Z(R \tensor_R U, D) \rar 
        & \Hom_\Z(R \tensor_R V, D) \rar 
        & \Hom_\Z(R \tensor_R W, D) \rar 
        & 0
    \end{cd}
    For a left $R$-module $X$ we have the hom-tensor adjunction:
    \[
        \_\Z(\Hom_\Z(\_\Z(\_{\Z}R_R \tensor_R \_RX), \_{\Z}D))
            \iso \_\Z(\Hom_R(\_RX_\Z, \_R(\Hom_\Z(\_{\Z}R_R, \_{\Z}D))))
    \]
    \begin{align*}
        \Hom_\Z(R \tensor_R X, D)
            &= \_\Z(\Hom_\Z(\_\Z(\_{\Z}R_R \tensor_R \_RX), \_{\Z}D)) \\
            &\iso \_\Z(\Hom_R(\_RX_\Z, \_R(\Hom_\Z(\_{\Z}R_R, \_{\Z}D)))) \\
            &= \Hom_R(X, \Hom_\Z(R, D))
    \end{align*}
    Hence, we obtain an exact sequence of abelian groups:
    \begin{cd}[column sep=tiny]
        0 \rar
        & \Hom_R(U, \Hom_\Z(R, D)) \rar 
        & \Hom_R(V, \Hom_\Z(R, D)) \rar 
        & \Hom_R(W, \Hom_\Z(R, D)) \rar 
        & 0
    \end{cd}
    In other words, $\Hom_R(-, \Hom_\Z(R, D)) : \R\lMod \to \Z\lMod$ is an exact functor, so by definition, $\Hom_\Z(R, D)$ is an injective left $R$-module.
\end{proof}


\begin{theorem}
    Let $R$ be any ring.
    Every left $R$-module can be embedded as a submodule of an injective left $R$-module.
\end{theorem}

\begin{proof}
    Let $\_RM \in \R\lMod$.
    There is a homomorphism of abelian groups defined as follows:
    \begin{align*}
        \phi : \_{\Z}M &\To \Hom_\Z(R, M) \\
            m &\Mapsto (\phi_m : r \mapsto rm)
    \end{align*}
    (Probably easy to check well-defined, i.e., that $\phi_m$ is $\Z$-linear for all $m \in M$, and that $\phi$ itself is $\Z$-linear.)

    We check that $\phi$ is injective.
    Suppose $\phi_m = 0 \in \Hom_\Z(R, M)$, i.e., that $\phi_m(r) = 0$ for all $r \in R$.
    Then in particular, $m = 1 \cdot m = \phi_m(1) = 0$, so $\ker \phi = 0$.

    By Lemma 23, there is an embedding $\iota : \_{\Z}M \inc \_{\Z}D$ with $D$ an injective abelian group.

    Left exactness of the hom-functor gives us an injection $\iota_* : \Hom_\Z(R, M) \to \Hom_\Z(R, D)$ of abelian groups.

    Hence, we have an injection of abelian group $\Phi := \iota_* \circ \phi : \_{\Z}M \to \Hom_\Z(R, D)$.

    Then $\Phi$ acts by $\Phi(m) = \iota_*\phi_m = \iota \circ \phi_m$ with $(\iota \circ \phi_m)(x) = \iota(\phi_m(x)) = \iota(xm) = xm$ in $D$.

    Recall that $\Hom_\Z(R, D)$ has a left $R$-module structure $r \cdot f = f(- \cdot r)$, i.e., $(r \cdot f)(x) = f(xr)$ (where $xr$ is simply multiplication in $R$) for all $r, x \in R$ and $f \in \Hom_\Z(R, D)$.

    We check that $\Phi$ is $R$-linear.
    For $r \in R$ and $m \in M$,
    \[
        \Phi(rm)
            = \iota_*\phi_{rm}
            = \iota \circ \phi_{rm}.
    \]
    Applying this to $x \in R$ we have the following in $D$:
    \[
        (\iota \circ \phi_{rm})(x)
            = \iota_*\phi_{rm}(x)
            = x \cdot rm
            = xr \cdot m
            = \iota_*\phi_m(xr)
            = (r \cdot \iota_*\phi_m)(x)
            = (r \cdot \Phi(m))(x)
    \]
    Hence, $\Phi(rm) = r \cdot \Phi(m)$, so indeed $\Phi$ is $R$-linear.

    In other words, $\Phi$ corresponds to a left $R$-module homomorphism $\tilde{\Phi} : \_RM \to \_R(\Hom_\Z(R, D))$ which is defined the same as $\Phi$ set-theoretically.
    Since $\Phi$ is injective, so is $\tilde{\Phi}$.

    Thus, we have embedded $\_RM$ into $\Hom_\Z(R, D)$, which is an injective left $R$-module by the previous lemma.
\end{proof}

\sepline

9/24/23

Derived Functors

\begin{theorem}[31 Long Exact Homology Sequence]
    Suppose $\AA$ is SES in $R\Comp$:
    \begin{cd}
        0 \rar & A_\bullet \rar["f"] & A_\bullet' \rar["f'"] & A_\bullet'' \rar & 0
    \end{cd}
    Then for each $n \in \Z$, there exists $\partial_n \in \Hom_R(H_n(A_\bullet''), H_{n-1}(A_\bullet))$ such that the following sequence is exact in $R\lMod$:
    \begin{cd}
        \cdots \rar
        & H_n(A_\bullet) \rar["H_n(f)"]
        & H_n(A_\bullet') \rar["H_n(f')"]
        & H_n(A_\bullet'') \rar["\partial_n"]
        & H_{n-1}(A_\bullet) \rar
        & \cdots
    \end{cd}

    In addition, we have naturality in the following sense:
    For any $\AA$ as above, the family $(\partial_n)_{n \in \Z} = (\partial_n^\AA)_{n \in \Z}$ satisfies the following condition:
    Whenever $h : \AA \to \BB$ is a morphism of SES's, i.e., the following diagram commutes:
    \begin{cd}
        0 \rar 
        & A_\bullet \rar["f"] \dar["h"]
        & A_\bullet' \rar["f'"] \dar["h'"]
        & A_\bullet'' \rar \dar["h''"]
        & 0 \\
        0 \rar & B_\bullet \rar["g"] & B_\bullet' \rar["g'"] & B_\bullet'' \rar & 0
    \end{cd}
    then the following diagram commutes:
    \begin{cd}
        H_n(A_\bullet'') \rar["\partial_n^\AA"] \dar["H_n(h'')"']
        & H_{n-1}(A_\bullet) \dar["H_{n-1}(h)"] \\
        H_n(B_\bullet'') \rar["\partial_n^\BB"] 
        & H_{n-1}(B_\bullet)
    \end{cd}
\end{theorem}

\sepline

Let $u : A_\bullet \to A_\bullet'$ be a morphism in $R\Comp$.
Say $u = (u_n)_{n \in \Z}$.

Call $u$ \keyword{nullhomotopic} if there exists $s_n \in \Hom_R(A_n, A_{n+1}')$ such that
\[
    u_n = d_{n+1}' \circ s_n + s_{n-1} \circ d_n.
\]
Looks like
\begin{cd}
    A_{n+1} \rar["d_{n+1}"] \dar["u_{n+1}"']
    & A_n \rar["d_n"] \dar["u_n"'] \dlar["s_n" description]
    & A_{n-1} \dar["u_{n-1}"] \dlar["s_{n-1}" description] \\
    A_{n+1}' \rar["d_{n+1}'"']
    & A_n' \rar["d_n'"']
    & A_{n-1}'
\end{cd}

Call $(s_n)_{n \in \Z}$ chain maps of degree $+1$.


\begin{proposition}[32]
    Suppose $u \in \Hom_{R\Comp}(A_\bullet, A_\bullet')$ is nullhomotopic.
    Then $H_n(u) = 0$ for all $n \in \Z$.
\end{proposition}


\sepline

Let $f, g : A_\bullet \to A_\bullet'$ be morphisms in $R\Comp$.

Say $f$ and $g$ are \keyword{homotopic}, denoted $f \htpy g$, if $f - g$ is nullhomotopic.

Two complexes are \keyword{homotopy equivalent}, $A_\bullet \htpy A_\bullet'$, if there exist $f : A_\bullet \to A_\bullet'$ and $f' : A_\bullet' \to A_\bullet$ such that $f' \circ f \htpy \id_{A_\bullet}$ and $f \circ f' \htpy \id_{A_\bullet'}$.

\begin{corollary}
    If $f \htpy g$ then $H_n(f) = H_n(g)$ for all $n \in \Z$.
\end{corollary}

\sepline

Deleted Resolution

Let $M \in R\lMod$ and
\begin{cd}
    \cdots \rar & P_n \rar & \cdots \rar & P_1 \rar & P_0 \rar & M \rar & 0
\end{cd}
be a projective resolution of $M$.

The \keyword{deleted projective resolution} is the complex $P_\bullet$:
\begin{cd}
    \cdots \rar & P_n \rar & \cdots \rar & P_1 \rar & P_0 \rar & 0
\end{cd}
(maybe not exact at $P_0$)


Given injective resolution
\begin{cd}
    0 \rar & M \rar & E_0 \rar & E_1 \rar & \cdots \rar & E_n \rar & \cdots
\end{cd}
the \keyword{deleted injective resolution} is complex $E_\bullet$:
\begin{cd}
    0 \rar & E_0 \rar & E_1 \rar & \cdots \rar & E_n \rar & \cdots
\end{cd}


\sepline

Plan: study model problem.

Look at $\FF = X \tensor_R -$.
For SES
\begin{cd}
    0 \rar & M \rar["f"] & M' \rar["f'"] & M'' \rar & 0
\end{cd}
in $R\lMod$, we want some information about $\ker(X \tensor_R f : X \tensor_R M \to X \tensor_R M')$.

Step 1.

One can construct projective resolutions of $M$, $M'$, $M''$ such we get a SES of deleted projective resolutions $0 \to P_\bullet \to P_\bullet' \to P_\bullet'' \to 0$.

In particular, if $\bar{f} = (f_n)_{n \in \Z}$ and 

some more stuff...

\sepline

\begin{lemma}[33 Comparison]
    Let $M, M' \in R\lMod$, $f \in \Hom_R(M, M')$.
    Let $P_\bullet, P_\bullet'$ be deleted projective resolutions.
    Then there exists $\bar{f} = (f_n)_{n \geq 0} \in \Hom_R(P_\bullet, P_\bullet')$ such that the following diagram commutes:
    \begin{cd}
        \cdots \rar
        & P_1 \rar["d_1"] \dar[dashed, "f_1"]
        & P_0 \rar["d_0"] \dar[dashed, "f_0"]
        & M \rar \dar["f"]
        & 0 \\
        \cdots \rar
        & P_1' \rar["d_1'"]
        & P_0' \rar["d_0'"]
        & M' \rar
        & 0 \\
    \end{cd}

    Moreover, given another chain map $\bar{g} \in \Hom_R(P_\bullet, P_\bullet')$, we obtain $\bar{f} \htpy \bar{g}$ (as chain maps of deleted projective resolutions).
\end{lemma}

\begin{corollary}
    Any two deleted projective resolutions are homotopy equivalent.
\end{corollary}

\begin{lemma}[Horseshoe]
    Given a SES
    \begin{cd}
        0 \rar & M \rar["f"] & M' \rar["f'"] & M'' \rar & 0
    \end{cd}
    and projective resolutions $P_\bullet \to M$ and $P_\bullet'' \to M''$, there is a projective resolution $P_\bullet' \to M'$ and chain maps $\bar{f}$ and $\bar{f}'$ such that stuff commute good:
    \begin{cd}
        0 \rar
        & P_\bullet \rar["\bar{f}"] \dar
        & P_\bullet' \rar["\bar{f}'"] \dar
        & P_\bullet'' \rar \dar
        & 0 \\
        0 \rar
        & M \rar["f"]
        & M' \rar["f'"] 
        & M'' \rar
        & 0
    \end{cd}
\end{lemma}

\sepline

Left Derived Functors

Let $T : R\lMod \to S\lMod$ be a covariant additive functor.

Want to extract functors $L_nT$ for $n \geq 0$ which sends a SES to a LES.

Also if $T$ is right exact, want $L_0(T) \iso T$.


Pick and fix a (deleted) projective resolution $\P_M \to M$ for each $M \in R\lMod$.

Define $L_nT$ on objects $M \in R\lMod$ by
\[
    L_nT(M) = H_n(T(\P_M)).
\]

Define $L_nT$ on morphisms $f \in \Hom_R(M, M')$.

By Lemma 33, choose and fix $\bar{f} \in \Hom_{R\Comp}(\P_M, \P_{M'})$ lying over $f$.
Then define
\[
    L_nT(f) = H_n(T(\bar{f})).
\]


\begin{theorem}[36]
    $L_nT$ is a well-defined covariant functor $R\lMod \to S\lMod$.

    Moreover, if $\hat{\P}_M$ is a different choice of deleted projective resolution for each $M \in R\lMod$, then the functor $\hat{L}_nT$ defined with respect to the $\P_M$'s satisfies $\hat{L}_nT \iso L_nT$.
\end{theorem}

\begin{theorem}[37]
    All notations as before.
    Suppose
    \begin{cd}
        0 \rar & M \rar["f"] & M' \rar["f'"] & M'' \rar & 0
    \end{cd}
    is a SES in $R\lMod$.
    Then there exists an exact sequence
    \begin{cd}
        \cdots \rar
        & L_{n+1}T(M'') \rar["\partial_{n+1}"]
        & L_nT(M) \rar["L_nT(f)"]
        & L_nT(M') \rar["L_nT(f')"]
        & L_nT(M'') \rar["\partial_n"]
        & \cdots \\
        & L_1(M'') \rar["\partial_1"]
        & L_0T(M) \rar["L_0T(f)"]
        & L_0T(M') \rar["L_0T(f')"]
        & L_0T(M'') \rar
        & 0
    \end{cd}
    (In particular, $L_0T$ is always right exact.)

    Moreover:
    \begin{enumerate}[(a)]
        \item If $T$ is right-exact then $L_0T \iso T$;
        \item If $M$ is projective then $L_nT(M) = 0$ for $n \geq 1$;
        \item All $\partial_n$ for $n \geq 1$ are natural in the sense specified in Theorem 31.
    \end{enumerate}
\end{theorem}


\begin{theorem}[38 Dimension Shifting]
    Notation as before.
    Then
    \[
        L_nT(M)
        \iso L_{n-1}T(K_1)
        \iso L_{n-1}(K_2)
        \iso \cdots
        \iso L_1T(K_{n-1})
    \]
    where $K_i$ is an $i$th syzygy of $M$.
\end{theorem}



\sepline

Other derived functors in a nutshell, i.e., left/right derived functors of co-/contravariant functors.

mnemonic:

left derived fill in the left, come from left $\infty$

right derived fill in the right, come from right $\infty$


$T : R\lMod \to S\lMod$, $M \in R\lMod$, get injective resolution $M \to \E_M$.
Apply $T$, take $n$th homology, define $R^nT(M) = H_n(T(\E_M))$.

From $0 \to M \to M' \to M'' \to 0$ get LES 
\begin{cd}
    0 \rar
    & R^0T(M) \rar
    & R^0T(M') \rar
    & R^0T(M'') \rar["\partial_1"]
    & R_1T(M) \rar
    & \cdots
\end{cd}


If $T$ left exact then $R^0T \iso T$.

If $M$ injective then $R^nT(M) = 0$ for $n \geq 1$.

For contravariant $T : R\lMod^\op \to S\lMod$, right used $\P_M$ and left uses $\E_M$.


\sepline

3/10/23

Change of index position in right derived functors.

Indexing going down from left to right $A_\bullet$:
\begin{cd}
    \cdots \rar & A_{n+1} \rar["d_{n+1}"] & A_n \rar["d_n"] & A_{n-1} \rar["d_{n-1}"] & \cdots
\end{cd}

Indexing going up from left to right $B_\bullet$:
\begin{cd}
    \cdots \rar & B_{n-1} \rar["d_n"] & B_n \rar["d_{n+1}"] & B_{n+1} \rar & \cdots
\end{cd}

(Some use indexing with superscripts)

$n$th homology of $A_\bullet$ is $H_n(A_\bullet) = \ker d_n / \im d_{n+1}$.

$n$th homology of $B_\bullet$ is $H^n(B_\bullet) = \ker d_{n+1} / \im d_n$.

``$n$th homology is the homology in the position which is labeled $n$''

$n$th right derived functor: homology of a complex of type $B_\bullet$, write $R^n$ instead of $R_n$.

\sepline

The Functors $\Tor$ and $\Ext$.

Let $\_SX_R$ be a bimodule, $T = X \tensor_R - : R\lMod \to S\lMod$.

We know $T$ is right exact but not left exact in general.

Define left derived functor $\Tor_n^R(X, Y) = L_nT(Y) = H_n(X \tensor_R \P_Y)$ with $\P_Y$ a deleted projective resolution of $Y$.

For $\Tor_n^R(X, f) = L_nT(f)$ for $f \in \Hom_R(Y, Y')$.

\sepline

Comments.

1. $T$ is right exact, hence $\Tor_0^R(X, Y) = X \tensor_R Y$.

2. If
\begin{cd}
    0 \rar & Y \rar["f"] & Y' \rar["f'"] & Y'' \rar & 0
\end{cd}
is exact in $R\lMod$, then we get an exact sequence in $S\lMod$:
\begin{cd}[column sep=tiny]
    \cdots \rar["\partial_2"]
    & \Tor_1^R(X, Y) \rar
    & \Tor_1^R(X, Y') \rar
    & \Tor_1^R(X, Y'') \rar["\partial_1"]
    & X \tensor_R Y \rar
    & X \tensor_R Y' \rar
    & X \tensor_R Y'' \rar
    & 0
\end{cd}

3. Let $Y \iso P/K$, $P$ projective, i.e.,
\begin{cd}
    0 \rar & K \rar[hook] & P \rar["\text{can}"] & Y \rar & 0
\end{cd}
is exact, then $\Tor_n^R(X, P) = 0$ for all $n \geq 1$.
So
\begin{cd}
    0 \rar
    & \Tor_1^R(X, P/K) \rar
    & X \tensor_R K \rar["{X \tensor \iota}"]
    & X \tensor_R P \rar
    & \cdots
\end{cd}
is exact.
So $\Tor_1^R(X, P/K) = \ker(X \tensor \iota)$

\sepline

Define analogously
\[
    \Tor_n^R(-, Y)(X) = H_n(\P_X \tensor_R Y) = L_n\tilde{T}(X).
\]
Then $\Tor_n^R(X, Y)$ has two definitions.

How are $H_n(X \tensor_R \P_Y)$ and $H_n(\P_X \tensor_R Y)$ related?

\begin{theorem}[39 Balancedness of arguments of $\Tor$]
    For $X \in \rMod{R}$, $Y \in R\lMod$,
    \[
        H_n(X \tensor_R \P_Y) \iso H_n(\P_X \tensor_R Y)
    \]
    in $S\lMod$.
    (Both defined as $\Tor_n^R(X, Y)$ up to isomorphism.)
\end{theorem}

\begin{proof}
    See Rotman
\end{proof}

\sepline

Comment.

1. The tor functors commute with arbitrary direct sums.

Reason: tensor and homology commutes with arbitrary direct sums, and tor is just the composition of these two devices.

2. The tor functors also commute with direct limits extending over directed partially ordered sets (not with arbitrary colimits nor limits).

A poset $(S, \leq)$ is \keyword{directed} if for all $x, y \in S$, there exists $z \in S$ with $x \leq z$ and $y \leq z$.

\sepline

\begin{theorem}[40]
    \
    \begin{enumerate}[(A)]
        \item For $X \in \rMod{R}$, TFAE:
        \begin{enumerate}[(i)]
            \item $X_R$ is flat,
            \item $\Tor_n^R(X, Y) = 0$ for all $n \geq 1$ and $Y \in R\lMod$,
            \item $\Tor_1^R(X, Y) = 0$ for all $Y \in R\lMod$.
        \end{enumerate}
        \item Analogous statement for $\_RY$.
    \end{enumerate}
\end{theorem}

\begin{proof}
    (i $\implies$ ii)

    Assume (i) and let $Y \in R\lMod$, $\cdots \to P_1 \to P_0 \to Y \to 0$ a projective resolution.

    By flatness, get exact $\cdots \to X \tensor P_1 \to X \tensor P_0 \to X \tensor Y \to 0$.

    Since $X$ is flat,
    \begin{cd}
        \cdots \rar
        & X \tensor P_4 \rar
        & X \tensor P_3 \rar
        & X \tensor P_2 \rar
        & X \tensor P_1 \rar
        & X \tensor P_0
    \end{cd}
    is an exact portion of $X \tensor \P_Y$.

    Somehow get $H_n(X \tensor \P_Y) = 0$ for all $n \geq 1$.

    (ii $\implies$ iii) trivial

    (iii $\implies$ i)

    Assume (iii) and let $0 \to Y \to Y' \to Y'' \to 0$ be an exact sequence in $R\lMod$.
    Then by long exact homology sequence get
    \begin{cd}
        \Tor_1^R(X, Y'') \rar
        & X \tensor Y \rar
        & X \tensor Y' \rar
        & X \tensor Y'' \rar
        & 0
    \end{cd}
    is exact.
    But $\Tor_1^R(X, Y'')$ is zero by hypothesis, i.e., $X \tensor_R -$ is exact.
\end{proof}

\sepline

$\Tor$ for $R = \Z$. (`tor' is short for `torsion,' comes from $\Z$) [Analogous for $R$ a PID.]


If $R$ is commutative, then $X \tensor_R Y \iso Y \tensor_R X$, and hence $\Tor_n^R(X, Y) \iso \Tor_n^R(Y, X)$.

\begin{proposition}[41]
    If $X, Y \in \Z\lMod$, then $\Tor_n^\Z(X, Y) = 0$ for $n \geq 2$.
\end{proposition}

\begin{proof}
    Know $\operatorname{pdim}_\Z Y = 1$, i.e., any first syzygy $K$ of $Y$ is projective.

    So by dimension-shifting, $\Tor_2^\Z(X, Y) \iso \Tor_1^\Z(X, K_1) = 0$ since $K_1$ is projective and hence flat.

    Rest by induction.
    For $n \geq 3$, get $\Tor_n^\Z(X, Y) \iso \Tor_{n-1}^\Z(X, K_1) = 0$ since $K_1$ projective.
\end{proof}

\sepline

Example.

$\Tor_1^\Z(\Q, \Z/6\Z) = 0$ because $\Q$ is torsionfree hence flat.


\sepline

3/13/23


\begin{proposition}[42]
    \
    \begin{enumerate}[(1)]
        \item $\Tor_1^\Z(X, Y) \iso \Tor_1^\Z(T(X), T(Y))$ where $T(X)$ is torsion subgroup of $X$.
        ($\Tor$ only sees torsion part.)
        \item $\Tor_1^\Z(X< \Z/n\Z) = \{x \in X \mid nx = 0\} =: X[n]$.
        \item $\Tor_1^\Z(\Z/m\Z, \Z/n\Z) \iso \Z/d\Z$ where $d = \gcd(m, n)$.
        \item $\Tor_1^\Z(X, Y)$ is torsion for all $X$ and $Y$.
        \item $\Tor_1^\Z(\Q/\Z, Y) \iso T(Y)$ for all $Y$.
    \end{enumerate}
\end{proposition}

\begin{proof}
    (1) Consider the exact sequence $0 \to T(Y) \to Y \to Y/T(Y) \to 0$.
    Then
    \begin{cd}
        \Tor_2^\Z(X, Y/T(Y)) \rar
        & \Tor_1^\Z(X, T(Y)) \rar
        & \Tor_1^\Z(X, Y) \rar
        & \Tor_1^\Z(X, Y/T(Y))
    \end{cd}
    is exact.

    But $\Tor_2^\Z(X, Y/T(Y)) = 2$ and $\Tor_1^\Z(X, Y/T(Y)) = 0$ because $n \geq 2$ and $Y/T(Y)$ is torsionfree, respectively.

    Hence, $\Tor_1^\Z(X, Y) \iso \Tor_1^\Z(X, T(Y)) \iso \Tor_1^\Z(T(X), T(Y))$.

    (2) Consider
    \begin{cd}
        0 \rar
        & n\Z \rar[hook, "\iota"]
        & \Z \rar["\text{can}"]
        & \Z/n\Z \rar
        & 0
    \end{cd}
    Then get
    \begin{cd}[row sep=tiny]
        \Tor_1^\Z(X, \Z) \rar \dar[equals]
        & \Tor_1^\Z(X, \Z/n\Z) \rar["\partial_1"]
        & X \tensor_\Z n\Z \rar
        & X \tensor_\Z \Z \\
        0
    \end{cd}
    This
    \begin{align*}
        \Tor_1^\Z(X, \Z/n\Z) 
            &\iso \ker(X \tensor \iota) \\
            &= \{x \tensor n \in X \tensor_\Z n\Z \mid x \tensor n = 0 \in X \tensor_\Z \Z\}.
    \end{align*}
    Note $x \tensor n = nx \tensor 1$ in $X \tensor_\Z \Z$ but not in $X \tensor_\Z n\Z$.
    In particular, $X \tensor_Z \Z \iso X$ via $X \tensor 1 \mapsto x$.
    \begin{align*}
        &= \{x \tensor n \in X \tensor_\Z n\Z \mid xn = 0 \in X\} \\
        &\iso \{x \in X \mid xn = 0\} \\
        &= X[n].
    \end{align*}
    Why is isomorphism true?
    Have map
    \begin{cd}[row sep=tiny]
        X \tensor n\Z \rar["\iso"] & X \tensor \Z \rar["\iso"] & X \\
        x \tensor n \rar[mapsto] & x \tensor 1 \\
        & x \tensor z \rar[mapsto] & zx
    \end{cd}

    (3) Have
    \[
        \Tor_1^\Z(\Z/m\Z, \Z/n\Z)
            \iso \Z/m\Z[n]
            = \tfrac{m}{d}\Z/m\Z
            \iso \Z/d\Z
    \]
    (Last is cyclic of order $d$.)

    (4) (a) For $Y$ finitely generated torsion, have $T(Y) = \bigoplus_{\text{finite}} C_i$ with $C_i \iso \Z/p_i^{m_i} \Z$.

    Hence,
    \[
        \Tor_1^\Z(X, Y)
            \iso \Tor_1^\Z(X, T(Y))
            \iso \bigoplus_{\text{finite}} \Tor_1^\Z(X, C_i)
            \iso \bigoplus_{\text{finite}} X[p_i^{m_i}].
    \]
    Each $X[p_i^{m_i}]$ is obviously torsion by definition, so direct sum is torsion.

    (b) $Y$ arbitrary.
    Then $Y = \bigcup_{i \in I} Y_i$ with $Y_i$ finitely generated.
    In other words, $Y$ is the direct limit of all of its finitely generated subgroups?
    That is, $Y = \varinjlim_{i \in I} Y_i$.

    Then
    \[
        \Tor_1^\Z(X, Y)
            \iso \varinjlim_{i \in I} \Tor_1^\Z(X, Y_i)  
    \]
    Each $\Tor_1^\Z(X, Y_i)$ is torsion, so the limit is torsion.


    (5) Consider the following exact sequence:
    \begin{cd}[row sep=small]
        \Tor_1^\Z(\Q, Y) \rar \dar[equals]
        & \Tor_1^\Z(\Q/\Z, Y) \rar
        & \Z \tensor_\Z Y \rar \dar["\iso"]
        & \Q \tensor_Z Y
        \\
        0 & & Y \urar["\psi = 1 \tensor -"']
    \end{cd}
    First since $\Q$ torsionfree.

    Hence
    \[
        \Tor_1^\Z(\Q/\Z, Y)
            \iso \ker(\iota \tensor Y)
            \iso \ker\psi
            \leq Y.
    \]
    Since $\Tor_1^\Z(\Q/\Z, Y)$ is torsion, then $\ker\psi$ is torsion and sitting inside $Y$, so $\ker\psi \seq T(Y)$.

    To see that $T(Y) \seq \ker\psi$, let $y \in T(Y)$ and $z \in \Z \setminus \{0\}$ with $zy = 0$.

    Then $\psi(y) = 1 \tensor y \in \Q \tensor_\Z Y$.
    Here,
    \[
        1 \tensor y
        = \tfrac{z}{z} \tensor y 
        = \tfrac{1}{z} \tensor zy
        = 0.
    \]

\end{proof}


\sepline

last class or two of A


\newpage
\sepline

4/3/23

starting 236B

(Re)introduce additive and abelian categories

Prototypical example: sheaves of abelian groups on a topological space.

Will define derived categories (objects: complexes of objects in an abelian category. morphism: want such that quasi-isomorphisms of complexes become isomorphisms of objects.)

Then get a natural way of talking about derived functors.

will find a few functors $f_*$, $f^*$, $f_!$, $f^!$, $\D$, $\udl{\Hom}$ for sheaves.

will define triangulated category

will define $t$-structures (extra data on triangulated category)

example is perverse sheaves

textbook: \textit{Methods of Homological Algebra} by Gelfand and Manin

\sepline

Examples
\begin{enumerate}[(a)]
    \item $\Ab$ is abelian groups with group homomorphisms
    \item $R\lMod$ is (left) $R$-modules with (left) $R$-module homomorphisms
    \item $\SAb$ is sheaves of abelian groups

    $\PAb$ is presheaves of abelian groups
    \item (quasi-)coherent sheaves of modules over a ringed space
\end{enumerate}

\sepline

Let $\CC$ be a (locally-small) category.

\paragraph{A1.}
Each set $\Mor_\CC(X, Y)$ is an abelian group ($+$), the composition of morphism is bi-additive.
In which case, usually write $\Hom_\CC$ in place of $\Mor_\CC$.

(note that it is ambiguous whether A1 is a structure or a property.)

In particular, $\Hom_\CC$ is a functor $\CC^\op \times \CC \to \Ab$.

And automatically get a zero morphism $0 \in \Hom_\CC(X, Y)$.

\paragraph{A2.}
There exists a zero object $0 \in \CC$, i.e., an object such that $\Hom_\CC(0, 0) = 0$.

This gives $\Hom_\CC(0, X) = 0$ and $\Hom_\CC(X, 0) = 0$ for all $X \in \CC$.

Always have identity $\id_0 : 0 \to 0$, but then $\id_0 = 0$.
So then for any $f : 0 \to X$, must have $f = f \circ \id_0 = f \circ 0 = 0$, using A1.

Exercise: any two zero objects are isomorphic.

\paragraph{A3.}
For any $X_1, X_2 \in \CC$, there exists an object $Y \in \CC$ any morphisms \begin{cd}
    X \rar["i_1", shift left] & Y \lar["p_1", shift left]
    & X_2 \rar["i_2", shift left] & Y \lar["p_2", shift left]
\end{cd}
such that $p_1i_1 = \id_{X_1}$, $p_2i_2 = \id_{X_2}$, $i_1p_1 + i_2p_2 = \id_Y$, and $p_2i_1 = p_1i_2 = 0$.

\begin{lemma}
    We have a cartesian diagram
    \begin{cd}
        Y \rar["p_1"] \dar["p_2"'] & X_1 \dar \\
        X_2 \rar & 0
    \end{cd}
    That is, for all $Z$ with $Z \to X_i$ making the diagram commute, there is a unique $Z \to Y$ which completes the diagram.
    and a co-cartesian diagram
    \begin{cd}
        Y & X_1 \lar["i_1"'] \\
        X_2 \uar["i_2"] & 0 \lar \uar
    \end{cd}

\end{lemma}


\begin{proof}
    Suppose we have commutative diagram
    \begin{cd}
        Z \ar[drr, bend left=20, "q_1"] \ar[ddr, bend right=20, "q_2"'] \drar[dashed, "\phi"]\\
        & Y \rar["p_1"] \dar["p_2"'] & X_1 \dar \\
        & X_2 \rar & 0
    \end{cd}
    Need to construct $\phi : Z \to Y$ such that $q_i = p_i \circ \phi$.

    Take $\phi = i_1q_1 + i_2q_2$, then
    \[
        p_1 \circ \phi
            = p_1i_1q_1 + p_1i_2q_2
            = \id_{X_1}q_1 + 0q_2
            = q_1.
    \]
    Similarly, $p_2 \circ \phi = q_2$.

    Verify uniqueness of $\phi$...
\end{proof}

An \keyword{additive category} is a category satisfying A1, A2, A3.

\sepline

To state A4, need more notation

Let $\CC$ be a category satisfying A1 and A2.

Let $\phi : X \to Y$ be a morphism in $\CC$.

A \keyword{kernel} of $\phi$ is a morphism $i : Z \to X$ such that
\begin{enumerate}[(a)]
    \item $\phi \circ i = 0$
    \item for any $i' : Z' \to X$ with $\phi \circ i' = 0$, there is a unique $g : Z' \to Z$ such that $i' = i \circ g$. (that is, $i$ is unique up to unique isomorphism with respect to (a)).
\end{enumerate}

A \keyword{cokernel} is the dual notion.

Exercise: this definition of kernel is equivalent to the following: for all $Z' \in \CC$, the sequence
\begin{cd}
    0 \rar & \Hom(Z', Z) \rar["i_*"] & \Hom(Z', X) \rar["\phi_*"] & \Hom(Z', X)
\end{cd}
is exact.
Similar for cokernel.

\paragraph{A4.}
For any morphism $\phi : X \to Y$, there is a sequence of morphisms
\begin{cd}
    K \rar["k"] & X \rar["i"] & I \rar["j"] & Y \rar["c"] & K'
\end{cd}
such that
\begin{enumerate}[(a)]
    \item $j \circ i = \phi$
    \item $K = \ker\phi$ and $K' = \coker\phi$
    \item $I = \coker k = \ker c$
\end{enumerate}

An \keyword{abelian category} is a category satisfying A1-4.

\sepline

4/5/23

II Sheaves

Examples from complex analysis

\paragraph{(a)} Holomorphic functions on the Riemann sphere $\P = \C \cup \{\infty\}$.

For each open subset $U \seq \P$, we consider the ring of holomorphic functions $f : U \to \C$.

There is a sheaf $\OO$ of holomorphic functions on $\P$ is given by the data
\[
    \{(f, V) \mid V \seq \P \text{ open}, f : V \to \C \text{ holomorphic}\}.
\]
Note that $V$ need not be the maximal open set on which $f$ is defined.

\paragraph{(b)} The sheaf of solutions of a linear ODE.

Let $U \seq \P$ be open, $a_i(z) \in \Gamma(U, \OO) := \{f : U \to \C \text{ holomorphic}\}$ for $i = 0, 1, \dots, n - 1$.

Denote by $S$ the collection of all $(V, f)$ such that $V \seq U$ is open and $f$ is holomorphic on $V$ such that
\[
    Lf := \frac{\dd^nf}{\dd z^n} + \sum_{i=0}^{n} a_i(z)\frac{\dd^if}{\dd  z^i} = 0.
\]

When $V$ is connected and simply connected, basic result of ODE gives
\[
    \Gamma(V) = \{f : V \to \C \text{ holomorphic with } Lf = 0\} \iso \C^n.
\]

In general, the solution may depend on the topology of $V$.

e.g., $U = \C \setminus \{0\}$, $L = \frac{\dd^2}{\dd{z}^2} + \frac{1}{z}\frac{\dd}{\dd{z}}$.
The solutions are $c_1\log{z} + c_2$ for any ``branch'' of $\log$, but $\Gamma(U) = \{\text{const}\}$.

(Riemann-Hilbert correspondence)

\sepline

A \keyword{presheaf (of sets)} $\FF$ on a topological space $Y$ consists of the following data:
\begin{itemize}
    \item a set $\FF(U)$ for each open subset $U \seq Y$,
    \item a (restriction) map $r_{UV} : \FF(U) \to \FF(V)$ for each pair of open sets $V \seq U$,
\end{itemize}
satisfying
\begin{itemize}
    \item $r_{UU} = \id_{\FF(U)}$,
    \item $r_{VW} \circ r_{UV} = r_{UW}$ whenever $W \seq V \seq U$.
\end{itemize}

A \keyword{sheaf} is a presheaf $\FF$ satisfying:
\begin{itemize}
    \item for any open covering $U = \bigcup_{i \in I} U_i$ and $s_i \in \FF(U_i)$ such that for all $i, j \in I$ with
    \[
        r_{U_i, U_i \cap U_j}(s_i) = r_{U_j, U_i \cap U_j}(s_j)
    \]
    then there exists a unique $s \in \FF(U)$ with $s_i = r_{U,U_i}(s)$ for all $i \in I$.
\end{itemize}


A \keyword{morphism of presheaves on $Y$} $f : \FF \to \GG$  consists of the following data:
\begin{itemize}
    \item a family of maps $f(U) : \FF(U) \to \GG(U)$ for each open subset $U \seq Y$
\end{itemize}
satisfying
\begin{itemize}
    \item for each pair of open sets $V \seq U$, the following diagram commutes:
    \begin{cd}
        \FF(U) \rar["f(U)"] \dar["r_{UV}^\FF"']
        & \GG(U) \dar["r_{UV}^\GG"] \\
        \FF(V) \rar["f(V)"'] & \GG(V)
    \end{cd}
\end{itemize}

A \keyword{morphism of sheaves} is a morphism of the underlying presheaves.


A \keyword{presheaf of groups/rings/vector spaces} is a presheaf $\FF$ of sets such that each $\FF(U)$ is a group/ring/vector space and each restriction map is a morphism is the corresponding category.

sheaves and morphisms defined in the natural way.

\sepline

A \keyword{abelian presheaf} is a presheaf of abelian groups. (this course only, not standard language)

Let $f : \FF \to \GG$ be a morphism of abelian presheaves.

Recall $f(U) : \FF(U) \to \GG(U)$.

Define a kernel $K(U) := \ker f(U)$ and a cokernel $C(U) := \coker f(U)$.

With natural restrictions, $K$ and $C$ are presheaves. (exercise)


A sequence of presheaves 
\begin{cd}
    \FF \rar["f"] & \GG \rar["g"] & \HH
\end{cd}
is exact if for all $U$, the sequence
\begin{cd}
    \FF(U) \rar["f(U)"] & \GG(U) \rar["g(U)"] & \HH(U)
\end{cd}
is exact.


This implies that $\PAb$ is an abelian category.

\sepline

What about abelian sheaves?

Let $f : \FF \to \GG$ be a morphism of abelian sheaves.

Can try to define kernel $K(U) = \ker f(U)$ and cokernel $C(U) := \coker f(U)$.
Know that both are presheaves, but are they sheaves?

\begin{proposition}
    \
    \begin{enumerate}[(a)]
        \item The kernel $K$ is an abelian sheaf.
        \item The cokernel presheaf $C$ may not be a sheaf.
    \end{enumerate}
\end{proposition}

\sepline

4/10/23

Functors Between Abelian Categories.

Let $\CC$ and $\CC'$ be additive categories.

A functor $F : \CC \to \CC'$ is \keyword{additive} if all maps
\[
    F : \Hom_\CC(X, Y) \to \Hom_\CC(FX, FY)
\]
are homomorphisms of abelian groups.

A \keyword{complex in $\CC$}, denoted by $X^\bullet$, is a sequence of objects and morphisms:
\begin{cd}
    \cdots \rar["d^{n-1}"]
    & X^n \rar["d^n"]
    & X^{n+1} \rar["d^{n+1}"]
    & \cdots
\end{cd}
with $d^n \circ d^{n-1} = 0$.

Assume $\CC$ and $\CC'$ are abelian.
Have diagram
\begin{cd}
    & \coker d^n \drar[dashed, "b^{n+1}"]
    \\
    X^n \rar["d^n"] \drar[dashed, "a^n"'] 
    & X^{n+1} \rar["d^{n+1}"] \uar
    & X^{n+1}
    \\
    & \ker d^{n+1} \uar[hook]
\end{cd}
The \keyword{$(n+1)$-cohomology} of $X^\bullet$ is
\[
    H^{n+1}(X^\bullet) := \coker a^n = \ker b^{n+1}.
\]

$X^\bullet$ is \keyword{acyclic} at $X^n$ if $H^n(X^\bullet) = 0$.

$X^\bullet$ is \keyword{exact/acyclic} if it is acyclic at $X^n$ for all $n$.

An additive functor $F : \CC \to \CC'$ is \keyword{exact} if it sends SES $0 \to X \to Y \to Z \to 0$ in $\CC$ to SES $0 \to FX \to FY \to FZ \to 0$ in $\CC'$.

\keyword{left exact} in case each $0 \to FX \to FY \to FZ$ is exact.

\keyword{right exact} in case each $FX \to FY \to FZ \to 0$ is exact.

\sepline

Examples.

1. $\CC$ abelian.
Then $\Hom_\CC(Y, -) : \CC \to \Ab$ and $\Hom_\CC(-, Y) : \CC^\op \to \Ab$ are left exact.

2. Consider $R\lMod$ and a right $R$-module $Y_R$.
Then $Y \tensor_R - : R\lMod \to \Ab$ is right exact.

\sepline

\begin{proposition}
    $X$ a topological space, $U \seq X$ open, consider the category $\SAb_X$ of sheaves of abelian groups on $X$.
    The functor $\SAb \to \Ab$, $\FF \mapsto \FF(U)$, is left exact.
\end{proposition}

\begin{proof}
    The inclusion functor $\iota : \SAb \inc \PAb$ is left exact: kernel does not need sheafification.

    The functor $\PAb \to \Ab$, $\FF \mapsto \FF(U)$ is exact by definition.

    Composing an exact functor with a left exact functor gives a left exact functor.
\end{proof}

\sepline

Let $\CC$ be an abelian category.

An object $Y$ is \keyword{projective} if $\Hom_\CC(Y, -)$ is exact.

An object $X$ is \keyword{injective} if $\Hom_\CC(-, Y)$ is exact.

A right $R$-module $Y_R$ is \keyword{flat} if $Y \tensor_R -$ is exact.

\sepline

Let $f : M \to N$ be a continuous map of topological spaces, $\FF$ a sheaf on $M$.

The \keyword{direct image} or \keyword{pushforward} $f_*\FF$ is a sheaf defined by
\[
    f_*\FF(U) := \FF(f^{-1}(U)).
\]
and restriction for $V \seq U$ induced from $r_{f^{-1}(U), f^{-1}(V)}$.

Exercise: This is a sheaf.

\sepline

\begin{proposition}
    \
    \begin{enumerate}[(a)]
        \item Let $F : M \to \mathrm{pt}$, then $f_*\FF = \Gamma(M, \FF) = \FF(M)$.
        \item $i : M \to N$ closed embedding, then have stalk
        \[
            (i_*\FF)_x = \begin{cases}
                \FF_x & x \in M \\
                0 & x \notin M
            \end{cases}
        \]
        Called ``extension by zero.''

        (Exercise: $i : M \inc N$ open embedding; $i_*\FF$ may have nonzero stalk at $x \in N \setminus M$)

        \item $f_* : \SAb_M \to \SAb_N$ is a functor: $(fg)_* = f_* \circ g_*$ and $\id_* = \id$ (not sure right id condition here).
    \end{enumerate}
\end{proposition}

\sepline

4/12/23

Let $f : M \to N$ be a continuous map.

Let $\FF \in \SAb_N$ be a sheaf on $N$.

Define $f^*_p\FF$ as a presheaf
\[
    f^*_p\FF(U) = \FF(f(U)) := \varinjlim_{f(U) \seq V \seq N} \FF(V)
\]
Then take sheafification $f^*\FF = s(f^*_p\FF)$, called the \keyword{inverse image} or \keyword{pullback}.

Exercise: for all $x \in M$, have stalk $(f^*\FF)_x = \FF_{f(x)}$.

\sepline

\begin{proposition}
    Adjunction $f^* \dashv f_*$.
    \[
        \Hom_{\SAb_M}(f^*\FF, \GG) \iso \Hom_{\SAb_N}(\FF, f_*\GG)
    \]
\end{proposition}

\begin{proof}
    As $s \dashv \iota$, we only need to show
    \[
        \Hom_{\PAb_M}(f^*_p\FF, \GG) \iso \Hom_{\PAb_N}(\FF, f_*\GG).
    \]

    Want to establish a functorial morphism $\FF \to f_*f^*_p\FF$.
    
    (Remark: Why should we expect this?
    In the original statement, take $\GG = f^*_p\FF$, then should have $\Hom(f^*_p\FF, f^*_p\FF) \iso \Hom(\FF, f_*f^*_p\FF)$.
    Identity $\id$ on the left should give something on the right, which is the thing we are talking about.)

    Note, for $V \seq N$ open, need to define morphism $\FF(V) \to f_*f^*_p\FF(V)$.
    Have
    \[
        f_*f^*_p\FF(V)
        = (f^*_p\FF)(f^{-1}(V))
        = \varinjlim_{f(f^{-1}(V)) \seq U \seq N} \FF(U)
    \]
    But $V$ is an open neighborhood of $f(f^{-1}(V))$, so there is a canonical morphism $\FF(V) \to \varinjlim \FF(U)$.

    This is compatible with restrictions and gives us a presheaf morphism $i_\FF : \FF \to f_*f^*_p\FF$.

    And this induces the arrow in the statement: given $\psi \in \Hom_{\PAb_M}(f^*_p\FF, \GG)$, send to the composition
    \begin{cd}
        \FF \rar["i_\FF"] & f_*f^*_p\FF \rar["f_*\psi"] \rar & f_*\GG
    \end{cd}

    The other direction uses $f^*_pf_*\GG \to \GG$.

    Exercise: construct this and check it is the desired inverse.
\end{proof}


(Think about adjunction $f^* \dashv f_*$ as a pair of natural transformations $\id \to f_*f^*$ and $f^*f_* \to \id$.)

\begin{proposition}
    \begin{tikzcd}
        \CC \rar[shift left=3pt, "F"] & \DD \lar[shift left=3pt, "G"]
    \end{tikzcd}
    and $F \dashv G$.
    Then $F$ is right exact and $G$ is left exact.
\end{proposition}

\begin{proof}
    We prove the second part, i.e., that $G$ is left exact.
    Given SES
    \begin{cd}
        0 \rar & Y' \rar["f"] & Y \rar["g"] & Y'' \rar & 0
    \end{cd}
    in $\DD$.
    Apply left exact functor $\Hom_\DD(FX, -)$ for all $X \in \CC$.
    Gives a
    \begin{cd}
        0 \rar & \Hom_\DD(FX, Y') \rar & \Hom_\DD(Fx, Y) \rar & \Hom_\DD(Fx, Y'')
    \end{cd}
    This is really the same as 
    \begin{cd}
        0 \rar & \Hom_\CC(X, GY') \rar & \Hom_\CC(X, GY) \rar & \Hom_\CC(X, GY'')
    \end{cd}
    This is true for any $X \in \CC$, which is equivalent to 
    \begin{cd}
        0 \rar & GY' \rar & GY \rar & GY''
    \end{cd}
    being exact.
\end{proof}

\begin{proposition}
    In $\SAb$, $f_*$ is left exact and $f^*$ is exact.
\end{proposition}

\begin{proof}
    By earlier exercise, $f^*$ is exact on stalks.
\end{proof}

\sepline

Direct Images with Compact Support. (III.8.7-10)

Assume all topological spaces are 
\begin{enumerate}
    \item locally compact: for any $x \in X$ there exists an open set $U$ and compact set $K$ such that $x \in U \seq K$;
    \item first-countable: every point has a countable local basis;
    \item (often) Hausdorff.
\end{enumerate}

A morphism between topological spaces is \keyword{proper} if the preimages of compact sets are compact.
(Sort of like a morphism with compact fibers.)

Let $f : X \to Y$ be continuous, $\FF$ a sheaf on $X$, $U \seq Y$ open.
Define the \keyword{direct image with compact support} sheaf $f_!\FF$ on $Y$ by
\[
    f_!\FF(U) = \{s \in \Gamma(f^{-1}(U), \FF) \mid f : \supp(s) \to U \text{ is proper}\}.
\]

Let $s \in \Gamma(V, \GG)$, the \keyword{support} of $s$ is
\[
    \supp(s) := \clo{\{x \in V \mid \eqc{s} \ne 0 \in \GG_x\}}.
\]
where $\Gamma(V, \GG) \to \GG_x$, $s \mapsto \eqc{s}$, takes the germ of $s$ at $x$.
In other words, the closure of the set of points where the germ of $s$ is nonzero.

\begin{theorem}
    \
    \begin{enumerate}[(a)]
        \item $f_!\FF$ is a subsheaf of $f_*\FF$;
        \item $f_! : \SAb_X \to \SAb_Y$ is a left exact functor
    \end{enumerate}
\end{theorem}

\sepline

4/14/23

\begin{proof}
    (a)
    It is clear that $f_!\FF$ is a subpresheaf of $f_*\FF$, since $f_!\FF(U) \seq f_*\FF(U)$ by definition.
    So the main thing to check is that $f_!\FF$ is a sheaf.

    Any set of compatible sections of $f_!\FF$ glue uniquely to a section of $f_*\FF$.
    We want this glued section to be a section of $f_!\FF$, i.e., that it satisfies the relevant condition.

    This comes down to a topological statement.

    Exercise: given a collection of proper maps $V_i \to U_i$, the glued map $\bigcup V_i \to \bigcup U_i$ is proper.

    (b)
    follows from the fact that $f_!$ is a subsheaf?
\end{proof}

\sepline

Consider the special case $f : X \to \mathrm{pt}$.
Then $f_!\FF \approx f_!\FF(\mathrm{pt})$ is the sections $s \in \FF(X)$ such that $\supp(s)$ is compact, denote this by $\Gamma_c(X, \FF)$.

(Remark: category of abelian sheaves over a point is equivalent (isomorphic?) to the category of abelian groups.)

Now consider a single point $y \in Y$.
Have the following diagram:
\begin{cd}
    X \rar["f"] & Y \\
    f^{-1}(y) \uar[hook, "i"] \rar & y \uar[hook]
\end{cd}

\begin{proposition}
    For a point $y \in Y$, we have the stalk
    \[
        (f_!\FF)_y = \Gamma_c(f^{-1}(y), \FF|_{f^{-1}(y)} = i^*\FF).
    \]
\end{proposition}

\begin{proof}
    First want to construct the morphism $\phi : (f_!\FF)_y \to \Gamma_c(f^{-1}(y), \FF|_{f^{-1}(y)})$.

    Let $s \in (f_!\FF)_y$.
    Choose a representative $\tilde{s} \in \Gamma(f^{-1}(U), \FF)$ with $U$ an open neighborhood of $y$ and $\supp(\tilde{s}) \to U$ proper.

    Then $\tilde{s}|_{f^{-1}(y)}$ is in $\Gamma_c(f^{-1}(y), \FF|_{f^{-1}(y)})$, since $\supp(\tilde{s}|_{f^{-1}(y)}) = \supp(\tilde{s}) \cap f^{-1}(y)$.

    Exercise: $\phi(s) := \tilde{s}|_{f^{-1}(y)}$ only depends on $s$.

    Now show $\phi$ is injective.
    Suppose $\phi(s) = 0$.
    Then induced section $tilde{s} = 0$, which says that $\supp(\tilde{s})$ is disjoint from $f^{-1}(y)$.
    In particular, $y \notin f(\supp\tilde{s})$.
    Note that the support is closed and $f$ is proper (also need space is locally compact), so $f(\supp\tilde{s})$ is closed.
    So there is a neighborhood of $y$ where a representative of $s$ is zero, hence $s = 0$.

    Now show $\phi$ is surjective.
    Choose a local basis $\{U_i\}$ around $y$ such that $\bigcap U_i = y$.
    Then $f^{-1}(y) = \bigcap f^{-1}(U_i)$.

    Exercise: locally compact implies $\Gamma_c(f^{-1}(y), \FF|_{f^{-1}(y)}) = \varinjlim A_i$, where
    \[
        A_i = \{t \in \Gamma(f^{-1}(U_i), \FF) \mid \supp t = K \cap f^{-1}(U_i) \text{ for some compact } K \seq X\}.
    \]
    (This is annoying to verify.)

    
\end{proof}

\sepline

Example

1. open immersion $U \inc X$, $\FF$ a sheaf on $U$.
Look at stalk:
\[
    (i_!\FF)_x = \begin{cases}
        \FF_x & x \in U \\
        0 & x \notin U
    \end{cases}
\]
For points in the complement, map is never proper.
Called ``extension by zero.''


2. $j : E \to X$ proper (in particular, closed embedding).
Then $j_!\GG = j_*\GG$.
The closed image of a compact set is also compact, so every map is proper.

\sepline

Derived Categories via Localizations.
(More classical approach)

Let $f : K^\bullet \to L^\bullet$ be a morphism of complexes in an abelian category $\AA$.

Say $f$ is a \keyword{quasi-isomorphism} if the induced morphisms $H^n(f) : H^n(K^\bullet) \to H^n(L^\bullet)$ are all isomorphisms.

Define $\Kom(\AA)$ to be the category of complexes in $\AA$.

There exists a category $D(\AA)$ and a functor $Q : \Kom(\AA) \to D(A)$ such that
\begin{enumerate}
    \item $Q(f)$ is an isomorphism for any quasi-isomorphism $f$;
    \item $Q$ is universal with respect to all such categories:
    \begin{cd}
        \Kom(\AA) \drar \ar[rr] && \DD' \\
        & D(\AA) \urar[dashed, "\exists!"']
    \end{cd}
\end{enumerate}
Call $D(\AA)$ the \keyword{derived category} of $\AA$.

\sepline

4/17/23

Remark: Consider full subcategory $\Kom^+(\AA)$ of $\Kom(\AA)$ consisting of all complexes $K^\bullet$ with $K^i = 0$ for $i \ll 0$, i.e., $K^\bullet$ is bounded from below.
Also have $\Kom^-(\AA)$ of complexes bounded above, and $\Kom^b(\AA)$ of complexes bounded from both sides.
These induce bounded derived categories $D^+(\AA)$, $D^-(\AA)$, and $D^b(\AA)$, respectively, which are full subcategories of $D(\AA)$.


\sepline

Construction of $D(\AA)$.

Think about it like ``localization of noncommutative rings.''

A class of morphisms $S \seq \Mor\BB$ is said to be \keyword{localizing} if
\begin{enumerate}[(a)]
    \item $S$ is closed under composition: $id_X \in S$ for all $X \in \BB$ and $s \circ t \in S$ for all $s, t \in S$ for which the composition is defined;
    \item extension: for all $f \in \Mor\BB$ and $s \in S$, there exists $g \in \Mor\BB$ and $t \in S$ such that the following diagram commutes:
    \begin{center}
        \begin{tikzcd}
            W \rar[dashed, "g"'] \dar[dashed, "t"'] & Z \dar["s"] \\
            X \rar["f"'] & Y
        \end{tikzcd}
        \hspace*{1cm}
        or
        \hspace*{1cm}
        \begin{tikzcd}
            W & Z \lar[dashed, "g"] \\
            X \uar[dashed, "t"] & Y \lar["f"] \uar["s"']
        \end{tikzcd}
    \end{center}
    \item $f, g \in \Mor(X, Y)$, the existence of $s \in S$ with $sf = sg$ is equivalence to the existence of $t \in S$ with $ft = gt$.
\end{enumerate}

The \keyword{localization} of $\BB$ at $S$ is the category $\BB[S^{-1}]$ with $\Ob\BB[S^{-1}] = \Ob\BB$.
A morphism $X \to Y$ in $\BB[S^{-1}]$ is the equivalence class of \keyword{roofs} of the form
\begin{cd}
    & X' \dlar["s"'] \drar["f"] \\
    X && Y
\end{cd}
with $s \in S$ and $f \in \Mor\BB$.
Two roofs are equivalence if there exists a third roof on top, which makes the following diagram commute:
\begin{cd}
    & X''' \dlar[dashed, "r"'] \drar[dashed, "h"] \\
    X' \dar["s"'] \ar[drr, "f" near start] && X'' \ar[dll, "t"' near start] \dar["g"] \\
    X && Y
\end{cd}
The identity $\id_X$ is the class of
\begin{cd}
    & X \dlar["\id_X"'] \drar["\id_X"] \\
    X && X
\end{cd}

The composition of roofs $(s, f)$ and $(t, g)$ is the class of $(st', gf')$ where
\begin{cd}
    && X'' \dlar[dashed, "t'"'] \drar[dashed, "f'"] \\
    & X' \dlar["s"'] \drar["f"] && Y' \dlar["t"'] \drar["g"] \\
    X && Y && Z
\end{cd}
Extension condition gives existence of square, so 
\[
    gt^{-1}fs^{-1}
        = gf't'^{-1}s^{-1}
        = (gf')(st')^{-1}.  
\]

\begin{lemma}
    \
    \begin{enumerate}[(a)]
        \item This is an equivalence relation on roofs,
        \item composition is well-defined.
    \end{enumerate}
\end{lemma}

\begin{proof}
    (a) Reflexive and symmetric easy.

    Suppose $(s, f) \sim (t, g)$ adn $(t, g) \sim (u, e)$.
    So we have
    \begin{cd}[column sep=small, row sep=small]
        && {Z'} && {Z''} \\
        & {X'} && {X''} && {X'''} \\
        \\
        X &&&&&& Y
        \arrow["s"', from=2-2, to=4-1]
        \arrow["f"'{pos=0.6}, from=2-2, to=4-7]
        \arrow["t"'{pos=0.6}, from=2-4, to=4-1]
        \arrow["e", from=2-6, to=4-7]
        \arrow["g"{pos=0.6}, from=2-4, to=4-7]
        \arrow["u"{pos=0.6}, from=2-6, to=4-1]
        \arrow["r"', from=1-3, to=2-2]
        \arrow["h", from=1-3, to=2-4]
        \arrow["p"', from=1-5, to=2-4]
        \arrow["i", from=1-5, to=2-6]
    \end{cd}
    First, consider
    \begin{cd}
        Z' \dar["sr"'] & W \lar[dashed, "v"'] \dar[dashed, "k"]\\
        X & Z'' \lar["tp"]
    \end{cd}
    Then $srv = tpk$.
    By first diagram, have $srv = thv$.
    So by (c), there exists $w : Z''' \to W$ in $S$ such that $hvw = pkw$.
    Now
    \begin{cd}
        & Z''' \dlar[dashed, "rvw"'] \drar[dashed, "ikw"] \\
        X' \dar["s"'] \ar[drr, "f" near start] && X''' \ar[dll, "u"' near start] \dar["e"] \\
        X && Y
    \end{cd}
\end{proof}

Construct morphism $F : \BB \to \BB[S^{-1}]$ which is the identity on objects and sends a morphism $f \in \Mor_\BB(X, Y)$ to the equivalence class of the roof
\begin{cd}
    & X \dlar["\id_X"'] \drar["f"] \\
    X && Y
\end{cd}

This satisfies the following universal property: If $T : \BB \to \DD$ is functor such that $T(s)$ is an isomorphism for all $s \in S$, then there exists a unique factorization
\begin{cd}
    \BB \ar[rr, "T"] \drar["F"'] && \DD \\
    & \BB[S^{-1}] \urar[dashed, "\exists!"']
\end{cd}

\sepline

4/19/23

\begin{proposition}
    Let $T : \BB \to \DD$ be a functor such that $T(s)$ is an isomorphism for any $s$ in a localizing class $S$ of $\BB$.
    Then $T$ uniquely factors through $\BB[S^{-1}]$:
    \begin{cd}
        \BB \ar[rr, "T"] \drar["F"'] && \DD \\
        & \BB[S^{-1}] \urar[dashed, "\exists!G"']
    \end{cd}
\end{proposition}

\begin{proof}
    To construct $G : \BB[S^{-1}] \to \DD$ such that $T = G \circ F$, must have
    \[
        G(X \in \BB[S^{-1}]) = T(X \in \BB)
    \]
    and
    \[
        G([s, f]) = T(f) \circ T(s)^{-1}.
    \]
    Exercise: well-defined and unique.
\end{proof}

\sepline

Issue: The quasi-isomorphisms in $\Kom(\AA)$ do not form a localizing class.

Fix $n$.
For $K^\bullet = (K^\bullet, d_K^\bullet)$, define a complex $K[n]^\bullet
$ by
\[
    (K[n])^i := K^{n + i}
    \isp{and}
    d_{K[n]} := (-1)^n d_K.
\]
(A shift to the left by $n$.)

\begin{cd}
    \cdots \rar K^i \rar["d_K^i"] & K^{i+1} \rar["d_K^{i+1}"] & \cdots
\end{cd}

For $f : \to K^\bullet \to L^\bullet$, define $f[n] : K[n]^\bullet \to L[n]^\bullet$ to coincide with $f$ componentwise.

This describes a \keyword{translation functor} $T^n : \Kom(\AA) \to \Kom(\AA)$, which is an autoequivalence.


Let $f : K^\bullet \to L^\bullet$ be a morphism of complexes, the \keyword{cone} of $f$ is the complex $c(f)^\bullet$ with
\[
    c(f)^i = K[1]^i \oplus L^i
    \isp{and}
    d_{c(f)}(K^{i+1}, L_i) = (-d_K K^{i+1}, f(K^{i+1}) + d_L(L^i))
\]
\begin{cd}[column sep=large]
    K^{i+1} \rar["-d_K"] \dar[draw=none, "\oplus" description] \drar["f^{i+1}"]
    & K^{i+1} \dar[draw=none, "\oplus" description] \\
    L_i \rar["d_L"] & L^{i+1}
\end{cd}
Check $d_{c(f)}^2 = 0$.

\sepline

Example: $f$ is a morphism of a ``$0$-complex,'' i.e., $f : K^0 \to L^0$ where $K^0$ and $L^0$ have at most one nonzero entry, in the $0$th position.
Then $c(f)$ is the complex
\begin{cd}
    \cdots \rar & 0 \rar & K^0 \rar["f"] & L^0 \rar & 0 \rar & \cdots \\
    & & \deg -1 \uar & \deg 0 \uar
\end{cd}

In particular, $H^{-1}(c(f)) = \ker f$ and $H^0(c(f)) = \coker f$.

\sepline

The \keyword{cylinder} $\Cyl(f)$ is a complex
\[
    \Cyl(f)^\bullet = K^\bullet \oplus K[1]^\bullet \oplus L^\bullet
\]
and $d_{Cyl(f)}^i$ given by
\begin{cd}[column sep=large]
    K^i \rar["d_K"] & K^{i+1} \\
    K^{i+1} \urar["-\id"] \rar["-d_K"] \drar["f"] & K^{i+2} \\
    L_i \rar["d_L"] & L^{i+1}
\end{cd}

\begin{lemma}
    The following diagram commutes with exact rows:
    \begin{cd}
        & 0 \rar 
        & L^\bullet \rar \dar["\alpha"]
        & c(f) \rar \dar[equal]
        & K[1]^\bullet \rar & 0
        \\
        0 \rar 
        & K^\bullet \rar \dar[equal] 
        & \Cyl(f) \rar \dar["\beta"]
        & c(f) \rar
        & 0
        \\
        & K^\bullet \rar["f"]
        & L^\bullet
    \end{cd}
    It's functorial in $f$.
    The morphisms $\alpha$ and $\beta$ are quasi-isomorphisms: $\beta\alpha = \id_L$ and $\alpha\beta$ is homotopic to $\id_{\Cyl(f)}$.
\end{lemma}

Say $f, g : K^\bullet \to L^\bullet$ are \keyword{homotopic}, written $f \sim g$ if there exists $k$ such that $f - g = kd + dk$ where
\begin{cd}
    K^{i-1} \rar 
    & K^i \rar \dlar["k^i"'] \dar["f - g"'] 
    & K^{i+1} \dlar["k^{i+1}"']
    \\
    L^{i-1} \rar
    & L^i \rar
    & L^{i+1}
\end{cd}

\begin{proof}
    $\alpha$ is the ``inclusion.''

    And $\beta$ maps
    \begin{cd}
        (K^i, K^{i+1}, L^i) \rar["d_{\Cyl(f)}"] \dar["\beta_i"]
        & (\quad) \dar["\beta^{i+1}"]
        \\
        (f(K^i) + L^i) \rar["d_L"]
        & (\quad)
    \end{cd}
    Consider map
    \begin{cd}[row sep=tiny]
        \Cyl(f)^i \rar["k^i"] & \Cyl(f)^{i-1} \\
        (K^i, K^{i+1}, L^i) \rar[mapsto] & (0, K^i, 0)
    \end{cd}
    Then $\alpha\beta = \id + dk + kd$.
\end{proof}

\sepline

4/21/23

Let $\AA$ be an abelian category.

The \keyword{homotopy category} $K(\AA)$ is
\begin{itemize}
    \item $\Ob K(\AA) = \Ob \Kom(\AA)$;
    \item $\Mor K(\AA) = \Mor\Kom(\AA) / \{\text{homotopy equivalence}\}$
\end{itemize}

Recall that homotopy equivalence on morphisms induces the same morphisms on cohomology, so it makes sense to talk about quasi-isomorphisms inside $K(\AA)$.

\begin{theorem}[III.4.4]
    The class of quasi-isomorphisms in $K(\AA)$ is localizing.
\end{theorem}

Will give a more conceptual proof later, based on two more results:
\begin{itemize}
    \item $K(\AA)$ is a triangulated category;
    \item quasi-isomorphisms are ``obtained by cohomological functors.''
\end{itemize}

Taste of proof of localization:

obviously closed under composition

Want to complete the following diagram:
\begin{cd}
    N^\bullet \rar[dashed, "k", "\text{q-iso}"'] \dar[dashed, "h"']
    & M^\bullet \dar["g"]
    \\
    K^\bullet \rar["f", "\text{q-iso}"']
    & L^\bullet
\end{cd}
This comes from the following diagram:
\begin{cd}
    c(\pi g)^\bullet[-1] \rar["k"] \dar[dashed, "\exists h"]
    & M^\bullet \rar["\pi g"] \dar["g"]
    & c(f)^\bullet \dar[equals] \rar
    & c(\pi g)^\bullet \dar["{h[1]}"]
    \\
    K^\bullet \rar["f"']
    & L^\bullet \rar["\pi"']
    & c(f)^\bullet \rar
    & K^\bullet[1]
\end{cd}
This diagram commutes up to homotopy.

$c(f)$ is cyclic (no cohomology), will imply something about $M^\bullet$.

Bottom row is common thing.

\sepline

Can now define derived category as $D(\AA) := K(A)[\{\text{q-iso}\}^{-1}]$.

\sepline

$D(\AA)$ is an additive category.

For two morphisms
\begin{center}
    \begin{tikzcd}
        & Z \dlar["s"'] \drar["f"] \\
        X && Y
    \end{tikzcd}
    \hspace*{1cm}
    \begin{tikzcd}
        & Z' \dlar["s'"'] \drar["f'"] \\
        X && Y
    \end{tikzcd}
\end{center}
take
\begin{cd}
    U \dar[dashed, "r"'] \rar[dashed, "r'"] 
    & Z' \dar["s'"]
    \\
    Z \rar["s"]
    & X
\end{cd}
Both $r$ and $r'$ are quasi-isomorphisms.
\begin{cd}
    & U \dlar["r"'] \drar["r'"]
    \\
    Z \dar["s"'] \ar[drr, "f" near start] 
    && Z' \ar[dll, "s'"' near start] \dar["f'"] \\
    X && Y
\end{cd}
Take
\[
    (s, f) + (s', f') = (sr = s'r', fr + f'r')
\]

Exercise: this is an additive structure.

\sepline

Let $\DD$ be an additive category.

A \keyword{triangle} in $\DD$ is a sequence
\begin{cd}
    X \rar["u"] & Y \rar["v"] & Z \rar["w"] & X[1]
\end{cd}
and a morphism of triangles is a diagram
\begin{cd}
    X \rar \dar["f"] 
    & Y \rar \dar["g"] 
    & Z \rar \dar["k"] 
    & X[1] \dar["{f[1]}"]
    \\
    X' \rar
    & Y' \rar
    & Z' \rar
    & X'[1]
\end{cd}

A \keyword{$\Delta$-category structure} on $\DD$ consists of the data
\begin{enumerate}[(a)]
    \item Additive autoequivalence $T : \DD \to \DD$ called the translation, denote ($X[n] := T^nX$ and $f[n] := T^nf$);
    \item a class of \textit{distinguished triangles} 
\end{enumerate}
satisfying
\begin{enumerate}[TR1.]
    \item \begin{enumerate}[(1)]
        \item Every triangle of the form
        \begin{cd}
            X \rar["\id"] & X \rar & 0 \rar & X[1]
        \end{cd}
        is distinguished,
        \item any triangle isomorphic to a distinguished triangle is distinguished,
        \item any morphism $u : X \to Y$ can be completed to a distinguished triangle.
    \end{enumerate}
    \item A triangle
    \begin{cd}
        X \rar["u"] & Y \rar["v"] & Z \rar["w"] & X[1]
    \end{cd}
    is distinguished if and only if
    \begin{cd}
        Y \rar["v"] & Z \rar["w"] & X[1] \rar["{-u[1]}"] & Y[1]
    \end{cd}
    is distinguished.
    \item Assume we have two distinguished triangles and morphisms $f$ and $g$, then there exists $h$ (not necessarily unique) which completes the following diagram:
    \begin{cd}
        X \rar \dar["f"] 
        & Y \rar \dar["g"]
        & Z \rar \dar[dashed, "\exists h"]
        & X[1] \dar["{f[1]}"]
        \\
        X' \rar & Y' \rar & Z' \rar & X[1]
    \end{cd}
    \item Starting with commutative square
    \begin{cd}
        X \rar["f"] \dar["g"']
        & Y \dar["w"]
        \\
        X' \rar["h"]
        & Y'
    \end{cd}
    Then there is a (unique?) way to complete it to a diagram:
    \begin{cd}
        X \rar["f"] \dar["g"']
        & Y \dar["w"] \rar
        & Z \dar
        \\
        X' \rar["h"] \dar
        & Y' \rar \dar
        & Z' \dar
        \\
        X'' \rar 
        & Y'' \rar
        & Z''
    \end{cd}
\end{enumerate}

(Unknown if TR4 follows from the previous axioms)

generalized by stable $\infty$-categories.

\sepline

Octahedron:
Given three triangles of the form
\begin{cd}
    X \rar["f"] & Y \rar["h"] & Y/X \rar & X[1] \\
    Y \rar["g"] & Z \rar["k"] & Z/Y \rar & Y[1] \\
    X \rar["g \circ f"] & Z \rar["\ell"] & Z/X \rar & X[1]
\end{cd}

\sepline

4/24/23

\begin{proposition}
    Let $\DD$ be a triangulated category, and
    \begin{cd}
        X \rar["u"] & Y \rar["v"] & Z \rar["w"] & X[1]
    \end{cd}
    Then there exists $U \in \Ob\DD$ such that the following sequences are exact
    \begin{cd}
        \cdots \rar 
        & \Hom(U, X[i]) \rar["{(u[i])_*}"]
        & \Hom(U, Y[i]) \rar["{(v[i])_*}"]
        & \Hom(U, Z[i]) 
        \\
        && \rar["{(w[i])_*}"]
        & \Hom(U, X[i+1]) \rar
        & \cdots
    \end{cd}
    and
    \begin{cd}
        \cdots \rar
        & \Hom(X[i+1]) \rar["{(w[i])^*}"]
        & \Hom(Z[i], U) \rar
        & \cdots
    \end{cd}
\end{proposition}

\begin{proof}
    By TR2, we only need to check exactness at $\Hom(U, Y)$ (i.e., we can shift to get Lemma applies elsewhere).

    First, we show that $vu = 0$.
    By TR3 (applied to TR1 and assumption)
    \begin{cd}
        X \rar["\id"] \dar["\id"]
        & X \rar["0"] \dar["u"]
        & 0 \rar \dar[dashed, "h"]
        & X[1] \dar["\id"]
        \\
        X \rar["u"]
        & Y \rar["v"]
        & Z \rar["w"]
        & X[1]
    \end{cd}
    The only option for $h$ is $0$ so $0 = vu$.
    Thus $v_*u_* = 0$ so the hom sequence is a complex.

    Now, we show exactness.
    Let $f : U \to Y$ be such that $vf = 0$.
    ($f \in \ker v_*$, show $f \in \im u_*$.)
    Need to find $g : U \to X$ such that $f = ug$.
    By TR2 and TR3
    \begin{cd}
        U \rar["\id"] \dar[dashed, "g"]
        & U \rar \dar["f"]
        & 0 \rar \dar
        & U[1] \dar[dashed, "{g[1]}"]
        \\
        X \rar["u"]
        & Y \rar["v"]
        & Z \rar
        & X[1]
    \end{cd}
    So $f = ug$, so exact at $\Hom(U, Y)$ so exact.
\end{proof}

\begin{corollary}
    If $f$ and $g$ uin TR3 are isomorphisms, then $h$ is also an isomorphism.
    In particular, distinguished triangles in TR1c is unique up to (non-unique) isomorphism.
\end{corollary}

\begin{proof}
    \begin{cd}
        X \rar \dar["f"]
        & Y \rar \dar["g"]
        & Z \dar[dashed, "h"]
        \\
        X' \rar
        & Y' \rar
        & Z'
    \end{cd}
\end{proof}
Apply $\Hom(Z', -)$ to get
\begin{cd}[column sep=tiny]
    \Hom(Z', X) \rar \dar["f_*", "\iso"']
    & \Hom(Z', Y) \rar \dar["g_*", "\iso"']
    & \Hom(Z', Z) \rar \dar[dashed, "h_*"]
    & \Hom(Z', X[1]) \rar \dar["{f[1]_*}", "\iso"']
    & \Hom(Z', Y[1]) \dar["{g[1]_*}", "\iso"']
    \\
    \Hom(Z', X') \rar
    & \Hom(Z', Y') \rar
    & \Hom(Z', Z') \rar
    & \Hom(Z', X'[1]) \rar
    & \Hom(Z', Y'[1])
\end{cd}
So by the five lemma, $h_*$ is an isomorphism of abelian groups, so there exists $\phi : Z' \to Z$ such that $h\phi = \id_{Z'}$.

By applying $\Hom(-, Z)$, get left inverse of $h$, which implies $h$ is an isomorphism.

\begin{corollary}
    Diagram
    \begin{cd}
        & X \rar["u"] \dar[dashed, "f"]
        & Y \rar["v"] \dar["g"]
        & Z \rar["w"] \dar[dashed, "h"]
        & X[1] \dar["{f[1]}"]
        \\
        Z'[-1] \rar
        & X' \rar["u'"]
        & Y' \rar["v'"]
        & Z' \rar["w'"]
        & X'[1]
    \end{cd}
    There exists $f, h$ if $v'gu = 0$.

    If, moreover, $\Hom(X, Z'[-1]) = 0$, then this morphism of distinguished triangles is unique.
\end{corollary}

\sepline

By TR1c and corollary, any $u : X \to Y$ determines an object $C(u)$ (unique up to isomorphism) in distinguished triangle
\begin{cd}
    X \rar["u"] & Y \rar & C(u) \rar & X[1]
\end{cd}
This object is called the \keyword{cone} of $u$.

Recall TR4:
\begin{cd}
    X \rar["u"] \dar[equals]
    & Y \rar \dar["v"]
    & C(u) \dar
    \\
    X \rar["vy"] \dar
    & Z \rar \dar
    & C(vu) \dar
    \\
    0 \rar
    & C(v) \rar
    & C(v)
\end{cd}
third column is important distinguished triangle, like an isomorphism theorem
\begin{cd}
    0 \rar & Y/X \rar & Z/X \rar & Z/Y \rar & 0
\end{cd}






\sepline

A triangle in $\Kom(\AA)/K(A)$ is distinguished if it is 

\sepline

4/26/23

Recall $K(\AA)$ with autoequivalence $T = [1]$.

A distinguished triangle is a triangle isomorphic to
\begin{cd}
    K^\bullet \rar["\bar{f}"] & \Cyl(f) \rar["\pi"] & c(f) \rar["\delta"] & K[1]^\bullet
\end{cd}
for some $f : K^\bullet \to L^\bullet$.

\begin{theorem}
    This gives $K(\AA)$ the structure of a triangulated category.
\end{theorem}

\begin{proof}
    Must check TR1-4.

    Only show TR1.

    Must construct
    \begin{cd}
        X \rar["\id"] \dar[equals]
        & X \rar \dar[equals]
        & 0 \rar \dar
        & X[1] \dar[equals]
        \\
        X \rar["\id"]
        & X \rar
        & c(\id) \rar
        & X[1]
    \end{cd}
    This is to show $0 : c(\id) \to c(\id)$ is homotopic to $\id : c(\id) \to c(\id)$.

    Note: $c(\id) = X[1] \oplus X$ by definition.
    The differential of this cone is
    \begin{cd}
        X^{i+1} \rar["-d^{i+1}"] \drar["\id"]
        & X^{i+2}
        \\
        X^i \rar["d^i"']
        & X^{i+1}
    \end{cd}
    Write
    \[
        h = \mat{X^{i+1} && X^i} \mat{0 & 0 \\ \id & 0}  = \mat{X^i & 0}
    \]
    so
    \[
        \id_{c(\id)} = h \circ d + d \circ h
    \]
    Then
    \begin{cd}
        & X^1 \oplus X^0 \dlar["h"'] \dar["\id"] \rar 
        & X^2 \oplus X^1 \dlar["h"]
        \\X
        X^0 \oplus X^{-1} \rar
        & X^1 \oplus X^0
    \end{cd}
    with
    \begin{cd}
        & (a, b) \dlar["h"'] \dar["\id"] \rar 
        & (-da, a + db) \dlar["h"]
        \\X
        (b, 0) \rar
        & (-db, b)
    \end{cd}
\end{proof}

\sepline

Let $\DD$ be a triangulated category and $\AA$ an abelian category.

A functor $H : \DD \to \AA$ is \keyword{cohomological} if its additive and for any distinguished triangle
\begin{cd}
    X \rar["u"] & Y \rar["v"] & Z \rar["w"] & X[1]
\end{cd}
in $\DD$ the sequence
\begin{cd}
    H(X) \rar["H(u)"] & H(Y) \rar["H(v)"] & H(Z)
\end{cd}
in $\AA$ is exact.

Then TR2 gives us a long exact sequence
\begin{cd}
    \cdots \rar & H(X[i]) \rar & H(Y[i]) \rar & H(Z[i]) \rar & H(X[i+1]) \rar & \cdots
\end{cd}

\sepline

Examples

1. $\DD = K(\AA) \to \AA$, $C^\bullet \mapsto H^0(C^\bullet)$.

2. $\DD \to \Ab$, $X \mapsto \Hom_\DD(U, X)$ for some fixed $U \in \DD$.

\begin{proposition}
    Let $H : \DD \to \AA$ be a cohomological functor.
    Consider $\SS$, the class of morphisms $s \in \Hom\DD$ such that $H^i(s) := H(s[i])$ is an isomorphism in $\AA$ for any $i \in \Z$.
    Then $\SS$ is a localizing class (a-c).
    Moreover, $\SS$ is compatible with $\DD$, i.e.,
    \begin{enumerate}[(d)]
        \item $s \in \SS$ iff $T(s) \in \SS$;
        \item In TR3, if $f, g \in \SS$ then there exists $h \in \SS$ completing the diagram.
    \end{enumerate}
\end{proposition}

Example: If $s : C^\bullet \to D^\bullet$, then $s[i]  : C^\bullet[i] \to D^\bullet[i]$.
Need $H(s[i]) : H^i(C^\bullet) \to H^i(D^\bullet)$ to be an isomorphism.
Then $\SS$ is the quasi-isomorphisms.

\begin{proof}
    (a) ok

    (b) Let $f : X \to Y$ and $s : Z \to Y$.
    Take distinguished triangle
    \begin{cd}
        & X \dar["f"]
        \\
        Z \rar["s"]
        & Y \rar["u"]
        & C \rar
        & Z[1]
    \end{cd}
    Then make square
    \begin{cd}
        & X \dar["f"] \rar["uf"]
        & C \dar[equals]
        \\
        Z \rar["s"]
        & Y \rar["u"]
        & C \rar
        & Z[1]
    \end{cd}
    Complete top row to a triangle
    \begin{cd}
        W \rar["t"] \dar["g"']
        & X \rar["uf"] \dar["f"]
        & C \rar \dar[equals]
        & W[1] \dar["{g[1]}"]
        \\
        Z \rar["s"]
        & Y \rar["u"]
        & C \rar
        & Z[1]
    \end{cd}
    with $t \in \SS$
    Then 
    \begin{cd}
        H(Z[i]) \rar["\iso"] & H(Y[i]) \rar & H(C[i]) = 0
    \end{cd}

    (c) Consider $f, g \in \Mor(X, Y)$.
    If there exists $s : Y \to Y'$ such that $s(f - g) = 0$, consider distinguished triangle
    \begin{cd}
        Z \rar["u"] & Y \rar["s"] & Y' \rar["\delta"] & Z[1]
    \end{cd}
    Then
    \begin{cd}[row sep=tiny]
        \Hom(X, Z) \rar["u_*"] & \Hom(X, Y) \rar["s_*"] & \Hom(X, Y')
        \\
        v \rar[mapsto] & f - g \rar[mapsto] & 0
    \end{cd}
    There exists $v : X \to Z$ such that $f - g = uv$.

    Consider distinguished triangle
    \begin{cd}
        X' \rar["t"] & X \rar["v"] & Z \rar & X'[1]
    \end{cd}
    Then $vt = 0$ implies $(f - g)t = uvt = 0$.

    Hve $s \in \SS$, implies $H^*(Z) = 0$ implies $t \in \SS$.

    (d) ok

    (e) consequence of 5-Lemma.
\end{proof}

\sepline

\begin{theorem}
    Let $\DD$ be a triangulated category and $\SS$ a localizing class compatible with $\DD$.
    Then $\DD[S^{-1}]$ is a triangulated category with a natural translation $T$, distinguished triangles defined to be the triangles isomorphic to the images of distinguished triangles in $\DD$ under the canonical functor $\DD \to \DD[S^{-1}]$.
\end{theorem}

Example: $D(\AA) = K(\AA)[\{\text{q-iso}\}^{-1}]$ is a triangulated category.

\sepline

4/28/23

\begin{proof}
    Only need to show how to complete $a : X \to Y$ in $\DD[S^{-1}]$ to a distinguished triangle.
    By definition, $a$ is represented by a roof
    \begin{cd}
        & Y' \dlar["s"'] \drar["f"] \\
        X && Y
    \end{cd}
    with $s \in S$.
    Then there exists
    \begin{cd}
        & Y' \dlar["s"'] \drar["f"]
        \\
        X \drar[dashed, "f'"']
        && Y \dlar[dashed, "s'"]
        \\
        & Y''
    \end{cd}
    In $\DD$, take distinguished triangle
    \begin{cd}
        X \rar["f'"] & Y'' \rar["g"] & Z \rar["h"] & X[1]
    \end{cd}
    This gives a distinguished triangle in $\DD[S^{-1}]$
    \begin{cd}
        X \rar["a"]
        & Y \ar[rr, "gs'"] \drar["s"']
        && Z \rar["h"]
        & X[1]
        \\
        && Z'' \urar["g"']
    \end{cd}

    TR3 \& TR4 [Stacks Project, Prop 13.5.5]
\end{proof}


\begin{corollary}
    $D(\AA)$ is a triangulated category.
\end{corollary}

\sepline

\begin{proposition}
    Consider the functor $Q : \AA \to D(\AA)$ sending an object $E$ to the complex
    \begin{cd}[row sep=small]
        \cdots \rar & 0 \rar & E \rar & 0 \rar & \cdots \\
        && \deg 0 \uar
    \end{cd}
    and sending morphisms in the obvious unique way.
    This yields an equivalence of $\AA$ awith the subcategory of $D(\AA)$ formed bu $H^0$-complexes (i.e., complexes $K^\bullet$ such that $H^i(K^\bullet) = 0$ for $i \neq 0$).
\end{proposition}

\begin{proof}
    Have sequence
    \begin{cd}
        \AA \rar & K(\AA) \rar & D(\AA)
    \end{cd}
    First functor is fully faithful.

    Let $X, Y \in \AA$.
    Need to show that $\Hom_\AA(E, F) = \Hom_{K(A)}(E, F)$ is equal (isomorphic) to $\Hom_{D(\AA)}(QE, QF)$.
    Take map
    \begin{cd}[row sep=tiny]
        \Hom_\AA(E, F) \rar["a"] & \Hom_{D(\AA)}(E, F) \\
        f \rar[mapsto] & {[(\id_X, f)]}
    \end{cd}
    The inverse $b$ is induced by $H^0 : D(\AA) \to A$.
    By definition $b \circ a = \id$.

    To check $a \circ b = \id$, say $\tilde{f} : X \to F$ (between $0$-complexes) in $D(\AA)$ is represented by
    \begin{cd}
        & Z \dlar["s"'] \drar["f"] \\
        X && Y
    \end{cd}
    As $s$ is a quasi-isomorphism, $b(\tilde{f})$ is\
    \begin{cd}
        & X \dlar["\id_X"'] \drar["g"] \\
        X && Y
    \end{cd}
    where $g = H^0(f) \circ H^0(s)^{-1}$
        
\end{proof}

copy from notebook

\sepline

5/1/23

copy

\sepline

5/3/23

Natural morphism $K^+(I) \to D^+(\AA)$

If there exists enough injectives in $\AA$, then this is an equivalence.

Concretely, for any $C^\bullet \in D^+(\AA)$ there exists a quasi-isomorphism $C^\bullet \to I^\bullet$ with $I^\bullet \in K^+(I)$.

In the following, we will not use ``injective'' property.
Will only use: for all $C \in \AA$ there exists $I \in \mathcal{I}$ with $C \inc I$.

suppose we have a complex
\begin{cd}
    \cdots \rar 
    & 0 \rar
    & C^0 \rar["d_C^0"] \dar["t^0"]
    & C^1 \rar["d_C^1"] \dar["t^1"]
    & C^2 \rar \dar["t^2"]
    & \cdots
    \\
    & 0 \rar
    & I^0 \rar["d_I^0"]
    & I^1 \rar["d_I^1"]
    & I^2 \rar
    & \cdots
\end{cd}
with $t^i$ quasi-isomorphisms and $I^i \in \mathcal{I}$.

A \keyword{cocartesian product} (fibered coproduct) is the completion of the following diagram:
\begin{cd}
    A \rar["f"] \dar["g"']
    & B \dar[dashed]
    \\
    C \rar[dashed]
    & B \sqcup_A C
\end{cd}
In abelian category,
\begin{cd}
    A \rar["{(f, g)}"] & B \oplus C \rar & B \sqcup_A C \rar & 0
\end{cd}

Now take
\begin{cd}
    C^0 \rar \dar[hook, "t^0"]
    & C^1 \dar \drar[dashed, "t^1"]
    \\
    I^0 \rar \ar[rr, dashed, bend right, "d_I^0"']
    & I^0 \sqcup_{C^0} C^1 \rar[hook]
    & I^1
\end{cd}
Can check $t^0 : \ker d_I^0 \to \ker d_C^0$ is an isomorphism---basically by definition of $\sqcup$.

Next, consider
\begin{cd}
    & C^1 \dlar["t^1"'] \dar \rar
    & C^2 \dar \drar[dashed, "t^2"]
    \\
    I^1 \rar \ar[rrr, dashed, bend right=15, "d_I^1"']
    & \coker d_I^0 \rar
    & C^2 \sqcup_{C^1} \coker d_I^0 \rar[hook]
    & I^2
\end{cd}
Can check this is quasi-isomorphism at degree $1$.

All the rest will be essentially the same story.

Details in III.5.25.


\sepline

Derived Functors.

Let $F : \AA \to \BB$ be an additive functor between abelian categories.
This induces a functor $\Kom(\AA) \to \Kom(\BB)$, by applying $F$ termwise.
This clearly preserves homotopy, so it gives a well-defined functor $K(F) : K(\AA) \to K(\BB)$.

Issue: $K(F)$ preserves quasi-isomorphisms only when $F$ is exact.
But we want to define functors $D(\AA) \to D(\BB)$ for left/right exact functors $F$.

Assume: $F$ is left exact and we work with $\Kom^+$, $K^+$, $D^+$.

Recall functor $Q_A : K^+(\AA) \to D^+(\AA)$.

\begin{theorem}
    There exists a \keyword{derived functor} $RF : D^+(\AA) \to D^+(\BB)$, together with a natural transformation
    \begin{cd}
        Q_B \circ K^+(F) \rar[Rightarrow, "\eps_F"] & RF \circ Q_A
    \end{cd}
    such that
    \begin{enumerate}[(1)]
        \item $RF$ is exact: $T_B \circ RF = RF \circ T_A$ and it sends distinguished triangles to distinguished triangles;
        \item universal property: given $G : D^+(\AA) \to D^+(\BB)$ and $\eps_G : Q_B \circ K^+(F) \Rightarrow G \circ Q_A$, then there exists a unique $\eta : RF \Rightarrow G$ such that $\eps_G = \eta \circ \eps_F$.
    \end{enumerate}
\end{theorem}


A class of objects $R$ in $\AA$ is \keyword{adapted} to $F$ if
\begin{enumerate}[(1)]
    \item $R$ is closed under finite direct sum.
    \item For every $X^\bullet \in \Kom^+(\AA)$ acyclic complex of objects in $R$, we have $F(X^\bullet)$ is also acyclic.
    \item There exists enough objects in $R$ such that for any $M \in \AA$, there exists an injection $M \inc B$ for some $B \in R$.
\end{enumerate}

View $R$ as an additive full subcategory of $\AA$, us convention $\Kom^+(R)$ and $K^+(R)$ as subcategories of $\Kom^+(\AA)$ and $K^+(\AA)$, respectively, as these don't necessarily require the abelian structure.

\begin{lemma}
    \
    \begin{itemize}
        \item For every $X^\bullet \in \Kom^+(\AA)$, there exists a quasi-isomorphism $X^\bullet \to M^\bullet$ with $M^\bullet \in \Kom^+(R)$.
        \item The class of quasi-isomorphisms in $K^+(R)$ is localizing.
        \item $\GG := K^+(R)[\{\text{q-iso}\}^{-1}] \to D^+(\AA)$ is an equivalence
    \end{itemize}
\end{lemma}

Now have diagram
\begin{cd}
    & K^+(\AA) \rar["K^+(F)"] 
    & K^+(\BB) \dar["Q_B"]
    \\
    K^+(R) \dar \urar
    && D^+(\BB)
    \\
    K^+(R)[\{\text{q-iso}\}^{-1}] \ar[rr, "\iso"] \ar[urr, dashed]
    && D^+(\AA) \uar[dashed, "RF"']
\end{cd}

\sepline

5/5/23

Note that a morphism $u$ in $K^+(R)$ is a quasi-isomorphism iff $c(u)$ is acyclic.
Then by (2) of adapted class, acyclic $c(u) \in K^+(R)$ tells us that $F(c(u)) = C(F(u))$ is acyclic (strictly speaking, this is $K^+(F)(i(c(u)))$).
Then $F(u)$ is quasi-isomorphic.

Quasi-isomorphism in $K^+(R)$ sent to isomorphism in $D^+(\BB)$ by top path in diagram.
This map factors through the localization, equivalent to $D^+(\AA)$, so we get the functor $RF : D^+(\AA) \to D^+(\BB)$.

Explicitly, for any $X^\bullet \in D^+(\AA)$, we take a quasi-isomorphism $X^\bullet \to M^\bullet \in \Kom^+(R)$ and $RF(X^\bullet) = F(M^\bullet)$

\begin{proposition}
    This is a derived functor $RF$ with $\eps_F(X^\bullet) : F(X^\bullet) \to F(M^\bullet)$ induced by $X^\bullet \to M^\bullet$, giving natural transformation $Q_B \circ F \Rightarrow RF \circ Q_A$.
\end{proposition}

\begin{proof}
    Exercise
\end{proof}

Exercise: If $\AA$ has enough injectives, then the class of injective objects is adapted to any left exact functor. 
(Only need to verify (2): exercise from last time.)

\sepline

\begin{theorem}
    Let $F : \AA \to \BB$ and $G : \BB \to \CC$ be left exact functors.
    Suppose we have adapted classed $R_\AA$ for for $F$ and $R_\BB$ for $G$ such that $F(M) \in R_\BB$ if $M \in R_\AA$.
    Then there is a natural isomorphism $RG \circ RF \Rightarrow R(G \circ F)$.
\end{theorem}

\begin{proof}
    By definition, replace $X^\bullet$ with some $M^\bullet$ and map to $FM^\bullet \in \Kom^+(R_\BB)$ then get $G(FM^\bullet)$.
\end{proof}

\sepline

Want to define $\RHom$.

Let $\AA$ be an abelian category with enough injectives.

Let $X^\bullet \in \Kom(\AA)$ and $Y^\bullet \in \Kom^+(\AA)$.

Define \keyword{hom complex}
\[
    \Hom_\AA^n(X^\bullet, Y^\bullet) := \prod_{i \in \Z} \Hom_\AA(X^i, Y^{n+i})
\]
with differentials
\[
    d((f_i)_{i \in \Z}) := (d_Y \circ f_i + (-1)^{n+1} f_{i+1} \circ d_X)_{i \in \Z}
\]

Key Observation: $Z^0(\Hom^\bullet) = \Hom_{\Kom(\AA)}(X^\bullet, Y^\bullet)$.

$B^0(\Hom^\bullet) = \{f \text{ homotopic to } 0\}$.

So $H^0(\Hom^\bullet(X^\bullet, Y^\bullet)) = \Hom_{K(\AA)}(X^\bullet, Y^\bullet)$.

\sepline

Choose a quasi-isomorphism $Y^\bullet \to I^\bullet \in \Kom^+(\mathcal{I})$, where $\mathcal{I}$ is class of injectives in $\AA$.
Define
\[
    \RHom_\AA(X^\bullet, Y^\bullet) := \Hom^\bullet(X^\bullet, I^\bullet) \in D(\Ab).
\]

Won't write dot, so remember $X, Y \in \AA$.

Note
\begin{align*}
    \Ext_\AA^i(X, Y) 
        &:= \Hom_{D(\AA)}(X[0], Y[i]) \\
        &= \Hom_{D(\AA)}(X[0], I^\bullet[i]) \\
        &= \Hom_{K(\AA)}(X[0], I^\bullet[i]) \\
        &= H^i(\RHom_\AA(X, Y))
\end{align*}
where $Y \to I^\bullet$ is some quasi-isomorphism.


\end{document}