\documentclass[12pt]{article}

% Packages
\usepackage[margin=1in]{geometry}
\usepackage{fancyhdr, parskip}
\usepackage{amsmath, amsthm, amssymb}

% Page Style
\fancypagestyle{plain}{
    \fancyhf{}
    \renewcommand{\headrulewidth}{0pt}
    \renewcommand{\footrulewidth}{0pt}
    \fancyfoot[R]{\thepage}
}
\pagestyle{plain}

% Problem Box
\setlength{\fboxsep}{4pt}
\newsavebox{\savefullbox}
\newenvironment{fullbox}{\begin{lrbox}{\savefullbox}\begin{minipage}{\dimexpr\textwidth-2\fboxsep\relax}}{\end{minipage}\end{lrbox}\begin{center}\framebox[\textwidth]{\usebox{\savefullbox}}\end{center}}
\newenvironment{pbox}[1][]{\begin{fullbox}\ifx#1\empty\else\paragraph{#1}\fi}{\end{fullbox}}

% Theorem Environments
%\theoremstyle{definition}
%\newtheorem{proposition}{Proposition}
%\newtheorem{lemma}{Lemma}

% Options
%\allowdisplaybreaks
%\addtolength{\jot}{4pt}

% Default Commands
\newcommand{\isp}[1]{\quad\text{#1}\quad}
\newcommand{\N}{\mathbb{N}} 
\newcommand{\Z}{\mathbb{Z}}
\newcommand{\Q}{\mathbb{Q}}
\newcommand{\R}{\mathbb{R}}
\newcommand{\C}{\mathbb{C}}
\newcommand{\eps}{\varepsilon}
\renewcommand{\phi}{\varphi}
\renewcommand{\emptyset}{\varnothing}
\newcommand{\<}{\langle}
\renewcommand{\>}{\rangle}
\newcommand{\isom}{\cong}
\newcommand{\eqc}{\overline}
\newcommand{\clo}{\overline}

% Extra Commands
\newcommand{\A}{\mathbb{A}}
\renewcommand{\P}{\mathbb{P}}
\newcommand{\teq}{\trianglelefteq}
\newcommand{\inc}{\hookrightarrow}
\newcommand{\Ip}{I_{\mathrm{p}}}
\newcommand{\Vp}{V_{\mathrm{p}}}
\newcommand{\rad}{\sqrt}
\DeclareMathOperator{\codim}{codim}

% Document Info
\fancypagestyle{title}{
    \renewcommand{\headrulewidth}{0.4pt}
    \setlength{\headheight}{15pt}
    \fancyhead[R]{Harry Coleman}
    \fancyhead[L]{MATH 120AG Homework 5}
    \fancyhead[C]{May 3, 2021}
}

% Begin Document
\begin{document}
\thispagestyle{title}

\begin{pbox}[Exercise 5.23]
    Use diagonals to prove the following statements:
\end{pbox}

\begin{pbox}[(a)]
    The intersection of any two affine open subsets of a variety is again an affine open subset.
\end{pbox}



\begin{proof}
    Let $U, V \subseteq X$ be affine open subsets of the variety $X$. We know $U \cap V$ is open in $X$, therefore an open subprevariety of $X$. It remains the prove affine. Since $U$ and $V$ are affine varieties, then their product $U \times V$ is again an affine variety. There is an inclusion morphism of prevarieties $\iota : U \times V \inc X \times X$ which, in particular, is continuous. By definition, $X$ being a variety means the diagonal $\Delta_X$ is closed in $X \times X$. Then the preimage $\iota^{-1}(\Delta_X)$ is closed in $U \times V$. As sets, we have
    \[
        \iota^{-1}(\Delta_X)
            = (U \times V) \cap \Delta_X
            = \Delta_{U \cap V}.
    \]
    As a closed subset of the affine variety $U \times V$, we deduce that $\Delta_{U \cap V}$ is an affine variety. There are inclusion morphisms of prevarieties, $U \cap V \inc U$ and $U \cap V \inc V$, which uniquely define the morphism of prevarieties
    \begin{align*}
        i :  U \cap V &\to \Delta_{U \cap V} \\
            x &\mapsto (x, x).
    \end{align*}
    Moreover, $i$ is an isomorphism of prevarieties, with inverse given by the restriction to $\Delta_{U \cap V}$ of either the projection $\pi_U : U \times V \to U$ or the projection $\pi_V : U \times V \to V$, i.e.,
    \[
        i^{-1} = \pi_U|_{\Delta_{U \cap V}} = \pi_V|_{\Delta_{U \cap V}}.
    \]
    Hence, $U \cap V \isom \Delta_{U \cap V}$ as prevarieties. And since $\Delta_{U \cap V}$ is an affine variety, then so is $U \cap V$.

\end{proof}


\newpage
\begin{pbox}[(b)]
    If $X, Y \subseteq \A^n$ are two pure-dimensional affine varieties then every irreducible component of $X \cap Y$ has dimension at least $\dim X + \dim Y - n$.
\end{pbox}

\begin{proof}
    There is an inclusion morphism $\iota : X \times Y \inc \A^n \times \A^n$. Since $\A^n$ is an affine variety, then $\Delta_{\A^n}$ is closed in $\A^n \times \A^n$, with
    \[
        \Delta_{\A^n} = V(x_1 - y_1, \dots, x_n - y_n),
    \]
    where each $x_j - y_j \in A(\A^n \times \A^n)$. Then, similar to (a), we find
    \[
        \iota^{-1}(\Delta_{\A^n}) = (X \times Y) \cap \Delta_{\A^n} = \Delta_{X \cap Y},
    \]
    with $X \cap Y \isom \Delta_{X \cap Y}$ as affine varieties. Then
    \[
        \Delta_{X \cap Y} = V(x_1 - y_1, \dots, x_n - y_n),
    \]
    where each $x_j - y_j \in A(X \times Y)$. Then
    \[
        I(\Delta_{X \cap Y}) = \rad{\<x_1 - y_1, \dots, x_n - y_n\>} \teq A(X \times Y).
    \]
    
    With the evaluation homomorphism
    \begin{align*}
        K[x_1, \dots, x_n, y_1, \dots, y_n] &\to K[x_1, \dots, x_n] \\
            p(x_1, \dots, x_n, y_1, \dots, y_n) &\mapsto p(x_1, \dots, x_n, x_1, \dots, x_n),
    \end{align*}
    we obtain
    \[
        K[x_1, \dots, x_n, y_1, \dots, y_n]/\<x_1 - y_1, \dots, x_n - y_n\> \isom K[x_1, \dots, x_n].
    \]
    This is an integral domain, implying that $\<x_1 - y_1, \dots, x_n - y_n\>$ is a is prime ideal in $K[x_1, \dots, x_n, y_1, \dots, y_n]$. Therefore, its projection onto the quotient
    \[
        A(X \times Y) = K[x_1, \dots, x_n, y_1, \dots, y_n]/I(X \times Y)
    \]
    will again be a prime ideal, therefore a radical ideal. Then
    \[
        I(\Delta_{X \cap Y}) = \<x_1 - y_1, \dots, x_n - y_n\> \teq A(X \times Y).
    \]

    That is, $I(\Delta_{X \cap Y})$ is an ideal of the Noetherian ring $A(X \times Y)$, generated by $n$ elements. By Krull's height theorem, the minimal prime ideals of $A(X \times Y)$ containing $I(\Delta_{X \cap Y})$ have height at most $n$. This corresponds to the irreducible components of $\Delta_{X \cap Y}$ having codimension in $X \times Y$ at most $n$. Each irreducible component of $X \cap Y$ is isomorphic to an irreducible component $Z$ of $\Delta_{X \cap Y}$, of the same dimension and for which we have
    \[
        \dim(X \times Y) = \dim Z + \codim_{X \times Y} Z \leq \dim Z + n,
    \]
    so
    \[
        \dim X + \dim Y - n \leq \dim Z.
    \]

\end{proof}


\newpage
\begin{pbox}[Exercise 6.13]
    Let $a \in \P^n$ be a point. Show that the one-point set $\{a\}$ is a projective variety, and compute explicit generators for the ideal $\Ip(\{a\}) \teq K[x_0, \dots, x_n]$.
\end{pbox}

\begin{proof}
    Suppose $a = [a_0, a_1, \dots, a_n]$. We know that not all $a_j$'s are zero, otherwise $a$ would not describe a linear subspace of $\A^{n+1}$. Without loss of generality, assume $a_0 \ne 0$ (otherwise, we replace $a_0$ with a nonzero $a_j$ in the following argument), then
    \[
        a = [a_0, a_1, \dots, a_n] = \left[1, \tfrac{a_1}{a_0}, \dots, \tfrac{a_n}{a_0}\right].
    \]
    For any $x \in \P^n$, we have $x = a$ if and only if $x = [x_0, x_1, \dots, x_n]$ with $x_0 \ne 0$ and
    \[
        \left[1, \tfrac{x_1}{x_0}, \dots, \tfrac{x_n}{x_0}\right] = \left[1, \tfrac{a_1}{a_0}, \dots, \tfrac{a_n}{a_0}\right].
    \]
    This last equality holds if and only if, for $j = 1, \dots, n$, we have $\tfrac{x_j}{x_0} = \tfrac{a_j}{a_0}$, which is equivalent to $a_0x_j - a_jx_0 = 0$. Therefore, the singleton $\{a\}$ is given as the zero locus
    \[
        \{a\} = \Vp\left(x_1 - \tfrac{a_1}{a_0}x_0, \dots, x_n - \tfrac{a_n}{a_0}x_0\right)
    \]
    of homogeneous polynomials. So in fact, $\{a\}$ is a projective variety. 

\end{proof}

Let
\[
    J = \left\<x_1 - \tfrac{a_1}{a_0}x_0, \dots, x_n - \tfrac{a_n}{a_0}x_0\right\> \teq K[x_0, \dots, x_n]
\]
be the ideal generated by homogeneous polynomials; we claim $\Ip(\{a\}) = J$.

    By the projective Nullstellensatz, we have
\[
    \Ip(\{a\}) = \Ip(\Vp(J)) = \rad{J}.
\]
To show that $J$ is a radical ideal, we will show that it is a prime ideal, by showing that the quotient ring $K[x_0, \dots, x_n]/J$ is an integral domain. Consider the evaluation homomorphism of rings
\begin{align*}
    \phi : K[x_0, \dots, x_n] &\to K[x] \\
        p(x_0, x_1, \dots, x_n) &\mapsto p\left(x, \tfrac{a_1}{a_0}x, \dots, \tfrac{a_n}{a_0}x\right).
\end{align*}
This map is surjective, as the restriction to $K[x_0]$ is simply the isomorphism $K[x_0] \to K[x]$ by $x_0 \mapsto x$. Moreover, its kernel is precisely the ideal $J$, so
\[
    K[x_0, \dots, x_n]/J = K[x_0, \dots, x_n]/\ker\phi \isom K[x].
\]
Since $K[x]$ is an integral domain, this proves $J$ is a prime, therefore radical, ideal. Hence,
\[
    \Ip(\{a\}) = \rad{J} = J = \left\<x_1 - \tfrac{a_1}{a_0}x_0, \dots, x_n - \tfrac{a_n}{a_0}x_0\right\>.
\]

\end{document}