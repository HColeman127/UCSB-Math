\documentclass[12pt]{article}

% Packages
\usepackage[margin=1in]{geometry}
\usepackage{fancyhdr, parskip}
\usepackage{amsmath, amsthm, amssymb}
\usepackage{tikz, tikz-cd}

% Page Style
\makeatletter
\fancypagestyle{title}{
    \renewcommand{\headrulewidth}{0.4pt}
    \setlength{\headheight}{15pt}
    \fancyhead[R]{\@author}
    \fancyhead[L]{\@title}
    \fancyhead[C]{\@date}
}
\makeatother
\renewcommand{\maketitle}{\thispagestyle{title}}
\fancypagestyle{plain}{
    \fancyhf{}
    \renewcommand{\headrulewidth}{0pt}
    \renewcommand{\footrulewidth}{0pt}
    \fancyfoot[R]{\thepage}
}
\pagestyle{plain}

% Problem Box
\setlength{\fboxsep}{4pt}
\newlength{\myparskip}
\setlength{\myparskip}{\parskip}
\newsavebox{\savefullbox}
\newenvironment{fullbox}{\begin{lrbox}{\savefullbox}\begin{minipage}{\dimexpr\textwidth-2\fboxsep\relax}\setlength{\parskip}{\myparskip}}{\end{minipage}\end{lrbox}\framebox[\textwidth]{\usebox{\savefullbox}}}
\newenvironment{pbox}[1][]{\begin{fullbox}\ifx#1\empty\else\paragraph{#1}\phantom{}\fi}{\end{fullbox}}

% Theorem Environments
\theoremstyle{definition}
\newtheorem{lemma}{Lemma}
\newtheorem{corollary}{Corollary}


% Tikz Environments
\newenvironment{drawing}{\begin{center}\begin{tikzpicture}}{\end{tikzpicture}\end{center}}
% \tikzcdset{row sep/normal=0pt}
\newenvironment{cd}{\begin{center}\begin{tikzcd}}{\end{tikzcd}\end{center}}

% Default Commands
\newcommand{\isp}[1]{\quad\text{#1}\quad}
\newcommand{\N}{\mathbb{N}} 
\newcommand{\Z}{\mathbb{Z}}
\newcommand{\Q}{\mathbb{Q}}
\newcommand{\R}{\mathbb{R}}
\newcommand{\C}{\mathbb{C}}
\newcommand{\A}{\mathbb{A}}
\renewcommand{\P}{\mathbb{P}}
\newcommand{\eps}{\varepsilon}
\renewcommand{\phi}{\varphi}
\renewcommand{\emptyset}{\varnothing}
\newcommand{\<}{\langle}
\renewcommand{\>}{\rangle}
\newcommand{\isom}{\cong}
\newcommand{\eqc}{\overline}
\newcommand{\clo}{\overline}
\newcommand{\seq}{\subseteq}
\newcommand{\teq}{\trianglelefteq}
\DeclareMathOperator{\id}{id}
\DeclareMathOperator{\im}{im}
\newcommand{\inc}{\hookrightarrow}

% Extra Commands
\newcommand{\dd}{\mathrm{d}}
\newcommand{\DD}{\mathrm{D}}

\newcommand{\pdv}[2]{\frac{\partial #1}{\partial #2}}
\newcommand{\mat}[1]{\begin{bmatrix}#1\end{bmatrix}}

% Document
\begin{document}
\title{MATH 221C Homework 2}
\author{Harry Coleman}
\date{April 11, 2022}
\maketitle

I worked with Joseph Sullivan and Gahl Shemy.

\begin{pbox}[1 Exercise 1.1.1]
    If $k < \ell$ we can consider $\R^k$ to be the subset $\{(a_1, \dots, a_k, 0, \dots, 0)\}$ in $\R^\ell$.
    
\end{pbox}

\begin{lemma}
    Let $n \geq 1$ and $\pi : \R^\ell \to \R^k$ be the standard submersion.
    If $G \in C^n(\R^k, \R)$, then
    \[
        \DD_{i_1, \dots, i_n} (G \circ \pi) = \begin{cases}
            \DD_{i_1, \dots, i_n} G \circ \pi & \text{all } i_j \leq k, \\
            0 & \text{some } i_j > k.
        \end{cases}
    \]
    In particular, $G \circ \pi \in C^n(\R^\ell, \R)$ as all mixed $n$th partial derivatives exist and are continuous.
\end{lemma}

\begin{proof}
    We prove the lemma by induction on $n$.

    For the base case, we simply calculate the first partial derivatives.
    Suppose $G \in C^1(\R^k, \R)$, $x \in \R^\ell$, and $i \leq k$.
    Then
    \begin{align*}
        \lim_{h \to 0} \frac{(G \circ \pi)(x + he_i) - (G \circ \pi)(x)}{h}
            &= \lim_{h \to 0} \frac{G(\pi(x + he_i)) - G(\pi(x))}{h} \\
            &= \lim_{h \to 0} \frac{G(\pi(x) + he_i) - G(\pi(x))}{h} \\
            &= \DD_i G(\pi(x)),
    \end{align*}
    so indeed $\DD_i (G \circ \pi) = \DD_i G \circ \pi$.
    For $i > k$,
    \begin{align*}
        \lim_{h \to 0} \frac{(G \circ \pi)(x + he_i) - (G \circ \pi)(x)}{h}
            &= \lim_{h \to 0} \frac{G(\pi(x + he_i)) - G(\pi(x))}{h} \\
            &= \lim_{h \to 0} \frac{G(\pi(x)) - G(\pi(x))}{h} \\
            &= 0,
    \end{align*}
    so $\DD_i G = 0$.
    Thus, the base case holds.

    Assume that the result holds for $n-1$.
    Let $G \in C^n(\R^k, \R)$ and $i_1, \dots, i_n \in \{1, \dots, \ell\}$.
    If some $j < n$ has $i_j > k$, then the inductive hypothesis gives
    \[
        \DD_{i_1, \dots, i_n} (G \circ \pi)
            = \DD_{i_n}(\DD_{i_1, \dots, i_{n-1}} (G \circ \pi))
            = D_{i_n} 0
            = 0.
    \]
    Supposing that $i_j \leq k$ for all $j < n$, the inductive hypothesis gives
    \[
        \DD_{i_1, \dots, i_n} (G \circ \pi)
            = \DD_{i_n}(\DD_{i_1, \dots, i_{n-1}} (G \circ \pi))
            = \DD_{i_n}(\DD_{i_1, \dots, i_{n-1}} G \circ \pi).
    \]
    Note that $\DD_{i_1, \dots, i_{n-1}} G \in C^1(\R^\ell, \R)$,
    so as an instance of the base case,
    \[
        \DD_{i_n}(\DD_{i_1, \dots, i_{n-1}} G \circ \pi) = \begin{cases}
            \DD_{i_n}(\DD_{i_1, \dots, i_{n-1}} G) \circ \pi & i_n \leq k, \\
            0 & i_n > k.
        \end{cases}
    \]
    This completes the inductive step.
\end{proof}

\begin{lemma}
    Let $n \geq 1$ and $\iota : \R^k \to \R^\ell$ be the standard immersion.
    Let $U \seq \R^\ell$ be an open neighborhood of $\R^k$.
    If $G \in C^n(U, \R)$, then
    \[
        \DD_{i_1, \dots, i_n} (G \circ \iota) = \DD_{i_1, \dots, i_n} G \circ \iota.
    \]
    In particular, $G \circ \iota \in C^n(\R^k, \R)$ as all mixed $n$th partial derivatives exist and are continuous.
\end{lemma}

\begin{proof}
    We prove the lemma by induction on $n$.

    For the base case, we simply calculate the first partial derivatives.
    Suppose $G \in C^1(U, \R)$, $x \in \R^k$, and $i \leq k$.
    Then
    \begin{align*}
        \lim_{h \to 0} \frac{(G \circ \iota)(x + he_i) - (G \circ \iota)(x)}{h}
            &= \lim_{h \to 0} \frac{G(\iota(x + he_i)) - G(\iota(x))}{h} \\
            &= \lim_{h \to 0} \frac{G(\iota(x) + he_i) - G(\iota(x))}{h} \\
            &= \DD_i G(\iota(x)),
    \end{align*}
    so indeed $\DD_i (G \circ \iota) = \DD_i G \circ \iota$.

    Assume that the result holds for $n-1$.
    Let $G \in C^n(U, \R)$ and $i_1, \dots, i_n \in \{1, \dots, k\}$.
    Then applying the inductive hypothesis to $G$ and the base case to the $(n-1)$th partial derivative of $G$ gives
    \[
        \DD_{i_1, \dots, i_n} (G \circ \iota)
            = \DD_{i_n}(\DD_{i_1, \dots, i_{n-1}} (G \circ \iota))
            = \DD_{i_n}(\DD_{i_1, \dots, i_{n-1}} G \circ \iota).
            = \DD_{i_1, \dots, i_n}G \circ \iota.
    \]
\end{proof}

\begin{pbox}[]
    Show that the smooth functions on $\R^k$, considered as a subset of $\R^\ell$, are the same as usual.
\end{pbox}

\begin{proof}
    Recall that a function $f = (f_1, \dots, f_m) : \R^n \to \R^m$ is smooth if and only if all its component functions $f_i : \R^n \to \R$ are smooth.
    Therefore, it suffices to consider only functions with codomain $\R$.

    Let $\iota : \R^k \inc \R^\ell$ be standard immersion and $\pi : \R^\ell \to \R^k$ be the standard submersion.

    Given a smooth function $f \in C^\infty(\R^k, \R)$, define a new function $F = f \circ \pi : \R^\ell \to \R$.
    It follows from Lemma 1 that $f$ being smooths implies $F$ is also smooth.
    Moreover,
    \[
        F|_{\R^k} = f \circ \pi|_{\R^k} = f \circ \id_{\R^k} = f,
    \]
    which means $F$ is a smooth extension of $f$ to the open neighborhood $\R^\ell$ of $\R^k$.
    By definition, this means $f$ is smooth on $\R^k \seq \R^\ell$.

    Suppose $f : \R^k \to \R$ is smooth on $\R^k \seq \R^\ell$ and let $F : U \to \R$ be a smooth extension to an open neighborhood $\R^k \seq U \seq \R^\ell$.
    It follows from Lemma 2 that $F \circ \iota$ is a smooth function on $\R^k$ in the usual sense.
    Moreover,
    \[
        F \circ \iota = F|_{\R^k} = f,
    \]
    so indeed $f \in C^\infty(\R^k, \R)$.
\end{proof}


\newpage
\begin{pbox}[2 Exercise 1.1.2]
    Suppose that $X$ is a subset of $\R^N$ and $Z$ is a subset of $X$.
    Show that the restriction to $Z$ of any smooth map on $X$ is a smooth map on $Z$.
\end{pbox}

\begin{proof}
    As in Problem 1, it suffices to consider only functions with codomain $\R$.

    Let $f : X \to \R$ be a smooth function and let $F : U \to \R$ be a smooth extension to an open neighborhood $X \seq U \seq \R^N$ be an open.
    Then $Z \seq X \seq U$ and $F|_Z = f|_Z$, so $F$ is also a smooth extension of $f|_Z$ to the open neighborhood $Z \seq U \seq \R^N$.
    By definition, this means $f|_Z$ is smooth.
\end{proof}

\begin{pbox}[3 Exercise 1.1.3]
    Let $X \seq \R^N$, $Y \seq \R^M$, $Z \seq \R^L$ be arbitrary subsets, and let $f : X \to Y$, $g : Y \to Z$ be smooth maps.
    Then the composite $g \circ f : X \to Z$ is smooth.
\end{pbox}

\begin{proof}
    Let $x \in X$ and $F : U \to \R^M$ be a smooth local extension of $f$ at $x$.
    Next, let $G : V \to \R^L$ be a smooth local extension of $g$ at $f(x)$.
    Set $W = U \cap F^{-1}(V)$, then $F|_W : W \to V$ is a smooth extension of $f$ at $x$.
    From analysis, we know that the composition of smooth maps on open sets is smooth, so $G \circ F|_W : W \to \R^L$ is a smooth local extension of $g \circ f$ at $x$.
    Hence, $g \circ f$ is smooth at every point $x \in X$, hence $g \circ f$ is smooth.
\end{proof}


\begin{pbox}[]
    If $f$ and $g$ are diffeomorphisms, so is $g \circ f$.
\end{pbox}

\begin{proof}
    Since $f$ and $g$ are diffeomorphisms, they have smooth inverses $f^{-1} : Y \to X$ and $g^{-1} : Z \to Y$, respectively.
    By the previous result, the function inverse $(g \circ f)^{-1} = f^{-1} \circ g^{-1}$ is smooth.
\end{proof}


\begin{pbox}[4 Exercise 1.1.4(a)]
    Let $B_a$ be the open ball $\{x : |x|^2 < a^2\}$ in $\R^k$.
    ($|x|^2 = \sum x_i^2$) Show that the map
    \[
        x \longmapsto \frac{ax}{\sqrt{a^2 - |x|^2}}
    \]
    is a diffeomorphism of $B_a$ onto $\R^k$.
    [Hint: Compute its inverse directly.]
\end{pbox}

\begin{proof}
    Denote the map by $f : B_a \to \R^k$ and define the function $g : \R^k \to \R^k$ by
    \[
        g(x) = \frac{ax}{\sqrt{a^2 + |x|^2}}.
    \]
    We check that the image of $g$ is contained in $B_a$:
    \[
        \bigg|\frac{ax}{\sqrt{a^2 + |x|^2}}\bigg|^2
            = \frac{a^2|x|^2}{a^2 + |x|^2}
            = \frac{a^2}{\frac{a^2}{|x|^2} + 1}
            \leq a^2.
    \]
    As the composition of smooth functions, both $f$ and $g$ are smooth---we claim that they are inverse to each other.
    First, for any $x \in \R^k$,
    \[
        f(g(x))
            = \frac{a\frac{ax}{\sqrt{a^2 + |x|^2}}}{\sqrt{a^2 - \Big|\frac{ax}{\sqrt{a^2 + |x|^2}}\Big|^2}}
            = \frac{a^2x}{\sqrt{(a^2 + |x|^2)\Big(a^2 - \frac{a^2|x|^2}{a^2 + |x|^2}}\Big)}
            = \frac{a^2x}{\sqrt{a^4 + a^2|x|^2 - a^2|x|^2}}
            = x.
    \]
    In particular, this tells us that $f$ is surjective, since for any point $x \in \R^k$, $g(x) \in B_a$ is a point in the domain of $f$ mapping to $x$.
    Second, for any $x \in B_a$,
    \[
        g(f(x))
            = \frac{a\frac{ax}{\sqrt{a^2 - |x|^2}}}{\sqrt{a^2 + \Big|\frac{ax}{\sqrt{a^2 - |x|^2}}\Big|^2}}
            = \frac{a^2x}{\sqrt{(a^2 - |x|^2)\Big(a^2 + \frac{a^2|x|^2}{a^2 - |x|^2}}\Big)}
            = \frac{a^2x}{\sqrt{a^4 - a^2|x|^2 + a^2|x|^2}}
            = x.
    \]
    Hence, $f$ is a diffeomorphism with smooth inverse $g$.
\end{proof}

\begin{pbox}[5 Exercise 1.1.6]
    A smooth bijective map of manifolds need not be a diffeomorphism.
    In fact, show that $f : \R^1 \to \R^1$, $f(x) = x^3$ is an example.
\end{pbox}

\begin{proof}
    The function inverse of $f$ is $g(x) = \sqrt[3]{x}$, but this is not differentiable at $0$, since
    \[
        \lim_{h \to 0} \frac{g(0 + h) - g(0)}{h}
            = \lim_{h \to 0} \frac{\sqrt[3]{h}}{h}
            = \lim_{h \to 0} \frac{1}{h^{2/3}}
            = \infty.
    \]
\end{proof}

\begin{pbox}[6 Exercise 1.1.7]
    Prove that the union of the two coordinate axes in $\R^2$ is not a manifold.
\end{pbox}

\begin{proof}
    Assume in contradiction that $X = \{xy = 0\} \seq \R^2$ is a manifold.

    We first show that $X$ must be a $1$-dimensional manifold.
    The point $(1, 0) \in X$ has an open neighborhood $(0, 2) \times 0 \seq X$, which is diffeomorphic to the open interval $(0, 2) \seq \R^1$ via projecting onto the first coordinate.
    In particular, $X$ is locally $1$-dimensional at the point $(1, 0)$, so by definition it must be globally $1$-dimensional.

    Let $\phi : U \to V$ be a smooth chart of $X$ around the origin, i.e., $U \seq X$ is an open neighborhood of the origin, $V \seq \R^1$ is open, and $\phi$ is a diffeomorphism.
    Since $X \seq \R^2$ has the subspace topology, we may restrict our attention to a small open ball $B_r(0) \seq \R^2$ whose intersection $U' = B_r(0) \cap X \seq U$.
    Set $V' = \phi(U') \seq V$, then the fact that $U' \seq U$ is open implies the restricted chart $\phi|_{U'} : U' \to V'$ is still a diffeomorphism.

    \begin{drawing}[commutative diagrams/every diagram]
        \begin{scope}[xshift=-3cm]
            \draw (-1, 0) -- (1, 0);
            \draw (0, -1) -- (0, 1);

            \draw[dotted] (1, 0) -- (1.5, 0);
            \draw[dotted] (-1, 0) -- (-1.5, 0);
            \draw[dotted] (0, 1) -- (0, 1.5);
            \draw[dotted] (0, -1) -- (0, -1.5);

            \fill (0, 0) circle (2pt) node[below left] {$0$};

            \node at (1, 1) {$U'$};
        \end{scope}

        \node (A) at (-1, 1) {};
        \node (B) at (1, 1) {};
        \path[commutative diagrams/.cd, every arrow, every label]
            (A) edge[bend left] node {$\phi|_{U'}$} (B);

        \begin{scope}[xshift=3cm]
            \draw (-1, 0) -- (1, 0);

            \draw[dotted] (1, 0) -- (1.5, 0);
            \draw[dotted] (-1, 0) -- (-1.5, 0);

            \fill (0, 0) circle (2pt) node[below] {$\phi(0)$};

            \node at (-1, 1) {$V'$};
        \end{scope}
    \end{drawing}
    
    On one hand, $U'$ is star-shaped (with all line segments to the origin) so it is a connected space.
    The homeomorphism $\phi|_{U'}$ preserves connectedness, therefore $V'$ must be an open interval.
    On the other hand, $U' \setminus \{0\}$ is an open subset with four connected components---namely $\{\pm x > 0\}$ and $\{\pm y > 0\}$.
    Another restriction of $\phi$ gives a diffeomorphism to
    \[
        \phi(U' \setminus \{0\}) = \phi(U') \setminus \{\phi(0)\} = V' \setminus \{\phi(0)\}.
    \]
    However, removing a single point from an open interval leaves us with only two disjoint intervals, which make up its two connected components.
    This is a contradiction since homeomorphisms preserve the number of connected components.
\end{proof}

\begin{pbox}[7 Exercise 1.1.8]
    Prove that the paraboloid in $\R^3$, defined by $x^2 + y^2 - z^2 = a$, is a manifold if $a > 0$.
\end{pbox}

\begin{proof}
    Define $f(x, y, z) = x^2 + y^2 - z^2$, so $f : \R^3 \to \R$ is a smooth function.
    We claim that $a \ne 0$ is a regular value of $f$.
    Suppose $p = (x, y, z) \in f^{-1}(a)$, then we have the Jacobian
    \[
        J_f(p) = \mat{2x & 2y & -2z}.
    \]
    Since $x^2 + y^2 - z^2 = a \ne 0$, then some component of $p$ must be nonzero.
    In particular, the Jacobian has rank $1$, so $\dd{f}_p$ is surjective.
    This means $a$ is a regular value of $f$, therefore the paraboloid $f^{-1}(a) \seq \R^3$ is a smooth manifold of dimension $2$.
\end{proof}

\begin{pbox}[]
    Why doesn't $x^2 + y^2 - z^2 = 0$ define a manifold?
\end{pbox}

Note that when $a = 0$, the origin $0 \in f^{-1}(0)$ has Jacobian $J_f(0) = 0$.
In particular, $a$ is not a regular value of $f$, so the above argument does not work to show $f^{-1}(0)$ is a manifold.
We will prove more explicitly that it is not a manifold.

\begin{proof}
    Suppose $X$ is a smooth manifold
\end{proof}


\begin{pbox}[8 Exercise 1.1.14]
    
\end{pbox}

\begin{lemma}\label{diff-prod}
    Let $F \in C^1(U, \R^k)$ and $G \in C^1(V, \R^\ell)$ for $U \seq \R^n$ and $V \seq \R^m$ open.
    Then the product $F \times G \in C^1(U \times V, \R^k \times \R^\ell)$ and $\dd(F \times G) = \dd{F} \times \dd{G}$.
\end{lemma}

\begin{proof}
    We consider the sum norm $\|(x, y)\| = \|x\| + \|y\|$ on both $\R^n \times \R^m$ and $\R^k \times \R^\ell$.
    Then for all $z = (x, y) \in U \times V$ and $\Delta z = (\Delta x, \Delta y) \in \R^n \times \R^m$, we have
    \begin{align*}
        &\frac{\|(F \times G)(z + \Delta z) - (F \times G)(z) - (\dd{F} \times \dd{G})_{z}(\Delta z)\|}{\|\Delta z\|} \\[2ex]
        & \hspace{2em} = \frac{\|F(x + \Delta x) - F(x) - \dd{F}_x(\Delta x)\|}{\|\Delta x\| + \|\Delta y\|} + \frac{\|G(y + \Delta y) - G(y) - \dd{G}_y(\Delta y)\|}{\|\Delta x\| +  \|\Delta y\|}.
    \end{align*}
    When $\Delta x = 0$, the left term is zero and the right term is simply the argument of the limit in the definition of the derivative $\dd{G}$ (and vice versa when $\Delta y = 0$).
    When both $\Delta x$ and $\Delta y$ are nonzero, the expression is bounded above by
    \[
        \frac{\|F(x + \Delta x) - F(x) - \dd{F}_x(\Delta x)\|}{\|\Delta x\|} + \frac{\|G(y + \Delta y) - G(y) - \dd{G}_y(\Delta y)\|}{\|\Delta y\|}.
    \]
    Replacing the respective term by zero whenever $\Delta x$ or $\Delta y$ is zero, this is in fact always an upper bound.
    Therefore, when taking the limit as $\|\Delta z\| \to 0$, we may simply take the limit of the left term as $\|\Delta x\| \to 0$ and the limit of the right term as $\|\Delta y\| \to 0$.
    By definition of the derivative, both of these are zero, so indeed $F \times G$ is differentiable at $z$ with $\dd(F \times G)_z = \dd{F}_x \times \dd{G}_y$ for all $z = (x, y) \in U \times V$.
    Moreover, $\dd(F \times G) = \dd{F} \times \dd{G}$ is continuous as the product of continuous maps, hence $F \times G \in C^1(U \times V, \R^k \times \R^\ell)$.
\end{proof}

\begin{corollary}
    Let $F \in C^r(U, \R^k)$ and $G \in C^r(V, \R^\ell)$ for $U \seq \R^n$ and $V \seq \R^m$ open.
    Then the product $F \times G \in C^r(U \times V, \R^k \times \R^\ell)$ and $\dd^r(F \times G) = \dd^r{F} \times \dd^r{G}$.
\end{corollary}

\begin{proof}
    Lemma \ref{diff-prod} provides the base case for induction.
    Assume the result holds for all orders less than $r \geq 1$.
    Given $F \in C^r(U, \R^k)$ and $G \in C^r(V, \R^\ell)$, we know that $F \times G$ is at least $r - 1$ times continuously differential, with
    \[
        \dd^{r-1}(F \times G) = \dd^{r-1}{F} \times \dd^{r-1}{G}.
    \]
    By assumption, both $\dd^{r-1}F$ and $\dd^{r-1}{G}$ are continuously differential, and the base case implies that so is their product, with
    \[
        \dd(\dd^{r-1}{F} \times \dd^{r-1}{G})
            = \dd(\dd^{r-1}F) \times \dd(\dd^{r-1}{G})
            = \dd^rF \times \dd^rG.
    \]
    This completes the inductive step.
\end{proof}

\begin{pbox}[]
    If $f : X \to X'$ and $g : Y \to Y'$ are smooth maps, define a \textit{product map} $f \times g : X \times Y \to X' \times Y'$ by
    \[
        (f \times g)(x, y) = (f(x), g(y)).
    \]
    Show that $f \times g$ is smooth.
\end{pbox}

\begin{proof}
    Say $X \seq \R^n$, $X' \seq \R^k$, $Y \seq \R^m$, and $Y' \seq \R^\ell$.
    Given a point $(x, y) \in X \times Y$, choose smooth local extensions $F : U \to \R^k$ of $f$ at $x$ and $G : V \to \R^\ell$ of $g$ at $y$.
    Then the product map $F \times G : U \times V \to \R^k \times \R^\ell$ is a local extension of $f \times g$ at $(x, y)$.
    It follows from Corollary 1 that $F \times G$ is smooth.
\end{proof}

\begin{pbox}[9 Exercise 1.2.3]
    Let $V$ be a vector subspace of $\R^N$.
    Show that $T_x(V) = V$ if $x \in V$.
\end{pbox}\

\begin{proof}
    Let $v_1, \dots, v_n \in V$ form a basis and define a parameterization $\phi : \R^M \to V$ in terms of basis vectors $e_i \mapsto v_i$.
    This is indeed a diffeomorphism since it is linear and therefore smooth, with a smooth inverse $v_i \mapsto e_i$.
    Given a point $x \in V$, let $y = \phi^{-1}(x) \in \R^M$.
    Then the fact that $\phi$ is linear implies $\dd{\phi}_y = \phi : \R^M \to \R^N$.
    Therefore, we compute the tangent space to be
    \[
        T_x(V) = \im\dd{\phi}_y = \im \phi = V.
    \]
\end{proof}

\begin{pbox}[10 Exercise 1.2.6]
    The tangent space to $S^1$ at a point $(a, b)$ is a one-dimensional subspace of $\R^2$.
    Explicitly calculate the subspace in terms of $a$ and $b$.
\end{pbox}

The function $(\cos, \sin) : \R^1 \to \R^2$ defined by $x \mapsto (\cos x, \sin x)$ is a smooth surjection on $S^1$; choose $p \in \R^1$ such that $p \mapsto (a, b) \in S^1$.
On the interval $U = (p - \pi, p + \pi) \seq \R^1$, this map restricts to an injection, denoted by $\phi : U \to V = S^1 \setminus\{(-a, -b)\}$.
The inverse of $\phi$ is a smooth map $V \to U$ given by the appropriate branches of the inverse sine and cosine functions, hence $\phi$ is a diffeomorphism.

We now compute the derivative of $\phi$ at $p$:
\[
    \dd{\phi}_{p}(x)
        = \dd(\cos, \sin)_p(x)
        = (-\sin, \cos)_p(x)
        = (-\sin p, \cos p) \cdot x
        = (-bx, ax).
\]
Hence, we have the tangent space
\[
    T_{(a, b)}(S^1)
        = \im \dd{\phi}_{p}
        = \{(-bx, ax) : x \in \R^1\}
        = \operatorname{span}_\R\{-be_1 +  ae_2\}
\]



\begin{pbox}[\relax11 Exercise 1.2.8]
    What is the tangent space to the paraboloid defined by $x^2 + y^2 - z^2 = a$ at $(\sqrt{a}, 0, 0)$, where ($a > 0$)?
\end{pbox}

Let $M = \{x^2 + y^2 - z^2 = a\} \seq \R^3$ be the manifold in question.
Since $x^2 + y^2 = z^2 + a > 0$, there is a well-defined function $f : M \to \R^3$ where
\[\textstyle
    f(x, y, z)
        = \left(\frac{x}{\sqrt{x^2 + y^2}}, \frac{y}{\sqrt{x^2 + y^2}}, z\right)
        = \left(\frac{x}{\sqrt{z^2 + a}}, \frac{y}{\sqrt{z^2 + a}}, z\right).
\]
The first two components describe a unit vector in $\R^2$, which means we may consider this to be a smooth map $f : M \to N = S^1 \times \R^1$.
In fact, this is a diffeomorphism with smooth inverse $g : S^1 \times \R^1 \to M$ defined by
\[
    g((u, v), z) = \left(u\sqrt{z^2 + a}, v\sqrt{z^2 + a}, z\right).
\]
Then at each point $p \in N$, the derivative $\dd{g}_p : T_{p}(N) \to T_{g(p)}(M)$ is an isomorphism of tangent spaces (Exercise 1.1.4).
We compute the Jacobian matrix at $p = ((u, v), z)$ to be
\[
    J_g(p) = \mat{
        \sqrt{z^2 + a} & 0 & \frac{uz}{\sqrt{z^2 + a}} \\
        0 & \sqrt{z^2 + a} & \frac{vz}{\sqrt{z^2 + a}} \\
        0 & 0 & 1
    }.
\]
Moreover, (Textbook Exercise 1.2.9 and) Problem 10 give us
\[
    T_p(N)
        = T_{((u, v), z)}(S^1 \times \R^1)
        = T_{(u, v)}(S^1) \times T_z(\R^1)
        = \operatorname{span}_\R\{-ve_1 + ue_2, e_3\}.
\]
Taking the image under $\dd{g}_p$ yields
\[
    T_{g(p)}(M)
        = \im \dd{g}_p
        = J_g(p) \cdot T_p(N)
        = \operatorname{span}_\R\left\{
            \mat{-v\sqrt{z^2 + a} \\ u\sqrt{z^2 + a} \\ 0},
            \mat{\frac{uz}{\sqrt{z^2 + a}} \\ \frac{vz}{\sqrt{z^2 + a}} \\ 1}
        \right\}.
\]
At the point $(\sqrt{a}, 0, 0) = g((1, 0), 0) \in M$, we have
\[
    T_{(\sqrt{a}, 0, 0)}(M) = \operatorname{span}_\R\left\{
        \mat{-0\sqrt{0^2 + a} \\ 1\sqrt{0^2 + a} \\ 0},
        \mat{\frac{1 \cdot 0}{\sqrt{0^2 + a}} \\ \frac{0 \cdot 0}{\sqrt{0^2 + a}} \\ 1}
    \right\} = \operatorname{span}_\R\{e_2, e_3\} = 0 \times \R^2.
\]


\begin{pbox}[12 Exercise 1.2.10 \\ (a)]
    Let $f : X \to X \times X$ be the mapping $f(x) = (x, x)$.
    Check that $\dd{f}_x(v) = (v, v)$.
\end{pbox}

\begin{proof}
    Suppose $X \seq \R^n$ and $L : \R^n \to \R^n \times \R^n = \R^{n + n}$ is the linear map defined on bases by $e_i \mapsto (e_i, e_i) = e_i + e_i'$, where $\{e_i' = e_{i+n}\}$ is the standard basis for the second copy of $\R^n$.
    Then $f$ is simply the restriction of $L$ to $X$, so for all $x \in X$ its derivative is $\dd{f}_x = L$.
    Hence, for all $v \in \R^n$, we indeed have $\dd{f}_x(v) = L(v) = (v, v)$.
\end{proof}

\begin{pbox}[(b)]
    If $\Delta$ is the diagonal of $X \times X$, show that its tangent space $T_{(x, x)}(\Delta)$ is the diagonal of $T_x(X) \times T_x(X)$.
\end{pbox}

\begin{proof}
    Note that $f : X \to \Delta$ from (a) is a diffeomorphism, whose inverse is projection onto either component, hence the derivative $\dd{f}_x : T_x(X) \to T_{(x, x)}(\Delta)$ is an isomorphism of tangent spaces.
    Applying the result of (a), we find
    \[
        T_{(x, x)}(\Delta)
            = \im \dd{f}_x
            = \{(v, v) : v \in T_x(X)\},
    \]
    which is precisely the diagonal of $T_x(X) \times T_x(X)$.
\end{proof}


\end{document}