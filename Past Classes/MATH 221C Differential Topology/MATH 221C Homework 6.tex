\documentclass[12pt]{article}

% Packages
\usepackage[margin=1in]{geometry}
\usepackage{fancyhdr, parskip}
\usepackage{amsmath, amsthm, amssymb}
\usepackage{tikz, tikz-cd}

% Page Style
\makeatletter
\fancypagestyle{title}{
    \renewcommand{\headrulewidth}{0.4pt}
    \setlength{\headheight}{15pt}
    \fancyhead[R]{\@author}
    \fancyhead[L]{\@title}
    \fancyhead[C]{\@date}
}
\makeatother
\renewcommand{\maketitle}{\thispagestyle{title}}
\fancypagestyle{plain}{
    \fancyhf{}
    \renewcommand{\headrulewidth}{0pt}
    \renewcommand{\footrulewidth}{0pt}
    \fancyfoot[R]{\thepage}
}
\pagestyle{plain}

% Problem Box
\setlength{\fboxsep}{4pt}
\newlength{\myparskip}
\setlength{\myparskip}{\parskip}
\newsavebox{\savefullbox}
\newenvironment{fullbox}{\begin{lrbox}{\savefullbox}\begin{minipage}{\dimexpr\textwidth-2\fboxsep\relax}\setlength{\parskip}{\myparskip}}{\end{minipage}\end{lrbox}\framebox[\textwidth]{\usebox{\savefullbox}}}
\newenvironment{pbox}[1][]{\begin{fullbox}\def\temp{#1}\ifx\temp\empty\else\paragraph{#1}\phantom{}\fi}{\end{fullbox}}

\newcommand{\pnum}[1]{\paragraph{#1}}

% Theorem Environments
\theoremstyle{definition}
\newtheorem{lemma}{Lemma}

% Tikz Environments
\newenvironment{drawing}{\begin{center}\begin{tikzpicture}}{\end{tikzpicture}\end{center}}
% \tikzcdset{row sep/normal=0pt}
\newenvironment{cd}{\begin{center}\begin{tikzcd}}{\end{tikzcd}\end{center}}

% Default Commands
\newcommand{\isp}[1]{\qquad\text{#1}\qquad}
\newcommand{\N}{\mathbb{N}} 
\newcommand{\Z}{\mathbb{Z}}
\newcommand{\Q}{\mathbb{Q}}
\newcommand{\R}{\mathbb{R}}
\newcommand{\C}{\mathbb{C}}
\newcommand{\A}{\mathbb{A}}
\renewcommand{\P}{\mathbb{P}}
\newcommand{\eps}{\varepsilon}
\renewcommand{\phi}{\varphi}
\renewcommand{\emptyset}{\varnothing}
\newcommand{\<}{\langle}
\renewcommand{\>}{\rangle}
\newcommand{\iso}{\cong}
\newcommand{\eqc}{\overline}
\newcommand{\clo}{\overline}
\newcommand{\seq}{\subseteq}
\newcommand{\teq}{\trianglelefteq}
\DeclareMathOperator{\id}{id}
\DeclareMathOperator{\im}{im}
\newcommand{\inc}{\hookrightarrow}
\newcommand{\dd}{\mathrm{d}}

% Extra Commands
\newcommand{\mat}[1]{\begin{bmatrix}#1\end{bmatrix}}
\newcommand{\htpy}{\simeq}
\newcommand{\bd}{\partial}
\DeclareMathOperator{\inter}{int}

% Document
\begin{document}
\title{MATH 221C Homework 6}
\author{Harry Coleman}
\date{May 9, 2022}
\maketitle

I worked with Joseph Sullivan and Gahl Shemy.

\pnum{Exercise 1.4.8}

Denote the map by $p : \C \to \C$.
Then the derivative at a point $z \in \C$ is the linear map
\begin{align*}
    \dd{p}_z : \C &\longrightarrow \C, \\
        v &\longmapsto p'(z)v.
\end{align*}
This map is surjective if and only if $p'(z)$ is nonzero.
Note that
\[
    p'(z) = mz^{m-1} + (m-1)a_1 z^{m-2} + \cdots + a_{m-1}
\]
is a complex polynomial, so the fundamental theorem of algebra tells us that $p'(z)$ has exactly $m-1$ roots in $\C$, counted with multiplicity.
In particular, $p'(z)$ is nonzero except at finitely many points $z \in \C$.





\pnum{Exercise 1.5.8}

Let define the functions $h, s : \R^3 \to \R$ by
\[
    h(x, y, z) = x^2 + y^2 - z^2 - 1
    \isp{and}
    s(x, y, z) = x^2 + y^2 + z^2 - a.
\]
At a point $(x, y, z) \in \R^3$, their Jacobians are
\[
    J_h(x, y, z) = \mat{2x & 2y & -2z}
    \isp{and}
    J_s(x, y, z) = \mat{2x & 2y & 2z}.
\]
We deduce that $0$ is a regular value of both $h$ and $s$, so they define the hyperbola $H = h^{-1}(0)$ and the sphere $S = s^{-1}(0)$ as submanifolds of $\R^3$.
By Homework 5 Exercise 1.5.5, the tangent spaces are given by
\[
    T_{(x, y, z)}X = \ker\dd{h}_{(x, y, z)}
    \isp{and}
    T_{(x, y, z)}Y = \ker\dd{s}_{(x, y, z)}.
\]
Of course, these are the same as the kernels of the corresponding Jacobian matrices.
We will use this fact to understand the tangent spaces.

The intersection $H \cap S$ is the set of points in $\R^3$ which are roots of both $h$ and $s$.
In particular, for or such a point we must have
\[
    1 + z^2 = x^2 + y^2 = a - z^2,
\]
which implies $z = \pm\sqrt{(a - 1)/2}$.

If $a < 1$, there is no real solution for $z$, so $H \cap S = \emptyset$.
In this case, $H$ and $S$ are vacuously transverse.

If $a = 1$, then $z = 0$ and their intersection is the circle $\{x^2 + y^2 = 1\}$ in the $xy$-plane.
However, $H$ and $S$ are not transverse at any of these points.
For example, consider the point $e_1 = (1, 0, 0) \in H \cap S$.
The tangent space of $H$ at $e_1$ is
\[
    T_{e_1}H
        = \ker J_h(e_1)
        = \ker \mat{1 & 0 & 0}
        = \<e_2, e_3\>
\]
and the tangent space of $S$ at $e_1$ is
\[
    T_{e_1}S
        = \ker J_s(e_1)
        = \ker \mat{1 & 0 & 0}
        = \<e_2, e_3\>.
\]
In other words, both tangent spaces equal the $yz$-plane.
But these do not span $T_{e_1}\R^3 = \R^3$, so we conclude that $H$ and $S$ are not transverse for $a = 1$.

If $a > 1$, then there are two possible values of $z$---one positive and one negative.
In this case, $H \cap S$ is the union of two circles $\{x^2 + y^2 = r\}$ with $r = a - z^2 =(a + 1)/2$.
Each circle is parallel to the $xy$-plane, but shifted up or down by the corresponding value of $z$.
At all of theses points, $H$ and $S$ are transverse.
A vertical reflection of $\R^3$ restricts to a diffeomorphism on each of $H$ and $S$ which preserves the overall geometry inside $\R^3$, so it suffices to check that $H$ and $S$ are transverse on just one of the two circles.
Moreover, a rotation of $\R^3$ about the $z$ axis restricts to the same sort of geometry-preserving diffeomorphism, so we need only check transversality at a single point of the intersection.
We consider the point
\[
    p = (\sqrt{(a + 1)/2}, 0, \sqrt{(a - 1)/2}) \in H \cap S.
\]
The tangent space of $H$ at $p$ is
\[
    T_p H
        = \ker \mat{\sqrt{2(a + 1)} & 0 & -\sqrt{2(a - 1)}}
        = \<\sqrt{2(a - 1)}e_1 + \sqrt{2(a + 1)}e_3,\; e_2\>
\]
and the tangent space of $S$ at $p$ is
\[
    T_p S
        = \ker \mat{\sqrt{2(a + 1)} & 0 & \sqrt{2(a - 1)}}
        = \<\sqrt{2(a - 1)}e_1 - \sqrt{2(a + 1)}e_3,\; e_2\>.
\]
These spaces do span $\R^3$, so we conclude that $H$ and $S$ are transverse for $a > 1$.




\pnum{Exercise 1.6.3}

Per the hint, we check that path-connectivity defines an equivalence relation between points.
The relation is reflexive since the constant map $I \to \{x\}$ is smooth.
The relation is symmetric since if $f : I \to X$ is a smooth path from $x$ to $y$ then $f(1 - t)$ gives a smooth path from $y$ to $x$.
Lastly, the relation is transitive since a smooth path $I \to X$ from $x$ to $y$ can be considered as a homotopy between the maps $\{0\} \to \{x\}$ and $\{0\} \to \{y\}$ from the $0$-manifold $\{0\} = \R^0$.
In other words, the transitivity of paths is a special case of the transitivity of homotopies, which we have from Homework 5 Exercise 1.6.2.
We deduce that $X$ is the disjoint union of its path-components.

We check that the path-components are open.
Let $P \seq X$ be a path component and $x \in P$.
Pick a chart $\phi : U \to \R^k$ on $X$ with $\phi(x) = 0$.
Then the image $\phi(U) \seq \R^k$ is an open neighborhood of the origin, so it must contain some open ball of radius $\eps > 0$.
Since $B_\eps(0)$ is path-connected and open in $\phi(U)$, its image $\phi^{-1}(B_\eps(0))$ is similarly path-connected and open in $U$.
Since this set is path-connected and contains $x$, it must be contained in $P$.
And since $U$ is open in $X$, the set is also open in $X$.
Hence, we have found an open neighborhood of $x$ contained in $P$, so $P$ is open in $X$.

To summarize, we have found that every manifold is the disjoint union of its path-components, all of which are open.
If $X$ is connected, this is only possible if $X$ has a single path-component, hence $X$ must be path-connected.



\pnum{Exercise 1.6.4}

Suppose $X$ is contractible and let $R : X \times I \to X$ be a smooth homotopy from the identity $R_0 = \id_X$ to a constant map $R_1 = \mathrm{c}_x : X \to \{x\}$.
Let $Y$ be an arbitrary manifold and $f : Y \to X$ a smooth map.
Combining maps with smoothness-preserving operations, we construction a smooth map $H : Y \times I \to X$ as follows:
\begin{cd}[column sep=large]
    Y \times I \rar["f \times \id_Y"] & X \times I \rar["R"] & X
\end{cd}
Then $H$ is a homotopy from
\[
    H_0
        = R_0 \circ f
        = \id_X \circ f
        = f
\]
to
\[
    H_1
        = R_1 \circ f
        = \mathrm{c}_x \circ f
        = \mathrm{c}_x.
\]
This proves $f \htpy \mathrm{c}_x$ for all maps $f : Y \to X$.
From Homework 5 Exercise 1.6.2, we know that homotopy is an equivalence relation.
Then for all maps $f, g : Y \to X$ we have $f \htpy \mathrm{c}_x \htpy g$, which implies $f \htpy g$.

Conversely, assume that for any manifold $Y$ all maps $Y \to X$ are homotopic.
In particular, take $Y = X$ and consider the identity map $\id_X$ and a constant map $\mathrm{c}_x$ for some $x \in X$.
By assumption, we have $\id_X \htpy \mathrm{c}_x$, so indeed $X$ is contractible.




\pnum{Exercise 1.6.5}

We will apply Exercise 1.6.4.
Define $M : \R^k \times I \to \R^k$ by scalar multiplication, i.e., $M(x, t) = tx$.
Then $M$ is smooth and thus defines a homotopy $M_0 \htpy M_1$.
By construction, we have
\[
    M_1(x) = 1x = x = \id_{\R^k}(x)
\]
and
\[
    M_0(x) = 0x = 0 = \mathrm{c}_0(x).
\]
Thus, we have found a homotopy $\id_{\R^k} \htpy \mathrm{c}_0$, so indeed $\R^k$ is contractible.




\pnum{Exercise 1.6.6}

Assume $X$ is contractible.
For any pair of points $x, y \in X$, there is a homotopy between the maps $\{0\} \to \{x\}$ and $\{0\} \to \{y\}$, where $\{0\} = \R^0$ is a $0$-manifold with a single point.
These two maps are homotopic by Exercise 1.6.4, and a homotopy between them is precisely a path between $x$ and $y$ in $X$.
Therefore, $X$ is (path-)connected.
(The proof of Exercise 1.6.3 shows that connectedness and path-connectedness are equivalent conditions for manifolds.)
Another application of Exercise 1.6.4 tells us that every map $S^1 \to X$ is homotopic to a constant map, so indeed $X$ is simply-connected.

As a likely counterexample to the converse, consider $S^2$.
In more general algebraic topology, we do not require the homotopies to be smooth---only continuous.
Any smooth homotopy is also a continuous (not necessarily smooth) homotopy, but a priori the existence of a continuous homotopy does not in general imply the existence of a smooth one.
I presume that the same techniques for showing $S^2$ is continuously simply-connected would work to show that is is smoothly simply connected, so long as we are careful to preserve smoothness.
On the other hand, it is obvious that $S^2$ is not smoothly contractible since we know more generally that it is not continuously contractible.


\pnum{Exercise 1.6.7}

Per the hint, let $R_t : \R^2 \to \R^2$ be the the linear map defined by the rotation matrix
\[
    [R_t] = \mat{\cos\pi t & - \sin{\pi t} \\ \sin{\pi t} & \cos{\pi t}}.
\]
Then $R$ restricts to a homotopy of maps $S^2 \to S^2$ corresponding to the matrices
\[
    [R_0] = \mat{1 & 0 \\ 0 & 1} = I_2
    \isp{and}
    [R_1] = \mat{0 & 1 \\ -1 & 0} = -I_2.
\]
The former is corresponds to the identity map and the latter to the antipodal map.
Then for $k = 2n - 1$ with $n \in \N$, we have $S^k \seq \R^{2n}$ and define the linear map $F_t : \R^{2n} \to \R^{2n}$ by the block matrix built out of $n$ copies of $[R_t]$ along the diagonal (and zeros elsewhere):
\[
    [F_t] = \mat{[R_t] && 0 \\ & \ddots \\ 0 && [R_t]}.
\]
As in the previous step, $F$ restricts to a homotopy of maps $S^k \to S^k$ corresponding to 
\[
    [F_0] = \mat{I_2 && 0 \\ & \ddots \\ 0 && I_2} = I_{2n}
    \isp{and}
    [F_1] = \mat{-I_2 && 0 \\ & \ddots \\ 0 && -I_2} = -I_{2n}.
\]
Also like the previous step, the former is corresponds to the identity map and the latter to the antipodal map.




\pnum{Exercise 2.1.7}

Let $x \in \bd X$ and take parameterizations $\phi : U \to X$ and $\psi : V \to X$ with $\phi(0) = x = \psi(0)$, where $U$ and $V$ are open subsets of $H^k$.
By shrinking $U$ and $V$ we may assume without loss of generality that $\phi(U) = \psi(V)$.
Then $h = \psi^{-1} \circ \phi : U \to V$ is a diffeomorphism, so its derivative $\dd{h}_0$ is an isomorphism of tangent spaces
\[
    \R^k = T_0(U) \longrightarrow T_0(V) = \R^k.
\]
We claim that $\dd{h}_0(H^k) = H^k$.
To show this, notice that we can write
\[
    H^k = \bd H^k + \R_{\geq 0}e_k,
\]
where $\bd H^k = \R^{k-1}$ and $\R_{\geq 0}e_k$ is the nonnegative multiples of $e_k$.
The linearity of $\dd{h}_0$ gives
\[
    \dd{h}_0(H^k)
        = \dd{h}_0(\bd H^k + \R_{\geq 0}e_k)
        = \dd{h}_0(\bd H^k) + \R_{\geq 0}\dd{h}_0(e_k)
\]

Now considering boundaries, the map $\bd h = \bd\psi^{-1} \circ \bd\phi : \bd U \to \bd V$ is a diffeomorphism with $\bd U$ and $\bd V$ both open sets in $\bd H^k = \R^{k-1}$.
In other words, $\bd h$ is a parameterization of $\bd V$ as a $(k-1)$-manifold, so
\[
    \dd{h}_0(\bd H^k)
        = \im \dd(\bd h)_0
        = T_0\bd V
        = T_0\bd H^k
        = \bd H^k.
\]

We now inspect the directional derivative
\[
    \dd{h}_0(e_k)
        = \mathrm{D}_{e_k}h(0)
        = \lim_{t \to 0} \frac{h(0 + te_k) - h(0)}{t}
        = \lim_{t \to 0} \frac{h(te_k)}{t}
\]
For all $t > 0$ we have $te_k \in H^k$.
Since $U$ is an open neighborhood of $0$ in $H^k$, then for small $t > 0$ we have $te_k \in U$.
Then $h(te_k) \in V \seq H^k$, and scaling by $1/t > 0$ keeps the value inside of $H^k$.
Since $H^k$ is closed as a subset $\R^k$, we have
\[
    \dd{h}_0(e_k) = \lim_{t \to 0} \frac{h(te_k)}{t} \in H^k.
\]
Moreover, since $e_k$ is not contained in the invariant subspace $\bd H^k = \dd{h}_0(\bd H^k)$ and $\dd{h}_0$ is a linear isomorphism, $\dd{h}_0(e_k)$ must also not be in $\bd H^k$.
In other words, $\dd{h}_0(e_k)$ is is the (strictly) positive upper half-space $H^k \setminus \bd{H^k}$, so
\[
    \dd{h}_0(H^k) = \bd H^k + \R_{\geq 0}\dd{h}_0(e_k) = H^k.
\]
We now conclude that
\[
    \dd{\phi}_0(H^k) = \dd{\psi}_0(\dd{h}_0(H^k)) = \dd{\psi}_0(H^k).
\]


\pnum{Exercise 2.1.8}

Since $T_x\bd X$ is a subspace of codimension in $T_xX$, then the orthogonal complement $(T_x\bd X)^\perp$ is a $1$-dimensional subspace of $T_xX$.
A $1$-dimensional real vector space has exactly two unit vectors and they are opposites of each other.
That is, if $v \in (T_x\bd X)^\perp$ is a unit vector, the other unit vector is $-v$.
In the proof of Exercise 2.1.7, we found that $H_xX$ is a half-space of $T_xX$ whose boundary is precisely $T_x\bd X$.
By construction, exactly one of $\pm v$ is contained in this half-space.

For $X = H^k$, we have $\vec{n}_{H^k}(x) = -e^k$ for all $x \in \bd H^k$.
In other words, $\vec{n}_{H^k}$ is a constant map, which is in particular smooth.
For $x \in \bd X$, let $\phi : U \to X$ be a local parameterization with $\phi(0) = x$.
For each $u \in U$, the derivative $\dd{\phi}_u : \R^k \to T_{\phi(u)}X$ is an isomorphism of vector spaces.
There is a map
\begin{align*}
    U &\longrightarrow X \times (\R^N)^k \\
    u &\longmapsto \big(\phi(u), (\dd{\phi}_u(e_1), \dots, \dd{\phi}_u(e_k))\big),
\end{align*}
where each point $\phi(u)$ is paired with a basis of the tangent space $T_{\phi(u)}X$.
This map is smooth, as it can be constructed out of restrictions of the global derivative map $\dd{\phi} : T(U) \to R(X)$ between tangent bundles.

For any vector space $V$ and linearly independent set $v_1, \dots, v_k$, we can perform the Gram-Schmidt process to obtain an orthonormal set $w_1, \dots, w_k$ such that the span of $v_1, \dots, v_i$ is the same as the span of $w_1, \dots, w_i$ for each $i = 1, \dots, k$.
The subset $Y \seq (\R^N)^k$ consing only of $k$-tuples of linearly independent vectors forms a manifold and the map $Y \to Y$ which performs the Gram-Schmidt process is a smooth map.

Applying this process to $\dd{\phi}_u(e_1), \dots, \dd{\phi}_u(e_k)$ gives us $w_1, \dots, w_k$ with $w_1, \dots, w_{k-1}$ spanning $T_{\phi(u)}\bd X$ and $w_k$ a unit vector in $T_{\phi(u)}X$ orthogonal to $T_{\phi(u)}\bd X$.
Moreover, from the proof of Exercise 2.1.7, we know that $\dd{\phi}_u(e_k)$ and therefore $w_k$ are both contained in $H_{\phi(u)}X$, so in fact $\vec{n}(\phi(u)) = -w_k$.

Composing all the necessary maps gives $\vec{n}$ as a smooth map.

\pnum{Exercise 2.1.9}

\pnum{(a)}

We prove that $\inter X = X \setminus \bd X$ is open.
Let $x \in \inter X$ and choose a local parameterization $\phi : U \to X$ where $U$ is an open set in $H^k$.
Since $x$ is not on the boundary of $X$, the point $u \in U$ with $\phi(u) = x$ must not be on the boundary of $H^k$.
Since $\bd H^k = \R^{k-1}$ is a closed subspace of $\R^k$, there is a small enough neighborhood $V \seq \R^k$ of $u$ such that $V \cap \R^{k-1} = \emptyset$.
Then $W = U \cap V$ is an open set in $H^k$ with $W \cap \bd H^k = \emptyset$.

Then $\phi(W)$ is an open neighborhood of $x$ in $X$.
Moreover, $\phi^{-1}$ gives a chart of $\phi(W)$ such that $\phi^{-1}(y) \notin \bd H^k$ for all $y \in \phi(W)$.
By Exercise 2.1.1, we conclude that $\phi(W)$ and $\bd X$ are disjoint so $\phi(W) \in \inter X$, hence $\inter X$ is open.


\pnum{(b)}

Trivially, any non-compact manifold without boundary, since the emptyset is compact.

The $1$-manifold $H^1 \seq \R^1$ is not compact since it is unbounded, but $\bd H^1 = \{0\}$ is compact.




\end{document}