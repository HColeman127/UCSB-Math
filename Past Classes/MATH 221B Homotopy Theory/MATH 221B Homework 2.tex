\documentclass[12pt]{article}

% Packages
\usepackage[margin=1in]{geometry}
\usepackage{fancyhdr, parskip}
\usepackage{amsmath, amsthm, amssymb}
\usepackage{tikz, tikz-cd}
\usepackage[shortlabels]{enumitem}

% Page Style
\makeatletter
\fancypagestyle{title}{
    \renewcommand{\headrulewidth}{0.4pt}
    \setlength{\headheight}{15pt}
    \fancyhead[R]{\@author}
    \fancyhead[L]{\@title}
    \fancyhead[C]{\@date}
}
\makeatother
\renewcommand{\maketitle}{\thispagestyle{title}}
\fancypagestyle{plain}{
    \fancyhf{}
    \renewcommand{\headrulewidth}{0pt}
    \renewcommand{\footrulewidth}{0pt}
    \fancyfoot[R]{\thepage}
}
\pagestyle{plain}

% Problem Box
\setlength{\fboxsep}{4pt}
\newlength{\myparskip}
\setlength{\myparskip}{\parskip}
\newsavebox{\savefullbox}
\newenvironment{fullbox}{\begin{lrbox}{\savefullbox}\begin{minipage}{\dimexpr\textwidth-2\fboxsep\relax}\setlength{\parskip}{\myparskip}}{\end{minipage}\end{lrbox}\framebox[\textwidth]{\usebox{\savefullbox}}}
\newenvironment{pbox}[1][]{\begin{fullbox}\ifx#1\empty\else\paragraph{#1}\phantom{}\fi}{\end{fullbox}}

% Theorem Environments
\theoremstyle{definition}
\newtheorem{lemma}{Lemma}

% Tikz Environments
\newenvironment{drawing}{\begin{center}\begin{tikzpicture}}{\end{tikzpicture}\end{center}}
% \tikzcdset{row sep/normal=0pt}
\newenvironment{cd}{\begin{center}\begin{tikzcd}}{\end{tikzcd}\end{center}}

% Default Commands
\newcommand{\isp}[1]{\quad\text{#1}\quad}
\newcommand{\N}{\mathbb{N}} 
\newcommand{\Z}{\mathbb{Z}}
\newcommand{\Q}{\mathbb{Q}}
\newcommand{\R}{\mathbb{R}}
\newcommand{\C}{\mathbb{C}}
\newcommand{\A}{\mathbb{A}}
\renewcommand{\P}{\mathbb{P}}
\newcommand{\eps}{\varepsilon}
\renewcommand{\phi}{\varphi}
\renewcommand{\emptyset}{\varnothing}
\newcommand{\<}{\langle}
\renewcommand{\>}{\rangle}
\newcommand{\isom}{\cong}
\newcommand{\eqc}{\overline}
\newcommand{\clo}{\overline}
\newcommand{\seq}{\subseteq}
\newcommand{\teq}{\trianglelefteq}
\DeclareMathOperator{\id}{id}
\DeclareMathOperator{\im}{im}

% Extra Commands
\newcommand{\inc}{\hookrightarrow}
\newcommand{\htpy}{\simeq}
\newcommand{\bd}{\partial}


% Document
\begin{document}
\title{MATH 221B Homework 2}
\author{Harry Coleman}
\date{January 24, 2022}
\maketitle


\begin{pbox}[1 Exercise 0.9]
    Show that a retract of a contractible space is contractible.
\end{pbox}

\begin{proof}
    Suppose $X$ is a contractible space with $A \seq X$ a retract.
    Let $F : X \times I \to X$ be a contraction, i.e., a map with $F_0 = \id_X$ and $F_1(X) = \{x_0\}$ for some point $x_0 \in X$.
    And let $r : X \to A$ be a retract, i.e., a map with $r|_A = \id_A$.

    We define a homotopy $G : A \times I \to A$ by the following composition:
    \begin{cd}
        A \times I \rar[hookrightarrow] & X \times I \rar["F"] & X \rar["r"] & A.
    \end{cd}
    By construction, for each $a \in A$, we have
    \[
        G_0(a) = r(F_0(a)) = r(a) = a
    \]
    and
    \[
        G_1(a) = r(F_1(a)) = r(x_0).
    \]
    In other words, $G_0 = \id_A$ and $G_1(A) = \{r(x_0)\}$, hence $G$ describes a contraction of $A$ to the point $r(x_0) \in A$.
\end{proof}

\newpage
\begin{pbox}[2 Exercise 0.10]
    Show that a space $X$ is contractible iff every map $f : X \to Y$, for arbitrary $Y$, is nullhomotopic.
\end{pbox}

\begin{proof}
    Suppose $X$ has a contraction $R : X \times I \to X$, and consider a map $f : X \to Y$.
    Then the composition $fR = f \circ R : X \times I \to Y$ defines a homotopy between the maps
    \[
        (fR)_0 = f \circ R_0 = f \circ \id_X = f
    \]
    and
    \[
        (fR)_1 = f \circ R_1,
    \]
    the latter of which is a constant map since $R_1$ is constant.
    Hence, $f$ is nullhomotopic.

    On the other hand, if every map $X \to Y$ is nullhomotopic, then in particular the identity $\id_X : X \to X$ is nullhomotopic, which means $X$ is contractible.
\end{proof}

\begin{pbox}
    Similarly, show $X$ is contractible iff every map $f : Y \to X$ is nullhomotopic.
\end{pbox}

\begin{proof}
    Suppose $X$ has a contraction $R : X \times I \to X$, and consider a map $f : Y \to X$.
    Then the composition $Rf = R \circ (f \times \id_I) : Y \times I \to X$ defines a homotopy between the maps
    \[
        (Rf)_0 = R_0 \circ f = \id_X \circ f = f
    \]
    and
    \[
        (fR)_1 = R_1 \circ f,
    \]
    the latter of which is a constant map since $R_1$ is constant.
    Hence, $f$ is nullhomotopic.

    On the other hand, if every map $Y \to X$ is nullhomotopic, then in particular the identity $\id_X : X \to X$ is nullhomotopic, which means $X$ is contractible.
\end{proof}



\newpage
\begin{pbox}[3 Exercise 0.11]
    Show that $f : X \to Y$ is a homotopy equivalence if there exist maps $g, h : Y \to X$ such that $fg \htpy \id_Y$ and $hf \htpy \id_X$.
    
\end{pbox}

\begin{proof}
    Suppose $G : Y \times I$ is a homotopy with $G_0 = f \circ g$ and $G_1 = \id_Y$.
    Similarly, suppose $H : X \times I \to X$ is a homotopy with $H_0 = \id_X$ and $H_1 = h \circ f$.
    We represent $G$ and $H$ with the following diagram (which commutes only in the sense that difference paths through the diagram are homotopic, rather than equal):
    \begin{cd}[column sep=large]
        Y
        \rar["g"]
        \ar[rr, bend right=40, "\id_Y"' name=BotMor]
        &
        | [alias=MidX] | X
        \ar[from=MidX, to=BotMor, shorten=1.5mm, Rightarrow, "G" xshift=1mm]
        \rar["f"]
        \ar[rr, bend left=40, "\id_X" name=TopMor]
        &
        | [alias=MidY] | Y
        \rar["h"]
        \ar[from=TopMor, to=MidY, shorten=1.5mm, Rightarrow, "H" xshift=1mm]
        &
        X
    \end{cd}
    Intuitively, the diagram suggest a homotopy between $g$ (the top path in the diagram) and $h$ (the bottom path).
    To make this explicit, we construct the following ``horizontal compositions'' of homotopies (also used in Problem 2):
    \[
        Hg = H \circ (g \times \id_I) : Y \times I \to X
    \]
    and
    \[
        hG = h \circ G : Y \times I \to X.
    \]
    The first of these describes a homotopy between the maps
    \[
        (Hg)_0 = H_0 \circ g = \id_X \circ g = g
    \]
    and
    \[
        (Hg)_1 = H_1 \circ g = (h \circ f) \circ g = h \circ f \circ g.
    \]
    The second describes a homotopy between the maps
    \[
        (hG)_0 = h \circ G_0 = h \circ (f \circ g) = h \circ f \circ g
    \]
    and
    \[
        (hG)_1 = h \circ G_1 = h \circ \id_Y = h.
    \]
    The ``vertical compositions'' of these homotopies then gives us
    \[
        g \htpy h \circ f \circ g \htpy h.
    \]

    Lastly, if $K : Y \times I \to X$ is a homotopy with $K_0 = h$ and $K_1 = g$, then we could draw the following diagram of homotopies:
    \begin{cd}[column sep=large]
        X
        \rar["f"]
        \ar[rr, bend left=40, "\id_X" name=Moridx]
        &
        | [alias=ObY] | Y
        \rar["h" name = Morh]
        \ar[from=Moridx, to=ObY, shorten=1.5mm, Rightarrow, "H" xshift=1mm]
        \rar[bend right=50, "g"' name=Morg]
        \ar[from=Morh, to=Morg, shorten=1.5mm, Rightarrow, "K" xshift=1mm]
        &
        X
    \end{cd}
    Again, the diagram suggests a homotopy between $\id_X$ and $g \circ f$ which we will make explicit.
    Constructing the horizontal composition
    \[
        Kf = K \circ (f \times \id_I) : X \times I \to X
    \]
    gives us a homotopy between the maps
    \[
        (Kf)_0 = K_0 \circ f = h \circ f
    \]
    and
    \[
        (Kf)_1 = K_1 \circ f = g \circ f.
    \]
    Vertically composing this homotopy with $H$ allows us to conclude
    \[
        \id_X \htpy h \circ f \htpy g \circ f.
    \]
    By assumption, $f \circ g \htpy \id_Y$, so in fact $g$ is a homotopy inverse of $f$, hence $f$ is a homotopy equivalence.
\end{proof}

\begin{pbox}
    More generally, show that $f$ is a homotopy equivalence if $fg$ and $hf$ are homotopy equivalences.
\end{pbox}

\begin{proof}
    Suppose $k : Y \to Y$ and $\ell : X \to X$ are homotopy inverses to $f \circ g$ and $h \circ f$, respectively.
    Then let $G : Y \times I \to Y$ and $H : X \times I \to X$ be homotopies with
    \[
        f \circ g \circ k = G_0 \htpy G_1 = \id_Y
    \]
    and
    \[
        \id_X = H_0 \htpy H_1 = \ell \circ h \circ f.
    \]
    We describe the situation we the following diagram:
    \begin{cd}
        Y \rar["k"]
            \ar[rrr, bend right=35, "\id_Y"' name=idY]
        & Y \rar["g" name=g]
            \ar[from=g, to=idY, shorten=3mm, Rightarrow, "G" xshift=1mm]
        & X \rar["f"]
            \ar[rrr, bend left=40, "\id_X" name=idX]
        & Y \rar["h" name=h]
            \ar[from=idX, to=h, shorten=2mm, Rightarrow, "H" xshift=1mm] 
        & X \rar["\ell"] 
        & X
    \end{cd}
    Notice that this is essentially the same diagram as the one we used in the first part, but with $g$ and $h$ replaced with $g \circ k$ and $\ell \circ h$, respectively.
    Moreover, we can use the same techniques of horizontal and vertical composition of homotopies to obtain
    \[
        g \circ k \htpy (g \circ k) \circ f \circ (\ell \circ h) \htpy \ell \circ h.
    \]
    And again, if $K : Y \times I \to X$ is a homotopy with
    \[
        \ell \circ h = K_0 \htpy K_1 = g \circ k,
    \]
    then we can draw the following diagram of homotopies:
    \begin{cd}[column sep=large]
        X \rar["f"] 
            \ar[rr, bend left=40, "\id_X" name=idX]
        &|[alias=Y]| Y \rar["\ell \circ h" name=loh]
            \ar[from=idX, to=Y, shorten=1.5mm, Rightarrow, "H" xshift=1mm]
            \rar[bend right=50, "g \circ k"' name=gok]
            \ar[from=loh, to=gok, shorten=1.5mm, Rightarrow, "K" xshift=1mm]
        & X
    \end{cd}
    This, in turn, gives us a homotopy
    \[
        \id_X \htpy (\ell \circ h) \circ f \htpy (g \circ k) \circ f.
    \]
    By assumption, $f \circ (g \circ k) \htpy \id_Y$, so in fact $g \circ k$ is a homotopy inverse of $f$, hence $f$ is a homotopy equivalence.
\end{proof}


\newpage
\begin{pbox}[4 Exercise 0.12]
    Show that a homotopy equivalence $f : X \to Y$ induces a bijection between the set of path-components of $X$ and the set of path-components of $Y$, and that $f$ restricts to a homotopy equivalence from each path-component of $X$ to the corresponding path-component of $Y$.
\end{pbox}

\begin{proof}
    We claim that a pair of points $a, b \in X$ are path-connected in $X$ if and only if their images $f(a)$ and $f(b)$ are path-connected in $Y$.

    On one hand, if $\gamma : I \to X$ is a path from $a$ to $b$, then the composition $f \circ \gamma : I \to Y$ is a path from $f(a)$ to $f(b)$.

    On the other hand, suppose $\gamma : I \to Y$ is a path from $f(a)$ to $f(b)$.
    Additionally, let $g : Y \to X$ be a homotopy inverse of $f$, and let $H : X \times I \to X$ be a homotopy
    \[
        \id_X = H_0 \htpy H_1 = g \circ f.
    \]
    We construct maps $\alpha, \beta : I \to X$ by
    \[
        \alpha(t) = H_t(a) \isp{and} \beta(t) = H_t(b).
    \]
    That is $\alpha$ is a path from
    \[
        \alpha(0) = H_0(a) = \id_X(a) = a \isp{to} \alpha(1) = H_1(a) = g(f(a))
    \]
    and $\beta$ is a path from
    \[
        \beta(0) = H_0(b) = \id_X(b) = b \isp{to} \beta(1) = H_1(b) = g(f(b)).
    \]
    Lastly, $g \circ \gamma : I \to X$ is a path from $g(f(a))$ to $g(f(b))$.
    Hence, there is a path from $a$ to $b$ in $X$, obtained as the product of paths
    \[
        \alpha \cdot (g \circ \gamma) \cdot \overline{\beta},
    \]
    where $\overline{\beta}$ is the inverse path of $\beta$, i.e., $\overline{\beta}(t) = \beta(1 - t)$.

    Therefore, it is indeed true that $a, b \in X$ are path-connected in $X$ if and only if $f(a)$ and $f(b)$ are path-connected in $Y$.
    It follows that there is a well-defined injective function
    \begin{align*}
        \tilde{f} : \{PC(x) \mid x \in X\} &\longrightarrow \{PC(y) \mid y \in Y\}, \\
            PC(x) &\longmapsto PC(f(x)),
    \end{align*}
    where $PC(-)$ denotes the path-component of a point in its respective space.
    (Note $\tilde{f}$ is well-defined and injective since $PC(a) = PC(b)$ if and only if $PC(f(a)) = PC(f(b))$.)
    By the same argument, there is a well-defined injective function
    \begin{align*}
        \tilde{g} : \{PC(y) \mid y \in Y\} &\longrightarrow \{PC(x) \mid x \in X\}, \\
            PC(y) &\longmapsto PC(g(y)).
    \end{align*}
    
    In fact, $\tilde{f}$ and $\tilde{g}$ are inverses of each other.
    A portion of the above argument shows that any given point $x \in X$ is path-connected to $g(f(x))$, by a restriction of the homotopy $\id_X \htpy g \circ f$.
    In particular, this means
    \[
        PC(x) = PC(g(f(x))) = \tilde{g}(\tilde{f}(PC(x)))
    \]
    for all $x \in x$, and similarly that
    \[
        PC(y) = PC(f(g(y))) = \tilde{f}(\tilde{g}(PC(y))),
    \]
    for all $y \in Y$.
    Hence, $\tilde{f}$ is a bijection with $\tilde{g}$ its inverse (where $g$ is any homotopy inverse).
\end{proof}

\begin{pbox}
    Prove also the corresponding statements with components instead of path-components.
\end{pbox}

\begin{proof}
    If $C \seq X$ is a connected component, the continuous image under $f$ is connected in $Y$.
    Then $f(C)$ is contained in some connected component of $Y$, say $\hat{f}(C)$.
    This gives us a well-defined function
    \begin{align*}
        \hat{f} : \{\text{connected components of } X\} &\longrightarrow \{\text{connected components of } Y\}, \\
            C &\longmapsto \hat{f}(C).
    \end{align*}
    Similarly, we have another function
    \begin{align*}
        \hat{g} : \{\text{connected components of } Y\} &\longrightarrow \{\text{connected components of } X\}, \\
            C &\longmapsto \hat{g}(C).
    \end{align*}

    For a connected component $C \seq X$ and a point $x \in C$, we have
    \[
        f(x) \seq f(C) \seq \hat{f}(C).
    \]
    Note that since $\hat{f}(C)$ is a connected component and $PC(f(x))$ is a connected subspace of $Y$ intersecting $\hat{f}(C)$, then we must have
    \[
        \tilde{f}(PC(x)) = PC(f(x)) \seq \hat{f}(C).
    \]
    By the same argument and the fact that $\tilde{f}$ and $\tilde{g}$ are inverses, we have
    \[
        PC(x) = \tilde{g}(\tilde{f}(PC(x))) \subseteq \hat{g}(\hat{f}(C)).
    \]
    Since $C$ and $\hat{g}(\hat{f}(C))$ are connected components of $X$ which intersect (at least at $x$), then in fact $C = \hat{g}(\hat{f}(C))$.
    Since $\hat{f}$ and $\hat{g}$ are symmetric, we conclude that they are inverses, and therefore describe a bijection.
\end{proof}

\begin{pbox}
    Deduce from this that if the components and path-components of a space coincide, then the same is true for any homotopy equivalent space.
\end{pbox}

Since a homotopy equivalence induces a bijection between both the path-components and the connected components of each space, the two types of components coincide for one space if and only if they coincide for the other.


\newpage
\begin{pbox}[5 Exercise 0.13]
    Show that any two deformation retractions $r^0_t$ and $r^1_t$ of a space $X$ onto a subspace $A$ can be joined by a continuous family of deformation retractions $r^s_t$, $0 \leq s \leq 1$, of $X$ onto $A$, where continuity means that the map $X \times I \times I \to X$ sending $(x, s, t)$ to $r^s_t(x)$ is continuous.
\end{pbox}

\begin{proof}
    Denote $Y = X \times I$, so that $r^0, r^1 : Y \to X$.
    We define a map $F : Y \times I \to X$ by
    \[
        F^s(x, t) = r^0_t(r^1_{st}(x)).
    \]
    (One may construct $F$ explicitly as the result of combining continuous maps with continuity-preserving operations such as composition, product, multiplication, etc.)
    Then $F$ describes a sort of ``second-order'' homotopy between
    \[
        F^0 = r^0 \circ r^1_0 = r^0 \circ \id_X = r^0
    \]
    and
    \[
        F^1 = r^0 \circ r^1.
    \]
    Note that each $F^s : X \times I \to X$ is a homotopy with $F^s_t(x) = F^s(x, t)$.
    We find that
    \[
        F^s_0 = r^0_0 \circ r^1_0 = \id_X \circ \id_X = \id_X.
    \]
    For all $a \in A$ we have
    \[
        F^s_t(a) = r^0_t(r^1_{st}(a)) = r^0_t(a) = a,
    \]
    since $r^0_t|_A = r^1_t|_A = \id_A$ for all $t$, i.e., $F^s_t|_A = \id_A$ for all $t$.
    And for all $x \in X$ we have
    \[
        F^s_1(x) = R_s(x, 1) = r^0_1(r^1_s(x)) \in A,
    \]
    since $r^1_s(x) \in X$ and $r^0_1(X) \seq A$.
    In other words, each $F^s : X \times I \to X$ is a deformation retraction of $X$ onto $A$.
    We conclude that $R$ describes a second-order homotopy $r^0 \htpy r^0 \circ r^1$, where each intermediate homotopy $F^s$ is a deformation retraction of $X$ onto $A$.

    We define another map $G : Y \times I \to X$ by
    \[
        G^s(x, t) = r^0_{st}(r^1_t(x)),
    \]
    which describes a second-order homotopy
    \[
        r^1 = G^0 \htpy G^1 = r^0 \circ r^1.
    \]
    One can check that, once again, each intermediate homotopy $G^s : X \times I \to X$ is a deformation retraction of $X$ onto $A$.
    Composing these second-order homotopies, we obtain
    \[
        r^0 \htpy r^0 \circ r^1 \htpy r^1,
    \]
    via deformation retractions of $X$ onto $A$.
\end{proof}

\newpage
\begin{pbox}[6 Exercise 0.17]
    Construct a $2$-dimensional cell complex that contains both an annulus $S^1 \times I$ and a M\"obius band as deformation retracts.
\end{pbox}

Note that a quotient of a cell complex where we identify a pair of closed $n$-cells is still a cell complex, as we could reconstruct the new cell complex by adding in only a single $n$-cell for the chosen pair, identifying smaller cells as necessary and modifying all higher attaching maps to reflect this.

For example, we could construct a filled-in square as a $2$-dimensional cell complex as follows:
\begin{drawing}
    \draw[fill=gray!20] (0,0) rectangle (2,2);
    \foreach \x in {0, 2} {
        \foreach \y in {0, 2} {
            \filldraw (\x, \y) circle (2pt);
        }
    }
\end{drawing}
We could then identify the pair of horizontally opposing $1$-cells to obtain an annulus:
\begin{drawing}
    \begin{scope}[xshift=-4cm, yshift=-1cm]
        \draw[fill=gray!20] (0,0) rectangle (2,2);
        \foreach \x in {0, 2} {
            \foreach \y in {0, 2} {
                \filldraw (\x, \y) circle (2pt);
            }
        }
        \draw[->] (0, 0) -- (0, 1);
        \draw[->] (2, 0) -- (2, 1);
    \end{scope}

    \node at (-1, 0) {$\leadsto$};

    \draw[fill=gray!20] (1.5, 0) circle (1.5);
    \draw[fill=white] (1.5, 0) circle (0.5);
    \filldraw (0, 0) circle (2pt) -- (1, 0) circle (2pt);
\end{drawing}

With this in mind, we construct a cell complex as the quotient of the following cell complex, where $1$-cells are identified by name:
\begin{drawing}
    \begin{scope}[xshift=-2cm]
        \draw[fill=gray!20] (0,0) rectangle (2,2);
        \draw (0, 1) -- (2, 1) node[midway, anchor=south] {$e$};
        \foreach \x in {0, 2} {
            \foreach \y in {0, 1, 2} {
                \filldraw (\x, \y) circle (2pt);
            }
        }

        \node[anchor=east] at (0, 1.5) {$a$};
        \node[anchor=west] at (2, 1.5) {$a$};
        \node[anchor=east] at (0, 0.5) {$b$};
        \node[anchor=west] at (2, 0.5) {$b$};

        \draw[->] (0, 0) -- (0, 0.5);
        \draw[->] (0, 1) -- (0, 1.5);
        \draw[->] (2, 0) -- (2, 0.5);
        \draw[->] (2, 1) -- (2, 1.5);
    \end{scope}
    \begin{scope}[xshift=2cm]
        \draw[fill=gray!20] (0,0) rectangle (2,2);
        \draw (0, 1) -- (2, 1) node[midway, anchor=south] {$e$};
        \foreach \x in {0, 2} {
            \foreach \y in {0, 1, 2} {
                \filldraw (\x, \y) circle (2pt);
            }
        }
        
        \node[anchor=east] at (0, 1.5) {$c$};
        \node[anchor=west] at (2, 1.5) {$d$};
        \node[anchor=east] at (0, 0.5) {$d$};
        \node[anchor=west] at (2, 0.5) {$c$};

        \draw[->] (0, 0) -- (0, 0.5);
        \draw[->] (0, 1) -- (0, 1.5);
        \draw[->] (2, 2) -- (2, 1.5);
        \draw[->] (2, 1) -- (2, 0.5);
    \end{scope}
\end{drawing}
The left component becomes the annulus and the right component becomes the M\"obius band.
Each component can deformation retract onto the shared centerline $e$.
Performing one of these deformation retractions while fixing the other component induces a deformation retraction of the whole complex onto the fixed component.

\newpage
\begin{pbox}[7 Exercise A.3]
    Show that a CW complex is path-connected iff its 1-skeleton is path-connected.
\end{pbox}

\begin{proof}
    Assume $X = \bigcup_k X^k$ is a path-connected CW complex.

    Let $\gamma : I \to X$ be a path in $X$.
    Since $I$ is compact and $\gamma$ is continuous, the image of $\gamma$ is compact subspace of $X$. By Hatcher Proposition A.1, $\gamma(I)$ is contained in some finite subcomplex of $X$---in particular, in some $k$-skeleton.
    We conclude that every path in $X$ is contained in some $k$-skeleton.

    Let $x, y \in X^1$ be arbitrary distinct points in the $1$-skeleton.
    Choose the minimum $k \geq 1$ such that $X^k$ contains a path between $x$ and $y$; let $\gamma : I \to X^k$ be such a path.
    (Note that $k$ is well-defined since some path in $X$ exists, and the previous result tells us the path is contained in some skeleton.)
    Assume, for contradiction, that $k > 1$.
    Again, $\gamma(I) \seq X$ is compact, and therefore contained in finitely many cells.
    In particular, $\gamma$ passes through finitely many $k$-cells $e^k_1, \dots, e^k_n$.

    Since $k \geq 2$, the boundary $\bd D^k_i$ of each of these $k$-cells is path-connected.
    Moreover, $x$ and $y$ are not in any of these $k$-cells, which means $\gamma$ does not start or end inside of a $k$-cell.
    Therefore, we can construct a new path between $x$ and $y$ by following $\gamma$ inside of $X^{k-1}$, but anywhere $\gamma$ would pass through a $k$-cell $e^k_i$, we replace with a path through the boundary $\bd D^k_i$. This is possible since $\gamma$ can only enter/exit the $k$-cell through its boundary, and any time $\gamma$ enters a $k$-cell, it must eventually exit.

    Since the boundaries $\bd D^k_i$ are contained in $X^{k-1}$, this new path is completely contained in $X^{k-1}$.
    But this contradicts the choice of $k$, implying that $k = 1$.
    Hence, $x$ and $y$ are connected via a path in $X^1$, and we conclude that $X^1$ is path-connected.

    Assume now that $X = \bigcup_k X^k$ is a CW complex with $X^1$ path-connected.
    We show by induction on $k$ that each $k$-skeleton is path-connected.
    Assume $X^{k-1}$ is path-connected, and $X^k$ is formed by attaching $k$-cells $e^k_\alpha$ to $X^{k-1}$.
    For a point $x \in X^k$, either $x \in X^{k-1}$ or $x \in e^k_\alpha$ for some $\alpha$.
    In the latter case, $e^k_\alpha$ is path-connected to its boundary, which is contained in $X^{k-1}$.
    Hence, every point of $X^k$ is path-connected to $X^{k-1}$.
    Then a path between any two points of $X^k$ is obtained by first connecting each point to a point in $X^{k-1}$ (if necessary), then finding a path connecting these points $X^{k-1}$.
    Thus, $X^k$ is path connected, which completes the induction.
    Then any two points in $X$ are contained in some $k$-skeleton, which we have just shown to be path-connected.
    The path in $X^k$ is also a path in $X$, hence $X$ is path-connected.
\end{proof}


\newpage
\begin{pbox}[8 Exercise A.4]
    Show that a CW complex is locally compact iff each point has a neighborhood that meets only finitely many cells.
\end{pbox}

\begin{proof}
    Let $X$ be a CW complex.
    In particular, $X$ is Hausdorff, so we have the following equivalent criteria for $X$ to be locally compact:
    \begin{enumerate}[(i)]
        \item Each point of $X$ has a compact neighborhood contained in any given neighborhood.
        \item Each point of $X$ has a compact neighborhood.
    \end{enumerate}
    Definition (i) is given by Hatcher, but we will use (ii) for this proof since, as noted, they are equivalent for Hausdorff spaces.

    Suppose $X$ is locally compact.
    Let $x \in X$ and let $K \seq X$ be a compact neighborhood of $x$.
    By Proposition A.1, we know that $K$ is contained in a finite subcomplex of $X$.
    In particular, $K$ meets only finitely many cells of $X$.

    Suppose $x \in X$ has a neighborhood $N \seq X$ that meets only finitely many cells $e_1, \dots, e_n$ of $X$.
    If $e_i$ is a $k$-cell, then its closure $\clo{e_i} \seq X$ is homeomorphic to a closed ball in $\R^k$, which is compact.
    Therefore the finite union of compact sets $K = \clo{e_i} \cup \cdots \cup \clo{e_n}$ is compact.
    And since $N$ is a neighborhood of $x$ contained in $K$, then $K$ is a compact neighborhood of $x$, as desired.
\end{proof}

\end{document}