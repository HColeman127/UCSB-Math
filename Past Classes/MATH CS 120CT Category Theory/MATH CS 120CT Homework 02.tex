\documentclass[12pt]{article}

% packages
\usepackage[margin=1in]{geometry}
\usepackage{framed}
\usepackage[table]{xcolor}
\usepackage{colortbl, multirow}
\usepackage{amsmath,amsthm,amssymb,wasysym}
\usepackage{mathrsfs, mathtools}
\usepackage{tikz,pgf,pgfplots}
\usetikzlibrary{arrows, angles, quotes, decorations.pathreplacing, math, patterns, calc}
\usepackage{graphicx}

% custom commands
\newcommand{\N}{\mathbb{N}}
\newcommand{\Z}{\mathbb{Z}}
\newcommand{\I}{\mathbb{I}}
\newcommand{\R}{\mathbb{R}}
\newcommand{\Q}{\mathbb{Q}}
\newcommand{\p}{^{\prime}}
\newcommand{\powerset}{\raisebox{.15\baselineskip}{\Large\ensuremath{\wp}}}
\DeclarePairedDelimiter{\ceil}{\lceil}{\rceil}
\DeclarePairedDelimiter\floor{\lfloor}{\rfloor}

 
\begin{document}
 
\title{Homework 2\\
    \large MATH CS 120CT Category Theory}
\author{Harry Coleman}
\date{January 10, 2020}

\maketitle

\section*{Exercise 1}
\fbox{
    \parbox{\textwidth} {
         Let $f:\N\rightarrow\N$ be $f(x)=2x$. What is its left adjoint? What is its right adjoint?
    }
}
\\

In the natural numbers, with their typical ordering of $\leq$, $f$ is clearly a monotone map. Since for any two $x,y\in\N$ such that $x\leq y$, we find that $2x\leq2x$ so $f(x)\leq f(y)$. Now let
\[g:\N\rightarrow\N \text{ with } g(x)=\floor{x/2}.\]
This is also monotone, since for any two $x,y\in\N$ such that $x\leq y$, we have $\floor{x/2}\leq\floor{y/2}$, so $g(x)\leq g(y)$. We assert that $f$ and $g$ are the left and right adjoint, respectively, for a Galois connection. For $f$ and $g$ to be a Galois Connection, it must be the case that for all
\[p,q\in\N,(f(p)\leq q \iff p\leq g(q)).\]

We first show the conditional in one direction. Let $p,q\in\N$ such that $f(p)\leq q$. This gives us that $2p\leq q$ and $p\leq q/2$. Since the floor function is monotone, $\floor{p}\leq\floor{q/2}$. And since $p\in\N$, $p=\floor{p}$, so $p\leq\floor{q/2}$, which is $p\leq g(p)$, so the conditional to the right is true.

We now show the conditional in the other direction. Let $p,q\in\N$ such that $p\leq g(q)$. So $p\leq\floor{q/2}\leq q/2$, and $2p\leq q$. This gives us $f(p)\leq q$, so the conditional to the right is true.

So since $f$ and $g$ are monotone functions where for all $p,q\in\N,(f(p)\leq q \iff p\leq g(q))$, then $f$ and $g$ are the left and right adjoint, respectively, for a Galois connection.


\end{document}