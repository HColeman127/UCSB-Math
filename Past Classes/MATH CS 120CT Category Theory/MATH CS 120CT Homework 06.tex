\documentclass[12pt]{article}

% packages
\usepackage{kantlipsum}
\usepackage[margin=1in]{geometry}
\usepackage[labelfont=it]{caption}
\usepackage[table]{xcolor}
\usepackage{subcaption,framed,colortbl,multirow}
\usepackage{amsmath,amsthm,amssymb,wasysym,mathrsfs,mathtools}
\usepackage{tikz,graphicx,pgf,pgfplots}
\usetikzlibrary{arrows, angles, quotes, decorations.pathreplacing, math, patterns, calc}
\pgfplotsset{compat=1.16}


% custom commands
\newcommand{\N}{\mathbb{N}}
\newcommand{\Z}{\mathbb{Z}}
\newcommand{\I}{\mathbb{I}}
\newcommand{\R}{\mathbb{R}}
\newcommand{\Q}{\mathbb{Q}}
\newcommand{\C}{\mathsf{C}}
\newcommand{\D}{\mathsf{D}}
\newcommand{\F}{\mathbb{F}}
\newcommand{\p}{^{\prime}}
\newcommand{\powerset}{\raisebox{.15\baselineskip}{\Large\ensuremath{\wp}}}
\DeclarePairedDelimiter{\ceil}{\lceil}{\rceil}
\DeclarePairedDelimiter\floor{\lfloor}{\rfloor}
\newcommand{\id}[1]{\text{id}_{#1}}
\newcommand{\set}{\mathsf{Set}}
\newcommand{\homc}{\text{Hom}_\C}

\setlength{\fboxsep}{4pt}
\newcommand{\ex}[2]{\section*{Exercise #1}\begin{center}\framebox{\begin{minipage}{\textwidth-10pt}#2\end{minipage}}\end{center}}

\newenvironment{drawing}{\begin{center}\begin{tikzpicture}}{\end{tikzpicture}\end{center}}

 
\begin{document}
 
\title{Homework 6\\
    \large MATH CS 120CT Category Theory}
\author{Harry Coleman}
\date{March 4, 2020}
\maketitle

\ex{1.1.i}{
    Show that a morphism can have at most one inverse isomorphism.
}

Suppose we have the morphism $f: X\rightarrow Y$ with at least one inverse isomorphism. Let $g,g':Y\rightarrow X$ be inverse morphisms of $f$, so\[f \circ g = \id{Y} = f\circ g'.\]
Where $f^{-1}$ is an arbitrary inverse of $f$,
\begin{align*}
    f^{-1} \circ (f \circ g) &= f^{-1} \circ (f \circ g'), \\
    (f^{-1} \circ f) \circ g &= (f^{-1} \circ f) \circ g', \\
    \id{X} \circ g &= \id{X}\circ g', \\
    g &= g'.
\end{align*}
Since $g$ and $g'$ are equal, the inverse morphism of $f$ must be unique.


\ex{1.3.iii}{
    Find an example to show that the objects and morphisms in the image of a functor $F: \C \rightarrow \D$ do not necessarily define a subcategory of $\D$.
}

Let $\C$ be the category:
\begin{drawing}
    \draw (0,0) node{$c_1$};
    \draw (1,0) node{$c_2$};
    \draw (3,0) node{$c_3$};
    \draw (4,0) node{$c_4$};
    
    \draw[->, line width=0.5pt] (0.25,0) -- (0.75,0) node[midway,anchor=south]{$f$};
    \draw[->, line width=0.5pt] (3.25,0) -- (3.75,0)node[midway,anchor=south]{$g$};
\end{drawing}
And $F:\C\rightarrow\D$ be the functor such that $c_2$ and $c_3$ map to the same object in $D$. That is, we have the image $F(C)$:
\begin{drawing}
    \draw (0,0) node{$Fc_1$};
    \draw (3,0.25) node{$Fc_2$};
    \draw (3,-0.25) node{$Fc_3$};
    \draw (6,0) node{$Fc_4$};
    
    \draw[->, line width=0.5pt] (0.5,0) -- (2.5,0) node[midway,anchor=south]{$Ff$};
    \draw[->, line width=0.5pt] (3.5,0) -- (5.5,0)node[midway,anchor=south]{$Fg$};
\end{drawing}
However, since $f$ and $g$ are not composable in $C$, there is no composition $Fg\circ Ff$, so $F(C)$ is not a (sub)category. 

\ex{1.4.ii}{
    What is a natural transformation between a parallel pair of functions between groups, regarded as one-object categories?
}

Let $X,Y$ be groups, regarded as single-object categories, with the group homomorphisms and natural transformation:
\begin{drawing}
    \draw (0,0) node{$X$};
    \draw (2,0) node{$Y$};
    
    \draw[->, line width=0.5pt] (0.25,0.25) to [out=30,in=150] (1.75,0.25);
    \draw[->, line width=0.5pt] (0.25,-0.25) to [out=-30,in=-150] (1.75,-0.25);
    
    \draw (1,0.75) node{$F$};
    \draw (1,0) node{$\Downarrow\alpha$};
    \draw (1,-0.75) node{$H$};
\end{drawing}

Since $F$ and $H$ must satisfy the functorality axioms,
\[F(a\circ b) = F(a)\circ F(b),\]
\[H(a\circ b) = H(a)\circ H(b),\]
$FX$ and $HX$ are subgroups of $Y$. Since $\alpha$ has only one component for the one object $y$ in $Y$, then $\alpha_y\circ Fa= Ha\circ \alpha_y$ for all elements $a$ of $X$. Additionally, since $\alpha_y$ has an inverse, we have $Fa=\alpha_y^{-1}\circ Ha\circ \alpha_y$. In other words, the natural transformation $\alpha$ tells us that the subgroups $FX$ and $HX$ are conjugate to each other.

\ex{2.1.ii}{
    Prove that if $F:\C \to \set$ is representable, then $F$ preserves monomorphisms, i.e., sends every monomorphism in $\C$ to an injective function. Use the contrapositive to find a covariant set-valued functor defined on your favorite concrete category that is not representable. 
}

Let $F:\C\rightarrow\set$ be a representable functor with natural isomorphism
\[\alpha: F \Rightarrow \homc(x,-)\]
where $x$ is an object of $\C$. Let $f:a\rightarrow b$ be a monomorphism in $\C$. Since $F$ is representable, the following diagram commutes in $\set$:
\begin{drawing}
    \draw (0,2) node{$Fa$};
    \draw (4,2) node{$Fb$};
    \draw (0,0) node{$\homc(x,a)$};
    \draw (4,0) node{$\homc(x,b)$};
    
    \draw[->] (0.5,2) -- (3.5,2) node[midway, anchor=south]{$Ff$};
    \draw[->] (1.25,0) -- (2.75,0) node[midway, anchor=north]{$f\circ-$};
    
    \draw[->] (0,1.5) -- (0,0.5) node[midway,anchor=east]{$\alpha_a$};
    \draw[->] (4,1.5) -- (4,0.5) node[midway,anchor=west]{$\alpha_b$};
\end{drawing}

For $F$ to preserve monomorphisms, we must have $Ff$ be an injective function. The function $Ff$ is an injection if and only if $\alpha_b^{-1}\circ (f\circ-)\circ\alpha_a$ is an injection. Since $\alpha_a$ and $\alpha_b$ are bijections, this is equivalent to $f\circ-$ being an injection.

Let $g,h\in\homc(x,a)$, such that $f\circ g = f\circ h$. Since $f$ is a monomorphism, this implies that $g=h$. Therefore, $f\circ-$ is an injection, and so is $Ff$. Thus, $F$ preserves monomorphisms.

\newpage
Now let $\C$ be the category free category on the following graph:
\begin{drawing}
    \draw[fill=black] (0,0) circle (2pt) node[anchor=north]{$a$};
    \draw[fill=black] (2,0) circle (2pt) node[anchor=north]{$b$};
    \draw (0,0) -- (2,0) node[midway]{$>$};
    \node at (1,-0.5){$f$};
\end{drawing}
The only morphism into $a$ is $\id{a}$ and clearly,
\begin{align*}
    f\circ \id{a} &= f\circ \id{a}, \\
    \id{a} &= \id{a}.
\end{align*}
Therefore, $f$ is a monomorphism. Now let
\begin{align*}
    F:\C &\rightarrow\set \\
        a &\mapsto \{0,1\} \\
        b &\mapsto \{0\} \\
        f &\mapsto \{(0,0), (1,0)\}
\end{align*}
Evidently, $Ff$ is not an injection since $Ff(0)=Ff(1)$, but $0\ne1$. Therefore, $F$ is not representable.


\newpage
\ex{3.1.vi}{
    Prove that if
    \begin{drawing}
        \draw (0,0) node{$E$};
        \draw (2,0) node{$A$};
        \draw (4,0) node{$B$};
        
        \draw[->, line width=0.5pt] (0.25,0) -- (1.75,0) node[midway, anchor=south]{\footnotesize$h$};
        \draw[->, line width=0.5pt] (2.25,0.1) -- (3.75,0.1) node[midway, anchor=south]{\footnotesize$f$};
        \draw[->, line width=0.5pt] (2.25,-0.1) -- (3.75,-0.1) node[midway, anchor=north]{\footnotesize$g$};
    \end{drawing}
    is an equalizer diagram, then $h$ is a monomorphism.
}

Suppose it is an equalizer diagram, so
\[f\circ h = g\circ h.\]
Consider the object $X$ with morphisms $k$ and $l$
\begin{drawing}
    \draw (-2,0) node{$X$};
    \draw (0,0) node{$E$};
    \draw (2,0) node{$A$};
    \draw (4,0) node{$B$};
    
    \draw[->, line width=0.5pt] (0.25,0) -- (1.75,0) node[midway, anchor=south]{\footnotesize$h$};
    \draw[->, line width=0.5pt] (2.25,0.1) -- (3.75,0.1) node[midway, anchor=south]{\footnotesize$f$};
    \draw[->, line width=0.5pt] (2.25,-0.1) -- (3.75,-0.1) node[midway, anchor=north]{\footnotesize$g$};
    
    \draw[->, line width=0.5pt] (-1.75,0.1) -- (-0.25,0.1) node[midway, anchor=south]{\footnotesize$k$};
    \draw[->, line width=0.5pt] (-1.75,-0.1) -- (-0.25,-0.1) node[midway, anchor=north]{\footnotesize$l$};
\end{drawing}
such that $h\circ k = h\circ l$. To show $h$ is a monomorphism, we must show that $k=l$.

Consider the morphisms
\begin{align*}
    h\circ k&:X\rightarrow A, \\
    h\circ l&:X\rightarrow A.
\end{align*}
The first of these, $h\circ k$ (and for the same reasons $h\circ l$), equalizes $f$ and $g$ since
\begin{align*}
    f\circ(h\circ k)
        &= (f\circ h)\circ k \\
        &= (g\circ h)\circ k \\
        &= g\circ (h\circ k).
\end{align*}
Therefore, there exists a unique morphism $u:X\rightarrow E$ such that
\[h\circ u = h\circ k.\]
Since both $k$ and $l$ satisfy this, $u=k=l$. Thus, $h$ is a monomorphism.



\end{document}