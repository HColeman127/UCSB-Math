\documentclass[12pt]{article}

% Packages
\usepackage[margin=1in]{geometry}
\usepackage{parskip}
\usepackage{amsmath, amsthm, amssymb}
\usepackage{tikz, tikz-cd}
\usepackage[shortlabels]{enumitem}

\usepackage{bbm}

\usepackage{suffix}
\usetikzlibrary{decorations.pathmorphing}

% Problem Box
\setlength{\fboxsep}{4pt}
\newlength{\myparskip} 
\setlength{\myparskip}{\parskip}
\newsavebox{\savefullbox}
\newenvironment{fullbox}{\begin{lrbox}{\savefullbox}\begin{minipage}{\dimexpr\textwidth-2\fboxsep\relax}\setlength{\parskip}{\myparskip}}{\end{minipage}\end{lrbox}\framebox[\textwidth]{\usebox{\savefullbox}}}

% Environments
\setlist[enumerate]{nosep}
\newcommand{\keyword}[1]{\textbf{#1}}
\newcommand{\sepline}{\rule{\textwidth}{0.4pt}}

% Tikz Environments
\newenvironment{drawing}{\begin{center}\begin{tikzpicture}}{\end{tikzpicture}\end{center}}
% \tikzcdset{row sep/normal=0pt}
\newenvironment{cd}{\begin{center}\begin{tikzcd}}{\end{tikzcd}\end{center}}


% Document Formatting
\newtheoremstyle{mythmstyle}% name of the style to be used
{ }% measure of space to leave above the theorem. E.g.: 3pt
{ }% measure of space to leave below the theorem. E.g.: 3pt
{ }% name of font to use in the body of the theorem
{ }% measure of space to indent
{\scshape}% name of head font
{.}% punctuation between head and body
{ }% space after theorem head; " " = normal interword space
{\thmname{#1}\thmnumber{ #2}\thmnote{ (#3)}}% Manually specify head

\theoremstyle{definition}
\newtheorem{theorem}{Theorem}
\newtheorem{corollary}{Corollary}
\newtheorem{lemma}{Lemma}
\newtheorem{proposition}{Proposition}
\newtheorem{claim}{Claim}


% Math Formatting
\newcommand{\isp}[1]{\quad\text{#1}\quad}

% mathbb
\newcommand{\N}{\mathbb{N}}
\newcommand{\Z}{\mathbb{Z}}
\newcommand{\Q}{\mathbb{Q}}
\newcommand{\R}{\mathbb{R}}
\newcommand{\C}{\mathbb{C}}
\renewcommand{\k}{\mathbbm{k}}

% mathcal
\newcommand{\LL}{\mathcal{L}}
\newcommand{\FF}{\mathcal{F}}

% Symbols
\newcommand{\eps}{\varepsilon}
\renewcommand{\phi}{\varphi}
\renewcommand{\emptyset}{\varnothing}

% Delimiters
\newcommand{\<}{\left\langle}
\renewcommand{\>}{\right\rangle}

% Relations
\newcommand{\iso}{\cong}
\newcommand{\seq}{\subseteq}

\newcommand{\inc}{\hookrightarrow}
\newcommand{\To}{\longrightarrow}
\newcommand{\Mapsto}{\longmapsto}

% Math Roman
\newcommand{\dd}{\mathrm{d}}
\newcommand{\DD}{\mathrm{D}}
\newcommand{\bd}{\partial}

\DeclareMathOperator{\im}{im}
\newcommand{\GL}{\mathrm{GL}}

% Other
\newcommand{\pdv}[2]{\frac{\partial #1}{\partial #2}}
\newcommand{\mat}[1]{\begin{bmatrix}#1\end{bmatrix}}
\newcommand{\clo}{\overline}
\newcommand{\conj}{\overline}
\renewcommand{\hat}{\widehat}


\title{Algebraic Topology \\
    \large Fall 2022
}
\author{}
\date{}

\begin{document}
\maketitle

requires homotopy stuff and some group theory.
helpful to have some category theory.

\sepline

9/22/2022

Why algebraic topology?

We want to learn about spaces, but spaces are squishy and complicated so we want to reduce them down to something more rigid.

Idea: look for algebraic `invariants' (maps from spaces to algebraic things, where homeomorphic and homotopy equivalent spaces give the same thing).

Example: Euler Characteristic.
Take a surface and tile it with polygons, then
\[
    \chi = (\#\text{vertices}) - (\#\text{edges}) + (\#\text{faces}).
\]
Claim: this doesn't depend on the particular tiling, only the original surface.
To prove this, show that certain moves/operations on the tilings doesn't change $\chi$ and that the set of moves is enough to get between any two tilings.
Homology will tell us how this works out.

More generally: someone gives you a space defined implicitly, e.g., given $(X, x_0)$ nice space define $\Omega X$ to be the set of loops in $X$ based at $x_0$ with the $C^0$ topology (supremum distance between loops as functions $[0, 1] \to X$).

\sepline

What is homology?

Slogan: $H_n(X)$ ($n$th homology group of $X$) measures the ``$n$-dimensional holes'' of $X$.


What should This mean?

Example: $n$-sphere $S^n$ should have an $n$-dimensional hole.

How to formalize this?
\begin{enumerate}
    \item homotopy groups.
    \[
        \pi_n(X, x_0) = \{\text{based maps } (S^n, s_0) \to (X, x_0)\} / \text{homotopy}
    \]
    Problems: hard to compute; $\pi_2(S^2) = \Z$, $\pi_3(S^2) = \Z$, $\pi_4(S^2) = Z_2$, $\pi_5(S^2) = Z_2$, $\pi_6(S^2) = Z_{12}$, $\pi_7(S^2) = Z_2$, etc.
    \item cycles and boundaries.
    An $n$-chain is some sort of $n$-dimensional subsurface.
    Say two $n$-chains are equivalent if together they form the boundary of an $(n+1)$-chain.
    Say an $n$-dimensional hole is a $n$-cycle which is not the boundary of an $(n+1)$-chain.
    Issues when trying to make this rigorous: multiplicity, orientation, intersections, need machinery to compute.
\end{enumerate}

How is $H_1$ different from $\pi_1$?
e.g., punctured torus deformation retracts to figure-eight with $\pi_1 \iso F_2$.
But the embedded figure-eight is simply the boundary of the $2$-chain formed by the rest of the surface; it turns out $H_1$ is abelian (in fact $H_1$ is the abelianization $\pi_1/\text{commutators}$).

\sepline

9/27/22

Simplicial Homology

Idea: Impose strong restriction on what is a chain.

The \keyword{standard $n$-simplex} is constructed as
\[\textstyle
    \Delta^n = \{(t_0, \dots, t_n) \in \R^{n+1} \mid t_i \geq 0, \sum_i t_i = 1\}.
\]
Equivalently, $\Delta^n$ is the convex hull of the vertices $e_1, \dots, e_{n+1}$.

Call the tuple $(t_0, \dots, t_n)$ the \keyword{Barycentric coordinates}.

A \keyword{face} of $\Delta^n$ is a copy of $\Delta^{n-1}$ obtained by setting $t_i = 0$ (called the $i$-th face for each $i$).


\sepline

An \keyword{abstract simplicial complex} consists of
\begin{itemize}
    \item a set $V$ of vertices;
    \item a set $D$ of finite subsets of $V$, closed under $\seq$ (taking faces/subsimplices), which tells you which simplices are included.
\end{itemize}

The \keyword{geometric realization} of an abstract simplicial complex is constructed by taking simplices corresponding to the elements of $D$ and gluing faces according to the recipe.
Give this space the quotient topology.


\sepline

Examples

$S^1$ homeomorphic to
\begin{drawing}
    \fill (0, 0) circle (2pt);
    \fill (1, 0) circle (2pt);
    \fill (0, 1) circle (2pt);

    \draw (0, 0) -- (1, 0) -- (0, 1) -- (0, 0);
\end{drawing}


Torus $T^2$ homeomorphic to
\begin{drawing}
    \draw (0, 0) rectangle (3, 3);
    \draw (0, 1) -- (3, 1);
    \draw (0, 2) -- (3, 2);
    \draw (1, 0) -- (1, 3);
    \draw (2, 0) -- (2, 3);


\end{drawing} 

\begin{drawing}
    \node (A) at (0 : 2) {};

    \foreach \a in {0, 60, 120, 180, 240, 300} {
        \fill (\a : 2) circle (2pt);
    }

    \fill (0, 1) circle (2pt);
    fill 


    \draw (0 : 2) -- (60 : 2) -- (120 : 2) -- (180 : 2) -- (240 : 2) -- (300 : 2) -- (0 : 2);

    



\end{drawing}

\sepline

An \keyword{$n$-chain} in a simplicial complex $X$ is a formal linear combination of $n$-simplices. 

There is a \keyword{boundary homomorphism}
\begin{cd}
    C_n(X) \rar["\bd_n"] & C_{n-1}(X)
\end{cd}
which sends each simplex to its boundary and extends by linearity.

$\bd[v_0, v_1] = [v_1] - [v_0]$.

$\bd[v_0, v_1, v_2] = [v_1, v_2] - [v_0, v_2] + [v_0, v_1]$.

$\bd[v_0, v_1, v_2, v_3] = [v_1, v_2, v_3] - [v_0, v_2, v_3] + [v_0, v_1, v_3] - [v_0, v_1, v_2]$.

In general,
\[
    \bd[v_0, \dots, v_n] = \sum_{i} (-1)^{i} [v_0, \dots, \hat{v}_i, \dots, v_n].
\]


To ways to fully formalize this:
\begin{enumerate}
    \item Global ordering on vertices, giving fixed orientation on each simplex;
    \item include ordering on the notation, adopt convention of putting arrows on edges and writing $[v_0, v_1] = -[v_1, v_0]$.
    \item Then $[v_0, \dots, v_0] = \mathrm{sign}(\sigma)[v_{\sigma(0)}, \dots, v_{\sigma(n)}]$ for any permutation $\sigma$ (equivalently, should be sign of determinant of transformation taking one to the other).
\end{enumerate}


\sepline

Let $X$ be a simplicial complex.

Have $n$-chains $C_n(X)$.

Have $n$-cycles $Z_n(X) = \ker\bd_n \seq C_n(X)$.

Have $n$-boundaries $B_n(X) = \im\bd_{n+1} \seq Z_n(X)$.

Have \keyword{chain complex}
\begin{cd}
    \cdots \rar["\bd_{n+2}"]
    & C_{n+1}(X) \rar["\bd_{n+1}"]
    & C_n(X) \rar["\bd_n"]
    & C_{n-1}(X) \rar["\bd{n-1}"]
    & \cdots
\end{cd}

The \keyword{$n$th simplicial homology group} of $X$ is $H_n(X) = Z_n(X)/B_n(X)$.
An element of $H_n(X)$ is called a \keyword{homology class}.
Two cycles in the same class are said to be \keyword{homologous}, i.e., they differ by the boundary of something.

\sepline

Example

$S_1$
\begin{drawing}
    \fill (0, 0) circle (2pt) node[anchor=north east] {$v_0$};
    \fill (1, 0) circle (2pt) node[anchor=west] {$v_1$};
    \fill (0, 1) circle (2pt) node[anchor=south] {$v_2$};

    \draw (0, 0) -- (1, 0) -- (0, 1) -- (0, 0);
\end{drawing}

$H_0(S^1) = \<v_0, v_1, v_2\> / \<v_1 - v_0, v_2 - v_1, v_0 - v_2\> \iso \Z$ with $[v_0] = [v_1] = [v_2]$.

$H_1(S^1) = Z_1(S^1)/B_1(S^1)$.
Have $B_1(S^1) = 0$ because no $2$-simplices, and
\[
    Z_1(S^1) = \<[v_1, v_2] - [v_0, v_1] + [v_0, v_1]\> \iso \Z.
\]

\sepline

$\Delta$-Complexes




\end{document}