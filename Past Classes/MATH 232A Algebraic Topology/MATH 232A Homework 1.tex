\documentclass[12pt]{article}

% Packages
\usepackage[margin=1in]{geometry}
\usepackage{fancyhdr, parskip}
\usepackage{amsmath, amsthm, amssymb}
\usepackage{tikz, tikz-cd}

\usetikzlibrary{decorations.markings}

\tikzset{->-/.style={decoration={
    markings,
    mark=at position 0.5 with {\arrow{>}}},postaction={decorate}}}

\tikzset{->>-/.style={decoration={
    markings,
    mark=at position 0.5 with {\arrow{>>}}},postaction={decorate}}}

\tikzset{->>>-/.style={decoration={
    markings,
    mark=at position 0.5 with {\arrow{>>>}}},postaction={decorate}}}

% Page Style
\makeatletter
\fancypagestyle{title}{
    \renewcommand{\headrulewidth}{0.4pt}
    \setlength{\headheight}{15pt}
    \fancyhead[R]{\@author}
    \fancyhead[L]{\@title}
    \fancyhead[C]{\@date}
}
\makeatother
\renewcommand{\maketitle}{\thispagestyle{title}}
\fancypagestyle{plain}{
    \fancyhf{}
    \renewcommand{\headrulewidth}{0pt}
    \renewcommand{\footrulewidth}{0pt}
    \fancyfoot[R]{\thepage}
}
\pagestyle{plain}

% Problem Box
\setlength{\fboxsep}{4pt}
\newlength{\myparskip}
\setlength{\myparskip}{\parskip}
\newsavebox{\savefullbox}
\newenvironment{fullbox}{\begin{lrbox}{\savefullbox}\begin{minipage}{\dimexpr\textwidth-2\fboxsep\relax}\setlength{\parskip}{\myparskip}}{\end{minipage}\end{lrbox}\framebox[\textwidth]{\usebox{\savefullbox}}}
\newenvironment{pbox}[1][]{\begin{fullbox}\def\temp{#1}\ifx\temp\empty\else\paragraph{#1}\phantom{}\fi}{\end{fullbox}}

% Theorem Environments
\theoremstyle{definition}
\newtheorem{lemma}{Lemma}

% Tikz Environments
\newenvironment{drawing}{\begin{center}\begin{tikzpicture}}{\end{tikzpicture}\end{center}}
% \tikzcdset{row sep/normal=0pt}
\newenvironment{cd}{\begin{center}\begin{tikzcd}}{\end{tikzcd}\end{center}}

% Default Commands
\newcommand{\isp}[1]{\quad\text{#1}\quad}
\newcommand{\N}{\mathbb{N}} 
\newcommand{\Z}{\mathbb{Z}}
\newcommand{\Q}{\mathbb{Q}}
\newcommand{\R}{\mathbb{R}}
\newcommand{\C}{\mathbb{C}}
\newcommand{\A}{\mathbb{A}}
\renewcommand{\P}{\mathbb{P}}
\newcommand{\eps}{\varepsilon}
\renewcommand{\phi}{\varphi}
\renewcommand{\emptyset}{\varnothing}
\newcommand{\<}{\langle}
\renewcommand{\>}{\rangle}
\newcommand{\iso}{\cong}
\newcommand{\eqc}{\overline}
\newcommand{\clo}{\overline}
\newcommand{\seq}{\subseteq}
\newcommand{\teq}{\trianglelefteq}
\DeclareMathOperator{\id}{id}
\DeclareMathOperator{\im}{im}
\newcommand{\inc}{\hookrightarrow}
\newcommand{\dd}{\mathrm{d}}
\newcommand{\mat}[1]{\begin{bmatrix}#1\end{bmatrix}}

% Extra Commands
\newcommand{\bd}{\partial}

% Document
\begin{document}
\title{MATH 232A Homework 1}
\author{Harry Coleman}
\date{October 2, 2022}
\maketitle

I worked with Joseph Sullivan and Gahl Shemy.

\begin{pbox}[1]
    Let $A$ be a $\Delta$-complex and build a new $\Delta$-complex $X$ by adding a single new $n$-simplex $D$.
    Using simplicial homology, compute the difference between $H_*(A)$ and $H_*(X)$.
\end{pbox}

First note that $C_n(X) = C_n(A) \oplus \Z D$ and $C_i(X) = C_i(A)$ for $i \ne n$.

Because of this, the kernels and images of the boundary maps $\bd_i$ for $i \ne n$ are entirely unchanged.
Therefore, when $i > n$ or $i \leq n - 2$ we have
\[
    H_i(X) 
        = Z_i(X) / B_i(X)
        = Z_i(A) / B_i(A)
        = H_i(A).
\]

It remains to check $H_n$ and $H_{n-1}$.

$H_{n-1}(X) = Z_{n-1}(X) / B_{n-1}(X) = Z_{n-1}(A) / (B_{n-1}(A) + \Z\bd_n(D))$


$H_n(X) = Z_n(X) /B_n(X) = Z_n(X) / B_n(A)$

\newpage
\begin{pbox}[2 Hatcher 2.1.1]
    What familiar space is the quotient $\Delta$-complex of a $2$-simplex $[v_0, v_1, v_2]$ obtained by identifying the edges $[v_0, v_1]$ and $[v_1, v_2]$, preserving the ordering of vertices?
\end{pbox}

We want to form the following quotient:

\begin{drawing}
    \node (A) at (0, 1) {};
    \node (B) at (1, 0) {};
    \node (C) at (2, 1) {};

    \fill (A) circle (2pt) node[anchor=east] {$v_0$};
    \fill (B) circle (2pt) node[anchor=north] {$v_1$};
    \fill (C) circle (2pt) node[anchor=west] {$v_2$};

    \draw[->-] (A) -- (B);
    \draw[->-] (B) -- (C);
    \draw (A) -- (C);
\end{drawing}

Before identifying the edges, we make the following cut:
\begin{drawing}
    \draw[->-] (0, 2) -- (2, 0);
    \draw (0, 2) -- (2, 2);
    \draw[->>-, dashed] (2, 0) -- (2, 2);

    \draw (3, 2) -- (5, 2);
    \draw[->-] (3, 0) -- (5, 2);
    \draw[->>-, dashed] (3, 0) -- (3, 2);
\end{drawing}

We now glue the original pair of edges to obtain the following:
\begin{drawing}
    \draw (0, 2) -- (2, 2);
    \draw (0, 0) -- (2, 0);
    \draw[->>-] (0, 2) -- (0, 0);
    \draw[->>-] (2, 0) -- (2, 2);

    \draw[->-, dashed] (0, 2) -- (2, 0);
\end{drawing}

Regluing the cut we made gives us the M\"obius strip.



\newpage
\begin{pbox}[3 Hatcher 2.1.5]
    Compute the simplicial homology groups of the Klein bottle using the $\Delta$-complex structure described at the beginning of this section.
\end{pbox}

We use the following $\Delta$-complex construction:
\begin{drawing}
    \fill (0, 0) circle (2pt) node[anchor=north east] {$v$};
    \fill (2, 0) circle (2pt) node[anchor=north west] {$v$};
    \fill (0, 2) circle (2pt) node[anchor=south east] {$v$};
    \fill (2, 2) circle (2pt) node[anchor=south west] {$v$};

    \draw[->-] (0, 0) -- (0, 2) node[midway, anchor=east] {$a$};
    \draw[->-] (2, 0) -- (2, 2) node[midway, anchor=west] {$a$};

    \draw[->>-] (0, 2) -- (2, 2) node[midway, anchor=south] {$b$};
    \draw[->>-] (2, 0) -- (0, 0) node[midway, anchor=north] {$b$};

    \draw[->>>-] (0, 0) -- (2, 2) node[midway, anchor=west] {$c$};

    \node at (0.5, 1.5) {$U$};
    \node at (1.5, 0.5) {$L$};
\end{drawing}

We first compute the boundaries of the 1-simplices:
\[
    \bd_1(a) = \bd_1(b) = \bd_1(c) = v - v = 0.
\]
This allows us to compute the 0th homology group
\[
    H_0 = \<v\>/0 \iso \Z.
\]

We now compute the boundaries of the 2-simplices:
\begin{align*}
    \bd_2(U) &= a + b - c, \\
    \bd_2(L) &= -a + b + c.
\end{align*}
This gives us
\[
    H_1 = \<a, b, c\>/\<a + b - c, -a + b + c\>.
\]

Consider the map
\begin{align*}
    f : \Z^2 &\longrightarrow \Z^3 = \<a, b, c\>, \\
        e_1 &\longmapsto a + b - c, \\
        e_2 &\longmapsto -a + b + c.
\end{align*}
Then we can write $H_2 = \Z^3 / \im f$.
If $P : \Z^2 \to \Z^2$ and $Q : \Z^3 \to \Z^3$ are invertible linear maps, then we also have $H_2 = \Z^3/\im(QfP)$.
Write the linear map $f$ as the matrix
\[
    \mat{1 & -1 \\ 1 & 1 \\ -1 & 1},
\]
then $P$ and $Q$ correspond to row and column operations on this matrix.
We find the following equivalent matrix:
\[
    \mat{1 & -1 \\ 1 & 1 \\ -1 & 1}
    \xrightarrow{r_1 = r_1 + r_3}
    \mat{0 & 0 \\ 1 & 1 \\ -1 & 1}
    \xrightarrow{r_3 = r_3 + r_2}
    \mat{0 & 0 \\ 1 & 1 \\ 0 & 2}
    \xrightarrow{c_2 = c_2 + c_1}
    \mat{0 & 0 \\ 1 & 0 \\ 0 & 2}.
\]
Thus, we can now compute the 1st homology group
\[
    H_1
        = \<a, b, c\> / \<b, 2c\>
        = \<a\> \oplus \<c\>/\<2c\>
        \iso \Z \oplus \Z/2\Z.
\]

We compute the kernel of $\bd_2$:
\begin{align*}
    0
        &= \bd_2(\alpha U + \beta L) \\
        &= \alpha(a + b - c) + \beta(-a + b + c) \\
        &= (\alpha - \beta)a + (\alpha + \beta)b + (-\alpha + \beta)c.
\end{align*}
This implies $\alpha = \beta = -\beta$, so we must have $\alpha = \beta = 0$.
Hence, we compute the 2nd homology group to be
\[
    H_2 = 0.
\]


\newpage
\begin{pbox}[4 Hatcher 2.1.8]
    Construct a $3$-dimensional $\Delta$-complex $X$ from $n$ tetrahedra $T_1, \dots, T_n$ by the following two steps.
    First, arrange the tetrahedra in a cyclic pattern as in the figure, so that each $T_i$ shares a common vertical face with its two neighbors $T_{i-1}$ and $T_{i+1}$, subscripts being taken mod $n$.
    Then identify the bottom face of $T_i$ with the top face of $T_{i+1}$ for each $i$.
    Show the simplicial homology groups of $X$ in dimensions 0, 1, 2, 3 are $\Z$, $\Z/n\Z$, 0, $\Z$, respectively.
\end{pbox}

After making the prescribed identifications, we may label the simplices of $T_i$ as follows:
\begin{drawing}
    \fill (0, 0) circle (2pt) node[anchor=east] {$x$};
    \fill (2, -1) circle (2pt) node[anchor=north] {$x$};
    \fill (5, 1) circle (2pt) node[anchor=north west] {$y$};
    \fill (5, 3) circle (2pt) node[anchor=south west] {$y$};

    \draw[->-] (0, 0) -- (2, -1) node[midway, anchor=north east] {$a$};
    \draw[->-] (5, 1) -- (5, 3) node[midway, anchor=west] {$b$};


    \draw[->-] (0, 0) -- (5, 3) node[midway, above, sloped] {$c_i$};
    \draw[->-, dashed] (0, 0) -- (5, 1) node[midway, above, sloped] {$c_{i+1}$};
    \draw[->-] (2, -1) -- (5, 3) node[midway, above, sloped] {$c_{i+1}$};
    \draw[->-] (2, -1) -- (5, 1) node[midway, below, sloped] {$c_{i+2}$};
\end{drawing}
Additionally, we label the top face $U_i$, the bottom face $U_{i+1}$, the back/left face $L_i$, and the front/right face $L_{i+1}$.

We compute the boundaries of the 1-simplices:
\begin{align*}
    \bd_1(a) &= x - x = 0, \\
    \bd_1(b) &= y - y = 0, \\
    \bd_1(c_i) &= y - x.
\end{align*}
We compute the 0th homology group
\[
    H_0
        = \<x, y\> / \<y - x\>
        = \<x\>
        \iso \Z.
\]

We compute the kernel of $\bd_1$:
\[\textstyle
    0
        = \bd_1(\alpha a + \beta b + \sum_i \gamma_i c_i)
        = \sum_i \gamma_i(y - x).
\]
This implies $\sum_i \gamma_i = 0$ so $\gamma_1 = -\sum_{i=2}^{n} \gamma_i$.
So in $C_2 = \<a, b, c_1, \dots, c_n\>$, the kernel of of the boundary map consists of elements represented as vectors of the form
\begin{align*}
    \mat{\alpha \\ \beta \\ -\sum_{i=2}^{n} \gamma_i \\ \gamma_2 \\ \vdots \\ \gamma_n}
        &= \alpha\mat{1 \\ 0 \\ \vdots \\ 0}
            + \beta\mat{0 \\ 1 \\ 0 \\ \vdots \\ 0}
            + \gamma_2\mat{0 \\ 0 \\ -1 \\ 1 \\ 0 \\ \vdots \\ 0}
            + \gamma_2\mat{0 \\ 0 \\ -1 \\ 0 \\ 1 \\ 0 \\ \vdots \\ 0}
            + \gamma_2\mat{0 \\ 0 \\ -1 \\ 0 \\ \vdots \\ 0 \\ 1} \\
        &= \alpha a + \beta b + \gamma_2(c_2 - c_1) + \gamma_3(c_3 - c_1) + \cdots + \gamma_n(c_n - c_1) \\
        &= \alpha a  + \beta b + \sum_{i=2}^{n} \gamma_i(c_i - c_1).
\end{align*}
So $\ker\bd_1 = \<a, b, c_2 - c_1, c_3 - c_1, \dots, c_n - c_1\>$.

We can manipulate the generators to get $\ker\bd_1 = \<a, b, c_2 - c_1, c_3 - c_2, \dots, c_n - c_{n-1}\>$.

Define the 1-chains $d_i = c_{i + 1} - c_i$ for $i = 1, \dots, n-1$.
The $d_i$'s are linearly independent, so we can write $\ker\bd_1 = \<a, b, d_1, \dots, d_{n-1}\> \iso \Z^{n+1}$.

We compute the boundaries of the 2-simplices:
\begin{align*}
    \bd_2(U_i) &= a + c_{i+1} - c_i = a + d_i, \\
    \bd_2(L_i) &= b + c_{i+1} - c_i = b + d_i.
\end{align*}

Then the 1st homology group is
\[
    H_1 = \<a, b, d_1, \dots, d_{n-1}\> / \<a + d_1, \dots, a + d_n, b + d_1, \dots, b + d_n\>,
\]
where $d_n = -d_1 - \cdots - d_{n-1}$.
We consider the kernel as the image of a linear map $\Z^{2n} \to \Z^{n+1}$ which we represent as the matrix
\[
    \begin{array}{c}
        a \\
        b \\
        d_1 \\
        d_2 \\
        \vdots \\
        d_{n-1}
    \end{array}
    \mat{
        1 & 1 & \cdots & 1 & 1 & 0 & 0 & \cdots & 0 & 0 \\
        0 & 0 & \cdots & 0 & 0 & 1 & 1 & \cdots & 1 & 1 \\
        1 & 0 & \cdots & 0 & -1 & 1 & 0 & \cdots & 0 & -1 \\
        0 & 1 & \cdots & 0 & -1 & 0 & 1 & \cdots & 0 & -1 \\ 
        \vdots & & \ddots & & \vdots & \vdots & & \ddots & & \vdots \\
        0 & 0 & \cdots & 1 & -1 & 0 & 0 & \cdots & 1 & -1 \\ 
    }
    =
    \begin{array}{c}
        a \\
        b \\
        d_i
    \end{array}
    \mat{
        1 & 1 & 0 & 0 \\
        0 & 0 & 1 & 1 \\
        I_{n-1} & -1 & I_{n-1} & -1
    }.
\]

We now perform row and column operations on this matrix to produce a more useful presentation of the quotient.
Note that when a column is repeated, this corresponds to a repeated relation, so we can safely remove that any repeated columns.
Additionally, if a column contains all 0's except for a 1 in the $j$th row, this corresponds to the $j$th generator reducing to zero in the quotient.
In this case, we can safely remove that column and the $j$th row and discard that generator.

\begin{align*}
    \mat{
        1 & 1 & 0 & 0 \\
        0 & 0 & 1 & 1 \\
        I_{n-1} & -1 & I_{n-1} & -1
    }
        &\to
        \mat{
            1 & 1 & -1 & -1 \\
            0 & 0 & 1 & 1 \\
            I_{n-1} & -1 & 0 & 0
        } && \text{column operations} \\
        &\to
        \mat{
            1 & 1 & -1 \\
            0 & 0 & 1 \\
            I_{n-1} & -1 & 0 
        } && \text{remove repeat columns} \\
        &\to
        \mat{
            0 & n & -1 \\
            0 & 0 & 1 \\
            I_{n-1} & -1 & 0 
        } && \text{row operations} \\
        &\to
        \mat{
            n & -1 \\
            0 & 1 \\
        } && \text{remove null generators} \\
        &\to \mat{
            n & 0 \\
            0 & 1 \\
        } && \text{row operations} \\
        &\to \mat{n} && \text{remove null generators}
\end{align*}
Thus, the 1st homology group is
\[
    H_1 = \<a\> / \<na\> \iso \Z/n\Z.
\]


We compute the kernel of $\bd_2$:
\begin{align*}
    \bd_2(\textstyle\sum_{i=1}^{n} \alpha_i U_i + \sum_{i=1}^{n} \beta_i L_i)
        &= \sum_{i=1}^{n} \alpha_i(a + c_{i+1} - c_i) + \beta_i(b + c_{i+1} - c_i) \\
        &= \sum_{i=1}^{n} \alpha_i a + \sum_{i=1}^{n} \beta_i b + \sum_{i=1}^{n} (\alpha_{i-1} - \alpha_i + \beta_{i-1} - \beta_i)c_i.
\end{align*}
Elements of the kernel are of the form
\begin{align*}
    \mat{
        \sum_i \alpha_i \\
        \sum_i \beta_i \\
        \alpha_n - \alpha_1 + \beta_n - \beta_1 \\
        \alpha_1 - \alpha_2 + \beta_1 - \beta_2 \\
        \vdots \\
        \alpha_{n-1} - \alpha_n + \beta_{n-1} - \beta_n
    }
        &= \alpha_1 \mat{1 \\ 0 \\ -1 \\ 1 \\ 0 \\ \vdots \\ 0}
            % + \alpha_2 \mat{1 \\ 0 \\ 0 \\ -1 \\ 1 \\ 0 \\ \vdots \\ 0} 
            + \cdots
            + \alpha_n \mat{1 \\ 0 \\ 1 \\ 0 \\ \vdots \\ 0 \\ -1}
            + \beta_1 \mat{0 \\ 1 \\ -1 \\ 1 \\ 0 \\ \vdots \\ 0}
            % + \beta_2 \mat{0 \\ 1 \\ 0 \\ -1 \\ 1 \\ 0 \\ \vdots \\ 0} 
            + \cdots
            + \beta_n \mat{0 \\ 1 \\ 1 \\ 0 \\ \vdots \\ 0 \\ -1} \\
        &= \sum_{i=1}^{n} \alpha_i(a + c_{i+1} - c_i) +   \sum_{i=1}^{n} \beta_i(b + c_{i+1} - c_i).
\end{align*}
Then
\[
    \ker\bd_2 = \<a + c_{i+1} - c_i, b + c_{i+1} - c_i\>.
\]

We compute the image of $\bd_3$:
\begin{align*}
    \bd_3(U_i) &= a + c_{i+1} - c_i, \\
    \bd_3(L_i) &= b + c_{i+1} - c_i.
\end{align*}

Hence, the 2nd homology group is
\[
    H_2
        = \<a + c_{i+1} - c_i, b + c_{i+1} - c_i\>/\<a + c_{i+1} - c_i, b + c_{i+1} - c_i\>
        = 0.
\]

We compute the kernel of $\bd_3$:
\begin{align*}
    0 
        &= \bd_3(\textstyle\sum_i \alpha_i T_i) \\
        &= \sum_{i=1}^{n} \alpha_i(L_{i+1} - L_i + U_i - U_{i+1}) \\
        &= \sum_{i=1}^{n} (\alpha_{i-1} - \alpha_i)L_i + \sum_{i=}^{n} (\alpha_i - \alpha_{i-1}) U_i 
\end{align*}
This implies $\alpha_1 = \cdots = \alpha_n$, so $\ker\bd_3 = \<T_1 + \cdots + T_n\>$.
Hence, the 3rd homology group is
\[
    H_3 = \<T_1 + \cdots + T_n\>/0 \iso \Z.
\]


\end{document}