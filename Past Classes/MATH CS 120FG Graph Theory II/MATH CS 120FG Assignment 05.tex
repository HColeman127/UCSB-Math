\documentclass[12pt]{article}

% packages
\usepackage{kantlipsum}
\usepackage[margin=1in]{geometry}
\usepackage[labelfont=it]{caption}
\usepackage[table]{xcolor}
\usepackage{subcaption,framed,colortbl,multirow,enumitem}
\usepackage{amsmath,amsthm,amssymb,wasysym,mathrsfs,mathtools}
\usepackage{tikz,graphicx,pgf,pgfplots}
\usetikzlibrary{arrows, angles, quotes, decorations.pathreplacing, math, patterns, calc}
\pgfplotsset{compat=1.16}

% Set Names
\newcommand{\N}{\mathbb{N}}
\newcommand{\Z}{\mathbb{Z}}
\newcommand{\I}{\mathbb{I}}
\newcommand{\R}{\mathbb{R}}
\newcommand{\Q}{\mathbb{Q}}
\newcommand{\C}{\mathbb{C}}

% Misc Characters
\newcommand{\F}{\mathbb{F}}
\newcommand{\powerset}{\raisebox{.15\baselineskip}{\Large\ensuremath{\wp}}}
\newcommand{\eps}{\varepsilon}

% Paired Delimiters
\DeclarePairedDelimiter{\ceil}{\lceil}{\rceil}
\DeclarePairedDelimiter\floor{\lfloor}{\rfloor}
\DeclarePairedDelimiter{\ang}{\langle}{\rangle}

% Homework Sections
\setlength{\fboxsep}{4pt}
\newcommand{\generic}[2]{\section*{#1}\begin{center}\framebox{\begin{minipage}{\textwidth-10pt}#2\end{minipage}}\end{center}}
\newcommand{\ex}[2]{\generic{Exercise #1}{#2}}
\newcommand{\prob}[2]{\generic{Problem #1}{#2}}
\newcommand{\ques}[2]{\generic{Question #1}{#2}}

% Environments
\newenvironment{drawing}{\begin{center}\begin{tikzpicture}}{\end{tikzpicture}\end{center}}

% MATH CS 117 Intro to Real Analysis
\newcommand{\ds}{\displaystyle}
\newcommand{\seq}[1]{\left\{#1\right\}_{n=1}^\infty}
\newcommand{\isp}[1]{\quad\text{#1}\quad}
 
\begin{document}
 
\title{Assignment 5\\
    \large MATH CS 120FG Graph Theory II}
\author{Harry Coleman}
\date{May 27, 2020}
\maketitle

\ex{1}{
    Verify that if $G$ is a simple graph with incidence matrix $M$ and Laplacian $L$, then
    \[L = MM^T.\]
}

Suppose $G$ has $n$ vertices and $m$ edges, so
\[V-\{v_1,\dots,v_n\}, \quad\text{and}\quad E=\{e_1,\dots,e_m\}.\]
The Laplacian matrix $L$ is the $n\times n$ is given by
\[L=D-A.\]
The $n\times n$ matrix $D$ is the diagional matrix defined by
\[D_{ii} = \deg(v_i),\]
and the $n\times n$ matrix $A$ is defined by
\[A_{ij} = 
    \begin{cases}
        1 &\text{if } v_i\sim v_j, \\
        0 &\text{otherwise.}
    \end{cases}
\]
This means that $L$ is defined by
\[L_{ij} = 
    \begin{cases}
        \deg(v_i)   &\text{if } i=j, \\
        -1          &\text{if } v_i\sim v_j, \\
        0           &\text{otherwise.}
    \end{cases}
\]
We assign arbitrary orientations to the edges of $G$. In particular, for each edge $e_k\in E$ incident to vertices $v_i,v_j$, we define one vertex to be the source, $s(e_k)=v_i$, and the other to be the target, $t(e_k)=v_j$. We now consider the $n\times m$ incidence matrix $M$ defined by
\[M_{ik} = 
    \begin{cases}
        -1  &\text{if } s(e_k)=v_i, \\
        1   &\text{if } t(e_k)=v_i, \\
        0   &\text{otherwise.}
    \end{cases}
\]

We now consider the product $MM^T$:
\[(MM^T){ij} = \sum_{k=1}^mM_{ik}M^T_{kj} = \sum_{k=1}^mM_{ik}M_{jk}.\]
If $i=j$, then $M_{ik}M_{jk}$ will be 1 if $e_k$ is incident to $v_i$ (source or target) and 0 otherwise. So taking the sum over all edges, we get the number of edges incident to $v_i$, which is $\deg(v_i)$. If $i\ne j$, then $M_{ik}M_{jk}$ will be zero for all $k$ except when $e_k$ is an edge between $v_i$ and $v_j$, in which case $M_{ik}M_{jk}=-1$. Since $G$ is simple, there can be at most one edge between two vertices, so summing over all edges gives $-1$ if $i\sim j$ and 0 otherwise. In conclusion,
\[(MM^T)_{ij} = 
    \begin{cases}
        \deg(v_i)   &\text{if } i=j, \\
        -1          &\text{if } v_i\sim v_j, \\
        0           &\text{otherwise,}
    \end{cases}
\]
which is precisely $L_{ij}$. So $L=MM^T$.


\newpage
\ex{2}{
    Calculate $\kappa(Q_n)$ for the hypercube $Q_n$ using the eigenvalues of the adjacency matrix of $Q_n$.
}

If we let $L$ be the Laplacian of $Q_n$ and $A$ be the adjacency matrix of $Q_n$, then $L = nI - A$. We want to calculate $\kappa(Q_n)$, which is given by
\[\kappa(Q_n) = \frac1{2^n}\prod_{v\in\Z_2^n\setminus\{w\}}\mu_v,\]
where $\mu_v$ are the eigenvalues of $L$ indexed by the vertices of $Q_n$, and $\mu_w=0$. In particular, $\mu_v\in\sigma(L)$ if and only if $\det(\mu_vI-L)=0$. And since
\[\det(\mu_vI-L) = \det(\mu_vI - (nI-A)) = \det(A-(\mu_v+n)I).\]
Now let $\lambda_v$ denote the eigenvalues of $A$ indexed by the vertices of $Q_n$. Then $\det(A-(\mu_v+n)I)=0$ if and only if $\mu+n\in\sigma(A)$, that is, $\mu_v+n=\lambda_v$ for some $\lambda_v$. Now since $\lambda_v=n-2\omega(v)$, where $\omega(v)$ is the number of 1's in $v$,
\begin{align*}
    \kappa(Q_n)
        &= \frac1{2^n}\prod_{v\in\Z_2^n\setminus\{w\}}(\lambda_v-n) \\
        &= \frac1{2^n}\prod_{v\in\Z_2^n\setminus\{w\}}(n-2\omega(v)-n) \\
        &= \frac1{2^n}\prod_{v\in\Z_2^n\setminus\{w\}}2\omega(v).
\end{align*}
Notice that if $\mu_w=0$, then $2\omega(v)=0$, so $w$ is all zeros. For each $k=1,\dots,n$, there are ${n \choose k}$ vertices $v\in\Z_2^n$ such that $\omega(v)=k$. So we can rewrite the above sum more explicitly as
\begin{align*}
    \kappa(Q_n)
        &= \frac1{2^n}\prod_{k=1}^n(2k)^{n \choose k}.
\end{align*}


\newpage
\ex{3}{
    Take $K_n$ and delete one edge to form $G$. Use the Matrix-Tree Theorem to calculate $\kappa(G)$.
}

Suppose from $G$ we delete the edge connecting vertices $v_i$ and $v_j$. Let $L$ be the Laplacian matrix of $K_n$ and $L'$ be the Laplacian matrix of $G$. Then
\[L'_{k\ell} =
    \begin{cases}
        L_{k\ell}-1 &\text{if } k\ell=ii \text{ or } k\ell=jj, \\
        L_{k\ell}+1 &\text{if } k\ell=ij \text{ or } k\ell=ji, \\
        L_{k\ell} &\text{otherwise.}
    \end{cases}
\]
This characterization of $L'$ comes from the fact that the vertices $v_i,v_j$ have one less degree in $G$ than in $K_n$, and are not adjacent to each other in $G$. We denote by $L_0$ and $L'_0$, respectively, the minor matrices of $L$ and $L'$ obtained by removing the $j$th row and column. Note that $L_0$ and $L'_0$ only differ in the $ii$th element, with $(L'_0)_{ii} = (L_0)_{ii}-1 = n-2$. Then the matrix-tree theorem gives us that,
\[\kappa(G) = \det(L'_0).\]
Applying cofactor expansion along the $i$th row, we find
\begin{align*}
    \kappa(G)
        &= \sum_{m=1}^{n-1}(L'_0)_{im}(-1)^{i+m}\det(L'_0(i|m)) \\
        &= \sum_{m=1,m\ne i}^{n-1}(L'_0)_{im}(-1)^{i+m}\det(L'_0(i|m)) + (L'_0)_{ii}(-1)^{i+i}\det(L'_0(i|i))\\
        &= \sum_{m=1,m\ne i}^{n-1}(L'_0)_{im}(-1)^{i+m}\det(L'_0(i|m)) + (n-2)\det(L'_0(i|i)).
\end{align*}
Notice now that since $L_0$ and $L'_0$ differ only at the $ii$th element, $L'_0(i|m)=L_0(i|m)$ for all $m$, and $(L'_0)_{im}=(L_0)_{im}$ for all $m\ne i$. So
\begin{align*}
    \kappa(G)
        &= \sum_{m=1,m\ne i}^{n-1}(L_0)_{im}(-1)^{i+m}\det(L_0(i|m)) + (n-2)\det(L_0(i|i))\\
        &= \det(L_0)-(L_0)_{ii}(-1)^{i+i}\det(L_0(i|i)) + (n-2)\det(L_0(i|i))\\
        &= n^{n-2}-(n-1)\det(L_0(i|i)) + (n-2)\det(L_0(i|i)).
\end{align*}
We now notice that $L_0(i|i)=nI_{n-2}-J_{n-2}$, which has similar eigenvalues to $J_{n-2}$, but shifted by $n$. So
\begin{align*}
    \kappa(G)
        &= n^{n-2}-(n-1)2n^{n-3} + (n-2)2n^{n-3}\\
        &= n\cdot n^{n-3}-2n^{n-3} \\
        &= (n-2)n^{n-3}.
\end{align*}


\newpage
\ex{4}{
    Let $p \geq 5$ and let $G_p$ be the graph with vertex set $\Z_p$ and edges $\{i,i+1\}$ and $\{i,i + 2\}$ for $i \in\Z_p$. Show that $\kappa(G) = pF_p^2$ where $F_p$ is the $p$th Fibonacci number (with $F_1 = 1$ and $F_2 = 1$).
}

The minor of the laplactian of $G_p$ is given by the $(p-1)\times(p-1)$ matrix
\[L_0 =
    \begin{bmatrix}
        4 & -1 & -1 & 0 & \cdots & 0 & -1 \\
        -1 & \ddots & \ddots & \ddots & \ddots & & 0 \\
        -1 & \ddots & \ddots & \ddots & \ddots & \ddots & \vdots \\
        0 & \ddots & \ddots & \ddots & \ddots & \ddots & 0 \\
        \vdots & \ddots & \ddots & \ddots & \ddots & \ddots & -1 \\
        0 & & \ddots & \ddots & \ddots & \ddots & -1 \\
        -1 & 0 & \cdots & 0 & -1 & -1 & 4 \\
    \end{bmatrix}
\]
I think the way to go about this would be to do cofactor expansion a few times until matrix structures start to repeat, at which point you get a recurrence relation that would solve to be $pF_p^2$ or something equivalent. I was unable to do this.




\ex{5}{
    Take $K_n$ and for $1 \leq m \leq n$, form $K_n - K_m$ by deleting a copy of the edges of $K_m$ from $K_n$. Calculate $\kappa(K_n - K_m)$.
}

The Laplacian of $K_n-K_m$ is given by
\[L=
    \begin{bmatrix}
        a      & -1     & \cdots & -1     & -1     & \cdots & \cdots & -1     \\
        -1     & \ddots & \ddots & \vdots & \vdots & \ddots &        & \vdots \\
        \vdots & \ddots & \ddots & -1     & \vdots &        & \ddots & \vdots \\
        -1     & \cdots & -1     & a      & -1     & \cdots & \cdots & -1     \\
    
        -1     & \cdots & \cdots & -1     & b      & 0      & \cdots & 0      \\
        \vdots & \ddots &        & \vdots & 0      & \ddots & \ddots & \vdots \\
        \vdots &        & \ddots & \vdots & \vdots & \ddots & \ddots & 0      \\
        -1     & \cdots & \cdots & -1     & 0      & \cdots & 0      & b      \\
    \end{bmatrix}
    =
    \begin{bmatrix}
        aI_{n-m}-J_{n-m}   & -J_{n-m,m} \\
        -J_{m,n-m} & bI_{m}
    \end{bmatrix}
\]
where $a=n-1$ and $b=n-m-1$. We obtain the minor matrix
\[L_0 =
    \begin{bmatrix}
        aI_{n-m}-J_{n-m}   & -J_{n-m,m-1} \\
        -J_{m-1,n-m} & bI_{m-1}
    \end{bmatrix}
\]
by removing the last row and column from $L$. The block $B_0$ is the same form as $B$, but with size $(m-1)\times(m-1)$. To make calculating the determinant of this matrix easier, we perform row operations which do no change the value of the determinant to simplify $L_0$. In particular we add $1/b$ times each of the lower rows to each of the upper rows to eliminate the upper right block. this gives us the matrix
\[L_0' =
    \begin{bmatrix}
        aI_{n-m}-(\frac{m}{b}+1)J_{n-m}   & 0 \\
        -J_{m-1,n-m} & bI_{m-1}
    \end{bmatrix}.
\]
So then
\begin{align*}
    \kappa (K_n-K_m)
        &= \det(L_0')\\
        &= \det\left(aI_{n-m}-\left(\frac{m}{b}+1\right)J_{n-m}\right)\det(bI_{m-1}) \\
        &= \det\left(\left(\frac{m}{b}+1\right)\left(\frac{ab}{m+b}I_{n-m}-J_{n-m}\right)\right)b^{m-1} \\
        &= b^{m-1}\left(\frac{m}{b}+1\right)^{n-m}\det\left(\frac{ab}{m+b}I_{n-m}-J_{n-m}\right) \\
\end{align*}
Since $J_{n-m}$ has eigenvalues 0 with multiplicity $n-m-1$ and $n-m$ with multiplicity $1$, then the matrix inside the determinant has eigenvalues $\frac{ab}{m+b}$ with multiplicity $n-m-1$ and $\frac{ab}{m+b}-n+m$ with multiplicity 1. So now
\begin{align*}
    \kappa (K_n-K_m)
        &= b^{m-1}\left(\frac{m}{b}+1\right)^{n-m}\left(\frac{ab}{m+b}\right)^{n-m-1}\left(\frac{ab}{m+b}-n+m\right) \\
        &= b^{m-2}a^{n-m-1}[ab-n(m+b)+m(m+b)] \\
        &= (n-m-1)^{m-2}(n-1)^{n-m-1}[(n-1)(n-m-1) \\
            &\quad-n(m+n-m-1)+m(m+n-m-1)] \\
        &= (n-m-1)^{m-2}(n-1)^{n-m-1}[(n-1)(n-m-1)-n(n-1)+m(n-1)] \\
        &= (n-m-1)^{m-2}(n-1)^{n-m-1}(n-1)[n-m-1-n+m] \\
        &= (n-m-1)^{m-2}(n-1)^{n-m} \\
\end{align*}


\ex{6}{
    Let $G$ be a finite graph on $n$ vertices with Laplacian matrix $L(G)$. Let $G'$ be obtained from $G$ by adding a new vertex and joining it to all vertices of $G$. Express $\kappa(G')$ in terms of the eigenvalues of $L(G)$.
}

By the definition of $G'$, we have
\[L(G') = 
    \begin{bmatrix}
        L(G)+I & -1 \\
        -1 & n
    \end{bmatrix}.
\]
Then
\begin{align*}
    \kappa(G')
        &= \det(L_0(G')) \\
        &= \det(L(G) + I) \\
        &= \lambda_1\cdots\lambda_{n-1},
\end{align*}
where $\lambda_1,\dots,\lambda_{n-1}$ are the nonzero eigenvalues of $L(G)+I$. To find them, we let $\mu_1,\dots,\mu_{n-1}$ be the nonzero eigenvalues of $L(G)$. Then
\[\det(xI-(L(G)+I)) = \det((x-1)I-L(G))\]
equals zero if and only if $x-1$ is an eigenvalue of $L(G)$. So in fact $\mu_1+1,\dots,\mu_{n-1+1}$ are the eigenvalues of $L(G)+I$. So
\[\kappa(G') = (\mu_1+1)\cdots(\mu_{n-1}+1).\]


\end{document}