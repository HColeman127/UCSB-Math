\documentclass[12pt]{article}

% Packages
\usepackage[margin=1in]{geometry} % proper margins
\usepackage{enumitem} % custom numbering for lists
\usepackage{amsmath} % align, cases, eqref, matrices, dots, roots, delimiters, math mode functions, mod, arrows
\usepackage{amsthm} % theorems, proofs
\usepackage{amssymb} % fancy letters, niche relations/negations
\usepackage{mathrsfs} % much more loopy calligraphy font

% Theorems
\newtheorem{theorem}{Theorem}
\newtheorem{lemma}{Lemma}
\newtheorem{proposition}{Proposition}

% Problem Box
\setlength{\fboxsep}{4pt}
\newsavebox{\mybox}
\newenvironment{problem}
    {\begin{lrbox}{\mybox}\begin{minipage}{0.98\textwidth}}
    {\end{minipage}\end{lrbox}\framebox[\textwidth]{\usebox{\mybox}}}

% Formatting
\newcommand{\ds}{\displaystyle}
\newcommand{\isp}[1]{\quad\text{#1}\quad}

% Alternate Characters
\let\eps\varepsilon % double curved, rather than single curve with midline
\let\phi\varphi % single stroke loop, rather than vertical line through circle
\let\emptyset\varnothing % circular, rather than tall

% Named Sets
\newcommand{\N}{\mathbb{N}} % natural numbers
\newcommand{\Z}{\mathbb{Z}} % integers 
\newcommand{\Q}{\mathbb{Q}} % rational numbers
\newcommand{\R}{\mathbb{R}} % real numbers
\newcommand{\C}{\mathbb{C}} % complex numbers

% Fancy Characters
\newcommand{\F}{\mathbb{F}} % arbitrary field
\renewcommand{\P}{\mathbb{P}} % probability measure (apparently, sometimes prime numbers)
\newcommand{\FF}{\mathcal{F}} % sigma algebra
\newcommand{\BB}{\mathcal{B}} % Borel sigma algebra

% Paired Delimiters
\newcommand{\ceil}[1]{\left\lceil #1 \right\rceil} % ceiling
\newcommand{\floor}[1]{\left\lfloor #1 \right\rfloor} % floor
\newcommand{\<}{\left\langle} % left angle bracket
\renewcommand{\>}{\right\rangle} % right angle bracket

% Functions
\renewcommand{\Im}{\operatorname{Im}} % imaginary part of a complex number
\renewcommand{\Re}{\operatorname{Re}} % real part of a complex number
\newcommand{\Arg}{\operatorname{Arg}} % principal argument (angle) of complex number

% Simplified Notation
\newcommand{\id}[1]{\operatorname{id_{\mathnormal{#1}}}} % identity operator
\newcommand{\seq}[2][n]{\left\{#2\right\}_{#1\in\N}} % sequence
\newcommand{\pdv}[3][1]{\ifnum#1=1{\frac{\partial #2}{\partial#3}}\else{\frac{\partial^{#1}#2}{\partial#3^{#1}}}\fi} % partial derivative
\newcommand{\intr}[1]{\accentset{\circ}{#1}} % interior of a set

% Renaming
\let\sm\setminus % set minus (difference)
\let\clo\overline % closure of a set
\let\conj\overline % conjugate of an object
\let\eqc\overline % equivalence class of an object
\let\teq\trianglelefteq % normal subgroup
\let\iso\cong % isomorphic (groups)

% Notes
% medskip for header-less paragraph
% intertext{} for short text inside big display structure
% dots is dynamic based on surroundings
% dfrac and tfrac to force large or small fractions
% operatorname for new operators instead of text
% consider using nath; it seems to break most formatting and is not compatible with many packages

\begin{document}
 
\title{Homework 7\\
    %\large MATH CS 121 Intro to Probability
    %\large MATH 111A Intro to Abstract Algebra
    %\large MATH CS 122A Complex Analysis I
    \large MATH 118A Intro to Real Analysis
    %\large MATH 104A Intro to Numerical Analysis
}
\author{Harry Coleman}
\date{November 19, 2020}
\maketitle

\section*{Exercise 1}
\begin{problem}
    Let $\{a_n\}_{n\in\N}$ and $\{b_n\}_{n\in\N}$ be sequences of nonnegative real numbers. Assume that $a_n \le b_n$ for all $n\in\N$. Prove the following statements:
\end{problem}

\subsection*{Exercise 1(a)}
\begin{problem}
    If $\sum_n b_n$ converges, then $\sum_n a_n$ converges.
\end{problem}

\begin{proof}
    Suppose $\sum_n b_n$ converges and let $B = \ds\sum_{n=1}^\infty b_n$. Let $A_n$ and $B_n$ denote the partial sums of $\sum_n a_n$ and $\sum_n b_n$, respectively. First, note that for any $n\in\N$,
    \[A_n = \sum_{j=1}^n a_j \leq \sum_{j=1}^n a_j + a_{n+1} = \sum_{j=1}^{n+1}a_j = A_{n+1},\]
    so $\{A_n\}$ is a non-decreasing sequence. By the same argument, $\{B_n\}$ is non-decreasing, which implies $B_n\leq B$ for all $n\in\N$. Now, for any $n\in\N$,
    \[A_n = \sum_{j=1}^n a_j \leq \sum_{j=1}^n b_j = B_n \leq B.\]
    Thus, $\{A_n\}$ is a non-decreasing sequence which is bounded above, so it is convergent, implying that $\sum_n a_n$ converges.
    
\end{proof}

\newpage
\subsection*{Exercise 1(b)}
\begin{problem}
    If $\sum_n a_n$ diverges, then $\sum_n b_n$ diverges.
\end{problem}
\medskip

Note: this is precisely the contrapositive of exercise 1(a), so they are equivalent.

\begin{proof}
    Suppose $\sum_n a_n$ diverges. Suppose, for contradiction, that $\sum_n b_n$ converges. By exercise 1(a), this implies that $\sum_n a_n$ converges, which is a contradiction. Therefore, $\sum_n b_n$ diverges.
    
\end{proof}

\subsection*{Exercise 1(c)}
\begin{problem}
    Can we remove the nonnegativity condition? Explain.
\end{problem}
\medskip

No. Suppose $a_n=-1$ for all $n\in\N$ and $\seq{b_n}$ were any nonnegative sequence such that $\sum_b b_n$ converges. Then $a_n \leq b_n$ for all $n\in\N$, however,
\[A_n = \sum_{j=1}^n-1 = -n\]
diverges to $-\infty$ as $n\to\infty$.

\newpage
\section*{Exercise 2}
\begin{problem}
    Suppose $\{a_n\}$ and $\{b_n\}$ are sequences of nonnegative real numbers such that 
    \begin{equation}
        \lim_{n\to\infty} \frac{a_n}{b_n} = L > 0.
    \end{equation}
    Prove that either $\sum_n a_n$ or $\sum_n b_n$ both converge or both diverge.
\end{problem}

\begin{proof}
    Let $A_n$ and $B_n$ denote the partial sums of $\sum_n a_n$ and $\sum_n b_n$, respectively. Suppose $\sum_n a_n$ converges; then, $\seq{A_n}\subseteq\R$ is convergent and, therefore, Cauchy. Let $\eps>0$ be given and let $c\in\R$ such that $0<c<L$. Let $n_1\in\N$ such that
    \[n\geq n_1 \implies \left|\frac{a_n}{b_n} - L\right| < c,\]
    and let $n_2\in\N$ such that
    \[n,m\geq n_2 \implies |A_m - A_n| < \eps.\]
    First note that $n\geq n_1$ gives us
    \begin{align*}
        -c &< \frac{a_n}{b_n} - L < c, \\[1em]
        L - c &< \frac{a_n}{b_n}, \\[1em]
        b_n(L-c) &< a_n, \\[1em]
        b_n &< \frac{a_n}{L-c}.
    \end{align*}
    Now define $n_0=\max\{n_1,n_2\}$. Let $n,m\geq n_0$ and, without loss of generality, assume $n\leq m$. Then
    \allowdisplaybreaks
    \begin{align*}
        |B_m - B_n|
            &= \left|\sum_{j=1}^m b_j - \sum_{j=1}^n b_j\right| \\
            &= \left|\sum_{j=n+1}^m b_j\right| \\
            &= \sum_{j=n+1}^m b_j \\
            &\leq \sum_{j=n+1}^m \frac{a_j}{L-c} \\
            &= \frac1{L-c} \sum_{j=n+1}^m a_j \\
            &= \frac1{L-c} \left|\sum_{j=n+1}^m a_j\right| \\
            &= \frac1{L-c}\left|\sum_{j=1}^m a_j - \sum_{j=1}^n a_j\right| \\
            &= \frac1{L-c}|A_m - A_n| \\
            &< \frac1{L-c}\eps.
    \end{align*}
    Since $c<L$, then $\frac1{L-c}$ is a positive constant with respect to $n,m$, and $\eps$. So the above implies that $\seq{B_n}\subseteq\R$ is Cauchy. Therefore $\seq{B_n}$ is convergent and so is $\sum_n b_n$.
    
    Suppose $\sum_n b_n$ converges; then, $\seq{A_n}\subseteq\R$ is convergent and, therefore, Cauchy. Let $\eps>0$ be given and let $c>0$ be arbitrary. Let $n_1\in\N$ such that
    \[n\geq n_1 \implies \left|\frac{a_n}{b_n} - L\right| < c,\]
    and let $n_2\in\N$ such that
    \[n,m\geq n_2 \implies |B_m - B_n| < \eps.\]
    First note that $n\geq n_1$ gives us
    \begin{align*}
        -c < \frac{a_n}{b_n} - L &< c, \\[1em]
        \frac{a_n}{b_n} &< L+c, \\[1em]
        a_n &< b_n(L+c).
    \end{align*}
    Now define $n_0=\max\{n_1,n_2\}$. Let $n,m\geq n_0$ and, without loss of generality, assume $n\leq m$. Then
    \begin{align*}
        |A_m - A_n|
            &= \left|\sum_{j=1}^m a_j - \sum_{j=1}^n a_j\right| \\
            &= \left|\sum_{j=n+1}^m a_j\right| \\
            &= \sum_{j=n+1}^m a_j \\
            &\leq \sum_{j=n+1}^m b_j(L+c) \\
            &= (L+c) \sum_{j=n+1}^m b_j \\
            &= (L+c)\left|\sum_{j=n+1}^m b_j\right| \\
            &= (L+c)\left|\sum_{j=1}^m b_j - \sum_{j=1}^n b_j\right| \\
            &= (L+c)|B_m - B_n| \\
            &< (L+c)\eps.
    \end{align*}
    This implies $\seq{A_n}\subseteq\R$ is Cauchy. Therefore, $\seq{A_n}$ converges and so does $\sum_n a_n$.
    
    In conclusion, $\sum_n a_n$ converges if and only if $\sum_n b_n$ converges. In other words, either $\sum_n a_n$ or $\sum_n b_n$ both converge or both diverge.
    
\end{proof}

\newpage
\section*{Exercise 3}
\begin{problem}
    Study the convergence of the following series:
\end{problem}

\subsection*{Exercise 3(a)}
\begin{problem}
    \begin{equation}\label{118A-H07-3a}
        \sum_{n=1}^\infty \frac{3n^2-n+1}{4n^3-n}.
    \end{equation}
\end{problem}

\begin{proposition}
    The series \eqref{118A-H07-3a} diverges.
\end{proposition}

\begin{proof}
    For any $n\in\N$,
    \begin{align*}
        \frac{3n^2-n+1}{4n^3-n}
            &\geq \frac{3n^2-n+1}{4n^3} \\[1em]
            &\geq \frac{3n^2-n}{4n^3} \\[1em]
            &\geq \frac{3n^2-n^2}{4n^3} \\[1em]
            &= \frac{4n}{4n^2} \\[1em]
            &= \frac{1}{n}.
    \end{align*}
    Thus, \eqref{118A-H07-3a} is term-wise larger than the harmonic series. Since the harmonic series diverges, so does \eqref{118A-H07-3a}.
    
\end{proof}

\newpage
\subsection*{Exercise 3(b)}
\begin{problem}
    \begin{equation}\label{118A-H07-3b}
        \sum_{n=1}^\infty \frac{1}{\sqrt{n^3+n+5}}.
    \end{equation}
\end{problem}

\begin{proposition}
    The series \eqref{118A-H07-3b} converges.
\end{proposition}

\begin{proof}
    For any $n\in\N$,
    \[0\leq n^3 + n + 5 \implies 0\leq \frac{1}{\sqrt{n^3+n+5}},\]
    so \eqref{118A-H07-3b} has nonnegative terms.
    \begin{align*}
        \frac{1}{\sqrt{n^3+n+5}}
            &\leq \frac{1}{\sqrt{n^3}} \\
            &= \frac1{n^{3/2}}.
    \end{align*}
    Now since $\frac32 > 1$, then the series
    \[\sum_{n=1}^\infty \frac1{n^{3/2}}\]
    converges to some $L\in\R$. Then the partial sums of \eqref{118A-H07-3b} are bounded above by $L$, since they are term-wise bounded by the partial sums of the series converging to $L$. Thus, \eqref{118A-H07-3b} is a series of nonnegative terms whose partial sums are bounded above; therefore, it converges.
    
\end{proof}

\newpage
\subsection*{Exercise 3(c)}
\begin{problem}
    \begin{equation}\label{118A-H07-3c}
        \sum_{n=1}^\infty \frac{(-1)^{n-1} 10^n}{n!}.
    \end{equation}
\end{problem}

\begin{proposition}
    The series \eqref{118A-H07-3c} converges.
\end{proposition}

\begin{proof}
    We will apply the ratio test. For any $n\in\N$,
    \begin{align*}
        \left|\frac{\dfrac{(-1)^{n+1-1} 10^{n+1}}{(n+1)!}}{\dfrac{(-1)^{n-1} 10^n}{n!}}\right|
            &= \left|\frac{(-1)10}{n+1}\right| \\
            &= \frac{10}{n+1}. 
    \end{align*}
    Then
    \[\limsup_{n\to\infty} \frac{10}{n+1} = \lim_{n\to\infty} \frac{10}{n+1} = 0 < 1,\]
    so \eqref{118A-H07-3c} converges.
    
\end{proof}

\subsection*{Exercise 3(d)}
\begin{problem}
    \begin{equation}\label{118A-H07-3d}
        \sum_{n=1}^\infty \frac{n!}{p^n},
    \end{equation}
    for any real, positive $p$.
\end{problem}

\begin{proposition}
    The series \eqref{118A-H07-3d} diverges.
\end{proposition}

\begin{proof}
    We will apply the ratio test. For any $n\in\N$,
    \begin{align*}
        \left|\frac{\dfrac{(n+1)!}{p^{n+1}}}{\dfrac{n!}{p^n}}\right|
            &= \left|\frac{n+1}{p}\right| \\
            &= \frac{n+1}{p}. 
    \end{align*}
    Let $n_0\in\N$ such that $n_0\geq p - 1$. Then for $n\geq n_0$,
    \[\frac{n+1}{p} \geq \frac{p-1+1}{p} = \frac{p}{p} = 1,\]
    so \eqref{118A-H07-3d} diverges.
    
\end{proof}

\newpage
\section*{Exercise 4}
\begin{problem}
    Determine the radius of convergence of 
\end{problem}

\subsection*{Exercise 4(a)}
\begin{problem}
    \begin{equation}\label{118A-H07-4a}
        \sum_{n=1}^\infty (-1)^n z^n.
    \end{equation}
    Find the value of the series (for each $z$ within the radius of convergence).
\end{problem}
\medskip

Since this is a power series with $a_n = (-1)^n$ for all $n\in\N$ and $z_0 = 0$, then we have the radius of convergence
\begin{align*}
    \frac1{\ds\limsup_{n\to\infty}\sqrt[n]{|(-1)^n|}}
        &= \frac1{\ds\limsup_{n\to\infty}\sqrt[n]{|-1|^n}} \\
        &= \frac1{\ds\limsup_{n\to\infty}\sqrt[n]{1^n}} \\
        &= \frac1{\ds\limsup_{n\to\infty}1} \\
        &= \frac11 \\
        &= 1.
\end{align*}

Let $z\in\C$ such that $|z|<1$, then \eqref{118A-H07-4a} is a geometric series with $r=-z$, so
\[\sum_{n=1}^\infty (-1)^n z^n. = \sum_{n=1}^\infty (-z)^n = \frac1{1-(-z)} = \frac1{1+z}.\]

\newpage
\subsection*{Exercise 4(b)}
\begin{problem}
    \begin{equation}\label{118A-H07-4b}
        \sum_{n=1}^\infty  \frac{z^n}{2^n}.
    \end{equation}
    Find the value of the series (for each $z$ within the radius of convergence).
\end{problem}
\medskip

Since this is a power series with $a_n = \frac1{2^n}$ for all $n\in\N$ and $z_0 = 0$, then we have the radius of convergence
\begin{align*}
    \frac1{\ds\limsup_{n\to\infty}\sqrt[n]{\left|\dfrac1{2^n}\right|}}
        &= \frac1{\ds\limsup_{n\to\infty}\sqrt[n]{\dfrac1{2^n}}} \\
        &= \frac1{\ds\limsup_{n\to\infty}\dfrac12} \\
        &= \frac1{\dfrac12}\\
        &= 2.
\end{align*}
Let $z\in\C$ such that $|z|<2$, then \eqref{118A-H07-4b} is a geometric series with $r=\frac{z}{2}$, so
\[\sum_{n=1}^\infty  \frac{z^n}{2^n} = \sum_{n=1}^\infty \left(\frac{z}{2}\right)^n = \frac{\frac{z}{2}}{1-\frac{z}{2}} = \frac{z}{2-z}.\]

\newpage
\section*{Exercise 5}
\begin{problem}
    For any sequence of real numbers $\{a_n\}_{n\in\N}$, show that 
    \begin{equation}
        \limsup_n (-a_n) = - \liminf a_n.
    \end{equation}
\end{problem}

\begin{proof}
    Recall that for any set $A$, we have
    \[\inf(-A) = -\sup A,\]
    where $-A = \{-a : a\in A\}$. Then
    \begin{align*}
        \limsup_n (-a_n)
            &= \inf_n\sup_{k\geq n} (-a_k) \\[1em]
            %&= \inf_n\sup \{-a_k : k\geq n\} \\[1em]
            %&= \inf_n(-\inf(- \{-a_k : k\geq n\})) \\[1em]
            %&= \inf_n(-\inf\{a_k : k\geq n\}) \\[1em]
            &= \inf_n(-\inf_{k\geq n} a_k) \\[1em]
            %&= \inf\{-\inf_{k\geq n} a_k : n\in\N\} \\[1em]
            %&= -\sup(-\{-\inf_{k\geq n} a_k : n\in\N\}) \\[1em]
            %&= -\sup\{\inf_{k\geq n} a_k : n\in\N\} \\[1em]
            &= -\sup_n\inf_{k\geq n} a_k \\[1em]
            &= - \liminf_n a_n.
    \end{align*}
    
\end{proof}





\end{document}