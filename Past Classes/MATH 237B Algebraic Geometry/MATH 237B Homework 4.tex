\documentclass[12pt]{article}

% Packages
\usepackage[margin=1in]{geometry}
\usepackage{fancyhdr, parskip}
\usepackage{amsmath, amsthm, amssymb}
\usepackage{tikz, tikz-cd}

% Page Style
\makeatletter
\fancypagestyle{title}{
    \renewcommand{\headrulewidth}{0.4pt}
    \setlength{\headheight}{15pt}
    \fancyhead[R]{\@author}
    \fancyhead[L]{\@title}
    \fancyhead[C]{\@date}
}
\makeatother
\renewcommand{\maketitle}{\thispagestyle{title}}
\fancypagestyle{plain}{
    \fancyhf{}
    \renewcommand{\headrulewidth}{0pt}
    \renewcommand{\footrulewidth}{0pt}
    \fancyfoot[R]{\thepage}
}
\pagestyle{plain}

% Problem Box
\setlength{\fboxsep}{4pt}
\newlength{\myparskip}
\setlength{\myparskip}{\parskip}
\newsavebox{\savefullbox}
\newenvironment{fullbox}{\begin{lrbox}{\savefullbox}\begin{minipage}{\dimexpr\textwidth-2\fboxsep\relax}\setlength{\parskip}{\myparskip}}{\end{minipage}\end{lrbox}\framebox[\textwidth]{\usebox{\savefullbox}}}
\newenvironment{pbox}[1][]{\begin{fullbox}\ifx#1\empty\else\paragraph{#1}\phantom{}\fi}{\end{fullbox}}

% Theorem Environments
\theoremstyle{definition}
\newtheorem{lemma}{Lemma}
\newtheorem{proposition}{Proposition}

% Tikz Environments
\newenvironment{drawing}{\begin{center}\begin{tikzpicture}}{\end{tikzpicture}\end{center}}
% \tikzcdset{row sep/normal=0pt}
\newenvironment{cd}{\begin{center}\begin{tikzcd}}{\end{tikzcd}\end{center}}

% Default Commands
\newcommand{\isp}[1]{\quad\text{#1}\quad}
\newcommand{\N}{\mathbb{N}} 
\newcommand{\Z}{\mathbb{Z}}
\newcommand{\Q}{\mathbb{Q}}
\newcommand{\R}{\mathbb{R}}
\newcommand{\C}{\mathbb{C}}
\newcommand{\A}{\mathbb{A}}
\renewcommand{\P}{\mathbb{P}}
\newcommand{\eps}{\varepsilon}
\renewcommand{\phi}{\varphi}
\renewcommand{\emptyset}{\varnothing}
\newcommand{\<}{\langle}
\renewcommand{\>}{\rangle}
\newcommand{\isom}{\cong}
\newcommand{\eqc}{\overline}
\newcommand{\clo}{\overline}
\newcommand{\seq}{\subseteq}
\newcommand{\teq}{\trianglelefteq}
\DeclareMathOperator{\id}{id}
\DeclareMathOperator{\im}{im}

% Extra Commands
\newcommand{\OO}{\mathcal{O}}
\newcommand{\FF}{\mathcal{F}}
\newcommand{\GG}{\mathcal{G}}

\DeclareMathOperator{\Spec}{Spec}
\DeclareMathOperator{\Proj}{Proj}

\newcommand{\inc}{\hookrightarrow}

% Document
\begin{document}
\title{MATH 237B Homework 4}
\author{Harry Coleman}
\date{February 1, 2022}
\maketitle


\begin{pbox}[1 Exercise 6.7.8]
    Let $C_1 = \{f_1 = 0\}$ and $C_2 = \{f_2 = 0\}$ be affine curves in $\A_k^2$, and let $P \in C_1 \cap C_2$ be a point.
    Show that the intersection multiplicity of $C_1$ and $C_2$ at $P$ (i.e. the length of the component at $P$ of the intersection scheme $C_1 \cap C_2$) is equal to the dimension of the vector space $\OO_{\A^2, P}/\<f_1, f_2\>$ over $k$.
\end{pbox}

\begin{proof}
    We have a $k[x, y]$-module isomorphism
    \[
        A(C_1 \cap C_2)_P
            = (k[x, y]/\<f_1, f_2\>)_P
            \isom k[x, y]_P / \<f_1, f_2\>_P
            = \OO_{\A^2, P}/\<f_1, f_2\>.
    \]
    In particular, this is a $k$-linear isomorphism, so
    \[
        \mu_P(C_1, C_2)
            = \dim_k A(C_1 \cap C_2)_P
            = \dim_k \OO_{\A^2, P}/\<f_1, f_2\>.
    \]
\end{proof}

\newpage
\begin{pbox}[2 Exercise 6.7.9]
    
\end{pbox}

\begin{lemma}
    Any five distinct points in $\P^2$ (with no four collinear) determine a unique quadric curve passing through them.
\end{lemma}

\begin{proof}
    The quadric curves in $\P^2 = \Proj k[x, y, z]$ correspond to the homogeneous quadric polynomials
    \[
        A_0x^2 + A_1xy + A_2y^2 + A_3xz + A_4yz + A_5z^2,
    \]
    up to nonzero scaling.
    That is, we can parameterize the quadric curves in $\P^2$ by the projective space $\P^5 = \Proj k[A_0, \dots, A_5]$.
    For a fixed point $[x : y : z] \in \P^2$, the above polynomial is a homogeneous linear polynomial in the variables $A_0, \dots, A_5$.
    That is, each point of $\P^2$ determines a hyperplane in $\P^5$, by requiring that the above polynomial is zero at that point.
    One can check that five points in $\P^2$ determine independent hyperplanes in $\P^5$ if and only if no three of the points are collinear.

    On one hand, if no three points are collinear, the intersection of the five hyperplanes is a unique point of $\P^5$.
    This point of $\P^5$, in turn, determines an irreducible quadric curve in $\P^2$.

    On the other hand, if three of the points are collinear, then the remaining two determine a second line.
    The union of these two lines is a quadric in $\P^2$.
\end{proof}

\begin{pbox}
    Let $C_1, C_2 \seq \P^2$ be distinct smooth cubic curves, and assume that $C_1$ and $C_2$ intersect in $9$ (distinct) points $P_1, \dots, P_9$.
    Prove that every cubic curve passing through $P_1, \dots, P_8$ also has to pass through $P_9$.
\end{pbox}

\begin{proof}
    Write $C_i = Z(f_i) \seq \P^2$, with $f_i$ an irreducible smooth homogeneous cubic polynomial.
    Note that B\'ezout's theorem implies no four of the given points $P_1, \dots, P_4$ are collinear, since the intersection of a line and $C_i$ is at most three distinct points.
    So Lemma 1 implies that any five of the given points uniquely determine a quadric curve.

    Suppose, for contradiction, that $C = Z(f) \seq \P^2$ is a cubic curve passing through $P_1, \dots, P_8$, but not $P_9$.
    In particular, note that the polynomials $f, f_1, f_2$ are $k$-linearly independent; otherwise, we could write $f = af_1 + bf_2$, for some $a, b \in k$, implying $f(P_9) = 0$.

    We will prove the following two facts: no three of the given points are collinear and no six of the given points lie on the same quadric curve.

    Assume, for contradiction, that three of the given points are collinear---say $P_1, P_2, P_3$ lie on the line $L \seq \P^2$.
    Then the remaining points $P_4, \dots, P_8$ do not lie on $L$ and uniquely determine a quadric curve $D \seq \P^2$.
    Choose points $Q \in L \setminus \{P_1, P_2, P_3\}$ and $Q' \in \P^2 \setminus (L \cup D)$.
    Choose values $a, b, c \in k$ which give a nontrivial solution to the following system of linear equations:
    \[
        \begin{bmatrix}
            f(Q) & f_1(Q) & f_2(Q) \\
            f(Q') & f_1(Q') & f_2(Q')
        \end{bmatrix}
        \begin{bmatrix}
            a \\ b \\ c
        \end{bmatrix}
        = 0.
    \]
    Then, since $f, f_1, f_2$ are $k$-linearly independent, we obtain a nonzero homogeneous cubic polynomial
    \[
        g = af + bf_1 + cf_2,
    \]
    which defines a cubic curve $Z(g) \seq \P^2$ passing through the points $P_1, \dots, P_8$, $Q$, and $Q'$.
    In particular, $Z(g)$ passes through the collinear points $P_1, P_2, P_3$, and $Q$.
    Applying B\'ezout's theorem, we deduce that $Z(g)$ must contain the line $L$ as a component.
    Therefore, if $L = Z(\ell)$ for a homogeneous linear polynomial $\ell$, then we can write $g = \ell \cdot h$ for a homogeneous quadric polynomial $h$.
    It follows that $Z(h)$ is a quadric curve passing through the points $P_4, \dots, P_8$, so in fact $Z(h) = D$ by uniqueness.
    We conclude that $Z(g) = L \cup D$, but the choice of $Q' \notin L \cup D$ contradicts the fact that $Q' \in Z(g)$.
    Hence, no three of the points $P_1, \dots, P_8$ are collinear.

    Assume, for contradiction, that six of the given points lie on a quadric---say $P_1, \dots P_6$ lie on the quadric curve $D \seq \P^2$.
    Let $L \seq \P^2$ be the line through $P_7$ and $P_8$.
    Similar to the previous step, we choose points $Q \in D \setminus \{P_1, P_6\}$ and $Q' \in \P^2 \setminus (L \cup D)$, then construct a nonzero cubic polynomial
    \[
        g = af + bf_1 + cf_2,
    \]
    with zeros at $P_1, \dots, P_8$, $Q$, and $Q'$.
    In particular, $Z(g)$ passes through the points $P_1, \dots, P_6, Q$ which lie on $D$.
    Applying B\'ezout's theorem, we deduce that $Z(g)$ must contain the quadric curve $D$ as a component.
    (By the previous result, $D$ cannot be the union of two lines, and is therefore irreducible.)
    Then the other component of $Z(g)$ is simply the line $L$, passing through $P_7$ and $P_8$.
    We conclude that $Z(g) = L \cup D$, but the choice of $Q' \notin L \cup D$ contradicts the fact that $Q' \in Z(g)$.
    Hence, no six of the points $P_1, \dots, P_8$ lie on a quadric.

    With these two facts, we now let $L \seq \P^2$ be the line through $P_1$ and $P_2$, and let $D \seq \P^2$ be the quadric curve through $P_3, \dots, P_7$; it follows that $P_8 \notin L \cup D$.
    For a third time, we choose distinct points $Q, Q' \in L \setminus \{P_1, P_2\}$ and construct a nonzero cubic polynomial
    \[
        g = af + bf_1 + cf_2,
    \]
    with zeros at $P_1, \dots, P_8$, $Q$, and $Q'$.
    Since the cubic curve $Z(g)$ passes through the collinear points $P_1, P_2, Q, Q'$, B\'ezout's theorem implies that $Z(g)$ must contain the line $L$ as a component.
    Then the other component of $Z(g)$ is a quadric curve passing through $P_3, \dots, P_7$, which is uniquely $D$.
    We conclude that $Z(g) = L \cup D$, but the fact that $P_8 \notin L \cup D$ contradicts the fact that $P_8 \in Z(g)$.
    Hence, no such cubic curve $C$ exists.
\end{proof}


\begin{pbox}
    Can you find a stronger version of this statement that applies in the case that the intersection multiplicities in $C_1 \cap C_2$ are not all equal to $1$?
\end{pbox}

Some parts of the proof do not require that the points be distinct, as we rely more often on B\'ezout's theorem to count the length of $0$-dimensional intersections, which accounts for higher multiplicities.
However, to use entirely the same sort of proof, we would need to be able to construct certain types of curves (e.g., lines and quadrics) given a set of points on another curve, with desired multiplicities.
I am not sure how to do this nor under what conditions it is possible.



\newpage
\begin{pbox}[3 Exercise 6.7.10]
    Let $C$ be a smooth cubic curve of the form
    \[
        C = \{[x : y : z] \mid y^2z = x^3 + axz^2 + bz^3\} \seq \P_k^2
    \]
    for some given $a, b \in k$.
    (It can be shown that every cubic can be brought into this form by a change of coordinates.)
    Pick the point $P_0 = [0 : 1 : 0]$ as the zero element for the group structure on $C$.
    For given points $P_1 = [x_1 : y_1 : 1]$ and $P_2 = [x_2 : y_2 : 1]$ compute explicitly the coordinates of the inverse $\ominus P_1$ and of the sum $P_1 \oplus P_2$.
    Conclude that the group structure on $C$ is well-defined even if $k$ is not necessarily algebraically closed.
\end{pbox}

Denote $f = x^3 - y^2z + axz^2 + bz^3 \in k[x, y, z]^{(3)}$, so that $C = Z(f) \seq \P^2$.

For any $P, Q \in C$, let $\phi(P, Q)$ be the third point on the intersection of $C$ and $\overline{PQ}$ (or the tangent line if $P = Q$).
Then we have
\[
    \ominus P = \phi(\phi(P_0, P_0), P) \isp{and} P \oplus Q = \phi(\phi(P, Q), P_0).
\]

Consider the affine open neighborhood $\{y = 1\} \isom \A^2$ of $P_0$.
Here, $P_0$ is $(0, 0) \in \A^2$ and $f$ can be written as 
\[
    F = x^3 -z + axz^2 + bz^3 \in k[x, z].
\]
Then the equation of the tangent line at $(0, 0)$ is given by
\[\textstyle
    0
        = \frac{\partial F}{\partial x}\big|_{(0, 0)} \cdot x + \frac{\partial F}{\partial z}\big|_{(0, 0)} \cdot z
        = -z
\]
Taking the projective closure, we find that the line tangent to $C$ at $P_0$ is given by
\[
    L = \{[x : y : 0] \mid [x : y] \in \P^1\} \seq \P^2.
\]
Since $f|_L = x^3$, we have $C \cdot L = 3P_0$.
That is, $\phi(P_0, P_0) = P_0$, so
\[
    \ominus P = \phi(\phi(P_0, P_0), P) = \phi(P_0, P).
\]
The line through $P_0 = [0 : 1 : 0]$ and $P_1 = [x_1 : y_1 : 1]$ is given by
\[
    \overline{P_0P_1} = \{[x_1s : y_1s + t : s] \mid [s : t] \in \P^1\} \seq \P^2.
\]
At $s = 0$, we get $P_0$, so consider the points on this line with $s = 1$.
Evaluating $f$ at such a point gives us
\[
    x_1^3 + ax_1 - (y_1 + t)^2 + b.
\]
Then the possible values of $t$ which give us a point in $C \cap \overline{P_0P_1}$ are $-y_1 \pm \sqrt{x_1^3 + ax_1 + b}$.
Note that since $f([x_1 : y_1 : 1]) = 0$, then in fact $y_1^2 = x_1^3 + ax_1 + b$, so we have $t = -y_1 \pm y_1$.
Since $t = 0$ gives $P_1$, then the third point on the intersection of $C$ and $\overline{P_0P_1}$ occurs at $t = -2y_1$, which gives
\[
    \ominus P_1 = \phi(P_0, P_1) = [x_1 : -y_1 : 1].
\]

We parameterize the line through $P_1$ and $P_2$ as follows:
\[
    \overline{P_1P_2} = \{[x_1s + x_2t : y_1s + y_2t : s + t] \mid [s : t] \in \P^1\} \seq \P^2.
\]
Note that $s = 0$ gives us $P_2$, so we consider the points on this line with $s = 1$.
Evaluating $f$ at such a point gives us
\[
    (x_1 + x_2t)^3 - (y_1 + y_2t)^2(1 + t) + a(x_1 + x_2t)(1 + t)^2 + b(1 + t)^3.
\]
(There might be a better overall approach, but I could not find a straightforward way to compute the other value of $t$.)
If we can solve this for another value of $t$, it gives us the third point on the intersection of $C$ and $\overline{P_1P_2}$:
\[
    \phi(P_1, P_2) = ???.
\]
Then if it happens that this point is of the form $[x : y : 1]$, we can use the previous result to finish computing
\[
    P_1 \oplus P_2
        = \phi(\phi(P_1, P_2), P_0)
        = \ominus \phi(P_1, P_2).
\]




\newpage
\begin{pbox}[4 Exercise 7.8.1]
    Let $\FF'$ be a presheaf on a topological space $X$, and let $\FF$ be its sheafification.
    Show that
\end{pbox}

\begin{pbox}[(i)]
    There is a natural morphism $\theta: \FF' \to \FF$.    
\end{pbox}

For each open set $U \seq X$, we define a map
\begin{align*}
    \theta_U : \FF'(U) &\longrightarrow \FF(U) \\
        \phi &\longmapsto (\phi_P)_{P \in U}.
\end{align*}
This is a homomorphism since localization is a homomorphism, i.e., $(\phi + \psi)_P = \phi_P + \psi_P$ (and $(\phi\psi)_P = \phi_P \psi_P$ if a presheaf of rings).
Then $\theta$ defines a morphism of presheaves as it commutes with restriction, i.e., for all $\phi \in \FF'(U)$ we have
\[
    \theta_V(\phi|_V)
        = ((\phi|_V)_P)_{P \in V}
        = (\phi_P)_{P \in V}
        = \theta_U(\phi)|_V.
\]

\begin{pbox}[(ii)]
    Any morphism from $\FF'$ to a sheaf factors uniquely through $\theta$.
\end{pbox}

\begin{proof}
    Suppose $f : \FF' \to \GG$ is a morphism of presheaves, with $\GG$ a sheaf.
    Let $U \seq X$ be an open set and $\phi \in \FF(U)$; we want to construct a natural section $f^\#_U(\phi) \in \GG(U)$.
    For each point $P \in U$, let $V_P \seq U$ be an open neighborhood of $P$ such that for some $\psi^P \in \FF'(V_P)$ we have $(\psi^P)_Q = \phi_Q \in \FF'_Q$ for all $Q \in V_P$.
    Then $\{V_P\}_{P \in U}$ is an open cover of $U$ with local sections $f_{V_P}(\psi^P) \in \GG(V_P)$, such that
    \[
        f_{V_P}(\psi^P)_Q
            = f_Q((\psi^P)_Q)
            = f_Q(\phi_Q)
    \]
    for all $Q \in V_P$.
    In particular, since the stalks of all local sections agree, we deduce that
    \[
        f_{V_P}(\psi^P)|_{V_P \cap V_{P'}}
            = (f_Q(\phi_Q))_{Q \in V_P \cap V_{P'}}
            = f_{V_{P'}}(\psi^{P'})|_{V_P \cap V_{P'}}.
    \]
    Hence, by the gluing property of the sheaf $\GG$, there is a unique section $f^\#_U(\phi) \in \GG(U)$ such that $f^\#_U(\phi)|_{V_P} = f_{V_P}(\psi^P)$ for all $P \in U$.

    The fact that $f^\#_U$ is homomorphism follows from the fact that each $f_{V_P}$ is a homomorphism.
    Moreover, commuting with restrictions is inherent to its construction, so we obtain a morphism $f^\# : \FF \to \GG$.
    Additionally, $f = f^\# \circ \theta$, since for all $\phi \in \FF'(U)$ we have $\theta(\phi) = (\phi_P)_{P \in U} \in \FF(U)$, which gives
    \[
        f^\#_U(\theta(\phi))
            = (f_P(\phi_P))_{P \in U}
            = f_U(\phi)
            \in \GG(U).
    \]
\end{proof}






\newpage
\begin{pbox}[5 Exercise 7.8.2]
    Let $f : \FF \to \GG$ be a morphism of sheaves of abelian groups on a topological space $X$.
    Show that $f$ is injective/surjective/an isomorphism if and only if all induces maps $f_P : \FF_P \to \GG_P$ on the stalks are injective/surjective/an isomorphism.
\end{pbox}

\begin{proposition}
    $f$ is injective if and only if each $f_P$ is injective.
\end{proposition}

\begin{proof}
    Suppose $f$ is injective, i.e., each $f_U : \FF(U) \to \GG(U)$ is injective.
    Let $\phi \in \FF(U)$ such that $\phi_P \in \ker f_P$.
    Note that by definition,
    \[
        0 = f_P(\phi_P) = f_U(\phi)_P \in \GG_P.
    \]
    This means there is some open neighborhood $V$ of $P$ such that $f_U(\phi)|_V = 0 \in \GG(V)$.
    Then
    \[
        f_V(\phi|_V) = f_U(\phi)|_V = 0 \in \GG(V),
    \]
    which means $\phi|_V \in \ker f_V$.
    Since $f$ is injective, this mean $\phi|_V = 0 \in \FF(V)$, so in fact
    \[
        \phi_P = (\phi|_V)_P = 0_P = 0 \in \FF_P.
    \]
    Hence, $f_P$ is injective.

    Suppose each $f_P$ is injective and let $\phi \in \ker f_U$, i.e., $f_U(\phi) = 0 \in \GG(U)$.
    Then
    \[
        f_P(\phi_P) = f_U(\phi)_P = 0_P = 0 \in \GG_P.
    \]
    This implies $\phi_P = 0$ for all $P \in U$.
    This means there is an open neighborhood $V_P$ of $P$ such that $\phi|_{V_P} = 0 \in \FF(V_P)$.
    Then $\{V_P\}_{P \in U}$ is an open cover of $U$ with local sections $\phi|_{V_P}$.
    By the gluing property of the sheaf $\FF$, there is a unique section in $\FF(U)$ which restricts to these local sections on each $V_P$.
    However, both $\phi$ and $0$ are sections in $\FF(U)$ satisfying this property, so in fact $\phi = 0$.
    Hence, $f_U$ is injective, which means $f$ is injective.
\end{proof}

\begin{proposition}
    $f$ is surjective if and only if each $f_P$ is surjective.
\end{proposition}

\begin{proof}
    (Note that there is a natural inclusion $\im' f \inc \GG$ induced by the construction of the image presheaf $(\im' f)(U) = \im f_U \seq \GG(U)$.
    Since $\GG$ is a sheaf, Problem 4 implies that this inclusion uniquely factors through the inclusion into the sheafification, giving us a natural inclusions $\im' f \inc \im f \inc \GG$.
    Under these morphisms, we will consider both the image presheaf and image sheaf to be sort of ``subsheaves'' of $\GG$.
    Then $f$ is considered to be surjective if $\im f = \GG$, under these identifications.)

    Suppose $f$ is surjective.
    Let $\phi \in \GG(U) = (\im f)(U)$, which means there is an open cover $\{V_i\}$ of $U$ and local sections $\psi_i \in (\im' f)(V_i) = \im f_{V_i}$ that glue together to form $\phi$ (where $\im' f$ is the image presheaf, whose sheafification is the image sheaf $\im f$).
    For a given $P \in U$, suppose $V_i$ is a neighborhood of $P$, then
    \[
        (\psi_i)_P = (\phi|_{V_i})_P = \phi_P \in \GG_P.
    \]
    Additionally, since $\psi_i \in \im f_{V_i}$, then in fact
    \[
        \phi_P = (\psi_i)_P \in \im f_P.
    \]
    Since we can choose $P$ and $\phi \in \GG(U)$ for a neighborhood $U$ of $P$ arbitrarily, this proves that $\im f_P = \GG_P$, i.e., $f_P$ is surjective.

    Suppose each $f_P$ is surjective.
    Let $\phi \in \GG(U)$, then for all $P \in U$ we have
    \[
        \phi_P \in \GG_P = \im f_P = (\im' f)_P.
    \]
    Therefore, there is an open neighborhood $V_P$ of $P$ and a section $\psi^P \in (\im' f)(V_P)$ such that
    $(\psi^P)_P = \phi_P$.
    In particular, we can choose $V_P$ small enough so that $\psi^P = \phi|_{V_P}$.
    Note
    \[
        (\im' f)(V_P) = \im f_{V_P} \seq \GG(V_P).
    \]
    Then $\{V_P\}_{P \in U}$ is an open cover of $U$ with local sections $\psi^P \in (\im' f)(V_P)$, which agree on overlaps.
    By the definition of sheafification, this means there is a unique section in $(\im f)(U) \seq \GG(U)$ which restricts to $\psi^P = \phi|_{V_P}$ on each $V_P$.
    However, since such a section must also be unique in $\GG(U)$ (by the gluing property of the sheaf $\GG$), then $\phi$ must precisely be this section.
    In particular, $\phi \in (\im f)(U)$, and we conclude that $\im f = \GG$, hence $f$ is surjective.
\end{proof}

\begin{proposition}
    $f$ is an isomorphism if and only if each $f_P$ is an isomorphism.
\end{proposition}

\begin{proof}
    Suppose $f$ is an isomorphism.
    (I am a little unsure of what definition of isomorphism we are using here, but I will assume this means each $f_U$ is an isomorphism, which is equivalent to $f$ being both injective and surjective.)
    Then by Propositions 1 and 2, we know that each $f_P$ is both injective and surjective, which suffices to conclude $f_P$ is an isomorphism.

    Suppose each $f_P$ is an isomorphism.
    By Propositions 1 and 2, we know that $f$ is therefore both injective and surjective.
    Equivalently, each $f_U$ is an isomorphism, we we conclude that $f$ is an isomorphism.
\end{proof}


\end{document}