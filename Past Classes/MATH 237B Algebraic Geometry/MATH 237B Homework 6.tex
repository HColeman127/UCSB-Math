\documentclass[12pt]{article}

% Packages
\usepackage[margin=1in]{geometry}
\usepackage{fancyhdr, parskip}
\usepackage{amsmath, amsthm, amssymb}
\usepackage{tikz, tikz-cd}

% Page Style
\makeatletter
\fancypagestyle{title}{
    \renewcommand{\headrulewidth}{0.4pt}
    \setlength{\headheight}{15pt}
    \fancyhead[R]{\@author}
    \fancyhead[L]{\@title}
    \fancyhead[C]{\@date}
}
\makeatother
\renewcommand{\maketitle}{\thispagestyle{title}}
\fancypagestyle{plain}{
    \fancyhf{}
    \renewcommand{\headrulewidth}{0pt}
    \renewcommand{\footrulewidth}{0pt}
    \fancyfoot[R]{\thepage}
}
\pagestyle{plain}

% Problem Box
\setlength{\fboxsep}{4pt}
\newlength{\myparskip}
\setlength{\myparskip}{\parskip}
\newsavebox{\savefullbox}
\newenvironment{fullbox}{\begin{lrbox}{\savefullbox}\begin{minipage}{\dimexpr\textwidth-2\fboxsep\relax}\setlength{\parskip}{\myparskip}}{\end{minipage}\end{lrbox}\framebox[\textwidth]{\usebox{\savefullbox}}}
\newenvironment{pbox}[1][]{\begin{fullbox}\ifx#1\empty\else\paragraph{#1}\phantom{}\fi}{\end{fullbox}}

% Theorem Environments
\theoremstyle{definition}
\newtheorem{lemma}{Lemma}

% Tikz Environments
\newenvironment{drawing}{\begin{center}\begin{tikzpicture}}{\end{tikzpicture}\end{center}}
% \tikzcdset{row sep/normal=0pt}
\newenvironment{cd}{\begin{center}\begin{tikzcd}}{\end{tikzcd}\end{center}}

% Default Commands
\newcommand{\isp}[1]{\quad\text{#1}\quad}
\newcommand{\N}{\mathbb{N}} 
\newcommand{\Z}{\mathbb{Z}}
\newcommand{\Q}{\mathbb{Q}}
\newcommand{\R}{\mathbb{R}}
\newcommand{\C}{\mathbb{C}}
\newcommand{\A}{\mathbb{A}}
\renewcommand{\P}{\mathbb{P}}
\newcommand{\eps}{\varepsilon}
\renewcommand{\phi}{\varphi}
\renewcommand{\emptyset}{\varnothing}
\newcommand{\<}{\langle}
\renewcommand{\>}{\rangle}
\newcommand{\isom}{\cong}
\newcommand{\eqc}{\overline}
\newcommand{\clo}{\overline}
\newcommand{\seq}{\subseteq}
\newcommand{\teq}{\trianglelefteq}
\DeclareMathOperator{\id}{id}
\DeclareMathOperator{\im}{im}

% Extra Commands
\newcommand{\OO}{\mathcal{O}}
\newcommand{\FF}{\mathcal{F}}
\newcommand{\GG}{\mathcal{G}}
\newcommand{\LL}{\mathcal{L}}
\newcommand{\mm}{\mathfrak{m}}

\DeclareMathOperator{\Spec}{Spec}
\DeclareMathOperator{\Proj}{Proj}

\newcommand{\inc}{\hookrightarrow}
\newcommand{\tensor}{\otimes}

\newcommand{\dd}{\mathrm{d}}

% Document
\begin{document}
\title{MATH 237B Homework 6}
\author{Harry Coleman}
\date{February 14, 2022}
\maketitle

\begin{pbox}[1 Vakil Exercise 21.4.A]
    What is the degree of the invertible sheaf $\Omega_{C/k}$?
\end{pbox}

\begin{pbox}[2 Vakil Exercise 21.4.B]
    Show that $h^0(C, \Omega_{C/k}) = g$ as follows.
\end{pbox}

\begin{pbox}[(a)]
    Show that $\frac{\dd x}{y}$ is a (regular) differential on $\Spec k[x, y]/\<y^2 - f(x)\>$ (i.e., and element of $\Omega_{(k[x, y]/\<y^2 - f(x)\>)/k}$).
\end{pbox}

\begin{pbox}[(b)]
    Show that for $0 \leq i < g$, $x^i(\dd x)/y$ extends to a global differential $\omega_i$ on $C$ (i.e., with no poles).
\end{pbox}

\begin{pbox}[(c)]
    Show that the $\omega_i$ ($0 \leq i < g$) are all linearly independent differentials.
\end{pbox}

\begin{pbox}[(d)]
    Show that the $\omega_i$ form a basis for the differentials.
\end{pbox}

\begin{pbox}[3 Gathmann Exercise 7.8.7]
    
\end{pbox}

\begin{pbox}[4 Gathmann Exercise 7.8.8]
    
\end{pbox}

\end{document}