\documentclass[12pt]{article}

% Packages
\usepackage[margin=1in]{geometry}
\usepackage{amsmath, amsthm, amssymb, physics}

% Problem Box
\newsavebox{\mybox}
\newenvironment{pbox}{\begin{lrbox}{\mybox}\begin{minipage}{0.98\textwidth}}{\end{minipage}\end{lrbox}\begin{center}\framebox[\textwidth]{\usebox{\mybox}}\end{center}}

% Options
%\renewcommand{\thesubsection}{\thesection(\alph{subsection})}
\allowdisplaybreaks
\addtolength{\jot}{1em}
\theoremstyle{definition}

% Default Commands
\newtheorem{proposition}{Proposition}
\newtheorem{lemma}{Lemma}
\newcommand{\ds}{\displaystyle}
\newcommand{\isp}[1]{\quad\text{#1}\quad}
\newcommand{\N}{\mathbb{N}}
\newcommand{\Z}{\mathbb{Z}}
\newcommand{\Q}{\mathbb{Q}}
\newcommand{\R}{\mathbb{R}}
\newcommand{\C}{\mathbb{C}}
\newcommand{\eps}{\varepsilon}
\renewcommand{\phi}{\varphi}
\renewcommand{\emptyset}{\varnothing}

% Extra Commands
\renewcommand{\mod}{\bmod}
\newcommand{\isom}{\cong}
\newcommand{\eqc}{\overline}

% Document Info
\title{Homework 4\\
    \large MATH 111B
}
\author{Harry Coleman}
\date{February 5, 2021}

% Begin Document
\begin{document}
\maketitle

\section{Problem 7.6.5}
\begin{pbox}
    Let $n_1, n_2, \dots, n_k$ be integers which are relatively prime in pairs: $(n_i, n_j) = 1$ for all $i \ne j$.
\end{pbox}


\subsection{Problem 7.6.5(a)}
\begin{pbox}
    Show that the Chinese Remainder Theorem implies that for any $a_1, \dots, a_k \in \Z$ there is a solution $x \in \Z$ to the simultaneous congruences
    \[
        x \equiv a_1 \mod n_1, \quad x \equiv a_2 \mod n_2, \quad \dots, \quad x \equiv a_k \mod n_k
    \]
    and that the solution $x$ is unique mod $n = n_1 n_2 \cdots n_k$.
\end{pbox}

\begin{proof}
    For all $i \ne j$, $(n_i, n_j) = 1$ implies that $n_i\Z$ and $n_j\Z$ are comaximal ideals in $\Z$. So the Chinese Remainder Theorem tells us that the ring homomorphism
    \begin{align*}
        \Z &\to \Z/n_1\Z \times \cdots \times \Z/n_k\Z \\
        x &\mapsto (x \mod n_1, \dots, x \mod n_k)
    \end{align*}
    is surjective. So there exists $x \in \Z$ such that
    \[
        x \mapsto (a_1 \mod n_1, \dots, a_k \mod n_k).
    \]
    In other words, $x \equiv a_i \mod n_i$ for $i = 1, \dots, k$. Moreover, the kernel of this map is
    \[
        n_1\Z \cap \cdots \cap n_k\Z = n_1\Z \cdots n_k\Z = n\Z,
    \]
    which implies
    \[
        \Z/n\Z \isom \Z/n_1\Z \times \cdots \times \Z/n_k\Z.
    \]
    Then if $y \in \Z$ also satisfies the simultaneous congruences, $x \equiv y \mod n$.
    
\end{proof}

\subsection{Problem 7.6.5(b)}
\begin{pbox}
    Let $n_i' = m/n_i$ be the quotient of $n$ by $n_i$, which is relatively prime to $n_i$ by assumption, Let $t_i$ be the inverse of $n_i' \mod n_i$. Prove that the solution $x$ in (a)  is given by
    \[
        x = a_1t_1n_1' + a_2t_2n_2' + \cdots + a_kt_kn_k' \mod n.
    \]
\end{pbox}

\begin{proof}
    Let $x = \sum_{i=1}^k a_it_in_i'$. Then for a fixed $j$ we find
    \[
        x \mod n_j
            = \left(\sum_{i=1}^k a_it_in_i'\right) \mod n_j 
            = \sum_{i=1}^k \left(a_it_in_i' \mod n_j\right).
    \]
    If $i \ne j$, then $n_j \mid n_i'$, which implies $n_i' \equiv 0 \mod n_j$, so
    \[
        x \mod n_j = a_jt_jn_j' \mod n_j = a_j \mod n_j.
    \]
    Hence, $x \equiv a_j \mod n_j$ for $j = 1, \dots, k$
    
\end{proof}

\subsection{Problem 7.6.5(c)}
\begin{pbox}
    Solve the simultaneous system of congruences
    \[
        x \equiv 1 \mod 8, \quad x \equiv 2 \mod 25, \quad x \equiv 3 \mod 81
    \]
    and the simultaneous system
    \[
        y \equiv 5 \mod 8, \quad y \equiv 12 \mod 25, \quad y \equiv 47 \mod 81.
    \]
\end{pbox}

A solution to the first system is given by $x = \sum_{i=1}^k a_it_in_i'$, where
\begin{align*}
    a_1 &= 1, & n_1' &= 25 \cdot 81, \\
    a_2 &= 2, & n_2' &= 8 \cdot 81, \\
    a_3 &= 3, & n_3' &= 8 \cdot 25.
\end{align*}
Moreover, $t_i$ is given by the inverse of $n_i' \mod n_i$, which we can find using the Euclidean Algorithm. So
\[
    t_1 = 1, \quad t_2 = 12, \quad t_3 = 32,
\]
and we obtain
\[
    x = 1 \cdot 1 \cdot 25 \cdot 81 + 2 \cdot 12 \cdot 8 \cdot 81 + 3 \cdot 32 \cdot 8 \cdot 25 = 36777.
\]
Since $36777 \equiv 4377 \mod n$, where $n = 8 \cdot 25 \cdot 81 = 16200$, then any solution $x$ to the system has
\[
    x \equiv 4377 \mod n.
\]

A solution to the second system is given by $y = \sum_{i=1}^k a_it_in_i'$, where
\begin{align*}
    a_1 &= 5, & n_1' &= 25 \cdot 81, \\
    a_2 &= 12, & n_2' &= 8 \cdot 81, \\
    a_3 &= 47, & n_3' &= 8 \cdot 25.
\end{align*}
Moreover, $t_i$ is given by the inverse of $n_i' \mod n_i$ and, again, we have
\[
    t_1 = 1, \quad t_2 = 12, \quad t_3 = 32.
\]
Hence,
\[
    y = 5 \cdot 1 \cdot 25 \cdot 81 + 12 \cdot 12 \cdot 8 \cdot 81 + 47 \cdot 32 \cdot 8 \cdot 25 \equiv 15437 \mod n.
\]



\section{}
\begin{pbox}
    Use the Chinese Remainder Theorem to prove that $\Z[x] / (x(x+1)) \isom \Z \times \Z$.
\end{pbox}

\begin{proof}
    By definition, $(x)$ and $(x + 1)$ are ideals of $\Z[x]$. Moreover, if $p(x) \in \Z[x]$, then
    \begin{align*}
        p(x)
            &= \sum_{k=0}^n a_k x^k \\
            &= \sum_{k=2}^n a_k x^k + a_1x + a_0 \\
            &= \sum_{k=2}^n a_k x^k + a_1x - a_0x + a_0x + a_0 \\
            &= \left(\sum_{k=2}^n a_k x^{k-1} + a_1 - a_0\right) x + a_0(x + 1).
    \end{align*}
    That is, $p(x)$ can be written as an element of $(x) + (x + 1)$, so the two ideals are comaximal in $\Z[x]$. So, by the Chinese Remainder Theorem, we have
    \[
        \Z/(x(x+1)) \isom \Z/(x) \times \Z/(x + 1) \isom \Z \times \Z.
    \]
    
\end{proof}

\newpage
\section{}
\subsection{}
\begin{pbox}
    Show that $(x^2 - 2, x^4 - x + 2)$ is a principal ideal of $\Q[x]$.
\end{pbox}

\begin{proof}
    Since $\Q$ is a field, then $\Q[x]$ is a principal ideal domain, so $(x^2 - 2, x^4 - x + 2)$ is a principal ideal of $\Q[x]$.
    
\end{proof}

\subsection{}
\begin{pbox}
    Use the Euclidean Algorithm to show that $(x^2 - 2, x^4 - x + 2)$ is generated by some $a \in \Q^\times$.
\end{pbox}

\begin{proof}
    Using the Euclidean Algorithm, we find
    \begin{align*}
        x^4 - x + 2 &= (x^2 - 2)(x^2 + 2) - x + 6, \\
        x^2 - 2 &= (-x + 6)(-x - 6) + 34, \\
        -x + 6 &= 34\left(\tfrac{-1}{34}x + \tfrac{6}{34}\right) + 0.
    \end{align*}
    Since $34 \in \Q^\times$ is the last nonzero remainder in the Euclidean Algorithm, then we have the principal ideal
    \[
        (x^2 - 2, x^4 - x + 2) = (34).
    \]
    In particular, $(x^2 - 2, x^4 - x + 2) = \Q[x]$ since $34$ is a unit of $\Q[x]$.
    
\end{proof}

\subsection{}
\begin{pbox}
    Find the inverse of $\eqc{x^4 - x + 2}$ in $\Q[x]/(x^2 - 2)$
\end{pbox}

Using back substitution on the Euclidean on the above application of the Euclidean Algorithm, we find
\begin{align*}
    34
        &= (x^2 - 2) - (-x + 6)(-x - 6), \\
        &= (x^2 - 2) + ((x^4 - x + 2) - (x^2 - 2)(x^2 + 2))(x + 6), \\
        &= (1 + (x^2 + 2)(x + 6))(x^2 - 2) + (x + 6)(x^4 - x + 2)
\end{align*}
Then in $\Q[x]/(x^2 - 2)$, we have
\begin{align*}
    \eqc{1}
        &= \eqc{\tfrac{1}{34}(1 + (x^2 + 2)(x + 6))(x^2 - 2) + \tfrac{1}{34}(x + 6)(x^4 - x + 2)} \\
        &= \eqc{\tfrac{1}{34}(1 + (x^2 + 2)(x + 6))} \cdot \eqc{x^2 - 2} + \eqc{\tfrac{1}{34}(x + 6)} \cdot \eqc{x^4 - x + 2} \\
        &= \eqc{\tfrac{1}{34}(x + 6)} \cdot \eqc{x^4 - x + 2}.
\end{align*}
Thus, $\eqc{\tfrac{1}{34}(x + 6)}$ is the inverse of $\eqc{x^4 - x + 2}$ in $\Q[x]/(x^2 - 2)$.


\section{}
\begin{pbox}
    Decide whether $(x^4 - 1, x^3 + 2x^2 - x - 2)$ is a maximal ideal of $\Q[x]$.
\end{pbox}

\begin{proposition}
    $(x^4 - 1, x^3 + 2x^2 - x - 2)$ is not a maximal ideal of $\Q[x]$.
\end{proposition}

\begin{proof}
    Using the Euclidean Algorithm, we find
    \begin{align*}
        x^4 - 1 &= (x^3 + 2x^2 - x - 2)(x - 2) + 5x^2 - 5 \\
        x^3 + 2x^2 - x - 2 &= (5x^2 - 5)\left(\tfrac{1}{5} x + \tfrac{2}{5}\right) + 0.
    \end{align*}
    Then by back substitution,
    \begin{align*}
        5x^2 - 5 &= (x^4 - 1) - (x - 2)(x^3 + 2x^2 - x - 2) \\
        x^2 - 1 &= \tfrac15(x^4 - 1) - \tfrac15(x - 2)(x^3 + 2x^2 - x - 2).
    \end{align*}
    Hence,
    \[
        (x^4 - 1, x^3 + 2x^2 - x - 2) = (x^2 - 1).
    \]
    But since $(x^2 - 1) = (x - 1)(x + 1)$, then it is not a prime ideal in $\Q[x]$ and, therefore, not a maximal ideal.
    
\end{proof}

\end{document}