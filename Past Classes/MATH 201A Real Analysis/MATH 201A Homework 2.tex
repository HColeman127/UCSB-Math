\documentclass[12pt]{article}

% Packages
\usepackage[margin=1in]{geometry}
\usepackage{fancyhdr, parskip}
\usepackage{amsmath, amsthm, amssymb}
\usepackage[shortlabels]{enumitem}

% Page Style
\makeatletter
\fancypagestyle{title}{
    \renewcommand{\headrulewidth}{0.4pt}
    \setlength{\headheight}{15pt}
    \fancyhead[R]{\@author}
    \fancyhead[L]{\@title}
    \fancyhead[C]{\@date}
}
\makeatother
\renewcommand{\maketitle}{\thispagestyle{title}}
\fancypagestyle{plain}{
    \fancyhf{}
    \renewcommand{\headrulewidth}{0pt}
    \renewcommand{\footrulewidth}{0pt}
    \fancyfoot[R]{\thepage}
}
\pagestyle{plain}

% Problem Box
\setlength{\fboxsep}{4pt}
\newlength{\myparskip}
\setlength{\myparskip}{\parskip}
\newsavebox{\savefullbox}
\newenvironment{fullbox}{\begin{lrbox}{\savefullbox}\begin{minipage}{\dimexpr\textwidth-2\fboxsep\relax}\setlength{\parskip}{\myparskip}}{\end{minipage}\end{lrbox}\framebox[\textwidth]{\usebox{\savefullbox}}}
\newenvironment{pbox}[1][]{\begin{fullbox}\ifx#1\empty\else\paragraph{#1}\fi}{\end{fullbox}}

% Default Commands
\newcommand{\isp}[1]{\quad\text{#1}\quad}
\newcommand{\N}{\mathbb{N}} 
\newcommand{\Z}{\mathbb{Z}}
\newcommand{\Q}{\mathbb{Q}}
\newcommand{\R}{\mathbb{R}}
\newcommand{\C}{\mathbb{C}}
\newcommand{\eps}{\varepsilon}
\renewcommand{\phi}{\varphi}
\renewcommand{\emptyset}{\varnothing}
\newcommand{\<}{\langle}
\renewcommand{\>}{\rangle}
\newcommand{\isom}{\cong}
\newcommand{\eqc}{\overline}
\newcommand{\clo}{\overline}
\newcommand{\teq}{\trianglelefteq}
\DeclareMathOperator{\id}{id}

% Extra Commands
\renewcommand{\AA}{\mathcal{A}}

% Document
\begin{document}
\title{MATH 201A Homework 2}
\author{Harry Coleman}
\date{October 21, 2021}
\maketitle

\begin{pbox}[1]
    Give an example of a topological space $X$ and a measure $\mu$ on $X$ so that $\mu$ is Borel but not Borel-regular.
\end{pbox}

Let $X = \{0, 1\}$ with the indiscrete topology and $\mu$ the cardinality measure. Clearly, $\mu$ is Borel, since $\emptyset$ and $X$, the only Borel sets, are always measurable. Consider $\{0\} \subseteq X$; the only Borel subset containing $\{0\}$ is $X$ itself, but $\mu(\{0\}) = 1 \ne 2 = \mu(X)$. Hence, $\mu$ is not Borel-regular.



\begin{pbox}[2]
    Let $X$ be a nonempty set and let $\{\mu_n\}_{n=1}^{\infty}$ be a sequence of measures on $X$. Assume for any subset $A \subseteq X$ the limit $\lim_{n \to \infty} \mu_n(A)$ exists and denote $\mu(A) = \lim_{n \to \infty} \mu_n(A)$.
\end{pbox}

\begin{pbox}[(i)]
    Is it true that $\mu$ is a measure on $X$ if for any $A \subseteq X$ the sequence $\{\mu_n(A)\}$ is increasing?
\end{pbox}

Yes.

\begin{proof}
    First,
    \[
        \mu(\emptyset)
            = \lim_{n \to \infty} \mu_n(\emptyset)
            = \lim_{n \to \infty} 0
            = 0.
    \]
    

    Suppose $A_1, A_2, \ldots \subseteq X$ and $A \subseteq \bigcup_{i=1}^{\infty} A_i$. By the increasing condition, $\mu_n(A_i) \leq \mu(A_i)$, for all $n, i \in \N$. Then, for all $n \in \N$,
    \[
        \mu_n(A)
            \leq \sum_{i=1}^{\infty} \mu_n(A_i)
            \leq \sum_{i=1}^{\infty} \mu(A_i).
    \]
    Letting $n \to \infty$, we obtain
    \[
        \mu(A)
            = \lim_{n \to \infty} \mu_n(A)
            \leq \sum_{i=1}^{\infty} \mu(A_i).
    \]
    This is the countable subadditivity and monotonicity, hence $\mu$ is a measure on $X$.

\end{proof}


\newpage
\begin{pbox}[(ii)]
    Assume in addition that $\mu_1(X) < \infty$, and that each of the measures $\mu_n$ is Borel-regular. Is it true that $\mu$ is a measure on $X$ if for any $A \subseteq X$ the sequence $\{\mu_n(A)\}$ is decreasing?
\end{pbox}

Yes.

\begin{proof}
    First,
    \[
        \mu(\emptyset)
            = \lim_{n \to \infty} \mu_n(\emptyset)
            = \lim_{n \to \infty} 0
            = 0.
    \]
    
    Next, $\mu$ is is monotone. Suppose $A \subseteq B \subseteq X$, then
    \[
        \mu(A)
            = \lim_{n \to \infty} \mu_n(A)
            \leq \lim_{n \to \infty} \mu_n(B)
            = \mu(B).
    \]



    Let $A_1, A_2, \dots \subseteq X$ be mutually disjoint Borel sets. Since each $\mu_n$ is Borel-regular, then each $A_i$ is $\mu_n$-measurable. Then,
    \[
        \mu\left(\bigcup_{i=1}^{\infty} A_i\right)
            = \lim_{n \to \infty} \mu_n\left(\bigcup_{i=1}^{\infty} A_i\right)
            = \lim_{n \to \infty} \sum_{i=1}^{\infty} \mu_n(A_i).
    \]
    Since $\mu_n(A_i) \leq \mu_1(A_i)$ and $\sum_{i=1}^{\infty} \mu_1(A_i) = \mu_1(\bigcup_{i=1}^{\infty} A_i) < \infty$, we may perform the interchange
    \[
        \mu\left(\bigcup_{i=1}^{\infty} A_i\right)
            = \lim_{n \to \infty} \sum_{i=1}^{\infty} \mu_n(A_i)
            = \sum_{i=1}^{\infty} \lim_{n \to \infty} \mu_n(A_i)
            = \sum_{i=1}^{\infty} \mu(A_i).
    \]
    This shows $\mu$ is countably additive on disjoint Borel sets.

    Now, suppose the $A_i$'s are not necessarily disjoint Borel sets. Define the Borel sets $B_1 = A_1$ and $B_i = A_i \setminus B_{i-1}$ for $i \geq 2$. Then $\bigcup_{i=1}^{\infty} A_i = \bigcup_{i=1}^{\infty} B_i$ and $B_i \subseteq A_i$, so
    \[
        \mu\left(\bigcup_{i=1}^{\infty} A_i\right)
            = \sum_{i=1}^{\infty} \mu(B_i)
            \leq \sum_{i=1}^{\infty} \mu(A_i).
    \]
    This shows $\mu$ is countably subadditive on Borel sets.

    Now, suppose the $A_i$'s are arbitrary sets. For each $n, i$, there is a Borel set $B_{n, i} \subseteq X$ such that $A_i \subseteq B_{n,i}$ and $\mu_n(A_i) = \mu_n(B_{n,i})$. Define the Borel set $B_i = \bigcap_{n=1}^{\infty} B_{n,i}$, then $A_i \subseteq B_i \subseteq B_{n,i}$. So
    \[
        \mu_n(A_i) \leq \mu_n(B_i) \leq \mu_n(B_{n,i}) = \mu_n(A_i),
    \]
    implying $\mu_n(A_i) = \mu_n(B_i)$. Letting $n \to \infty$, we find $\mu(A_i) = \mu(B_i)$. Then
    \[
        \mu\left(\bigcup_{i=1}^{\infty} A_i\right)
            \leq \mu\left(\bigcup_{i=1}^{\infty} B_i\right)
            \leq \sum_{i=1}^{\infty} \mu(B_i)
            = \sum_{i=1}^{\infty} \mu(A_i).
    \]
    This shows $\mu$ is countably subadditive on all sets.


\end{proof}


\newpage
\begin{pbox}[3]
    Let X be a nonempty set and $F$ be a collection of functions $f : X \to \R$ with the following properties:
    \begin{enumerate}[(i)]
        \item The constant function $f(x) \equiv 1 \in F$, and if $f, g \in F$ and $c \in \R$, then $f + g, fg, cf \in F$.
        \item If a sequence $\{f_n\} \subseteq F$ has as pointwise limit in $X$: $f(x) = \lim_{n \to \infty} f_n(x)$ for all $x \in X$, then $f \in F$.
    \end{enumerate}
    Prove that the collection $\AA = \{A \subseteq X : \chi_A \in F\}$ is a $\sigma$-algebra, where $\chi_A$ is the characteristic function of the set $A$.
\end{pbox}

\begin{proof}
    Since $\chi_X = 1 \in F$, we know $X \in \AA$.

    If $A \in F$, then $\chi_{A^c} = 1 - \chi_A \in F$, implying $A^c \in \AA$.

    Suppose $A_1, A_2, \ldots \in \AA$ and let $A = \bigcap_{i=1}^{\infty} A_i$. For $n \in \N$, define $B_n = \bigcap_{i=1}^{n} A_i$, then
    \[
        \chi_{B_n} = \prod_{i=1}^{n} \chi_{A_i} \in F.
    \]
    Then $\chi_{B_n} \to \chi_A$ pointwise in $X$, so $\chi_A \in F$, implying $A \in \AA$.

\end{proof}



\begin{pbox}[4]
    Prove that any open subset of $\R^n$ can be expressed as a countable union of closed balls in $\R^n$

    \textbf{Remark.} The statement is true for any separable metric space $X$.
\end{pbox}

\begin{proof}
    Let $X$ be a separable metric space and let $Y \subseteq X$ be a countable dense subset. Let $U \subseteq X$ be an open subset. For each $x \in U$, define the radius $r_x = d(x, U^c) > 0$ and the closed ball $E_x = \clo{B_{r_x/2}(x)}$. We claim that $U = \bigcup_{y \in U \cap Y} E_y$ (a countable union of closed balls).

    By construction, $E_y \subset B_{r_y}(y) \subseteq U$ for all $y \in U \cap Y$, so $\bigcup_{y \in U \cap Y} E_y \subseteq U$.

    For each $x \in U$, consider the open neighborhood $B_{r_x/4}(x) \subseteq U$ of $x$. Since $Y$ is a dense subset of $X$, we can find some $y \in B_{r_x/4}(x)$. Then
    \[
        r_x
            \leq d(x, y) + d(y, U^c) 
            = d(x, y) + r_y
            \leq \tfrac{1}{4}r_x + r_y,
    \]
    so $\frac{3}{4}r_x \leq r_y$. Then
    \[
        d(x, y)
            \leq \tfrac{1}{4}r_x
            \leq \tfrac{1}{2} \cdot \tfrac{3}{4}r_x
            \leq \tfrac{1}{2}r_y,
    \]
    so $x \in E_y$. Hence, $U \subseteq \bigcup_{y \in U \cap Y} E_y$, proving the equality.

\end{proof}

\end{document}