\documentclass[12pt]{article}

% Packages
\usepackage[margin=1in]{geometry}
\usepackage{fancyhdr, parskip}
\usepackage{amsmath, amsthm, amssymb}
\usepackage{tikz, tikz-cd}

% Page Style
\makeatletter
\fancypagestyle{title}{
    \renewcommand{\headrulewidth}{0.4pt}
    \setlength{\headheight}{15pt}
    \fancyhead[R]{\@author}
    \fancyhead[L]{\@title}
    \fancyhead[C]{\@date}
}
\makeatother
\renewcommand{\maketitle}{\thispagestyle{title}}
\fancypagestyle{plain}{
    \fancyhf{}
    \renewcommand{\headrulewidth}{0pt}
    \renewcommand{\footrulewidth}{0pt}
    \fancyfoot[R]{\thepage}
}
\pagestyle{plain}

% Problem Box
\setlength{\fboxsep}{4pt}
\newlength{\myparskip}
\setlength{\myparskip}{\parskip}
\newsavebox{\savefullbox}
\newenvironment{fullbox}{\begin{lrbox}{\savefullbox}\begin{minipage}{\dimexpr\textwidth-2\fboxsep\relax}\setlength{\parskip}{\myparskip}}{\end{minipage}\end{lrbox}\framebox[\textwidth]{\usebox{\savefullbox}}}
\newenvironment{pbox}[1][]{\begin{fullbox}\ifx#1\empty\else\paragraph{#1}\phantom{}\fi}{\end{fullbox}}

% Theorem Environments
\theoremstyle{definition}
\newtheorem{lemma}{Lemma}
\newtheorem{corollary}{Corollary}


% Tikz Environments
\newenvironment{drawing}{\begin{center}\begin{tikzpicture}}{\end{tikzpicture}\end{center}}
\tikzcdset{row sep/normal=0pt}
\newenvironment{cd}{\begin{center}\begin{tikzcd}}{\end{tikzcd}\end{center}}

% Default Commands
\newcommand{\isp}[1]{\quad\text{#1}\quad}
\newcommand{\N}{\mathbb{N}} 
\newcommand{\Z}{\mathbb{Z}}
\newcommand{\Q}{\mathbb{Q}}
\newcommand{\R}{\mathbb{R}}
\newcommand{\C}{\mathbb{C}}
\newcommand{\A}{\mathbb{A}}
\renewcommand{\P}{\mathbb{P}}
\newcommand{\eps}{\varepsilon}
\renewcommand{\phi}{\varphi}
\renewcommand{\emptyset}{\varnothing}
\newcommand{\<}{\langle}
\renewcommand{\>}{\rangle}
\newcommand{\isom}{\cong}
\newcommand{\eqc}{\overline}
\newcommand{\clo}{\overline}
\newcommand{\teq}{\trianglelefteq}
\DeclareMathOperator{\id}{id}
\DeclareMathOperator{\im}{im}

% Extra Commands
\newcommand{\dd}{\,\mathrm{d}}

\newcommand{\odv}[3][]{\frac{\mathrm{d}^{#1}#2}{\mathrm{d}{#3}^{#1}}}

\newcommand{\dto}{\longrightarrow}

\newcommand{\qty}[1]{\left(#1\right)}
\newcommand{\abs}[1]{\left|#1\right|}
\newcommand{\pfrac}[2]{\qty{\frac{#1}{#2}}}

\newcommand{\tint}{{\textstyle\int}}
\newcommand{\lint}{{\textstyle\int_*}}
\newcommand{\uint}{{\textstyle\int^*}}


% Document
\begin{document}
\title{MATH 201A Homework 5}
\author{Harry Coleman}
\date{December 2, 2021}
\maketitle

\begin{pbox}[1 (Integrability of the Product)]
    Let $X$ be a nonempty set and let $\mu$ be a measure on $X$.
    Prove that if $\mu$-measurable functions $f, g : X \to [-\infty, \infty]$ are such that $f$ is $\mu$-summable on $X$ and $g$ is bounded on $X$ ($|g(x)| \leq M$ for $\mu$-a.e.\ $x \in X$), then the product $fg$ is $\mu$-summable and
    \[
        \int_X |fg| \dd\mu \leq M \int_X |f| \dd\mu.
    \]
\end{pbox}

\begin{proof}
    First, assume $f$ and $g$ are nonnegative, then their product $fg$ is nonnegative and $\mu$-measurable---therefore $\mu$-integrable.
    Denote $A = g^{-1}([-M, M])$, which is $\mu$-measurable since $g$ is $\mu$-measurable.
    Note that $\mu(X \setminus A) = 0$ so $\int_X h \dd\mu = \int_A h \dd\mu$ for all $\mu$-integrable functions $h : X \to [-\infty, \infty]$.
    Then
    \[
        \int_X fg \dd\mu
            = \int_A fg \dd\mu
            \leq \int_A fM \dd\mu
            = M \int_A f \dd\mu
            = M \int_X f \dd\mu.
    \]
    Since $f$ is $\mu$-summable and $M < \infty$, we conclude that $fg$ is $\mu$-summable.
    
    For $f$ and $g$ arbitrary write
    \[
        fg
            = (f^+ - f^-)(g^+ - g^-)
            = f^+g^+ - f^+g^- - f^-g^+ + f^-g^-.
    \]
    Then each $f^\pm$ is $\mu$-summable and each $g^\pm$ is bounded $\mu$-a.e.\ on $X$ by $M$.
    An application of the first case tells us each $f^\pm g^\pm$ is $\mu$-summable with $\int_X f^\pm g^\pm \dd\mu \leq M \int_X f^\pm \dd\mu$.
    It follows that $fg$ is $\mu$-summable as a finite $\R$-linear combination of $\mu$-summable functions.
    
    Similarly, $|f|$ is $\mu$-summable and $|g|$ is bounded $\mu$-a.e.\ on $X$ by $M$, so the first case gives
    \[
        \int_X |fg| \dd\mu
            = \int_X |f||g| \dd\mu
            \leq M \int_X |f| \dd\mu.
    \]
\end{proof}

\newpage
\begin{pbox}[2]
    Let $X$ be a nonempty set and let $\mu$ be a measure on $X$.
    Assume $\mu$-summable functions $f, f_n : X \to [-\infty, \infty]$ are such that
    \[
        f_n \dto f \qquad \text{$\mu$-a.e.\ in $X$}
    \]
    and
    \[
        \int_X |f_n| \dd\mu \dto \int_X |f| \dd\mu.
    \]
    Prove that
    \[
        \int_X |f_n - f| \dd\mu \dto 0.
    \]
\end{pbox}

\begin{proof}
    For $n \in \N$ define the nonnegative $\mu$-summable function $g_n = |f_n| + |f|$, then define
    \[
        g = \liminf_{n \to \infty} g_n.
    \]
    It follows that $g$ and $2|f|$ agree $\mu$-a.e.\ in $X$.
    If $A = (g - 2|f|)^{-1}(0)$ then $\mu(X \setminus A) = 0$, so
    \[
        \int_X g \dd\mu
            = \int_A g \dd\mu
            = \int_A 2|f| \dd\mu
            = \int_X 2|f| \dd\mu.
    \]
    For $n \in \N$ consider the nonnegative $\mu$-summable function $g_n - |f_n - f|$.
    By Fatou's Lemma,
    \[
        \int_X g \dd\mu 
        = \int_X \liminf_{n \to \infty} (g_n - |f_n - f|) \dd\mu
            \leq \liminf_{n \to \infty} \int_X (g_n - |f_n - f|) \dd\mu.
    \]
    Then
    \[
        \int_X \liminf_{n \to \infty} g_n \dd\mu
            = \int_X |f| \dd\mu + \lim_{n \to \infty} \int_X |f_n| \dd\mu
            = \int_X 2|f| \dd\mu.
    \]
    Since $f$ is $\mu$-summable we can subtract the integral of $2|f|$ from both sides to obtain
    \[
        0 \leq -\liminf_{n \to \infty} \int_X |f_n - f| \dd\mu,
    \]
    or equivalently
    \[
        \limsup_{n \to \infty} \int_X |f_n - f| \dd\mu \leq 0.
    \]
    Hence, $\int_X |f_n - f| \dd\mu \to 0$.
\end{proof}

\newpage
\begin{pbox}[3]
    Let $X$ be a topological space and let $\mu$ be a finite measure on $X$, i.e., $\mu(X) < \infty$.
    A family of $\mu$-measurable functions $f_n : X \to \R$ is called \textbf{uniformly integrable} in $X$ if for any $\eps > 0$ there exists $M > 0$ such that
    \[
        \int_{\{x \,:\, |f_n(x)| > M\}} |f_n(x)| \dd\mu < \eps \qquad\text{for all } n = 1, 2, \dots.
    \]
    Similarly $\{f_n\}$ is called \textbf{uniformly absolutely continuous} if for any $\eps > 0$ there exists $\delta > 0$ such that for any $\mu$-measurable set $A \subseteq X$ with $\mu(A) < \delta$ one has
    \[
        \abs{\int_A f_n(x) \dd\mu} < \eps \qquad\text{for all } n = 1, 2, \dots.
    \]
\end{pbox}

\begin{lemma}
    If $\{f_n\}$ is uniformly absolutely continuous then so are $\{f_n^+\}$ and $\{f_n^-\}$ (with the same choice of $\delta$).
\end{lemma}

\begin{proof}
    Given $\eps > 0$ let $\delta > 0$ be as in the definition.
    For $n \in \N$ define the set
    \[
        U_n = f_n^{-1}([0, \infty)) = \{x \in X : f_n(x) \geq 0\},
    \]
    which is $\mu$-measurable since $f_n$ is $\mu$-measurable, so that $f_n^+ = f_n\chi_{U_n}$.
    Given a $\mu$-measurable set $A \subseteq X$ with $\mu(A) < \delta$ we have
    \[
        \int_A f_n^+ \dd\mu
            = \int_A f_n \chi_{U_n} \dd\mu
            = \int_{A \cap U_n} f_n \dd\mu.
    \]
    Then $A \cap U_n$ is $\mu$-measurable with $\mu(A \cap U_n) \leq \mu(A) < \delta$, hence
    \[
        \abs{\int_A f_n^+ \dd\mu}
            = \abs{\int_{A \cap U_n} f_n \dd\mu}
            < \eps.
    \]

    The proof for $\{f^-\}$ is similar with $L_n = f_n^{-1}((-\infty, 0])$ and $f_n^- = -f_n\chi_{L_n}$.
\end{proof}


\begin{pbox}
    Prove that $\{f_n\}$ is uniformly integrable if and only if
    \[
        \sup_{n} \int_X |f_n(x)| \dd\mu < \infty
    \]
    and $\{f_n\}$ is uniformly absolutely continuous.
\end{pbox}

\begin{proof}
    Suppose $\{f_n\}$ is uniformly integrable.
    Choose $\eps = 1$ and let $M > 0$ be as in the definition.
    For $n \in \N$ define the set
    \[
        A_n = f_n^{-1}([-M, M]) = \{x \in X : |f_n(x)| \leq M\},
    \]
    which is $\mu$-measurable since $f_n$ is $\mu$-measurable.
    We split the integral over $X$ into two parts:
    \[
        \int_X |f_n| \dd\mu
            = \int_{A_n} |f_n| \dd\mu + \int_{X \setminus A_n} |f_n| \dd\mu.
    \]
    By assumption, we have
    \[
        \int_{X \setminus A_n} |f_n| \dd\mu < 1.
    \]
    Monotonicity of the $\mu$-integral gives us
    \[
        \int_{A_n} |f_n| \dd\mu
            \leq  \int_{A_n} M\dd\mu
            = M \int_X \chi_{A_n} \dd\mu
            = M \mu(A_n)
            \leq M \mu(X).
    \]
    Letting $R = M \mu(X) + 1 < \infty$, we deduce that $\int_X |f_n| \dd\mu < R$ for all $n \in \N$.
    Passing to the supremum, we obtain
    \[
        \sup_n \int_X |f_n| \dd\mu \leq R < \infty.
    \]
    For an unknown $\delta > 0$ suppose that $A \subseteq X$ is a $\mu$-measurable set with $\mu(A) < \delta$.
    Then
    \[
        \abs{\int_A f_n \dd\mu}
            \leq \int_A |f_n| \dd\mu
            \leq \int_A M\dd\mu
            = M \mu(A)
            < M \delta.
    \]
    So given $\eps > 0$ we can choose $\delta \leq \eps/M$, hence $\{f_n\}$ is uniformly absolutely continuous.

    We now show the reverse direction---suppose $R = \sup_n \int_X |f_n| \dd\mu < \infty$ and $\{f_n\}$ is uniformly absolutely continuous.
    For $n \in \N$ and an unknown $M > 0$ define the set
    \[
        B_n = X \setminus f_n^{-1}([-M, M]) = \{x \in X : |f_n(x)| > M\},
    \]
    which is $\mu$-measurable since $f_n$ is $\mu$-measurable.
    Then
    \[
        \mu(B_n)
            = \frac{1}{M} \int_{B_n} M \dd\mu
            \leq \frac{1}{M} \int_{B_n} |f_n| \dd\mu
            \leq \frac{1}{M} \int_X |f_n| \dd\mu
            \leq \frac{R}{M}.
    \]
    Let $\eps > 0$ be given.
    By Lemma 1 we can choose $\delta > 0$ such that for all $\mu$-measurable sets $A \subseteq X$ with $\mu(A) < \delta$ we have
    \[
        \int_A f_n^+ \dd\mu < \frac{\eps}{2}
        \isp{and}
        \int_A f_n^- \dd\mu < \frac{\eps}{2}.
    \]
    Choosing $M > R/\delta$ gives us $\mu(B_n) < \delta$, so
    \[
        \int_{B_n} |f_n| \dd\mu
            = \int_{B_n} (f_n^+ + f_n^-) \dd\mu
            = \int_{B_n} f_n^+ \dd\mu + \int_{B_n} f_n^- \dd\mu
            < \eps.
    \]
    Hence, $\{f_n\}$ is uniformly integrable.
\end{proof}




\newpage
\begin{pbox}[4]
    Compute the limit
    \[
        \lim_{n \to \infty} \int_{0}^{n} \qty{1 - \frac{x}{n}}^n \ln\qty{2 + \cos\pfrac{x}{n}} \dd{x} 
    \]
\end{pbox}

For $n \in \N$ and $x \in [0, \infty)$ define the function
\[
    f_n(x) = \begin{cases}
        \qty{1 - \frac{x}{n}}^n \ln\qty{2 + \cos\pfrac{x}{n}} &\text{if } x \leq n, \\
        0 &\text{otherwise}.
    \end{cases}
\]
Then we can write
\[
    \int_{0}^{n} \qty{1 - \frac{x}{n}}^n \ln\qty{2 + \cos\pfrac{x}{n}} \dd{x} 
        = \int_{0}^{\infty} f_n(x) \dd{x}.
\]

Note that $\cos x$ is positive and decreasing on $[0, \pi/2]$, so for $n \in \N$ and $x \in [0, n]$ we have
\[
    \cos\pfrac{x}{n} \leq \frac{x}{n + 1}.
\]
Since the logarithm is increasing on $(0, \infty)$, we obtain
\[
    \ln\qty{2 + \cos\pfrac{x}{n}} \leq \ln\qty{2 + \cos\pfrac{x}{n+1}}.
\]

Check that $0 \leq (1 - \frac{x}{n})^n \leq (1 - \frac{x}{n+1})^{n+1}$ for all $n \in \N$ and $x \in [0, n]$.

It follows that $0 \leq f_n(x) \leq f_{n+1}(x)$ for all $n \in \N$ and $x \in [0, \infty)$.
Monotone convergence of the Lebesgue integral gives us
\[
    \lim_{n \to \infty} \int_{0}^{\infty} f_n(x) \dd{x}
        = \int_{0}^{\infty} e^{-x} \ln(3 + \cos 0) \dd{x}
        = \ln 3 \int_{-\infty}^{0} e^x \dd{x}
        = \ln 3 \cdot e^0
        = \ln 3.
\] 

\end{document}