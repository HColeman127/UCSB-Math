\documentclass[12pt]{article}

% Packages
\usepackage[margin=1in]{geometry}
\usepackage{amsmath, amsthm, amssymb, physics}

% Problem Box
\setlength{\fboxsep}{4pt}
\newsavebox{\mybox}
\newenvironment{problem}
    {\begin{lrbox}{\mybox}\begin{minipage}{0.98\textwidth}}
    {\end{minipage}\end{lrbox}\begin{center}\framebox[\textwidth]{\usebox{\mybox}}\end{center}}

% Options
\renewcommand{\thesubsection}{\thesection(\alph{subsection})}
\allowdisplaybreaks
\addtolength{\jot}{1em}
\theoremstyle{definition}

% Default Commands
\newtheorem{proposition}{Proposition}
\newtheorem{lemma}{Lemma}
\newcommand{\ds}{\displaystyle}
\newcommand{\isp}[1]{\quad\text{#1}\quad}
\newcommand{\N}{\mathbb{N}}
\newcommand{\Z}{\mathbb{Z}}
\newcommand{\Q}{\mathbb{Q}}
\newcommand{\R}{\mathbb{R}}
\newcommand{\C}{\mathbb{C}}
\newcommand{\eps}{\varepsilon}
\renewcommand{\phi}{\varphi}
\renewcommand{\emptyset}{\varnothing}

% Extra Commands
\newcommand{\bd}{\partial}
\newcommand{\Log}{\operatorname{Log}}


% Document Info
\title{Homework 1\\
    \large MATH CS 122B
}
\author{Harry Coleman}
\date{January 24, 2021}

% Begin Document
\begin{document}
\maketitle

\section{Exercise VI.1.1}
\begin{problem}
    Find all possible Laurent expansions centered at $0$ of the following functions:
\end{problem}

\subsection{Exercise VI.1.1(a)}
\begin{problem}
    \[
        \frac{1}{z^2 - z}
    \]
\end{problem}

The function
\[
    f(z) = \frac{1}{z^2 - z} = \frac{1}{z(z - 1)}
\]
has singularities at $0$ and $1$, so it is analytic on the annuli $A_{0,1}(0)$ and $A_{1,\infty}(0)$. First expanding for $z \in A_{0,1}(0)$, the partial fraction decomposition yields a geometric series
\[
    f(z) = \frac{-1}{z} + \frac{1}{z - 1} = -z^{-1} - \sum_{k=0}^\infty z^k.
\]
Hence, the Laurent expansion
\[
    f(z) = \sum_{k=-1}^\infty -z^k, \quad z \in A_{0,1}(0).
\]
To expand $f$ for $z \in A_{1,\infty}(0)$, we consider the function
\[
    g(w) = f(w^{-1}) = \frac{1}{w^{-2} - w^{-1}} = \frac{w^2}{1 - w},
\]
which is analytic in $B_1(0)$. Moreover, its power series expansion at $0$ for $w \in B_1(0)$ is given by the geometric series
\[
    g(w) = w^2\sum_{k=0}^\infty w^k = \sum_{k=2}^\infty w^k.
\]
Then, taking $f(z) = g(z^{-1})$, we obtain the Laurent series expansion
\[
    f(z) = \sum_{k=-\infty}^{-2} w^k, \quad z \in A_{1,\infty}(0).
\]

\subsection{Exercise VI.1.1(b)}
\begin{problem}
    \[
        \frac{z - 1}{z + 1}
    \]
\end{problem}

The function
\[
    f(z) = \frac{z - 1}{z + 1}
\]
has a singularity at $-1$, so it is analytic on $B_1(0)$ and $A_{1,\infty}(0)$. For $z \in B_1(0)$, we write $f$ in terms of a geometric series
\[
    f(z) = \frac{-(1 - z)}{1 - (-z)} = -(1 - z)\sum_{k=0}^\infty (-z)^k.
\]
Simplifying further, we obtain
\begin{align*}
    f(z)
        &= -\left(\sum_{k=0}^\infty (-z)^k + (-z)\sum_{k=0}^\infty (-z)^k\right) \\
        &= -\left(1 + \sum_{k=1}^\infty (-z)^k + \sum_{k=1}^\infty (-z)^k\right) \\
        &= -1 - \sum_{k=1}^\infty 2(-z)^k.
\end{align*}
Hence, the Laurent series expansion
\[
    f(z) = -z^0 + \sum_{k=1}^\infty 2(-1)^{k+1} z^k, \quad z \in B_1(0).
\]
To expand $f$ in $A_{1,\infty}(0)$, we consider the function on $B_1(0)$ defined by
\[
    g(w) = f(w^{-1}) = \frac{w^{-1} - 1}{w^{-1} + 1} = \frac{1 - w}{1 + w}.
\]
Since this is equal to $-f(w)$ in $B_1(0)$, then we have
\[
    g(w) = -f(w) = w^0 + \sum_{k=1}^\infty 2(-1)^k w^k.
\]
Taking $f(z) = g(z^{-1})$, we obtain the Laurent series expansion
\[
    f(z) = z^0 + \sum_{k=-\infty}^{-1} 2(-1)^k z^k, \quad z \in A_{1,\infty}(0).
\]


\subsection{Exercise VI.1.1(c)}
\begin{problem}
    \[
        \frac{1}{(z^2 -1)(z^2 - 4)}
    \]
\end{problem}

The function
\[
    f(z) = \frac{1}{(z^2 -1)(z^2 - 4)} = \frac{1}{(1 - z^2)(4 - z^2)}
\]
has singularities at $\pm1$ and $\pm2$, so it is analytic on $B_1(0)$, $A_{1,2}(0)$, and $A_{2,\infty}(0)$. Taking the partial fraction decomposition, we have
\begin{align*}
    f(z)
        &= \frac16\left(\frac{1}{1-z} + \frac{1}{1+z}\right) + \frac{-1}{12}\left(\frac{1}{2-z} + \frac{1}{2+z}\right) \\
        &= \frac16\left(\frac{2}{1-z^2}\right) + \frac{-1}{12}\left(\frac{4}{4-z^2}\right) \\
        &= \frac13\left(\frac{1}{1-z^2}\right) + \frac{-1}{12}\left(\frac{1}{1-\frac{z^2}{4}}\right).
\end{align*}
For $|z| < 1$, these can be rewritten in terms of geometric series to obtain the Laurent series expansion in $B_1(0)$:
\begin{align*}
    f(z)
        &= \frac13 \sum_{k=0}^\infty (z^2)^k + \frac{-1}{12} \sum_{k=0}^\infty \left(\frac{z^2}{4}\right)^k \\
        &= \sum_{k=0}^\infty \frac{z^{2k}}{3}  + \sum_{k=0}^\infty \frac{-z^{2k}}{3 \cdot 4^{k+1}} \\
        &= \sum_{k=0}^\infty \frac{1}{3}\left(1  - \frac{1}{4^{k+1}}\right)z^{2k}
\end{align*}

For $|z| < 2$, we still have the the geometric series
\[
    f_0(z)
        = \frac{-1}{12}\left(\frac{1}{1-\frac{z^2}{4}}\right)
        = \frac{-1}{12} \sum_{k=0}^\infty \left(\frac{z^2}{4}\right)^k
        = \sum_{k=0}^\infty \frac{-z^{2k}}{3 \cdot 4^{k+1}}.
\]
For $|z| > 1$, we expand the following function at infinity:
\[
    f_1(z^{-1})
        = \frac13\left(\frac{1}{1-z^{-2}}\right)
        = \frac{-z^2}{3}\left(\frac{1}{1 - z^2}\right)
        = \sum_{k=0}^\infty \frac{-z^{2k + 2}}{3}
\]
Then the Laurent series expansion of $f$ in $A_{1, 2}(0)$ is
\begin{align*}
    f(z) 
        &= f_0(z) + f_1(0) \\
        &= \sum_{k=0}^\infty \frac{-z^{2k}}{3 \cdot 4^{k+1}} + \sum_{k=-\infty}^{-1} \frac{-z^{2k}}{3} \\
        &= \sum_{k=-\infty}^\infty \frac{-1}{3}\left(1 + \frac{1}{4^{k+1}}\right)z^{2k}
\end{align*}

For $|z| > 2$, we keep
\[
    f_1(z) = \sum_{k=-\infty}^{-1} \frac{-z^{2k}}{3}.
\]
Then expanding $f_0$ at infinity, 
\[
    f_0(z^{-1})
        = \frac{-1}{12}\left(\frac{1}{1-\frac{z^{-2}}{4}}\right) \\
        = \frac{z^2}{3}\left(\frac{1}{1-4z^2}\right) \\
        = \sum_{k=0}^\infty \frac{4^k z^{2k+2}}{3}
\]
Then the Laurent series expansion of $f$ in $A_{2, \infty}(0)$ is
\begin{align*}
    f(z)
        &= f_0(z) + f_1(0) \\
        &= \sum_{k=-\infty}^{-1} \frac{z^{2k}}{3\cdot 4^{k+1}} + \sum_{k=-\infty}^{-1} \frac{-z^{2k}}{3} \\
        &= \sum_{k=-\infty}^{-1} \frac{1}{3}\left(-1 + \frac{1}{4^{k+1}}\right) z^{2k}
\end{align*}



\newpage
\section{Exercise VI.1.4.}
\begin{problem}
    Suppose that $f(z) = f_0(z) + f_1(z)$ is the Laurent decomposition of an analytic function $f(z)$ on the annulus $\{A < |z| < B\}$. Show that if $f(z)$ is an even function, then $f_0(z)$ and $f_1(z)$ are even functions, and the Laurent series expansion of $f(z)$ has only even powers of $z$. Show that if $f(z)$ is an odd function, then $f_0(z)$ and $f_1(z)$ are odd functions, and the Laurent series expansion of $f(z)$ has only odd powers of $z$.
\end{problem}

\begin{proof}
    We have the power series expansions
    \[
        f_0(z) = \sum_{k=0}^\infty a_k z^k, \quad |z| < B,
    \]
    and
    \[
        f_1(z) = \sum_{k=-\infty}^{-1} a_k z^k, \quad |z| > A.
    \]
    Then $f$ has the Laurent series expansion
    \[
        f(z) = \sum_{-\infty}^\infty a_k z^k, \quad A < |z| < B.
    \]
    If $f$ is even, we also have another Laurent series expansion given by
    \[
        f(z) = f(-z) = \sum_{-\infty}^\infty a_k (-z)^k = \sum_{-\infty}^\infty a_k(-1)^k z^k.
    \]
    Since the coefficients of the Laurent series expansion are unique, then we must have $a_k = a_k(-1)^k$ for all $k \in \Z$. For odd $k$, this means $a_k = -a_k$, so $a_k = 0$. Thus, the Laurent series expansion of $f$ has only even powers of $z$. Moreover, since $f_0$ and $f_1$ have power series expansions with all even powers of $z$, and $z^2 = (-z)^2$, then $f_0$ and $f_1$ are even functions.
    
    If $f$ is odd, we have a second Laurent series expansion given by
    \[
        f(z) = -f(-z) = -\sum_{-\infty}^\infty a_k (-z)^k = \sum_{-\infty}^\infty a_k(-1)^{k+1} z^k.
    \]
    In this case, when $k$ is even, we have $a_k = -a_k$ implying $a_k = 0$. So the Laurent series expansion of $f$ has only odd powers of $z$ and $f_0$ and $f_1$ are odd functions.
    
\end{proof}

\newpage
\section{Exercise VI.1.6.}
\begin{problem}
    Fix an annulus $D = \{a < |z| < b\}$, and let $f(z)$ be a continuous function on its boundary $\bd D$. Show that $f(z)$ can be approximated uniformly on $\bd D$ by polynomials in $z$ and $1/z$ if and only if $f(z)$ has a continuous extension to the closed annulus $D \cup \bd D$ that is analytic on $D$.
\end{problem}

\begin{proof}
    Suppose $f$ can be approximated uniformly on $\bd D$ by polynomials in $z$ and $1/z$. i.e., there exist power series
    \[
        f_0(z) = \sum_{k=0}^\infty a_k z^k \isp{and} f_1(z) = \sum_{k=-\infty}^{-1} a_k z^k
    \]
    with $f_0 + f_1$ converging uniformly to $f$ on $\bd D$. Let $R$ be the radius of convergence such that $f_0$ converges for $|z| < R$ and diverges for $|z| > R$. Similarly, let $r$ be the radius of convergence such that $f_1$ converges for $|z| > r$ and diverges for $|z| < r$. Since $f_0$ converges uniformly for all $|z| = b$, then we must have $b < R$. And since $f_1$ converges uniformly for all $|z| = a$, then we must have $r < a$. Since $D \cup \bd D$ is a closed subset of the annulus of convergence $\{r < |z| < R\}$, then both $f_0$ and $f_1$ converge uniformly on $D \cup \bd D$, so $f_0 + f_1$ is analytic on $D \cup \bd D$.
    
    Now suppose that $f$ has a continuous extension to the closed annulus $D \cup \bd D$ that is analytic on $D$. Then we have the Laurent series expansion
    \[
        f(z) = \sum_{k=-\infty}^\infty a_k z^k, \quad z \in D.
    \]
    In other words, the sequence of functions
    \[
        f_n(z) = \sum_{k=-n}^n a_k z^k
    \]
    converges to $f$ on $D$. Moreover, this convergence is uniform for any closed annulus strictly smaller than $D$. We want to show that this convergence is also uniform on $\bd D$. 
    (I was unable to complete this proof, but might be able to force uniform continuity at boundary using uniform continuity of each $f_n$ on $D \cup \bd D$, uniform convergence on a closed subannulus, then picking a point close to $z_0$ then using triangle inequality to get $f_n(z_0)$ close to $f(z_0)$. Maybe there is a correct order to apply continuity to prove this way, but I could not get it)
    
\end{proof}

\newpage
\section{Exercise VI.2.1}
\begin{problem}
    Find the isolated singularities of the following functions and determine whether they are removable, essential, or poles. Determine the order of any pole, and find the principal part at each pole.
\end{problem}

\subsection{}
\begin{problem}
    \[
        z/(z^2 - 1)^2
    \]
\end{problem}

The function
\[ 
    f(z) = \frac{z}{(z^2 - 1)^2} = \frac{z}{(z - 1)^2 (z+1)^2}
\]
has singularities at $\pm1$. Consider
\[
    g(z) = \frac{z}{(z + 1)^2},
\]
which is analytic at $1$ and has a power series expansion
\[
    g(z) = \sum_{k=0}^\infty a_k(z - 1)^k, \quad |z - 1| < r,
\]
where
\[
    a_k = \frac{g(k)(1)}{k!}.
\]
Since $g(1) \ne 0$ and $f(z) = g(z)/(z - 1)^2$, then $f$ has a double pole at $1$ and
\[
    f(z) = \sum_{k = -2}^\infty a_{k-2}(z - 1)^k, \quad |z - 1| < r.
\]
Then the principal part of $f$ near $1$ is
\[
    \frac{g(1)}{(z - 1)^2} + \frac{g'(1)}{(z - 1)}
        = \frac{1/4}{(z - 1)^2} + \frac{0}{(z - 1)}
        = \frac{1}{4(z - 1)^2}.
\]

Similarly,
\[
    g(z) = \frac{z}{(z - 1)^2}
\]
is analytic at $-1$, $g(-1) \ne 0$, and $f(z) = g(z)/(z + 1)^2$, so $f$ has a double pole at $-1$. As above, we obtain a Laurent series expansion of $f$ around $-1$, from which we obtain the principal part
\[
    \frac{g(-1)}{(z + 1)^2} + \frac{g'(-1)}{(z + 1)}
        = \frac{-1/4}{(z + 1)^2} + \frac{0}{(z + 1)}
        = \frac{-1}{4(z + 1)^2}.
\]


\subsection{}
\begin{problem}
    \[
        \frac{z e^z}{z^2 - 1}
    \]
\end{problem}

The function
\[ 
    f(z) = \frac{z e^z}{z^2 - 1} = \frac{z e^z}{(z - 1)(z + 1)}
\]
has singularities at $\pm1$. Since
\[
    g(z) = \frac{z e^z}{z + 1}
\]
is analytic at $1$ with $g(1) \ne 0$ and $f(z) = g(z)/(z - 1)$, then $f$ has a simple pole at $1$. Moreover, we have the principal part of $f$ near $1$ given by
\[
    \frac{g(1)}{z - 1}
        = \frac{e/2}{z - 1}.
\]
Similarly, 
\[
    g(z) = \frac{z e^z}{z - 1}
\]
is analytic at $-1$ with $g(-1) \ne 0$ and $f(z) = g(z)/(z + 1)$, so $f$ has a simple pole at $-1$. And we have the principal part of $f$ near $-1$ given by
\[
    \frac{g(-1)}{z - 1}
        = \frac{-e/2}{z - 1}.
\]

\subsection{}
\begin{problem}
    \[
        \frac{e^{2z} - 1}{z}
    \]
\end{problem}

The function
\[
    f(z) = \frac{e^{2z} - 1}{z} = z^{-1}(e^{2z} - 1)
\]
has a singularity at $0$. We write the exponential as a power series to obtain
\[
    f(z) 
        = z^{-1} \left(\sum_{k=0} \frac{(2z)^k}{k!} - 1\right) 
        = z^{-1}\sum_{k=1} \frac{(2z)^k}{k!} 
        = \sum_{k=0} \frac{(2z)^k}{(k+1)!}. 
\]
Thus, $0$ is a removable singularity.

\subsection{}
\begin{problem}
    \[
        \tan z = \frac{\sin z}{\cos z}
    \]
\end{problem}

\subsection{}
\begin{problem}
    \[
        z^2 \sin(\frac1z)
    \]
\end{problem}

We have
\[
    z^2 \sin(\frac1z)
        = z^2 \sum_{k=0}^\infty \frac{(-1)^k  z^{-2k - 1}}{(2k +1)!}
        = \sum_{k=-\infty}^0 \frac{(-1)^k  z^{2k + 1}}{(-2k +1)!}.
\]
Since we have infinitely many negative powers of $z$, we have an essential singularity at $0$.

\subsection{}
\begin{problem}
    \[
        \frac{\cos z}{z^2 - \pi^2/4}
    \]
\end{problem}

\subsection{}
\begin{problem}
    \[
        \Log\left(1 - \frac1z\right)
    \]
\end{problem}

\subsection{}
\begin{problem}
    \[
        \frac{\Log z}{(z - 1)^3}
    \]
\end{problem}

\subsection{}
\begin{problem}
    \[
        e^{1/(z^2 + 1)}
    \]
\end{problem}

\newpage
\section{Exercise VI.2.4}
\begin{problem}
    Suppose $f(z)$ is meromorphic on the disc $\{|z| < s\}$, with only a finite number of poles in the disc. Show that the Laurent decomposition of $f(z)$ with respect to the annulus $\{s - \eps < |z| < s\}$ has the form $f(z) = f_0(z) + f_1(z)$, where $f_1(z)$ is the sum of the principal parts of $f(z)$ at its poles.
\end{problem}

\begin{proof}
    Suppose $f(z)$ has poles $z_1, \dots, z_N$ on the disc $\{|z| < s\}$. For each $n = 1, \dots, N$, we choose a radius $r_n$ such that the annulus $\{0 < |z - z_n| < r_n\}$ contains no poles of $f(z)$. Let $P_n(z)$ be the principal part of $f(z)$ at the pole $z_n$. Then $P_n(z)$ is analytic on $\{0 < |z|\}$ and $f(z) - P_n(z)$ is analytic at $z_n$.
    
    Let $f_1(z) = P_1(z) + \cdots + P_N(z)$ and $f_0(z) = f(z) - f_1(z)$. Since $f_1(z)$ is analytic on all of $\C$ except for the poles $z_1, \dots, z_N$, then $f_0(z)$ is analytic on $\{|z| < s\}$. Moreover, $f_1(z)$ is analytic on $\{s - \eps < |z|\}$ and $f_1(\infty) = 0$ since each $P_n(z)$ is the sum of negative powers of $z$. Therefore, we have the Laurent decomposition $f(z) = f_0(z) + f_1(z)$ on the annulus $\{s - \eps < |z| < s\}$.

\end{proof}


\section{Exercise VI.2.6}
\begin{problem}
    Show that if $f(z)$ is continuous on a domain $D$, and if $f(z)^8$ is analytic on $D$, then $f(z)$ is analytic on $D$.
\end{problem}

\begin{proof}
    First note that $f(z) = 0$ if and only if $f(z)^8 = 0$. Since $f(z)^8$ is analytic on $D$, then its zeros are isolated, so the zeros of $f(z)$ are isolated. Let $z_0 \in D$ such that $f(z_0) \ne 0$ and let $B_r(z_0)$ be a ball around $z_0$ containing no zeros of $f(z)$. Let $g(z) = \sqrt[8]{z}$ be the branch of the complex $8$th root such that $g(f(z_0)^8) = f(z_0)$. We pick a smaller ball $B_s(z_0)$ such that the image $B_s(f(z_0))$ contains no other branches of the complex $8$th root. Then since $f(z)$ is continuous, we must have $f(z) = g(f(z)^8)$ for all $z \in B_s(z_0)$. Since $g(f(z)^8)$ is analytic at $z_0$, then so is $f(z)$. Hence $f(z)$ is analytic on $D$ as all points which are not zeros of $f(z)$. Then for every zero $z_0$ of $f(z)$, we have $f(z) \to f(z_0) = 0$ as $z \to z_0$ since $f(z)$ is continuous. This implies that $z_0$ cannot be a singularity of $f(z)$.
    
\end{proof}




\end{document}