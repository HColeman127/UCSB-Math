\documentclass[12pt]{article}

% Packages
\usepackage[margin=1in]{geometry}
\usepackage{fancyhdr}
\usepackage{amsmath, amsthm, amssymb, physics}
\usepackage{tikz}
\usetikzlibrary{arrows,arrows.meta}

\newenvironment{drawing}{\begin{center}\begin{tikzpicture}}{\end{tikzpicture}\end{center}}

% Page Style
\fancypagestyle{plain}{
    \fancyhf{}
    \renewcommand{\headrulewidth}{0pt}
    \renewcommand{\footrulewidth}{0pt}
    \fancyfoot[R]{\thepage}
}
\pagestyle{plain}

% Problem Box
\setlength{\fboxsep}{4pt}
\newsavebox{\savefullbox}
\newenvironment{fullbox}{\begin{lrbox}{\savefullbox}\begin{minipage}{\dimexpr\textwidth-2\fboxsep\relax}}{\end{minipage}\end{lrbox}\begin{center}\framebox[\textwidth]{\usebox{\savefullbox}}\end{center}}
\newenvironment{pbox}[1][]{\begin{fullbox}\ifx#1\empty\else\paragraph{#1}\fi}{\end{fullbox}}

% Options
\renewcommand{\thesubsection}{\thesection(\alph{subsection})}
\allowdisplaybreaks
\addtolength{\jot}{4pt}
\theoremstyle{definition}

% Default Commands
\newtheorem{proposition}{Proposition}
\newtheorem{lemma}{Lemma}
\newcommand{\ds}{\displaystyle}
\newcommand{\isp}[1]{\quad\text{#1}\quad}
\newcommand{\N}{\mathbb{N}}
\newcommand{\Z}{\mathbb{Z}}
\newcommand{\Q}{\mathbb{Q}}
\newcommand{\R}{\mathbb{R}}
\newcommand{\C}{\mathbb{C}}
\newcommand{\eps}{\varepsilon}
\renewcommand{\phi}{\varphi}
\renewcommand{\emptyset}{\varnothing}
\newcommand{\pfrac}[2]{\left(\frac{#1}{#2}\right)}

% Extra Commands
\newcommand{\bd}{\partial}
\newcommand{\Arg}{\operatorname{Arg}}

% Document Info
\fancypagestyle{title}{
    \renewcommand{\headrulewidth}{0.4pt}
    \setlength{\headheight}{15pt}
    \fancyhead[R]{Harry Coleman}
    \fancyhead[L]{MATH CS 122B Midterm}
    \fancyhead[C]{February 19, 2021}
}

% Begin Document
\begin{document}
\thispagestyle{title}


\begin{pbox}[1]
    Show that if $f(z) : D \to \C$ is continuous in the domain $D \subset \C$ and $f^8(z)$ is analytic in $D$, then in fact $f(z)$ is analytic in $D$.
\end{pbox}

\begin{proof}
    Since $f^8(z)$ is analytic in the domain $D$, then its zeros are isolated. And since the zeros of $f^8(z)$ are precisely the zeros of $f(z)$, then the zeros of $f(z)$ are isolated. Since $f(z)$ is continuous, it suffices to prove that it is analytic in $D' = \{z \in D : f(z) \ne 0\}$. If some zero $z_0 \in D \setminus D'$ were a singularity of $f(z)$, then it would have to be a removable singularity, as the continuity of $f(z)$ implies its boundedness around $z_0$. In other words, $f(z)$ would not have a singularity at $z_0$, since it could be made analytic there by defining $f(z_0) = \lim_{z\to z_0} f(z) = 0$, which is simply the actual value of $f(z_0)$.
    
    Let $z_0 \in D'$ and define $w_0 = f(z_0)$. Consider the open neighborhood of $w_0$ given by
    \[
        U =  \{w \in f(D') : |\Arg w - \Arg w_0| < \tfrac{\pi}{8}\},
    \]
    which can be thought of as the intersection $f(D')$ and a sector of $\C$ around $w_0$. Now consider the function $h(w) = w^8$ defined on $U$, then $f^8(z) = h(f(z))$. We will show that $h(w)$ is injective. Suppose $w_1, w_2 \in U$ with $h(w_1) = h(w_2)$. Then
    \[
        |w_1|^8e^{i8\Arg w_1} = w_1^8 = w_2^8 = |w_2|^8e^{i8\Arg w_2}, 
    \]
    which implies $|w_1| = |w_2|$ and $\Arg w_1 = \Arg w_2 + \frac{\pi}{4}k$ for some $k \in \Z$. By the triangle inequality and the fact that $w_1, w_2 \in U$, we have
    \[
        |\Arg w_1 - \Arg w_2| \leq |\Arg w_1 - \Arg w_0| + |\Arg w_0 - \Arg w_2| < \tfrac{\pi}{4},
    \]
    so we must have $k = 0$. Thus, $h(w)$ is injective on $U$, and its inverse will be some branch of the complex $8$th root function.
    
    In particular, we can make a branch cut opposite of $w_0^8$ to obtain the slit plane
    \[
        S = \C \setminus \{re^{-i8\Arg w_0} : r \geq 0\} \supseteq h(U),
    \]
    on which we define $g(z)$ to be the branch of $\sqrt[8]{z}$ such that $|\Arg g(z) - \Arg w_0| < \pi/8$. Since every nonzero complex number has eight $8$th roots, separated from each other by an angle of $\pi/4$, then each point in $S$ has an $8$th root within an angle of $\pi/8$ of $w_0$. In other words, $g(z)$ chooses the $8$th root of $z$ which is contained in $U$. Then $g(z)$ is analytic on $S$ and, in particular, on $h(U)$.
    
    Recall that $U$ is an open neighborhood of $w_0 = f(z_0)$ contained in $f(D')$. We take the preimage $V = f^{-1}(U)$, which is an open neighborhood of $z_0$ contained in $D'$. Then $f^8(z)$ is analytic in $V$, so the composition $g(f^8(z)) = g(h(f(z)) = f(z)$ is analytic in $V$. Thus, $f(z)$ is analytic in $D'$ and, therefore, all of $D$.
    
    
    
\end{proof}



\newpage
\begin{pbox}[2]
    Let $f$ be a bounded analytic function on the unit disc $D = \{z \in \C : |z| < 1\}$. Suppose that there are a finite number of points on the boundary of $D$ such that $f(z)$ extends continuously to the arcs of $\bd D$ separating the points and satisfies $|f(e^{i\theta})| \leq M$ on the arcs. Show that $|f(z)| \leq M$ for all $z \in D$.
\end{pbox}

\begin{proof}
    Let $C \geq 0$ bound the modulus of $f(z)$, that is, $|f(z)| \leq C$ for all $z \in D$. Let $E = \{z_1, \dots, z_n\}$ be the set of points on the boundary of $D$ such that $f(z)$ extends continuously to the rest of the boundary, and define $\Gamma = \bd D \setminus E$ to be the arcs of the unit circle separating the points of $E$. For $\eps > 0$, we define the function
    \[
        g(z) = ((z - z_1) \cdots (z - z_n))^\eps f(z),
    \]
    which is analytic in $D$. Since $f(z)$ extends continuously to $\Gamma$, and $((z - z_1) \cdots (z - z_n))^\eps$ is analytic in $\C \setminus E$, then $g(z)$ extends continuously to $\Gamma$. If $z \in \Gamma$, then
    \begin{align*}
        |g(z)|
            &= (|z - z_1| \cdots |z - z_n|)^\eps |f(z)| \\
            & \leq (2^n)^\eps M.
    \end{align*}
    For any $z_j \in E$ and $z \in D$, we find
    \begin{align*}
        |g(z)| 
            &= (|z - z_1| \cdots |z - z_n|)^\eps |f(z)| \\
            &\leq |z - z_j|^\eps(2^{n-1})^\eps C,
    \end{align*}
    so $|g(z)| \to 0$ as $z \to z_j$. Thus, $g(z)$ extends continuously to the whole boundary $\bd D$, defining $g(z_j) = 0$ for all $z_j \in E$, and we conclude that $|g(z)| \leq (2^n)^\eps M$ for all $z \in \bd D$. The maximum principle tells us that in fact $|g(z)| \leq (2^n)^\eps M$ for all $z \in D$. In other words, we have shown for all $z \in D$ and $\eps > 0$,
    \[
        |f(z)| \leq \pfrac{2^n}{|z-z_1|\cdots|z-z_n|}^\eps M.
    \]
    For a fixed $z \in D$, letting $\eps \to 0$ gives us $|f(z)| \leq M$. Thus, $|f(z)| \leq M$ for all $z \in D$.
    
    
    
    
\end{proof}



\newpage
\begin{pbox}[3]
    By integrating around a keyhole contour show that
    \[
        \int_{0}^{\infty} \frac{\log x}{x^a(x+1)} \dd{x} = \frac{\pi^2 \cos(\pi a)}{\sin^2(\pi a)}, \quad 0 < a < 1.
    \]
\end{pbox}

Let $0 < \eps < 1 < R$, and we consider the keyhole contour.
\begin{drawing}
    %\draw[help lines, color=gray, dashed] (-2.9,-2.9) grid (2.9,2.9);
    
    
    %\draw[->, >=Triangle] (1.77,-1.77) arc (-45 : 45 : 2.5);
    
    \draw[] (2.5, 0) arc (0 : 90 : 2.5) node[midway, sloped]{\footnotesize{$<$}};
    \draw[] (0, 2.5) arc (90 : 180 : 2.5) node[midway, sloped]{\footnotesize{$<$}};
    \draw[] (-2.5, 0) arc (-180 : -90 : 2.5) node[midway, sloped]{\footnotesize{$>$}};
    \draw[] (0, -2.5) arc (-90 : 0 : 2.5) node[midway, sloped]{\footnotesize{$>$}};
    
    %\draw[->, >=Triangle] (1.77,1.77) arc (45 : 135 : 2.5);
    %\draw[->, >=Triangle] (-1.77,1.77) arc (135 : 225 : 2.5);
    %\draw[->, >=Triangle] (-1.77,-1.77) arc (225 : 315 : 2.5);
    
    \fill[fill=white] (2.5, 0) circle (0.12);
    \draw[] (-3,0)--(3,0);
    \draw[] (0,-3)--(0,3);
    
    \draw[fill=black] (2.5,0) circle (2pt) node[anchor=north west]{$R$};
    \draw[fill=black] (-2.5,0) circle (2pt) node[anchor=north east]{$-R$};
    
    
    \draw[] (0.215, -0.125) arc (330 : 30 : 0.25);
    \draw[] (0.215, -0.125) arc (330 : 120 : 0.25) node[midway, sloped]{\footnotesize{$<$}};
    \draw[] (0.215, 0.125) -- (2.5, 0.125) node[midway, sloped]{\footnotesize{$>\quad$}};
    \draw[] (0.215, -0.125) -- (2.5, -0.125) node[midway, sloped]{\footnotesize{$\quad<$}};
    
    \draw[] (2,2) node[]{$\Gamma_R$};
    \draw[] (-0.4,0.4) node[]{$\gamma_\eps$};
    \draw[] (1, 1) node[]{$D_R$};
\end{drawing}
Let $f(z) = (\log z)/(z^a(z+1))$ be the branch defined on the slit plane $\C \setminus [0, +\infty)$ by
\[
    f(z) = \frac{\log r + i\theta}{r^a e^{i\theta a}(z + 1)}, \qquad z = re^{i\theta},\; \theta \in (0, 2\pi).
\]
We extend to function to the positive real axis with $\theta = 0$ from above and $\theta = 2\pi$ from below. Then
\[
    \int_{\bd D_R} f(z) \dd{z}
        = \int_{\Gamma_R} f(z) \dd{z}
            + \int_{\gamma_\eps} f(z) \dd{z}
            + \int_{\eps}^{R} \frac{\log x}{x^a(x + 1)} \dd{x}
            + \int_{R}^{\eps} \frac{\log x + i2\pi}{x^ae^{i2\pi a}(x + 1)} \dd{x}
\]
Since $f(z)$ has one simple pole inside of $D_R$ at $-1 = e^{i\pi}$, then by the residue theorem
\begin{align*}
    \int_{\bd D_R} f(z) \dd{z}
        &= 2\pi i \cdot \Res[f(z), -1] \\
        &= 2\pi i \cdot \lim_{z \to -1} (z + 1) f(z) \\
        &= 2\pi i \cdot \frac{\log 1 + i\pi}{1^a e^{i\pi a}} \\
        &= \frac{-2\pi^2}{e^{i\pi a}}
\end{align*}
Applying the $ML$-estimate to the integral over $\Gamma_R$, we find
\[
    \left|\int_{\Gamma_R} \frac{\log r + i\theta}{r^a e^{i\theta a}(z + 1)} \dd{z}\right|
        \leq \frac{\log R + 2\pi}{R^a(R - 1)} \cdot 2\pi R
        \sim \frac{\log R}{R^a}.
\]
We confirm that this tends to zero with L'H\^opital's rule, 
\[
    \lim_{R \to \infty} \frac{\log R}{R^a}
        = \lim_{R \to \infty} \frac{1/R}{aR^{a-1}}
        = \lim_{R \to \infty} \frac{1}{aR^a}
        = 0.
\]
Applying the $ML$-estimate to the integral over $\gamma_\eps$, we find
\[
    \left|\int_{\gamma_\eps} \frac{\log r + i\theta}{r^a e^{i\theta a}(z + 1)} \dd{z}\right|
        \leq \frac{\log \eps + 2\pi}{\eps^a(1 - \eps)} \cdot 2\pi \eps
        \sim \frac{\log \eps}{\eps^{a-1}}.
\]
Again, we verify the limit
\[
    \lim_{\eps \to 0} \frac{\log \eps}{\eps^{a-1}}
        = \lim_{R \to \infty} \frac{\log \frac{1}{R}}{\pfrac{1}{R}^{a-1}}
        = \lim_{R \to \infty} \frac{-\log R}{R^{1-a}}
        = 0.
\]
For the integrals along the real line, we have
\begin{align*}
    \int_{R}^{\eps} \frac{\log x + i2\pi}{x^ae^{i2\pi a}(x + 1)} \dd{x}
        &= -e^{-i2\pi a} \int_{\eps}^{R} \frac{\log x}{x^a(x + 1)} \dd{x} - \frac{i2\pi}{e^{i2\pi a}} \int_{\eps}^{R} \frac{x^{-a}}{(x + 1)} \dd{x}.
\end{align*}
Letting $R \to \infty$ and $\eps \to 0$ in the original equation for the integral over $\bd D$, we obtain
\[
    \frac{-2\pi^2}{e^{i\pi a}}
        = (1 - e^{-i2\pi a}) \int_{0}^{\infty} \frac{\log x}{x^a(x + 1)} \dd{x} - \frac{i2\pi}{e^{i2\pi a}} \int_{0}^{\infty} \frac{x^{-a}}{(x + 1)} \dd{x}.
\]
We have previously seen that
\[
    \int_{0}^{\infty} \frac{x^{-a}}{(x + 1)} \dd{x} = \frac{\pi}{\sin(\pi a)}.
\]
Then
\begin{align*}
    \int_{0}^{\infty} \frac{\log x}{x^a(x + 1)} \dd{x}
        &= \frac{1}{1 - e^{-i2\pi a}}\left(\frac{i2\pi^2}{e^{i2\pi a}\sin(\pi a)} - \frac{2\pi^2}{e^{i\pi a}}\right) \\
        &=  \frac{i2\pi^2}{e^{i\pi a} - e^{-i\pi a}}\left(\frac{1}{e^{i\pi a}\sin(\pi a)} + i\right) \\
        &=  \frac{i2\pi^2}{i2\sin(\pi a)}\left(\frac{e^{-i\pi a}}{\sin(\pi a)} + i\right) \\
        &=  \frac{\pi^2}{\sin^2(\pi a)}\left(e^{-i\pi a} + i\sin(\pi a)\right) \\
        &=  \frac{\pi^2}{\sin^2(\pi a)} \left(\cos(\pi a) - i\sin(\pi a) + i\sin(\pi a)\right) \\
        &=  \frac{\pi^2\cos(\pi a)}{\sin^2(\pi a)}
\end{align*}





\end{document}