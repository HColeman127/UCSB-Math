\documentclass[12pt]{article}

% Packages
\usepackage[margin=1in]{geometry}
\usepackage{fancyhdr}
\usepackage{amsmath, amsthm, amssymb, physics}
\usepackage{tikz}
\usetikzlibrary{arrows, arrows.meta, decorations.markings}

\newenvironment{drawing}{\begin{center}\begin{tikzpicture}}{\end{tikzpicture}\end{center}}

% Page Style
\fancypagestyle{plain}{
    \fancyhf{}
    \renewcommand{\headrulewidth}{0pt}
    \renewcommand{\footrulewidth}{0pt}
    \fancyfoot[R]{\thepage}
}
\pagestyle{plain}

% Problem Box
\setlength{\fboxsep}{4pt}
\newsavebox{\savefullbox}
\newenvironment{fullbox}{\begin{lrbox}{\savefullbox}\begin{minipage}{\dimexpr\textwidth-2\fboxsep\relax}}{\end{minipage}\end{lrbox}\begin{center}\framebox[\textwidth]{\usebox{\savefullbox}}\end{center}}
\newenvironment{pbox}[1][]{\begin{fullbox}\ifx#1\empty\else\paragraph{#1}\fi}{\end{fullbox}}

% Options
\renewcommand{\thesubsection}{\thesection(\alph{subsection})}
\allowdisplaybreaks
\addtolength{\jot}{4pt}
\theoremstyle{definition}

% Default Commands
\newtheorem{proposition}{Proposition}
\newtheorem{lemma}{Lemma}
\newcommand{\ds}{\displaystyle}
\newcommand{\isp}[1]{\quad\text{#1}\quad}
\newcommand{\N}{\mathbb{N}}
\newcommand{\Z}{\mathbb{Z}}
\newcommand{\Q}{\mathbb{Q}}
\newcommand{\R}{\mathbb{R}}
\newcommand{\C}{\mathbb{C}}
\newcommand{\eps}{\varepsilon}
\renewcommand{\phi}{\varphi}
\renewcommand{\emptyset}{\varnothing}
\newcommand{\pfrac}[2]{\left(\frac{#1}{#2}\right)}

% Extra Commands
\newcommand{\bd}{\partial}
\def\[#1\]{\begin{align*}#1\end{align*}}
\newcommand{\conj}{\overline}
\DeclareMathOperator{\Arg}{Arg}


% Document Info
\fancypagestyle{title}{
    \renewcommand{\headrulewidth}{0.4pt}
    \setlength{\headheight}{15pt}
    \fancyhead[R]{Harry Coleman}
    \fancyhead[L]{MATH CS 122B Final}
    \fancyhead[C]{March 17, 2021}
}

% Begin Document
\begin{document}
\thispagestyle{title}


\begin{pbox}[Problem 1]
    Show that the sum of the residues of a rational function on the extended complex plane is equal to zero.
\end{pbox}

\begin{proof}
    Let $f(z)$ be a rational function. Then $f$ is meromorphic on $\C$, and let $z_1, \dots, z_m$ be its poles. For a large radius $R > \max|z_j|$, all the poles are contained in the disc $D = \{|z| < R\}$. By the residue theorem, we have
    \[
        \frac{1}{2\pi i} \int_{\bd D} f(z) \dd{z} = \sum_{j=1}^{m} \Res[f(z), z_j].
    \]
    By our choice of radius, $f$ has no poles in the exterior domain $E = \{|z| > R\}$, so
    \[
        \frac{1}{2\pi i} \int_{\bd E} f(z) \dd{z} = \Res[f(z), \infty].
    \]
    Since $D$ and $E$ have the same boundary, $\{|z| = R\}$, but with opposite orientation for integration, then
    \[
        \int_{\bd E} f(z) \dd{z} = - \int_{\bd D} f(z) \dd{z}.
    \]
    Rewriting both integrals in terms of residues, we obtain
    \[
         \Res[f(z), \infty] + \sum_{j=1}^{m} \Res[f(z), z_j] = 0.
    \]
    
\end{proof}


\newpage
\begin{pbox}[Problem 2]
    Find the number of zeros of the polynomial
    \[
        P(z) = z^4 + z^3 + 4z^2 + 3z + 2
    \]
    in each quadrant.
\end{pbox}

If $x \geq 0$, then
\[
    P(x) = x^4 + x^3 + 4x^2 + 3x + 2 \geq 2,
\]
so $P(z)$ has no positive real roots. For $-1 < -x < 0$, we find
\[
    P(-x)
        &= x^4 - x^3 + 4x^2 - 3x + 2 \\
        &= x^4 + 2 + (x^2 - x^3) + 3(x^2 - x) \\
        &\geq x^4 + 2.
\]
And for $-x \leq 1$, we find
\[
    P(-x)
        = (x^4 - x^3) + (4x^2 - 3x) + 2 
        \geq 2.
\]
Thus, $P(z)$ has no real roots. For values $iy$ on the positive imaginary axis, we have
\[
    P(iy) = y^4 - 4y^2 + 2 - i(y^3 - 3y).
\]
And in order for this to be zero, we would need $y = 0$ or $y = \pm\sqrt{3}$, and none of these are roots of the polynomial. Thus, $P(z)$ has no roots on either of the axes.

Since $P(z)$ has four complex roots, there are two pairs of conjugate roots. Let $z_1, z_2$ be the roots in the upper half-plane, with $\conj{z_1}, \conj{z_2}$ being the roots in the lower half-plane. Choose a large radius $R > \max\{|z_1|, |z_2|\}$, so any roots of $P(z)$ which may be located in the first quadrant are contained in the upper-right quarter-disc domain $D_R$.
\begin{drawing}
    %\draw[help lines, color=gray, dashed] (-0.5, -0.5) grid (4.5, 3.5);
    
    \draw (-0.5, 0)--(3.5, 0);
    \draw (0, -0.5)--(0, 3.5);
    \draw (3, 0) arc (0 : 90 : 3);
    
    \begin{scope}[-{Stealth[scale=1.5]}]
        \draw (1.5, 0) -- +(0.15, 0) node[anchor=north]{$\gamma_1$};
        \draw (0, 1.5) -- +(0, -0.15) node[anchor=east]{$\gamma_2$};
        \draw (45 : 3) arc (45 : 47.5 : 3) node[anchor=south west]{$\Gamma_R$};
    \end{scope}
    
    \draw[fill=black] (3,0) circle (2pt) node[anchor=north]{$R$};
    \draw[fill=black] (0,3) circle (2pt) node[anchor=east]{$iR$};
    \draw[] (1.25, 1.25) node{$D_R$};
\end{drawing}
Since $P(z)$ is entire, the argument principal tells us that if $N_0$ is the number of zeros of $P(z)$ in the first quadrant, then
\[
    2\pi N_0
        = \int_{\bd D_R} \dd{\arg P(z)}
        = \int_{\gamma_1} \dd{\arg P(z)} + \int_{\Gamma_R} \dd{\arg P(z)} + \int_{\gamma_2} \dd{\arg P(z)}.
\]

Since $\gamma_1$ is a subset of the real axis, and we have seen that $P(x) > 0$ for real $x$, then the argument of $P(z)$ is constantly zero along $\gamma_1$, giving us the change in argument
\[
    \int_{\gamma_1} \dd{\arg P(z)} = 0.
\]

For $z \in \Gamma_R$, we take $z = Re^{i\theta}$ with $\theta \in [0, \pi/2]$. For sufficiently large $R > 0$, the $z^4$ term of $P(z)$ dominates, so we can make the approximation $\arg P(z) \approx \arg z^4 = 4\theta$. Then the change in argument is given by
\[
    \int_{\Gamma_R} \dd{\arg P(z)}
        \approx 4(\pi/2) - 4(0)
        = 2\pi.
\]
Explicitly, the change in argument of $P(z)$ over $\Gamma_R$ tends to $2\pi$ as $R \to \infty$.


For $z \in \gamma_2$ we take $z = iy$ with $y \in [0, R]$, and
\[
    P(iy) = y^4 - 4y^2 + 2 - i(y^3 - 3y).
\]
For sufficiently large $R > 0$, the $y^4$ term dominates the real component, so $\Re P(iR) > 0$, and the $y^3$ term dominates the imaginary component, so $\Im P(iR) < 0$. So $P(iR)$ is in the fourth quadrant, and we make the approximation $\arg P(iR) \approx 2\pi$. We now analyse the path of $P(iy)$ as $y$ changes from $R$ to $0$. As previously noted, $\Im P(iy) = 0$ only for $y = 0$ and $y = \pm\sqrt{3}$.  So from $R$ to $\sqrt{3}$, the curve $P(iy)$ remains strictly in the lower half-plane. And
\[
    P(i\sqrt{3}) = 3^3 - 4(3) + 2 = -1,
\]
so the curve crosses the negative real axis at this point. Then it does not cross the real axis again until $y=0$, at which point, it hits the positive real axis at $P(i0) = 2$. Therefore, we can take the branch $\arg z \in [0, 2\pi]$, with upper and lower edges on the positive real axis, and the change in argument is given by
\[
    \int_{\gamma_2} \dd{\arg P(z)}
        = \arg P(i0) - \arg P(iR)
        \approx 0 - 2\pi.
\]

Substituting the changes in argument over the three segments of $\bd D$, we obtain
\[
    2\pi N_0 = 0 + 2\pi - 2\pi = 0.
\]
That is, $N_0 = 0$, so $P(z)$ has no roots in the first quadrant. Therefore, we conclude that there are two roots in the second quadrant, and conjugate to them are two roots in the third quadrant.







\newpage
\begin{pbox}[Problem 3]
    Show that if $m, n \in \N$, then the polynomial
    \[
        p(z) = \sum_{k=0}^{m} \frac{z^k}{k!} + 3z^n
    \]
    has exactly $n$ zeros in the unit disc.
\end{pbox}

\begin{proof}
    Let $f(z) = 3z^n$ and $h(z) = \sum_{k=0}^{m} z^k/k!$, which are both entire functions. For any point $|z| = 1$ on the unit circle,
    \[
        |h(z)|
            \leq \sum_{k=0}^{m} \frac{|z|^k}{k!} 
            \leq \sum_{k=0}^{\infty} \frac{1}{k!}
            = e
            < 3
            = |f(z)|.
    \]
    So by Rouch\'e's theorem, $f(z)$ and $p(z) = f(z) + h(z)$ have the same number of zeros in the unit disc, counting multiplicities. And since $f(z) = 3z^n$ has a exactly $n$ zeros (in particular, $0$ is its only zero and has multiplicity $n$), then in fact $p(z)$ the same number of zeros inside the unit disc.
    
\end{proof}





\newpage
\begin{pbox}[Problem 4]
    Show that
    \[
        \int_{0}^{\infty} \sin(x^2) \dd{x} = \frac{\sqrt{\pi}}{2\sqrt{2}}.
    \]
\end{pbox}

Define the function $f(z) = e^{iz^2}$, which is entire. For a large radius $R > 0$, let $D_R$ be the upper-right eighth-disc which touches the real axis.
\begin{drawing}
    %\draw[help lines, color=gray, dashed] (-0.5, -0.5) grid (4.5, 3.5);
    
    \draw (-0.5, 0)--(3.5, 0);
    \draw (0, -0.5)--(0, 2.5);
    \draw (3, 0) arc (0 : 45 : 3) -- (0, 0);
    
    \begin{scope}[-{Stealth[scale=1.5]}]
        \draw (1.5, 0) -- +(0.15, 0);
        \draw (45 : 1.5) node[anchor=south east]{$\gamma_R$} -- +(45 : -0.15);
        \draw (22.5 : 3) node[anchor=west, xshift=3]{$\Gamma_R$} arc (22.5 : 25 : 3);
    \end{scope}
    
    \draw[fill=black] (3,0) circle (2pt) node[anchor=north]{$R$};
    \draw[fill=black] (45 : 3) circle (2pt) node[anchor=south west]{$Re^{i\pi/4}$};
    \draw (22.5 : 2) node{$D_R$};
\end{drawing}
Cauchy's theorem yields
\[
    \int_{\bd D_R} f(z) \dd{z}
        = \int_{0}^{R} e^{ix^2} \dd{x} + \int_{\Gamma_R} f(z) \dd{z} + \int_{\gamma_R} f(z) \dd{z}
        = 0.
\]
We parameterize the integral over $\Gamma_R$ by $z = Re^{i\theta}$ for $\theta \in [0, \pi/4]$. Then $\dd{z} = iRe^{i\theta} \dd{\theta}$ and we find
\[
    \int_{\Gamma_R} f(z) \dd{z}
        = \int_{0}^{\pi/4} e^{i(Re^{i\theta})^2}iRe^{i\theta} \dd{\theta}
        = iR\int_{0}^{\pi/4} e^{iR^2e^{2i\theta}}e^{i\theta} \dd{\theta}.
\]
The modulus of the integrand is given by
\[
    \abs{e^{iR^2e^{2i\theta}}e^{i\theta}}
        = \abs{e^{iR^2(\cos 2\theta + i\sin 2\theta)}}
        = e^{-R^2 \sin 2\theta}.
\]
A change of variables gives us the bound
\[
    \abs{\int_{\Gamma_R} f(z) \dd{z}}
        \leq R \int_{0}^{\pi/4} e^{-R^2 \sin 2\theta} \dd{\theta}
        = \frac{R}{2} \int_{0}^{\pi/2} e^{-R^2 \sin \theta} \dd{\theta}
        < R \int_{0}^{\pi} e^{-R^2 \sin \theta} \dd{\theta},
\]
so by Jordan's lemma,
\[
    \abs{\int_{\Gamma_R} f(z) \dd{z}}
        < R \cdot \frac{\pi}{R^2}
        = \frac{\pi}{R}
        \xrightarrow{R \to \infty} 0.
\]
We parameterize the integral over $\gamma_R$ by $z = te^{i\pi/4}$ for $t \in [0, R]$. Then $\dd{z} = e^{i\pi/4} \dd{t}$ and we find
\[
    \int_{\gamma_R} f(z) \dd{z}
        = - \int_{0}^{R} e^{i(te^{i\pi/4})^2} e^{i\pi/4} \dd{t}
        = - e^{i\pi/4} \int_{0}^{R} e^{-t^2} \dd{t}.
\]
Taking the limit as $R \to \infty$, this becomes
\[
    - e^{i\pi/4} \int_{0}^{\infty} e^{-t^2} \dd{t}
        = -e^{i\pi/4} \frac{\sqrt{\pi}}{2}
        = -(1 + i)\frac{\sqrt{\pi}}{2\sqrt{2}}
\]
Letting $R \to \infty$ in our original equation, we find
\[
    \int_{0}^{\infty} e^{ix^2} \dd{x} = (1 + i)\frac{\sqrt{\pi}}{2\sqrt{2}},
\]
and taking imaginary parts gives us
\[
    \int_{0}^{\infty} \sin x^2 \dd{x} = \frac{\sqrt{\pi}}{2\sqrt{2}}.
\]







\newpage
\begin{pbox}[Problem 5]
    For $a \in \R$ evaluate the integral
    \[
        \int_{0}^{\infty} \frac{x \sin(ax)}{x^4 + 1} \dd{x}.
    \]
\end{pbox}

We will first assume that $a \geq 0$. Define the function
\[
    f(z)
        = \frac{ze^{iaz}}{z^4 + 1},
\]
which is meromorphic on $\C$ with simple poles at $(\pm1 \pm i)/\sqrt{2}$. The poles in the upper half-plane are $z_1 = (1 + i)/\sqrt{2}$ and $z_2 = (-1 + i)/\sqrt{2}$. For a large radius $R > 1$, let $D_R$ be the upper half-disc domain.
\begin{drawing}
    %\draw[help lines, color=gray, dashed] (-0.5, -0.5) grid (4.5, 3.5);
    
    \draw (-2.5, 0)--(2.5, 0);
    \draw (0, -0.5)--(0, 2.5);
    \draw (2, 0) arc (0 : 180 : 2);
    
    \begin{scope}[-{Stealth[scale=1.5]}]
        \draw (-1, 0) -- +(0.15, 0);
        \draw (1, 0) -- +(0.15, 0);
        \draw (45 : 2) arc (45 : 47.5 : 2) node[anchor=south west]{$\Gamma_R$};
        \draw (135 : 2) arc (135 : 137.5 : 2);
    \end{scope}
    
    \draw[fill=black] (0.6, 0.6) circle (2pt) node[anchor=north]{$z_1$};
    \draw[fill=black] (-0.6, 0.6) circle (2pt) node[anchor=north]{$z_2$};
    
    \draw[fill=black] (2, 0) circle (2pt) node[anchor=north]{$R$};
    \draw[fill=black] (-2, 0) circle (2pt) node[anchor=north, xshift=-5]{$-R$};
    \draw[] (0.75, 1.25) node{$D_R$};
\end{drawing}
The residue theorem yields
\[
    \int_{\bd D_R} f(z) \dd{z}
        = \int_{-R}^{R} \frac{xe^{iax}}{x^4 + 1} \dd{x} + \int_{\Gamma_R} f(z) \dd{z}
        = 2\pi i \sum_{j=1,2}\Res[f(z), z_j].
\]
For $z \in \Gamma_R$, we have $z = Re^{i\theta}$ with $\theta \in [0, \pi]$. Note that 
\[
    iaz
        = iaR(\cos\theta + i\sin\theta)
        = -aR\sin\theta + iaR\cos\theta.
\]
Since $a \geq 0$, $R > 0$, and $\sin\theta \geq 0$, then we have the bound
\[
    |f(z)|
        = \abs{\frac{ze^{iaz}}{z^4 + 1}}
        = \frac{R e^{-aR\sin\theta}}{|z^4 + 1|}
        \leq \frac{R}{R^4 - 1}.
\]
Then the $ML$-estimate gives us
\[
    \abs{\int_{\Gamma_R} f(z) \dd{z}}
        \leq \frac{R}{R^4 - 1} \cdot \pi R
        \sim \frac{1}{R^2}
        \xrightarrow{R \to \infty} 0.
\]
The residue at $z_1$ is
\[
    \Res[f(z), z_1]
        = \eval[\frac{ze^{iaz}}{4z^3}|_{z = z_1}
        = \eval[\frac{e^{iaz}}{4z^2}|_{z = z_1} 
        = \frac{e^{ia(1 + i)/\sqrt{2}}}{4i}
        = \frac{-i}{4} e^{-a/\sqrt{2}}e^{ia/\sqrt{2}}.
\]
The residue at $z_2$ is
\[
    \Res[f(z), z_2]
        = \eval[\frac{ze^{iaz}}{4z^3}|_{z = z_2}
        = \eval[\frac{e^{iaz}}{4z^2}|_{z = z_2}
        = \frac{e^{ia(-1 + i)/\sqrt{2}}}{-4i}
        = \frac{i}{4} e^{-a/\sqrt{2}}e^{-ia/\sqrt{2}}.
\]
Letting $R \to \infty$ in our original equation, we obtain
\[
    \int_{-\infty}^{\infty} \frac{xe^{iax}}{x^4 + 1} \dd{x}
        &= 2\pi i\qty(\frac{-i}{4} e^{-a/\sqrt{2}}e^{ia/\sqrt{2}} + \frac{i}{4} e^{-a/\sqrt{2}}e^{-ia/\sqrt{2}}) \\
        &= \frac{\pi}{2} e^{-a/\sqrt{2}} \qty(e^{ia/\sqrt{2}} - e^{-ia/\sqrt{2}}) \\
        &= \frac{\pi}{2} e^{-a/\sqrt{2}} \qty(2i\sin \frac{a}{\sqrt{2}}) \\
        &= i\pi e^{-a/\sqrt{2}}\sin \frac{a}{\sqrt{2}},
\]
and taking imaginary parts gives us
\[
    \int_{-\infty}^{\infty} \frac{x \sin(ax)}{x^4 + 1} \dd{x} = \pi e^{-a/\sqrt{2}}\sin \frac{a}{\sqrt{2}}.
\]
We can see that the integrand is even, since for all $x \in (-\infty, \infty)$
\[
    \frac{(-x) \sin(-ax)}{(-x)^4 + 1}
        = \frac{x \sin(ax)}{x^4 + 1}.
\]
Therefore, for $a \geq 0$, we have found
\[
    \int_{0}^{\infty} \frac{x \sin(ax)}{x^4 + 1} \dd{x}
        = \frac{1}{2} \int_{-\infty}^{\infty} \frac{x \sin(ax)}{x^4 + 1} \dd{x}
        = \frac{\pi}{2} e^{-a/\sqrt{2}} \sin \frac{a}{\sqrt{2}}.
\]
On the other hand, if $a < 0$, then $-a > 0$ and we find
\[
    \int_{0}^{\infty} \frac{x \sin(ax)}{x^4 + 1} \dd{x}
        &= -\int_{0}^{\infty} \frac{x \sin(-ax)}{x^4 + 1} \dd{x} \\
        &= -\frac{\pi}{2} e^{-(-a)/\sqrt{2}} \sin \frac{-a}{\sqrt{2}} \\
        &= \frac{\pi}{2} e^{a/\sqrt{2}} \sin \frac{a}{\sqrt{2}}.
\]
And we conclude that, for all $a \in \R$,
\[
    \int_{0}^{\infty} \frac{x \sin(ax)}{x^4 + 1} \dd{x} = \frac{\pi}{2} e^{-|a|/\sqrt{2}} \sin \frac{a}{\sqrt{2}}.
\]


\end{document}