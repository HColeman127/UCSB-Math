\documentclass[12pt]{article}

% Packages
\usepackage[margin=1in]{geometry} % proper margins
\usepackage{enumitem} % custom numbering for lists
\usepackage{amsmath} % align, cases, eqref, matrices, dots, roots, delimiters, math mode functions, mod, arrows
\usepackage{amsthm} % theorems, proofs
\usepackage{amssymb} % fancy letters, niche relations/negations
\usepackage{mathrsfs} % much more loopy calligraphy font

% Theorems
\newtheorem{theorem}{Theorem}
\newtheorem{lemma}{Lemma}
\newtheorem{proposition}{Proposition}

% Problem Box
\setlength{\fboxsep}{4pt}
\newsavebox{\mybox}
\newenvironment{problem}
    {\begin{lrbox}{\mybox}\begin{minipage}{0.98\textwidth}}
    {\end{minipage}\end{lrbox}\framebox[\textwidth]{\usebox{\mybox}}}

% Formatting
\newcommand{\ds}{\displaystyle}
\newcommand{\isp}[1]{\quad\text{#1}\quad}
\newcommand*{\defeq}{\mathrel{\vcenter{\baselineskip0.5ex \lineskiplimit0pt\hbox{\scriptsize.}\hbox{\scriptsize.}}}=}

% Alternate Characters
\let\eps\varepsilon % double curved, rather than single curve with midline
\let\phi\varphi % single stroke loop, rather than vertical line through circle
\let\emptyset\varnothing % circular, rather than tall

% Named Sets
\newcommand{\N}{\mathbb{N}} % natural numbers
\newcommand{\Z}{\mathbb{Z}} % integers 
\newcommand{\Q}{\mathbb{Q}} % rational numbers
\newcommand{\R}{\mathbb{R}} % real numbers
\newcommand{\C}{\mathbb{C}} % complex numbers

% Fancy Characters
\newcommand{\F}{\mathbb{F}} % arbitrary field
\renewcommand{\P}{\mathbb{P}} % probability measure (apparently, sometimes prime numbers)
\newcommand{\FF}{\mathcal{F}} % sigma algebra
\newcommand{\BB}{\mathcal{B}} % Borel sigma algebra

% Paired Delimiters
\newcommand{\ceil}[1]{\left\lceil #1 \right\rceil} % ceiling
\newcommand{\floor}[1]{\left\lfloor #1 \right\rfloor} % floor
\newcommand{\<}{\left\langle} % left angle bracket
\renewcommand{\>}{\right\rangle} % right angle bracket

% Functions
\renewcommand{\Im}{\operatorname{Im}} % imaginary part of a complex number
\renewcommand{\Re}{\operatorname{Re}} % real part of a complex number
\newcommand{\Arg}{\operatorname{Arg}} % principal argument (angle) of complex number

% Simplified Notation
\newcommand{\id}[1]{\operatorname{id_{\mathnormal{#1}}}} % identity operator
\newcommand{\seq}[2][n]{\left\{#2\right\}_{#1\in\N}} % sequence
\renewcommand{\d}[1]{\operatorname{d}\!#1} % differential operator
\newcommand{\od}[3][1]{\ifnum#1=1{\frac{\d #2}{\d #3}}\else{\frac{\d^{#1}#2}{\d #3^{#1}}}\fi} % ordinary derivative
\newcommand{\pd}[3][1]{\ifnum#1=1{\frac{\partial #2}{\partial#3}}\else{\frac{\partial^{#1}#2}{\partial#3^{#1}}}\fi} % partial derivative
\newcommand{\intr}[1]{\accentset{\circ}{#1}} % interior of a set

% Renaming
\let\sm\setminus % set minus (difference)
\let\clo\overline % closure of a set
\let\conj\overline % conjugate of an object
\let\eqc\overline % equivalence class of an object
\let\teq\trianglelefteq % normal subgroup
\let\iso\cong % isomorphic (groups)

% Notes
% medskip for header-less paragraph
% intertext{} for short text inside big display structure
% dots is dynamic based on surroundings
% dfrac and tfrac to force large or small fractions
% operatorname for new operators instead of text
% consider using nath; it seems to break most formatting and is not compatible with many packages

\begin{document}
 
\title{Homework 8\\
    %\large MATH CS 121 Intro to Probability
    \large MATH 111A Intro to Abstract Algebra
    %\large MATH CS 122A Complex Analysis I
    %\large MATH 118A Intro to Real Analysis
    %\large MATH 104A Intro to Numerical Analysis
}
\author{Harry Coleman}
\date{December 2, 2020}
\maketitle

\section{Exercise 4.3.13}
\begin{problem}
    Find all finite groups which have exactly two conjugacy classes.
\end{problem}

\begin{proposition}
    A finite group with exactly two conjugacy classes is isomorphic to the cyclic group $Z_2$.
\end{proposition}

\begin{proof}
    Suppose $G$ is a finite group with exactly two conjugacy classes. We consider $G$ to act on itself by conjugation, so the conjugacy classes are the orbits of this action. Every group, in particular $G$, has the singleton conjugacy class of the identity, $\{1\}$. Then for some $x \in G \setminus \{1\}$, the orbit of $x$ is the other conjugacy class in $G$. In other words, $G$ is partitioned as $G = \{1\} \sqcup (G \cdot x)$, so
    \[|G| = |\{1\}| + |G \cdot x| = 1 + |G \cdot x|.\]
    Since $G$ acts on itself by conjugation, we have
    \[|G \cdot x| = [G : G_x] = \frac{|G|}{|G_x|}.\]
    Letting $n = |G| \in \N$, we find that
    \begin{align*}
        n &= 1 + \frac{n}{|G_x|}, \\
        |G_x| &= \frac{n}{n-1}.
    \end{align*}
    Now since $|G_x| \in \N$, and $n = 2$ is the only natural number for which $n/(n-1)$ is also a natural number, then we must have $|G| = 2$. Denote by $x$ the element of $G$ which is not the identity, so $G = \{1, x\}$. Since $G$ is a group, we must have $x^{-1} \in G$. And since $1x = x \ne 1$, then $x$ must be its own inverse. That is, $x^2 = 1$, which implies that $G$ is generated by $x$. Since $G$ is a cyclic group of order 2, it is isomorphic to $Z_2$.
    
\end{proof}

\newpage
\section{Exercise 4.3.29}
\begin{problem}
    Let $p$ be a prime and let $G$ be a group of order $p^\alpha$. Prove that $G$ has a subgroup of order $p^\beta$, for every $\beta$ with $0 \leq \beta \leq \alpha$. [Use Theorem 8 and induction on $\alpha$.]
\end{problem}

\begin{proof}
    We prove the property by induction on $\alpha$. Consider $\alpha = 0$, then $\beta = 0$ and we have the subgroup $\{1\} \leq G$ with order $p^\beta$; this proves the base case. Suppose that the property holds for all powers of $p$ up to $\alpha - 1$, for some $\alpha \geq 1$. Additionally, we assume that $\beta > 0$, since we always have $\{1\} \leq G$. Now since $p$ is prime and $|G| = p^\alpha$ with $\alpha \geq 1$, then $Z(G)$ is nontrivial. Because $Z(G) \leq G$, then $|Z(G)|$ divides $|G|$. Since $|Z(G)| \ne 1$ and the factors of $|G|$ are powers of $p$, then $|Z(G)|$ is some nonzero power of $p$. This tells us that $p$ divides $|Z(G)|$, so by Cauchy's theorem, there exists a subgroup $N \leq Z(G)$ of order $p$. Since $N \leq Z(G) \teq G$, then we know that $N \teq G$ and we consider the quotient group $G/N$, which has order
    \[|G/N| = \frac{|G|}{|N|} = \frac{p^\alpha}{p} = p^{\alpha - 1}.\]
    Since $0 < \beta \leq \alpha$, then $0 \leq \beta - 1 \leq \alpha - 1$. So by the inductive hypothesis, there exists a subgroup $\conj{P} \leq G/N$ of order $p^{\beta - 1}$. Then by the fourth isomorphism theorem, there exists some subgroup $P \leq G$ such that $P/N = \conj{P}$. Moreover,
    \[p^{\beta - 1} = |\conj{P}| = |P/N| = \frac{|P|}{|N|} = \frac{|N|}{p},\]
    so $|P| = p^{\beta}$ and the induction is complete.
    
\end{proof}


\end{document}