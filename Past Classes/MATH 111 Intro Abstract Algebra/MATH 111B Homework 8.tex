\documentclass[12pt]{article}

% Packages
\usepackage[margin=1in]{geometry}
\usepackage{fancyhdr}
\usepackage{amsmath, amsthm, amssymb, physics}

% Page Style
\fancypagestyle{plain}{
    \fancyhf{}
    \renewcommand{\headrulewidth}{0pt}
    \renewcommand{\footrulewidth}{0pt}
    \fancyfoot[R]{\thepage}
}
\pagestyle{plain}

% Problem Box
\setlength{\fboxsep}{4pt}
\newsavebox{\savefullbox}
\newenvironment{fullbox}{\begin{lrbox}{\savefullbox}\begin{minipage}{\dimexpr\textwidth-2\fboxsep\relax}}{\end{minipage}\end{lrbox}\begin{center}\framebox[\textwidth]{\usebox{\savefullbox}}\end{center}}
\newenvironment{pbox}[1][]{\begin{fullbox}\ifx#1\empty\else\paragraph{#1}\fi}{\end{fullbox}}

% Options
\renewcommand{\thesubsection}{\thesection(\alph{subsection})}
\allowdisplaybreaks
\addtolength{\jot}{4pt}
\theoremstyle{definition}

% Default Commands
\newtheorem{proposition}{Proposition}
\newtheorem{lemma}{Lemma}
\newcommand{\ds}{\displaystyle}
\newcommand{\isp}[1]{\quad\text{#1}\quad}
\newcommand{\N}{\mathbb{N}}
\newcommand{\Z}{\mathbb{Z}}
\newcommand{\Q}{\mathbb{Q}}
\newcommand{\R}{\mathbb{R}}
\newcommand{\C}{\mathbb{C}}
\newcommand{\eps}{\varepsilon}
\renewcommand{\phi}{\varphi}
\renewcommand{\emptyset}{\varnothing}
\newcommand{\pfrac}[2]{\left(\frac{#1}{#2}\right)}

% Extra Commands
\newcommand{\isom}{\cong}
\newcommand{\eqc}{\overline}
\newcommand{\Hom}{\operatorname{Hom}}
\newcommand{\End}{\operatorname{End}}
\newcommand{\Tor}{\operatorname{Tor}}
\newcommand{\Ann}{\operatorname{Ann}}
\newcommand{\tsum}{\textstyle\sum\limits}

% Document Info
\fancypagestyle{title}{
    \renewcommand{\headrulewidth}{0.4pt}
    \setlength{\headheight}{15pt}
    \fancyhead[R]{Harry Coleman}
    \fancyhead[L]{MATH 111B Homework 8}
    \fancyhead[C]{March 5, 2021}
}

% Begin Document
\begin{document}
\thispagestyle{title}


\begin{pbox}[Q1]
    Let $F$ be a field. Let $V = F^n$ viewed as an $F[x]$-module with respect to the linear transformation $T : V \to V$, $T = \ds\mqty(x_1 \\ \vdots \\ x_{n-1} \\ x_n) = \mqty(x_2 \\ \vdots \\ x_n \\ 0)$. View the quotient ring $F[x]/(x^n)$ as an $F[x]$-module by
    \[
        F[x] \times \qty\big(F[x]/(x^n)) \to F[x]/(x^n), \qquad \qty\big(a(x), \eqc{p(x)}) \to \eqc{a(x)p(x)}.
    \]
    Show that $V \isom F[x]/(x^n)$ as $F[x]$-modules.
\end{pbox}

\begin{proof}
    Note that the equivalence classes of $F[x]/(x^n)$ are the polynomials of $F[x]$ with degree at most $n-1$. We define the map
    \begin{align*}
        \phi : F[x]/(x^n) &\to V \\
            \tsum_{k=0}^{n-1} a_k x^k &\mapsto \tsum_{k=0}^{n-1} a_k e_{n-k},
    \end{align*}
    where $\{e_1, \dots, e_n\}$ is the standard basis for $V = F^n$. We will show that $\phi$ is an $F[x]$-module isomorphism. First, for any $p(x), q(x) \in F[x]/(x^n)$, with coefficients $a_0, \dots, a_{n-1} \in F$ and $b_0, \dots, b_{n-1} \in F$, we find
    \begin{align*}
        \phi\qty(p(x) + q(x))
            &= \phi\qty(\tsum_{k=0}^{n-1} a_k x^k + \tsum_{k=0}^{n-1} b_k x^k) \\
            &= \phi\qty(\tsum_{k=0}^{n-1} (a_k + b_k) x^k) \\
            &= \sum_{k=0}^{n-1} (a_k + b_k) e_{n-k} \\
            &= \sum_{k=0}^{n-1} a_k e_{n-k} + \sum_{k=0}^{n-1} b_k e_{n-k} \\
            &= \phi\qty(\sum_{k=0}^{n-1} a_k x^k) + \phi\qty(\sum_{k=0}^{n-1} b_k x^k) \\
            &= \phi(p(x)) + \phi(q(x)).
    \end{align*}
    To show that $\phi$ respects the actions of all elements in $F[x]$, we first consider $r \in F$. For elements in both $F[x]$-modules, $r$ acts simply by multiplication. Then
    \begin{align*}
        \phi(rp(x))
            &= \phi\qty(\tsum_{k=0}^{n-1} ra_k x^k) \\
            &= \sum_{k=0}^{n-1} ra_k e_{n-k} \\
            &= r\sum_{k=0}^{n-1} a_k e_{n-k} \\
            &= r \phi(p(x)). 
    \end{align*}
    Next, we consider $x \in F[x]$. For $p(x) = a_0 + a_1 x + \cdots a_{n-1}x^{n-1} \in F[x]/(x^n)$, we have
    \begin{align*}
        xp(x)
            &= \eqc{xp(x)} \\
            &= \eqc{\tsum_{k=0}^{n-1} a_k x^{k+1}} \\
            &= \sum_{k=1}^{n-1} a_{k-1} x^k
    \end{align*}
    For $v = \mqty[v_1 & \cdots & v_n]^T \in V$, we have
    \begin{align*}
        xv
            &= T\qty(\tsum_{k=1}^{n} v_k e_k) \\
            &= \sum_{k=1}^{n} v_kT(e_k) \\
            &= \sum_{k=2}^{n} v_k e_{k-1}.
    \end{align*}
    Now,
    \begin{align*}
        \phi(xp(x))
            &= \phi\qty(\tsum_{k=1}^{n-1} a_{k-1} x^k) \\
            &= \sum_{k=1}^{n-1} a_{k-1} e_{n-k} \\
            &= \sum_{k=1}^{n-1} a_{k-1} T(e_{n-k+1}) \\
            &= \sum_{k=0}^{n-1} a_k T(e_{n-k}) \\
            &= T\qty(\sum_{k=0}^{n-1} a_k e_{n-k}) \\
            &= x\phi(p(x)).
    \end{align*}
    Then for any $r(x) = r_0 + r_1x + \cdots r_{n-1}x^{n-1} + \cdot \in F[x]$, we find
    \begin{align*}
        \phi(r(x)p(x))
            &= \phi\qty(\sum_{k=0}^{n-1} r_k x^k p(x)) \\
            &= \sum_{k=0}^{n-1} r_k x^k \phi(p(x)) \\
            &= r(x) \phi(p(x)).
    \end{align*}
    Thus, $\phi$ is an $F[x]$-module homomorphism. Then for any $p(x) \in \ker\phi$, we have
    \begin{align*}
        0
            &= \phi(p(x)) \\
            &= \sum_{k=0}^{n-1} a_k e_{n-k},
    \end{align*}
    which implies $a_0 = \cdots = a_n = 0$, so $p(x) = 0$. Therefore, $\ker\phi = \{0\}$, implying $\phi$ is injective. For any $v = \mqty[v_1 & \cdots & v_n]^T \in V$, we can choose $p(x) \in F[x]/(x^n)$ by
    \[
        p(x) = \sum_{k=0}^{n-1} v_{k+1} x^k.
    \]
    Then $\phi(p(x)) = v$, so $\phi$ is surjective. Hence, $\phi$ is an $F[x]$-module isomorphism, so $F[x]/(x^n) \isom V$.
    
\end{proof}



\newpage
\begin{pbox}[Q2 Problem 10.2.4]
    Let $A$ be any $\Z$-module, let $a$ be any element of $A$ and let $n$ be a positive integer. Prove that the map $\phi_a : \Z/n\Z \to A$ given by $\phi_a(\eqc{k}) = ka$ is a well-defined $\Z$-module homomorphism if and only if $na = 0$. Prove that $\Hom_\Z(\Z/n\Z, A) \isom A_n$, where $A_n = \{a \in A \mid na = 0\}$ (so $A_n$ is the annihilator in $A$ of the ideal $(n)$ of $\Z$).
\end{pbox}

\begin{proof}
    Suppose $\phi_a$ is a well-defined $\Z$-module homomorphism. For $k \in \Z$, we have $\eqc{k} = \eqc{k + n}$ in $\Z/n\Z$. Then
    \begin{align*}
        \phi_a(\eqc{k}) &= \phi_a(\eqc{k + n}) \\
        ka &= (k + n)a \\
        ka &= ka + na \\
        0 &= na.
    \end{align*}
    
    Now suppose $na = 0$. Then for $k, \ell \in \Z$ with $\eqc{k} = \eqc{\ell}$ in $\Z/n\Z$, then $\ell = k + mn$ for some $m \in \Z$. So
    \[
        \phi_a(\eqc{\ell})
            = (k + mn)a \\
            = ka \\
            = \phi_a(\eqc{k}),
    \]
    implying $\phi_a$ is well-defined. Suppose $\eqc{k}, \eqc{\ell} \in \Z/n\Z$ and $r \in \Z$, then
    \begin{align*}
        \phi_a(r\eqc{k} + \eqc{\ell})
            &= \phi_a(\eqc{rk + \ell}) \\
            &= (rk + \ell)a \\
            &= rka + \ell a \\
            &= r\phi_a(\eqc{k}) + \phi_a(\eqc{\ell}).
    \end{align*}
    Hence, $\phi_a$ is a $Z$-module homomorphism. Define the map
    \begin{align*}
        F : A_n &\to \Hom_\Z(\Z/n\Z, A) \\
            a &\mapsto \phi_a,
    \end{align*}
    where $\phi_a(\eqc{k}) = ka$. Since $na = 0$ for all $a \in A_n$, then $F$ is well-defined, we will show that is is a $\Z$-module isomorphism. For any $r \in \Z$ and $a, b \in A_n$, we have $F(ra + b) = \phi_{ra + b}$ and $rF(a) + F(b) = r\phi_a + \phi_b$. Then for any $\eqc{k} \in \Z/n\Z$, we find
    \begin{align*}
        \phi_{ra+b}(\eqc{k})
            &= k(ra + b) \\
            &= rka + kb \\
            &= r\phi_a(\eqc{k}) + \phi_b(\eqc{k}) \\
            &= (r\phi_a + \phi_b)(\eqc{k}).
    \end{align*}
    Hence, $F(ra + b) = rF(a) + F(b)$, so $F$ is a $\Z$-module homomorphism. Given $\phi \in \Hom_\Z(\Z/n\Z, A)$, define $a = \phi(\eqc{1})$. First, we see that
    \[
        na
            = n\phi(\eqc{1})
            = \phi(n\eqc{1})
            = \phi(\eqc{n})
            = \phi(\eqc{0})
            = 0.
    \]
    Therefore, $a \in A_n$ and $F(a) = \phi_a \in \Hom_\Z(\Z/n\Z, A)$. Moreover, for any $\eqc{k} \in \Z/n\Z$, we find
    \[
        \phi_a(\eqc{k})
            = ka
            = k\phi(\eqc{1}) \\
            = \phi(k\eqc{1}) \\
            = \phi(\eqc{k}).
    \]
    So $F(a) = \phi$, implying $F$ is surjective. Suppose $a \in \ker F$, then $F(a) = \phi_a$ is the zero $\Z$-module homomorphism. So $\phi_a(\eqc{k}) = 0$ for all $\eqc{k} \in \Z/n\Z$. In particular, $0 = \phi_a(\eqc{1}) = 1a = a$, so $\ker F = \{0\}$, implying $F$ is injective. Thus, $F$ is a $\Z$-module isomorphism.
    
\end{proof}


\newpage
\begin{pbox}[Q3]
    View $R[x]$ as an $R$-module with the action of $R$ given by
    \[
        R \times R[x] \to R[x], \qquad (r, p(x)) \mapsto rp(x).
    \]
    Show that $R[x]$ is not a finitely generated $R$-module.
\end{pbox}

\begin{proof}
    Suppose, for contradiction, that for some finite subset of polynomials
    \[
        A = (p_1(x), \dots, p_m(x)) \subset R[x],
    \]
    we have
    \[
        R[x] = RA = \{r_1p(x) + \cdots r_mp_m(x) \mid r_1, \dots, r_m \in R\}.
    \]
    However, if $n = \max\{\deg p_1(x), \dots, \deg p_m(x)\}$, then for any $q(x) \in RA$, we must have $\deg q(x) \leq n$. But since $x^{n+1} \in R[x]$ and $\deg x^{n+1} = n+1 > n$, this is a contradiction.
    
    
\end{proof}


\begin{pbox}[Q4 Problem 10.3.7]
    Let $N$ be a submodule of $M$, Prove that if both $M/N$ and $N$ are finitely generated then so is $M$.
\end{pbox}

\begin{proof}
    Suppose $N = RA$ for some finite subset $A = \{a_1, \dots, a_n\} \subset R$ and $M/N = RB$ for some finite subset $B = \{\eqc{b_1}, \dots, \eqc{b_m}\} \subset M/N$, where $\eqc{b_j} = b_j + N$ for $j = 1, \dots m$. If $C = A \cup \{b_1, \dots, b_m\}$, then we claim that $M = RC$. For any $x \in M$, we have $\eqc{x} \in M/n = RB$, so
    \[
        \eqc{x}
            = r_1\eqc{b_1} + \cdots r_m\eqc{b_m}
            = \eqc{r_1b_1 + \cdots + r_mb_m}
    \]
    for some $r_1, \dots, r_m \in R$. Therefore, we have
    \[
        x - (r_1b_1 + \cdots + r_mb_m) \in N,
    \]
    implying that
    \[
        x - (r_1b_1 + \cdots + r_mb_m) = s_1a_1 + \cdots + s_na_n
    \]
    for some $s_1, \dots, s_n \in R$. Thus,
    \[
        x = r_1b_1 + \cdots r_mb_m + s_1a_1 + \cdots + s_na_n \in RC.
    \]
    Hence, $M$ is finitely generated, in particular $M = RC$.
    
\end{proof}



\newpage
\begin{pbox}[Q5 Problem 10.3.14]
    Let $R$ be a commutative ring and let $F$ be the free $R$-module of rank $n$. Prove that $\Hom_R(F, M) \isom M \times \cdots \times M$ ($n$ times).
\end{pbox}

\begin{proof}
    Since $F$ is the free $R$-module of rank $n$, then it has a basis $A = \{a_1, \dots, a_n\} \subset F$.
    Define the map
    \begin{align*}
        \Phi : \Hom_R(F, M) &\to M \times \cdots \times M \\
            \phi &\mapsto (\phi(a_1), \dots, \phi(a_n)),
    \end{align*}
    which we claim to the an $R$-module isomorphism. Let $r \in R$ and $\phi, \psi \in \Hom_R(F, M)$, then
    \begin{align*}
        \Phi(r\phi + \psi)
            &= ((r\phi + \psi)(a_1), \dots, (r\phi + \psi)(a_n)) \\
            &= (r\phi(a_1) + \psi(a_1), \dots, r\phi(a_n) + \psi(a_n)) \\
            &= r(\phi(a_1), \dots, \phi(a_n)) + (\psi(a_1), \dots, \psi(a_n)) \\
            &= r\Phi(\phi) + \Phi(\psi).
    \end{align*}
    So $\Phi$ is an $R$-module homomorphism. Then for any $(x_1, \dots, x_n) \in M \times \cdots \times M$, since $F$ is the free module generated by $A$, then there is a unique $\phi \in \Hom_R(F, M)$ such that $\phi(a_j) = x_j$ for $j = 1, \dots, n$. In which case, $\Phi(\phi) = (x_1, \dots, x_n)$, so $\Phi$ is surjective. Now suppose $\phi \in \ker\Phi$, so
    \[
        (\phi(a_1), \dots, \phi(a_n)) = (0, \dots, 0).
    \]
    Then for any $x \in F$, there is some $r_1, \dots, r_n \in R$ such that
    \[
        x = r_1a_1 + \cdots + r_na_n.
    \]
    Then
    \[
        \phi(x)
            = r_1\phi(a_1) + \cdots + r_n\phi(a_n)
            = 0,
    \]
    which implies that $\phi$ is the zero homomorphism. Hence, $\Phi$ is injective and, therefore, an $R$-module isomorphism.
    
\end{proof}


\end{document}