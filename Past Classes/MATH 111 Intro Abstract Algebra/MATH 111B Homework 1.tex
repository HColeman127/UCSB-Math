\documentclass[12pt]{article}

% Packages
\usepackage[margin=1in]{geometry}
\usepackage{amsmath, amsthm, amssymb, physics}

% Problem Box
\setlength{\fboxsep}{4pt}
\newsavebox{\mybox}
\newenvironment{problem}
    {\begin{lrbox}{\mybox}\begin{minipage}{0.98\textwidth}}
    {\end{minipage}\end{lrbox}\begin{center}\framebox[\textwidth]{\usebox{\mybox}}\end{center}}

% Options
%\renewcommand{\thesubsection}{\thesection(\alph{subsection})}
\allowdisplaybreaks
\addtolength{\jot}{1em}
\theoremstyle{definition}

% Default Commands
\newtheorem{proposition}{Proposition}
\newtheorem{lemma}{Lemma}
\newcommand{\ds}{\displaystyle}
\newcommand{\isp}[1]{\quad\text{#1}\quad}
\newcommand{\N}{\mathbb{N}}
\newcommand{\Z}{\mathbb{Z}}
\newcommand{\Q}{\mathbb{Q}}
\newcommand{\R}{\mathbb{R}}
\newcommand{\C}{\mathbb{C}}
\newcommand{\eps}{\varepsilon}
\renewcommand{\phi}{\varphi}
\renewcommand{\emptyset}{\varnothing}

% Extra Commands



% Document Info
\title{Homework 1\\
    \large MATH 111B
}
\author{Harry Coleman}
\date{January 15, 2021}

% Begin Document
\begin{document}
\maketitle

\section{Problem 7.1.6}
\begin{problem}
    Decide which of the following are subrings of the ring of all functions from the closed interval $[0, 1]$ to $\R$:
\end{problem}

Let $R$ denote the ring of all functions from $[0,1]$ to $\R$.

\subsection{Problem 7.1.6(a)}
\begin{problem}
    the set of all functions $f(x)$ such that $f(q) = 0$ for all $q \in \Q \cap [0, 1]$
\end{problem}

\begin{proposition}
    The subset $S = \{f \in R : f(q) = 0 \text{ for all } q \in \Q \cap [0, 1]\}$ is a subring of $R$.
\end{proposition}

\begin{proof}
    First, $S$ contains the zero function which maps all values of $[0, 1]$ to $0 \in \R$, so it is nonempty. We now let $f, g \in S$ and consider the functions $f - g$ and $fg$. For any $q \in \Q \cap [0, 1]$, we have
    \[
        (f - g)(q) = f(q) - g(q) = 0 - 0 = 0,
    \]
    and
    \[
        (fg)(q) = f(q)g(q) = 0 \cdot 0 = 0.
    \]
    Thus, we have $f - g$ and $fg$ in $S$, so the subring criterion tells us that $S$ is a subring of $R$.
    
\end{proof}

\newpage
\subsection{Problem 7.1.6(d)}
\begin{problem}
    the set of all functions which have an infinite number of zeros.
\end{problem}

\begin{proposition}
    The subset $S = \{f \in R : f(x) = 0 \text{ for infinitely many } x \in [0, 1]\}$ is not a subring of $R$.
\end{proposition}

\begin{proof}
    Consider the functions $f, g$ defined for $x \in [0, 1]$ by
    \begin{align*}
        f(x) &= \begin{cases}
                    0 &\text{if $x$ is rational} \\
                    1 &\text{if $x$ is irrational}
                \end{cases}\\
        g(x) &= \begin{cases}
                    1 &\text{if $x$ is rational} \\
                    0 &\text{if $x$ is irrational}.
                \end{cases}
    \end{align*}
    Since there are infinitely many of both rational and irrational numbers on the interval $[0, 1]$, then both functions are in $S$. However, their sum $f + g$ is simply the constant function mapping all values in $[0, 1]$ to $1 \in \R$, which is not in $S$. So $S$ is not closed under addition and, therefore, it is not a subring.

\end{proof}

\subsection{Problem 7.1.6(e)}
\begin{problem}
    the set of all functions $f$ such that $\ds\lim_{x \to 1^-} f(x) = 0$
\end{problem}

\begin{proposition}
    The subset $S = \{f \in R : \ds\lim_{x \to 1^-} f(x) = 0\}$ is a subring of $R$.
\end{proposition}

\begin{proof}
    The zero function is in $S$, so it is nonempty. For some $f, g \in S$, we consider the functions $f - g$ and $fg$. Since the left-hand limits of both $f$ and $g$ at $1$ are defined and finite, we have
    \[
        \lim_{x \to 1^-} (f - g)(x) = \lim_{x \to 1^-} (f(x) - g(x)) = \lim_{x \to 1^-} f(x) - \lim_{x \to 1^-} g(x) = 0 - 0 = 0
    \]
    and
    \[
        \lim_{x \to 1^-} (fg)(x) = \lim_{x \to 1^-} (f(x)g(x)) = \left(\lim_{x \to 1^-} f(x)\right) \left(\lim_{x \to 1^-} g(x)\right) = 0 \cdot 0 = 0.
    \]
    Thus, we have $f - g$ and $fg$ in $S$, so the subring criterion tells us that $S$ is a subring of $R$.
    
\end{proof}

\newpage
\section{Problem 7.1.11}
\begin{problem}
    Prove that if $R$ is an integral domain and $x^2 = 1$ for some $x \in R$ then $x = \pm 1$.
\end{problem}

\begin{proof}
    Let $R$ be an integral domain and suppose $x \in R$ such that $x^2 = 1$. Since $(R, +)$ is a group, then the additive identity of $1$ is in $R$, i.e., $-1 \in R$. Adding this to both sides of $x^2 = 1$, we obtain $x^2 - 1 = 0$. The distributive law for rings gives us
    \[
        (x + 1)(x - 1) = xx + x(-1) + 1x + (-1)1 = x^2 - 1.
    \]
    Now since $R$ has no zero divisors and $(x + 1)(x - 1) = 0$, then we must have
    \[
        x + 1 = 0 \isp{or} x - 1 = 0.
    \]
    In the first case, adding $-1$ to both sides gives us $x = -1$ and, in the second case, adding $1$ to both sides gives us $x = 1$.
    
\end{proof}

\newpage
\section{Problem 7.1.14}
\begin{problem}
    Let $x$ be a nilpotent element of the commutative ring $R$.
\end{problem}

\subsection{Problem 7.1.14(a)}
\begin{problem}
    Prove that $x$ is either zero or a zero divisor.
\end{problem}

\begin{proof}
    If $x = 0$, then we are done. Suppose $x \ne 0$ and consider the set
    \[
        \{n \in \Z^+ : x^n = 0\}.
    \]
    Since $x$ is nilpotent, this set is nonempty and the well-ordering principle tells us that its minimum exists; let it be $k$. On one hand, $k - 1 < k$ implies that $x^{k - 1} \ne 0$. On the other hand,
    \[
        xx^{k - 1} = x^k = 0.
    \]
    So $x$ is a zero divisor (and so is $x^{k-1}$).

\end{proof}

\subsection{Problem 7.1.14(b)}
\begin{problem}
    Prove that $rx$ is nilpotent for all $r \in R$.
\end{problem}

\begin{proof}
    Let $r \in R$. We will prove by induction on $k$ that $(rx)^k = r^k x^k$. For the base case, take $k = 1$, then
    \[
        (rx)^1 = rx = r^1x^1.
    \]
    Now suppose $(rx)^k = r^k x^k$, for some $k \in \Z^+$. Since $R$ is a a commutative ring, we find
    \[
        (rx)^{k + 1} = rx(rx)^k = rxr^kx^k = rr^kxx^k = r^{k+1}x^{k+1},
    \]
    completing the induction. Now since $x$ is nilpotent, let $n \in \Z^+$ such that $x^n = 0$. Then
    \[
        (rx)^n = r^nx^n = r^n0 = 0,
    \]
    so $rx$ is nilpotent.
    
\end{proof}

\newpage
\subsection{Problem 7.1.14(c)}
\begin{problem}
    Prove that $1 + x$ is a unit in $R$.
\end{problem}

\begin{lemma}
    $1 - y$ is a unit in $R$ for all nilpotent $y \in R$.
\end{lemma}

\begin{proof}
    Let $n \in \Z^+$ such that $y^n = 0$. Consider the element of $R$ given by
    \[
        \sum_{k=0}^{n-1} y^k = 1 + y + y^2 + \cdots + y^{n-1}.
    \]
    Since $R$ is commutative, it suffices to show that this is a right inverse of $1 - y$. By the distributive law of rings, we find
    \begin{align*}
        (1 - y) \left( \sum_{k=0}^{n-1} y^k \right) 
            &= \sum_{k=0}^{n-1} y^k - \sum_{k=0}^{n-1} y^{k+1} \\
            &= 1 + \sum_{k=1}^{n-1} y^k - \sum_{k=1}^{n-1} y^k - y^n \\
            &= 1 - y^n \\
            &= 1.
    \end{align*}
    
\end{proof}

\begin{proposition}
    $1 + x$ is a unit in $R$.
\end{proposition}

\begin{proof}
    From (b), we know $-x = (-1)x$ is nilpotent. Then Lemma 1 tells us that $1 - (-x) = 1 + x$ is a unit in $R$.
    
\end{proof}

\subsection{Problem 7.1.14(d)}
\begin{problem}
    Deduce that the sum of a nilpotent element and a unit is a unit.
\end{problem}

\begin{proof}
    Let $u, x \in R$ such that $u$ is a unit and $x$ is nilpotent. Part (b) tells us $u^{-1}x$ is nilpotent and part (c) tells us $1 + u^{-1}x$ is a unit. Since $R^\times$ is a group, then it is closed under multiplication, so
    \[
        u(1 + u^{-1}x) = u + x
    \]
    is a unit.
    
\end{proof}

\newpage
\section{Problem 7.3.1}
\begin{problem}
    Prove that the rings $2\Z$ and $3\Z$ are not isomorphic.
\end{problem}

\begin{proof}
    Assume, for contradiction, that there exists an isomorphism $\phi : 2\Z \to 3\Z$. Then
    \begin{align*}
        2 + 2 &=  2 \cdot 2 \\
        \phi(2 + 2) &= \phi(2 \cdot 2) \\
        \phi(2) + \phi(2) &= \phi(2) \cdot \phi(2).
    \end{align*}
    Since $\phi(2) \in 3\Z$, then $\phi(2) = 3k$ for some $k \in \Z$. Therefore,
    \begin{align*}
        3k + 3k &= 3k \cdot 3k \\
        6k &= 9k^2 \\
        2 &= 3k.
    \end{align*}
    This implies $k = \frac23$, which is a contradiction.
    
\end{proof}

\newpage
\section{Problem 7.3.7}
\begin{problem}
    Let $R = \{\mqty(a & b \\ 0 & d) : a, b, d \in \Z\}$ be the subring of $M_2(\Z)$ of upper right triangular  matrices. Prove that the map
    \[
        \phi : R \to \Z \times \Z \isp{defined by} \phi : \mqty(a & b \\ 0 & d) \mapsto (a, d)
    \]
    is a surjective homomorphism and describe its kernel.
\end{problem}

\begin{proof}
    For any $(a, b) \in \Z \times \Z$, we have the upper triangular matrix
    \[
        A = \mqty(a & 0 \\ 0 & b) \in R,
    \]
    with $\phi(A) = (a, b)$. Hence, $\phi$ is surjective. Let $A, B \in R$ be the upper triangular matrices
    \[
        A = \mqty(a_1 & a_2 \\ 0 & a_3) \isp{and} B = \mqty(b_1 & b_2 \\ 0 & b_3).
    \]
    Then $\phi$ maps their sum as follows:
    \[
        A + B = \mqty(a_1 + b_1 & a_2 + b_2 \\ 0 & a_3 + b_3) \mapsto (a_1 + b_1,\; a_3 + b_3).
    \]
    And this does agree with the sum of the images of $A$ and and $B$ under $\phi$:
    \[
        \phi(A) + \phi(B) = (a_1, a_3) + (b_1, b_3) = (a_1 + b_1,\; a_3 + b_3).
    \]
    We now look at how $\phi$ maps their product:
    \[
        AB
            = \mqty(a_1b_1 + a_20 & a_2b_1 + a_3b_3 \\ a_10 + 0b_3 & a_20 + a_3b_3)
            = \mqty(a_1b_1 & a_2b_1 + a_3b_3 \\ 0 & a_3b_3)
            \mapsto (a_1b_1,\; a_3b_3).
    \]
    And this does agree with the product of the images of $A$ and $B$ under $\phi$:
    \[
        \phi(A)\phi(B) = (a_1, a_3)(b_1, b_3) = (a_1b_1,\; a_3b_3).
    \]
    Hence, $\phi$ is a ring homomorphism.
    
\end{proof}

The kernel of $\phi$ is precisely the set of strictly upper triangular matrices, i.e., the set
\[
    \{\mqty(0 & a \\ 0 & 0) : a \in \Z\}.
\]
Moreover, $\ker \phi$ is isomorphic to $\Z$ with the usual addition and multiplication. It is trivial to check that the map $\mqty(0 & a \\ 0 & 0) \mapsto a$ is an isomorphism.

\end{document}