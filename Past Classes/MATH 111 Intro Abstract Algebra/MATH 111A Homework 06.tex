\documentclass[12pt]{article}

% Packages
\usepackage[margin=1in]{geometry} % proper margins
\usepackage{enumitem} % custom numbering for lists
\usepackage{amsmath} % align, cases, eqref, matrices, dots, roots, delimiters, math mode functions, mod, arrows
\usepackage{amsthm} % theorems, proofs
\usepackage{amssymb} % fancy letters, niche relations/negations
\usepackage{mathrsfs} % much more loopy calligraphy font

% Theorems
\newtheorem{theorem}{Theorem}
\newtheorem{lemma}{Lemma}
\newtheorem{proposition}{Proposition}

% Problem Box
\setlength{\fboxsep}{4pt}
\newsavebox{\mybox}
\newenvironment{problem}
    {\begin{lrbox}{\mybox}\begin{minipage}{0.98\textwidth}}
    {\end{minipage}\end{lrbox}\framebox[\textwidth]{\usebox{\mybox}}}

% Formatting
\newcommand{\ds}{\displaystyle}
\newcommand{\isp}[1]{\quad\text{#1}\quad}

% Alternate Characters
\let\eps\varepsilon % double curved, rather than single curve with midline
\let\phi\varphi % single stroke loop, rather than vertical line through circle
\let\emptyset\varnothing % circular, rather than tall

% Named Sets
\newcommand{\N}{\mathbb{N}} % natural numbers
\newcommand{\Z}{\mathbb{Z}} % integers 
\newcommand{\Q}{\mathbb{Q}} % rational numbers
\newcommand{\R}{\mathbb{R}} % real numbers
\newcommand{\C}{\mathbb{C}} % complex numbers

% Fancy Characters
\newcommand{\F}{\mathbb{F}} % arbitrary field
\renewcommand{\P}{\mathbb{P}} % probability measure (apparently, sometimes prime numbers)
\newcommand{\FF}{\mathcal{F}} % sigma algebra
\newcommand{\BB}{\mathcal{B}} % Borel sigma algebra

% Paired Delimiters
\newcommand{\ceil}[1]{\left\lceil #1 \right\rceil} % ceiling
\newcommand{\floor}[1]{\left\lfloor #1 \right\rfloor} % floor
\newcommand{\<}{\left\langle} % left angle bracket
\renewcommand{\>}{\right\rangle} % right angle bracket

% Functions
\renewcommand{\Im}{\operatorname{Im}} % imaginary part of a complex number
\renewcommand{\Re}{\operatorname{Re}} % real part of a complex number
\newcommand{\Arg}{\operatorname{Arg}} % principal argument (angle) of complex number

% Simplified Notation
\newcommand{\id}[1]{\operatorname{id_{\mathnormal{#1}}}} % identity operator
\newcommand{\seq}[2][n]{\left\{#2\right\}_{#1\in\N}} % sequence
\newcommand{\pdv}[3][1]{\ifnum#1=1{\frac{\partial #2}{\partial#3}}\else{\frac{\partial^{#1}#2}{\partial#3^{#1}}}\fi} % partial derivative
\newcommand{\intr}[1]{\accentset{\circ}{#1}} % interior of a set

% Renaming
\let\sm\setminus % set minus (difference)
\let\clo\overline % closure of a set
\let\conj\overline % conjugate of an object
\let\eqc\overline % equivalence class of an object
\let\teq\trianglelefteq % normal subgroup
\let\iso\cong % isomorphic (groups)

% Notes
% medskip for header-less paragraph
% intertext{} for short text inside big display structure
% dots is dynamic based on surroundings
% dfrac and tfrac to force large or small fractions
% operatorname for new operators instead of text
% consider using nath; it seems to break most formatting and is not compatible with many packages

\begin{document}
 
\title{Homework 6\\
    %\large MATH CS 121 Intro to Probability
    \large MATH 111A Intro to Abstract Algebra
    %\large MATH CS 122A Complex Analysis I
    %\large MATH 118A Intro to Real Analysis
    %\large MATH 104A Intro to Numerical Analysis
}
\author{Harry Coleman}
\date{November 18, 2020}
\maketitle

\section*{Q1 Exercise 1.7.3}
\begin{problem}
    Show that the additive group $\Z$ acts on itself by $z\cdot a = z + a$ for all $z,a\in\Z$.
\end{problem}

\begin{proof}
    The first axiom of a group action is satisfied by the associativity of addition in $\Z$. If $z_1,z_2,a\in\Z$, then
    \[(z_1 + z_2) \cdot a = (z_1 + z_2) + a = z_1 + (z_2 + a) = z_1 + (z_2\cdot a) = z_1\cdot(z_2\cdot a).\]
    And the second axiom is satisfied simply by the additive identity in $\Z$:
    \[0\cdot a = 0 + a = a.\]
    
\end{proof}

\newpage
\section*{Q2 Exercise 4.1.1}
\begin{problem}
    Let $G$ act on the set $A$. Prove that if $a,b\in A$ and $b= g\cdot a$ for some $g\in G$, then $G_b = gG_ag^{-1}$ ($G_a$ is the stabilizer of $a$). Deduce that if $G$ acts transitively on $A$, then the kernel of the action is $\ds\bigcap_{g\in G} gG_ag^{-1}$.
\end{problem}

\begin{proposition}
    If $a,b\in A$ and $b= g\cdot a$ for some $g\in G$, then $G_b = gG_ag^{-1}$ ($G_a$ is the stabilizer of $a$).
\end{proposition}

\begin{proof}
    Suppose we have $a,b\in A$ and $g\in G$ such that $b = g\cdot a$. Note that
    \[gG_ag^{-1} = \{ghg^{-1} : h\in G_a\}.\]
    Suppose $h\in G_b$, so $h \cdot b = b$. Then
    \begin{align*}
        g^{-1}hg \cdot a 
            &= g^{-1}h \cdot (g \cdot a) \\
            &= g^{-1}h \cdot b\\
            &= g^{-1} \cdot (h \cdot b)\\
            &= g^{-1} \cdot b\\
            &= g^{-1} \cdot (g \cdot a)\\
            &= g^{-1}g \cdot a\\
            &= 1 \cdot a\\
            &= a.
    \end{align*}
    This implies that $g^{-1}hg \in G_a$, so
    \[h = g(g^{-1}hg)g^{-1} \in gG_ag^{-1}.\]
    
    Now suppose $h \in gG_ag^{-1}$, so $h = gkg^{-1}$ for some $k\in G_a$. Then
    \begin{align*}
        h \cdot b
            &= gkg^{-1} \cdot (g\cdot a) \\
            &= gkg^{-1}g \cdot a \\
            &= gk \cdot a \\
            &= g \cdot (k\cdot a) \\
            &= g \cdot a\\
            &= b.
        \end{align*}
        Thus, $h\in G_b$ and we have the equality
        \[G_b = gG_ag^{-1}.\]
    
\end{proof}

\newpage
\begin{proposition}
    If $G$ acts transitively on $A$, then the kernel of the action is $\ds\bigcap_{g\in G} gG_ag^{-1}$.
\end{proposition}

\begin{proof}
    Suppose $G$ acts transitively on $A$, i.e., $G\cdot x = A$ for all $x\in A$. In particular, $G\cdot a = A$, so $x\in G\cdot a$ for all $x\in A$. Consider the set of all stabilizers, $\{G_x : x\in A\}$, and the set $\{gG_ag^{-1} : g\in G\}$. We claim that these sets are equal. Suppose $G_x$ is an arbitrary element of the first set. Since $G\cdot a = A$, there exists some $g\in G$ such that $g\cdot a = x$. Then by proposition 1, we have $G_x = gG_ag^{-1}$, so
    \[G_x = gG_ag^{-1} \in \{gG_ag^{-1} : g\in G\}.\]
    Now suppose $gG_ag^{-1}$ is an arbitrary element of the second set. By the definition of $G$ acting on $A$, we have $g\cdot a = x$ for some $x\in A$. Then by proposition 1, we have $G_x = gG_xg^{-1}$, so
    \[gG_xg^{-1} = G_x \in \{G_x : x\in A\}.\]
    Therefore,
    \[\{G_x : x\in A\} = \{gG_ag^{-1} : g\in G\},\]
    and the kernel of the group action is
    \[\bigcap_{x\in A} G_x = \bigcap_{g\in G} gG_ag^{-1}.\]
    
\end{proof}

\newpage
\section*{Q3 Exercise 4.1.4}
\begin{problem}
    Let $S_3$ act on the set $\Omega$ of ordered pairs: $\{(i,j) : 1\leq i, j \leq 3\}$ by $\sigma((i,j)) = (\sigma(i),\sigma(j))$. Find the orbits of $S_3$ on $\Omega$.
\end{problem}
\medskip

For any two pairs $(i,i)$ and $(j,j)$, with $i\ne j$, we have the permutation $(ij)\in S_3$. Then
\[(ij)\cdot(i,i) = (j,j),\]
so $(i,i)$ and $(j,j)$ are in the same orbit. Moreover, for any $(i,i)$ and $\sigma\in S_3$, if $\sigma(i) = j$, then
\[\sigma\cdot(i,i) = (j,j),\]
so the set
\[\{(1,1), (2,2), (3,3)\}.\]
is one orbit of $S_3$ on $\Omega$. For any pair $(i,j)$ with $i\ne j$, we have $(ij)\in S_3$. Then
\[(ij) \cdot (i,j) = (j,i),\]
So $(i,j)$ and $(j,i)$ are in the same orbit. For any two pairs $(i,j),(i,k)$, with $\{i,j,k\} = \{1,2,3\}$, we have $(jk)\in S_3$. Then
\[(jk) \cdot (i,j) = (i,k),\]
so $(i,j)$ and $(i,k)$ are in the same orbit. Moreover, $(i,j)$ and $(k,i)$ are in the same orbit since $(i,k)$ and $(k,i)$ are in the same orbit. Therefore, the other orbit of $S_3$ on $\Omega$ is
\[\{(1,2),(2,1),(1,3),(3,1),(2,3),(3,2)\}.\]





\end{document}