\documentclass[12pt]{article}

% Packages
\usepackage[margin=1in]{geometry}
\usepackage{amsmath, amsthm, amssymb, physics}

% Problem Box
\setlength{\fboxsep}{4pt}
\newsavebox{\mybox}
\newenvironment{problem}
    {\begin{lrbox}{\mybox}\begin{minipage}{0.98\textwidth}}
    {\end{minipage}\end{lrbox}\begin{center}\framebox[\textwidth]{\usebox{\mybox}}\end{center}}

% Options
\renewcommand{\thesubsection}{\thesection(\alph{subsection})}
\allowdisplaybreaks
\addtolength{\jot}{1em}
\theoremstyle{definition}

% Default Commands
\newtheorem{proposition}{Proposition}
\newtheorem{lemma}{Lemma}
\newcommand{\ds}{\displaystyle}
\newcommand{\isp}[1]{\quad\text{#1}\quad}
\newcommand{\N}{\mathbb{N}}
\newcommand{\Z}{\mathbb{Z}}
\newcommand{\Q}{\mathbb{Q}}
\newcommand{\R}{\mathbb{R}}
\newcommand{\C}{\mathbb{C}}
\newcommand{\eps}{\varepsilon}
\renewcommand{\phi}{\varphi}
\renewcommand{\emptyset}{\varnothing}

% Extra Commands
\newcommand{\<}{\left\langle}
\renewcommand{\>}{\right\rangle}
\newcommand{\inc}{\hookrightarrow}
\newcommand{\isom}{\cong}
\newcommand{\conj}{\overline}


% Document Info
\title{Homework 3\\
    \large MATH 111B
}
\author{Harry Coleman}
\date{January 29, 2021}

% Begin Document
\begin{document}
\maketitle

\section{Problem 7.4.2}
\begin{problem}
    Assume $R$ is commutative. Prove that the augmentation ideal in the group ring $RG$ is generated by $\{g - 1 \mid g \in G\}$. Prove that if $G = \< \sigma \>$ is cyclic then the augmentation ideal is generated by $\sigma - 1$.
\end{problem}

\begin{proposition}
    The augmentation ideal in the group ring $RG$ is generated by $\{g - 1 \mid g \in G\}$.
\end{proposition}

\begin{proof}
    Let $G = \{g_1, \dots, g_n\}$, let $A = \{\sum_{i=1}^n a_i g_i \mid \sum_{i=1}^n a_i = 0\}$ be the augmentation ideal of the group ring $RG$, and let $S = \{g-1 \mid g \in G\}$. We aim to prove $(S) = A$. For any $g-1 \in S$ we have, more explicitly,
    \[
        g-1 = 1_Rg - 1_R1_G.
    \]
    And since $1_R - 1_R = 0$, then $g-1 \in A$, so $S \subseteq A$. Because $(S)$ is the smallest ideal of $RG$ containing $S$ and $A$ is an ideal containing $S$, then $(S) \subseteq A$. Suppose $x \in A$, then
    \[
        x = \sum_{i=1}^n a_i g_i \isp{and} \sum_{i=1}^n a_i = 0.
    \]
    Regarding $\sum_{i=1}^n a_i = 0$ as the element in the group ring $RG$, we have
    \[
        x
            = \sum_{i=1}^n a_i g_i - \sum_{i=1}^n a_i
            = \sum_{i=1}^n(a_ig_i - a_i)
            = \sum_{i=1}^n a_i(g_i - 1).
    \]
    For $i = 1, \dots, n$, we have $a_i = a_i1_G \in RG$ and $g_i - 1 \in S$, so $x \in RGS \subseteq (S)$. Therefore, $A \subseteq (S)$, giving us equality $(S) = A$.
    
\end{proof}

\newpage
\begin{proposition}
    If $G = \< \sigma \>$ is cyclic then the augmentation ideal is generated by $\sigma - 1$.
\end{proposition}

\begin{proof}
    Let $S = \{g - 1 \mid g \in G\}$, then by Proposition 1, $(S)$ is the augmentation ideal of $RG$. We aim to prove $(\sigma - 1) = (S)$. Since $\sigma - 1 \in S$, then $(\sigma-1) \subseteq (S)$. Let $g - 1 \in S$, then $g = \sigma^n$ for some $n \in \Z^+$ and
    \[
        g - 1 = \sigma^n - 1 = (\sigma - 1)\sum_{k=0}^{n-1} \sigma^k \in (\sigma - 1)RG \subseteq (\sigma - 1).
    \]
    Thus, $S \subseteq (\sigma - 1)$, which implies $(S) \subseteq (\sigma - 1)$ since $(\sigma - 1)$ is an ideal, giving us equality $(\sigma - 1) = (S)$.
    
\end{proof}

\section{Problem 7.4.8}
\begin{problem}
    Let $R$ be an integral domain. Prove that $(a) = (b)$ for some elements $a, b \in R$, if and only if $a = ub$ for some unit $u$ or $R$.
\end{problem}

\begin{proof}
    Suppose $a, b \in R$ with $(a) = (b)$. In particular, $a \in (b)$ and $b \in (a) $, so $a = ub$ and $b = va$ for some $u, v \in R$. Therefore, $a = ub = uva$ and, because $R$ is an integral domain, the cancellation law gives us $1 = uv$, i.e., $v = u^{-1}$. So $a = ub$ with $u \in R^\times$.
    
    Now suppose $a = ub$ for some $u \in R^\times$. In particular, $a = ub \in (b)$ so $(a) \subseteq (b)$. Moreover, $b = u^{-1}a \in (a)$ so $(b) \subseteq (a)$, implying equality $(a) = (b)$.
    
\end{proof}

\section{Problem 7.4.11}
\begin{problem}
    Assume $R$ is commutative. Let $I$ and $J$ be ideals of $R$ and assume $P$ is a prime ideal of $R$ that contains $IJ$ (for example, if $P$ contains $I \cap J$). Prove either $I$ or $J$ is contained in $P$.
\end{problem}

\begin{proof}
    Assuming $J \not\subseteq P$, we will show $I \subseteq P$. Let $y \in J$ such that $y \notin P$. Then for all $x \in I$ we have $xy \in IJ \subseteq P$. Because $P$ is a prime ideal, either $x \in P$ or $y \in P$. But, since $y \notin P$, we must have $x \in P$. Hence, $I \subseteq P$.
    
\end{proof}

\newpage
\section{Problem 7.4.13}
\begin{problem}
    Let $\phi : R \to S$ be the homomorphism of commutative rings.
\end{problem}

\subsection{Problem 7.4.13(a)}
\begin{problem}
    Prove that if $P$ is a prime ideal of $S$ then either $\phi^{-1}(P) = R$ or $\phi^{-1}(P)$ is a prime ideal of $R$. Apply this to the special case when $R$ is a subring of $S$ and $\phi$ is the inclusion homomorphism to deduce that if $P$ is a prime ideal of $S$ then $P \cap R$ is either $R$ or a prime ideal of $R$.
\end{problem}

\begin{proof}
    Let $P$ be a prime ideal of $S$ and assume $\phi^{-1}(P) \ne R$; we will show that $\phi^{-1}(P)$ is a prime ideal of $R$. In particular, $P$ is an ideal of $S$, so $\phi^{-1}(P)$ is an ideal of $R$. Suppose $ab \in \phi^{-1}(P)$ for some $a, b \in R$, then
    \[
        \phi(a)\phi(b) = \phi(ab) \in P.
    \]
    So either $\phi(a) \in P$ or $\phi(b) \in P$. In other words, either $a \in \phi^{-1}(P)$ or $\phi^{-1}(P)$, implying that $P$ is a prime ideal.
    
    In the case that $\phi : R \inc S$ is the inclusion map, then
    \[
        \phi^{-1}(P) = \{x \in R \mid \phi(x) = x \in P\} = R \cap P.
    \]
    Hence, $R \cap P$ is either $R$ or a prime ideal of $R$.
    
\end{proof}

\subsection{Problem 7.4.13(b)}
\begin{problem}
    Prove that if $M$ is a maximal ideal of $S$ and $\phi$ is surjective then $\phi^{-1}(M)$ is a maximal ideal of $R$. Give an example to show that this need not be the case if $\phi$ is not surjective.
\end{problem}

\begin{proof}
    Let $M$ be a maximal ideal of $S$, then $S/M$ is a field. Suppose $\phi : R \to S$ is surjective and consider the natural projection $\pi : S \to S/M$, which is a surjective ring homomorphism with $\ker\phi = M$. Then the composition $\psi = \pi \circ \phi : R \to S/M$ is a surjective ring homomorphism with
    \begin{align*}
        \ker \psi 
            &= \{x \in R \mid \psi(x) = \pi(\phi(x)) = 0\} \\
            &= \{x \in R \mid \phi(x) \in \ker \pi = M\} \\
            &= \{x \in R \mid x \in \phi^{-1}(M)\} \\
            &= \phi^{-1}.
    \end{align*}
    Then $R/\phi^{-1}(M) = R/\ker\psi \isom S/M$. Since $S/M$ is a field, then so is $R/\phi^{-1}(M)$, implying that $\phi^{-1}(M)$ is a maximal ideal of $R$.
    
\end{proof}

To see why this is not necessarily the case if $\phi$ is not surjective, consider $R = S = \Z$, $M = 2\Z$, and let $\phi$ be the zero homomorphism $\phi : R \to \{0\} \subseteq S$. Then $M$ is a maximal ideal of $S$, but $\phi^{-1}(M) = R$ is not a maximal ideal of $R$.

\newpage
\section{Problem 7.4.17}
\begin{problem}
    Let $x^4 - 2x + 1$ be an element of the polynomial ring $E = \Z[x]$ and use the bar notation to denote passage to the quotient ring $\Z[x]/(x^3 - 2x + 1)$. Let $p(x) = 2x^7 - 7x^5 + 4x^3 - 9x + 1$ and let $q(x) = (x - 1)^4$.
\end{problem}


\subsection{Problem 7.4.17(a)}
\begin{problem}
    Express each of the following elements of $\conj{E}$ in the form $\conj{f(x)}$ for some polynomial $f(x)$ of degree $\leq 2$: $\conj{p(x)}$, $\conj{q(x)}$, $\conj{p(x) + q(x)}$, and $\conj{p(x)q(x)}$.
\end{problem}

\begin{align*}
    \conj{p(x)} &= \conj{-x^2 - 11x + 3} \\
    \conj{q(x)} &= \conj{8x^2 - 13x + 5} \\
    \conj{p(x) + q(x)} &= \conj{7x^2 - 24x + 8} \\
    \conj{p(x)q(x)} &= \conj{146x^2 - 236 + 90}
\end{align*}

\subsection{Problem 7.4.17(b)}
\begin{problem}
    Prove that $\conj{E}$ is not an integral domain.
\end{problem}

\begin{proof}
    Let $f(x) = x - 1$ and $g(x) = x^2 + x - 1$, then $\conj{f(x)}, \conj{g(x)} \in \conj{E}$, with
    \begin{align*}
        \conj{f(x)}\conj{g(x)} 
            &= \conj{f(x)g(x)} \\
            &= \conj{(x - 1)(x^2 + x - 1)} \\
            &= \conj{x^3 - 2x + 1} \\
            &= \conj{0}.
    \end{align*}
    Thus, $\conj{f(x)}$ and $\conj{g(x)}$ are zero divisors in $\conj{E}$, so $\conj{E}$ is not an integral domain.
    
\end{proof}

\subsection{Problem 7.4.17(c)}
\begin{problem}
    Prove that $\conj{x}$ is a unit of $\conj{E}$.
\end{problem}

\begin{proof}
    Consider $\conj{-x^2 + 2} \in \conj{E}$. We have
    \[
        \conj{x} \cdot \conj{-x^2 + 2} = \conj{-x^3 + 2x} = \conj{1},
    \]
    hence, $\conj{x} \in \conj{E}^\times$.
    
\end{proof}



\end{document}