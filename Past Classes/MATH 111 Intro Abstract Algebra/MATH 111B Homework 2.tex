\documentclass[12pt]{article}

% Packages
\usepackage[margin=1in]{geometry}
\usepackage{amsmath, amsthm, amssymb, physics}

% Problem Box
\setlength{\fboxsep}{4pt}
\newsavebox{\mybox}
\newenvironment{problem}
    {\begin{lrbox}{\mybox}\begin{minipage}{0.98\textwidth}}
    {\end{minipage}\end{lrbox}\begin{center}\framebox[\textwidth]{\usebox{\mybox}}\end{center}}

% Options
\renewcommand{\thesubsection}{\thesection(\alph{subsection})}
\allowdisplaybreaks
\addtolength{\jot}{1em}
\theoremstyle{definition}

% Default Commands
\newtheorem{proposition}{Proposition}
\newtheorem{lemma}{Lemma}
\newcommand{\ds}{\displaystyle}
\newcommand{\isp}[1]{\quad\text{#1}\quad}
\newcommand{\N}{\mathbb{N}}
\newcommand{\Z}{\mathbb{Z}}
\newcommand{\Q}{\mathbb{Q}}
\newcommand{\R}{\mathbb{R}}
\newcommand{\C}{\mathbb{C}}
\newcommand{\eps}{\varepsilon}
\renewcommand{\phi}{\varphi}
\renewcommand{\emptyset}{\varnothing}

% Extra Commands
\newcommand{\isom}{\cong}
\newcommand{\conj}{\overline}
\newcommand{\inc}{\hookrightarrow}


% Document Info
\title{Homework 2\\
    \large MATH 111B
}
\author{Harry Coleman}
\date{January 22, 2021}

% Begin Document
\begin{document}
\maketitle

\section{Problem 7.3.10}
\begin{problem}
    Decide which of the following are an ideal of the ring $\Z[x]$:
\end{problem}

\subsection{Problem 7.3.10(a)}
\begin{problem}
    the set of all polynomials whose constant term is a multiple of $3$.
\end{problem}

The composition of ring homomorphisms
\[
    \begin{array}{cccccc}
        \phi : & \Z[x] & \to & \Z & \to & \Z/3\Z \\
             & p(x) & \mapsto & p(0) & \mapsto & \conj{p(0)}
    \end{array}
\]
is a ring homomorphism. Moreover, its kernel is
\[
    \ker\phi
        = \{p(x) \mid \conj{p(0)} = 0\}
        = \{p(x) \mid p(0) \in 3\Z\},
\]
which is precisely the polynomials whose constant term is a multiple of $3$. As the kernel of a ring homomorphism, it is an ideal of $\Z[x]$.



\subsection{Problem 7.3.10(b)}
\begin{problem}
    the set of all polynomials whose coefficient of $x^2$ is a multiple of $3$.
\end{problem}

Consider $2x + 3x^2$ in this set. Then $x(2x + 3x^2) = 2x^2 + 3x^3$, which is not in the set, so it is not an ideal.

\subsection{Problem 7.3.10(d)}
\begin{problem}
    $\Z[x^2]$ (i.e., the polynomials in which on even powers of $x$ appear)
\end{problem}

Consider $x^2$ in this set. Then $x(x^2) = x^3$, which is not in the set, so it is not an ideal.

\section{Problem 7.3.24}
\begin{problem}
    Let $\phi : R \to S$ be a ring homomorphism.
\end{problem}

\subsection{Problem 7.3.24(a)}
\begin{problem}
    Prove that if $J$ is an deal of $S$ then $\phi^{-1}(J)$ is an ideal of $R$. Apply this to the special case when $R$ is a subring of $S$ and $\phi$ is the inclusion homomorphism to deduce that if $J$ is an ideal of $S$ then $J\cap R$ is an ideal of $R$.
\end{problem}

\begin{proof}
    Since $\phi$ is a group homomorphism and $J$ is a subgroup of $S$ for addition, then $\phi^{-1}(J)$ is a subgroup of $R$ for addition. Let $r \in R$ and $x \in \phi^{-1}(J)$, then
    \[
        \phi(rx) = \phi(r)\phi(x) \in \phi(r)J = J,
    \]
    \[
        \phi(xr) = \phi(x)\phi(r) \in J\phi(r) = J.
    \]
    Thus, $rx, xr \in \phi^{-1}(J)$. so $\phi^{-1}(J)$ is an ideal of $R$. If $\phi : R \inc S$ is the inclusion homomorphism, then
    \[
        \phi^{-1}(J) = \{x \in R : \phi(x) = x \in J\} = R \cap J
    \]
    is an ideal of $R$.

\end{proof}

\subsection{Problem 7.3.24(b)}
\begin{problem}
    Prove that if $\phi$ is surjective and $I$ is an ideal of $R$ then $\phi(I)$ is an ideal of $S$. Give an example where this fails if $\phi$ is not surjective.
\end{problem}

\begin{proof}
    Since $\phi$ is a group homomorphism and $I$ is a subgroup of $R$ for addition, then $\phi(I)$ is a subgroup of $S$ for addition. Let $s \in S$ and $y \in \phi(I)$. Since $\phi$ is surjective, there is some $r \in R$ and $x \in I$ such that $\phi(r) = s$ and $\phi(x) = y$. Then
    \[
        sy = \phi(r)\phi(x) = \phi(rx) \in \phi(I),
    \]
    \[
        ys = \phi(x)\phi(r) = \phi(xr) \in \phi(I),
    \]
    so $\phi(I)$ is an ideal of $S$.
    
\end{proof}

Consider the inclusion ring homomorphism  $\Z \inc \Q$ be. Then $\Z$ is an ideal of $\Z$, but not and ideal of $\Q$. Consider $1 \in \Z$ and $\frac12 \in \Q$, then $\frac12\cdot 1 = \frac12 \notin\Z$.

\newpage
\section{Problem 7.3.34}
\begin{problem}
    Let $I$ and $J$ be ideals of $R$.
\end{problem}

\subsection{Problem 7.3.34(a)}
\begin{problem}
    Prove that $I + J$ is the smallest ideal of $R$ containing both $I$ and $J$.
\end{problem}

\begin{proof}
    Both $I$ and $J$ are subgroups of $R$ for addition, so $I + J$ is a subgroup of $R$ for addition. Let $r \in R$, then
    \begin{align*}
        r(I + J) 
            &= \{r(a + b) \mid a \in I, b \in J\} \\
            &= \{ra + rb \mid a \in I, b \in J\} \\
            &= rI + rJ \\
            &\subseteq I + J
    \end{align*}
    and
    \begin{align*}
        (I + J)r
            &= \{(a + b)r \mid a \in I, b \in J\} \\
            &= \{ar + br \mid a \in I, b \in J\} \\
            &= Ir + Jr \\
            &\subseteq I + J,
    \end{align*}
    so $I + J$ is an ideal of $R$. Suppose $K$ is an ideal of $R$ containing both $I + J$. Then for all $a \in I$ and $b \in J$, we have $a, b \in K$. Since $K$ is a subgroup of $R$ for addition, $a + b \in K$, implying $I + J \subseteq K$. 
    
\end{proof}

\newpage
\subsection{Problem 7.3.34(b)}
\begin{problem}
    Prove that $IJ$ is an ideal contained in $I \cap J$.
\end{problem}

\begin{proof}
    We first show $IJ$ is a subgroup of $R$ for addition. Since $0 \in I$ and $0 \in J$, then $0 \cdot 0 \in IJ \ne \emptyset$. Let $x, y \in IJ$. Then we can write $x$ and $y$ as
    \[
        x = \sum_{k=1}^n a_k b_k \isp{and} y = \sum_{k=n+1}^m a_k b_k,
    \]
    where $a_k \in I$ and $b_k \in J$ for $k = 1, \dots, m$. Then
    \[
        x - y = \sum_{k=1}^m a_k' b_k,
    \]
    where $a_k' = a_k$ for $k \leq n$, and $a_k' = -a_k$ for $k \geq n+1$. Then $a_k' \in I$ for $k = 1, \dots, m$, so $x - y \in IJ$ and $IJ$ is a subgroup of $R$ for addition. Let $r \in R$ and $x \in IJ$. Then
    \[
        rx = r\sum_{k=1}^n a_k b_k = \sum_{k=1}^n ra_k b_k = \sum_{k=1}^n a_k' b_k
    \]
    where $a_k' = ra_k$. Since $I$ is an ideal, $a_k' \in I$, so $rx \in IJ$. And
    \[
        xr = \left(\sum_{k=1}^n a_k b_k\right)r = \sum_{k=1}^n a_k b_kr = \sum_{k=1}^n a_k b_k'
    \]
    where $b_k' = b_kr$. Since $J$ is an ideal, $b_k' \in J$, so $xr \in IJ$. Thus, $IJ$ is an ideal of $R$. For any $x \in IJ$, we have
    \[
        x = \sum_{k=1}^n a_k b_k.
    \]
    Because $I$ and $J$ are both ideals,  $a_kb_k \in I \cap J$ for $k = 1, \dots n$. Since both $I$ and $J$ are subgroups of $R$ for addition, so is $I \cap J$ and we have $x \in I \cap J$.
    
\end{proof}

\subsection{Problem 7.3.34(c)}
\begin{problem}
    Give an example where $IJ \ne I \cap J$.
\end{problem}

Consider the ideal $I = 2\Z$ of $\Z$, then $I \cap I = 2\Z$, but
\begin{align*}
    I^2 
        &= \{a_1b_1 + \cdots + a_nb_n \mid a_j, b_j \in 2\Z\} \\
        &= \{(2k_1)(2\ell_1) + \cdots + (2k_n)(2\ell_n) \mid k, \ell \in \Z\} \\
        &= \{4(k_1\ell_1 + \cdots + k_n\ell_n) \mid k, \ell \in \Z\} \\
        &= 4\Z.
\end{align*}


\subsection{Problem 7.3.34(d)}
\begin{problem}
    Prove that if $R$ is commutative with identity and if $I + J = R$ then $IJ = I \cap J$.
\end{problem}

\begin{proof}
    Let $x \in I \cap J$. Since $R = I + J$ and $1 \in R$, then $1 = a + b$ for some $a \in I$ and $b \in J$. Since $R$ is commutative and $x$ is in both $I$ and $J$,
    \[
        x = x(a + b) = xa + xb = ax + xb \in IJ.
    \]
    Having already shown the opposite inclusion, we have equality $IJ = I \cap J$.
    
\end{proof}

\newpage
\section{}
\begin{problem}
    Let $A$ be a subring of a ring $R$ and $I$ be an ideal of $R$. Prove the following statements.
\end{problem}

\subsection{}
\begin{problem}
    $A + I = \{a + r \mid a \in A, r \in I\}$ is a subring of $R$.
\end{problem}

\begin{proof}
    Since $A$ and $I$ are both subgroups of $R$ for addition, then so is $A + I$. Let $a, b \in A$ and $x, y \in I$, so $a + x$ and $b + y$ are arbitrary elements of $A + I$. Then
    \[
        (a + x)(b + y) = a(b + y) + x(b + y) = ab + ay + xb + xy.
    \]
    Since $A$ is a subring, $ab \in A$. Since $I$ is an ideal, then $ay, xb \in I$. And since $I$ is a subring, then $(ay + xb + xy) \in I$. Hence, $ab + (ay + xb + xy) \in A + I$.
\end{proof}

\subsection{}
\begin{problem}
    $I$ is an ideal of $A + I$ and $A \cap I$ is an ideal of $A$.
\end{problem}

\begin{proof}
    Since $A$ is a subgroup of $R$ for addition, $0 \in A$, so $I = 0 + I \subseteq A + I$. Because $I$ and $A + I$ are subgroups of $R$ for addition and $I \subseteq A + I$, then $I$ is a subgroup of $A + I$ for addition. Since $rI, Ir \subseteq I$ for all $r \in R$, then, in particular, $rI, Ir \subseteq I$ for all $r \in A + I$. Hence, $I$ is an ideal of $A + I$.
    
    Since $A$ and $I$ are both subgroups of $R$ for addition, then so is $A \cap I$. Since $A$ is a subgroup of $R$ for addition and $A \cap I \subseteq A$, then $A + I$ is a subgroup of $A$ for addition. If $a \in A$ and $x \in A \cap I$, then $ax \in A$ since $A$ is a subring of $R$ and $ax \in I$ since $I$ is an ideal of $R$. This implies $ax \in A \cap I$ and $xa \in A \cap I$ for the same reasons. Hence, $A \cap I$ is an ideal of $A$.
    
\end{proof}

\subsection{}
\begin{problem}
    $A + I/I \isom A/A \cap I$.
\end{problem}

\begin{proof}
    Since $A$ and $A + I$ are subrings of $R$ with $A \subseteq A + I$, then $A$ is a subring of $A + I$ and the inclusion map $A \inc (A + I)$ is a ring homomorphism. Since $I$ is and ideal of $A + I$, then the natural projection $\pi : (A + I) \to (A + I)/I$  is a ring homomorphism. So their composition
    \[
        \begin{array}{cccccc}
            \phi : & A & \inc & A + I & \to & (A + I)/I \\
                 & a & \mapsto & a + 0 & \mapsto & a + I
        \end{array}
    \]
    is a ring homomorphism. Its kernel is
    \[
        \ker \phi = \{a \in A \mid a + I = I\} = \{a \in A \mid a \in I\} = A \cap I,
    \]
    so the first isomorphism theorem for rings gives us $A/(A \cap I) \isom (A + I)/I$.
    
\end{proof}

\section{}
\begin{problem}
    Let $I$ and $J$ be ideals of a ring $R$ with $I \subseteq J$. Prove the following statements.
\end{problem}

\subsection{}
\begin{problem}
    $I$ is an ideal of $J$ and $J/I$ is an ideal of $R/I$.
\end{problem}

\begin{proof}
    Since $I$ and $J$ are subrings of $R$ with $I \subseteq J$, then $I$ is a subring of $J$. In particular, $I$ is a subgroup of $J$ for addition. Since $rI, Ir \subseteq I$ for all $r \in R$, then, in particular, $rI, Ir \subseteq I$ for all $r \in J$. 
    
    The natural projection $\pi : R \to R/I$ is a surjective homomorphism and $J$ is an ideal of $R$, so $\pi(J) = J/I$ is an ideal of $\pi(R) = R/I$.
    
\end{proof}

\subsection{}
\begin{problem}
    $(R/I)/(J/I) \isom R/J$.
\end{problem}

\begin{proof}
    Let $\phi : R/I \to R/J$ be the map $(r + I) \mapsto (r + J)$. This map is well-defined since $r_1 + I = r_2 + I$ only if $r_1 - r_2 \in I \subseteq J$, implying that $r_1 + J = r_2 + J$. Note that $\phi$ is surjective since $\phi(r + I) = r + J$ for any $r + J \in R/J$. Then
    \begin{align*}
        \phi((a + I) + (b + I))
            &= \phi((a + b) + I) \\
            &= (a + b) + J \\
            &= (a + J) + (b + J) \\
            &= \phi(a + I) + \phi(b + I) 
    \end{align*}
    and
    \begin{align*}
        \phi((a + I)(b + I))
            &= \phi(ab + I) \\
            &= ab + J \\
            &= (a + J)(b + J) \\
            &= \phi(a + I)\phi(b + I).
    \end{align*}
    Thus, $\phi$ is a surjective ring homomorphism. Moreover, its kernel is
    \[
        \ker \phi = \{r + I \mid r + J = J\} = \{r + I \mid r \in J\} = J/I,
    \]
    so the first isomorphism theorem gives us $(R/I)/(J/I) \isom R/J$.
    
\end{proof}

\end{document}