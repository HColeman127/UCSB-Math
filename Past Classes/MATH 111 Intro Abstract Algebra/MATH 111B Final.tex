\documentclass[12pt]{article}

% Packages
\usepackage[margin=1in]{geometry}
\usepackage{fancyhdr}
\usepackage{amsmath, amsthm, amssymb, physics}

% Page Style
\fancypagestyle{plain}{
    \fancyhf{}
    \renewcommand{\headrulewidth}{0pt}
    \renewcommand{\footrulewidth}{0pt}
    \fancyfoot[R]{\thepage}
}
\pagestyle{plain}

% Problem Box
\setlength{\fboxsep}{4pt}
\newsavebox{\savefullbox}
\newenvironment{fullbox}{\begin{lrbox}{\savefullbox}\begin{minipage}{\dimexpr\textwidth-2\fboxsep\relax}}{\end{minipage}\end{lrbox}\begin{center}\framebox[\textwidth]{\usebox{\savefullbox}}\end{center}}
\newenvironment{pbox}[1][]{\begin{fullbox}\ifx#1\empty\else\paragraph{#1}\fi}{\end{fullbox}}

% Options
\renewcommand{\thesubsection}{\thesection(\alph{subsection})}
\allowdisplaybreaks
\addtolength{\jot}{4pt}
\theoremstyle{definition}

% Default Commands
\newtheorem{proposition}{Proposition}
\newtheorem{lemma}{Lemma}
\newcommand{\ds}{\displaystyle}
\newcommand{\isp}[1]{\quad\text{#1}\quad}
\newcommand{\N}{\mathbb{N}}
\newcommand{\Z}{\mathbb{Z}}
\newcommand{\Q}{\mathbb{Q}}
\newcommand{\R}{\mathbb{R}}
\newcommand{\C}{\mathbb{C}}
\newcommand{\eps}{\varepsilon}
\renewcommand{\phi}{\varphi}
\renewcommand{\emptyset}{\varnothing}
\newcommand{\pfrac}[2]{\left(\frac{#1}{#2}\right)}

% Extra Commands
\DeclareMathOperator{\Hom}{Hom}

% Document Info
\fancypagestyle{title}{
    \renewcommand{\headrulewidth}{0.4pt}
    \setlength{\headheight}{15pt}
    \fancyhead[R]{Harry Coleman}
    \fancyhead[L]{MATH 111B Final}
    \fancyhead[C]{March 15, 2021}
}

% Begin Document
\begin{document}
\thispagestyle{title}


\section*{1(a)}

Yes. Suppose $(x, y), (u, v) \in M \times M$. Then
\begin{align*}
    \phi((x,y) + (u,v)) 
        &= \phi(x+u, y+v) \\
        &= (x+u) - (y+v) \\
        &= x + u - y - v \\
        &= x - y + u - v \\
        &= \phi(x, y) + \phi(u, v).
\end{align*}
And for $r \in R$ and $(x, y) \in M \times M$, we find
\begin{align*}
    \phi(r(x, y))
        &= \phi(rx, ry) \\
        &= rx - ry \\
        &= r(x - y) \\
        &= r\phi(x, y).
\end{align*}

\section*{1(b)}

No. Consider $1, x \in \Q[x]$. First, we have
\[
    \phi(x1)
        = \phi(x)
        = \dv{x} x
        = 1.
\]
However,
\[
    x\phi(1)
        = x\dv{x}1
        = x0
        = 0.
\]
Thus, $\phi(x1) \ne x\phi(1)$, so $\phi$ is not a $\Q[x]$-module homomorphism.


\section*{2}

Yes. It is a field if and only if the ideal is maximal in $\Q[x]$. Since $\Q[x]$ is a UFD, ideals are maximal if and only if prime, and elements are prime if and only if irreducible. The polynomial has integer coefficients with GCD 1, so it is irreducible in $\Q[x]$ if and only if it is irreducible in $\Z[x]$. Since it is monic and $3$ divide all but the leading coefficient, but $3^2$ does not divide the constant term, then by Eisenstein's criterion, the polynomial is irreducible in $\Z[x]$. Hence, the quotient ring is a field.


\newpage
\section*{3}

Here, all integers implicitly represent their equivalence class mod $11$. First, we find the characteristic polynomial.
\begin{align*}
    c_A(x)
        &= \det(xI_3 - A) \\
        &= \det\mqty[x - 1 & -2 & 0 \\ -3 & x - 4 & -5 \\ -2 & 0  & x + 1] \\
        &= (x-1)\det\mqty[x - 4 & -5 \\ 0 & x + 1] - (-2)\det\mqty[-3 & -5 \\ -2 & x + 1] \\
        &= (x-1)((x-4)(x+1) - (-5)0) + 2(-3(x+1) - (-5)(-2)) \\
        &= (x-1)(x-4)(x+1) + 2(-3x - 13) \\
        &= (x^2 - 1)(x - 4) - 6x - 26 \\
        &= x^3 - 4x^2 - 7x + 11 \\
        &= x(x^2 - 4x - 7) \\
        &= x(x^2 - 4x + 4) \\
        &= x(x - 2)^2.
\end{align*}
The possibilities for the minimal polynomial are therefore $x(x - 2)^2$ or $x(x - 2)$. We check if the latter evaluates to zero at $A$.
\[
    A(A - 2I_3)
        = \mqty[1 & 2 & 0 \\ 3 & 4 & 5 \\ 2 & 0 & -1]\mqty[-1 & 2 & 0 \\ 3 & 2 & 5 \\ 2 & 0 & -3] \\
        = \mqty[5 & * & * \\ * & * & * \\ * & * & *] \\
        \ne 0.
\]
Therefore, the minimal polynomial is $m_A(x) = x(x - 2)^2$.

\section*{3(a)}

The invariant factor is $x(x - 2)^2$.


\section*{3(b)}

The elementary divisors are $x$ and $(x - 2)^2$.

\section*{3(c)}

The Jordan canonical form is $\mqty[2 & 1 & 0\\ 0 & 2 & 0 \\ 0 & 0 & 0]$.



\newpage
\section*{4(a)}

Let $s \in R$ and $x, y \in M$. Then
\begin{align*}
    \phi_r(sx + y)
        &= r(sx + y) \\
        &= rsx + ry \\
        &= srx + ry \\
        &= s\phi_r(x) + \phi_r(y).
\end{align*}
Since $1 \in R$, this proves $\phi_r$ is an $R$-module homomorphism.

\section*{4(b)}

Let $r, s, t \in R$. We want to show that $f(rs + t) = rf(s) + f(t)$, i.e., that
\[
    \phi_{rs + t} = r\phi_s + \phi_t.
\]
Let $x \in M$, then
\begin{align*}
    \phi_{rs + t}(x)
        &= (rs + t)(x) \\
        &= rsx + tx \\
        &= r\phi_s(x) + \phi_t(x) \\
        &= (r\phi_s + \phi_t)(x).
\end{align*}
Hence, $f$ is an $R$-module homomorphism.

\section*{4(c)}

Let $x \in M$ such that $M = Rx$ and $\phi \in \Hom_R(M, M)$. Since $M$ is cyclic, then for some $r \in R$ we have $rx = \phi(x)$. We claim that $f(r) = \phi_r = \phi$. Let $sx \in M$ (arbitrary element since $M = Rx$), then
\begin{align*}
    \phi_r(sx)
        &= rsx \\
        &= srx \\
        &= s\phi(x) \\
        &= \phi(sx).
\end{align*}
Hence, $f(r) = \phi$, so $f$ is surjective.


\newpage
\section*{5}

Let $I \subseteq R$ be an ideal. Since $I$ is a free $R$-module, then there exists a basis $\{x_1, \dots, x_n\}$ for $I$, with $x_1, \dots, x_n \in R$ nonzero. Suppose for contradiction that $n > 1$, so $x_1, x_2 \in R$ nonzero. Then we have the $R$-linear combination of basis elements
\[
    (x_2)x_1 + (-x_1)x_2 + 0x_3 + \cdots + 0x_n = x_1x_2 - x_1x_2 = 0.
\]
This implies that all the coefficients are zero, so $x_2 = -x_1 = 0$. This is a contradiction, since all basis elements are assumed nonzero. Therefore, $\{x_1\}$ is a basis for $I$, meaning that $I = Rx_1 = (x_1)$. Hence, all ideals of $R$ are principal, so $R$ is a PID.

\section*{6(a)}

Since $\phi$ is an $R$-module homomorphism, its image $\phi(M) \subseteq M$ is an $R$-submodule of $M$. Since $M$ is irreducible, this implies that $\phi(M) = 0$ or $\phi(M) = M$. Since $\phi$ is nonzero, then we must have $\phi(M) = M$, i.e., $\phi$ is surjective. Now, $\ker\phi \subseteq M$ is also an $R$-submodule of $M$, so we must have $\ker\phi = 0$ or $\ker\phi = M$. Since $\phi$ is nonzero, then we cannot have $\ker\phi = M$, as that would imply $\phi(M) = 0$. Therefore, $\ker\phi = 0$, so $\phi$ is injective. Thus, $\phi$ is a bijective $R$-module homomorphism, so it is an $R$-module isomorphism.

\section*{6(b)}

Since $V$ is irreducible it has only itself and $0$ as $\C[x]$-submodules. The $\C[x]$-submodules correspond bijectively to the $T_A$-stable subspaces. That is, the only $T_a$-stable subspaces are $0$ and $V$, so must have $A = 0$? If $A = \lambda I_n$ with $\lambda$ nonzero, then the span of any basis vector would be $T_A$-stable, so maybe I'm missing something.


\end{document}