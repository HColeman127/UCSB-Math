\documentclass[12pt]{article}

% Packages
\usepackage[margin=1in]{geometry}
\usepackage{fancyhdr, parskip}
\usepackage{amsmath, amsthm, amssymb}
\usepackage{tikz, tikz-cd}
\usepackage[shortlabels]{enumitem}

% Page Style
\makeatletter
\fancypagestyle{title}{
    \renewcommand{\headrulewidth}{0.4pt}
    \setlength{\headheight}{15pt}
    \fancyhead[R]{\@author}
    \fancyhead[L]{\@title}
    \fancyhead[C]{\@date}
}
\makeatother
\renewcommand{\maketitle}{\thispagestyle{title}}
\fancypagestyle{plain}{
    \fancyhf{}
    \renewcommand{\headrulewidth}{0pt}
    \renewcommand{\footrulewidth}{0pt}
    \fancyfoot[R]{\thepage}
}
\pagestyle{plain}

% Problem Box
\setlength{\fboxsep}{4pt}
\newlength{\myparskip}
\setlength{\myparskip}{\parskip}
\newsavebox{\savefullbox}
\newenvironment{fullbox}{\begin{lrbox}{\savefullbox}\begin{minipage}{\dimexpr\textwidth-2\fboxsep\relax}\setlength{\parskip}{\myparskip}}{\end{minipage}\end{lrbox}\framebox[\textwidth]{\usebox{\savefullbox}}}
\newenvironment{pbox}[1][]{\begin{fullbox}\def\temp{#1}\ifx\temp\empty\else\paragraph{#1}\phantom{}\fi}{\end{fullbox}}

% Theorem Environments
\theoremstyle{definition}
\newtheorem{lemma}{Lemma}

% Tikz Environments
\newenvironment{drawing}{\begin{center}\begin{tikzpicture}}{\end{tikzpicture}\end{center}}
% \tikzcdset{row sep/normal=0pt}
\newenvironment{cd}{\begin{center}\begin{tikzcd}}{\end{tikzcd}\end{center}}

% Default Commands
\newcommand{\isp}[1]{\quad\text{#1}\quad}
\newcommand{\N}{\mathbb{N}} 
\newcommand{\Z}{\mathbb{Z}}
\newcommand{\Q}{\mathbb{Q}}
\newcommand{\R}{\mathbb{R}}
\newcommand{\C}{\mathbb{C}}
\newcommand{\A}{\mathbb{A}}
\renewcommand{\P}{\mathbb{P}}
\newcommand{\eps}{\varepsilon}
\renewcommand{\phi}{\varphi}
\renewcommand{\emptyset}{\varnothing}
\newcommand{\<}{\langle}
\renewcommand{\>}{\rangle}
\newcommand{\iso}{\cong}
\newcommand{\eqc}{\overline}
\newcommand{\clo}{\overline}
\renewcommand{\tilde}{\widetilde}
\renewcommand{\hat}{\widehat}
\newcommand{\seq}{\subseteq}
\newcommand{\teq}{\trianglelefteq}
\DeclareMathOperator{\id}{id}
\DeclareMathOperator{\im}{im}
\newcommand{\inc}{\hookrightarrow}
\newcommand{\dd}{\mathrm{d}}
\newcommand{\mat}[1]{\begin{bmatrix}#1\end{bmatrix}}

% Extra Commands
\newcommand{\CP}{\mathbb{C}\mathrm{P}}

% Document
\begin{document}
\title{MATH 240A Midterm}
\author{Harry Coleman}
\date{October 30, 2022}
\maketitle

\section*{1}

Let $X = \C^2 \setminus \{0\}$.

Consider the projection $\pi : X \to \CP^1$.

Since $X$ is second-countable, it is clear that $\CP^1$ is as well.

We check that $\pi$ is an open map.
Let $U \seq X$ be open, then
\[
    \pi^{-1}(\pi(U))
        = \bigcup_{z \in U} [z]
        = \bigcup_{z \in U} \bigcup_{\lambda \ne 0} \{\lambda z\}
        = \bigcup_{\lambda \ne 0} \bigcup_{z \in U} \{\lambda z\}
        = \bigcup_{\lambda \ne 0} \lambda U.
\]
For $\lambda \ne 0$, the map $z \mapsto \lambda z$ is a homeomorphism of $X$ to itself.
Therefore, $\lambda U$ as the image of an open set is open.
Hence, $\pi^{-1}(\pi(U))$ is open, and we conclude that $\pi(U)$ is open.

We check that $\CP^1$ is Hausdorff.

Define a relation on $X$ by $z \sim w$ whenever $\pi(z) = \pi(w)$.
Let $R = \{(z, w) \mid z \sim w\} \seq X \times X$ be the set of pairs of points that are identified under the projection $\pi$.
Equivalently, $z \sim w$ if and only if $z^1w^2 = z^2w^1$.
Consider the polynomial map
\begin{align*}
    f : X \times X &\longrightarrow \C, \\
        (z, w) &\longmapsto z^1w^2 - z^2w^1.
\end{align*}
This map is continuous, so its zero locus $R = f^{-1}(0)$ is closed in $X \times X$.
It follows that the quotient space $\CP^1 = X/{\sim}$ is Hausdorff.

Define the open cover $\{U_1, U_2\}$ of $\CP^1$ as usual.
Let $\phi_1 : [1 : w] \mapsto w$ and $\phi_2 : [z : 1] \to z$ be the usual charts.
We check that these charts are smoothly compatible:
\[
    \phi_2 \circ \phi_1^{-1}(w)
        = \phi_2([1 : w])
        = \phi_2([\tfrac{1}{w} : 1])  
        = \tfrac{1}{w}
\]
is a smooth map $\phi_1(U_1 \cap U_2) = \C \setminus 0 \to \phi_2(U_1 \cap U_2) = \C \setminus \{0\}$ and 
\[
    \phi_1 \circ \phi_2^{-1}(z)
        = \phi_1([z : 1])
        = \phi_1([1 : \tfrac{1}{z}])  
        = \tfrac{1}{z}
\]
is a smooth map $\phi_2(U_1 \cap U_2) = \C \setminus 0 \to \phi_1(U_1 \cap U_2) = \C \setminus \{0\}$.
Hence, we have found a smooth structure on $\CP^1$.



\newpage

\section*{2}

We claim that the differential $i_* : T_pS^n \to T_p\R^{n+1} \iso \R^{n+1}$ is injective.
The radial projection $\pi : \R^{n+1} \setminus \{0\} \to S^n$ is a retraction, i.e., $\pi \circ \iota = \id$.
Then functorality of the differential gives us $\pi_* \circ \iota_* = \id$, hence $\iota_*$ is injective.


\newpage

\section*{3}

\subsection*{a}

Say $c \in I$ is a nonzero constant function and $f \in C^\infty(M)$ is arbitrary.
Then $\frac{1}{c} \in C^\infty(M)$ is a nonzero constant function and therefore
\[
    f = \frac{f}{c} \cdot c \in I.
\]
Hence, $I = C^\infty(M)$.

Now suppose $f \notin I_p$, i.e., $f(p) \ne 0$.
Then $g = f(p) - f \in I_p$.
However, $g + f = f(p)$ is a nonzero constant function so $I_p + \<f\>$ must be all of $C^\infty(M)$.
In other words, $I_p$ is a maximal ideal.


\subsection*{b}

Assume for contradiction that $I \teq C^\infty(M)$ is a maximal ideal which is not of the form $I_p$.
In particular, $I \nsubseteq I_p$ for all $p \in M$.
For each $p \in M$, let $f_p \in I$ be such that $f_p(p) \ne 0$.
We can choose some open neighborhood $U_p$ of $p$ on which $f_p$ is nonzero, e.g., $f_p^{-1}(B_\eps(f_p(p))$ for small $\eps > 0$.
Then $\{U_p\}$ is an open cover of $M$ and we can select a finite subcover, indexed by $p_1, \dots, p_k$.
Now $f_{p_i}^2 \in I$ is nonnegative function which is positive on at least $U_p$, so the sum $f = \sum_{i=1}^{k} f_{p_i}^2 \in I$ is strictly positive on all of $M$.
But then $\frac{1}{f} \in C^\infty(M)$, so the constant function $1 \equiv \frac{1}{f} \cdot f$ is an element of $I$.
By part (a), we conclude that $I = C^\infty(M)$, which is a contradiction.

\end{document}