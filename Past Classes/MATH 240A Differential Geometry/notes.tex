\documentclass[12pt]{article}

% Packages
\usepackage[margin=1in]{geometry}
\usepackage{parskip}
\usepackage{amsmath, amsthm, amssymb}
\usepackage{tikz, tikz-cd}
\usepackage[shortlabels]{enumitem}

\usepackage{bbm}

\usepackage{suffix}
\usetikzlibrary{decorations.pathmorphing}

% Problem Box
\setlength{\fboxsep}{4pt}
\newlength{\myparskip} 
\setlength{\myparskip}{\parskip}
\newsavebox{\savefullbox}
\newenvironment{fullbox}{\begin{lrbox}{\savefullbox}\begin{minipage}{\dimexpr\textwidth-2\fboxsep\relax}\setlength{\parskip}{\myparskip}}{\end{minipage}\end{lrbox}\framebox[\textwidth]{\usebox{\savefullbox}}}

% Environments
\setlist[enumerate]{nosep}
\newcommand{\keyword}[1]{\textbf{#1}}
\newcommand{\sepline}{\rule{\textwidth}{0.4pt}}

% Tikz Environments
\newenvironment{drawing}{\begin{center}\begin{tikzpicture}}{\end{tikzpicture}\end{center}}
% \tikzcdset{row sep/normal=0pt}
\newenvironment{cd}{\begin{center}\begin{tikzcd}}{\end{tikzcd}\end{center}}


% Document Formatting
\newtheoremstyle{mythmstyle}% name of the style to be used
{ }% measure of space to leave above the theorem. E.g.: 3pt
{ }% measure of space to leave below the theorem. E.g.: 3pt
{ }% name of font to use in the body of the theorem
{ }% measure of space to indent
{\scshape}% name of head font
{.}% punctuation between head and body
{ }% space after theorem head; " " = normal interword space
{\thmname{#1}\thmnumber{ #2}\thmnote{ (#3)}}% Manually specify head

\theoremstyle{definition}
\newtheorem{theorem}{Theorem}
\newtheorem{corollary}{Corollary}
\newtheorem{lemma}{Lemma}
\newtheorem{proposition}{Proposition}
\newtheorem{claim}{Claim}


% Math Formatting
\newcommand{\isp}[1]{\quad\text{#1}\quad}

% mathbb
\newcommand{\N}{\mathbb{N}}
\newcommand{\Z}{\mathbb{Z}}
\newcommand{\Q}{\mathbb{Q}}
\newcommand{\R}{\mathbb{R}}
\newcommand{\C}{\mathbb{C}}
\renewcommand{\k}{\mathbbm{k}}
\renewcommand{\P}{\mathbb{P}}

% mathcal
\newcommand{\LL}{\mathcal{L}}
\newcommand{\FF}{\mathcal{F}}
\newcommand{\UU}{\mathcal{U}}

% Symbols
\newcommand{\eps}{\varepsilon}
\renewcommand{\phi}{\varphi}
\renewcommand{\emptyset}{\varnothing}

% Delimiters
\newcommand{\<}{\left\langle}
\renewcommand{\>}{\right\rangle}

% Relations
\newcommand{\iso}{\cong}
\newcommand{\seq}{\subseteq}

\newcommand{\inc}{\hookrightarrow}
\newcommand{\To}{\longrightarrow}
\newcommand{\Mapsto}{\longmapsto}

\newcommand{\csub}{\subset\joinrel\subset}


% Math Roman
\newcommand{\dd}{\mathrm{d}}
\newcommand{\DD}{\mathrm{D}}

\DeclareMathOperator{\im}{im}
\newcommand{\GL}{\mathrm{GL}}

\newcommand{\id}{\mathrm{id}}
\newcommand{\proj}{\mathrm{proj}}
\DeclareMathOperator{\supp}{supp}
\DeclareMathOperator{\inter}{int}
\DeclareMathOperator{\rank}{rank}
\DeclareMathOperator{\Jac}{Jac}

% Other
\newcommand{\pdv}[2]{\frac{\partial #1}{\partial #2}}
\newcommand{\odv}[2]{\frac{\dd #1}{\dd #2}}
\newcommand{\eval}[1]{\left.#1\right|}

\newcommand{\mat}[1]{\begin{bmatrix}#1\end{bmatrix}}


\newcommand{\clo}{\overline}
\newcommand{\conj}{\overline}
\renewcommand{\hat}{\widehat}
\renewcommand{\tilde}{\widetilde}

\newcommand{\tsum}{{\textstyle\sum}}



\title{Differential Geometry \\
    \large Fall 2022
}
\author{}
\date{}


\begin{document}
\maketitle

this requires some topology

\sepline

\section*{9/22/2022}

Geometry: shape of spaces

Differential: use differential calculus

We wil begin by looking at smooth manifolds

\sepline

Topological manifolds are spaces built with Euclidean spaces, i.e., a topological space which is locally Euclidean

Example: a line with two origins.
\begin{drawing}
    \draw (-2, 0) -- (-0.1, 0);
    \draw (0.1, 0) -- (2, 0);

    \fill (0, 0.1) circle (1pt);
    \fill (0, -0.1) circle (1pt);
\end{drawing}
This space is locally Euclidean but not Hausdorff (since the distinct origins are not separated by disjoint open neighborhoods).


An $n$-dimensional \keyword{topological manifold} $M$ is topological space which is
\begin{itemize}
    \item Hausdorff,
    \item second countable;
    \item locally Euclidean, i.e., for any point $p \in M$ there exists an open neighborhood $p \in U \seq M$ and open set $\hat{U} \seq \R^n$ with a homeomorphism $\phi : U \to \hat{U}$.
\end{itemize}

\sepline

Hausdorff

A topological space $X$ is \keyword{Hausdorff} if for any $p, q \in X$ with $p \ne q$ there exist open neighborhood $p \in U \seq X$ and $q \in V \seq X$ such that $U \cap V = \emptyset$.


$\R^n$ is Hausdorff metric space by using balls.

Hausdorff is hereditary: any subspace of a Hausdorff space is Hausdorff.

In particular, any subspace of Euclidean space is Hausdorff.

\sepline

Second Countable

A topological space is \keyword{second countable} if it has a countable basis/base.

$\R^n$ is second countable by open balls of rational center and radius.

Second countability is hereditary.

In particular, any subspace of Euclidean space is second countable.

\sepline

Example: $(n-1)$-sphere as subspace $S^{n-1} = \{x_1^2 + \cdots + x_n^2 = 1\} \seq \R^n$.
By heritability, have Hausdorff and second countable.
Check locally Euclidean by projecting hemispheres onto flat $(n-1)$-discs.

Take $U_i^{\pm} = \{x \in S^{n-1} \mid x_i \gtrless 0\}$ then
\begin{align*}
    \phi_i^\pm : U_i^\pm &\To \hat{U}_i = \{(y_1, \dots, y_{n-1}) \in \R^{n-1} \mid \textstyle{\sum} y_i^2 < 1\}, \\
    (x_1, \dots, x_n) &\Mapsto (x_1, \dots, \hat{x}_i, \dots, x_n).
\end{align*}

Non-Example:
\begin{drawing}
    \draw (-1.5, 0) -- (0, 0) -- (1, 1);
    \draw (0, 0) -- (1, -1);
    \fill (0, 0) circle (2pt);
\end{drawing}

\sepline

Remarks

Topological properties such as Hausdorff-ness or second countability are fundamentally ``global'' in nature.
In other words, they have to do with the whole space and cannot be deduced from the ``local'' property of being locally Euclidean.

Topological conditions ensure that our spaces are ``nice'':
\begin{itemize}
    \item Hausdorff: finite sets are closed and compact sets are closed.
    \item second countable: go from local to global.
\end{itemize}

Any open set of $\R^n$ is a topological manifold, e.g.,
\[\textstyle
    \GL_n(\R)
        = \det^{-1}(\R \setminus \{0\})
        \seq M_n(\R)
        = \R^{n^2}.
\]
So $\GL_n(\R)$ is a topological manifold, and it is also a group, i.e., a Lie group.

\sepline

Let $M$ be a topological manifold.

Each point $p \in M$ has an open neighborhood $U \seq M$ and an open set $\hat{U} \seq \R^n$ with a homeomorphism $\phi : U \to \hat{U}$.
Call $(U, \phi)$ (or simply $\phi$) a \keyword{chart}.

A collection of charts $\{(U_\alpha, \phi_\alpha)\}$ is called an \keyword{atlas} if $\{U_\alpha\}$ covers $M$.

Given an atlas $\{(U_\alpha, \phi_\alpha)$, the \keyword{transition maps}/\keyword{coordinate changes} are the maps
\[
    \phi_\beta \circ \phi_\alpha^{-1} : \phi_\alpha(U_\alpha \cap U_\beta) \To \phi_\beta(U_\alpha \cap U_\beta).
\]

\sepline

\section*{9/27/22}

Remark: under translation and scaling, we can assume that a chosen point $p \in U \seq M$ is sent to the origin and that $B_1(0) \seq \hat{U}$.
Because of this, we will call $\phi$ a chart at $p$.

\sepline

Smooth Manifolds

Let $M$ be a topological manifold.

Two charts $(U_\alpha, \phi_\alpha)$ and $(U_\beta, \phi_\beta)$ are said to be \keyword{smoothly compatible} (or $C^\infty$ compatible) if whenever $U_\alpha \cap U_\beta \ne \emptyset$ the transition map $\phi_\beta \circ \phi_\alpha^{-1}$ is smooth (in $C^\infty$).
Moreover, $\phi_\alpha \circ \phi_\beta^{-1}$ is the smooth inverse.

\sepline

Example (bad)

Two charts $(\R, \phi)$ and $(\R, \psi)$ with $\phi(x) = x$ and $\psi(x) = x^3$.
Then
\[
    \phi \circ \psi^{-1}(y) = y^{-1/3}.
\]
This is not differentiable at $y = 0$.

\sepline

Roughly speaking, an atlas of charts which are pairwise smoothly compatible determines a smooth structure (on the topological manifold), which makes it into a smooth manifold.

A \keyword{smooth structre} (or $C^\infty$ or differentiable structure) on a topological manifold $M$ consists of an atlas $\UU = \{(U_\alpha, \phi_\alpha)\}$ such that
\begin{itemize}
    \item charts in $\UU$ are pairwise smoothly compatible ($\UU$ is a \keyword{smooth atlas});
    \item $\UU$ is maximal in the sense that if $(U, \phi)$ is a chart on $M$ which is smoothly compatible with all members of $\UU$, then it must be in $\UU$.
\end{itemize}



A \keyword{smooth manifold} is a topological manifold with a smooth structure.

\sepline

Example

$S^n$ with $\UU = \{(U_i^\pm, \phi_i^\pm)\}$ is pairwise smoothly compatible, but not maximal.

To make it maximal, put $\tilde{\UU} = \{(U, \phi) \text{ smoothly compatible with } \UU\}$.

\sepline

\begin{proposition}[1.17]
    Let $M$ be a topological manifold.
    Every smooth atlas on $M$ is contained in a unique maximal smooth atlas.
\end{proposition}

\begin{proposition}
    Given $\UU$, make $\tilde{\UU}$.
    If this is smoothly compatible, it is clearly maximal.
    For $(U, \phi), (V, \psi) \in \tilde{U}$, we have
    \[
        \phi \circ \psi^{-1} = (\phi \circ \phi_\alpha) \circ (\phi_\alpha^{-1} \circ \psi^{-1}),
    \]
    which is clearly smooth.
\end{proposition}


From now on, we specify a smooth structure by specifying a smooth atlas 


\sepline

$S^n$ with $\UU = \{(U_i^\pm, \phi_i^\pm)\}$ is standard smooth structure.

Stereographic projection gives the same smooth structure.

\sepline

$(\R, \id_\R)$ is a smooth structure.

$(\R, x^3)$ is also a smooth structure.

$(\R, x^{2n+1})$ is in fact a smooth structure.

Still too much redundancy.

\sepline

Any topological manifold with a single chart is a smooth manifold.

e.g.\ $\GL_n(\R) \seq M_n(\R) = \R^{n^2}$.

Also, $F : \R \to \R, x \mapsto |x|$.
Take graph $\Gamma = \{(x, F(x)) : x \in \R\}$. then projection $(x, F(x)) \mapsto x$ is a chart $\phi : \Gamma \to \R$.
Turns $\Gamma$ into a smooth manifold.
However, considered as sitting inside $\R^2$, $\Gamma$ is not smooth because of the corner at the origin.

\sepline

Sphere $S^n$ can be covered with two charts.


\sepline

\section*{9/29/22}

insert beginning notes

\sepline

Goal: show $\R\P^n$ is smooth manifold.

\begin{lemma}
    Quotient map $\pi : \R^{n+1} \setminus \{0\} \to \R\P^n$ is an open mapping.
\end{lemma}

\begin{proof}
    Let $U \seq \R^{n+1} \setminus \{0\}$ be open.

    Want $\pi(U) \seq \R\P^n$ to be open.

    i.e.\ $\pi^{-1}(\pi(U)) \seq \R^{n+1} \setminus \{0\}$ is open.

    To see this, we note
    \[
        \pi^{-1}(\pi(U))
            = \bigcup_{x \in U} [x]
            = \bigcup_{x \in U} \bigcup_{\lambda \neq 0} \{\lambda x\}
            = \bigcup_{\lambda \ne 0} \bigcup_{x \in U} \{\lambda x\}
            = \bigcup_{\lambda \ne 0} \lambda U.
    \]
    Since $\lambda \ne 0$, then $\lambda \cdot -$ is a homeomorphism, so $\lambda U$ is open hence $\pi^{-1}(\pi(U)) = \bigcup_{\lambda \ne 0} \lambda U$ is also open.
\end{proof}


\begin{lemma}
    If $X$ a top space with equivalence relation $\sim$, then the quotient space $X/{\sim}$ is Hausdorff if and only if the set $R = \{(x, y) \mid x \sim y\}$ is closed in $X \times X$ (with the box/product topology).
\end{lemma}

\begin{proof}
    ($\impliedby$)

    Assume $R$ closed, so $R^C$ open.

    For all $[x], [y] \in X/{\sim}$, have $[x] \ne [y]$ iff $(x, y) \notin R$.

    Then can find box neighborhood $(x, y) \in U \times V \seq R^C$.

    Then $\pi(U) \cap \pi(V) = \emptyset$, hence $X/{\sim}$ is Hausdorff.

\end{proof}

Given this lemma, we now see $\R\P^n$ is Hausdorff.

Let $R = \{(x, y) \mid x \sim y\}$.
Note $x \sim y$ if and only if $y = \lambda x$ for $\lambda \ne 0$.

In this case, $y_i = \lambda x_i$, so condition is equivalent to $x_jy_i = y_jx_i$ for all $i, j$.

Hence, $R$ is the zero locus of a polynomial (in particular, continuous) function $X \times X \to \R$, hence it is closed.
Say $F(x, y) = \sum_{i, j} (x_iy_j - x_jy_i)^2$, then $R = F^{-1}(0)$.

Hence, $\R\P^n$ is Hausdorff.

\sepline

Finding smooth atlas on $\R\P^n$.

Example: $\R\P^1 = \R^2 \setminus \{0\} / {\sim}$.

Can represent point $[x_1 : x_2] \in \R\P^1$ with slope $x_2/x_1$ provided $x_1 \ne 0$.
Cover with two charts.

Use open sets $U_i = \{[x_1 : x_2] \mid x_i \ne 0\}$ for $i = 1, 2$.

Then maps $\phi_1 : [x_1 : x_2] \mapsto x_2/x_1$ and $\phi_2 : [x_1 : x_2] \mapsto x_1/x_2$.

Can check that $\phi_i$'s are smoothly compatible, so this is a smooth atlas.

For $\R\P^n$, we generalize it.

\sepline

What does $\R\P^n$ look like?

Have $\R\P^n = \R^{n+1} \setminus \{0\} / {\sim}$, but could also consider $\R\P^n = S^n/\{x \sim -x\}$.

e.g.\ can only embed $\R\P^2$ in $\R^4$, not $\R^3$.


\sepline

Let $M^n$ be a smooth manifold.

A function $f : M \to \R$ is \keyword{smooth} if for any point $p \in M$ there there is a smooth chart $(U, \phi)$ at $p$ such that $f \circ \phi^{-1} : \phi(U) \to \R$ is a smooth map (in the Euclidean sense, as $\phi(U) \seq \R^n$).

\begin{cd}
    U \dar["\phi"'] \rar["f"] & \R \\
    \hat{U} \urar[dashed]
\end{cd}

Note that if $(V, \psi)$ is another smooth chart at $p$, then can write $f \circ \psi = (f \circ \phi^{-1}) \circ (\phi \circ \psi^{-1})$, which is smooth.
So this definition is independent of the choice of chart.

\sepline

Example

$M$ smooth manifold and $(U, \phi)$ a smooth chart.
Then $U \seq M$ is open, and therefore inherits a smooth manifold structure; in fact, $U$ has the global chart $(U, \phi)$.

Have $\phi : U \to \phi(U) \seq \R^n$, can define $f: U \to \R$ by
\begin{cd}[row sep=tiny]
    U \rar["\phi"] & \phi(U) \seq \R^n \rar["\pi_i"] & \R, \\
    & (x_1, \dots, x_n) \rar[mapsto] & x_i.
\end{cd}
Then $f_i$ defines a smooth function on $U$ since $f_i \circ \phi^{-1}$ is smooth.


\sepline
\section*{10/4/22}

$M^n$ smooth, $(U, \phi)$ smooth chart

\begin{cd} 
    U \rar["\phi"] \ar[rr, bend right=30, "f_i"'] & \phi(U) \seq \R^n \rar["\proj_i"] & \R
\end{cd}

$f_i$ is a smooth function on $U$--local coordinate function.

$\phi = (f_1, \dots, f_n)$.

Remark: $C^\infty(M)$ has addition, multiplication, and scalar multiplication; it is a vector space algebra.
It will turn out that smooth structure can be characterized by the algebra on this space.


\sepline

Smooth Maps

Let $M$ and $N$ be smooth manifolds.

A map $F : M \to N$ is called \keyword{smooth} (or $C^\infty$) if for every point $p \in M$ there are charts $(U, \phi)$ at $p$ and $(V, \psi)$ with at $F(p)$ such that $F(U) \seq V$ and $\hat{F} = \psi \circ F \circ \phi^{-1} : \hat{U} \to \hat{V}$ is smooth.
\begin{cd}
    U \rar["F"] \dar["\phi"'] & V \dar["\psi"] \\
    \hat{U} \rar[dashed, "\hat{F}"'] & \hat{V}
\end{cd}

Call $\hat{F}$ a \keyword{local representative} (or representation?).

\sepline

Examples

A function $f : M \to \R$ is smooth as a function if and only if it is smooth as a map between manifolds.

Every constant map is smooth.

Identity map is smooth.

Projection $M \times N \to M$ is smooth, where product has obvious smooth structure.

\sepline

Example

Take $F : S^2 \to \C\P^1$ defined by
\[
    F(x_1, x_2, x_3) = \begin{cases}
        [1 + x_1 : x_2 + ix_3] &\text{if } x_1 \ne -1, \\
        [x_2 - ix_3 : 1 - x_1] &\text{if } x_1 \ne 1.
    \end{cases}
\]
Check well-defined: $x_1 \ne \pm1$ implies $[1 + x_1 : x_2 + ix_3] = [x_2 - ix_3, 1 - x_1]$.
Cross multiply:
\[
    (1 + x_1)(1 - x_1) = 1 - x_1^2 = x_2^2 + x_3^2 = (x_2 + ix_3)(x_2 - ix_3).
\]

Check smooth.
Look at open set $U_1^+ = \{(x_1, x_2, x_3) \in S^2 \mid x_1 > 0\}$, with map $\phi_1^+ : U_1^+ \to \R^2$ sending points to $(x_2, x_3)$.

Consider $V_1 = \{[z_1 : z_2] \in \C\P^1 \mid z_1 \ne 0\}$ with map $\psi_1 : V_1 \to \C \iso \R^2$ sending points to $z_2/z_1$.

Then need to work out
\begin{align*}
    \psi_1 \circ F \circ (\phi_1^+)^{-1}(y_1, y_2)
        &= \psi_1 \circ F(\textstyle\sqrt{1 - y_1^2 - y_2^2}, y_1, y_2) \\
        & \vdots \\
        &= \frac{y_1 + iy_2}{1 + \sqrt{1 - y_1^2 - y_2^2}}
\end{align*}
This is smooth since $y_1^2 + y_2^2 < 1$.
Then others can be checked similarly.

Turns out that $F$ is bijective and $F^{-1} : \C\P^1 \to S^2$ is also smooth; hence $F$ is a diffeomorphism.

\sepline

A smooth map $F : M \to N$ is called a \keyword{diffeomorphism} if it is a bijection whose inverse $F^{-1} : N \to M$ is also smooth.

In such a case, we say $M$ and $N$ are \keyword{diffeomorphic} as manifolds.

\sepline

Example

Two different smooth structures on $\R$: 
\begin{itemize}
    \item $\R$ with $(\R, \phi = \id_\R)$
    \item $\tilde{\R}$ with $(\R, \psi(x) = x^3)$
\end{itemize}
Can define a diffeomorphism $F : \R \to \tilde{\R}, x \mapsto x^{1/3}$.

Check smooth:
\[
    \psi \circ F \circ \phi^{-1}(x)
        = \psi \circ F(x)
        = \psi(x^{1/3})
        = x.
\]
Can also check $F^{-1}$ is smooth.

So in fact, $\R$ and $\tilde{\R}$ are diffeomorphic (basically the same in the smooth world).

\sepline

Remark

Q: How should we count smooth structures?

Up to diffeomorphism.

Q: How many smooth structures are there up to diffeomorphism?

$\R^1$ has one.

$\R^n$: $n \leq 2$ unique (Radon), $n = 3$ unique (Moise), $n = 4$ uncountably many exotic (Donaldson theory), $n \geq 5$ unique (Stalling).


Q: What about compact manifolds?

Simplest: $S^n$

Holy Grail: classify

Smooth category $\subset$ Topological (continuous) category $\subset$ Homotopy category

Poincar\'e Conjecture: Every closed $n$-dimensional topological manifold homotopic to $S^n$ is actually homeomorphic to $S^n$.
For $n \geq 5$ yes (Smale 1966), $n = 4$ yes (Freedman 1986), $n = 3$ yes (Perelman 2003); all Field's medals.

Q: What about smooth categories?
i.e., how many smooth structures up to diffeomorphism can you find on $S^n$?

$S^7$ has 28 smooth structures (Milnor 1962, Field's medal).

Smooth Poincar\'e Conjecture: $S^4$? Field's medal up for grabs.


\sepline

\begin{lemma}
    $F : M \to N$ smooth implies $F$ is continuous.
    ($C^\infty \seq C^0$)
\end{lemma}

\begin{proof}
    Suffices to check locally.
    Given $p \in M$, then by definition there are charts $(U, \phi)$ and $(V, \psi)$ at $p$ and $F(p)$, respectively, such that $F(U) \seq V$ and $\hat{F} = \psi \circ F \circ \phi^{-1} : \hat{U} \to \hat{V}$ is smooth.
    
    Can write $F|_U = \psi^{-1} \circ \hat{F} \circ \phi$, where $\psi^{-1}$ is homeomorphism, $\hat{F}$ is smooth on $\R^n$ which is continuous, and $\phi$ is homeomorphism.
    Hence, $F|_U$ is continuous and so is all of $F$.
\end{proof}

\sepline

\section*{10/6/22}

Remark

For any $k = 0, 1, 2, \dots, \infty, \omega$ can similarly define $C^k$-structure on manifolds which only require up to $k$ times continuous differentiability.

$C^0$ is topological structure.

$C^\omega$ is real analytic structure.

Then $C^0 \supset C^k (k \geq 1) = C^\infty \supset C^\omega$.

Difference between $C^\infty$ and $C^\omega$ is partition of unity.

\sepline

Algebra of smooth functions $C^\infty(M)$.

Q: Can it separate points?

\sepline

Existence of bump functions: local Euclidean feature.

Define smooth function $f : \R \to \R$ by
\[
    f(t) = \begin{cases}
        e^{-1/t} & t > 0, \\
        0 & t \leq 0.
    \end{cases}
\]

Smooth cutoff function $h : \R \to [0, 1]$ with property that $h(t) = 1$ for $t \leq 1$ and $h(t) = 0$ for $t \geq 2$.
Construct it by
\[
    h(t) = \frac{f(2 - t)}{f(2 - t) + f(t - 1)}.
\]

Pass to smooth function $H : \R^n \to [0, 1]$ defined by $H(x) = h(\|x\|)$.
Has property that $H \equiv 1$ on $B_1(0)$ and $H \equiv 0$ outside $B_2(0)$.

Transplant to smooth manifold $M^n$.
For each point $p \in M$ there is a chart $(U, \phi)$ such that $\phi(p) = 0$ and $\phi(U) \supseteq B_3(0)$.
Construction bump function $H \circ \phi \in C^\infty(M, [0, 1])$ by setting to zero outside of $U$.

For $f \in C^\infty(M)$ define \keyword{support} of $f$ to be the closed set
\[
    \supp f = \clo{\{p \in M \mid f(p) \ne 0\}} \seq M.
\]

Then we have $\supp(H \circ \phi) \seq \phi^{-1}(\clo{B_2(0)}) \seq \phi^{-1}(B_3(0))$.

\sepline

\begin{theorem}
    $M^n$ a smooth manifold, $F \seq M$ closed, $K \csub M$ (compact) with $K \cap F = \emptyset$.
    Then there exists $f \in C^\infty(M, [0, 1])$ such that $f \equiv 1$ on $K$ and $f \equiv 0$ on $F$.
\end{theorem}

\begin{proof}
    By assumption, compact set $K$ is contained in open set $M \setminus F$.

    For all $P \in K$ there is a chart $(U_p, \phi_p)$ such that $\phi_p(U_p) \supseteq \clo{B_2(0)}$ and $U_p \seq M \setminus F$.

    Now look at open cover of $K$ by $\{\phi_p^{-1}(B_{1/2}(0))\}_{p \in K}$.
    Compactness gives us a finite subcover---call it $\{\phi_{p_i}^{-1}(B_{1/2}(0))\}_{i=1}^{k}$.

    Put $U_i = U_{p_i} \supseteq \phi_{p_i}^{-1}(B_{1/2}(0))$ and $\phi_i = \phi_{p_i}$.

    Define smooth function $f \in C^\infty(M, [0, 1])$ by
    \[
        f(x) = 1 - \prod_{i=1}^{k}(1 - H \circ \phi_i)(x).
    \]

    Remains to check $f \equiv 1$ on $K$ and $f \equiv 0$ on $F$.



\end{proof}

\sepline

Partition of Unity

$1 = \sum_\alpha f_\alpha$ with $f_\alpha \in C^\infty(M)$ bump functions.

For a function $f$ which we know how to do locally, then we can turn it into a global function by taking $f = f \cdot 1 = \sum_\alpha ff_\alpha$.

However, in general this sum may not be finite.
The next best thing is locally finite, i.e., near each point there are only finitely many terms.

Necessary condition for this to make sense:
\begin{enumerate}
    \item $\{\supp f_\alpha\}_\alpha$ forms and open cover of $M$
    \item to deal with summation, we demand that the sum has only finitely many terms locally
\end{enumerate}

\sepline

Let $X$ be a top space.

A collection of subsets $\{S_\alpha\}$ is called \keyword{locally finite} if for any $p \in X$ there is an open neighborhood $U$ such that $\#\{S_\alpha \mid S_\alpha \cap U \ne \emptyset\} < \infty$.

\sepline

Example

$\{f_\alpha\} \seq C^\infty(M)$.
Then $\{\supp f_\alpha\}$ is locally finite if and only if for any $p \in M$ there is an open neighborhood $U$ such that all but finitely many of the the $f_\alpha$'s vanish on $U$.

\sepline

Let $M$ be a smooth manifold and $\UU = \{U_\alpha$ an open cover of $M$.
A \keyword{partition of unity subordinate to $U$} is a collection of (smooth) bump functions $\{\psi_\alpha \in C^\infty(M)\}$ such that
\begin{enumerate}[(i)]
    \item $0 \leq \psi_\alpha \leq 1$,
    \item $\supp \psi_\alpha \seq U_\alpha$,
    \item $\{\supp \psi_\alpha\}$ is locally finite,
    \item $\sum_\alpha \psi_\alpha \equiv 1$.
\end{enumerate}


\begin{theorem}[Existence of Partitions of Unity]
    Given a (smooth) manifold and an open cover $\UU = \{U_\alpha\}$, there exists a (smooth) partition of unity subordinate to $\UU$.
\end{theorem}

Example: $p \ne q \in M$ with $U = M \setminus \{p\}$ and $V = M \setminus \{q\}$.
Then $\{U, V\}$ is open cover.
By theorem, there is a smooth partition of unity $\{\psi_1, \psi_2\}$
That is, $\supp \psi_1 \seq U$ implying $\psi_1(p) = 0$.
And similarly, $\psi_2(q) = 0$.
But $\psi_1 + \psi_2 \equiv 1$ implies $\psi_1(q) = 1$.

\sepline

\section*{10/11/22}

In definition of partition of unity, condition (iv) is equivalent to having strictly positive $\sum_\alpha \psi_\alpha > 0$.
Moreover, this is implied by having $\{\phi_\alpha^{-1}(\R_+)\}$ to be an open cover.

\begin{proof}
    First assume $M$ is compact.
    For all $x \in M$, there is a chart $(V_x, \phi_x)$ (normalized) at $x$ such that $\phi_x(x) = 0$ and $\phi_x(V_x) \supseteq B_2(0)$.
    In addition, $V_x \seq U_{\alpha(x)}$ for some index $\alpha(x)$.

    Let $W_x = \phi_x^{-1}(B_1(0))$, then $\{W_x\}_{x \in M}$ is an open cover of $M$.
    Compactness gives us finite subcover $\{W_1, \dots, W_k\}$.
    Set $f_i = H \circ \phi_i \in C^\infty(M)$ with $f_i \equiv 1$ on $W_i$.

    Since $\{W_i\}_{i=1}^{k}$ covers $M$, then $\sum_{i=1}^{k} f_i \geq 1$.
    Now define
    \[
        g_i = \frac{f_i}{\sum_{i=1}^{k} f_i} \in C^\infty(M, [0, 1])
    \]
    which has $\sum_{i=1}^{k} g_i \equiv 1$.

    Finally, must check that this partition of unity is subordinate to $\UU$.
    Have
    \[
        \supp g_i = \supp f_i \seq V_i \seq U_{\alpha_i}.
    \]
    Must still re-index: define
    \[
        \psi_\alpha = \sum_{\alpha_i = \alpha} g_i.
    \]
    Now $\{\psi_\alpha\}$ is a partition of unity subordinate to $\UU$.

    For the general case (where $M$ may not be compact), we use second countability to write $M$ as a nested sequence of compact subsets:
    \[
        M = \bigcup_{k=1}^{\infty} A_k
    \]
    where each $A_k$ is compact and $A_k \seq \inter A_{k+1}$.
    Then perform the compact construction inside each $A_k \setminus \inter A_{k-2}$ and combine them all together.
\end{proof}

\sepline

Tangent Vectors and Tangent Spaces

Manifolds are curved spaces, but we would like to approximate them by flat pieces (linear spaces).

In a geometric view, the tangent space of $\R^n$ at a fixed point $a \in \R^n$ is the set of all vectors in $\R^n$ placed such that their tail is at the point $a$; write
\[
    T_a\R^n = \{v_a = (a, v) \mid v \in \R^n\}.
\]
In most respects, this is essentially $T_a\R^n \iso \R^n$.
However, this cannot really be generalized to curved spaces.

Identify geometric $v_a \in T_a\R^n$ with a directional derivative $\DD_{v_a} : C^\infty(\R^n) \to \R$ where
\[
    \DD_{v_a}(f)
        = \left.\odv{}{t}\right|_{t=0} f(a + tv)
        = v \cdot \nabla f(a)
        = \sum_{i=1}^{n} v_i \pdv{f}{x_i}(a).
\]
Can check this map is linear and satisfies Leibniz rule:
\[
    \DD_{v_a}(fg) = f(a)\DD_{v_a}g + (\DD_{v_a}f) g(a).
\]
These two properties mean that the map is a derivation.

\sepline

For any point $a \in \R^n$, a \keyword{derivation} at $a$ is a linear map $X : C^{\infty}(\R^n) \to \R$ satisfying the \keyword{Leibniz rule}:
\[
    X(fg) = f(a)X(g) + X(f)g(a).
\]

The \keyword{abstract tangent space}
\[
    \tilde{T_a\R^n} = \{X : C^\infty(\R^n) \to \R \text{ derivation}\}.
\]
This is a vector space.

\sepline

\begin{proposition}[3.2]
    The map $T_a\R^n \to \tilde{T_a\R^n}$, $v_a \mapsto \DD_{v_a}$ is a linear isomorphism.
\end{proposition}

This gives us a way to generalize the tangent space to manifolds.

\begin{proof}
    linearity: easy.
    
    injectivity: if $v_a \in T_a\R^n$ such that $\DD_{v_a} = 0$ in $\tilde{T_a\R^n}$ then $v_a = 0$.
    Write $0 = \DD_{v_a}(f) = \sum_{i=1}^{n} v_i\pdv{f}{x_i}(a)$.
    Choose $f(x) = x_i$, then must have $v_i = 0$, so indeed $v = 0$.
    
    surjectivity: for all $X \in \tilde{T_a\R^n}$ need to find $v_a \in T_a\R^n$ such that $X = \DD_{v_a}$.
    Equivalently, need $Xf = \DD_{v_a}f$ for all $f \in C^\infty(\R^n)$.
    To find $v_a$, look at $f(x) = x_i$ then $v_i = X(x_i)$.
    By this choice, know $X(x_i) = \DD_{v_a}(x_i)$ but need to check for all functions.

    TBC
\end{proof}


\begin{lemma}[1]
    (1) If $f$ is a constant function and $X \in \tilde{T_a\R^n}$, then $Xf = 0$.

    (2) If $f, g \in C^\infty(\R^n)$ and $f(a) = g(a) 0$, then $X(fg) = 0$.
\end{lemma}

\begin{proof}
    (1) $X(1) = X(1 \cdot 1) = 1X(1) + X(1) 1 = 2X(1)$ implies $X(1) = 0$.

    (2) Leibniz rule.
\end{proof}

\begin{lemma}[2]
    (Taylor expansion with remainder) For any $f \in C^\infty(\R^n)$ we can write
    \[
        f(x) = f(a) + \sum_{i=1}^{n} g_i(x) (x_i - a_i)
    \]
    with each $g_i \in C^\infty(\R^n)$ and $g_i(a) = \pdv{f}{x_i}(a)$.

\end{lemma}

\sepline

\section*{10/13/22}

Geometric tangent space ``looks good'' and the abstract tangent space ``acts good.''

We continue the above proof to show surjectivity

\begin{proof}
    Compute
    \begin{align*}
        Xf
            &= X(f(a)) + \sum_{i=1}^{n} \eval{[g_i(a) X(x_i - a_i) + (Xg_i)(x_i - a_i)]}_{x = a} \\
            &= \sum_{i=1}^{n} \pdv{f}{x_i}(a) [X(x_i) - X(a)] \\
            &= \sum_{i=1}^{n} v_i \pdv{f}{x_i}(a) \\
            &= D_{v_a}f
    \end{align*}
\end{proof}


\begin{proof}[Proof of Lemma 2]
    \begin{align*}
        f(x) - f(a)
            &= \int_{0}^{1} \odv{}{t} f(a + t(x - a)) \,\dd{t} \\
            &= \int_{0}^{1} \pdv{f}{x_i}(a + t(x - a)) \cdot (x_i - a_i) \,\dd{t} \\
            &= \sum_{i=1}^{n} g_i(x)(x_i - a_i),
    \end{align*}
    where
    \[
        g_i = \int_{0}^{1} \pdv{f}{x_i}(a + t(x - a)) \,\dd{t} \in C^\infty(\R^n).
    \]
\end{proof}


\begin{corollary}
    $\dim \tilde{T_a\R^n} = n$ with a basis given by the directional derivative along the standard basis vectors $e_1, \dots, e_n$, e.g., $(e_i)_a \mapsto \DD_{(e_i)_a} \in \tilde{T_a\R^n}$ where we define
    \[
        \eval{\pdv{}{x_i}}_a f := \DD_{(e_i)_a}f = \pdv{f}{x_i}(a),
    \]
    i.e., $D_{(e_i)_a} = \pdv{}{x_i}\big|_a$.
\end{corollary}

\sepline

Given $p \in M^n$.
A \keyword{derivation} at $p$ is a linear map $X : C^\infty(M) \to \R$ such that the Leibniz rule holds:
\[
    X(fg) = f(p)Xg + (Xf)g(p).
\]

Call $X$ a \keyword{tangent vector} at $p$ and denote the \keyword{tangent space} of $M$ at $p$ by
\[
    T_pM = \{X : C^\infty(M) \to \R \text{ derivation at } p\}.
\]
This is a vector space in an obvious way.

Example: $M = \R^n$, then $T_p\R^n \iso \R^n$ with a basis given by $\pdv{}{x_1}\big|_p, \dots, \pdv{}{x_n}\big|_p$.

\sepline

Let $M$ and $N$ be smooth manifolds and $F : M \to N$ a smooth map.
Then for any $p \in M$, $F$ induces a linear map
\begin{align*}
    F_{*,p} : T_pM &\To T_{f(p)}N, \\
        X &\Mapsto (f \mapsto X(f \circ F) = X(F^*f)),
\end{align*}
called the \keyword{differential} of $F$ at $p$.

\begin{proposition}
    Basic properties of $F_*$:
    \begin{enumerate}[(1)]
        \item $F_*$ is linear
        \item If $F : M \to N$ and $G : N \to P$ are smooth maps, then the composition $G \circ F : M \to P$ is a smooth map with differential $(G \circ F)_* = G_* \circ F_*$.
        This is an abstract version of the chain rule.
        \item $(\id_M)_{*,p} = \id_{T_pM}$
        \item $F : M \to N$ diffeomorphism implies $F_* : T_pM \to T_{F(p)}N$ isomorphism
    \end{enumerate}
\end{proposition}

\sepline

\begin{proposition}
    $U \seq M$ open, then for all $p \in U$ we have $T_pM \iso T_pU$ (as vector spaces).
    In other words, tangent space is a local property.
\end{proposition}

\begin{proof}
    We have a smooth inclusion $\iota : U \inc M$, which induces $\iota_* : T_pU \to T_pM$.
    We construct an inverse by hand: $\sigma : T_pM \to T_pU$.
    A derivation $X \in T_pM$ only works for functions defined on all of $M$.
    Given a function defined only on $U$, we must define $\sigma(X)$ to act in a way that naturally follows from the action of $X$.
    To do this, we will use bump functions to extend a function on $U$ to all of $M$.

    Let $V$ be an open neighborhood of $p$ such that $K = \clo{V} \csub U$.
    Then there exists $\chi \in C^\infty(M)$ such that $\chi \equiv 1$ on $K$ and $\supp\chi \seq U$.
    Now for all $f \in C^\infty(U)$, we take the product $\chi f$, which is the same as $f$ on $V$ and zero on the rest of $U$.
    We can then trivially extend $\chi f$ to be a smooth function on all of $M$.
    Now we can define
    \[
        \sigma(X)f = X(\chi f).
    \]

    We must check that this makes sense, as it is not clearly independent of the choice of $\chi$.

    Need to check $\sigma$ is linear and an inverse of $\iota_*$; both need a lemma.
\end{proof}

\begin{lemma}
    If $X : C^\infty(M) \to \R$ is a derivation at $p$ and $f, g \in C^\infty(M)$ such that there is an open neighborhood $U$ of $p$ on which $f|_U = g|_U$, then in fact $Xf = Xg$.
\end{lemma}

\begin{proof}
    Let $h = f - g$; want to show $Xh = 0$.

    Assumption implies $h|_U \equiv 0$.

    For $p \in U$ there is $\psi \in C^\infty(M)$ such that $\psi(p) = 0$ and $\psi \equiv 1$ on $M \setminus U$.

    Then $h = \psi h$, which tells us
    \[
        Xh
            = X(\psi h)
            = (X\psi)h(p) + \psi(p) Xh
            = 0.  
    \]
\end{proof}


\sepline

\section*{10/18/22}

insert notes here

\sepline

\section*{10/20/22}

Smooth map $F : M \to N$ induces a linear map $F_* : T_pM \to T_{F(p)}N$ where $\dot{c}(0) \mapsto \dot{\tilde{c}}(0)$ with $\tilde{c}(t) = F(c(t))$.

A chart $(U, \phi)$ at $p$ and chart $(V, \psi)$ at $F(p)$.
Bases of $T_pM$ and $T_{F(p)}N$ are respectively
\[\textstyle
    E_i = (\phi^{-1})_*(\pdv{}{x_i})
    \isp{and}
    \tilde{E}_j = (\psi^{-1})_*(\pdv{}{y_j}).
\]

Matrix of $F_* = \mat{\pdv{\hat{F}_j}{x_i}}$, where $\hat{F} = \psi \circ F \circ \phi^{-1} : \phi(U) \to \psi(V)$.

\sepline

Tangent bundle 
\[
    TM^n
        = \bigsqcup_{p \in M} T_pM
        = \{(p, v) \mid p \in M, v \in T_pM\}.
\]
can be made into a smooth manifold of dimension $2n$.
Denote projection
\begin{align*}
    \pi : TM &\To M, \\
        (p, v) &\Mapsto p.
\end{align*}

Let $(U, \phi)$ be a chart of $M$.
We want to induce a chart $(\pi^{-1}(U), \tilde{\phi})$ on $TM$.
Note
\[
    \pi^{-1}(U)
        = \bigsqcup_{p \in U} T_pM
        = \bigsqcup_{p \in U} T_pU
        = TU.
\]
Construct
\begin{align*}
    \tilde{\phi} : \pi^{-1}(U) &\To \phi(U) \times \R^n \seq \R^{2n}, \\
        (p, v = \tsum v^iE_i) &\Mapsto (\phi(p), (v^1, \dots, v^n)).
\end{align*}
In other words using coordinates $T_pM \to \R^n$ given by $E_i \mapsto e_i$.

Check smooth compatibility: $(V, \psi)$ another chart of $M$ should induce chart $(\pi^{-1}(V), \tilde{\psi})$ on $TM$ which is smoothly compatible with the first chart.
For $x = (x_1, \dots, x_n) \in \phi(U)$ and $\xi = (\xi_1, \dots, \xi_n) \in \R^n$
\begin{align*}
    \tilde{\psi} \circ \tilde{\phi}^{-1}(x, \xi)
        &= \tilde{\psi}(\phi^{-1}(x), \xi_iE_i)
            & \text{Einstein Sum Convention } \xi_iE_i &= \tsum_i \xi_i E_i \\
        &= ((\psi \circ \phi^{-1})(x), ?)
            & &= \tsum_j ?\tilde{E}_j
\end{align*}

Claim
\[
    E_i = \sum_{j=1}^{n} \pdv{(\psi \circ \phi^{-1})_j}{x_i} \tilde{E_j}.
\]
To work out matrix, need to work out
\[
    F_*(E)_i = \pdv{\hat{F}_j}{x_i}\tilde{E}_j.
\]
Apply to $F = \id_M$ and we get the claim.

This result implies
\begin{align*}
    \sum_{i=1}^{n} \xi_i E_i
        &= \sum_{i=1}^{n} \sum_{j=1}^{n} \xi_i \pdv{(\psi \circ \phi^{-1})_j}{x_i} \tilde{E}_j \\
        &= \sum_{j=1}^{n} \left(\sum_{i=1}^{n} \xi_i \pdv{(\psi \circ \phi^{-1})_j}{x_i} \right) \tilde{E}_j.
\end{align*}
so
\[\textstyle
    \tilde{\psi} \circ \tilde{\phi}^{-1}(x, \xi)
        = \left((\psi \circ \phi^{-1})(x), \xi_i\pdv{(\psi \circ \phi^{-1})_1}{x_i}, \dots, \xi_i\pdv{(\psi \circ \phi^{-1})_n}{x_i}\right).
\]
This is smooth in all components.

We can give $TM$ a topology so that $\pi : TM \to M$ is continuous (initial topology?) and $\tilde{\phi}$ is then a homeomorphism (can check Hausdorff, second countable, etc.).

\sepline

Exercise to compute $T(TM)$


Example: $S^2$.
Let $S \in S^2$ be south pole.
Basis $E_1, E_2$ for $T_pS^2$ fpr $p \in S^2 \setminus \{S\}$, but no such basis is valid on all of $S^2$.
(Hairy Ball Theorem)


\sepline

Summarizing

\begin{proposition}
    $TM$ is a $2n$-dimensional smooth manifold and $\pi : TM \to M$ is smooth.
\end{proposition}

\sepline

Remark

$\pi^{-1}(p) = T_p(M)$ is a vector space.

$(U, \phi)$ a chart, $\pi^{-1}(U) \iso \phi(U) \times \R^n$ via $\tilde{\phi}$.

This gives the data of a vector bundle

\sepline

Submersions, Immersions, and Embeddings

The manifold $S^2$ can be represented as the solution set to the equation $x^2 + y^2 + z^2 = 1$ in $\R^3$.
But not all such equations give a manifold, e.g., $x^2 - y^2 = 0$ gives intersecting lines.

Curves and surfaces: $c : I \to \R^3$ smooth is a curve

Need to study smooth maps $F : M \to N$.

The differential $F_* : T_pM \to T_{F(p)}N$ is a linear approximation, so tells us about local information.

\sepline

Let $F : M \to N$ be a smooth map of manifolds.

The \keyword{rank} of $F$ at $p \in M$ is defined as
\begin{align*}
    \rank_pF
        &= \text{linear rank of } F_* : T_pM \to T_{F(p)}N \\
        &= \dim\im F_* \\
        &= \text{rank of Jacobian} \mat{\pdv{\hat{F}_j}{x_i}}.
\end{align*}

\sepline

Example

For $k \leq \min\{n, m\}$
\begin{align*}
    F : \R^n &\To \R^m, \\
        (x_1, \dots, x_n) &\Mapsto (x_1, \dots, x_k, 0, \dots, 0).
\end{align*}
Then
\[
    J_F = \mat{I_k & 0 \\ 0 & 0}
\]
so $\rank F = k$.

\sepline

Example

For $F = c : I \to N$, looking at $\phi \circ c : I \to \phi(U) \seq \R^n$.
So then
\[
    \rank c
        = \rank\mat{\odv{(\phi \circ c)_i}{t}}
        = \begin{cases}
            1 &\text{if one of } \odv{(\phi \circ c)_i}{t} \ne 0, \\
            0 &\text{otherwise}.
        \end{cases}
\]
Can also write $(\phi \circ c)_i' = \odv{(\phi \circ c)_i}{t}$.

Note $\dot{c} = (\phi \circ c)_i'E_i$, where $E_i = (\phi^{-1})_*(\pdv{}{x_i})$.
Then
\[
    \rank\dot{c} = \begin{cases}
        1 &\text{if } \dot{c} \ne 0. \\
        0 &\text{otherwise}.
    \end{cases}
\]

\sepline

Let $F: M^n \to N^m$ be a smooth map

$F$ is an \keyword{immersion} if $\rank_pF = n$ for all $p \in M$
(or, equivalently, if $F_* : T_pM \to T_{F(p)}N$ is injective for all $p \in M$).

$F$ is a \keyword{submersion} if $\rank_pF = m$ for all $p \in M$ (or, equivalently, if $F_* : T_pM \to T_pN$ is surjective for all $p \in M$).

\sepline

\section*{10/25/22}


\begin{theorem}[Rank]
    If $F : M \to N$ is smooth with $\rank_p F = k$ for all $p \in M$, then for all $p \in M$ there are charts $(U, \phi)$ at $p$ and $(V, \psi)$ at $F(p)$ such that
    \begin{align*}
        \hat{F} = \psi \circ F \circ \phi^{-1} : \phi(U) &\To \psi(V), \\
            (x_1, \dots, x_n) &\Mapsto (x_1, \dots, x_k, 0, \dots, 0).
    \end{align*}
    That is, in local coordinates, $F$ can look like the standard immersion/submersion.
\end{theorem}

\sepline

Example

$F : \R^2 \to \R^3$ with $F(x, y) = (x, y, f(x, y))$.
Then $\rank F \equiv 2$.

Make a change of coordinates by constructing a diffeomorphism on $\R^3$.
\begin{align*}
    G : \R^3 &\To \R^3, \\
        (x, y, z) &\Mapsto (x, y, z - f(x, y)) = (u, v, w).
\end{align*}

To see that $(u, v, w)$ is a valid coordinate system, we check that $G$ is a diffeomorphism.
Obviously, $G$ is smooth and its inverse
\[
    G^{-1}(u, v, w) = (u, v, w + f(u, v))
\]
is also smooth.

With respect to $(x, y)$ on $\R^2$ and $(u, v, w)$ on $\R^3$ we have $F(x, y) = (x, y, 0)$.

\sepline

\begin{theorem}[Inverse Function]
    $U \seq \R^n$ open and $G : U \to \R^n$ smooth.
    If $p \in U$ such that $J_G(p) = \mat{\pdv{G^j}{x_i}(p)}$ is invertible, then there is an open neighborhood $p \in V \seq U$ and $\tilde{V} \seq \R^n$ open such that $G : V \to \tilde{V}$ is a diffeomorphism.
    In other words, $G$ is a \keyword{local diffeomorphism}.
\end{theorem}


\sepline

\begin{proof}[Proof of Rank Theorem]
    $F$ smooth implies there are charts $(\tilde{U}, \tilde{\phi})$ at $p$ and $(\tilde{V}, \tilde{\psi})$ at $F(p)$ such that $F(\tilde{U}) \seq \tilde{V}$ and
    \[
        \tilde{F} = \tilde{\psi} \circ F \circ \tilde{\phi}^{-1} : \tilde{\phi}(\tilde{U}) \To \tilde{\psi}(\tilde{V}).
    \]
    Now we modify the charts to that $\tilde{F}$ is becoming simpler.

    For simplicity, $n = m = 2$ and $k = 1$, i.e., $\tilde{\phi}(\tilde{u}) \seq \R^2$ and $\tilde{\psi}(\tilde{V})$ and $\Jac\tilde{F} \equiv 1$.
    Consider
    \begin{align*}
        \tilde{F} : \tilde{\phi}(\tilde{U}) &\To \tilde{\psi}(\tilde{V}), \\
            (x, y) &\Mapsto (\tilde{F}_1(x, y), \tilde{F}_2(x, y)).
    \end{align*}
    Write
    \[
        \Jac\tilde{F} = \mat{
            \pdv{\tilde{F}_1}{x} & \pdv{\tilde{F}_1}{y} \\
            \pdv{\tilde{F}_2}{x} & \pdv{\tilde{F}_2}{y}.
        }
    \]
    Then $\rank_0 F = 1$ implies that one of the entries is not zero.
    Without loss of generality, say $\pdv{\tilde{F}_1}{x}(0, 0) \ne 0$.

    Consider
    \begin{align*}
        G : \tilde{\phi}(\tilde{U}) &\To \R^2, \\
            (x, y) &\Mapsto (\tilde{F}_1(x, y), y).
    \end{align*}
    Then
    \[
        \Jac_0 G = \mat{
            \pdv{\tilde{F}_1}{x}(0, 0) \ne 0 & \pdv{\tilde{F}_1}{y}(0, 0) \\
            0 & 1
        }
    \]
    is invertible.
    Then inverse function theorem implies $G$ is a local diffeomorphism at the origin.
    Say $W \seq \tilde{\tilde{U}}$ is open such that $G : W \to G(W)$ is a diffeomorphism.

    Check that $\tilde{F} \circ G^{-1}$ is already (much) simplified.

    Have
    \[
        G^{-1}(x_1, x_2) = (x, y)
    \]
    if and only if
    \[
        (x_1, x_2)
            = G(x, y)
            = (\tilde{F}_1(x, y), y).
    \]
    Then 
    \[
        x_1 = \tilde{F}_1(x, y)
        \isp{and}
        x_2 = y.
    \]
    Then
    \begin{align*}
        \tilde{F} \circ G^{-1}(x, y)
            &= \tilde{F}(x, y) \\
            &= (\tilde{F}_1(x, y), \tilde{F}_2(x, y)) \\
            &= (x_1, \overline{F}_2(x_1, x_2)).
    \end{align*}
    But we already have $\rank\Jac\tilde{F} \equiv 1$ so $\rank(\tilde{F} \circ G^{-1}) \equiv 1$.
    But
    \[
        \Jac(\tilde{F} \circ G^{-1}) = \mat{
            1 & 0 \\
            \pdv{\overline{F}_2}{x_1} & \pdv{\overline{F}_2}{x_2}
        }
    \]
    so must have $\pdv{\overline{F}_2}{x_2} \equiv 0$.
    That is, $\overline{F}_2(x_1, x_2) = f(x_1)$ smooth.

    That is $\tilde{F} \circ G^{-1}(x_1, x_2) = (x_1, f(x_1))$ as in our example.

    Make a change of coordinates on the target $\R^2$ as in the example, i.e., there is a diffeomorphism $H : \R^2 \to \R^2$ mapping $(x_1, x_2) \mapsto (x_1, x_2 - f(x_1))$.

    Then $H \circ \tilde{F} \circ G^{-1}(x_1, x_2) = (x_1, 0)$.

    All written out, $(H \circ \tilde{\psi}) \circ F \circ (\phi^{-1} \circ G^{-1}) = (H \circ \tilde{\psi}) \circ F \circ (G \circ \phi)^{-1}$.
    This is the modification of the charts.

    The general case simply deals with more coordinates.
\end{proof}


\sepline

\begin{corollary}
    $F : M^n \to N^m$ smooth
    \begin{enumerate}[(a)]
        \item If $F$ is an immersion, then $F$ is locally injective, i.e., for all $p \in M$ there is a neighborhood $U$ of $p$ such that $F|_U$ is injective.
        \item If $F$ is a submersion, then $F$ is an open mapping.
        (?: In particular, if $N$ is connected then $F$ is surjective.)
    \end{enumerate}
\end{corollary}

\sepline

Historically, manifolds arise as the image of a map.
First condition we need is that the map is an immersion.
Unfortunately, that's not enough!

example: $\gamma : (-\pi, \pi)  \to \R^2$ with $\gamma(t) = (\sin2t, \sin t)$.

$\rank \gamma \equiv 1$ as $\gamma'(t) \ne 0$.

But images is figure-eight---not a manifold.

\begin{drawing}
    \draw (0, 0) to[out=45, in=0] (0, 1);
    \draw (0, 0) to[out=135, in=180] (0, 1);

    \draw (0, 0) to[out=-45, in=0] (0, -1);
    \draw (0, 0) to[out=-135, in=180] (0, -1);
\end{drawing}

Two ways of giving the image a topology:
\begin{enumerate}
    \item subspace topology from $\R^2$
    \item use $\gamma$ to transplant the topology of the domain to the image 
\end{enumerate}
In this case, the two topologies are different.

\sepline

A smooth map $F : M^n \to N^m$ is called \keyword{embedding} if
\begin{enumerate}[(1)]
    \item $F$ is injective and an immersion,
    \item $F$ is a homeomorphism to its image with the subspace topology
\end{enumerate}

\sepline

\section*{10/27/22}

Example

Consider
\begin{align*}
    F: \R &\To T^2 = S^1 \times S^1 \seq \C \times \C \\
        t &\Mapsto (e^{2\pi it}, e^{2\sqrt{2}\pi it}).
\end{align*}
Can check smooth and injective immersion.

Since we chose irrational slope, line will densely fill up the torus.

Q: Is this an embedding?

No, it is not a homeomorphism with to its image as a subspace of the codomain.

Q: When is an injective immersion an embedding?

\sepline

\begin{proposition}[4.22]
    Let $F : M \to N$ be an injective immersion.
    Then $F$ is an embedding if any one of the following holds:
    \begin{enumerate}[(1)]
        \item $F$ is an open or closed map;
        \item $F$ is a proper map, i.e., preimage of compact is compact;
        \item $M$ is compact.
    \end{enumerate}
\end{proposition}

\begin{proof}
    (2) $\implies$ (3) easy since $M$ compact implies $F$ proper.

    (1) simply restating definition for $F^{-1}$ being continuous.

    Show (2).
    Assume $F$ is proper; will show $F$ is closed.
    Let $C \seq M$ be a closed set, and let $q$ be a limit point of $F(C) \seq N$.

    Let $U$ be an open neighborhood of $q$ such that $\clo{U}$ is compact.
    Then $F$ being proper implies $F^{-1}(\clo{U})$ is compact.
    Since $C$ is closed, $F^{-1}(\clo{U}) \cap C$ is again compact.
    Then $F$ continuous implies $F(F^{-1}(\clo{U}) \cap C) = \clo{U} \cap F(C)$ is compact and $q$ is a limit point of of $\clo{U} \cap F(C)$.
    This implies $q \in \clo{U} \cap F(C)$, so $q \in F(C)$.
\end{proof}

\sepline

Example

$\iota : S^n \inc \R^{n+1}$ by canonical inclusion.
Obviously injective.
$\iota_*$ injective implies immersion.
Then $S^n$ compact implies $\iota$ is an embedding.

\sepline

\begin{theorem}[4.25]
    If $F : M \to N$ immersion, then it is a local embedding.
    That is, for all $p \in M$ there is an open neighborhood $U$ of $p$ such that $F|_U : U \to N$ is an embedding.
\end{theorem}

\begin{proof}
    Rank theorem.

    Locally injective, choose open set such that closure is compact.
\end{proof}

\sepline

\begin{theorem}[Strong Whitney Embedding]
    Smooth manifold $M^n$ can always be embedded in $\R^{2n}$ and can be immersed in $\R^{2n-1}$.
\end{theorem}

Example: There is an embedding $\R\P^2 \inc \R^4$ but only immersed into $\R^3$.

\begin{proof}[Proof Sketch]
    Step 1: Construct embedding $F : M \to \R^N$ with $N$ sufficiently (very) large.

    Step 2: Try to reduce dimension of $\R^N$, e.g., project to $\R^{N-1}$.
    Uses Sand's Theorem.

    First, suppose $M$ is compact.
    Let $\{(U_i, \phi_i)\}_{i=1}^{k}$ be a finite atlas such that $\{\phi_i^{-1}(B_1(0))\}$ covers the manifold.
    Let $f_1, \dots, f_k \in C^\infty(M)$ be bump functions such that $f_i \equiv 1$ on $\phi_i^{-1}(B_1(0))$ and $\supp f_i \seq U_i$.
    Define
    \begin{align*}
        F : M &\To \R^{kn + k}, \\
            p &\Mapsto (f_1(p)\phi_1(p), \dots, f_k(p)\phi_k(p), f_1(p), \dots, f_k(p)).
    \end{align*}
    Obviously smooth injection.

    Check immersion on each $\phi_i^{-1}(B_1(0))$:
    \[
        F \circ \phi_i^{-1}(x) = (\dots, \underset{i\text{th}}{x}, \dots).
    \]
    So then
    \[
        \Jac F \circ \phi_i^{-1} = \mat{
            * \\
            & I_n \\
            & & & *
        }
    \]
    so rank is $n$, so $F$ is an immersion.

    $M$ compact, injective immersion implies embedding.
\end{proof}

\end{document}