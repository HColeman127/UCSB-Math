\documentclass[12pt]{article}

% Packages
\usepackage[margin=1in]{geometry}
\usepackage{fancyhdr, parskip}
\usepackage{amsmath, amsthm, amssymb}
\usepackage{tikz, tikz-cd}
\usepackage[shortlabels]{enumitem}

% Page Style
\makeatletter
\fancypagestyle{title}{
    \renewcommand{\headrulewidth}{0.4pt}
    \setlength{\headheight}{15pt}
    \fancyhead[R]{\@author}
    \fancyhead[L]{\@title}
    \fancyhead[C]{\@date}
}
\makeatother
\renewcommand{\maketitle}{\thispagestyle{title}}
\fancypagestyle{plain}{
    \fancyhf{}
    \renewcommand{\headrulewidth}{0pt}
    \renewcommand{\footrulewidth}{0pt}
    \fancyfoot[R]{\thepage}
}
\pagestyle{plain}

% Problem Box
\setlength{\fboxsep}{4pt}
\newlength{\myparskip}
\setlength{\myparskip}{\parskip}
\newsavebox{\savefullbox}
\newenvironment{fullbox}{\begin{lrbox}{\savefullbox}\begin{minipage}{\dimexpr\textwidth-2\fboxsep\relax}\setlength{\parskip}{\myparskip}}{\end{minipage}\end{lrbox}\framebox[\textwidth]{\usebox{\savefullbox}}}
\newenvironment{pbox}[1][]{\begin{fullbox}\def\temp{#1}\ifx\temp\empty\else\paragraph{#1}\phantom{}\fi}{\end{fullbox}}

% Theorem Environments
\theoremstyle{definition}
\newtheorem{lemma}{Lemma}

% Tikz Environments
\newenvironment{drawing}{\begin{center}\begin{tikzpicture}}{\end{tikzpicture}\end{center}}
% \tikzcdset{row sep/normal=0pt}
\newenvironment{cd}{\begin{center}\begin{tikzcd}}{\end{tikzcd}\end{center}}

% Default Commands
\newcommand{\isp}[1]{\quad\text{#1}\quad}
\newcommand{\N}{\mathbb{N}} 
\newcommand{\Z}{\mathbb{Z}}
\newcommand{\Q}{\mathbb{Q}}
\newcommand{\R}{\mathbb{R}}
\newcommand{\C}{\mathbb{C}}
\newcommand{\A}{\mathbb{A}}
\renewcommand{\P}{\mathbb{P}}
\newcommand{\eps}{\varepsilon}
\renewcommand{\phi}{\varphi}
\renewcommand{\emptyset}{\varnothing}
\newcommand{\<}{\langle}
\renewcommand{\>}{\rangle}
\newcommand{\iso}{\cong}
\newcommand{\eqc}{\overline}
\newcommand{\clo}{\overline}
\renewcommand{\tilde}{\widetilde}
\renewcommand{\hat}{\widehat}
\newcommand{\seq}{\subseteq}
\newcommand{\teq}{\trianglelefteq}
\DeclareMathOperator{\id}{id}
\DeclareMathOperator{\im}{im}
\newcommand{\inc}{\hookrightarrow}
\newcommand{\dd}{\mathrm{d}}
\newcommand{\mat}[1]{\begin{bmatrix}#1\end{bmatrix}}

% Extra Commands
\newcommand{\CP}{\C\mathrm{P}}
\newcommand{\RP}{\R\mathrm{P}}

\renewcommand{\Re}{\operatorname{Re}}
\renewcommand{\Im}{\operatorname{Im}}

\DeclareMathOperator{\supp}{supp}

\newcommand{\UU}{\mathcal{U}}

% Document
\begin{document}
\title{MATH 240A Homework 3}
\author{Harry Coleman}
\date{October 14, 2022}
\maketitle



\begin{pbox}[1]
    Let $F: S^2 \to \CP^1$ be the smooth map constructed in class.
    Show that $F$ is actually a diffeomorphism by constructing explicitly $F^{-1}$ and checking that it is also smooth.
    (\textbf{Hint:} try solve for $F^{-1}$ on each of the two charts. The quadratic formula comes handy here.)
\end{pbox}

On the open set $U_1 = \{[1 : w]\} \seq \CP^1$, we have
\[
    F^{-1}([1 : w]) = \left(\frac{1 - |w|^2}{1 + |w|^2}, \frac{2\Re w}{1 + |w|^2}, \frac{2\Im w}{1 + |w|^2}\right).
\]
This map is clearly smooth since $U_1$ is diffeomorphic to $\C$ via the map $[1 : w] \mapsto w$, and this map is built out of smooth maps in the variable $w$.

On the open set $U_2 = \{[z : 1]\} \seq \CP^1$, we have
\[
    F^{-1}([z : 1]) = \left(\frac{|z|^2 - 1}{|z|^2 + 1}, \frac{2\Re z}{|z|^2 + 1}, \frac{2\Im z}{|z|^2 + 1}\right).
\]
This map is clearly smooth since $U_2$ is diffeomorphic to $\C$ via the map $[z : 1] \mapsto z$, and this map is built out of smooth maps in the variable $z$.

One can check that these maps agree on the overlap $U_1 \cap U_2$.



\newpage
\begin{pbox}[2]
    Let $M^n$ be a smooth manifold of dimension $n$. Denote by $C^\infty(M)$ the space of $C^\infty$ functions on $M$.
    Recall that this is a vector space with the usual addition, and scalar multiplication.
\end{pbox}

\begin{pbox}[(a)]
    Show that $C^\infty(\R^n)$ $(n>0)$ is a vector space of infinite dimension.
\end{pbox}

\begin{proof}
    Any polynomial function on $\R^n$, i.e., an element of $\R[x_1, \dots, x_n]$, is in particular a smooth function.
    In other words, we have a subspace $\R[x_1, \dots, x_n] \leq C^\infty(\R^n)$.
    The space of polynomial functions is an infinite-dimensional real vector space with basis given by all monomials of the form $x_1^{a_1} \cdots x_n^{a_n}$ for $a_i \in \Z_{\geq0}$.
    Therefore, the entire space $C^\infty(\R^n)$ must be infinite-dimensional.
\end{proof}

\begin{pbox}[(b)]
    Show that $C^\infty(M)$ $(n>0)$ is a vector space of infinite dimension.
\end{pbox}

\begin{proof}
    Choose a chart $(U, \phi)$ on $M$ such that $\phi(U) = \R^n$.
    For $k \in \N$ set $p_k = \phi^{-1}(ke_1)$ and $U_k = \phi^{-1}(B_{1/4}(ke_1))$.
    By this construction, the $U_k$'s are completely disjoint sets, and in fact even their closures are disjoint.
    Applying a variant of Problem 3, let $f_k \in C^\infty(M)$ be constructed such that $f(p_k) = 1$ and $f_k \equiv 0$ on $M \setminus U_k$.
    In particular, we have $\supp f_k \seq U_k$, so the supports of all the $f_k$'s are completely disjoint from one another.

    It follows that $\{f_k\}$ is a linearly independent subset of $C^\infty(M)$.
    Suppose $\sum_k a_kf_k = 0$ for some $a_k \in \R$.
    Since the supports of the $f_k$'s are pairwise disjoint, the only way for this to be possible is if $a_k = 0$ for all $k$.
    Hence, the $f_k$'s are linearly independent.
    We have found infinitely many linearly independent elements in $C^\infty(M)$, so the space must be infinite-dimensional.
\end{proof}





\newpage
\begin{pbox}[3]
    Let $M$ be a smooth manifold. If $U$ is an open set of $M$, and $p\in U$, show that there is a smooth function $f\in C^\infty(M)$ such that $f=1$ on $M\setminus U$ and $f(p)=0$.
\end{pbox}

Choose a chart $\phi : U \to \R^n$ such that $\phi(p) = 0$ and $\phi(U) \supseteq B_2(0)$.
Let $H : \R^n \to [0, 1]$ be a smooth bump function satisfying $H \equiv 1$ on $B_1(0)$ and $H \equiv 0$ outside $B_2(0)$.
Then we define the smooth function $f = 1 - H \circ \phi : U \to \R$ which satisfies $f \equiv 0$ on $\phi^{-1}(B_1(0))$ (in particular, $f(p) = 0$) and $f \equiv 1$ on $U \setminus \phi^{-1}(B_2(0))$.
We can now extend $f$ to the rest of $M$ by defining $f \equiv 1$ on $M \setminus U$.


\newpage
\begin{pbox}[4]
    Let $M$ be a compact smooth manifold and $h: M\to \R$ a continuous function. Show that, for any $\eps>0$, there is $f\in C^\infty(M)$ such that
    \[
        |h(p)- f(p)| < \eps
    \]
    for any $p\in M$. (\textbf{Hint:} recall that the ($n$-dimensional) Weierstrass approximation theorem says that a continuous function on a bounded domain in $\R^n$ can be approximated by polynomials.)
\end{pbox}

Choose a smooth atlas $\UU = \{(U_\alpha, \phi_\alpha)\}$ for $M$ such that each $\phi_\alpha(U_\alpha)$ is bounded, e.g., is contained in $B_1(0)$.
Let $\{\psi_\alpha\}$ be a partition of unity subordinate to $\UU$.
Then $h \circ \phi_\alpha^{-1} : \phi_\alpha(U_\alpha) \to \R$ is a continuous function on a bounded domain in $\R^n$.
By the Weierstrass approximation theorem, let $f_\alpha : \phi_\alpha(U_\alpha) \to \R$ be a polynomial approximation which is within a supremum distance of $\eps$, i.e.,
\[
    \|f_\alpha - h \circ \phi_\alpha^{-1}\|
        = \sup\{|f_\alpha(y) - (h \circ \phi_\alpha^{-1})(y)| : y \in \phi_\alpha(U_\alpha)\}
        < \eps.
\]

We now define
\[
    f
        = \sum_\alpha \psi_\alpha \cdot (f_\alpha \circ \phi_\alpha)
        \in C^\infty(M).
\]
We check that this approximates $h$.
For a given $x \in M$, say $x \in \supp\psi_{\alpha_i}$ for $i = 1, \dots n$ (since the partition of unity has locally finite support).
Then
\begin{align*}
    |f(x) - h(x)|
        &= \left|\sum_{i=1}^{n} \psi_{\alpha_i}(x)f_{\alpha_i}(\phi_{\alpha_i}(x)) - \sum_{i=1}^{n} \psi_{\alpha_i}(x)h(x)\right| \\
        &= \sum_{i=1}^{n} \psi_{\alpha_i}(x)\big|f_{\alpha_i}(\phi_{\alpha_i}(x)) - h(x)\big|. 
\end{align*}
Denote $y_i = \phi_{\alpha_i}(x) \in \phi_{\alpha_i}(U_{\alpha_i})$, then
\begin{align*}
    |f(x) - h(x)|
        &= \sum_{i=1}^{n} \psi_{\alpha_i}(x)\big|f_{\alpha_i}(y_i) - h(\phi_{\alpha_i}^{-1}(y_i))\big| \\
        &\leq \sum_{i=1}^{n} \psi_{\alpha_i}(x)\big\|f_{\alpha_i} - h \circ \phi_{\alpha_i}^{-1}\big\| \\
        &< \sum_{i=1}^{n} \psi_{\alpha_i}(x) \eps \\
        &= \eps.
\end{align*}

\end{document}