\documentclass[12pt]{article}

% Packages
\usepackage[margin=1in]{geometry}
\usepackage{fancyhdr, parskip}
\usepackage{amsmath, amsthm, amssymb}
\usepackage{tikz, tikz-cd}

% Page Style
\makeatletter
\fancypagestyle{title}{
    \renewcommand{\headrulewidth}{0.4pt}
    \setlength{\headheight}{15pt}
    \fancyhead[R]{\@author}
    \fancyhead[L]{\@title}
    \fancyhead[C]{\@date}
}
\makeatother
\renewcommand{\maketitle}{\thispagestyle{title}}
\fancypagestyle{plain}{
    \fancyhf{}
    \renewcommand{\headrulewidth}{0pt}
    \renewcommand{\footrulewidth}{0pt}
    \fancyfoot[R]{\thepage}
}
\pagestyle{plain}

% Problem Box
\setlength{\fboxsep}{4pt}
\newlength{\myparskip}
\setlength{\myparskip}{\parskip}
\newsavebox{\savefullbox}
\newenvironment{fullbox}{\begin{lrbox}{\savefullbox}\begin{minipage}{\dimexpr\textwidth-2\fboxsep\relax}\setlength{\parskip}{\myparskip}}{\end{minipage}\end{lrbox}\framebox[\textwidth]{\usebox{\savefullbox}}}
\newenvironment{pbox}[1][]{\begin{fullbox}\def\temp{#1}\ifx\temp\empty\else\paragraph{#1}\phantom{}\fi}{\end{fullbox}}

% Theorem Environments
\theoremstyle{definition}
\newtheorem{lemma}{Lemma}

% Tikz Environments
\newenvironment{drawing}{\begin{center}\begin{tikzpicture}}{\end{tikzpicture}\end{center}}
% \tikzcdset{row sep/normal=0pt}
\newenvironment{cd}{\begin{center}\begin{tikzcd}}{\end{tikzcd}\end{center}}

% Default Commands
\newcommand{\isp}[1]{\quad\text{#1}\quad}
\newcommand{\N}{\mathbb{N}} 
\newcommand{\Z}{\mathbb{Z}}
\newcommand{\Q}{\mathbb{Q}}
\newcommand{\R}{\mathbb{R}}
\newcommand{\C}{\mathbb{C}}
\newcommand{\A}{\mathbb{A}}
\renewcommand{\P}{\mathbb{P}}
\newcommand{\eps}{\varepsilon}
\renewcommand{\phi}{\varphi}
\renewcommand{\emptyset}{\varnothing}
\newcommand{\<}{\langle}
\renewcommand{\>}{\rangle}
\newcommand{\iso}{\cong}
\newcommand{\eqc}{\overline}
\newcommand{\clo}{\overline}
\newcommand{\seq}{\subseteq}
\newcommand{\teq}{\trianglelefteq}
\DeclareMathOperator{\id}{id}
\DeclareMathOperator{\im}{im}
\newcommand{\inc}{\hookrightarrow}
\newcommand{\dd}{\mathrm{d}}
\newcommand{\mat}[1]{\begin{bmatrix}#1\end{bmatrix}}

% Extra Commands
\newcommand{\B}{\mathbb{B}}
\renewcommand{\hat}{\widehat}
\newcommand{\UU}{\mathcal{U}}

% Document
\begin{document}
\title{MATH 240A Homework 1}
\author{Harry Coleman}
\date{September 30, 2022}
\maketitle

\begin{pbox}[Problem 1.1]
    Let $X$ be the set of all points $(x, y) \in \R^2$ such that $y = \pm 1$, and let $M$ be the quotient of $X$ by the equivalence relation generated by $(x, -1) \sim (x, 1)$ for all $x \ne 0$.
    Show that $M$ is locally Euclidean and second-countable, but not Hausdorff.
\end{pbox}

Let $q : X \to M$ be the quotient map.

For nonzero $x \in \R$, denote the point (equivalence class) $\{(x, 1), (x, -1)\} \in M$ by $[x]$.

Denote the origins of $M$ by $0_+$ and $0_-$.

We first show $M$ is locally Euclidean and second-countable.

\begin{proof}
    The space $M$ is covered by two open sets: $U_\pm = M \setminus \{0_\mp\}$.
    We can then construct maps $\phi_\pm : U_\pm \to \R$ which send $[x] \mapsto x$ for nonzero values of $x$ and send $0_\pm \mapsto 0$.
    These maps are homeomorphisms, hence provide an atlas for $M$.

    Note that $\R$ is second-countable; let $\UU$ be a countable base for $\R$.
    We can construct a countable base for $M$ by replacing each set in $\UU$ which contains the origin by a pair of subsets, one containing $0_+$ and one containing $0_-$.
\end{proof}

We show that $M$ is not Hausdorff.

\begin{proof}
    Consider the points $0_\pm \in M$.
    Let $U \seq M$ and $V \seq M$ be open neighborhoods of $0_+$ and $0_-$, respectively.
    Both $U$ and $V$ correspond to neighborhoods of the origin in $\R$.
    We can find some $\eps > 0$ small enough that it is contained in both of these neighborhoods.
    In which case, we the points $[\eps] \in M$ is an element of both $U$ and $V$, hence the two are not disjoint.
\end{proof}


\newpage
\begin{pbox}[Problem 1.2]
    Show that the disjoint union of uncountably many copies of $\R$ is locally Euclidean and Hausdorff, but not second-countable.
\end{pbox}

\begin{proof}
    Let $I$ be an uncountable indexing set.
    For each $\alpha \in I$, let $\R_\alpha$ be a copy of $\R$.
    Denote the disjoint union space $X = \bigsqcup_{\alpha \in I} \R_\alpha$.
    By construction, each $\R_\alpha$ is open subset of $X$ homeomorphic to $\R$.
    Hence, $X$ is locally Euclidean.

    For two distinct points $x, y \in X$, either both points are in the same copy of $\R$ or they are in different copies.
    In the first case, then the fact that $\R$ is Hausdorff allows us to choose a pair of disjoint open neighborhoods.
    In the second case, the open neighborhoods are simply the respective copies of $\R$, say $\R_\alpha$ and $\R_\beta$.
    Hence, $X$ is Hausdorff.

    Let $\UU$ be a base for $X$.
    For each $\alpha \in I$, there must be some $U_\alpha \in \UU$ such that $U_\alpha \seq \R_\alpha$, since $\R_\alpha$ is an open set in $X$.
    But then $\{U_\alpha\}_{\alpha \in I}$ is an uncountable collection of distinct sets in $\UU$, implying that $\UU$ must be uncountable.
    Hence, no countable base for $X$ exists
\end{proof}


\newpage
\begin{pbox}[Problem 1.6]
    Let $M$ be a nonempty topological manifold of dimension $n \geq 1$.
    If $M$ has a smooth structure, show that it has uncountably many distinct ones.
    [Hint: first show that for any $s > 0$, $F_s(x) = |x|^{s - 1}x$ defines a homeomorphism from $B^n$ to itself, which is a diffeomorphism if and only if $s = 1$.]
\end{pbox}

\begin{proof}
    Per the hint, we check the properties of $F_s$.
    Since the absolute value and power functions are smooth away from zero, so is $F_s$.
    Moreover, $F_{1/s}$ is the smooth inverse of $F_s$, away from zero.
    Hence $F_s$ is always a diffeomorphism away from zero.
    Next, $F_s$ is continuous at zero, since the limit as $x$ approaches zero is indeed zero, so $F_s$ is always a homeomorphism on $B^n$.
    In the case that $s = 1$, $F_s$ is the identity and therefore a diffeomorphism on $B^n$.
    If $s \ne 1$ then either $F_s$ or its inverse $F_{1/s}$ is not differentiable at the origin, hence $F_s$ is not a diffeomorphism.

    Let $\UU$ be a smooth (maximal) atlas on $M$.
    Fix a point $p \in M$ and a chart $(U, \phi) \in \UU$ at $p$.
    After restricting the codomain and scaling, we may assume that the image of $\phi$ is the unit ball $B^n \seq \R^n$.
    We now modify $\UU$ by removing the point $p$ from all the other charts; denote this new collection of charts by $\UU'$.
    Since $U$ still covers $p$ and the other charts cover the rest of $M$, $\UU'$ is another atlas on $M$.

    For a given $s > 0$, we create a collection of charts $\UU_s$ which contains all the same charts at $\UU'$. but replacing $(U, \phi)$ with $(U, F_s \circ \phi)$.
    Now for any other chart $(V, \psi) \in \UU_s$, we know that $p \notin U \cap V$.
    Moreover, $F_s$ is a diffeomorphism away from the origin, so
    \[
        \psi \circ (F_s \circ \phi)^{-1}
            = \psi \circ \phi^{-1} \circ F_s^{-1}
    \]
    is a diffeomorphism by composition from $\phi(U \cap V)$ to $\psi(U \cap V)$.
    In other words, $(U, F_s \circ \phi)$ is smoothly compatible with the rest of the charts in $\UU_s$.
    And since the rest of the charts are unchanged, we know they remain smoothly compatible with each other.

    To see that the $\UU_s$'s are distinct, we look near the point $p$.
    We check to see if the charts $(U, F_s \circ \phi)$ and $(U, F_t \circ \phi)$ are smoothly compatible for $s \ne t$.
    Consider
    \[
        (F_s \circ \phi) \circ (F_t \circ \phi)^{-1}
            = F_s \circ \phi \circ \phi^{-1} \circ F_t^{-1}
            = F_s \circ F_{1/t}
            = F_{s/t}.
    \]
    Since $s \neq t$, then $s/t \ne 1$, so $F_{s/t}$ is not a diffeomorphism.
    Hence, $\UU$ adn $\UU'$ are distinct atlases.
\end{proof}

\newpage
\begin{pbox}[Problem 1.7]
    Let $N$ denote the north pole $(0, \dots, 0, 1) \in S^n \seq \R^{n+1}$, and let $S$ denote the south pole $(0, \dots, 0, -1)$.
    Define the stereographic projection $\sigma : S^n \setminus \{N\} \to \R^n$ by
    \[
        \sigma(x^1, \dots, x^{n + 1}) = \frac{(x^1, \dots, x^n)}{1 - x^{n+1}}.
    \]
    Let $\tilde{\sigma}(x) = -\sigma(-x)$ for $x \in S^n \setminus \{S\}$. 
\end{pbox}

\begin{pbox}[(a)]
\end{pbox}

The line through $N$ and $x$ is parallel to the vector
\[
    x - N = (x^1, \dots, x^n, x^{n+1} - 1).
\]
We parameterize the line as $(x - N)t + N$ for $t \in \R$.
We solve for where this line crosses the $x^{n + 1} = 0$ plane:
\[
    (x^{n+1} - 1)t + 1 = 0 \implies t = \frac{1}{1 - x^{n + 1}}.
\]
The point of intersection is then
\[
    \frac{(x^1, \dots, x^n, 0)}{1 - x^{n+1}}.
\]
This point is precisely $\sigma(x)$.

The argument is the same for $\tilde{\sigma}$, but with some signs flipped.

\begin{pbox}[(b)]
    
\end{pbox}

We check that the given function is indeed the inverse of $\sigma$.

\begin{align*}
    \sigma\left(\frac{(2u^1, \dots, 2u^n, |u|^2 - 1)}{|u|^2 + 1}\right)
        &= \frac{(2u^1, \dots, 2u^n}{|u|^2 + 1} \cdot \frac{1}{1 - \frac{|u|^2 - 1}{|u|^2 + 1}} \\
        &= \frac{(2u^1, \dots, 2u^n)}{|u|^2 + 1 - (|u|^2 - 1)} \\
        &= \frac{(2u^1, \dots, 2u^n)}{2} \\
        &= (u^1, \dots, u^n).
\end{align*}

\begin{pbox}[(c)]
    
\end{pbox}

\begin{pbox}[(d)]
    
\end{pbox}

\end{document}