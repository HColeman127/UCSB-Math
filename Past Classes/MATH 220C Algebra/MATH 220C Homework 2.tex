\documentclass[12pt]{article}

% Packages
\usepackage[margin=1in]{geometry}
\usepackage{fancyhdr, parskip}
\usepackage{amsmath, amsthm, amssymb}
\usepackage{tikz, tikz-cd}
\usepackage[shortlabels]{enumitem}

% Page Style
\makeatletter
\fancypagestyle{title}{
    \renewcommand{\headrulewidth}{0.4pt}
    \setlength{\headheight}{15pt}
    \fancyhead[R]{\@author}
    \fancyhead[L]{\@title}
    \fancyhead[C]{\@date}
}
\makeatother
\renewcommand{\maketitle}{\thispagestyle{title}}
\fancypagestyle{plain}{
    \fancyhf{}
    \renewcommand{\headrulewidth}{0pt}
    \renewcommand{\footrulewidth}{0pt}
    \fancyfoot[R]{\thepage}
}
\pagestyle{plain}

% Problem Box
\setlength{\fboxsep}{4pt}
\newlength{\myparskip}
\setlength{\myparskip}{\parskip}
\newsavebox{\savefullbox}
\newenvironment{fullbox}{\begin{lrbox}{\savefullbox}\begin{minipage}{\dimexpr\textwidth-2\fboxsep\relax}\setlength{\parskip}{\myparskip}}{\end{minipage}\end{lrbox}\framebox[\textwidth]{\usebox{\savefullbox}}}
\newenvironment{pbox}[1][]{\begin{fullbox}\def\temp{#1}\ifx\temp\empty\else\paragraph{#1}\phantom{}\fi}{\end{fullbox}}

% Theorem Environments
\theoremstyle{definition}
\newtheorem{lemma}{Lemma}

% Tikz Environments
\newenvironment{drawing}{\begin{center}\begin{tikzpicture}}{\end{tikzpicture}\end{center}}
% \tikzcdset{row sep/normal=0pt}
\newenvironment{cd}{\begin{center}\begin{tikzcd}}{\end{tikzcd}\end{center}}

% Default Commands
\newcommand{\isp}[1]{\quad\text{#1}\quad}
\newcommand{\N}{\mathbb{N}} 
\newcommand{\Z}{\mathbb{Z}}
\newcommand{\Q}{\mathbb{Q}}
\newcommand{\R}{\mathbb{R}}
\newcommand{\C}{\mathbb{C}}
\newcommand{\A}{\mathbb{A}}
\renewcommand{\P}{\mathbb{P}}
\newcommand{\eps}{\varepsilon}
\renewcommand{\phi}{\varphi}
\renewcommand{\emptyset}{\varnothing}
\newcommand{\<}{\langle}
\renewcommand{\>}{\rangle}
\newcommand{\iso}{\cong}
\newcommand{\eqc}{\overline}
\newcommand{\clo}{\overline}
\newcommand{\seq}{\subseteq}
\newcommand{\teq}{\trianglelefteq}
\DeclareMathOperator{\id}{id}
\DeclareMathOperator{\im}{im}
\newcommand{\inc}{\hookrightarrow}
\newcommand{\dd}{\mathrm{d}}
\newcommand{\mat}[1]{\begin{bmatrix}#1\end{bmatrix}}

% Extra Commands
\newcommand{\Char}{\operatorname{char}}

% Document
\begin{document}
\title{MATH 220C Homework 2}
\author{Harry Coleman}
\date{June 3, 2022}
\maketitle

\begin{pbox}[1]
    Let $n \geq 3$ be an odd integer such that $\Char F \nmid n$.
\end{pbox}

\begin{pbox}[(a)]
    Show that the cyclotomic polynomial $\Phi_n \in F[X]$ has even degree.
\end{pbox}

\begin{proof}
    We perform (strong) induction on $n$.

    For the base case $n = 3$, we have $\Phi_3 = X^2 + X + 1$, so $\deg\Phi_3 = 2$.

    Assume that the result holds for all odd integers $d$ with $3 \leq d < n$.
    We now write
    \[
        X^n - 1 = \prod_{\substack{1 \leq d \leq n \\ d \mid n}} \Phi_d.
    \]
    Because $n$ is odd, then $d \mid n$ only if $d$ is also odd, so we can write
    \[
        X^n - 1 = \Phi_1 \Phi_n \prod_{\substack{3 \leq d < n \\ d \mid n}} \Phi_d.
    \]
    Then taking degrees, we obtain
    \[
        \deg\Phi_n
            = \deg(X^n - 1) - \deg\Phi_1 - \sum_{\substack{3 \leq d < n \\ d \mid n}} \deg\Phi_d
            = n - 1 - \sum_{\substack{3 \leq d < n \\ d \mid n}} \deg\Phi_d.
    \]
    By the inductive hypothesis, each $\deg\Phi_d$ is even.
    And since $n - 1$ is also even, this equation tells us that $\deg\Phi_n$ is even.
\end{proof}

\begin{pbox}[(b)]
    Now assume that $\Char F \nmid 2n$.
    Show that $\Phi_{2n} = \Phi_n(-X)$.
    [Hint: Show that in an algebraic closure $\clo{F}$ of $F$, the primitive $2n$th roots of unity are exactly the elements $-\zeta$ where $\zeta$ is a primitive $n$th root of unity in $\clo{F}$.]
\end{pbox}

\begin{proof}
    Per the hint, fix an algebraic closure $\clo{F}$ of $F$.

    Suppose $\zeta \in \clo{F}$ is a primitive $n$th root of unity.
    Then
    \[
        (-\zeta)^{2n} = ((-1)^2)^n(\zeta^n)^2 = 1^n \cdot 1^2 = 1,
    \]
    so $-\zeta$ is a $2n$th root of unity.
    Because $\zeta$ is a primitive $n$th root of unity, we know that $\zeta^j \ne 1$ for $1 \leq j < n$.
    Moreover, since $n$ is odd, we also know $\zeta^j \ne -1$ for $1 \leq j < n$, since $-1$ is the primitive $2$nd root of unity.
    It follows that $\zeta^{n + j} = -\zeta^j \ne 1$ for $1 \leq j < n$, and we conclude that $-\zeta$ is a primitive $2n$th root of unity.

    Use similar argument to show $\zeta$ primitive $2n$th root implies $-\zeta$ primitive $n$th root.

    By definition, we have $\Phi_n = \prod (X - \zeta)$, where the product is taken over all $\zeta \in \clo{F}$ primitive $n$th roots of unity.
    Since the $-\zeta$'s are precisely the primitive $2n$th roots of unity, we deduce
    \[
        \Phi_{2n}
            = \prod(X + \zeta)
            = \prod -(-X - \zeta)
            = (-1)^{\deg\Phi_n} \Phi_n(-X)
            = \Phi_n(-X).
    \]
\end{proof}



\newpage
\begin{pbox}[2]
    Suppose $\Char F = 0$, and let $f \in F[X]$ be irreducible.
    Moreover, assume there exists an extension by radicals $F \seq K$ such that $f$ has a root in $K$.
    Show that $f$ is solvable by radicals over $F$.
\end{pbox}

\begin{proof}
    Let $L$ be a normal closure of $K$ over $F$, then Lemma 8.6 tells us that $F \seq L$ is an extension by radicals.
    Since $F \seq L$ is normal and $f \in F[X]$ is irreducible, we know that $f$ must split over $L$.
    Hence, $f$ is solvable by radicals over $F$.
\end{proof}



\newpage
\begin{pbox}[3]
    Let $F \seq K$ be a Galois field extension with finite degree, such that $\Char(F) \ne 2$.
    An \textit{iterated square root extension of $F$} is any field $L$ for which there exists a tower
    \[
        L_0 = F \seq L_1 \seq \cdots \seq L_n = L
    \]
    with $L_i = L_{i-1}(a_i)$ and $a_i^2 \in L_{i-1}$ for $1 \leq i \leq n$.
    Prove that the following conditions are equivalent:
    \begin{enumerate}[(a)]
        \item $K$ is an iterated square root extension of $F$.
        \item $K$ is contained in an iterated square root extension of $F$.
        \item $G(K : F)$ is a $2$-group.
    \end{enumerate}
\end{pbox}

\begin{proof}
    Assuming (a) is true, then (b) follows trivially since $K \seq K$.

    Assume (b) is true---say $K$ is contained in an iterated square root extension $L$ of $F$.
    By repeatedly applying the tower rule to the tower in the definition of $L$, we obtain
    \[
        [L : F] = [L_n : L_{n-1}] \cdots [L_1 : L_0].
    \]
    To compute this degree, we consider each extension $L_i = L_{i-1}(a_i)$ in the tower.
    If $a_i \in L_{i-1}$ then $L_i = L_{i-1}$, so $[L_i : L_{i-1}] = 1$.
    If $a_i \notin L_{i-1}$ then the minimal polynomial of $a_i$ over $L_{i-1}$ is simply $X^2 - a_i^2$, so
    \[
        [L_i : L_{i-1}] = \deg(X^2 - a_i^2) = 2.
    \]
    Thus, $[L_i : L_{i-1}]$ is either $1$ or $2$ for all $i = 1, \dots, n$.
    It follows that $[L : F]$ is a power of $2$, i.e., $[L : F] = 2^m$ for some nonnegative integer $m$.
    Now apply the tower rule to $F \seq K \seq L$ to obtain
    \[
        2^m
            = [L : F]
            = [L : K][K : F].
    \]
    Hence, $[K : F]$ is also a power of $2$, i.e., $G(K : F)$ is a $2$-group.

    Lastly, we prove (c) implies (a) by induction on $m$, where $[K : F] = |G(K : F)| = 2^m$.

    For the base case, assume $[K : F] = 2$.
    Necessarily, we must have $K = F(b)$ where the minimal polynomial of $b$ over $F$ is a quadratic $X^2 + \beta X + \gamma \in F[X]$.
    Since $\Char F \ne 2$, then we can also write $K = F(a)$ for $a = \beta + 2b \in K$.
    Then squaring $a$ gives us
    \[
        a^2
            = (\beta + 2b)^2
            = \beta^2 + 4(b^2 + b\beta)
            = \beta^2 - 4\gamma,
    \]
    which is an element of $F$.
    In other words, $K$ is a square root extension of $F$.

    For the inductive step, assume the result holds for powers of $2$ up to $m - 1$ for $m \geq 2$.
    (That is, we assume that whenever we have a Galois group of order $2^k$ with $1 \leq k \leq m - 1$, the corresponding fields form an iterated square root extension.)
    Recall that, by assumption, $G = G(K : F)$ is a group of order $2^m$.
    We claim that there is a nontrivial proper normal subgroup of $G$.
    To see this, let $G$ act on itself via conjugation and let $x_1, \dots, x_r \in G$ be representatives of the non-center conjugacy classes, then the class equation gives us
    \[
        |G| = |Z(G)| + \sum_{i=1}^{r} [G : C_G(x_i)],
    \]
    where $Z(G)$ is the center of $G$ and $C_G(x_i)$ is centralizer of $x_i$ in $G$.
    Note that each $[G : C_G(x_i)]$ is a positive power of $2$---say $2^{k_i}$ for $k_i \geq 1$.
    Then we write
    \[
        2^m = |Z(G)| + \sum_{i=1}^{r} 2^{k_i}.
    \]
    Since $Z(G)$ contains at least the identity, then $|Z(G)| \geq 1$, but in order for the above equation to hold we must in fact have $|Z(G)| > 1$.
    In other words, the center of $G$ is nontrivial.
    
    We may now choose $H \teq G$ to be a nontrivial proper normal subgroup: if $G$ is abelian then any nontrivial proper subgroup will do, otherwise we take the center of $G$.
    By the main theorem of Galois theory, the fixed field $L = \operatorname{Fix}_K(H)$ is a intermediate field $F \seq L \seq K$ with both extensions $F \seq L$ and $L \seq K$ nontrivial and Galois.
    By the tower rule,
    \[
        2^m = [K : F] = [K : L][L : F],
    \]
    which implies
    \[
        |G(K : L)| = [K : L] = 2^{k}
        \isp{and}
        |G(L : F)| = [L : F] = 2^\ell,
    \]
    where $1 \leq k, \ell \leq m - 1$.
    By the inductive hypothesis, both $F \seq L$ and $L \seq K$ are iterated square root extensions.
    Joining their respective towers from the definition at $L$ shows that $F \seq K$ is also an iterated square root extension.
\end{proof}

\end{document}