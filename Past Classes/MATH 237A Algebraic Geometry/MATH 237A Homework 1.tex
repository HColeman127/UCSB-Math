\documentclass[12pt]{article}

% Packages
\usepackage[margin=1in]{geometry}
\usepackage{fancyhdr, parskip}
\usepackage{amsmath, amsthm, amssymb}
\usepackage{tikz}

\usepackage[absolute,overlay]{textpos}

% Page Style
\fancypagestyle{plain}{
    \fancyhf{}
    \renewcommand{\headrulewidth}{0pt}
    \renewcommand{\footrulewidth}{0pt}
    \fancyfoot[R]{\thepage}
}
\pagestyle{plain}

% Problem Box
\setlength{\fboxsep}{4pt}
\newlength{\myparskip}
\setlength{\myparskip}{\parskip}
\newsavebox{\savefullbox}
\newenvironment{fullbox}{\begin{lrbox}{\savefullbox}\begin{minipage}{\dimexpr\textwidth-2\fboxsep\relax}\setlength{\parskip}{\myparskip}}{\end{minipage}\end{lrbox}\framebox[\textwidth]{\usebox{\savefullbox}}}
\newenvironment{pbox}[1][]{\begin{fullbox}\ifx#1\empty\else\paragraph{#1}\fi}{\end{fullbox}}

% Drawing
\newenvironment{drawing}{\begin{center}\begin{tikzpicture}}{\end{tikzpicture}\end{center}}

% Theorem Environments
%\theoremstyle{definition}
%\newtheorem{proposition}{Proposition}
%\newtheorem{lemma}{Lemma}

% Options
%\allowdisplaybreaks
%\addtolength{\jot}{4pt}

% Default Commands
\newcommand{\isp}[1]{\quad\text{#1}\quad}
\newcommand{\N}{\mathbb{N}} 
\newcommand{\Z}{\mathbb{Z}}
\newcommand{\Q}{\mathbb{Q}}
\newcommand{\R}{\mathbb{R}}
\newcommand{\C}{\mathbb{C}}
\newcommand{\eps}{\varepsilon}
\renewcommand{\phi}{\varphi}
\renewcommand{\emptyset}{\varnothing}
\newcommand{\<}{\langle}
\renewcommand{\>}{\rangle}
\newcommand{\isom}{\cong}
\newcommand{\eqc}{\overline}
\newcommand{\clo}{\overline}

% Extra Commands
\DeclareMathOperator{\im}{im}
\DeclareMathOperator{\Hom}{Hom}
\DeclareMathOperator{\Frac}{Frac}
\newcommand{\A}{\mathbb{A}}

% Document Info
\fancypagestyle{title}{
    \renewcommand{\headrulewidth}{0.4pt}
    \setlength{\headheight}{15pt}
    \fancyhead[R]{Harry Coleman}
    \fancyhead[L]{MATH 237A Homework 1}
    \fancyhead[C]{October 1, 2021}
}

% Begin Document
\begin{document}
\thispagestyle{title}

\begin{textblock*}{0in}(7.5in, 0.75in)
    \makebox[0pt][r]{(worked w/ Joseph Sullivan)}
\end{textblock*}

\begin{pbox}[1 Exercise 1.1]
    Prove that the following conditions of a module $M$ over a commutative ring $R$ are equivalent.
    \begin{enumerate}
        \item $M$ is Noetherian (that is, every submodule of $M$ is finitely generated).
        \item Every ascending chain of submodules of $M$ terminates.
        \item Every set of submodules of $M$ contains elements maximal under inclusion.
        \item Given any sequence of elements $f_1, f_2, \ldots \in M$, there is a number $m$ such that for each $n > m$ there is an expression $f_n = \sum_{i = 1}^{m} a_if_i$ with $a_i \in R$.
    \end{enumerate}
\end{pbox}

\begin{proof}
    We will prove the following implications between the conditions:
    \[
        1 \implies 2 \implies 3 \implies 4 \implies 1.
    \]

    Assume condition 1 holds and let $N_1 \leq N_2 \leq \cdots \leq M$ be an ascending chain of submodules. We check that $N = \bigcup_{i \in \N} N_i \subseteq M$ is a submodule (i.e., subgroup) of $M$. If $x, y \in N$, then $x, y \in N_n$ for some $n \in \N$ large enough. Since $N_n \leq M$, then in fact $xy, x^{-1} \in N_n \subseteq N$. Applying condition 1, we have $N = R\{x_1, \dots, x_m\}$, where each $x_i \in N$, implying $x_i \in N_{n_i}$ for some $n_i \in \N$. Defining $n = \max n_i$, we obtain $x_1, \dots, x_m \in N_n$. Since $N_n \leq N$ and $N$ is generated by the $x_i$'s, which are contained in $N_n$, then $N_n = N$. Moreover, for all $k \geq n$, we have $N_n \leq N_k \leq N$, so $N_k = N$. In other words, the chain of submodules terminates (after at most $n$ items), i.e., condition 2 holds.

    (Condition 3 is immediately obtained from condition 2 by an application of Zorn's lemma, but can also be shown without the full axiom of choice. In particular, we can assume dependent choice.)

    Assume condition 2 holds and let $S$ be a set of submodules of $M$. Suppose, for contradiction, that $S$ contains no elements maximal under inclusion. Then, by the axiom of dependent choice, there is a strictly increasing chain $N_1 < N_2 < \cdots $ of submodules in $S$. This contradiction condition 2, so we conclude that $S$ must contain elements maximal under inclusion, i.e., condition 3 holds.

    Assume condition 3 holds and let $f_1, f_2, \ldots \in M$ be a sequence of elements. For each $n \in \N$, define the submodule $N_n = R\{f_1, \dots, f_n\} \leq M$. By condition 3, the set $\{N_n \mid n \in \N\}$ contains some $N_m$ which is maximal under inclusion. Given $n \geq m$, we know $N_m \leq N_n$, but $N_m$ is maximal, so $N_n = N_m$. By construction of the submodules, we now have
    \[
        f_n \in N_n = N_m = R\{f_1, \dots, f_m\},
    \]
    so there exist $a_1, \dots, a_m \in R$ such that $f_n = \sum_{i=1}^{m} a_if_i$, i.e., condition 4 holds.

    Assume condition 4 holds and let $N$ be a submodule of $M$. We construct a sequence of elements of $M$ inductively as follows (implicitly using dependent choice). Take $f_1$ to be any element of $N$. For $n \geq 2$, there are two cases: if $N \ne R\{f_1, \dots, f_{n-1}\}$, choose $f_n \in N \setminus R\{f_1, \dots, f_{n-1}\}$; otherwise, choose $f_n \in N$ arbitrarily. Applying condition 4, there is some $m$ such that $f_n \in R\{f_1, \dots, f_m\}$ for all $n \geq m$. In particular, $f_{m+1} \in R\{f_1, \dots, f_m\}$, which means that we must have $N = R\{f_1, \dots, f_m\}$. Hence, $N$ is finitely generated, i.e., condition 1 holds.

\end{proof}



\newpage
\begin{pbox}[2 Exercise 1.3]
    Let $M'$ be a submodule of $M$. Show that $M$ is Noetherian iff both $M'$ and $M/M'$ are Noetherian.
\end{pbox}

\begin{proof}
    Suppose $M$ is Noetherian. Any submodule of $M'$ is also a submodule of $M$ and, therefore, finitely generated. For any submodule $N \leq M/M'$, its preimage under the natural projection $\pi : M \to M/M'$ is a submodule of $M$ and, therefore, finitely generated. Supposing $\pi^{-1}(N) = R\{x_1, \dots x_m\}$, we obtain $N = R\{\pi(x_1), \dots, \pi(x_m)\}$. Hence, both $M'$ and $M/M'$ are Noetherian.

    Suppose $M'$ and $M/M'$ are Noetherian. Let $N$ be a submodule of $M$, then we have the submodules $N \cap M' \leq M'$ and $ (N + M')/M' \leq M/M'$. Since $M'$ and $M/M'$ are Noetherian, we have $N \cap M' = R\{x_1, \dots, x_n\}$ with $x_i \in N \cap M'$ and $(N + M')/M' = R\{\eqc{y_1}, \dots, \eqc{y_m}\}$ with $y_j \in N + M'$. Without loss of generality, we may assume all $y_j \in N$, since adding any element of $M'$ does not change the equivalence class modulo $M'$. We claim that
    \[
        N = R\{x_1, \dots, x_n, y_1, \dots, y_m\}.
    \]
    As $x_i, y_j \in N$, the left inclusion is immediate, so it remains to prove the right inclusion. For any $f \in N$, we have $\eqc{f} \in (N + M')/M'$, so $\eqc{f} = \sum_{j=1}^{m} b_j\eqc{y_j}$ for some $b_j \in R$. That is, $f = g + \sum_{j=1}^{m} a_jy_j$, for some $g \in M'$. Since $f, y_j \in N$, then we must also have $g \in N$, implying that $g \in N \cap M'$. So $g = \sum_{i=1}^{n} a_ix_i$ for some $a_i \in R$, giving us
    \[
        f 
            = \sum_{i=1}^{n} a_ix_i + \sum_{j=1}^{m} b_jy_j
            \in R\{x_1, \dots, x_n, y_1, \dots, y_m\}.
    \]
\end{proof}



\newpage
\begin{pbox}[3 Exercise 1.7]
    \,
\end{pbox}

\begin{pbox}[(a)]
    Suppose that $k$ is a field of characteristic $\ne 2$. Let the generator $g$ of the group $G := \Z/2$ act on the polynomial ring $k[x, y]$ in two variables by sending $x$ to $-x$ and $y$ to $-y$. Show that the ring of invariants is $k[x^2, xy, y^2]$.
\end{pbox}

\begin{proof}
    Denote $k[x^2, xy, y^2]$. Since $G$ is the identity on $k$ and invariant on the generators $x^2$, $xy$, and $y^2$, we can immediately see that $k[x^2, xy, y^2] \subseteq k[x, y]^G$.

    Degrees are preserved under the action of $G$, so any polynomial in $k[x, y]^G$ must be a $k$-linear combination of $G$-invariant monomials. Therefore, to show the opposite inclusion, it suffices to prove that all $G$-invariant monomials in $k[x, y]$ are contained in $k[x^2, xy, y^2]$.

    Suppose $x^ay^b \in k[x, y]^G$, where $a, b \in \Z_{\geq 0}$. Then
    \[
        x^ay^b 
            = g \cdot x^ay^b
            = (-x)^a(-y)^b
            = (-1)^{a + b}x^ay^b,
    \]
    which implies $a + b \equiv 0 \pmod{2}$. This means that $a$ and $b$ are either both even or both odd. In the first case, write $a = 2n$, $b = 2m$, then
    \[
        x^ay^b
            = x^{2n}y^{2m}
            = (x^2)^n(y^2)^m
            \in k[x^2, xy, y^2].
    \]
    In the second case, write $a = 2n + 1$, $b = 2m + 1$, then
    \[
        x^ay^b
            = x^{2n+1}y^{2m+1}
            = (x^2)^nxy(y^2)^m
            \in k[x^2, xy, y^2].
    \]
    Hence, $x^ay^b \in k[x^2, xy, y^2]$ for all monomials $x^ay^b \in k[x, y]^G$.

\end{proof}

\begin{pbox}
    Prove that $k[x^2, xy, y^2] \isom k[u, v, w] / (uw - v^2)$.
\end{pbox}

\begin{proof}
    Consider the map
    \begin{align*}
        \phi : k[u, v, w] &\to k[x^2, xy, y^2], \\
            p(u, v, w) &\mapsto p(x^2, xy, y^2),
    \end{align*}
    on which we will use the first isomorphism theorem to obtain the result.

    This map is surjective as $u, v, w$ map to $x^2, xy, y^2$, respectively. We claim that its kernel is precisely the ideal $(uw - v^2)$. Since $\phi(uw - v^2) = x^2y^2 - (xy)^2 = 0$, the ideal is contained in the kernel.
    
    Let $p(u, v, w) \in \ker\phi$, so $p(x^2, xy, y^2) = 0$. We perform polynomial long division on $p$ by $v^2 - uw$ in $(k[u, w])[v]$, giving us
    \[
        p = (v^2 - uw)q + r,
    \]
    for some $q, r \in k[u, v, w]$, where the $v$-degree of $r$ is at most one. We claim that $r = 0$. We have
    \[
        r(u, v, w) = r_1(u, w) + vr_2(u, w),
    \]
    for some $r_1, r_2 \in k[u, w]$, then
    \[
        0
            = p(x^2, xy, y^2)
            = r(x^2, y^2)
            = r_1(x^2, y^2) + xyr_2(x^2, y^2).
    \]
    Note that $r_1(x^2, y^2)$ has only even powers of $x$ and $y$, while $xyr_2(x^2, y^2)$ has only odd powers, i.e., the two share no terms. So
    \[
        0 = r_1(x^2, y^2) \isp{and} 0 = xyr_2(x^2, y^2) \implies 0 = r_2(x^2, y^2).
    \]
    Note that $\phi|_{k[u, w]}$ defines an isomorphism $k[u, w] \isom k[x^2, y^2]$. So in fact $r_1 = r_2 = 0 \in k[u, w]$. Hence, $r = 0$, so $p \in (uw - v^2)$.

\end{proof}

\begin{pbox}
    Show that this is not isomorphic to any polynomial ring over a field.
\end{pbox}

\begin{proof}
    A polynomial ring over a field is a UFD, so any ring which is not a UFD cannot be isomorphic to a polynomial ring over a field. We claim that $R = k[x^2, xy, y^2]$ is not a UFD. In particular, the element $x^2y^2$ has the decompositions
    \[
        (xy)(xy) = x^2y^2 = (x^2)(y^2).
    \]
    It can be seen that $x^2, xy, y^2$ are irreducible in $R$, as a decomposition in $R$ would also be a decomposition in $k[x, y]$. The only decompositions in $k[x, y]$ are $x^2 = xx, xy, y^2 = yy$. Since $x, y \notin R$, then these decompositions are not possible in $R$. Hence the elements in question are irreducible. Thus, we have two distinct irreducible decompositions of $x^2y^2$ in $R$, so $R$ is not a UFD.
\end{proof}

\begin{pbox}[(b)]
    More generally, let $G$ be any finite abelian group, acting linearly on the space of linear forms of the ring $S = k[x_1, \dots, x_r]$. Assume that $G$ acts by characters; that is, assume that there are homomorphisms $\alpha_i : G \to k^\times$, and $g(x_i) = \alpha_i(g)x_i$ for all $g \in G$, where $k^\times$ is the multiplicative group of the field $k$. Show that the invariants of $G$ are generated by those monomials $\prod x_i^{a_i}$ whose exponent vectors $(a_1, \dots, a_r)$ are in the kernel of a map from $\Z^r$ to a certain finite abelian group. 
\end{pbox}

\begin{proof}
    Since $\alpha_i(g) \in k^\times$, the action of $G$ on $S$ preserves degrees, so any polynomial in $S^G$ must be a $k$-linear combination of $G$-invariant monomials. For $x = (x_1, \dots, x_r)$ and $a = (a_1, \dots, a_r) \in \Z_{\geq 0}^r$, denote by $x^a$ the product $\prod x_i^{a_i}$. Define the map
    \begin{align*}
        \alpha : G &\to (k^\times)^r, \\
            g &\mapsto (\alpha_1(g), \dots, \alpha_r(g)),
    \end{align*}
    which is a group homomorphism since each component $\alpha_i$ is a group homomorphism. Then the action of $g$ on a monomial $x^a \in S$ is given by
    \[
        g \cdot x^a
            = (\alpha(g)x)^a
            = \alpha(g)^a x^a.
    \]
    This means that $x^a$ is $G$-invariant if and only if $\alpha(g)^a = 1$ for all $g \in G$. In other words, $x^a$ is $G$-invariant if and only if $a$ is in the kernel of the group homomorphism
    \begin{align*}
        \Phi : \Z^r &\to \Hom(G, k^\times), \\
            a &\mapsto \alpha(-)^a.
    \end{align*}
    (One can check that the map $g \mapsto \alpha(g)^a = \prod \alpha_i(g)^{a_i}$ is a group homomorphism $G \to k^\times$.) Since $\Hom(G, k^\times)$ is a group under multiplication (i.e., for $\phi, \psi \in \Hom(G, k^\times)$, $\phi\psi$ is the homomorphism $g \mapsto \phi(g)\psi(g)$), then $k^\times$ being abelian implies $\Hom(G, k^\times)$ is abelian.

    Lastly, we check that $\Hom(G, k^\times)$ is finite. Since $G$ is finite, then each $g \in G$ has some finite order $n \in \Z$, with $g^n = 1$. Then for any $\phi \in \Hom(G, k^\times)$, we have $\phi(g)^n = 1$, which means that $\phi(g)$ is some $n$th root of unity in $k$. Since there are only finitely many $n$th roots of unity, then each $g \in G$ has only finitely many possible destinations in $k^\times$ under a group homomorphism. And since there are only finitely many elements in $G$, we have an upper bound $\prod_{g \in G} |g|$ on the order of $\Hom(G, k^\times)$.

\end{proof}

\begin{pbox}
    Conclude that the quotient field of $S^G$ is isomorphic to a field of rational functions in $r$ variables.
\end{pbox}

\begin{proof}
    The monomials in $S^G$ are a semigroup under multiplication, and generate a multiplicative group $M$ isomorphic to the additive group $\ker\Phi \leq \Z^r$, under the map $x^a \leftrightarrow a$. This $M$ is precisely the set of monomials in the fraction (quotient) field $\Frac S^G$, which means that we have $\Frac S^G = \Frac k[M]$.
    
    Equivalently, we may consider these as modules over the PID $\Z$. Then $\ker\Phi$ is a free $\Z$-module, and we can choose a basis $x_1, \dots, x_r$ for $\Z^r$ such that $a_1x_1, \dots, a_sx_s$ is a basis for $\ker\Phi$, for some $0\leq s \leq r$ and $a_i \in \Z$.
    
    Now, $M \isom \ker\Phi \isom \Z^s$ and we claim that $s = r$. We have
    \[
        \Z^{r - s} \oplus \Z/a_1\Z \oplus \cdots \oplus \Z/a_s\Z
            \isom \Z^r/\ker\Phi
            \isom \im\Phi
            \leq \Hom(G, k^\times).
    \]
    Since $\Hom(G, k^\times)$ is finite, then we must have $r - s = 0$.Then, there is a basis $y_1, \dots, y_r$ for $M$ as a free $\Z$-module, which gives us $k[M] = k[y_1, \dots, y_r]$.
    
\end{proof}


\newpage
\begin{pbox}[4 Exercise 1.10]
    Find rings to represent the following figures.
    \begin{drawing}[line width=1pt]
        \draw (0, 0) circle (1);
        \draw plot[smooth, domain=-2:2] (\x, {0.5*\x*\x - 1});

        \draw (3, -1) -- (5, 1);
        \draw[line width=3pt, white] (3, 1) -- (5, -1);
        \draw (3, 1) -- (5, -1);
    \end{drawing}
    The first represents the union of a circle and a parabola in the plane, and the second shows the union of two skew lines in $3$-space.
\end{pbox}

The circle is the zero locus of $f = x^2 + y^2 - 1$ and the parabola is the zero locus of $g = x^2 - 2y - 2$. Their union is $Z(fg)$, so the coordinate ring of this affine variety is $k[x, y]/\sqrt{\<fg\>}$.

After an affine transformation of $\A^3$, we can assume one of the skew lines is the $z$-axis, which is $Z(x, y)$. The other line is a codimension $2$ subvariety of $\A^3$, so it is the zero locus of two polynomials $f, g \in k[x, y, z]$. Their union is the zero locus of the ideal $I = \<x, y\>\<f, g\>$, so the coordinate ring is $k[x, y, z]/\sqrt{I}$

\end{document}