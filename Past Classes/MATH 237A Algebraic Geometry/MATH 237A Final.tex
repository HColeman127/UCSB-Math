\documentclass[12pt]{article}

% Packages
\usepackage[margin=1in]{geometry}
\usepackage{fancyhdr, parskip}
\usepackage{amsmath, amsthm, amssymb}
\usepackage{tikz, tikz-cd}

% Page Style
\makeatletter
\fancypagestyle{title}{
    \renewcommand{\headrulewidth}{0.4pt}
    \setlength{\headheight}{15pt}
    \fancyhead[R]{\@author}
    \fancyhead[L]{\@title}
    \fancyhead[C]{\@date}
}
\makeatother
\renewcommand{\maketitle}{\thispagestyle{title}}
\fancypagestyle{plain}{
    \fancyhf{}
    \renewcommand{\headrulewidth}{0pt}
    \renewcommand{\footrulewidth}{0pt}
    \fancyfoot[R]{\thepage}
}
\pagestyle{plain}

% Problem Box
\setlength{\fboxsep}{4pt}
\newlength{\myparskip}
\setlength{\myparskip}{\parskip}
\newsavebox{\savefullbox}
\newenvironment{fullbox}{\begin{lrbox}{\savefullbox}\begin{minipage}{\dimexpr\textwidth-2\fboxsep\relax}\setlength{\parskip}{\myparskip}}{\end{minipage}\end{lrbox}\framebox[\textwidth]{\usebox{\savefullbox}}}
\newenvironment{pbox}[1][]{\begin{fullbox}\ifx#1\empty\else\paragraph{#1}\phantom{}\fi}{\end{fullbox}}

% Theorem Environments
\theoremstyle{definition}
\newtheorem{lemma}{Lemma}

% Tikz Environments
\newenvironment{drawing}{\begin{center}\begin{tikzpicture}}{\end{tikzpicture}\end{center}}
\tikzcdset{row sep/normal=0pt}
\newenvironment{cd}{\begin{center}\begin{tikzcd}}{\end{tikzcd}\end{center}}

% Default Commands
\newcommand{\isp}[1]{\quad\text{#1}\quad}
\newcommand{\N}{\mathbb{N}} 
\newcommand{\Z}{\mathbb{Z}}
\newcommand{\Q}{\mathbb{Q}}
\newcommand{\R}{\mathbb{R}}
\newcommand{\C}{\mathbb{C}}
\newcommand{\A}{\mathbb{A}}
\renewcommand{\P}{\mathbb{P}}
\newcommand{\eps}{\varepsilon}
\renewcommand{\phi}{\varphi}
\renewcommand{\emptyset}{\varnothing}
\newcommand{\<}{\langle}
\renewcommand{\>}{\rangle}
\newcommand{\isom}{\cong}
\newcommand{\eqc}{\overline}
\newcommand{\clo}{\overline}
\newcommand{\teq}{\trianglelefteq}
\DeclareMathOperator{\id}{id}
\DeclareMathOperator{\im}{im}

% Extra Commands
\newcommand{\mat}[1]{\begin{bmatrix}#1\end{bmatrix}}
\newcommand{\vmat}[1]{\begin{vmatrix}#1\end{vmatrix}}
\newcommand{\pdv}[3][]{\frac{\partial^{#1}#2}{\partial{#3}^{#1}}}

% Document
\begin{document}
\title{MATH 237A Final}
\author{Harry Coleman}
\date{December 9, 2021}
\maketitle

\begin{pbox}
    This exam concerns the cubic surface $S$ in $\P^3$ with homogeneous equation
    \[
        f = x^3 - xy^2 + z^3 - zt^2 = 0.
    \]
    You may make any assumptions about the field $k$ as you wish, but please be sure to state your assumptions.
\end{pbox}

Probably assuming algebraically closed and characteristic zero, i.e., just $\C$.

\begin{pbox}[1]
    Verify that the cubic surface $S$ is nonsingular, and that it contains the line $\ell = \{x = z = 0\}$.
\end{pbox}

By the Jacobi criterion, $S$ is singular only where the rank of the following matrix is zero:
\[
    \mat{\pdv{f}{x} & \pdv{f}{y} & \pdv{f}{z} & \pdv{f}{t}}
        = \mat{3x^2 - y^2 & -2xy & 3z^2 - t^2 & -2zt}.
\]
Assuming this matrix is zero, we deduce
\[
    -2xy = 0, \quad 3x^2 = y^2 \quad\implies\quad x = y = 0,
\]
and the same implies $z = t = 0$.
However, the homogeneous coordinate with all zeros is not a valid projective point, therefore $S$ is nonsingular at every point.

Note $S = Z(f)$.
We evaluate $f$ at an arbitrary point $P = [0 : y : 0 : t]$ in the line $\ell$:
\[
    f(P) = 0^3 - 0y^2 + 0^3 - 0t^2 = 0.
\]
That is, $P \in Z(f)$, so in fact $\ell \subseteq S$.


\newpage
\begin{pbox}[2]
    Show that $S$ contains $10$ other lines $\ell_i$, $i = 1, \dots, 10$, none of which coincides with $\ell$ and each of which meets $\ell$.

    \textit{Hint:} Consider the natural $1$-parameter family of projective planes containing $\ell$ and find a condition on the parameter which guarantees that the intersection with the plane is singular. 
\end{pbox}

A plane in $\P^3$ is given by $Z(g)$ for some homogeneous linear polynomial $g = ax + by + cz + dt$.
In order for $\ell \subseteq Z(g)$, we must have $g(P) = 0$ for all $P = [0 : y : 0 : t] \in \ell$, i.e.,
\[
    0 = g(P) = a \cdot 0 + by + c \cdot 0 + dt = by + dt.
\]
In particular,
\[
    0 = g(0, 1, 0, 0) = b
    \isp{and}
    0 = g(0, 0, 0, 1) = d.
\]
So we can write $g = ax + cz$ for some $[a : c] \in \P^1$,
where distinct choices of $[a : c]$ correspond to distinct planes $Z(g) \subseteq \P^3$ containing the line $\ell$.

In the case that $c \ne 0$, we may assume $c = 1$ giving us the following isomorphism:
\begin{align*}
    \P^2 &\longrightarrow Z(g) \subseteq \P^3 \\
    [x : y : t] &\longmapsto [x : y : -ax : t]
\end{align*}
In this copy of $\P^2$, the variety corresponding to $Z(f, g) \subseteq \P^3$ is defined by the homogeneous cubic polynomial
\[
    F(x, y, t)
        = f(x, y, -ax, t)
        = x\left((1 - a^3)x^2 - y^2 + at^2\right).
\]
The factor of $x$ tells us that $Z(F) \subseteq \P^2$ contains the line $Z(x) \subseteq \P^2$, which corresponds to the line $\ell \subseteq \P^3$.

Let $G = (1 - a^3)x^2 - y^2 + at^2$ be the homogeneous degree $2$ factor of $F$, so that $F = xG$ and $Z(F) = Z(x) \cup Z(G)$.
From Homework 2 Problem 2, we know that $Z(G)$ is either a single line, two coincident lines, or a smooth quadratic curve.
In particular, we require that $Z(G)$ is singular; by the Jacobi criterion, the following matrix must be rank zero (char $k \ne 2$):
\[
    \mat{\pdv{G}{x} & \pdv{G}{y} & \pdv{G}{t}}
        = \mat{2(1 - a^3)x & -2y & 2at}.
\]
If $x \ne 0$ then we must have $a^3 = 1$, giving us three possible values for $a$: $1$, $\zeta$, and $\zeta^2$, where $\zeta = e^{2\pi i/3}$ (i.e., some primitive third root of unity for $k$ algebraically closed).
In any of these cases, we obtain $G = -y^2 + at^2$, which has two distinct homogeneous linear factors, corresponding to the lines $Z(y \pm \sqrt{a}t) \subseteq \P^2$ (for some choice of square root).
If $x = 0$ then we must have $t \ne 0$, which implies $a = 0$.
In which case, $G = x^2 - y^2$ has homogeneous linear factors corresponding to lines $Z(x \pm y) \subseteq \P^2$.

In the case that $c = 0$, we may assume $a = 1$ giving us the following isomorphism:
\begin{align*}
    \P^2 &\longrightarrow Z(g) \subseteq \P^3 \\
    [y : z : t] &\longmapsto [0 : y : z : t]
\end{align*}
In this copy of $\P^2$, the variety corresponding to $Z(f, g) \subseteq \P^3$ is defined by the homogeneous cubic polynomial
\[
    F(y, z, t) 
        = f(0, y, z, t)
        = z^3 - zt^2
        = z(z + t)(z - t).
\]
Similar to the previous case, the factor of $z$ corresponds to the line $Z(z) \subseteq \P^2$ which, in turn, corresponds to the line $\ell \subseteq \P^3$.
The other two homogeneous linear factors of $F$ correspond to the lines $Z(z \pm t) \subseteq \P^2$.

Hence, for $g = ax + cz$ we have five possible values for $[a : c] \in \P^1$:
\[
    [1 : 0], \quad [0 : 1], \quad [1 : 1], \quad [\zeta : 1], \quad [\zeta^2 : 1].
\]
Each of these define a plane $Z(g) \subseteq \P^3$ such that the intersection with $S$ is the union of three mutually coincident lines, one of which is $\ell$.
For distinct values of $[a, c]$ we get distinct planes, so we obtain ten lines in $S$ intersecting $\ell$:
\begin{align*}
    \ell_1 &= Z(x, z + t), & \ell_2 &= Z(x, z - t), \\
    \ell_3 &= Z(z, x + y), & \ell_4 &= Z(z, x - y), \\
    \ell_5 &= Z(x + z, y + t), & \ell_6 &= Z(x + z, y - t), \\
    \ell_7 &= Z(\xi^2 x + z, y + \xi t), & \ell_8 &= Z(\xi^2 x + z, y - \xi t), \\
    \ell_9 &= Z(\xi^4 x + z, y + \xi^2 t), & \ell_{10} &= Z(\xi^4 x + z, y - \xi^2 t),
\end{align*}
where $\xi = \sqrt{\zeta} = e^{\pi i /3}$ is some primitive sixth root of unity.





\newpage
\begin{pbox}[3]
    Show that $S$ contains $8$ other lines $\ell_j$, $j = 11, \dots, 18$ none of which coincides with a previously constructed line, and each of which meets $\ell_1$.
    (There are also two previously constructed lines which meet $\ell_1$.)
\end{pbox}

We will use $\ell_1 = Z(x, z + t)$ from Problem 2.
Again, we consider a plane $Z(g) \subseteq \P^3$ for some homogeneous linear polynomial $g = ax + by + cz + dt$.
In order for $\ell_1 \subseteq Z(g)$, we must have $g(P) = 0$ for all $P = [0 : y : z : -z] \in \ell_1$, i.e.,
\[
    0 = g(P) = a \cdot 0 + by + cz + d(-z) = by + (c - d)z.
\]
In particular,
\[
    0 = g(0, 1, 0, 0) = b
    \isp{and}
    0 = g(0, 0, 1, 0) = c - d.
\]
So we can write $g = ax + c(z + t)$ for some $[a : c] \in \P^1$.
Again, distinct choices of $[a : c]$ give distinct planes $Z(g) \subseteq \P^3$ containing $\ell_1$.

In the case that $c \ne 0$, we may assume $c = 1$ giving us the following isomorphism:
\begin{align*}
    \P^2 &\longrightarrow Z(g) \subseteq \P^3 \\
    [x : y : z] &\longmapsto [x : y : z : -ax - z]
\end{align*}
In this copy of $\P^2$, the variety corresponding to $Z(f, g) \subseteq \P^3$ is defined by the homogeneous cubic polynomial
\[
    F(x, y, z)
        = f(x, y, z, -ax - z)
        = x(x^2 - y^2 - a^2xz - 2az^2).
\]
The factor of $x$ corresponds to the line $Z(x) \subseteq \P^2$ which, in turn, corresponds to the line $\ell_1 \subseteq \P^3$.

As in Problem 2, let $G = x^2 - y^2 - a^2xz - 2az^2$ be the homogeneous degree $2$ factor of $F$, then $Z(G)$ is singular where the following matrix is rank zero:
\[
    \mat{\pdv{G}{x} & \pdv{G}{y} & \pdv{G}{z}} 
        = \mat{2x - a^2z & -2y & -a^2x - 4az}.
\]
The third entry tells us that either $a = 0$ or $z = -ax/4$.
If $a = 0$ then $G = x^2 - y^2$, giving us the lines $Z(x \pm y) \subseteq \P^2$.

If $z = -ax/4$ then the first entry tells us $(8 + a^3)x = 0$.
Then $y = 0$ and $a \ne 0$ implies both $x$ and $z = -az/4$ are nonzero, so in fact $a^3 = -8$.
Then we have three possibilities for $a$: $-2$, $-2\zeta$, and $-2\zeta^2$, where $\zeta = e^{2\pi i/3}$ as before.
Also like before, each choice of $a$ gives a pair of lines $Z(x \pm y - az) \subseteq \P^2$ from the linear factors of $G$.

Lastly, when $c = 0$ we obtain the plane containing $\ell$ and $\ell_2$, which we have already counted.

Hence, for $g = ax + c(z + t)$ we have four new possible values for $[a : c] \in \P^1$:
\[
    [0 : 1], \quad [-2 : 1], \quad [-2\zeta : 1], \quad [-2\zeta^2 : 1],
\]
giving us eight new lines in $S$ intersecting $\ell_1$:
\begin{align*}
    \ell_{11} &= Z(z + t, x + y), & \ell_{12} &= Z(z + t, x - y), \\
    \ell_{13} &= Z(-2x + z + t, x + y + 2z), & \ell_{14} &= Z(-2x + z + t, x - y + 2z), \\
    \ell_{15} &= Z(-2\zeta x + z + t, x + y + 2\zeta y), & \ell_{16} &= Z(-2\zeta x + z + t, x - y + 2\zeta y), \\
    \ell_{17} &= Z(-2\zeta^2 x + z + t, x + y + 2\zeta^2 y), & \ell_{18} &= Z(-2\zeta^2 x + z + t, x - y + 2\zeta^2 y).
\end{align*}


\newpage
\begin{pbox}[4]
    Show that $S$ contains $8$ more lines $\ell_k$, $k = 19, \dots, 26$ none of which coincides with a previously constructed line, and each of which meets $\ell_{11}$.
    (There are also two previously constructed lines which meet $\ell_{11}$.)
\end{pbox}

We will use $\ell_{11} = Z(z + t, x + y)$ from Problem 3.
For $g = ax + by + cx + dt$ we find $a = b $ and $c = d$ so we can write $g = a(x + y) + c(z + t)$.

If $a \ne 0$ we can assume $a = 1$ giving us the following isomorphism:
\begin{align*}
    \P^2 &\longrightarrow Z(g) \subseteq \P^3 \\
    [x : z : t] &\longmapsto [x : -c(z + t) - x : z : t]
\end{align*}
In this copy of $\P^2$, the variety corresponding to $Z(f, g) \subseteq \P^3$ is defined by the homogeneous cubic polynomial
\[
    F(x, z, t)
        = f(x, -c(z + t) - x, z, t)
        = (z + t) \big(-2cx^2 - c^2x(z + t) + z^2 - zt\big).
\]
The factor of $z + t$ corresponds to the line $Z(z + t) \subseteq \P^2$ which, in turn, corresponds to the line $\ell_{11} \subseteq \P^3$.

Let $G = -2cx^2 - c^2x(z + t) + z^2 - zt$ be the homogeneous degree $2$ factor of $F$, then $Z(G)$ is singular where the following matrix is rank zero:
\[
    \mat{\pdv{G}{x} & \pdv{G}{z} & \pdv{G}{t}}
        = \mat{-4cx - c^2(z + t) & 2z - c^2x - t & -c^2x - t}.
\]
The first entry tells us that either $c = 0$ or $4x + c(z + t) = 0$.
If $c = 0$ then $G = z^2 - zt$, giving us the lines $Z(z), Z(z - t) \subseteq \P^2$.
However, $Z(z)$ corresponds to $\ell_3$ which we have already counted.

Otherwise, the second and third entries imply that $z = 0$ so the first entry tells us $4x + ct = 0$.
The third entry also tells us $t = -c^2x$, so in fact we have $(4 - c^3)x = 0$.
And since $z = 0$ and $c \ne 0$ then $x$ and $t = -c^2x$ are both nonzero, so we deduce that $c^3 = 4$.
So we have three possibilities for $c$: $\sqrt[3]{4}$, $\sqrt[3]{4}\zeta$, and $\sqrt[3]{4}\zeta^2$.
Each choice of $c$ gives a pair of lines corresponding to the linear factors of $G$ (I could not find these factors)

Lastly, when $a = 0$ we may assume $c = 1$ giving us the following isomorphism:
\begin{align*}
    \P^2 &\longrightarrow Z(g) \subseteq \P^3 \\
    [x : y : z] &\longmapsto [x : y : z : -z]
\end{align*}
In this copy of $\P^2$, the variety corresponding to $Z(f, g) \subseteq \P^3$ is defined by the homogeneous cubic polynomial
\[
    F(x, y, z)
        = f(x, y, z, -z)
        = x(x + y)(x - y).
\]
The linear factors correspond to $\ell_1$, $\ell_{11}$, and $\ell_{12}$, respectively.

Hence, for $g = a(x + y) + c(z + t)$ we have four new possible values for $[a : c] \in \P^1$:
\[
    [1 : 0], \quad [1 : \sqrt[3]{4}], \quad [1 : \sqrt[3]{4}\zeta], \quad [1 : \sqrt[3]{4}\zeta^2],
\]
giving us eight new lines in $S$ intersecting $\ell_{11}$:
\begin{align*}
    \ell_{19} &= Z(???), & \ell_{20} &= Z(x + y, z - t), \\
    \ell_{21} &= Z(x + y + \sqrt[3]{4}(z + t), ???), & \ell_{22} &= Z(x + y + \sqrt[3]{4}(z + t), ???), \\
    \ell_{23} &= Z(x + y + \sqrt[3]{4}\zeta(z + t), ???), & \ell_{24} &= Z(x + y + \sqrt[3]{4}\zeta(z + t), ???), \\
    \ell_{25} &= Z(x + y + \sqrt[3]{4}\zeta^2(z + t), ???), & \ell_{26} &= Z(x + y + \sqrt[3]{4}\zeta^2(z + t), ???).
\end{align*}


\textit{Note}: As mentioned above, I seem to have found $\ell_3$ again instead of a new line.
Just from guessing, I believe the missing line is
\[
    \ell_{19} = Z(x - y, z - t),
\]
but this line does not intersect $\ell_{11}$ so I am not exactly sure what happened.
It's possible I did not choose the correct lines to go off of.

Additionally, I did not find the second equations for $\ell_{21}, \dots, \ell_{26}$ since $G$ was too hard to factor in this case.

\begin{pbox}
    Conclude that $S$ contains at least $27$ different lines.
\end{pbox}

We conclude that $S$ contains at least $27$ different lines, namely $\ell, \ell_1, \dots, \ell_{26}$.

\end{document}