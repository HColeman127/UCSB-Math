\documentclass[12pt]{article}

% Packages
\usepackage[margin=1in]{geometry}
\usepackage{fancyhdr, parskip}
\usepackage{amsmath, amsthm, amssymb}
\usepackage[shortlabels]{enumitem}

% Page Style
\fancypagestyle{plain}{
    \fancyhf{}
    \renewcommand{\headrulewidth}{0pt}
    \renewcommand{\footrulewidth}{0pt}
    \fancyfoot[R]{\thepage}
}
\pagestyle{plain}

% Problem Box
\setlength{\fboxsep}{4pt}
\newlength{\myparskip}
\setlength{\myparskip}{\parskip}
\newsavebox{\savefullbox}
\newenvironment{fullbox}{\begin{lrbox}{\savefullbox}\begin{minipage}{\dimexpr\textwidth-2\fboxsep\relax}\setlength{\parskip}{\myparskip}}{\end{minipage}\end{lrbox}\framebox[\textwidth]{\usebox{\savefullbox}}}
\newenvironment{pbox}[1][]{\begin{fullbox}\ifx#1\empty\else\paragraph{#1}\fi}{\end{fullbox}}

% Default Commands
\newcommand{\isp}[1]{\quad\text{#1}\quad}
\newcommand{\N}{\mathbb{N}} 
\newcommand{\Z}{\mathbb{Z}}
\newcommand{\Q}{\mathbb{Q}}
\newcommand{\R}{\mathbb{R}}
\newcommand{\C}{\mathbb{C}}
\newcommand{\eps}{\varepsilon}
\renewcommand{\phi}{\varphi}
\renewcommand{\emptyset}{\varnothing}
\newcommand{\<}{\langle}
\renewcommand{\>}{\rangle}
\newcommand{\isom}{\cong}
\newcommand{\eqc}{\overline}
\newcommand{\clo}{\overline}

% Extra Commands
\newcommand{\A}{\mathbb{A}}
\renewcommand{\P}{\mathbb{P}}
\newcommand{\teq}{\trianglelefteq}
\DeclareMathOperator{\rank}{rank}
\newcommand{\mat}[1]{\begin{bmatrix}#1\end{bmatrix}}

% Document Info
\fancypagestyle{title}{
    \renewcommand{\headrulewidth}{0.4pt}
    \setlength{\headheight}{15pt}
    \fancyhead[R]{Harry Coleman\makebox[0pt][r]{\raisebox{-0.25in}[0pt][0pt]{(worked with Joseph Sullivan)}}}
    \fancyhead[L]{MATH 237A Homework 2}
    \fancyhead[C]{October 10, 2021}
}

% Begin Document
\begin{document}
\thispagestyle{title}

\begin{pbox}[1 Exercise 1.14]
    Prove that the map $k[x, y] \to k[t]$ sending $x$ to $t^2$ and $y$ to $t^3$ induces an isomorphism
    \[
        k[x, y]/(y^2 - x^3) \isom k[t^2, t^3] \subset k[t].
    \]
\end{pbox}

\begin{proof}
    The map is surjective, so the quotient of $k[x, y]$ by the kernel is isomorphic to $k[t^2, t^3]$; it remains the prove that the ideal $\<y^2 - x^3\>$ is the map's kernel. This ideal is contained in the kernel, as $y^2 - x^3 \mapsto t^6 - t^6 = 0$. For $f \in k[x, y]$ in the kernel, $f(x, y) \mapsto f(t^2, t^3) = 0$. Since $y^2 - x^3$ is monic in $y$, then we perform long division in the ring $(k[x])[y]$ to obtain
    \[
        f(x, y) = q(x, y)(y^2 - x^3) + r(x, y),
    \]
    where the $y$-degree of $r$ is less than $2$, the $y$-degree of $y^2 - x^3$. We can rewrite $r$ as 
    \[
        r(x, y) = r_0(x) + r_1(x)y,
    \]
    then
    \[
        0 = f(t^2, t^3) = q(t^2, t^3)(t^6 - t^6) + r(t^2, t^3) = r_0(t^2) + r_1(t^2)t^3.
    \]
    We have the polynomials $r_0(t^2)$ and $r_1(t^2)t^3$ in $k[t]$, where the former has only even degrees of $t$ and the latter has only odd. This implies that $r_0(t^2) = r_1(t^2) = 0$, so we must also have $r_0(x) = r_1(x) = 0$, since $k[x] \isom k[t^2]$. Hence, $r = 0$, and we conclude that $f \in \<y^2 - x^3\>$.

\end{proof}

\newpage
\begin{pbox}[2 Exercise 1.15]
    A conic in the affine real plane $\R^2$ belongs to one of the following eight types:
    \begin{enumerate}[a.]
        \item The empty set
        \item A single point
        \item A line
        \item The union of two coincident lines
        \item The union of two parallel lines
        \item A parabola
        \item A hyperbola
        \item An ellipse
    \end{enumerate}
\end{pbox}

\begin{pbox}[(a)]
    Show that in the complex affine plane $\C^2$ there are only five  types of loci defined by equations of degree $2$: Types a and b  disappear, and types g and h  coincide.
\end{pbox}

A quadratic in $\C[x, y]$ is of the form
\[
    ax^2 + bxy + cy^2 + dx + ey + f
\]
If $a \ne 0$, then every value of $y \in \C$  gives a quadratic in $x$:
\[
    ax^2 + (by + d)x + (cy^2 + ey + f).
\]
Since $\C$ is algebraically closed, this has two solutions in $x$, counting multiplicity. In other words, the quadratic has infinitely many solutions in $\C^2$. The same is true when $c \ne 0$. If $a = c = 0$, so $b \ne 0$, then the polynomial becomes
\[
    (by + d)x + ey + f.
\]
Any value of $y \in \C$, other than $-d/b$, again has infinitely many solutions in $x$. Hence, cases a and b are not possible.

A hyperbola ($xy - 1 = 0$) can be transformed into the form of an ellipse by the invertible linear transformation of coordinates $x \mapsto x + iy$, $y \mapsto x - iy$ (giving $x^2 + y^2 - 1 = 0$).

\newpage
\begin{pbox}[(b)]
    Show that in the complex projective plane $\P^2(\C)$ there are only three types of loci represented by quadratic equations; they are represented by types c, d, and h  on the above list. More generally, there are exactly $n$ types of nonzero quadratic forms in $n$ variables, classified by rank (where the rank of a quadratic form $\sum_{i < j} a_{ij}x_ix_j$ is defined to be the rank of the symmetric matrix $(a_{ij})$). 
\end{pbox}

\begin{proof}
    Each (homogeneous) quadratic $f \in \C[x_0, \dots, x_n]$ has a representation $f = x^TAx$, where $A \in \C^{(n+1) \times (n+1)}$ and $x = [x_0 \, \cdots \, x_n]^T$. Without loss of generality, we may assume $A$ is chosen to be symmetric. We define the rank of the quadratic form $f$ to be the rank of the symmetric matrix $A$.

    For quadratic forms $f = x^TAx$ and $g = x^TBx$, we define the relation $Z(f) \sim Z(g)$ if there is an $(n+1)\times(n+1)$ invertible matrix $P$ with entries in $\C$, such that the map $Z(f) \to Z(g)$, where $x \mapsto Px$, is an isomorphism or, equivalently, if $A = P^TBP$ and write $A \sim B$. One can check that this is an equivalence relation on the loci of quadratic forms. We claim that $A \sim B$ (i.e., $Z(f) \sim Z(g)$) if and only if $\rank A = \rank B$.
    
    If $A \sim B$, then $A = P^TBP$ for some invertible matrix $P$. Since rank is preserved under multiplication by an invertible matrix, $\rank A = \rank P^TBP = \rank B$.

    Suppose $\rank A = \rank B$. By the spectral theorem, $A$ and $B$ are unitarily diagonalizable into $A = P^TD_AP$ and $B = Q^TD_BQ$, where $D_A$, $D_B$ are diagonal and $P$, $Q$ are unitary. In particular, $A \sim D_A$ and $B \sim D_B$, and it remains to prove $D_A \sim D_B$. We have
    \[
        \rank D_A = \rank A = \rank B = \rank D_B,
    \]
    then write
    \[
        D_A = \mat{
            \alpha_1 \\
                & \ddots \\
                &   & \alpha_r \\
                &   &   & 0
        }
        \isp{and}
        D_B = \mat{
            \beta_1 \\
                & \ddots \\
                &   & \beta_r \\
                &   &   & 0
        }.
    \]
    Since $\C$ is algebraically closed, we can construct
    \[
        C_A = \mat{
            1/\sqrt{\alpha_1} \\
                & \ddots \\
                &   & 1/\sqrt{\alpha_r} \\
                &   &   & I_{n-r}
        }
        \isp{and}
        C_B = \mat{
            1/\sqrt{\beta_1} \\
                & \ddots \\
                &   & 1/\sqrt{\beta_r} \\
                &   &   & I_{n-r}
        },
    \]
    where principle square roots are taken. Then
    \[
        C_A^TD_AC_A = \mat{
            I_r \\
                & 0
        }
        = C_B^TD_BC_B,
    \]
    so $D_A \sim D_B$, implying $A \sim B$.

\end{proof}

\begin{pbox}[(c)]
    Show that the different types in part (a) correspond to the relative placement of the conic and the line at infinity, in the sense that a parabola is a rank-$3$ conic tangent to the line at infinity, while an ellipse/hyperbola is a rank-$3$ conic meeting the line at infinity at two distinct points.
\end{pbox}

Consider $\P_\C^2$ in the coordinates $[x, y, z]$. The line at infinity (for $\C^2$ in $(x, y)$) is where $z = 0$.

The line $x^2 = 0$ intersects at the point $[0, 1, 0]$.

The union of two coincident lines $xy = 0$ intersects the points $[0, 1, 0]$ and $[1, 0, 0]$.

The union of two parallel lines $x(x - 1) = 0$ homogenizes to $x(x - z) = 0$, which means $x = 0$ or $x = z$, so intersects at the point $[0, 1, 0]$.

A parabola $x^2 - y = 0$ homogenizes to $x^2 - yz = 0$. When $z = 0$, we must have $x^2 = 0$, so intersects at the point $[0, 1, 0]$.

A hyperbola $xy - 1 = 0$ homogenizes to $xy - z^2 = 0$. When $z = 0$, we must have $xy = 0$, so intersects at the points $[0, 1, 0]$ and $[1, 0, 0]$.


\newpage
\begin{pbox}[3 Exercise 1.17]
    Let $I \subset k[x_1, x_2, x_3]$ be the ideal $(x_1^2 + x_2, x_1^2 + x_3)$, and let $X \subset \A^3$ be the affine algebraic set $Z(I)$. Let $\clo{X} \subset \P^3$ be the projective closure of $X$. Show that the homogeneous ideal $I(\clo{X})$ is not generated by the homogenizations of $x_1^2 + x_2$ and $x_1^2 + x_3$.
\end{pbox}

\begin{proof}
    The homogeneous ideal $I(\clo{X})$ is the ideal of $k[x_0, \dots, x_3]$ generated by the homogenizations of every polynomial in $I$. In particular, we know that $x_1^2 + x_2, x_1^2 + x_3 \in I$, so the ideal generated by their homogenizations is contained in $I(\clo{X})$, i.e.,
    \[
        J = \<x_1^2 + x_0x_2, x_1^2 + x_0x_3\> \subseteq I(\clo{X}).
    \]
    Since $J$ is generated by homogeneous polynomials of degree $2$, all homogeneous polynomials in $J$ have degree at least $2$. However, $I$ contains the element
    \[
        (x_1^2 + x_2) - (x_1^2 + x_3) = x_2 - x_3,
    \]
    which is homogeneous of degree $1$, so $x_2 - x_3 \in I(\clo{X}) \setminus J$.
    
\end{proof}

\newpage
\begin{pbox}[4 Exercise 1.18]
    Let $k$ be a field. Compute the Hilbert function and polynomial for the ring
    \[
        k[x, y, z, w]/(x, y) \cap (z, w)
    \]
    corresponding to the disjoint union of two lines in projective $3$-space. Compare these to the Hilbert function and polynomial of the ring corresponding to one projective line, $k[x, y]$.
\end{pbox}

We can rewrite the ideal as
\[
    \<x, y\> \cap \<z, w\> = \<xz, xw, yz, yw\>,
\]
so
\[
    M = k[x, y, z, w]/\<x, y\> \cap \<z, w\> = k[x, y, z, w]/\<xz, xw, yz, yw\>.
\]
Then the monomials of degree $s$ in $M$ are $x^ay^{s-a}$ and $z^aw^{s-a}$, for $a = 0, 1, \dots, s$. For $s \geq 1$, are exactly $2s + 2$ of these monomials, so the Hilbert function is given by
\[
    H_M(s) = \begin{cases}
        2s + 2 &\text{if } s > 0, \\
        1 &\text{if } s = 0, \\
        0 &\text{if } s < 0.
    \end{cases}
\]
This means that the Hilbert polynomial is $P_M(s) = 2s + 2$, agreeing with $H_M(s)$ when $s \geq 1$.

The monomials of degree $s$ in $N = k[x, y]$ are $x^ay^{s-a}$ for $a = 0, 1, \dots, s$. For $s \geq 1$, there are exactly $s + 1$ of these monomials, so the Hilbert function is given by
\[
    H_N(s) = \begin{cases}
        s + 1 &\text{if } s > 0, \\
        1 &\text{if } s = 0, \\
        0 &\text{if } s < 0.
    \end{cases}
\]
The Hilbert polynomial is $H_N(S) = s + 1$, agreeing with $H_N(s)$ when $s \geq 1$.

So $H_M(s) = 2H_N(s)$ for $s \geq 1$, and $H_M(s) = H_N(s)$ otherwise. Similarly, $P_M(s) = 2P_N(s)$.


\end{document}