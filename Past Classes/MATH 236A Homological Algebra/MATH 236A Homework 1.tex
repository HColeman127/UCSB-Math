\documentclass[12pt]{article}

% Packages
\usepackage[margin=1in]{geometry}
\usepackage{fancyhdr, parskip}
\usepackage{amsmath, amsthm, amssymb}
\usepackage{tikz, tikz-cd}
\usepackage[shortlabels]{enumitem}

% Page Style
\makeatletter
\fancypagestyle{title}{
    \renewcommand{\headrulewidth}{0.4pt}
    \setlength{\headheight}{15pt}
    \fancyhead[R]{\@author}
    \fancyhead[L]{\@title}
    \fancyhead[C]{\@date}
}
\makeatother
\renewcommand{\maketitle}{\thispagestyle{title}}
\fancypagestyle{plain}{
    \fancyhf{}
    \renewcommand{\headrulewidth}{0pt}
    \renewcommand{\footrulewidth}{0pt}
    \fancyfoot[R]{\thepage}
}
\pagestyle{plain}

% Problem Box
\setlength{\fboxsep}{4pt}
\newlength{\myparskip}
\setlength{\myparskip}{\parskip}
\newsavebox{\savefullbox}
\newenvironment{fullbox}{\begin{lrbox}{\savefullbox}\begin{minipage}{\dimexpr\textwidth-2\fboxsep\relax}\setlength{\parskip}{\myparskip}}{\end{minipage}\end{lrbox}\framebox[\textwidth]{\usebox{\savefullbox}}}
\newenvironment{pbox}[1][]{\begin{fullbox}\def\temp{#1}\ifx\temp\empty\else\paragraph{#1}\phantom{}\fi}{\end{fullbox}}

% Theorem Environments
\theoremstyle{definition}
\newtheorem{lemma}{Lemma}

% Tikz Environments
\newenvironment{drawing}{\begin{center}\begin{tikzpicture}}{\end{tikzpicture}\end{center}}
% \tikzcdset{row sep/normal=0pt}
\newenvironment{cd}{\begin{center}\begin{tikzcd}}{\end{tikzcd}\end{center}}

% Default Commands
\newcommand{\isp}[1]{\quad\text{#1}\quad}
\newcommand{\N}{\mathbb{N}} 
\newcommand{\Z}{\mathbb{Z}}
\newcommand{\Q}{\mathbb{Q}}
\newcommand{\R}{\mathbb{R}}
\newcommand{\C}{\mathbb{C}}
\newcommand{\A}{\mathbb{A}}
\renewcommand{\P}{\mathbb{P}}
\newcommand{\eps}{\varepsilon}
\renewcommand{\phi}{\varphi}
\renewcommand{\emptyset}{\varnothing}
\newcommand{\<}{\langle}
\renewcommand{\>}{\rangle}
\newcommand{\iso}{\cong}
\newcommand{\eqc}{\overline}
\newcommand{\clo}{\overline}
\renewcommand{\tilde}{\widetilde}
\renewcommand{\hat}{\widehat}
\newcommand{\seq}{\subseteq}
\newcommand{\teq}{\trianglelefteq}
\DeclareMathOperator{\id}{id}
\DeclareMathOperator{\im}{im}
\newcommand{\inc}{\hookrightarrow}
\newcommand{\dd}{\mathrm{d}}
\newcommand{\mat}[1]{\begin{bmatrix}#1\end{bmatrix}}

% Extra Commands
\newcommand{\tensor}{\otimes}

\renewcommand{\_}[1]{{_{#1}}}

\DeclareMathOperator{\Hom}{Hom}
\DeclareMathOperator{\GL}{GL}

\newcommand{\CC}{\mathcal{C}}
\newcommand{\DD}{\mathcal{D}}

\newcommand{\mathcat}{\mathsf}
\newcommand{\newcat}[2]{\newcommand{#1}{\mathcat{#2}}}

\newcat{\Set}{Set}

\newcat{\Mod}{Mod}
\newcat{\lMod}{\text{-}Mod}
\newcat{\rMod}{Mod\text{-}}

\newcat{\lmod}{\text{-}mod}
\newcat{\rmod}{mod\text{-}}

\newcat{\Rep}{Rep}

% Document
\begin{document}
\title{MATH 236A Homework 1}
\author{Harry Coleman}
\date{February 17, 2022}
\maketitle

\begin{pbox}[1]
    Let $R$, $S$ be rings and $\_SB_R$ an $S$-$R$-bimodule. Prove that the functors $G = B \tensor_R - : R\lMod \to S\lMod$ and $F = \Hom_S(B,-) : S\lMod \to R\lMod$ form an adjoint pair $(G, F)$.
\end{pbox}

We will construct a natural isomorphism
\begin{cd}[column sep=huge]
    R\lMod \times S\lMod
        \rar[bend left=24, "{\Hom_S(B \tensor_R -, -)}" name=U]
        \rar[bend right=20, "{\Hom_R(-, \Hom_S(B, -))}"' name=L]
        \ar[from=U, to=L, shorten=1ex, Rightarrow, "\iso"]
    & \Set
\end{cd}

Fix a pair of left modules $M \in R\lMod$ and $N \in S\lMod$.

We define the component
\[
    \alpha_{M, N} : \Hom_S(B \tensor_R M, N) \to \Hom_R(M, \Hom_S(B, N)).
\]
Given a homomorphism of left $S$-modules $f : B \tensor_R M \to N$ and an element $m \in M$, we define a map $\alpha f(m) : B \to N$ by $b \mapsto f(b \tensor m)$.
We check that $\alpha f (m)$ is a homomorphism of left $S$-modules:
\begin{align*}
    \alpha f(m)(sb_1 + b_2)
        &= f((sb_1 + b_2) \tensor m) \\
        &= f(s(b_1 \tensor m) + b_2 \tensor m) \\
        &= s \alpha f(m)(b_1) + \alpha f(m)(b_2).
\end{align*}
Hence, $\alpha f : M \to \Hom_S(B, N)$ is a well-defined map.
Next, we check that $\alpha f$ is a homomorphism of left $R$-modules:
\begin{align*}
    \alpha f(rm_1 + m_2)(b)
        &= f((rm_1 + m_2) \tensor b) \\
        &= f(r(m_1 \tensor b) + m_2 \tensor b) \\
        &= r \alpha f(m_1)(b) + \alpha f(m_2)(b) \\
        &= (r \alpha f(m_1) + \alpha f(m_2))(b).
\end{align*}
Hence, $\alpha_{M, N} : \Hom_S(B \tensor_R M, N) \to \Hom_R(M, \Hom_S(B, N))$ is a well-defined map.

We now define the component
\[
    \beta_{M, N} : \Hom_R(M, \Hom_S(B, N)) \to \Hom_S(B \tensor_R M, N).
\]
Given a homomorphism of left $R$-modules $g : M \to \Hom_S(B, N)$, there is a map $\tilde{g} : B \times M \to N$ given by $\tilde{g}(b, m) = g(m)(b)$.
We check that $\tilde{g}$ is an $R$-balanced map:
\[
    \tilde{g}(b, m_1 + m_2)
        = g(m_1 + m_2)(b)
        = g(m_1)(b) + g(m_2)(b)
        = \tilde{g}(b, m_1) + \tilde{g}(b, m_2),
\]
\[
    \tilde{g}(b_1 + b_2, m)
        = g(m)(b_1 + b_2)
        = g(m)(b_1) + g(m)(b_2)
        = \tilde{g}(b_1, m) + \tilde{g}(b_2, m),
\]
\[
    \tilde{g}(br, m)
        = g(m)(br)
        = rg(m)(b)
        = g(rm)(b)
        = \tilde{g}(b, rm).
\]
The second equality in the third condition follows from the fact that $\Hom_S(B, N)$ is a left $R$-module by $(rh)(b) = h(br)$ for $h \in \Hom_S(B, N)$.
By the universal property of the tensor product, there is a unique homomorphism of left $S$-modules $\beta g$ which makes the following diagram commute:
\begin{cd}
    B \times M 
        \rar["\tilde{g}"]
        \dar["\tau"'] 
    & N \\
    B \tensor_R M 
        \urar[dashed, "\beta g"']
\end{cd}
Here, $\tau$ is the canonical map from the product to the tensor product.
By this construction, we have the characterization $\beta g(b \tensor m) = g(m)(b)$.
Hence, $\beta_{M, N} : \Hom_R(M, \Hom_S(B, N)) \to \Hom_S(B \tensor_R M, N)$ is a well-defined map.

Lastly, we check that $\alpha_{M, N}$ and $\beta_{M, N}$ are inverses to each other:
\[
    (\beta \alpha f)(b \tensor m)
        = \alpha f (m)(b)
        = f(b \tensor m),  
\]
\[
    (\alpha \beta g)(m)(b)
        = \beta g(b \tensor m)
        = g(m)(b).
\]

Hence, $\alpha$ and $\beta$ describe the desired natural isomorphism.

\newpage
\begin{pbox}[2]
    Let $K$ be a field. Show that the contravariant functor $D = \Hom_K(-, K): K\lMod \to K\lMod$ is not a natural equivalence. In particular, show that the vector spaces $V$ and $D(V)$ fail to be isomorphic if $V \iso K^{(\N)}$.
\end{pbox}

We consider the vector space $V = K^{(\N)} = \bigoplus_{i \in \N} K$.
It has a countable basis given by $\{e_i\}_{i \in \N}$.
In particular, the dimension of $V$ as a $K$-vector space is countably infinite.

Consider its dual space $D(V) = \Hom_K(V, K)$.
A linear functional on $V$ is determined by where in $K$ it sends the basis vectors of $V$.
Moreover, any choice of image in $K$ for each basis vector of $V$ determines a linear functional on $V$.
In other words, a choice of linear functional on $V$ is equivalent to a choice of element in $K$ for each natural number, so $D(V) \iso K^\N$ as $K$-vector spaces.

We claim that the dimension of $K^\N$ as a $K$-vector space is uncountably infinite.
In particular, this would mean $\dim_K V \ne \dim_K D(V)$, implying that $V$ and $D(V)$ could not possibly be isomorphic.

Let $F = K_0$ be the prime subfield of $K$.
Then $F = \Q$ or $F = \Z/p\Z$ for some prime $p$---in particular, $F$ is countable.

\begin{lemma}\label{lem:dimF^N}
    $\dim_F F^\N$ is uncountably infinite.
\end{lemma}

\begin{proof}
    The dimension of $F^{(\N)}$ as an $F$-vector space is countably infinite since it has a countably infinite basis given by $\{e_i\}_{i \in \N}$.

    The dimension of $F^\N$ is at least countably infinite, since $\{e_i\}_{i \in \N}$ is an $F$-linearly independent set in $F^\N$ (though not a basis).
    It other words, $\dim_F F^\N \geq \dim_F F^{(\N)}$; it remains to prove that this inequality is strict.

    We can write $F^{(\N)}$ as the direct limit $F^{(\N)} = \varinjlim F^n$ with respect to the natural inclusions $F^n \inc F^{n+1}$ for each $n \in \N$.
    This gives $F^{(N)} = F^1 \cup F^2 \cup F^3 \cup \cdots$, which is a countable union of countable sets, therefore the cardinality of $F^{(\N)}$ is countable.
    On the other hand, the cardinality of $F^\N$ is given by $|F^\N| = |F|^{|\N|}$, which is uncountable.

    Since the cardinalities of $F^{(\N)}$ and $F^\N$ as sets are different, there can be no bijection between them.
    In particular, there can be no isomorphism between the respective $F$-vector spaces.
    It follows that their dimensions must be different, so indeed $\dim_F F^\N > \dim_F F^{(\N)}$.
\end{proof}

By Lemma \ref{lem:dimF^N}, let $\mathcal{S} \seq F^\N$ be an uncountable $F$-linearly independent set.
Under the natural set inclusion $F^\N \inc K^\N$, we can interpret $\mathcal{S}$ as a set of vectors in $K^\N$.
We claim that $\mathcal{S}$ is also $K$-linearly independent.

To prove that $\mathcal{S}$ is $K$-linearly independent, we will prove the equivalent condition that every finite subset of $\mathcal{S}$ is $K$-linearly independent.
Let $S = \{x_1, \dots, x_n\} \seq \mathcal{S}$ be an arbitrary finite subset.
Of course, $S$ is $F$-linearly independent.

\begin{lemma}\label{lem:truncate}
    There exists $m \in \N$ such that $S_m = \{x_1^{(m)}, \dots, x_n^{(m)}\}$ is $F$-linearly independent, where $x_i^{(m)} \in F^m$ is the truncation of $x_i$ to the first $m$ components.
\end{lemma}

\begin{proof}
    For $m \in \N$ define the following subspace of $F^n$: 
    \[
        D_m = \{a \in F^n \mid \textstyle\sum_{i=1}^{n} a_i x_i^{(m)} = 0\}.
    \]
    Notice that $\dim_F D_m < \infty$ and $D_m \seq D_{m+1}$ for all $m \in \N$.
    Moreover, if it happens for some $m \in \N$ that $\dim_F D_m = 0$ then $S_m$ would be $F$-linearly independent.
    We will show that such an $m$ can be found.

    Suppose for a given $m \in \N$ that $\dim_F D_m > 0$.
    Choose any nonzero element $a \in D_m$, then $Ka$ is a $1$-dimensional subspace of $D_m$.
    Since $S$ is $F$-linearly independent and $a$ is nonzero, we must have $x = \sum_{i=1}^{n} a_i x_i \ne 0$.
    By assumption, the first $m$ components of $x$ are zero, but in order for $x$ to be nonzero there must be some $m' \geq m$ such that the $m'$th component is nonzero.
    Then $a \notin D_{m'}$ and we compute 
    \begin{align*}
        \dim_F D_{m'}
            &= \dim_F(D_{m'} + Ka) - \dim_F Ka + \dim_F(D_{m'} \cap Ka) \\
            &= \dim_F(D_{m'} + Ka) - 1 + 0 \\
            &\leq \dim_F D_m - 1 \\
            &< \dim_F D_m.
    \end{align*}
    Since $D_m$ is always finite dimensional, we can repeat this process which strictly reduces the dimension a finite number of times to find $m \in \N$ large enough that $\dim_F D_m = 0$.
\end{proof}

Choose $m \in \N$ as in Lemma \ref{lem:truncate}.
We will show that $S_m$ is $K$-linearly independent as a set of vectors in $K^m$.
Extend $S_m$ to an $F$-basis $\beta = S_m \cup \{y_1, \dots, y_{m - n}\}$ of $F^m$.
The fact that $\beta$ is a basis is equivalent to the invertibility of the following matrix:
\[
    A = \mat{
        | & & | & | & & | \\
        x_1^{(m)} & \dots & x_n^{(m)} & y_1 & \dots & y_{m - n} \\
        | & & | & | & & |
    } \in M_m(F).
\]
So there exists an inverse matrix $A^{-1} \in M_m(F)$.
Under the inclusion $M_m(F) \inc M_m(K)$, we may consider both $A$ and $A^{-1}$ to be matrices in $M_m(K)$.
Moreover, it is still true in $M_m(K)$ that $AA^{-1} = A^{-1}A = I_m$, i.e., that $A$ is invertible.
Equivalently, the columns of $A$---the elements of $\beta$ under the inclusion $F^m \inc K^m$---form a $K$-basis of $K^m$.
In particular, $S_m$ is $K$-linearly independent.

From this, we deduce that $S$ is $K$-linearly independent, since for any nonzero $a \in K^n$ we must have $\sum_{i=1}^{n} a_i x_i^{(m)} \ne 0$, which implies $\sum_{i=1}^{n} a_i x_i \ne 0$.
Therefore, every finite subset of $\mathcal{S}$ is $K$-linearly independent, which means $\mathcal{S}$ is $K$-linearly independent.
Thus, we have found an uncountable set of $K$-linearly independent vectors in $K^\N$, so indeed $\dim_K K^\N$ is uncountably infinite.
In particular, $\dim_K K^\N > \dim_K K^{(\N)}$ so $K^\N \ncong K^{(\N)}$.



\rule{\textwidth}{0.4pt}


\begin{lemma}\label{lem:coprod_iso}
    Let $\{X_i\}_{i \in I}$ be a collection of objects in a category.
    Suppose there exists a coproduct $X = \bigsqcup_{i \in I} X_i$ with inclusion morphisms $\iota_i : X_i \to X$.
    If there exists an isomorphism $\alpha : X \to \tilde{X}$, then $\tilde{X}$ with inclusion morphisms $\alpha \circ \iota_i : X_i \to \tilde{X}$ is also a categorical coproduct of the $X_i$'s.
\end{lemma}

\begin{proof}
    Suppose $f_i : X_i \to Y$ is a collection of morphisms.
    Then by the universal property of the coproduct $X$, there is a unique morphism $f : X \to Y$ such that $f \circ \iota_i = f_i$ for all $i \in I$.
    Then the composition $\tilde{f} = f \circ \alpha^{-1} : \tilde{X} \to Y$ is a morphism satisfying $\tilde{f} \circ (\alpha \circ \iota_i) = f_i$.
    In other words, the following diagram commutes for all $i \in I$:
    \begin{cd}[column sep=huge]
        \tilde{X} \ar[ddr, dashed, "\tilde{f}"]\\
        X \uar["\alpha", "\iso"'] \drar[dashed, "f"'] \\
        X_i \uar["\iota_i"] \rar["f_i"'] & Y
    \end{cd}
    To check the uniqueness of $\tilde{f}$, suppose $g : \tilde{X} \to Y$ is another morphism satisfying the same diagram.
    But then, considering $g \circ \alpha$, we have the following commutative diagram:
    \begin{cd}[column sep=huge]
        \tilde{X} \ar[ddr, dashed, "g"]\\
        X \uar["\alpha", "\iso"'] \drar[dashed, "g \circ \alpha"'] \\
        X_i \uar["\iota_i"] \rar["f_i"'] & Y
    \end{cd}
    By the universal property of the coproduct $X$, we must have $g \circ \alpha = f$, so $g = f \circ \alpha^{-1} = \tilde{f}$.
\end{proof}

\begin{lemma}\label{lem:coprod_equiv}
    Suppose $\CC$ and $\DD$ are equivalent categories via functors $F : \CC \to \DD$ and $G : \DD \to \CC$, and natural isomorphisms $\alpha : \id_\CC \Rightarrow GF$ and $\beta : \id_\DD \Rightarrow FG$.
    If $\iota_i : X_i \to X$ describes a categorical coproduct in $\CC$, then $F\iota_i : FX_i \to FX$ describes a categorical coproduct in $\DD$.
\end{lemma}

\begin{proof}
    Suppose we have a family of morphisms $g_i : FX_i \to Y$ in $\DD$.
    Then $G$ gives us a family of morphisms $Gg_i : GFX_i \to GY$ in $\CC$.
    Lemma \ref{lem:coprod_iso} and the naturality of $\alpha$ gives us a unique morphism $\tilde{g}$ which makes the following diagram in $\DD$ commute for all $i \in I$:
    \begin{cd}
        X \rar["\alpha_X", "\iso"']
        & GFX \drar[dashed, "\tilde{g}"]
        \\
        X_i \uar["\iota_i"] \rar["\alpha_{X_i}"', "\iso"]
        & GFX_i \uar["GF\iota_i"] \rar["Gg_i"']
        & GY
    \end{cd}
    Mapping the triangle of this diagram under $F$, and using the naturality of $\beta$, we obtain the following commutative diagram in $\CC$:
    \begin{cd}
        FX \rar["\beta_{FX}", "\iso"']
        & FGFX \drar["F\tilde{g}"]
        \\
        FX_i \uar["F\iota_i"] \rar["\beta_{FX_i}"', "\iso"]
        & FGFX_i \uar["FGF\iota_i"] \rar["FGg_i"']
        & FGY \rar["\beta_Y^{-1}"]
        & Y 
    \end{cd}
    Now, $g = \beta_Y^{-1} \circ F\tilde{g} \circ \beta_{FX} : FX \to Y$ is a morphism satisfying $g \circ F\iota_i = \beta_Y^{-1} \circ FGg_i \circ \beta_{FX_i}$, which equals $g_i$ by the naturality of $\beta$, for all $i \in I$.

    It remains to prove that $g$ is the unique such morphism.
    Suppose $h : FX \to Y$ is a morphism satisfying $h \circ F\iota_i = g_i$ for all $i \in I$.
    Mapping under $G$, we find that $Gh$ makes the following diagram in $\CC$ commute for all $i \in I$:
    \begin{cd}
        X \rar["\alpha_X", "\iso"']
        & GFX \drar["Gh"]
        \\
        X_i \uar["\iota_i"] \rar["\alpha_{X_i}"', "\iso"]
        & GFX_i \uar["GF\iota_i"] \rar["Gg_i"']
        & GY
    \end{cd}
    But then the universal property of $GFX$ as a coproduct of $X_i$'s tells us that $Gh = \tilde{g}$, so
    \[
        FGh
            = F\tilde{g}
            = \beta_Y \circ g \circ \beta_{FX}^{-1}.
    \]
    Combining this with the naturality of $\beta$, we find that the following  diagram in $\DD$ commutes:
    \begin{cd}
        FX \dar["h"'] \rar["\beta_{FX}", "\iso"']
        & FGFX \dar["FGh"']
        & FX \lar["\beta_{FX}"', "\iso"] \dar["g"]
        \\
        Y \rar["\beta_Y"', "\iso"]
        & FGY 
        & Y \lar["\beta_Y", "\iso"']
    \end{cd}
    The perimeter of this diagram gives us $h = g$, and we conclude that $F\iota_i : FX_i \to FX$ indeed describes a categorical coproduct in $\DD$.
\end{proof}



\rule{\textwidth}{0.4pt}

Finally, suppose for contradiction that $D$ is part of a natural equivalence of categories.
Consider the $K$-vector space $K^{(\N)} = \bigoplus_{i \in \N} K$ as above.
Notice that $K^{(\N)}$ is the categorical coproduct of countably infinitely many copies of $K$.
Applying Lemma \ref{lem:coprod_equiv}, we compute
\[
    K^{\N}
        \iso D(K^{(\N)}) 
        \iso \bigoplus_{i \in \N} D(K) 
        \iso \bigoplus_{i \in \N} K
        = K^{(\N)}
\]
But this is a contradiction, as we have already shown these two vector spaces to be non-isomorphic.





\newpage
\begin{pbox}[3]
    Given $M \in R\lMod$, consider the co- and contravariant hom-functors $F_1, F_2 : R\lMod \to \Z\lMod$, given by $F_1 = \Hom_R(M, -)$ and $F_2 = \Hom_R(-, M)$. For either choice of $k \in \{1, 2\}$, prove those of the following statements which are true in general, and provide a counterexample for those that are not. (Here $(N_i)_{i \in I}$ is an arbitrary family of left $R$-modules.)
\end{pbox}





\begin{pbox}[(a) 1]
    $\Hom_R(M, \bigoplus_{i\in I} N_i) \iso \bigoplus_{i\in I} \Hom_R(M, N_i)$
\end{pbox}

Probably false.



\begin{pbox}[(a) 2]
    $\Hom_R(\bigoplus_{i\in I} N_i, M) \iso \bigoplus_{i\in I} \Hom_R(N_i, M)$
\end{pbox}

False.

\begin{lemma}\label{lem:co_hom_coprod}
    $\Hom_R(\bigoplus_{i \in I} N_i, M) \iso \prod_{i \in I} \Hom_R(N_i, M)$.
\end{lemma}

\begin{proof}
    For $i \in I$, let $\iota_i : N_i \inc \bigoplus_{i\in I} N_i$ be the canonical projection.
    Define the map
    \begin{align*}
        \phi : \Hom_R(\textstyle\bigoplus_{i \in I} N_i, M) &\to \textstyle\prod_{i\in I} \Hom_R(N_i, M) \\
            f &\mapsto (f \circ \iota_i)_{i \in I}.
    \end{align*}
    We check that $\phi$ is a $\Z$-module homomorphism:
    \begin{align*}
        \phi(af + g)(n)
            &= ((af + g)\iota_i(n))_{i \in I} \\
            &= (af\iota_i(n) + g\iota_i(n))_{i \in I} \\
            &= a(f\iota_i(n))_{i \in I} + (g\iota_i(n))_{i \in I} \\
            &= a\phi(f)(n) + \phi(g)(n) \\
            &= (a\phi(f) + \phi(g))(n).
    \end{align*}
    The bijectivity of of $\phi$ follows from the universal property of the direct sum (categorical coproduct) of $\Z$-modules, i.e., given a family of $\Z$-module homomorphisms $f_i : N_i \to M$ with $i \in I$, there is a unique $\Z$-module homomorphism $f : \bigoplus_{i \in I} N_i \to M$ such that $f_i = f \circ \iota_i$ for all $i \in I$.
    In other words, $\phi(f) = (f_i)_{i \in I}$.
    The existence in this condition gives us surjectivity and the uniqueness gives us injectivity.
    Hence, $\phi$ is an isomorphism of $\Z$-modules.
\end{proof}

Take $R = \Z$, $M = \Z$, $N_i = \Z$, and $I = \N$
By \ref{lem:co_hom_coprod}, we have
\[\textstyle
    \Hom_\Z(\bigoplus_{i\in\N} \Z, \Z)
        \iso \prod_{i \in I} \Hom_\Z(\Z, \Z)
        \iso \prod_{i \in I} \Z
        = \Z^\N.
\]
However, we also have
\[
    \bigoplus_{i\in\N} \Hom_\Z(\Z, \Z)
        \iso \bigoplus_{i\in\N} \Z
        = \Z^{(\N)}.
\]
The set $\Z^\N$ is uncountable whilst $\Z^{(\N)}$ is countable.
In particular, the cardinalities of the two sets are different, so there can be no bijection between them.
Therefore, there can be no isomorphism between the respective $\Z$-modules.


\begin{pbox}[(b) 1]
    $\Hom_R(M, \prod_{i\in I} N_i) \iso \prod_{i\in I} \Hom_R(M, N_i)$
\end{pbox}

True.

For $i \in I$, let $\pi_i : \prod_{i\in I} N_i \to N_i$ be the canonical projection.
Define the map
\begin{align*}
    \phi : \Hom_R(M, \textstyle\prod_{i\in I} N_i) &\to \textstyle\prod_{i\in I} \Hom_R(M, N_i) \\
        f &\mapsto (\pi_i \circ f)_{i \in I}.
\end{align*}
We check that $\phi$ is a $\Z$-module homomorphism:
\begin{align*}
    \phi(af + g)(m)
        &= (\pi_i(af + g)(m))_{i \in I} \\
        &= (\pi_i(af(m) + g(m)))_{i \in I} \\
        &= (a\pi_if(m) + \pi_ig(m))_{i \in I} \\
        &= a(\pi_if(m))_{i \in I} + (\pi_ig(m))_{i \in I} \\
        &= a\phi(f)(m) + \phi(g)(m) \\
        &= (a\phi(f) + \phi(g))(m).
\end{align*}
The bijectivity of of $\phi$ follows from the universal property of the direct product (categorical product) of $\Z$-modules, i.e., given a family of $\Z$-module homomorphisms $f_i : M \to N_i$ with $i \in I$, there is a unique $\Z$-module homomorphism $f : M \to \prod_{i \in I} N_i$ such that $f_i = \pi_i \circ f$ for all $i \in I$.
In other words, $\phi(f) = (f_i)_{i \in I}$.
The existence in this condition gives us surjectivity and the uniqueness gives us injectivity.
Hence, $\phi$ is an isomorphism of $\Z$-modules.



\begin{pbox}[(b) 2]
    $\Hom_R(\prod_{i\in I} N_i, M) \iso \prod_{i\in I} \Hom_R(N_i, M)$
\end{pbox}

Probably false.






\newpage
\begin{pbox}[4]
    Let $\CC$ be an additive category; in particular, finite direct sums and products of objects in $\CC$ exist. Prove that for any choice of objects $C_1, \dots, C_n$ of $\CC$, the direct sum $\bigsqcup_{1\leq i \leq n} C_i$ is isomorphic to the direct product $\prod_{1\leq i \leq n} C_i$.
\end{pbox}


\begin{proof}
    Let $X = \prod_{i=1}^{n} C_i$ be the product with projection maps $\pi_i : X \to C_i$.
    For all $i$ and $j$, define the morphism $\lambda_{i,j} \in \Hom(C_i, C_j)$ by
    \[
        \lambda_{i,j} = \begin{cases}
            \id_{C_i} &\text{if } i = j \\
            0 &\text{otherwise}.
        \end{cases}
    \]
    For a fixed $i$, the universal property of the product gives us a unique morphism $\lambda_i$ which makes the following diagram commute for all $j$:
    \begin{cd}
        & X \dar["\pi_j"] \\
        C_i \rar["\lambda_{i,j}"'] \urar[dashed, "\lambda_i"]
        & C_j
    \end{cd}
    Consider the morphism $\sigma = \sum_{j=1}^{n} (\lambda_j \circ \pi_j) \in \Hom(X, X)$.
    Composing with $\pi_i$ gives
    \[
        \pi_i \circ \sigma
            = \sum_{j=1}^{n} (\pi_i \circ \lambda_j) \circ \pi_j
            = \sum_{j=1}^{n} \lambda_{j,i} \circ \pi_j
            = \id_{C_i} \circ \pi_i.
            = \pi_i.
    \]
    In other words, $\sigma$ is a morphism which makes the following diagram commute for all $i$:
    \begin{cd}
        & X \dar["\pi_i"] \\
        X \rar["\pi_i"'] \urar[dashed, "\sigma"] & C_i
    \end{cd}
    However, $\id_X$ makes the same diagram commute, so the universal property of the product $X$ gives us $\sigma = \id_X$.

    We claim that the morphisms $\lambda_i : C_i \to X$ describe a coproduct.
    Let $f_i : C_i \to Y$ be an arbitrary collection of morphisms.
    Define $f = \sum_{j=1}^n (f_j \circ \pi_j) \in \Hom(X, Y)$, for which we compute
    \[
        f \circ \lambda_i
            = \sum_{j=1}^n f_j \circ (\pi_j \circ \lambda_i)
            = \sum_{j=1}^n f_j \circ \lambda_{i,j}
            = f_i \circ \id_{C_i}
            = f_i.
    \]
    That is, $f$ is a morphism which makes the following diagram commute:
    \begin{cd}
        X \drar[dashed, "f"] \\
        C_i \rar["f_i"'] \uar["\lambda_i"] & Y
    \end{cd}
    Suppose $h : X \to Y$ is a morphism which makes the same diagram commute, i.e., $h \circ \lambda_i = f_i$ for all $i$.
    Then
    \[
        f
            = \sum_{i=1}^{n} (f_i \circ \pi_i)
            = \sum_{i=1}^{n} (h \circ \lambda_i) \circ \pi_i
            = h \circ \sum_{i=1}^{n} (\lambda_i \circ \pi_i)
            = h \circ \sigma
            = h \circ \id_X
            = h.
    \]
    We conclude that $\lambda_i : C_i \to X$ is indeed a coproduct.

    By the uniqueness of coproducts, there is a unique isomorphism $\alpha : \bigsqcup_{i=1}^{n} C_i \to X$.

    Moreover, it follows from Lemma \ref{lem:coprod_iso} that the inclusions $\lambda_i : C_i \to X$ we constructed are precisely the inclusions $\alpha \circ \iota_i$ induced on $X$ by the isomorphism $\alpha$, coming from the coproduct's inclusions $\iota_i : C_i \to \bigsqcup_{i=1}^{n} C_i$.
\end{proof}

\newpage
\begin{pbox}[5]
    Let $G$ be a group, $K$ a field. Consider the category $\Rep_KG$ whose objects are the group homomorphisms $\rho : G \to \GL(V)$, $g \mapsto \rho_g$, where $V$ is any vector space over $K$; a morphism from $\rho : G \to \GL(V)$ to $\sigma : G\to \GL(V)$ is a map $f \in \Hom_K(V, W)$ such that $f\circ \rho_g = \sigma_g \circ f$ for all $g \in G$. Clearly, $\Rep_K G$ is a pre-additive category.
    
    Moreover, consider the group algebra $KG$, defined as follows: As a $K$-vector space, $KG$ is the vector space on basis $G$, whence its elements can be represented as finite sums $\sum k_g g$ with $k_g \in K$ and $g \in G$. The algebra multiplication on $KG$ mimics the multiplication of $G$, namely
    \[
        \left(\sum_{g\in G, \text{finite}} k_g g\right)\left(\sum_{h\in G, \text{finite}} l_h h\right)
            := \sum_{u\in G, \text{finite}} \left(\sum_{gh=u} k_g l_h\right) u.
    \]
    Show that the categories $\Rep_KG$ and $KG\lMod$ are naturally equivalent by way of additive functors (they are even isomorphic as categories, a rare phenomenon).
\end{pbox}

We will construct a functor $\Phi : \Rep_KG \to KG\lMod$.

On objects $(\rho : G \to \GL(V) \in \Rep_KG)$, we define $\Phi(\rho)$ as follows.
The group homomorphism $\rho : G \to \GL(V)$ induces a $K$-algebra homomorphism $\tilde{\rho} : KG \to \operatorname{End}_K(V)$.
Here, $\tilde{\rho}$ is a $K$-linear map characterized on the basis $G$ by $\tilde{\rho}(g) = \rho_g$.
Moreover, we get a left $KG$-multiplication on $V$:
\[\textstyle
    (\sum k_g g) \cdot v
        = \sum k_g \rho_g(v).
\]
This multiplication makes $\Phi(\rho)$ a left $KG$-module with $V$ as its underlying set.

On morphisms $f \in \Hom(\rho : G \to \GL(V), \sigma : G \to \GL(W))$, we define $\Phi(f)$ as follows.
The data of $f$ is a $K$-linear map $V \to W$.
We use this as the underlying map of $\Phi(f) : \Phi(\rho) \to \Phi(\sigma)$.
We already know that $f$ is $K$-linear, so to ensure that it is a homomorphism of left $KG$-modules, we only need to check that it commutes with multiplication by elements of the basis $G$:
\[
    f(g \cdot v)
        = f(\rho_g(v))
        = (f \circ \rho_g)(v)
        = (\sigma_g \circ f)(v)
        = \sigma_g(f(v))
        = g \cdot f(v).
\]

Having defined $\Phi$ on objects and morphisms, we now check that it is indeed a functor.

The identity morphism on $\rho : G \to \GL(V)$ consists of the identity map on $V$.
The data of $\Phi(\rho)$ is also the identity map on $V$, which happens to also be the identity of the left $KG$-module $\Phi(\rho)$.
Thus, $\Phi$ preserves identity morphisms.

The fact that $\Phi$ preserves composition of morphisms follows similarly from the fact that the underlying data of morphisms in both $\Rep_KG$ and $KG\lMod$ are functions between the underlying sets, which $\Phi$ preserves.
Moreover, the composition of morphisms in each category is precisely the composition of the underlying functions.


We now define a functor in the reverse direction: $\Psi : KG\lMod \to \Rep_KG$.

There is a forgetful functor $U : KG\lMod \to K\lMod$ which remembers only the addition and multiplication by $K$.
For an object $M \in KG\lMod$, the multiplication on $M$ is given by a $K$-algebra homomorphism $m : KG \to \operatorname{End}_K(UM)$.
We define $\Psi(M)$ to be the representation $m|_G : G \to \GL(UM)$; since the elements of $G$ are invertible in $KG$ and $m$ is a homomorphism, the images $m(g)$ are invertible in $\operatorname{End}_K(UM)$, hence the map is well-defined.

On morphisms $f \in \Hom(M, N)$, the data of $\Psi(f)$ is provided by the $K$-linear map $Uf : UM \to UN$.
We check that $Uf$ commutes with images of the representations:
\[
    (Uf \circ m_g)(v)
        = f(g \cdot v)
        = g \cdot f(v)
        = n_g(f(v))
        = (n_g \circ Uf)(v).
\]

The fact that $\Psi$ is a functor follows similarly to the previous case in that the data of morphisms are functions between the underlying sets, which $\Psi$ preserves.

We now show that $\Phi$ and $\Psi$ are inverse to each other.

Let $\rho : G \to \GL(V) \in \Rep_KG$.
Then $\Phi(\rho)$ is a left $KG$-module $M$ with underlying set $V$. Moreover, the multiplication on $M$ is characterized by the representation $\rho$.
Then, $\Psi\Phi(\rho)$ is a representation of $G$ in $UM = V$, characterized by the multiplication on $M$, which is precisely the representation $\rho$.
Hence, $\Psi\Phi(\rho) = \rho$.

Conversely, let $M \in KG\lMod$.
Then $\Psi(M)$ is a representation of $G$ in $UM$, characterized by the multiplication on $M$.
Then, $\Phi\Psi(M)$ is a left $KG$-module with underlying set $UM$, characterized by the representation $\Psi(M)$, which is precisely the multiplication on $M$.
Hence, $\Phi\Psi(M) = M$.

Thus, we have strongly inverse functors $\Psi\Phi = \id_{\Rep_KG}$ and $\Phi\Psi = \id_{KG\lMod}$.

Moreover, both $\Phi$ and $\Psi$ are additive functors since the addition of the underlying functions is commutative and both maps simply preserve the underlying sets and functions.

\begin{pbox}
    Deduce that $\Rep_K G$ is an abelian category.
\end{pbox}

Since $\Psi : KG\lMod \to \Rep_KG$ is additive, it carries over the zero object and commutes with finite biproducts.
Moreover, since $\Psi$ is part of an equivalence of categories, it preserves monomorphisms and epimorphisms.
Additionally, the fact that $\Psi$ is part of an equivalence also means it is an exact functor and therefore preserves kernels and cokernels.
Thus, $\Psi$ carries over all the abelian structure of $KG\lMod$ to $\Rep_KG$.


\newpage
\begin{pbox}[6 (a)]
    Let $M\in R\lMod$. Verify that the contravariant functor $\Hom_R(-,M): R\lMod \to \Z\lMod$ is left exact.
\end{pbox}

Let
\begin{cd}
    A \rar["f"] & B \rar["g"] & C \rar & 0
\end{cd}
be an exact sequence in $R\lMod$ and consider the image sequence
\begin{cd}
    0 \rar & \Hom_R(C, M) \rar["g^*"] & \Hom_R(B, M) \rar["f^*"] & \Hom_R(A, M)
\end{cd}
in $\Z\lMod$.

Suppose $\phi \in \ker g^*$, so $0 = g^*\phi = \phi \circ g$.
This means that $\ker\phi \supseteq \im g = C$, so in fact $\phi$ is zero on all of $C$.
Hence, $\ker g^* = 0$.

We have $f^* \circ g^* = (g \circ f)^* = 0$, so $\im g^* \seq \ker f^*$.
It remains to check the opposite inclusion.

Suppose $\psi \in \ker f^*$, so $0 = f^*\psi = \psi \circ f$.
This means that $\ker\psi \supseteq \im f = \ker g$.
We define a map $\phi : C \to M$ as follows: for $c \in C$ pick any $b \in g^{-1}(c)$ then put $\phi(c) = \psi(b)$.
In order for this to be well-defined, we must check that $g(b) = g(b')$ implies $\psi(b) = \psi(b')$ for all $b \in B$.
Indeed, if $g(b) = g(b')$, then we have $g(b - b') = 0$ so 
$b - b' \in \ker g \seq \ker\psi$.
Then $\psi(b - b') = 0$ which implies $\psi(b) = \psi(b')$.
By construction, this gives $\psi = \phi \circ g = g^*\phi$.
Therefore, $\psi \in \im g^*$ and we conclude that $\im g^* = \ker f^*$.



\begin{pbox}[(b)]
    Now let $R=M=\Z$. Show that the functor $\Hom_\Z(-,\Z): \Z\lMod \to \Z\lMod$ fails to be right exact.
\end{pbox}

Consider the short exact sequence
\begin{cd}
    0 \rar & \Z \rar["\cdot n"] & \Z \rar["q"] & \Z/n\Z \to 0
\end{cd}
Taking the image under the contravariant hom-functor, we obtain the sequence
\begin{cd}
    0 \rar & \Hom_\Z(\Z/n\Z, \Z) \rar["q^*"] & \Hom_\Z(\Z, \Z) \rar["(\cdot n)^*"] \dar["\iso"] & \Hom_\Z(\Z, \Z) \rar \dar["\iso"] & 0 \\
    0 \rar & 0 \uar[equals] \rar & \Z \rar["\cdot n"'] & \Z \rar & 0
\end{cd}
But this sequence is not exact in general since multiplying my $n$ for $n \ne \pm1$ is not an isomorphism. 

\end{document}