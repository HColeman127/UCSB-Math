\documentclass[12pt]{article}

% Packages
\usepackage[margin=1in]{geometry}
\usepackage{amsmath, amsthm, amssymb, physics}

% Problem Box
\setlength{\fboxsep}{4pt}
\newsavebox{\mybox}
\newenvironment{problem}
    {\begin{lrbox}{\mybox}\begin{minipage}{0.98\textwidth}}
    {\end{minipage}\end{lrbox}\begin{center}\framebox[\textwidth]{\usebox{\mybox}}\end{center}}

% Options
\renewcommand{\thesubsection}{\thesection(\alph{subsection})}
\allowdisplaybreaks
\addtolength{\jot}{1em}
\theoremstyle{definition}

% Default Commands
\newtheorem{proposition}{Proposition}
\newtheorem{lemma}{Lemma}
\newcommand{\ds}{\displaystyle}
\newcommand{\isp}[1]{\quad\text{#1}\quad}
\newcommand{\N}{\mathbb{N}}
\newcommand{\Z}{\mathbb{Z}}
\newcommand{\Q}{\mathbb{Q}}
\newcommand{\R}{\mathbb{R}}
\newcommand{\C}{\mathbb{C}}
\newcommand{\eps}{\varepsilon}
\renewcommand{\phi}{\varphi}
\renewcommand{\emptyset}{\varnothing}

% Extra Commands



% Document Info
\title{Quiz 2\\
    \large GEOG 191
}
\author{Harry Coleman}
\date{January 19, 2021}

% Begin Document
\begin{document}
\maketitle

\section*{Question 1}

For the objective function, we define
\[
    X = \mqty[x_1 \\ \vdots \\ x_6] \isp{and} c = \mqty[1 \\ 2 \\ 9 \\ 7 \\ 3 \\ 5].
\]
For the constraints, we define
\[
    A = \mqty[
        4.5 & 4.15 & 8.2 & 7.2 & 8.19 & 13 \\
        1 & 1 &  &  &  &  \\
         &  & 1 & 1 &  &  \\
         &  &   &   & 1 & 1
    ]
    \isp{and}
    b = \mqty[20 \\ 1 \\ 1 \\ 1].
\]
Then an equivalent statement of the model is
\[
    \begin{array}{ll}
        \textbf{maximize}   & Z = c^T X \\
        \textbf{subject to} & AX \leq b, \\
                            & X \geq 0.
    \end{array}
\]
Since $x_i$ represents a choice of either going or not going to the $i$th location, we might also specify that this is a binary linear program and add the type constraint that $X$ has only binary values, i.e., $0$ or $1$.

\newpage
\section*{Question 2}

Looking at the objective function, $x_1$ and $x_2$ appear to contribute the least. However, setting both equal to zero makes the second constraint impossible to satisfy with equality. The next best choice, based on this criteria, would be to take the set of nonbasic variables to be $\textrm{NBV} = \{x_1, x_5\}$. Then our set of basic variables is $\textrm{BV} = \{x_2, x_3, x_4, x_6\}$. Then the constraint that $AX = b$ is reduced to
\begin{align*}
    4.15x_2 + 8.2x_3 + 7.2x_4 + 13x_6 &= 20, \\
    x_2 &= 1, \\
    x_3 + x_4 &= 1, \\
    x_6 &= 1.
\end{align*}
Substituting $x_2 = 1$ and $x_6 = 1$ into the first equation, we obtain
\[
    8.2x_3 + 7.2x_4 = 2.85.
\]
With the third equation, we find that $x_3 = - 4.35$ and $x_4 = 5.35$. Ignoring for now the conditions that $X \geq 0$ and $X$ is binary, we have the feasible solution
\[
    X = \mqty[0 \\ 1 \\ -4.35 \\ 5.35 \\ 0 \\ 1]
\]
Because this feasible solution does not satisfy all constraints, it is not a basic feasible solution.

\end{document}