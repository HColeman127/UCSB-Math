\documentclass[12pt]{article}

% Packages
\usepackage[margin=1in]{geometry}
\usepackage{amsmath, amsthm, amssymb}

% Problem Box
\setlength{\fboxsep}{4pt}
\newsavebox{\mybox}
\newenvironment{problem}
    {\begin{lrbox}{\mybox}\begin{minipage}{0.98\textwidth}}
    {\end{minipage}\end{lrbox}\framebox[\textwidth]{\usebox{\mybox}}}

% Default Commands
\renewcommand{\thesubsection}{\thesection(\alph{subsection})}
\newtheorem{proposition}{Proposition}
\newtheorem{lemma}{Lemma}
\newcommand{\ds}{\displaystyle}
\newcommand{\isp}[1]{\quad\text{#1}\quad}
\newcommand{\N}{\mathbb{N}}
\newcommand{\Z}{\mathbb{Z}}
\newcommand{\R}{\mathbb{R}}
\newcommand{\C}{\mathbb{C}}
\newcommand{\eps}{\varepsilon}
\renewcommand{\phi}{\varphi}
\renewcommand{\emptyset}{\varnothing}

% Extra Commands
\newcommand{\teq}{\trianglelefteq}
\newcommand{\Syl}{\operatorname{Syl}}
\newcommand{\<}{\left\langle}
\renewcommand{\>}{\right\rangle}
\newcommand{\Aut}{\operatorname{Aut}}
\newcommand{\id}{\operatorname{id}}
\newcommand{\isom}{\cong}

\begin{document}
 
\title{Homework 9\\
    \large MATH 111A Intro to Abstract Algebra
}
\author{Harry Coleman}
\date{December 9, 2020}
\maketitle

\section{Exercise 4.5.5}
\begin{problem}
    Show that a Sylow $p$-subgroup of $D_{2n}$ is cyclic and normal for every odd prime $p$.
\end{problem}

\begin{proof}
    Let $P$ be a Sylow $p$-subgroup of $D_{2n}$. Suppose, for contradiction, that $sr^j \in P$ for some integer $j$. Then since
    \[
        sr^jsr^j = sr^jr^{-j}s = ss = 1,
    \]
    we have that $\<sr^j\> = \{1, sr^j\} \leq P$. However, this implies that $2 \mid p^\alpha$, which contradicts the fact that $p$ is odd. Therefore, all elements of $P$ are of the form $r^j$ for some integer $j$. That is, $P \leq \<r\>$, implying that $P$ is cyclic. Additionally, for all $r^j \in P$, we have that the conjugacy class of $r^j$ in $D_{2n}$ is the pair $\{r^j, r^{-j}\}$. Since $P$ is a group, we also have $r^{-j} \in P$. Thus, for all $x \in D_{2n}$, we have
    \[
        xr^jx^{-1} \in \{r^j, r^{-j}\} \subseteq P,
    \]
     and $P$ is normal.
    
\end{proof}

\newpage
\section{Exercise 4.5.13}
\begin{problem}
    Prove that a group of order $56$ has a normal Sylow $p$-subgroup for some prime $p$ dividing its order.
\end{problem}

\begin{proof}
    Suppose $|G| = 56 = 7 \cdot 2^3$. By Sylow's theorem, we have
    \[
         n_7 \mid 8 \isp{and} n_7 \equiv 1 \pmod{7},
    \]
    where $n_7 = |\Syl_7(G)|$. The first of these conditions implies $n_7 \in \{1, 2, 4, 8\}$ and the second specifies that $n_7 \in \{1, 8\}$. If $n_7 = 1$, then the unique Sylow $7$-subgroup is normal. Otherwise, suppose $n_7 = 8$, then
    \[
        \Syl_7(G) = \{P_1, \dots, P_8\}.
    \]
    Since all Sylow $7$-subgroups are of prime order $7$, they are cyclic. If $x$ is a non-identity element of $P_i$, then we have $\{1\} \ne \<x\> \leq P_i$. Then the order of $x$ is not $1$, but divides $7$, implying that it is $7$. That is, for all $x \in P_i \setminus \{1\}$, we have $\<x\> = P_i$. Suppose $P_i$ and $P_j$ are distinct Sylow $7$-subgroups, and let $x \in P_i \cap P_j$. If $x$ is not the identity, then $P_i = \<x\> = P_j$, so we must have $x = 1$. Thus, the number of elements of $G$ which are contained in some Sylow $7$-subgroup is
    \[
        n_7 \cdot (7 - 1) = 8 \cdot 6 = 48.
    \]
    Suppose $Q$ is a Sylow $2$-subgroup, then $|Q| = 8$. For any non-identity element $x \in Q$, we have that $|x|$ divides $8$. In particular, this implies that $|x| \ne 7$, so $x$ is not in any Sylow $7$-subgroup. Since there are only seven non-identity elements in $G$ which do not lie in a Sylow $7$-subgroup, then $Q$ must be precisely these seven elements plus the identity. Thus, $Q$ is normal, being the the unique Sylow $2$-subgroup.
    
\end{proof}

\newpage
\section{Exercise 4.5.24}
\begin{problem}
    Prove that if $G$ is a group of order $231$ then $Z(G)$ contains a Sylow $11$-subgroup of $G$ and a Sylow $7$-subgroup is normal in $G$.
\end{problem}

\begin{lemma}
    If $G$ is a group, $N \teq G$, and
    \begin{align*}
        F : G &\to \Aut(N)      &  \phi_g : N &\to N \\
            g &\mapsto \phi_g   &             x &\mapsto gxg^{-1}
    \end{align*}
    then $F$ is a homomorphism.
\end{lemma}

\begin{proof}
    Fix some $g \in G$; we will show that $\phi_g \in \Aut(N)$. Because $N$ is a subgroup of $G$, the identity element of $G$ is also the identity element of $N$. Therefore, if $1$ is the identity element of $N$, then
    \[
        \phi_g(1) = g1g^{-1} = gg^{-1} = 1.
    \]
    And for any $x, y \in N$, we have
    \[
        \phi_g(xy) = g(xy)g^{-1} = gxg^{-1}gyg^{-1} = \phi_g(x)\phi_g(y).
    \]
    Thus, $\phi_g$ is a homomorphism. We now consider the map $\phi_{g^{-1}}$, which we claim to be the inverse of $\phi_g$. This is true, since for any $x \in N$, we have
    \[
        (\phi_g \circ \phi_{g^{-1}})(x) = gg^{-1}xgg^{-1} = x = g^{-1}gxg^{-1}g = (\phi_{g^{-1}} \circ \phi_g)(x).
    \]
    Therefore, as a bijective homomorphism, $\phi_g$ is an isomorphism and, in particular, an automorphism of $N$.
    
    We now show that $F$ is a homomorphism. First, we consider $F(1) = \phi_1$. For any $x \in N$, we find that
    \[
        \phi_1(x) = 1x1^{-1} = x.
    \]
    In other words, $F(1)$ is the identity map on $N$, which is the identity element of the automorphism group $\Aut(N)$. For a given pair $g, h \in G$, consider the maps
    \[
        F(gh) = \phi_{gh} \isp{and} F(g) \circ F(h) = \phi_g \circ \phi_h.
    \]
    For all $x \in P$,
    \begin{align*}
        \phi_{gh}(x) = ghxh^{-1}g^{-1} = g\phi_h(x)g^{-1} = (\phi_g \circ \phi_h)(x),
    \end{align*}
    implying that $F(gh) = F(g) \circ F(h)$, and $F$ is a homomorphism.
    
\end{proof}

\newpage
\begin{proposition}
    If $G$ is a group of order $231$ then $Z(G)$ contains a Sylow $11$-subgroup of $G$ and a Sylow $7$-subgroup is normal in $G$.
\end{proposition}


\begin{proof}
    Suppose $|G| = 231 = 3 \cdot 7 \cdot 11$. Sylow's theorem tells us that
    \[
        n_7 \mid 33 \isp{and} n_7 \equiv 1 \pmod{7}.
    \]
    The first condition tells us $n_7 \in \{1, 3, 11, 33\}$ and the second specifies that $n_7 = 1$. Therefore, the unique Sylow $7$-subgroup of $G$ is normal. Similarly, by Sylow's theorem, we have
    \[
        n_{11} \mid 21 \isp{and} n_{11} \equiv 1 \pmod{11},
    \]
    where first conditions implies $n_{11} \in \{1, 3, 7, 21\}$ and the second specifies $n_{11} = 1$. Therefore, the unique Sylow $11$-subgroup of $G$ is normal.
    
    Let $P$ be the unique Sylow $11$-subgroup of $G$. Let the homomorphism $F : G \to \Aut(P)$ be defined as in Lemma 1. Then we have
    \[
        G/\ker F \isom F(G) \leq \Aut(P),
    \]
    which implies that $|G/\ker F|$ divides $|\Aut(P)|$. Since $P$ is of order $11$, it is isomorphic to the cyclic group $Z_{11}$. Moreover, their automorphism groups are isomorphic, and we have
    \[
        \Aut(P) \isom \Aut(Z_{11}) \isom (\Z / 11\Z)^\times = \{\overline{a} \in \Z/11\Z : (a, 11) = 1\} = \{\overline{1}, \dots, \overline{10}\}.
    \]
    Thus, $|\Aut(P)| = 10$. Since $|G/\ker F|$ also divides $|G|$, then
    \[
        |G/\ker F| \leq (10, 231) = 1.
    \]
    This implies that $|\ker F| = |G|$, so $\ker F = G$. This means that for all $g \in G$, because $F(g)$ to be the identity map on $P$. Therefore, for all $x \in P$ and $g \in G$,
    \[
        gxg^{-1} = \phi_g(x) = F(g)(x) = x,
    \]
    implying that $P \subseteq Z(G)$.
    
\end{proof}

\newpage
\section{Exercise 4.5.32}
\begin{problem}
    Let $P$ be a Sylow $p$-subgroup of $H$ and let $H$ be a subgroup of $K$. If $P \teq H$ and $H \teq K$, prove that $P$ is normal in $K$. Deduce that if $P \in \Syl_p(G)$ and $H = N_G(P)$, then $N_G(H) = H$ (in words: normalizers of Sylow $p$-subgroups are self-normalizing).
\end{problem}

\begin{proposition}
    Let $P$ be a Sylow $p$-subgroup of $H$ and let $H$ be a subgroup of $K$. If $P \teq H$ and $H \teq K$, then $P$ is normal in $K$.
\end{proposition}

\begin{proof}
    Let $k \in K$ and let the automorphism $\phi_k : H \to H$ be defined as in Lemma 1. Then because $P$ is a subgroup of $H$, we have
    \[
        kPk^{-1} = \phi_k(P) \leq \phi_k(H) = H.
    \]
    Since $\phi_k$ is injective, then $|\phi_k(P)| = |P|$, implying that $kPk^{-1}$ is a Sylow $p$-subgroup of $H$. Since $P$ is a normal subgroup of $H$, then it is the unique Sylow $p$-subgroup of $H$. Thus, we have $kPk^{-1} = P$, and $P$ is normal in $K$.
    
\end{proof}

\begin{proposition}
    If $P \in \Syl_p(G)$ and $H = N_G(P)$, then $N_G(H) = H$
\end{proposition}

\begin{proof}
    Note that $|G| = m|P|$ where $p$ does not divide $m$. Because $P \leq H$, then $|H| = n|P|$ for some integer $n$. Because $H \leq G$, then $n$ divides $m$. Therefore, $p$ does not divide $n$ and $P$ is a Sylow $p$-subgroup of $H$.
    
    Let $K = N_G(H)$. Then we have $P \teq H$ and $H \teq K$, and, by Proposition 2, $P$ is normal in $K$. That is,
    \[
        N_G(H) = K \subseteq N_G(P) = H.
    \]
    And since $H \subseteq N_G(H)$, then we have equality $H = N_G(H)$.
    
\end{proof}


\end{document}