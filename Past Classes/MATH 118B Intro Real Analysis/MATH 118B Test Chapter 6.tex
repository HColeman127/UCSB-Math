\documentclass[12pt]{article}

% Packages
\usepackage[margin=1in]{geometry}
\usepackage{fancyhdr}
\usepackage{amsmath, amsthm, amssymb, physics}

% Page Style
\fancypagestyle{plain}{
    \fancyhf{}
    \renewcommand{\headrulewidth}{0pt}
    \renewcommand{\footrulewidth}{0pt}
    \fancyfoot[R]{\thepage}
}
\pagestyle{plain}

% Problem Box
\setlength{\fboxsep}{4pt}
\newsavebox{\savefullbox}
\newenvironment{fullbox}{\begin{lrbox}{\savefullbox}\begin{minipage}{\dimexpr\textwidth-2\fboxsep\relax}}{\end{minipage}\end{lrbox}\begin{center}\framebox[\textwidth]{\usebox{\savefullbox}}\end{center}}
\newenvironment{pbox}[1][]{\begin{fullbox}\ifx#1\empty\else\paragraph{#1}\fi}{\end{fullbox}}

% Options
\renewcommand{\thesubsection}{\thesection(\alph{subsection})}
\allowdisplaybreaks
\addtolength{\jot}{4pt}
\theoremstyle{definition}

% Default Commands
\newtheorem{proposition}{Proposition}
\newtheorem{lemma}{Lemma}
\newcommand{\ds}{\displaystyle}
\newcommand{\isp}[1]{\quad\text{#1}\quad}
\newcommand{\N}{\mathbb{N}}
\newcommand{\Z}{\mathbb{Z}}
\newcommand{\Q}{\mathbb{Q}}
\newcommand{\R}{\mathbb{R}}
\newcommand{\C}{\mathbb{C}}
\newcommand{\eps}{\varepsilon}
%\renewcommand{\phi}{\varphi}
\renewcommand{\emptyset}{\varnothing}
\newcommand{\pfrac}[2]{\left(\frac{#1}{#2}\right)}

% Extra Commands
\newcommand{\RR}{\mathcal{R}}


% Document Info
\fancypagestyle{title}{
    \renewcommand{\headrulewidth}{0.4pt}
    \setlength{\headheight}{15pt}
    \fancyhead[R]{Harry Coleman}
    \fancyhead[L]{MATH 118B Test Chapter 6}
    \fancyhead[C]{February 21, 2021}
}

% Begin Document
\begin{document}
\thispagestyle{title}


\begin{pbox}[1]
    Suppose that $f$ is continuous, nonnegative, and decreasing on $[1,+\infty)$. Consider the series $\sum_{n=1}^\infty f(n)$. 
\end{pbox}

\begin{pbox}[(a)]
    Prove that $\sum_{n=1}^\infty f(n)$ converges if and only if 
    \begin{equation}
    \int_1^\infty f(x)\,dx
    \end{equation}
    converges.
\end{pbox}

\begin{proof}
    Suppose $\sum_{n=1}^{\infty} f(n)$ converges, we want to show that
    \[
        \int_1^\infty f(x) \dd{x} = \lim_{b \to \infty} \int_1^b f(x) \dd{x}
    \]
    converges. For a given $b > 1$, let $n \in \N$ be the maximum natural number less than $b$. Since $f$ is continuous, $f \in \RR([1, b])$ and
    \[
        \int_1^b f(x) \dd{x} = \sum_{k=1}^{n-1}\int_{k}^{k+1} f(x) \dd{x} + \int_{n}^{b} f(x) \dd{x}.
    \]
    Since $f$ is decreasing, then $f(x) \leq f(k)$ for all $x \in [k, k+1]$ and $f(x) \leq n$ for all $x \in [n, b]$. Additionally, $b - n \leq 1$, otherwise, $n + 1 < b$ contradicting the choice of $n$. Therefore, we can bound the integral of $f$ over $[1, b]$ by
    \begin{align*}
        \int_1^b f(x) \dd{x}
            \leq \sum_{k=1}^{n-1} f(k)(k+1-k) + f(n)(b-n)
            \leq \sum_{k=1}^{n}f(k).
    \end{align*}
    Note that the choice of $n$ means $n \to \infty$ as $b \to \infty$. Since $f$ is nonnegative, then the integral over $[1,b]$ is increasing as $b \to \infty$. Thus, the convergence of the improper integral is implied by the upper bound
    \[
        \int_1^\infty f(x) \dd{x}
            = \lim_{b \to \infty} \int_1^b f(x) \dd{x}
            \leq \lim_{b \to \infty} \sum_{k=1}^{n}f(k)
            = \sum_{n=1}^{\infty}f(n).
    \]
    
    Suppose $\int_{1}^{\infty} f(x) \dd{x}$ converges. Then, in particular, the limit
    \[
        \lim_{n \to \infty} \int_{1}^{n} f(x) \dd{x}
            = \lim_{n \to \infty} \sum_{k=2}^{n} \int_{k-1}^{k} f(x) \dd{x}
    \]
    converges and equals the improper integral. Since $f$ is decreasing, then $f(x) \geq f(k)$ for all $x \in [k-1, k]$. Then we can bound the terms
    \[
        f(k)
            = f(k)(k - (k-1))
            \leq \int_{k-1}^{k} f(x) \dd{x} .
    \]
    Since $f$ is nonnegative, then the sequence of partial sums is increasing. So the convergence of the infinite sum is implied by the upper bound
    \[
        \sum_{n=2}^{\infty} f(n)
            = \lim_{n \to \infty} \sum_{k=2}^{n} f(k)
            \leq \lim_{n \to \infty} \sum_{k=2}^{n} \int_{k-1}^{k} f(x) \dd{x}
            = \int_1^\infty f(x) \dd{x}.
    \]
    And since $f(1)$ is simply some finite value, then $\sum_{1}^{\infty} f(n)$ also converges.
    
\end{proof}

\begin{pbox}[(b)]
    Use the previous part to show that 
    \begin{equation}
    \sum_{n=1}^\infty \frac{1}{n^p}
    \end{equation}
    converges if and only if $p>1$.
\end{pbox}

\begin{proof}
    If $p \leq 1$, then for any $n \in \N$, we have $n^{p} \leq n$, which implies $1/n \leq 1/n^p$. And we know that the harmonic series diverges, so $\sum_{1}^{\infty} 1/n^p$ diverges.
    
    Suppose $p>1$, and consider the function $f(x) = x^{-p}$. Then $f$ is continuous, nonnegative, and decreasing on $[1, +\infty)$. By the previous result, it suffices to show that $\int_{1}^{\infty} f(x) \dd{x}$ converges. If it exists, the improper integral is given by
    \[
        \int_{1}^{\infty} f(x) \dd{x} = \lim_{b \to \infty} \int_{1}^{b} f(x) \dd{x}.
    \]
    Consider the function $F(x) = x^{1-p}/(1-p)$, which is well defined as $1-p \ne 0$, and whose derivative we find using the power rule to be
    \[
        F'(x) = (1-p)\frac{x^{(1-p)-1}}{1-p} = x^{-p} = f(x).
    \]
    Then by the fundamental theorem of calculus, we have
    \[
        \int_{1}^{b} f(x) \dd{x}
            = F(b) - F(1)
            = \frac{b^{1-p} - 1}{1-p}.
    \]
    Now since $1-p < 0$, then $b^{1-p} \to 0$ as $b \to \infty$. So taking the limit, we obtain convergence of the improper integral
    \[
        \int_{1}^{\infty} f(x) \dd{x}
            = \lim_{b\to\infty} \frac{b^{1-p} - 1}{1-p}
            = \frac{1}{p-1}.
    \]
    By part (a), this implies convergence of the infinite sum
    \[
        \sum_{n=1}^{\infty} f(n) = \sum_{n=1}^{\infty} \frac{1}{x^p}.
    \]
    
\end{proof}






\begin{pbox}[2]
    Let $f:[a,b]\to \R$ be continuous and such that 
    \begin{equation}
    \int_c^df(x)\,dx = 0
    \end{equation}
    for any interval $[c,d]\subseteq [a,b]$. Show that $f$ is identically zero.
\end{pbox}

\begin{proof}
    Suppose, for contradiction, that $f$ is not identically zero and let $x_0 \in [a,b]$ such that $f(x_0) \ne 0$. Since $f$ is continuous, there exists some $\delta > 0$ such that $|x - x_0| < \delta$ implies $|f(x) - f(x_0)| < |f(x_0)|$. We define the points
    \[
        c = \max\{a, x_0 - \delta/2\},
        \isp{and}
        d = \min\{b, x_0 + \delta/2\},
    \]
    so $[c,d] \subseteq (x_0-\delta, x_0+\delta)$ and $[c,d] \subseteq [a,b]$. Then $|f(x) - f(x_0)| < |f(x_0)|$ for all $x \in [c, d]$, implying $f(x) \ne 0$ for all $x \in [c,d]$. However, the mean value theorem for integrals tells us that there exists some $y_0 \in [c, d]$ such that
    \[
        f(y_0) = \frac{1}{d-c}\int_{c}^{d} f(x) \dd{x} = 0,
    \]
    which is a contradiction.
    
\end{proof}





\newpage
\begin{pbox}[3]
    Let $f:[a,b]\to \R$ be a bounded function. Suppose that there exist partitions $\{P_n\}_{n\in\N}$ and $\{Q_n\}_{n\in \N}$, such that 
    \begin{equation}
    \lim_{n\to \infty} (U(P_n,f)-L(Q_n,f))=0. 
    \end{equation}
    Prove that $f\in \mathcal{R}([a,b])$ and 
    \begin{equation}
    \int_a^b f(x)\,dx = \lim_{n\to \infty} U(P_n,f)= \lim_{n\to \infty} L(Q_n,f).
    \end{equation}
\end{pbox}

\begin{proof}
    For each $n \in \N$, the partition $S_n = P_n \cup Q_n$ is a refinement of both $P_n$ and $Q_n$. So
    \[
        L(Q_n, f) \leq L(S_n, f) \leq U(S_n, f) \leq U(P_n, f),
    \]
    which implies
    \[
        0 \leq U(S_n, f) - L(S_n, f) \leq U(P_n, f) - L(Q_n, f).
    \]
    Hence,
    \[
        \lim_{n \to \infty} (U(S_n, f) - L(S_n, f)) = 0.
    \]
    In other words, for any $\eps > 0$ there exists a partition of $[a,b]$, namely $S_n$, such that
    \[
        U(S_n, f) - L(S_n, f) < \eps,
    \]
    so $f \in \RR([a,b])$. Now for all $n \in \N$, we know that
    \[
        L(Q_n, f) \leq \int_{a}^{b} f(x) \dd{x} \leq U(P_n, f),
    \]
    so
    \begin{align*}
        0 \leq \int_{a}^{b} f(x) \dd{x} - L(Q_n, f) &\leq U(P_n,f) - L(Q_n,f), \\
        0 \leq U(P_n, f) - \int_{a}^{b} f(x) \dd{x} &\leq U(P_n,f) - L(Q_n,f).
    \end{align*}
    By the squeeze theorem, both of these go to zero, and we obtain
    \[
        \lim_{n\to\infty} L(Q_n, f) = \int_{a}^{b} f(x) \dd{x} = \lim_{n\to\infty} U(P_n, f).
    \]
    
    
\end{proof}



\newpage
\begin{pbox}[4]
    Evaluate the following limits: 
\end{pbox}

\begin{pbox}[(a)]
    Let $0<a<b<+\infty$. Prove that 
    \begin{equation}
    \lim_{n\to \infty} \int_a^b \frac{\cos(nx)}{\sqrt{x}}\,dx = 0.
    \end{equation}
\end{pbox}

\begin{proof}
    For a fixed $n \in \N$, define the functions
    \[
        f(x) = \frac{\cos x}{\sqrt{x}}
        \isp{and}
        \phi(x) = nx.
    \]
    Then $\phi(x)$ is strictly increasing and differentiable on $[a,b]$ with $\phi'(x) = n$. By the change of variables, we have
    \begin{align*}
        \int_a^b \frac{\cos(nx)}{\sqrt{x}} \dd{x}
            &= \frac{1}{\sqrt{n}} \int_{a}^{b} f(\phi(x)) \phi'(x) \dd{x} \\
            &= \frac{1}{\sqrt{n}} \int_{\phi(a)}^{\phi(b)} f(x) \dd{x} \\
            &= \frac{1}{\sqrt{n}} \int_{na}^{nb} f(x) \dd{x}.
    \end{align*}
    Without loss of generality, assume $n$ large enough that $n(b-a) > 4\pi$. Define
    \[
        m_n = \min\{k \in \N : na < 2\pi k\}
        \isp{and}
        M_n = \max\{k \in \N : 2\pi k < nb\},
    \]
    then the integral over $[na, nb]$ can be rewritten as
    \[
        \int_{na}^{nb} f(x) \dd{x}
            = \int_{2\pi m_n}^{2\pi M_n} f(x) \dd{x}
                + \int_{na}^{2\pi m_n} f(x) \dd{x}
                + \int_{2\pi M_n}^{nb} f(x) \dd{x}.
    \]
    By the mean value theorem for integrals, there is some $c_n \in [na,2 \pi m_n]$ such that
    \[
        \int_{na}^{2\pi m_n} f(x) \dd{x} = f(c_n)(2\pi m_n - na).
    \]
    By construction of $m_n$, we have $na < 2\pi m_n \leq  na + 2\pi$, so
    \[
        |f(c_n)(2\pi m_n - na)|
            = \frac{|\cos c_n|}{\sqrt{c_n}} (2\pi m_n - an)
            \leq \frac{2\pi}{\sqrt{na}}
    \]
    Similarly, let $d_n \in [2\pi M_n, nb]$ such that
    \[
        \int_{2\pi M_n}^{nb} f(x) \dd{x} = f(d_n)(b - 2\pi M_n),
    \]
    then
    \[
        |f(c_n)(b - 2\pi M_n)|
            = \frac{|\cos d_n|}{\sqrt{d_n}} (b - 2\pi M_n)
            \leq \frac{2\pi}{\sqrt{na}}
    \]
    Thus, we have the bound
    \[
        \left|\int_{na}^{2\pi m_n} f(x) \dd{x} + \int_{2\pi M_n}^{nb} f(x) \dd{x}\right|
            \leq \frac{4\pi}{\sqrt{na}}
            \leq \frac{4\pi}{\sqrt{a}}.
    \]
    We now rewrite the integral over $[2\pi m_n, 2\pi M_n]$ to be
    \[
        \int_{2\pi m_n}^{2\pi M_n} f(x) \dd{x}
            = \sum_{k=m_n}^{M_n-1} \int_{2\pi k}^{2\pi(k+1)} f(x) \dd{x}.
    \]
    For a fixed $k$, we have
    \begin{align*}
        \int_{2\pi k}^{2\pi(k+1)} f(x) \dd{x}
            &= \int_{2\pi k}^{2\pi k + \pi} \frac{\cos x}{\sqrt{x}} \dd{x}
                + \int_{2\pi k + \pi}^{2\pi k + 2\pi} \frac{\cos x}{\sqrt{x}} \dd{x} \\
            &= \int_{2\pi k}^{2\pi k + \pi} \frac{\cos x}{\sqrt{x}} \dd{x}
                + \int_{2\pi k}^{2\pi k + \pi} \frac{\cos(x + \pi)}{\sqrt{x + \pi}} \dd{x} \\
            &= \int_{2\pi k}^{2\pi k + \pi} \frac{\cos x}{\sqrt{x}} \dd{x}
                + \int_{2\pi k}^{2\pi k + \pi} \frac{-\cos x}{\sqrt{x + \pi}} \dd{x} \\
                &= \int_{2\pi k}^{2\pi k + \pi} \cos x \left(\frac{1}{\sqrt{x}} - \frac{1}{\sqrt{x + \pi}}\right) \dd{x}.
    \end{align*}
    We bound the absolute value of each term by
    \begin{align*}
        \left|\cos x \left(\frac{1}{\sqrt{x}} - \frac{1}{\sqrt{x + \pi}}\right)\right|
            &\leq \left|\frac{1}{\sqrt{x}} - \frac{1}{\sqrt{x + \pi}}\right| \\
            &= \left|\frac{\sqrt{x + \pi} - \sqrt{x}}{\sqrt{x(x+\pi)}} \cdot \frac{\sqrt{x+\pi} + \sqrt{x}}{\sqrt{x+\pi} + \sqrt{x}}\right| \\
            &= \left|\frac{x + \pi - x}{\sqrt{x(x+\pi)^2} + \sqrt{x^2(x+\pi)}}\right| \\
            &= \frac{\pi}{(x+\pi)\sqrt{x} + x\sqrt{x+\pi}} \\
            &\leq \frac{\pi}{x\sqrt{x} + x\sqrt{x}} \\
            &= \frac{\pi}{2x^{3/2}} \\
            &\leq \frac{\pi}{2 (2\pi k)^{3/2}} \\
            &= \frac{1}{\sqrt{32\pi} k^{3/2}}.
    \end{align*}
    Then we can bound the absolute value of the integral by
    \[
        \left|\int_{2\pi k}^{2\pi(k+1)} f(x) \dd{x}\right|
            \leq \frac{1}{\sqrt{32\pi} k^{3/2}}(2\pi k + \pi - 2\pi k)
            = \frac{\sqrt{\pi/32}}{k^{3/2}}.
    \]
    With this, we bound the absolute value of the summation by
    \[
        \left|\sum_{k=m_n}^{M_n-1} \int_{2\pi k}^{2\pi(k+1)} f(x) \dd{x}\right|
            \leq \sum_{k=m_n}^{M_n-1} \frac{\sqrt{\pi/32}}{k^{3/2}}
            = (\pi/32)^{1/2} \sum_{k=m_n}^{M_n-1} \frac{1}{k^{3/2}}
    \]
    From 1(b), we know that the infinite sum
    \[
        S = \sum_{k=1}^{\infty} \frac{1}{k^{3/2}}
    \]
    converges, and since each term is positive,
    \[
        \sum_{k=m_n}^{M_n-1} \frac{1}{k^{3/2}} \leq S.
    \]
    Thus, we have bounded the absolute value of the integral
    \[
        \left|\int_{2\pi m_n}^{2\pi M_n} f(x) \dd{x}\right| \leq (\pi/32)^{1/2}S.
    \]
    This now gives us the bound
    \[
        \left|\int_{na}^{nb} f(x) \dd{x}\right| \leq  (\pi/32)^{1/2}S + \frac{4\pi}{\sqrt{a}} = C,
    \]
    where $C$ is a positive constant. We can now bound the absolute value of the original integral by
    \[
        \left|\int_a^b \frac{\cos(nx)}{\sqrt{x}} \dd{x}\right|
            = \left| \frac{1}{\sqrt{n}} \int_{na}^{nb} f(x) \dd{x}\right|
            \leq \frac{C}{\sqrt{n}}.
    \]
    And since $C/\sqrt{n} \to 0$ as $n \to \infty$, then we have
    \[
        \lim_{n \to \infty} \int_a^b \frac{\cos(nx)}{\sqrt{x}} \dd{x}
            = 0.
    \]
    
    
    
\end{proof}



\newpage
\begin{pbox}[(b)]
    \begin{equation}
    \lim_{n\to\infty} \frac{2}{n} \sqrt[n]{n!}.
    \end{equation}
\end{pbox}

For a fixed $n$, we write
\begin{align*}
    \frac{2}{n}\sqrt[n]{n!}
        &= 2 \pfrac{n!}{n^n}^{1/n} \\
        &= 2 \left(\prod_{k=1}^{n} \frac{k}{n} \right)^{1/n} \\
        &= 2 \exp(\frac{1}{n} \log(\prod\limits_{k=1}^{n} \frac{k}{n})) \\
        &= 2 \exp(\frac{1}{n} \sum\limits_{k=1}^{n}\log \frac{k}{n}).
\end{align*}
If the logarithm is Riemann integrable on $[0,1]$, then the integral is given by the
\[
    \int_0^1 \log x \dd{x} = \lim_{n\to \infty} \frac{1}{n} \sum\limits_{k=1}^{n}\log \frac{k}{n}.
\]
We show this limit exists by evaluating
\[
    \lim_{a \to 0} \int_a^1 \log x \dd{x}.
\]
Since the logarithm is continuous on $[a, 1]$ for $a \in (0,1)$, then it is integrable. Moreover, letting $g(x) = x\log x - x$, we find
\begin{align*}
    g'(x)
        &= x\dv{x} \log x + \log x \dv{x}x - \dv{x}x \\
        &= x\frac{1}{x} + \log x - 1 \\
        &= \log x.
\end{align*}
Then by the fundamental theorem of calculus, we have
\[
    \int_a^1 \log x \dd{x}
        = g(1) - g(a)
        = -1 - a\log a + a.
\]
We now evaluate the limit of $a \log a$ as $a \to 0$. We can rewrite this limit as
\[
    \lim_{a \to 0} a \log a
        = \lim_{x \to \infty} \frac{\log(1/x)}{x}.
\]
Applying L'H\^opital's rule, we find
\begin{align*}
    \lim_{x \to \infty} \frac{\log(1/x)}{x}
        &= \lim_{x \to \infty} \frac{\dv{x}\log(1/x)}{\dv{x}x} \\
        &= \lim_{x \to \infty} \frac{1}{1/x}\dv{x}\frac{1}{x} \\
        &= \lim_{x \to \infty} x\frac{-1}{x^2} \\
        &= \lim_{x \to \infty} \frac{-1}{x} \\
        &= 0.
\end{align*}
Therefore, we have the limit of the integral
\[
    \lim_{a \to 0} \int_a^1 \log x \dd{x}
        = \lim_{a \to 0}(-1 - a\log a + a)
        = -1.
\]
Since the exponential function is continuous, then we find
\begin{align*}
    \lim_{n\to\infty} \exp(\frac{1}{n} \sum\limits_{k=1}^{n}\log \frac{k}{n})
        &= \exp(\lim_{n\to\infty} \frac{1}{n} \sum\limits_{k=1}^{n}\log \frac{k}{n}) \\
        &= \exp(\int_0^1 \log x \dd{x}) \\
        &= \exp(-1) \\
        &= \frac{1}{e}.
\end{align*}
Hence,
\[
    \lim_{n\to\infty} \frac{2}{n}\sqrt[n]{n!} = \frac{2}{e}.
\]
    



\newpage
\begin{pbox}[5]
    Examine the existence of the following improper integrals:
\end{pbox}

\begin{pbox}[(a)]
    \begin{equation}
    \int_2^{+\infty} \frac{\sqrt{x+1}-\sqrt{x}}{x}\,dx.
    \end{equation}
\end{pbox}

For $x \geq 2$, we have
\begin{align*}
    \frac{\sqrt{x+1}-\sqrt{x}}{x}
        &= \frac{\sqrt{x+1}-\sqrt{x}}{x} \cdot \frac{\sqrt{x+1} + \sqrt{x}}{\sqrt{x+1} + \sqrt{x}} \\
        &= \frac{1}{x(\sqrt{x+1} + \sqrt{x})} \\
        &\leq \frac{1}{x(\sqrt{x} + \sqrt{x})} \\
        &= \frac{1}{2x^{3/2}}.
\end{align*}
From 1(a) and 1(b), we know that the improper integral
\[
    \int_{1}^{\infty} \frac{1}{x^{3/2}}
\]
converges since $3/2 > 1$. In particular, the integral
\[
    \int_{2}^{\infty} \frac{1}{x^{3/2}}
\]
converges. The function
\[
    \frac{\sqrt{x+1}-\sqrt{x}}{x}
\]
is continuous, and therefore integrable, on the interval $[2, b]$ for $b > 2$. Moreover, it is positive, so the bound
\[
    0 \leq \int_{2}^{b}\frac{\sqrt{x+1}-\sqrt{x}}{x} \leq \int_{2}^{b} \frac{1}{x^{3/2}}
\]
implies the convergence of the improper integral when $b \to \infty$,
\[
    0 \leq \int_{2}^{\infty} \frac{\sqrt{x+1}-\sqrt{x}}{x}
        \leq \int_{2}^{\infty} \frac{1}{x^{3/2}}.
\]


\newpage
\begin{pbox}[(b)]
    \begin{equation}
    \int_0^{\frac{\pi}{2}} \tan x\,dx.
    \end{equation}
\end{pbox}

If it exists, the improper integral is given by
\[
    \int_0^{\frac{\pi}{2}} \tan x \dd{x}
        = \lim_{b \to \pi/2} \int_0^{b} \tan x \dd{x}.
\]
Since the tangent function is continuous on $[0, b]$ for $b \in (0, \pi/2)$, then it is integrable. Additionally, letting $g(x) = -\log \cos x$, we find
\[
    g'(x) 
        = \frac{-1}{\cos x} \dv{x} \cos x
        = \frac{\sin x}{\cos x}
        = \tan x.
\]
So by the fundamental theorem of calculus, we have
\[
    \int_0^{b} \tan x \dd{x}
        = g(b) - g(0)
        = -\log\cos b + \log\cos 0
        = -\log\cos b.
\]
Since $\cos b \to 0$ as $b \to \pi/2$ and $\log x \to -\infty$ as $x \to 0$, then
\[
    \lim_{b \to \pi/2} \int_0^{b} \tan x \dd{x} = +\infty.
\]
In other words, the improper integral does not converge.

\end{document}