\documentclass[11pt]{article}
 
\usepackage[table]{xcolor}
\usepackage[margin=1in]{geometry} 
\usepackage{amsmath,amsthm,amssymb}
\usepackage{listings}
\usepackage{tikz}
\usepackage{colortbl}
\usepackage{verbatim}
\usetikzlibrary{arrows, angles, quotes}
\usepackage{framed}


\lstset{basicstyle=\footnotesize}
\usetikzlibrary{calc}

\newcommand{\N}{\mathbb{N}}
\newcommand{\Z}{\mathbb{Z}}
\newcommand{\I}{\mathbb{I}}
\newcommand{\R}{\mathbb{R}}
\newcommand{\Q}{\mathbb{Q}}
 
\begin{document}
 
\title{Homework 7\\
    \large CS128 Intro to Higher Mathematics}
\author{Harry Coleman}
\date{October 25, 2019}

\maketitle

\section*{Exercise 11}
\fbox{
    \parbox{\textwidth}{
       Show that every nonzero real number has a unique multiplicative inverse.
    }
}
\\

We can write the statement
\begin{center}
    \emph{Every nonzero real number has a unique multiplicative inverse.}
\end{center}
as the following:
\[\forall x(((x\in\R) \land (x \ne 0)) \implies \exists y((x \cdot y = 1) \land \forall z(((x \cdot z = 1)) \implies (y = z))))\]

To prove this is true, we'll assume the negation, and show that a contradiction follows. The negation of our formula is
\[\lnot \forall x(((x\in\R) \land (x \ne 0)) \implies \exists y((x \cdot y = 1) \land \forall z(((x \cdot z = 1)) \implies (y = z))))\]

We'll simplify this using a number of logical equivalences.

\begin{center}
    \rowcolors{1}{gray!10}{white}
    \begin{tabular}{c l}
        $\exists x(\lnot(((x\in\R) \land (x \ne 0)) \implies \exists y((x \cdot y = 1) \land \forall z((x \cdot z = 1) \implies (y = z)))))$ & by Negation of Quantifiers \\
        $\exists x(\lnot(\lnot ((x\in\R) \land (x \ne 0)) \lor \exists y((x \cdot y = 1) \land \forall z(\lnot (x \cdot z = 1) \lor (y = z)))))$ & by Definition of Implication \\
        $\exists x(\lnot(\lnot ((x\in\R) \land (x \ne 0))) \land \lnot \exists y((x \cdot y = 1) \land \forall z(\lnot (x \cdot z = 1) \lor (y = z))))$ & by De Morgan's Laws \\
        $\exists x(((x\in\R) \land (x \ne 0)) \land \lnot \exists y((x \cdot y = 1) \land \forall z(\lnot (x \cdot z = 1) \lor (y = z))))$ & by Double Negation \\
        $\exists x(((x\in\R) \land (x \ne 0)) \land \forall y(\lnot((x \cdot y = 1) \land \forall z(\lnot (x \cdot z = 1) \lor (y = z)))))$ & by Negation of Quantifiers \\
        $\exists x(((x\in\R) \land (x \ne 0)) \land \forall y(\lnot(x \cdot y = 1) \lor \lnot \forall z(\lnot (x \cdot z = 1) \lor (y = z))))$ & by De Morgan's Laws \\
        $\exists x(((x\in\R) \land (x \ne 0)) \land \forall y(\lnot(x \cdot y = 1) \lor \exists z(\lnot(\lnot (x \cdot z = 1) \lor (y = z)))))$ & by Negation of Quantifiers \\
        $\exists x(((x\in\R) \land (x \ne 0)) \land \forall y(\lnot(x \cdot y = 1) \lor \exists z(\lnot(\lnot (x \cdot z = 1)) \land \lnot (y = z))))$ & by De Morgan's Laws \\
        $\exists x(((x\in\R) \land (x \ne 0)) \land \forall y(\lnot(x \cdot y = 1) \lor \exists z((x \cdot z = 1) \land \lnot (y = z))))$ & by Double Negation \\
        $\exists x(((x\in\R) \land (x \ne 0)) \land \forall y((x \cdot y \ne 1) \lor \exists z((x \cdot z = 1) \land  (y \ne z))))$ & by $\lnot(a=b) \equiv (a\ne b)$\\
    \end{tabular}
\end{center}

Let $a$ be a specific constant which satisfies for $x$ in our equation.

Let $b$ be an arbitrary constant.

Let $c$ be a specific constant which satisfies for $z$ in our equation.

\newpage
By Existential and Universal Instantiation, we get
\[1 \quad ((a\in\R) \land (a \ne 0)) \land ((a \cdot b \ne 1) \lor ((a \cdot c = 1) \land  (b \ne c)))\]

\begin{center}
    \rowcolors{1}{gray!10}{white}
    \begin{tabular}{r c l l}
        $\#$ & Statement & Reasoning & on Steps \\
        \hline
        2 & $(a \cdot b \ne 1) \lor ((a \cdot c = 1) \land  (b \ne c))$ & by Simplification & 1 \\
        3 & $\lnot(a \cdot b \ne 1) \implies ((a \cdot c = 1) \land  (b \ne c))$ & by Definition of Implication & 2 \\
        4.1 & $\lnot(a \cdot b \ne 1)$ & Assumption for Conditional Method & \\
        4.2 & $\lnot(\lnot(a \cdot b = 1))$ & by $\lnot(a=b) \equiv (a\ne b)$ & 4.1 \\
        4.3 & $a \cdot b = 1$ & by Double Negation & 4.2 \\
        4.4 & $(a \cdot c = 1) \land  (b \ne c)$ & by Modus Ponens & 3, 4.1 \\
        4.5 & $a \cdot c = 1$ & by Simplification & 4.4 \\
        4.6 & $a \cdot b = a \cdot c$ & by Transitivity of Equality & 4.3, 4.5 \\
        4.7 & $b = c$ & by Multiplication Property of Equality & 4.6 \\
        4.8 & $b \ne c$ & Simplification & 4.4 \\
        4.9 & F & by Contradiction & 4.7,4.8 \\
        13 & $\lnot(\lnot(a \cdot b \ne 1) \implies ((a \cdot c = 1) \land  (b \ne c)))$ & by Conditional Method & 4 \\
        14 & F & by Contradiction & 3, 13
    \end{tabular}
\end{center}

Therefore, our assumption, which was the negation of our original statement, is false, so our original statement is true.
\begin{center}
    \emph{Every nonzero real number has a unique multiplicative inverse.}
\end{center}

\section*{Exercise 8}
\fbox{
    \parbox{\textwidth}{
       Show $S(a)$ from the premises:
        \begin{itemize}
            \item $\forall x (R(x) \leftrightarrow \forall y S(y))$
            \item $\exists x R(x)$.
        \end{itemize}
    }
}
\\

\begin{center}
    \rowcolors{1}{gray!10}{white}
    \begin{tabular}{r c l l}
        $\#$ & Statement & Reasoning & on Steps \\
        \hline
        1 & $\forall x (R(x) \leftrightarrow \forall y S(y))$ & Premise & \\
        2 & $\exists x R(x)$ & Premise & \\
        \hline
        3 & \multicolumn{2}{l}{Let $c$ be a specific constant which satisfies for $x$} & 2 \\
        4 & $R(c)$ & by Existential Instantiation & 2, 3 \\
        5 & $R(c) \leftrightarrow \forall y S(y)$ & Universal Instantiation & 1, 3 \\
        6 & $\forall y S(y)$ & Modus Ponens & 4, 5 \\
        7 & \multicolumn{3}{l}{Let $a$ be an arbitrary constant} \\
        8 & $S(a)$ & Universal Instantiation & 6, 7 \\
    \end{tabular}
\end{center}


\newpage
\section*{Exercise 13}
\fbox{
    \parbox{\textwidth}{
       Show that for every natural number $n$, if $n \equiv 2$ mod 3, then $n$ is not a square.
    }
}
\\

Every natural number in (mod 3) is either 0, 1, or 2.

\[\forall x ((x\in\N) \implies ((x\text{ mod } 3) \in \{0, 1, 2\}))\]

Therefore, any natural number squared would be one of
\[0^2 \equiv 0 \pmod{3}\]
\[1^2 \equiv 1 \pmod{3}\]
\[2^2 \equiv 1 \pmod{3}\]
So
\[\forall x ((x\in\N) \implies ((x^2\text{ mod } 3) \in \{0, 1\}))\]

Since we want to prove that
\begin{center}
    \emph{For every natural number $n$, if $n \equiv 2$ mod 3, then $n$ is not a square}
\end{center}
or
\[\forall n(((n\in\N) \land (n \equiv 2 \pmod{3})) \implies \lnot\exists k((k\in\N) \land (n = k^2))\]
we'll assume the negation is true and show it leads to a contradiction. The negation is
\[\lnot\forall n(((n\in\N) \land (n \equiv 2 \pmod{3})) \implies \lnot\exists k((k\in\N) \land (n = k^2))\]

Through a number of logical equivalences, we can get this to be

\[\exists n(((n\in\N) \land (n \equiv 2 \pmod{3})) \land \exists k((k\in\N) \land (n = k^2)))\]

Let $c$ be a specific constant which satisfies for $n$ in the above formula. 

Let $m$ be a specific constant which satisfies for $k$ in the above formula.

So through Existential Instantiation, we get

\[((c\in\N) \land (c \equiv 2 \pmod{3})) \land ((m\in\N) \land (c = m^2)))\]

\begin{center}
    \rowcolors{1}{gray!10}{white}
    \begin{tabular}{r c l l}
        $\#$ & Statement & Reasoning & on Steps \\
        \hline
        1 & $((c\in\N) \land (c \equiv 2 \pmod{3})) \land ((m\in\N) \land (c = m^2)))$ & Premise & \\
        2 & $\forall x ((x\in\N) \implies ((x^2\text{ mod } 3) \in \{0, 1\}))$ & Premise & \\
        \hline
        3 & $c \equiv 2 \pmod{3}$ & Simplification & 1 \\
        4 & $c = m^2$ & Simplification & 1 \\
        5 & $(m^2 \equiv 0 \pmod{3}) \lor (m^2 \equiv 1 \pmod{3})$ & Existential Instantiation & 2 \\
        6 & $(c \equiv 0 \pmod{3}) \lor (c \equiv 1 \pmod{3})$ & Substitution & 4, 5 \\
        7 & $\lnot(c \equiv 0 \pmod{3})$ & Equivalency & 3 \\
        8 & $\lnot(c \equiv 1 \pmod{3})$ & Equivalency & 3 \\
        9 & F & Condition of Anti-Tautology & 6, 7, 8 \\
    \end{tabular}
\end{center}

Therefore, our assumption is false, so
\begin{center}
    \emph{For every natural number $n$, if $n \equiv 2$ mod 3, then $n$ is not a square}
\end{center}


\section*{Exercise 4}
\fbox{
    \parbox{\textwidth}{
       Write a useful denial of the statement: ``For each pair of real numbers $a$ and $b$ with $a<b$, there is a rational number $r$ such that $a<r<b$.
    }
}
\\

Original statement:
\[\forall a \forall b \exists r(((a\in\R)\land(b\in\R)\land(a<b)) \implies ((r\in\Q)\land(a<r<b)))\]

Negation:
\[\lnot\forall a \forall b \exists r(((a\in\R)\land(b\in\R)\land(a<b)) \implies ((r\in\Q)\land(a<r<b)))\]

By Negation of Quantifiers:
\[\exists a \exists b \forall r(\lnot((a\in\R)\land(b\in\R)\land(a<b)) \implies ((r\in\Q)\land(a<r<b))))\]

By Definition of Implication:
\[\exists a \exists b \forall r(\lnot(\lnot((a\in\R)\land(b\in\R)\land(a<b))) \lor ((r\in\Q)\land(a<r<b))))\]

By De Morgan's:
\[\exists a \exists b \forall r(\lnot(\lnot((a\in\R)\land(b\in\R)\land(a<b))) \land \lnot((r\in\Q)\land(a<r<b)))\]

By Double Negation:
\[\exists a \exists b \forall r((a\in\R)\land(b\in\R)\land(a<b) \land \lnot((r\in\Q)\land(a<r<b)))\]

By De Morgan's:
\[\exists a \exists b \forall r((a\in\R)\land(b\in\R)\land(a<b) \land (\lnot(r\in\Q)\lor \lnot(a<r<b)))\]

By Definition of Implication:
\[\exists a \exists b \forall r((a\in\R)\land(b\in\R)\land(a<b) \land ((a<r<b) \implies (r\notin\Q)))\]

In Natural language:
\begin{center}
    \emph{There exist some pair of real numbers a, b, such that every number between them is not rational}
\end{center}
or
\begin{center}
    \emph{There exist some pair of real numbers a, b, such that there does not exist a rational number between them.}
\end{center}

\end{document}

