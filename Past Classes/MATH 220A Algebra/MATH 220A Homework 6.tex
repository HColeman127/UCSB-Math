\documentclass[12pt]{article}

% Packages
\usepackage[margin=1in]{geometry}
\usepackage{fancyhdr, parskip}
\usepackage{amsmath, amsthm, amssymb}

% Page Style
\makeatletter
\fancypagestyle{title}{
    \renewcommand{\headrulewidth}{0.4pt}
    \setlength{\headheight}{15pt}
    \fancyhead[R]{\@author}
    \fancyhead[L]{\@title}
    \fancyhead[C]{\@date}
}
\makeatother
\renewcommand{\maketitle}{\thispagestyle{title}}
\fancypagestyle{plain}{
    \fancyhf{}
    \renewcommand{\headrulewidth}{0pt}
    \renewcommand{\footrulewidth}{0pt}
    \fancyfoot[R]{\thepage}
}
\pagestyle{plain}

% Problem Box
\setlength{\fboxsep}{4pt}
\newlength{\myparskip}
\setlength{\myparskip}{\parskip}
\newsavebox{\savefullbox}
\newenvironment{fullbox}{\begin{lrbox}{\savefullbox}\begin{minipage}{\dimexpr\textwidth-2\fboxsep\relax}\setlength{\parskip}{\myparskip}}{\end{minipage}\end{lrbox}\framebox[\textwidth]{\usebox{\savefullbox}}}
\newenvironment{pbox}[1][]{\begin{fullbox}\ifx#1\empty\else\paragraph{#1}\fi}{\end{fullbox}}

% Default Commands
\newcommand{\isp}[1]{\quad\text{#1}\quad}
\newcommand{\N}{\mathbb{N}} 
\newcommand{\Z}{\mathbb{Z}}
\newcommand{\Q}{\mathbb{Q}}
\newcommand{\R}{\mathbb{R}}
\newcommand{\C}{\mathbb{C}}
\newcommand{\eps}{\varepsilon}
\renewcommand{\phi}{\varphi}
\renewcommand{\emptyset}{\varnothing}
\newcommand{\<}{\langle}
\renewcommand{\>}{\rangle}
\newcommand{\isom}{\cong}
\newcommand{\eqc}{\overline}
\newcommand{\clo}{\overline}
\newcommand{\teq}{\trianglelefteq}
\DeclareMathOperator{\id}{id}

% Theorem Environments
\theoremstyle{definition}
\newtheorem{proposition}{Proposition}
\newtheorem{lemma}{Lemma}

% Extra Commands
\DeclareMathOperator{\Hom}{Hom}
\DeclareMathOperator{\GL}{GL}
\DeclareMathOperator{\im}{im}


% Document
\begin{document}
\title{MATH 220A Homework 6}
\author{Harry Coleman\makebox[0pt][r]{\raisebox{-0.25in}[0pt][0pt]{(worked with Joseph Sullivan and Gahl Shemy)}}}
\date{November 9, 2021}
\maketitle

\begin{pbox}[Exercise XVIII.2]
    Let $S_4$ be the symmetric group of on $4$ elements.
\end{pbox}

\begin{lemma}
    Given a cycle $(i_1 \, \cdots \, i_m) \in S_n$ and $\sigma \in S_n$,
    \[
        \sigma(i_1 \, \cdots \, i_m)\sigma^{-1} = (\sigma(i_1)) \, \cdots \, \sigma(i_m)).
    \]
\end{lemma}

\begin{proof}
    For an index $i \in \{1, \ldots, n\}$, we have $i = \sigma(j)$ for some other index $j = \sigma^{-1}(i)$. Then
    \[
        (\sigma(i_1 \, \cdots \, i_m)\sigma^{-1})(i)
            = (\sigma(i_1 \, \cdots \, i_m)\sigma^{-1})(\sigma(j))
            = (\sigma(i_1 \, \cdots \, i_m))(j).
    \]
    If $j \ne i_k$ for $k = 1, \dots, m$, then
    \[
        (\sigma(i_1 \, \cdots \, i_m))(j)
            = \sigma(j)
            = i.
    \]
    If $j = i_k$ for some $k \in \{1, \dots, m - 1\}$, then
    \[
        (\sigma(i_1 \, \cdots \, i_m))(j)
            = (\sigma(i_1 \, \cdots \, i_m))(i_k)
            = \sigma(i_{k+1}).
    \]
    If $j = i_m$, then
    \[
        (\sigma(i_1 \, \cdots \, i_m))(j)
            = (\sigma(i_1 \, \cdots \, i_m))(i_m)
            = \sigma(i_1).
    \]
    Hence, we can write
    \[
        \sigma(i_1 \, \cdots \, i_m)\sigma^{-1} = (\sigma(i_1)) \, \cdots \, \sigma(i_m)).
    \]
\end{proof}

\begin{pbox}[1 Exercise XVIII.2(a)]
    Show that there are $5$ conjugacy classes.
\end{pbox}

\begin{proof}
    Every element $\sigma \in S_n$ has a decomposition into disjoint cycles $\sigma = \sigma_1 \cdots \sigma_m$. For any $\tau \in S_n$, we have
    \[
        \tau\sigma\tau^{-1} = (\tau\sigma_1\tau^{-1}) \cdots (\tau\sigma_m\tau^{-1}),
    \]
    where Lemma 1 tells us that $\tau\sigma_k\tau^{-1}$ is a cycle of the same length as $\sigma_k$. Then the cycle type of $\sigma$ is the same as the cycle type of $\tau\sigma\tau^{-1}$.

    Given cycles $(i_1 \, \cdots \, i_m)$ and $(j_1 \, \cdots \, j_m)$ in $S_n$ of the same length, we can choose $\tau \in S_4$ which is the identity on $i \ne i_k$ and $\tau(i_k) = j_k$. Then applying Lemma 1,
    \[
        \tau(i_1 \, \cdots \, i_m)\tau^{-1}
            = (\tau(i_1) \, \cdots \, \tau(i_m))
            = (j_1 \, \cdots \, j_m).
    \]
    In other words, all cycles of the same length are conjugate. Then given elements $\sigma, \tilde{\sigma} \in S_n$ of the same cycle type, we can factor each into products of disjoint cycles
    \[
        \sigma = \sigma_1 \cdots \sigma_m, \quad \tilde{\sigma} = \tilde{\sigma}_1 \cdots \tilde{\sigma}_m,
    \]
    where each $\sigma_k$ and $\tilde{\sigma}_k$ are cycles of the same length. Moreover, as the cycles are disjoint, we can construct $\tau \in S_n$ such that $\tau\sigma_k\tau^{-1} = \tilde{\sigma}_k$ for all $k$, then $\tau\sigma\tau^{-1} = \tilde{\sigma}$.

    We have shown that elements in $S_n$ are conjugate if and only if they have the same cycle type. In particular, this implies that the number of conjugacy classes in $S_n$ is the same as the number of cycle types, i.e., the number of partitions of $n$.

    There are $5$ partitions of $4$:
    \[
        1 + 1 + 1 + 1, \quad 1 + 1 + 2, \quad 2 + 2, \quad 1 + 3, \quad 4.
    \]
    Hence, there are $5$ conjugacy classes.
\end{proof}



\newpage
\begin{pbox}[2 Exercise XVIII.2(b)]
    Show that $A_4$ has a unique subgroup of order $4$, which is not cyclic, and which is normal in $S_4$. Show that the factor (quotient) group is isomorphic to $S_3$, so the representations of Exercise 1 give rise to representations of $S_4$.
\end{pbox}

Note that $A_4$ contains all the $3$-cycles $(i \, j \, k) = (i \, j)(j \, k)$ where $\{i, j, k\} \subseteq \{1, 2, 3, 4\}$. There are $8$ such elements, and each has order $3$. The remaining elements of $A_4$ consist of the identity and the three even elements of order $2$. So $A_4$ has a unique subgroup of order $4$:
\[
    N = \{e, (1 \, 2)(3 \, 4), (1 \, 3)(2 \, 4), (1 \, 4)(2 \, 3)\}.
\]
Since $N$ contains the only elements of $S_4$ with cycle type $2 + 2$, and cycle type is invariant under conjugation, $N$ must be a normal subgroup of $S_4$.

We now consider the quotient $S_4 / N$. First, we see that
\[
    |S_4 / N| = \frac{|S_4|}{|N|} = \frac{24}{4} = 6.
\]
There are two groups of order $6$: the cyclic group $\Z/6\Z$ and the symmetric group $S_3$. Note that $\Z/6\Z$ is generated by some element of order $6$. However, every element of $S_4$ has order at most $4$, so no element of $S_4/N$ has order greater than $4$. Therefore, $S_4/N$ cannot be cyclic, so in fact $S_4/N \isom S_3$.



\newpage
\begin{pbox}[3 Extra Problem]
    Show that the symmetric group $S_4$ has a representation on 
    \[
        V = \{(z_1, z_2, z_3, z_4) \mid z_1 + z_2 + z_3 + z_4 = 0\}
    \]
    which permutes the coordinates. What is the dimension of this representation?
\end{pbox}

\begin{proof}
    For each $\sigma \in S_4$, define a map on $V$ by
    \[
        \rho(\sigma) : (z_1, z_2, z_3, z_4) \to (z_{\sigma(1)}, z_{\sigma(2)}, z_{\sigma(3)}, z_{\sigma(4)}).
    \]
    It is clear that $\rho(\sigma)$ is a map from $V$ to $V$, since the sum of the components of $\rho(\sigma)(z)$ is the same as the sum of the components of $z$ for all $z \in V$. We check that $\rho(\sigma) \in \GL(V)$.

    First, $\rho(\sigma)$ is linear, since for any $a \in \C$ and $z, w \in V$, we have
    \[
        \rho(\sigma)(az + w)
            = a\rho(\sigma)(z) + \rho(\sigma)(w).
    \]
    Moreover, $\rho(\sigma)$ is invertible, since for any $z \in V$, we have
    \[
        \rho(\sigma^{-1})(\rho(\sigma)(z)) = z.
    \]

    Hence, we have a map $\rho : S_4 \to \GL(V)$. We check that $\rho$ is a group homomorphism. For $\sigma, \tau \in S_4$ and $z \in V$, we have
    \[
        \rho(\sigma\tau)(z)
            = (z_{(\sigma\tau)(1)}, \dots, z_{(\sigma\tau)(4)})
            = \rho(\sigma)(z_{\tau(1)}, \dots, z_{\tau(4)})
            = \rho(\sigma)(\rho(\tau)(z)).
    \]
    That is, $\rho(\sigma\tau) = \rho(\sigma) \circ \rho(\tau)$, so $\rho$ is a group homomorphism, hence a representation.


    To see that $\dim V = 3$, consider the map $T : \C^4 \to \C$ defined by $z \mapsto z_1 + z_2 + z_3 + z_4$. Notice that $T$ is linear, surjective, and has kernel equal to $V$. Then we have
    \[
        \dim V = \dim \ker T = \dim \C^4 - \dim \im T = 4 - 1 = 3.
    \]
\end{proof}


\newpage
\begin{pbox}[4 Exercise XVIII.2(e)]
    Let $\rho$ be the representation of [Extra Problem]. Define $\rho'$ by
    \[
        \rho'(\sigma) = \begin{cases}
            \rho(\sigma) & \text{if $\sigma$ is even}, \\
            -\rho(\sigma) & \text{if $\sigma$ is odd}.
        \end{cases}
    \]
    Show that $\rho'$ is also irreducible and is non-isomorphic to $\rho$. This concludes the description of all irreducible representations of $S_4$.
\end{pbox}

\begin{proof}
    We first show that $\rho$ is irreducible.
    
    Suppose $U \leq V$ is a nonzero invariant subspace; we claim that $U = V$.
    Note that it suffices to show $U$ contains the point $(1, 0, 0, -1)$, since this vector and its images under $\rho((1 \, 2))$ and $\rho((1 \, 3))$ span all of $V$.

    Choose any nonzero point $z = (z_1, z_2, z_3, z_4) \in U$, where we may assume $z_1 = 1$.
    
    If it also happens that $z_2 = 1$, then $z = (1, 1, z_3, -2 - z_3)$ and we have the following element in $U$:
    \[
        w = z + \rho((2 \, 3))z = (2, 2, -2, -2).
    \]
    Then $U$ also contains
    \[
        u = \tfrac{1}{4}(w + \rho((2 \, 3))w) = (1, 0, 0, -1),
    \]
    from which we obtain
    \[
        \rho((1 \, 2))u = (0, 1, 0, -1) \isp{and} \rho((1 \, 3))u = (0, 0, 1, -1).
    \]
    These elements span $V$, so we must have $U = V$.
    
    If $z_2 \ne 1$, we consider the following element of $U$:
    \[
        w = z - z_2\rho((1 \, 2))z = (1 - z_2^2, 0, (1 - z_2)z_3, (1 - z_2)z_4).
    \]
    $w = 0$, then $z_2 = -1$ and $z_3 = z_4 = 0$, so $z = (1, -1, 0, 0)$. Then $U$ contains
    \[
        \rho((2 \, 4))z = (1, 0, 0, -1),
    \]
    implying $U = V$. If $w \ne 0$, then either $z_2 = -1$ or $1 - z_2^2 \ne 0$. In the first case, we have
    \[
        \tfrac{1}{2}\rho((1 \, 3))w = (1, 0, 0, -1),
    \]
    and we are done. In the second case, a rescaling of $w$ gives us $u = (1, 0, w_3, -1 - w_3)$. Then $U$ contains
    \[
        v = \rho((1 \, 4))(-u - \rho((3 \, 4))u) = (1, 0, 1, -2)
    \]
    and
    \[
        \tfrac{1}{3}(2v + \rho((3 \, 4))v) = (1, 0, 0, -1).
    \]
    Hence, $U = V$ in all cases and we conclude that $\rho$ is an irreducible representation.

    To show $\rho'$ is also irreducible, we perform the same procedure, but multiplying by $-1$ as necessary to obtain the desired elements.

    Note that $\rho$ and $\rho'$ are isomorphic if and only if there is a change of basis $T \in \GL(V)$ such that $T \circ \rho(\sigma) \circ T^{-1} = \rho'(\sigma)$ for all $\sigma \in S_4$. In particular, this would require that the characters of $\rho$ and $\rho'$ are the same. However
    \[
        \chi_{\rho}((1 \, 2)) = \operatorname{tr} \begin{bmatrix}
            0 & 1 & 0 \\
            1 & 0 & 0 \\
            0 & 0 & 1 \\
        \end{bmatrix}
        = 1,
    \]
    but
    \[
        \chi_{\rho'}((1 \, 2)) = \operatorname{tr} \begin{bmatrix}
            0 & -1 & 0 \\
            -1 & 0 & 0 \\
            0 & 0 & -1 \\
        \end{bmatrix}
        = -1.
    \]
    So $\rho$ and $\rho'$ are not isomorphic representations.
\end{proof}


\end{document}