\documentclass[12pt]{article}

% Packages
\usepackage[margin=1in]{geometry}
\usepackage{fancyhdr, parskip}
\usepackage{amsmath, amsthm, amssymb}

% Page Style
\makeatletter
\fancypagestyle{title}{
    \renewcommand{\headrulewidth}{0.4pt}
    \setlength{\headheight}{15pt}
    \fancyhead[R]{\@author}
    \fancyhead[L]{\@title}
    \fancyhead[C]{\@date}
}
\makeatother
\renewcommand{\maketitle}{\thispagestyle{title}}
\fancypagestyle{plain}{
    \fancyhf{}
    \renewcommand{\headrulewidth}{0pt}
    \renewcommand{\footrulewidth}{0pt}
    \fancyfoot[R]{\thepage}
}
\pagestyle{plain}

% Problem Box
\setlength{\fboxsep}{4pt}
\newlength{\myparskip}
\setlength{\myparskip}{\parskip}
\newsavebox{\savefullbox}
\newenvironment{fullbox}{\begin{lrbox}{\savefullbox}\begin{minipage}{\dimexpr\textwidth-2\fboxsep\relax}\setlength{\parskip}{\myparskip}}{\end{minipage}\end{lrbox}\framebox[\textwidth]{\usebox{\savefullbox}}}
\newenvironment{pbox}[1][]{\begin{fullbox}\ifx#1\empty\else\paragraph{#1}\fi}{\end{fullbox}}

% Default Commands
\newcommand{\isp}[1]{\quad\text{#1}\quad}
\newcommand{\N}{\mathbb{N}} 
\newcommand{\Z}{\mathbb{Z}}
\newcommand{\Q}{\mathbb{Q}}
\newcommand{\R}{\mathbb{R}}
\newcommand{\C}{\mathbb{C}}
\newcommand{\eps}{\varepsilon}
\renewcommand{\phi}{\varphi}
\renewcommand{\emptyset}{\varnothing}
\newcommand{\<}{\langle}
\renewcommand{\>}{\rangle}
\newcommand{\isom}{\cong}
\newcommand{\eqc}{\overline}
\newcommand{\clo}{\overline}
\newcommand{\teq}{\trianglelefteq}
\DeclareMathOperator{\id}{id}

% Extra Commands
\newcommand{\mat}[1]{\begin{bmatrix}#1\end{bmatrix}}
\DeclareMathOperator{\im}{im}

% Document
\begin{document}
\title{MATH 220A Homework 4}
\author{Harry Coleman\makebox[0pt][r]{\raisebox{-0.25in}[0pt][0pt]{(worked with Joseph Sullivan, Gahl Shemy)}}}
\date{October 24, 2021}
\maketitle

\begin{pbox}[1 Exercise I.34 \\ (a)]
    Let $n$ be an even positive integer. Show that there exists a group of order $2n$, generated by two elements $\sigma$, $\tau$ such that $\sigma^n = e = \tau^2$, and $\sigma\tau = \tau\sigma^{n-1}$. This group is called the \textbf{dihedral group}.
\end{pbox}

\begin{proof}
    We have a group presentation
    \[
        D_{2n} = \<\sigma, \tau \mid \sigma^n = \tau^2 = e, \sigma\tau = \tau\sigma^{n-1}\>.
    \]
    The relation $\sigma\tau = \tau\sigma^{n-1}$ tells us that
    \[
        (\sigma^a\tau^b)(\sigma^{a'}\tau^{b'})
            = \sigma^a(\sigma^{(-1)^ba'}\tau^b)\tau^{b'}
            = \sigma^{a + (-1)^ba'}\tau^{b + b'}.
    \]
    So every element of $D_{2n}$ has a representation as $\sigma^a\tau^b$ for some $a, b \in \Z$. Moreover, the relations $\sigma^n = \tau^2 = e$ mean we can assume $0 \leq a < n$ and $0 \leq b < 2$. Distinct choices of $a$ and $b$ give distinct elements of $D_{2n}$, so the order of $D_{2n}$ is $2n$.


\end{proof}

\begin{pbox}[(b)]
    Let $n$ be an odd positive integer, Let $D_{4n}$ be the group generated by the matrices
    \[
        \mat{0 & -1 \\ 1 & 0} \isp{and} \mat{\zeta & 0 \\ 0 & \zeta^{-1}}
    \]
    where $\zeta$ is a primitive $n$-th root of unity. Show that $D_{4n}$ has order $4n$, and give the commutation relations between the above generators.
\end{pbox}

Denote by $\tau$ the first matrix and $\sigma$ the second matrix. A computation shows that $o(\tau) = 4$ and $o(\sigma) = n$, giving us the relations $\sigma^n = \tau^4 = e$. Another computation gives us the commutation relation $\sigma\tau = \tau\sigma^{n-1}$. This means that we have the group presentation
\[
    D_{4n} = \<\sigma, \tau \mid \sigma^n = \tau^4 = e, \sigma\tau = \tau\sigma^{n-1}\>.
\]
Then, each element of $D_{4n}$ has a representation as $\sigma^a\tau^b$ with $0 \leq a < n$ and $0 \leq b < 4$. Distinct choices of $a$ and $b$ give distinct elements of $D_{4n}$, so the order of $D_{4n}$ is $4n$.


\newpage
\begin{pbox}[2 Exercise I.42]
    Viewing $\Z$, $\Q$ as additive groups, show that $\Q/\Z$ is a torsion group, which has one and only one subgroup of order $n$ for each integer $n \geq 1$, and that this subgroup is cyclic.
\end{pbox}

\begin{proof}
    First, note that each element of $\Q/\Z$ has a representative $a/b \in \Q$ such that $a, b \in \Z$ are coprime, $b \geq 1$, and $0 \leq a < b$. Let $a/b \in \Q$; automatically, we may assume that $a, b \in \Z$ are coprime and $b \geq 1$, as all elements of $\Q$ have such a representation. If it is not already the case that $0 \leq a < b$, write $a = kb + a'$ for some $k, a' \in \Z$ with $0 \leq a' < b$. In which case, $\eqc{a/b} = \eqc{a'/b} \in \Q/\Z$, which is the desired representation. 

    If $x \in \Q/\Z$ with representative $a/b \in \Q$, as described, then $bx = \eqc{a} = 0 \in \Q/Z$, so $o(x) \leq b < \infty$. In particular, $\Q/\Z$ is a torsion group. Moreover, $kx = 0$ if and only if $b \mid ka$, implying that $o(x) = \operatorname{lcm}(a, b)/a$. Since $a$ and $b$ are coprime, $\operatorname{lcm}(a, b) = ab$, so $o(x) = b$. 
    
    From this, we deduce that each $x \in \Q/\Z$ has a representative $a/o(x) \in \Q$ with $a, o(x) \in \Z$ coprime and $0 \leq a < o(x)$. It can then be seen that such a representative is unique. If $n = o(x)$ and $0 \leq a, b < n$ such that $\eqc{a/n} = \eqc{b/n} \in \Q/\Z$, then we must have $a = kn + b$ for some $k \in \Z$. But only $k = 0$ is possibly, implying $a = b$.

    For each integer $n \geq 1$, we define a map $\phi_n : \Z/n\Z \to \Q/\Z$, $a \mapsto \eqc{a/n}$. This map is well-defined and injective since $a \equiv b \pmod{n}$ if and only if $a = kn + b$ for some $k \in \Z$, which is equivalent to $\eqc{a/n} = \eqc{b/n} \in \Q/\Z$. Moreover, $\phi_n$ is a group homomorphism since
    \[
        \phi_n(a + b) = \eqc{(a + b)/n} = \eqc{a/n} + \eqc{b/n} = \phi_n(a) + \phi_n(b).
    \]
    Hence, $\phi_n$ is a monomorphism with $\Z/n\Z \isom \im\phi_n \leq \Q/\Z$. We claim that $H_n = \im\phi_n$ is the unique subgroup of $\Q/\Z$ of order $n$.

    Now, if $x \in \Q/\Z$ with order $n$, we know $x$ has a unique representative $a/n \in \Q$ with $0 \leq a < n$. In which case, $x = \phi_n(a) \in H_n$. This means that $H_n$ contains every element of order $n$ in $\Q/\Z$, so $H_n$ is the unique cyclic subgroup of order $n$.
    
    Let $H \leq \Q/\Z$ be a finite subgroup; we will prove that $H = H_n$ for some $n$. If $x \in H$ with order $m$ and unique representative $a/m \in \Q$, there are some $k, h \in \Z$ such that $ka + hm = 1$, then $\eqc{1/m} = kx \in H$. In other words, $\eqc{1/m} \in H$ if and only if $H$ contains an order $m$ element.
    
    Let $n$ be the maximum positive integer such that $\eqc{1/n} \in H$ (which exists since $H$ is finite). Assume, for contradiction, that $\eqc{1/m} \in H$ but $m$ does not divide $n$. Define $g = \gcd(n, m) < m \leq n$, then $n = ag$ and $m = b$ for some $a, b > 1$. Moreover, are some $c, d \in \Z$ such that $cn + dm = g$, so $H$ contains the element
    \[
        \eqc{a/n} + \eqc{b/m} = \eqc{g/nm} = \eqc{1/nb}.
    \]
    But $nb > n$, which contradicts the definition of $n$. Hence, the order of every element of $H$ divides $n$. Therefore, if $x \in H$ has representative $a/o(x) \in \Q$, we know that $n = bo(x)$ for some $b \in \Z$, then $x = \eqc{ab/n} \in H_n$. This shows that $H \leq H_n$. Since we also have $\eqc{1/n} \in H$, so $H_n = \<\eqc{1/n}\> \leq H$, we conclude that $H = H_n$.
    

\end{proof}


\newpage
\begin{pbox}[3 Exercise I.43]
    Let $H$ be a subgroup of finite abelian group $G$. Show that $G$ has a subgroup that is isomorphic to $G/H$.
\end{pbox}

\begin{proof}
    By Theorem 8.1 in Lang, $G \isom \bigoplus_{p} G(p)$, where $p$ ranges over all primes. Since $H$ is itself a finite abelian group, we also have $H \isom \bigoplus_p H(p)$. By definition, $H(p) = G(p) \cap H$, which means, in particular, that $H(p) \leq G(p)$. Applying a fact about modules to the special case of abelian groups ($\Z$-modules), we obtain $G/H \isom \bigoplus_p G(p)/H(p)$. If there is a subgroup $K_p \leq G(p)$ isomorphic to $G(p)/H(p)$, for each prime $p$, we can construct a subgroup $\bigoplus_p K_p \leq G$ isomorphic to $G/H$. Therefore, it suffices to prove the result when $G$ is a $p$-group.

    Assume $H$ is a subgroup of a finite abelian $p$-group $G$. Then we can write both $G$ and $H$ as the direct sum of cyclic $p$-groups, and I think there is a way to get the components to line up such that the problem again reduces to the case of when $G$ is a cyclic $p$-group. I could not figure out how to show this.

\end{proof}



\newpage
\begin{pbox}[4 Exercise I.44]
    Let $f : A \to A'$ be a homomorphism of abelian groups. Let $B$ be a subgroup of $A$. Denote by $A^f$ and $A_f$ the image and kernel of $f$ in $A$ respectively, and similarly for $B^f$ and $B_f$. Show that $[A : B] = [A^f : B^f][A_f : B_f]$, in the sense that if two of these three indices are finite, so is the third, and the stated equality holds.
\end{pbox}

(For clarity, we use $G/H$ and $\frac{G}{H}$, interchangeably, to denote the quotient of $G$ by $H$.)

\begin{proof}
    First note that
    \[
        B/B_f
            = B/(B \cap A_f)
            \isom BA_f/A_f
    \]
    and
    \[
        A_f/B_f
            = A_f/(A_f \cap B)
            \isom BA_f/B.
    \]
    Then
    \[
        \frac{A^f}{B^f}
            \isom \frac{A/A_f}{B/B_f}
            \isom \frac{A/A_f}{BA_f/A_f}
            \isom \frac{A}{BA_f}
            \isom \frac{A/B}{BA_f/B}
            \isom \frac{A/B}{A_f/B_f},
    \]
    so
    \[
        [A : B]
            = [A/B : 1]
            = [A/B : A_f/B_f][A_f/B_f : 1]
            = [A^f : B^f][A_f : B_f],
    \]
    where the second equality holds in the desired sense.


\end{proof}


\end{document}