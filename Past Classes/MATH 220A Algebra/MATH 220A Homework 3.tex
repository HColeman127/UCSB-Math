\documentclass[12pt]{article}

% Packages
\usepackage[margin=1in]{geometry}
\usepackage{fancyhdr, parskip}
\usepackage{amsmath, amsthm, amssymb}

% Page Style
\fancypagestyle{plain}{
    \fancyhf{}
    \renewcommand{\headrulewidth}{0pt}
    \renewcommand{\footrulewidth}{0pt}
    \fancyfoot[R]{\thepage}
}
\pagestyle{plain}

% Problem Box
\setlength{\fboxsep}{4pt}
\newlength{\myparskip}
\setlength{\myparskip}{\parskip}
\newsavebox{\savefullbox}
\newenvironment{fullbox}{\begin{lrbox}{\savefullbox}\begin{minipage}{\dimexpr\textwidth-2\fboxsep\relax}\setlength{\parskip}{\myparskip}}{\end{minipage}\end{lrbox}\framebox[\textwidth]{\usebox{\savefullbox}}}
\newenvironment{pbox}[1][]{\begin{fullbox}\ifx#1\empty\else\paragraph{#1}\fi}{\end{fullbox}}

% Default Commands
\newcommand{\isp}[1]{\quad\text{#1}\quad}
\newcommand{\N}{\mathbb{N}} 
\newcommand{\Z}{\mathbb{Z}}
\newcommand{\Q}{\mathbb{Q}}
\newcommand{\R}{\mathbb{R}}
\newcommand{\C}{\mathbb{C}}
\newcommand{\eps}{\varepsilon}
\renewcommand{\phi}{\varphi}
\renewcommand{\emptyset}{\varnothing}
\newcommand{\<}{\langle}
\renewcommand{\>}{\rangle}
\newcommand{\isom}{\cong}
\newcommand{\eqc}{\overline}
\newcommand{\clo}{\overline}
\newcommand{\teq}{\trianglelefteq}
\DeclareMathOperator{\id}{id}

% Extra Commands


% Document Info
\fancypagestyle{title}{
    \renewcommand{\headrulewidth}{0.4pt}
    \setlength{\headheight}{15pt}
    \fancyhead[R]{Harry Coleman\makebox[0pt][r]{\raisebox{-0.25in}[0pt][0pt]{(worked with Joseph Sullivan, Gahl Shemy)}}}
    \fancyhead[L]{MATH 220A Homework 3}
    \fancyhead[C]{October 16, 2021}
}

% Begin Document
\begin{document}
\thispagestyle{title}


\begin{pbox}[1 Exercise I.19]
    Let $G$ be a finite group operating on a finite set $S$.
\end{pbox}

\begin{pbox}[(a)]
    For each $s \in S$ show that
    \[
        \sum_{t \in G \cdot s} \frac{1}{|G \cdot t|} = 1.
    \]
\end{pbox}

\begin{proof}
    Note that for all $t \in G \cdot s$, we have $G \cdot t = G \cdot s$, since the orbits of $G$ in $S$ form a partition of $S$. So
    \[
        \sum_{t \in G \cdot s} \frac{1}{|G \cdot t|}
            = \sum_{t \in G \cdot s} \frac{1}{|G \cdot s|}
            = |G \cdot s| \frac{1}{|G \cdot s|}
            = 1.
    \]

\end{proof}

\begin{pbox}[(b)]
    For each $x \in G$ define $f(x) = $ number of elements $s \in S$ such that $xs = s$. Prove that the number of orbits of $G$ in $S$ is equal to
    \[
        \frac{1}{|G|}\sum_{x \in G} f(x).
    \]
\end{pbox}

\begin{proof}
    By the orbit-stabilizer theorem,
    \[
        |G \cdot s|
            = [G : G_s]
            = \frac{|G|}{|G_s|}.
    \]
    Let $S/G$ be the set of orbits of $G$ in $S$. Then, with (a), the number of orbits of $G$ in $S$ is
    \[
        |S/G|
            = \sum_{O \in S/G} 1
            = \sum_{O \in S/G} \sum_{s \in O} \frac{1}{|G \cdot s|}
            = \sum_{s \in S} \frac{1}{|G \cdot s|}
            = \frac{1}{|G|} \sum_{s \in S} |G_s|.
    \]
    Let $C = \{(x, s) \in G \times S \mid xs = s\}$. Then, by Homework 2 Problem 4,
    \[
        \sum_{x \in G} f(x) = |C| = \sum_{s \in S} |G_s|,
    \]
    so indeed
    \[
        |S/G| = \frac{1}{|G|} \sum_{x \in G} f(x).
    \]


\end{proof}


\newpage
\begin{pbox}[2 Exercise I.21]
    Let $G$ be a finite group and $H$ a subgroup. Let $P_H$ be a $p$-Sylow subgroup of $H$. Prove that there exists a $p$-Sylow subgroup $P$ of $G$ such that $P_H = P \cap H$.
\end{pbox}

\begin{proof}
    We have $|P_H| = p^k$, for some $k \in \Z_{\geq 0}$; in particular, $P_H$ is a $p$-subgroup of $G$. So there is a $p$-Sylow subgroup $P \leq G$ containing $P_H$. We claim that $P_H = P \cap H$.

    Since $P_H \leq P$ and $P_H \leq H$, it is evident that $P_H \leq P \cap H$. Moreover, this implies that $|P_H| = p^k$ divides $|P \cap H|$. And since $P \cap H \leq P$, we also have $|P \cap H|$ dividing $|P|$, so $|P \cap H|$ is a power of $p$. Since $P_H$ is a $p$-Sylow subgroup of $H$, then $p^k$ is the maximum power of $p$ that divides the order of $H$. So in fact, $|P \cap H| = p^k = |P_H|$. Therefore, $P \cap H$ contains $P$ and has the same order, so they must be equal.

\end{proof}



\newpage
\begin{pbox}[3 Exercise I.22]
    Let $H$ be a normal subgroup of a finite group $G$ and assume that $|H| = p$. Prove that $H$ is contained in every $p$-Sylow subgroup of $G$.
\end{pbox}

\begin{proof}
    Since $H$ is a $p$-subgroup of $G$, it is contained in some $p$-Sylow subgroup $P \leq G$. For any other $p$-Sylow subgroup $P' \leq G$, we have $P' = gPg^{-1}$ for some $g \in G$. Then, since $H$ is normal in $G$, we have
    \[
        H = gHg^{-1} \subseteq gPg^{-1} = P'.
    \]

\end{proof}


\newpage
\begin{pbox}[4 Exercise I.23]
    Let $P$, $P'$ be $p$-Sylow subgroups of a finite group $G$.
\end{pbox}

\begin{pbox}[(a)]
    If $P' \subseteq N(P)$ (normalizer of $P$), then $P' = P$.
\end{pbox}

\begin{proof}
    Consider $PP'$, which the set of all elements $xy$ for $x \in P$ and $y \in P'$. We claim that it is subgroup of $G$. Let $x_1y_1, x_2y_2 \in PP'$; we want to show that the product is still in $G$. Since $y_1 \in P' \subseteq N(P)$, then $y_1x_2y_1^{-1} \in P$, so
    \[
        (x_1y_1)(x_2y_2)
            = x_1y_1x_2(y_1^{-1}y_1)y_2
            = (x_1y_1x_2y_1^{-1})(y_1y_2)
            \in PP'.
    \]
    Additionally, for $xy \in PP'$, we require $(xy)^{-1}$ in $PP'$. Since $y^{-1} \in P' \subseteq N(P)$, we know that $y^{-1}P = Py^{-1}$. Hence,
    \[
        (xy)^{-1}
            = y^{-1}x^{-1}
            \in y^{-1}P
            = Py^{-1}
            \subseteq PP'.
    \]
    We conclude that $PP'$ is a subgroup of $G$ containing $P$ and $P'$.
    
    By the diamond isomorphism theorem (listed in Lang as one of the canonical isomorphisms), $P' \leq N(P)$ implies $PP'/P' \isom P/(P \cap P')$, so
    \[
        |PP'| = \frac{|P| |P'|}{|P \cap P'|}.
    \]
    We deduce that $|PP'|$ is a power of $p$, as it divides $|P||P'|$, which is a power of $p$. Since $PP'$ contains $P$ as a subgroup, then $|PP'|$ is at least $|P|$, the maximum power of $p$ dividing $|G|$. So in fact, $P = PP' = P'$.

\end{proof}

\begin{pbox}[(b)]
    If $N(P') = N(P)$, then $P' = P$.
\end{pbox}

\begin{proof}
    Since $P' \subseteq N(P') = N(P)$, then by (a), we have $P' = P$.

\end{proof}

\begin{pbox}[(c)]
    We have $N(N(P)) = N(P)$.
\end{pbox}

\begin{proof}
    As always, $N(P) \subseteq N(N(P))$, so we prove the opposite inclusion. Let $x \in N(N(P))$, meaning $xN(P)x^{-1} = N(P)$. Since $P \subseteq N(P)$,
    \[
        xPx^{-1}
            \subseteq xN(P)x^{-1}
            = N(P).
    \]
    Since $xPx^{-1}$ is a $p$-Sylow subgroup of $G$, (a) implies $xPx^{-1} = P$. Hence, $x \in N(P)$.

\end{proof}

\end{document}