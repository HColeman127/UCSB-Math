\documentclass[12pt]{article}

% packages
\usepackage{kantlipsum}
\usepackage[margin=1in]{geometry}
\usepackage[labelfont=it]{caption}
\usepackage[table]{xcolor}
\usepackage{subcaption,framed,colortbl,multirow}
\usepackage{amsmath,amsthm,amssymb,wasysym,mathrsfs,mathtools}
\usepackage{tikz,graphicx,pgf,pgfplots}
\usetikzlibrary{arrows, angles, quotes, decorations.pathreplacing, math, patterns, calc}
\pgfplotsset{compat=1.16}

% Set Names
\newcommand{\N}{\mathbb{N}}
\newcommand{\Z}{\mathbb{Z}}
\newcommand{\I}{\mathbb{I}}
\newcommand{\R}{\mathbb{R}}
\newcommand{\Q}{\mathbb{Q}}
\newcommand{\C}{\mathbb{C}}

% Misc Characters
\newcommand{\F}{\mathbb{F}}
\newcommand{\powerset}{\raisebox{.15\baselineskip}{\Large\ensuremath{\wp}}}
\newcommand{\eps}{\varepsilon}

% Paired Delimiters
\DeclarePairedDelimiter{\ceil}{\lceil}{\rceil}
\DeclarePairedDelimiter\floor{\lfloor}{\rfloor}

% Homework Sections
\setlength{\fboxsep}{4pt}
\newcommand{\generic}[2]{\section*{#1}\begin{center}\framebox{\begin{minipage}{\textwidth-10pt}#2\end{minipage}}\end{center}}
\newcommand{\ex}[2]{\generic{Exercise #1}{#2}}
\newcommand{\prob}[2]{\generic{Problem #1}{#2}}
\newcommand{\ques}[2]{\generic{Question #1}{#2}}

% Environments
\newenvironment{drawing}{\begin{center}\begin{tikzpicture}}{\end{tikzpicture}\end{center}}

% MATH CS 117 Intro to Real Analysis
\newcommand{\ds}{\displaystyle}
\newcommand{\seq}[1]{\left\{#1\right\}_{n=1}^\infty}
\newcommand{\isp}[1]{\quad\text{#1}\quad}
 
\begin{document}
 
\title{Homework 1\\
    \large MATH CS 120 Convex Optimization}
\author{Harry Coleman}
\date{April 21, 2020}
\maketitle

\ex{2.3}{
    A set $C$ is midpoint convex if whenever two points $a,b$ are in $C$, the average or midpoint $(a+b)/2$ is in $C$. Obviously a convex set is midpoint convex. It can be proved that under mild conditions midpoint convexity implies convexity. As a simple case, prove that if $C$ is closed and midpoint convex, then $C$ is convex.
}

Let $C\in\R^n$ be a midpoint convex, closed set and let $a,b\in\C$. Let $c=\theta a+(1-\theta)b$ for some $\theta\in[0,1]$, that is, $c$ is a convex combination of $a$ and $b$. We aim to prove that $c\in C$, thereby proving $C$ is convex. To do so, we construct a sequence of line segments betwen $a_n$ and $b_n$ for all $n\in\N$ in the following way:
\[a_0 = a, \quad b_0=b,\]
and given $a_{n-1},b_n-1$, define $m_n = \ds\frac12(a_{n-1}+b_{n-1})$ and if $c$ is on the line segment from $a_n$ to $m_n$, define
\[a_n=a_{n-1}, \quad b_n = m_n.\]
Otherwise, if $c$ is on the line segment from $m_n$ to $b_n$, define
\[a_n=m_n, \quad b_n=b_{n-1}.\]
Since $c$ is on the line segment between $a_n$ and $b_n$ for all $n\in\N$, we have that
\[|a_n-c| \leq |a_n-b_n| = \frac1{2^n}|a-b|.\]
So the sequence $\seq{a_n}$ converges to $c$. And since each $a_n$ is define by a midpoint of elements of $C$, each $a_n\in\C$. Thus $c$ is a limit point of $C$, and since $C$ is closed, $c\in\C$. Therefore, $C$ is convex.


\newpage
\ex{2.8}{
    Which of the following sets $S$ are polyhedra? If possible, express $S$ in the form $S=\{x:Ax\preceq b, Fc=g\}$.
}

(a) $S=\{y_1a_1+y_2a_2 : -1\leq y_1\leq 1, -1\leq y_2\leq1\}$, where $a_1,a_2\in\R^n$.

Any point $x\in S$ corresponds to a convex combination
\[x=\theta_1(a_1+a_2) + \theta_2(a_1-a_2) + \theta_3(a_2-a_1) + \theta_4(-a_1-a_2),\]
where
\begin{align*}
    y_1 &= \theta_1 + \theta_2 - \theta_3 + \theta_4, \\
    y_2 &= \theta_1 + \theta_3 - \theta_2 + \theta_4.
\end{align*}
So $S$ can be equivalently defined as the simplex given by the convex hull of the points
\[(a_1+a_2), (a_1-a_2), (a_2-a_1), (-a_1-a_2).\]
If $a_1,a_2$ are affinely dependent, then this is simply a line segment. Either way, this is a polyhedron.

(b) $S=\{x\in\R^n : x\succeq0, 1^Tx=1, \sum_{i=1}^nx_ia_i=b_1, \sum_{i=1}^nx_ia_i^2=b_2\}$ where $a_1,\dots,a_n\in\R$ and $b_1,b_2\in\R$.

To see that $S$ is a polyhedron, the conditions for $S$ can simply be rewritten. If we define
\[A = 
    \begin{bmatrix}
        1 & \cdots & 1 \\
        a_1 & \cdots & a_n \\
        a_1^2 & \cdots & a_n^2
    \end{bmatrix},
\qquad b = 
    \begin{bmatrix}
        1 \\
        b_1 \\
        b_2
    \end{bmatrix},
\]
then we can equivalently write $S$ as
\[S = \{x\in\R^n: (-I_n)x \preceq 0, Ax = b\},\]
thus $S$ is a polyhedron.

(c) $S=\{x\in\R^n : x\succeq0, x^Ty\leq 1 \text{ for all } y \text{ with } |y|=1\}$.

This set is equivalent to the nonnegative vectors of the closed unit ball centered at the origin in $R^n$. This can be seen since any negative $x$ would not satisfy $x\succeq0$ and any vector with a length greater than 1 would not satisfy $x^Ty\leq 1$ for the unit vector $y$ in the direction of $x$. However, any nonnegative unit vector satisfies both of these. Because of this, $S$ cannot be expressed as the intersection of finitely many halfspaces and hyperplanes, and thus is not a polyhedron.

(d) $S=\{x\in\R^n : x\succeq 0, x^Ty\leq 1 \text{ for all } y \text{ with } \sum_{i=1}^n|y_i|=1\}$.

This set is equivalent to the set of vectors in $\R^n$ with each element between 0 and 1. Similar to (c), each $x$ must be nonnegative, but in this case each element of $x$ must be less than or equal to 1. For any such vector, the dot product $x^Ty$
is less than or equal to the sum of all $|y_i|$, which is equal to 1. However, for any $x$ with an element greater than 1, there is some $y$ for which the dot product is greater than 1. Therefore, we can equivalently define
\[S=\{x\in\R^n : \left[{-I_n \atop I_n}\right]x \preceq \left[{0 \atop 1}\right]\}.\]


\ex{2.10}{
    Let $C\in\R^n$ be the solution set of a quadratic inequality,
    \[C = \{x\in\R^n : x^TAx + b^Tx + c\leq 0\},\]
    with $A\in S^n,b\in\R^n,$ and $c\in\R$.
    (note: $S^n$ is the set of $n\times n$ symmetric matrices.)
}

(a) Show that $C$ is convex if $A\succeq0$.
Let $x,y\in C$ and $\theta\in[0,1]$. We aim to prove that the convex combination
\[x + \theta(y-x)\]
is in $C$. To show this, we substitute the point into the inequality for $C$ to find
\begin{align*}
    & (x+\theta(y-x))^TA(x+\theta(y-x)) + b^T(x+\theta(y-x)) + c \\
    &= (y-x)^TA(y-x)\theta^2 + (2(y-x)^TAx + b^T(y-x))\theta + x^TAx+b^Tx+c.
\end{align*}
Notice that this is is a quadratic function on $\theta$, which is less than or equal to zero for values of $\theta=0$ and $\theta=1$, since these values correspond to the inequality for $C$ on $x$ and $y$, respectively. Therefore, this expression is less than or equal to zero if and only if the $a_2$ term is positive. Since
\[a_2 = (y-x)^TA(y-x)\]
and $A$ is positive semidefinite, we in fact have $a_2\geq0$. Thus, the expression is less than or equal to 0, giving us $x+\theta(y-x)\in C$. The converse, however, is not true, since we could pick $A=-I_n, b=0, c=0$, and then the condition for $C$ becomes
\[-x^Tx \leq 0,\]
which is true for all $x\in\R^n$, making $C=R^n$, which is convex.

(b) Show that the intersecton of $C$ and the hyperplane defined by $g^Tx+h=0$ (where $g\ne0$) is convex if $A+\lambda gg^T \succeq 0$ for some $\lambda\in\R$.

Similar to (a), we let $x,y\in C\cap\{x\in\R^n : g^T+h=0\}$, $\theta\in\R$ and we aim to prove
\[x+\theta(y-x) \in C\cap\{x\in\R^n : g^Tx+h=0\}.\]
Since the hyperplane is convex, we already know that
\[x+\theta(y-x) \in \{x\in\R^n : g^Tx+h=0\}.\]
So we just need to show that the point is in $C$. And similar to (a), this is the case if
\[(y-x)^TA(y-x) \geq 0.\]
To show this is satisfied, we assume $A + \lambda gg^T \succeq 0$ for some $\lambda\in\R$. So
\begin{align*}
    0 &\leq (y-x)^T(A+\lambda gg^T)(y-x) \\
        &= (y-x)^TA(y-x) + \lambda(g^Ty - g^Tx)^2 \\
        &= (y-x)^TA(y-x) + \lambda(-h+h)^2 \\
        &= (y-x)^TA(y-x).
\end{align*}
Thus, $x+\theta(y-x) \in C\cap\{x\in\R^n : g^Tx+h=0\}$, so the intersection is convex.


\ex{2.20}{
    Suppose $A\in\R^{m\times n}, b\in\R^m$, with $b\in R(A)$. Show that there exists an $x$ satisfying
    \[x\succ 0, \quad Ax=b\]
    if and only if there exists no $\lambda$ with
    \[A^T\lambda \succeq 0, \quad A^T\lambda \ne 0, \quad b^T\lambda \leq 0.\]
    Hint. First prove the following fact from linear algebra: $c^Tx = d$ for all $x$ satisfying $Ax=b$ if and only if there is a vector $\lambda$ such that $c=A^T\lambda, d=b^T\lambda$
}

We first prove that $c^Tx = d$ for all $x$ satisfying $Ax=b$ if and only if there is a vector $\lambda$ such that $c=A^T\lambda, d=b^T\lambda$. Suppose there is a vector $\lambda$ such that $c=A^T\lambda, d=b^T\lambda$. Then for any $x$ such that $Ax=b$, we find
\[c^Tx = A^T\lambda x = \lambda^TAx = \lambda^Tb = b^T\lambda = d.\]
Now suppose that $c^Tx=d$ for all $x$ such that $Ax=b$. Let $v$ be such that $Av=b$, then for any $x$ such that $Ax=b$, there is some $u\in\ker A$ such that $x=u+v$. Now since $c^Tv=d$ and $c^Tx=d$, we find
\[d = c^t(u+v) = c^Tu + c^Tv,\]
which implies that $c^Tu=0$ for all $u\in\ker A$. Now since $c$ is orthogonal to all the vectors in the kernel of $A$, it must be in the orthogonal complement to $\ker A$ which is given by the rowspace of $A$. So $c$ is a linear combination of the rows of $A$, that is, $c = A^T\lambda$ for some $\lambda$. From this, we find
\[d = c^Tx = (A^T\lambda)^Tx = \lambda^TAx = \lambda^Tb = b^T\lambda.\]
Thus $c^Tx = d$ for all $x$ satisfying $Ax=b$ if and only if there is a vector $\lambda$ such that $c=A^T\lambda, d=b^T\lambda$.

Suppose there exists $x\succ0$ such that $Ax=b$. We now assume for the purpose of contradiction that there is some $\lambda$ such that
\[A^T\lambda \succeq 0, \quad A^T\lambda \ne 0, \quad b^T\lambda \leq 0.\]
Then we have
\[A^T\lambda x = b^T\lambda.\]
However, since $x$ is strictly positive,  $A^T\lambda$ is nonnegative, and $b^T\lambda$ is strictly positive. Therefore, no such $\lambda$ exists.

Lastly suppose that there does not exist an $x\succ0$ such that $Ax=b$. In other words, the set $G=\{x\in\R^n : Ax=b\}$ is disjoint with the set $H=\{x\in\R^n : x\succ0\}$. Since both of these sets are convex, there is a hyperplane defined by $c\in\R^n,k\in\R$ such that 
\[ \begin{cases}
    c^Tg \leq k, & \forall g\in G;  \\
    c^Th \geq k, & \forall h\in H.  \\
\end{cases}\]
From this, we see that $c$ must be nonnegative, since otherwise we could find $c^Th<k$ for large enough $h\in H$. And since $c^Th$ can be made arbitrarily close to zero for small $h$, we must have $k\leq 0$. And since $G$ is affine, $c^Tg$ must be the same for all $g\in G$, since otherwise we could picking two points with different values find a third along the line with an arbitrarily large value for $c^Tg$. Call this value $d$, so $c^Tx=d$ for all $x\in G$. Then the desired $\lambda$ is given by $c=A^T\lambda$ and $d=b^T\lambda$.



\ex{2.21}{
    Suppose that $C$ and $D$ are disjoint subsets of $\R^n$. Consider the set of $(a,b)\in\R^{n+1}$ for which $a^Tx\leq b$ for all $x\in C$, and $a^Tx\geq b$ for all $x\in D$. Show that this set is a convex cone (which is the singleton $\{0\}$ if there is no hyperplane that separates $C$ and $D$).
}

Let $(a,b),(c,d) \in \R^{n+1}$ define separating hyperplanes for $C$ and $D$. We consider for some $\theta_1,\theta_2\geq0$ the conic combination
\[(e,f) = (\theta_1 a + \theta_2c, \,\theta_1 b + \theta_2d).\]
Then for any $x\in C$, we have
\[e^Tx = (\theta_1 a + \theta_2b)^Tx = \theta_1 a^Tx + \theta_2c^Tx \leq \theta_1 b + \theta_2d = f.\]
And for any $x\in D$, we have
\[e^Tx = (\theta_1 a + \theta_2b)^Tx = \theta_1 a^Tx + \theta_2c^Tx \geq \theta_1 b + \theta_2d = f.\]
So in fact, $(e,f)$ defines a separating hyperplane of $C$ and $D$. Thus the set of separating hyperplanes is a convex cone.

\ex{2.25}{
    Let $C\in\R^n$ be a closed convex set, and suppose that $x_1,\dots,x_K$ are on the boundary of $C$. Suppose that for each $i$, $a_i^T(x-x_i)=0$ defines a supporting hyperplane for $C$ at $X_i$, i.e., $C\subseteq\{x : a_i^T(x-x_i)\leq0\}$. Consider the two polyhedra
    \[P_{\text{inner}} = \textbf{conv}\{x_1,\dots,x_K\}, \quad P_{\text{outer}}(x : a_i^T(x-x_i)\leq0, i=1,\dots,K\}.\]
    Show that $P_{\text{inner}} \subseteq C \subseteq P_{\text{outer}}$. Draw a picture illustrating this.
}

Let $x\in P_{\text{inner}}$. Since $C$ is closed, the points $x_1,\dots,x_K\in C$. And since $x$ is a convex combination of points in $C$, which is a convex set, we have $x\in C$. Thus,  $P_{\text{inner}}\subseteq C$.

Now let $x\in C$. Then for each of the defined supporting hyperplanes of $C$, we have $a_i^T(x-x_i)\leq0$, which is precisely the condition for $P_{\text{outer}}$, so $x\in P_{\text{outer}}$. Thus $C\subseteq P_{\text{outer}}$.


\end{document}