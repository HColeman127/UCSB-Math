\documentclass[12pt]{article}

% Packages
\usepackage[margin=1in]{geometry}
\usepackage{amsmath, amsthm, amssymb, physics, tikz}

% Problem Box
\setlength{\fboxsep}{4pt}
\newsavebox{\mybox}
\newenvironment{problem}
    {\begin{lrbox}{\mybox}\begin{minipage}{0.98\textwidth}}
    {\end{minipage}\end{lrbox}\begin{center}\framebox[\textwidth]{\usebox{\mybox}}\end{center}}

\newenvironment{drawing}{\begin{center}\begin{tikzpicture}}{\end{tikzpicture}\end{center}}

% Options
\renewcommand{\thesubsection}{\thesection(\alph{subsection})}
\allowdisplaybreaks
\addtolength{\jot}{1em}
\theoremstyle{definition}

% Default Commands
\newtheorem{proposition}{Proposition}
\newtheorem{lemma}{Lemma}
\newcommand{\ds}{\displaystyle}
\newcommand{\isp}[1]{\quad\text{#1}\quad}
\newcommand{\N}{\mathbb{N}}
\newcommand{\Z}{\mathbb{Z}}
\newcommand{\Q}{\mathbb{Q}}
\newcommand{\R}{\mathbb{R}}
\newcommand{\C}{\mathbb{C}}
\newcommand{\eps}{\varepsilon}
\renewcommand{\phi}{\varphi}
\renewcommand{\emptyset}{\varnothing}

% Extra Commands
\newcommand{\pfrac}[2]{\left(\frac{#1}{#2}\right)}



% Document Info
\title{Test Chapter 5\\
    \large MATH 118B
}
\author{Harry Coleman}
\date{January 31, 2021}

% Begin Document
\begin{document}
\maketitle

\section{}
\begin{problem}
    (5 points) Let $f:\R \to \R$ be differentiable at $a\in\R$. Evaluate the following limit:
    \begin{equation}
    \lim_{x\to a} \frac{a^nf(x)-x^nf(a)}{x-a},\quad n\in \N.
    \end{equation}
\end{problem}

For any $x \in \R$ with $x \ne a$, we can rewrite the expression inside the limit as
\begin{align*}
    \frac{a^nf(x)-x^nf(a)}{x-a}
        &= \frac{a^nf(x) - x^nf(x) + x^nf(x) - x^nf(a)}{x-a} \\
        &= \frac{a^nf(x) - x^nf(x)}{x-a} + \frac{x^nf(x) - x^nf(a)}{x-a} \\
        &= -f(x) \pfrac{x^n - a^n}{x-a} + x^n \pfrac{f(x) - f(a)}{x-a}.
\end{align*}
We can now see that this contains the expressions for the derivatives of $x^n$ and $f(x)$ at $a$. That is, we have
\[
    \lim_{x\to a} \frac{x^n - a^n}{x-a} = \left(\dv{x}x^n\right)(a) = na^{n-1}
\]
and
\[
    \lim_{x\to a} \pfrac{f(x) - f(a)}{x-a} = f'(a).
\]
And since both of these function are differentiable at $a$, then, in particular, they are continuous at $a$. Thus,
\[
    \lim_{x\to a} \left(-f(x) \pfrac{x^n - a^n}{x-a} + x^n \pfrac{f(x) - f(a)}{x-a}\right)
        = -f(a)na^{n-1} + a^nf'(a).
\]

\newpage
\section{}
\begin{problem}
    (10 points) Consider a function $f:[a,b] \to \mathbb{R}$, and assume it is differentiable in $(a,b)$. Consider the set $Z(f) = \{ x \in [a,b]:\ f(x) = 0\}$. Assume that there exists a limit point of $c\in (a,b)$ of  $Z(f)$. Prove that $f'(c) = 0$. 
\end{problem}

\begin{proof}
    Let $\{z_n\}_{n\in\N}$ be a sequence in $Z(f)$ converging to $c$. Since $f$ is differentiable at $c$, then, in particular, it is continuous at $c$, so
    \[
        f(c) = \lim_{x \to c} f(x).
    \]
    Since this limit exists, then it holds for any sequence converging to $c$, so
    \[
        f(c) = \lim_{n \to \infty} f(z_n) = \lim_{n \to \infty} 0 = 0.
    \]
    Since $f$ is differentiable at $c$, then
    \[
        f'(c) = \lim_{x \to c} \frac{f(x) - f(c)}{x - c} = \lim_{x \to c} \frac{f(x)}{x - c}.
    \]
    Again, this holds for any sequence converging to $c$, so
    \[
        f'(c) = \lim_{n \to \infty} \frac{f(z_n)}{z_n - c} = \lim_{n \to \infty} 0 = 0.
    \]
    
\end{proof}


\newpage
\section{}

\subsection{}
\begin{problem}
     (5 points) Consider $f(x) = \tan x$, for $x \in \left ( -\frac{\pi}{2}, \frac{\pi}{2} \right )$. Prove that $f$ is strictly increasing.
\end{problem}

\begin{proof}
    Using the quotient rule for derivatives, we find
    \begin{align*}
        f'(x)
            &= \dv{x} \frac{\sin x}{\cos x}\\
            &= \frac{\cos x \dv{x} \sin x - \sin x \dv{x} \cos x}{\cos^2 x} \\
            &= \frac{\cos^2 x + \sin^2 x}{\cos^2 x} \\
            &= \frac{1}{\cos^2 x} \\
            &> 1.
    \end{align*}
    Since $f'(x) > 0$ for all $x \in \left ( -\frac{\pi}{2}, \frac{\pi}{2} \right )$, then $f$ is strictly increasing.
    
\end{proof}

\subsection{}
\begin{problem}
     (5 points) Consider its inverse, $g(x) = \arctan x$, defined as a function $g:\R \to \left ( -\frac{\pi}{2},\frac{\pi}{2}\right )$. Prove that 
    \begin{equation}
    g'(x) = \frac{1}{1+x^2},\quad x\in \R.
    \end{equation}
\end{problem}

\begin{proof}
    Since $f$ and $g$ are differentiable inverses and $f'(x) = 1/\cos^2 x$, then
    \[
        g'(x) 
            = \frac{1}{f'(g(x))}
            = \frac{1}{\frac{1}{\cos^2g(x)}}
            = \cos^2 g(x).
            = \cos^2 \arctan x.
    \]
    The fact that $\theta = \arctan x$ can be geometrically interpreted as the following right triangle:
    \begin{drawing}
        \draw[] (0, 0) -- node[anchor=north]{$1$} (2, 0) -- node[anchor=west]{$x$} (2, 1) -- node[yshift=10pt, xshift = -10pt]{$\sqrt{1 + x^2}$} (0, 0);
        \draw[] (1.75, 0) -- (1.75, 0.25) -- (2, 0.25);
        \draw[] (0.75, 0) arc (0:26:0.75);
        
        \draw[] (1, 0.25) node[]{$\theta$};
    \end{drawing}
    Thus,
    \[
        g'(x) = \cos^2 \theta = \pfrac{1}{\sqrt{1 + x^2}}^2 = \frac{1}{1 + x^2}.
    \]
    
\end{proof}

\subsection{}
\begin{problem}
     (5 points) Prove that $\arctan$ is uniformly continuous in $\R$.
\end{problem}

\begin{proof}
    For any $x \in \R$, we have
    \[
        0 < 1 \leq 1 + x^2,
    \]
    which implies
    \[
        0 < \frac{1}{1 + x^2} \leq 1.
    \]
    That is, $|g'(x)| \leq 1$ for all $x \in \R$. Since the derivative of $g$ is bounded on $\R$, then it is uniformly continuous on $\R$.
    
\end{proof}


\section{}
\begin{problem}
    (10 points) If $f:[a,b]\to \R$ is three times differentiable in $[a,b]$ and if 
    \begin{equation}
     f(a) = f'(a) = f(b) = f'(b) = 0,
    \end{equation}
    prove that $f'''(c) = 0$ for some $c$ in $(a,b)$.
\end{problem}

\begin{proof}
    Since $f$ is differentiable on $[a, b]$ and $f(a) = f(b)$, then by Rolle's theorem, there exists some $y \in (a, b)$ such that
    \[
        f'(y) = 0.
    \]
    Similarly, since $f'$ is differentiable on $[a, y]$ and $[y, b]$, and $f'(a) = f'(y) = f'(b)$, then there exist some $x \in (a, y)$ and $z \in (y, b)$ such that
    \[
        f''(x) = 0 \isp{and} f''(z) = 0.
    \]
    Once more, since $f''$ is differentiable on $[x, z]$, then there exists some $c \in (x, z) \subseteq (a, b)$ such that
    \[
        f'''(c) = 0.
    \]
    
\end{proof}

\newpage
\section{}
\begin{problem}
     (10 points) Suppose $g:\R\to\R$ is differentiable, with bounded derivative (say $|g'|\le M$). Fix $\epsilon \in \R$, and define $f(x) = x  + \epsilon g(x)$ for $x\in \R$. Prove that $f$ is one-to-one if $|\epsilon|$ is small enough.
\end{problem}

\begin{proof}
    Since the derivative of $g$ is bounded by $M$, then $g$ is Lipschitz continuous with 
    \[
        |g(x) - g(y)| \leq M|x - y|
    \]
    for all $x, y \in \R$. Assume $|\eps| < 1/M$ and suppose $x, y \in \R$ with $f(x) = g(x)$, then
    \begin{align*}
        x + \eps g(x) &= y + \eps g(y) \\
        x - y &= \eps(g(y) - g(x)) \\
        |x - y| &= |\eps| |g(x) - g(y)| \\
        |x - y| &\leq |\eps| M |x - y|.
    \end{align*}
    If $x \ne y$, then $x - y \ne 0$ and we find
    \[
        |x - y| \leq |\eps| M |x - y| < |x - y|,
    \]
    which is a contradiction. Hence, $x = y$, so $f$ is injective.
    
\end{proof}




\end{document}