\documentclass[12pt]{article}

% Packages
\usepackage[margin=1in]{geometry}
\usepackage{fancyhdr}
\usepackage{amsmath, amsthm, amssymb, physics}

% Page Style
\fancypagestyle{plain}{
    \fancyhf{}
    \renewcommand{\headrulewidth}{0pt}
    \renewcommand{\footrulewidth}{0pt}
    \fancyfoot[R]{\thepage}
}
\pagestyle{plain}

% Problem Box
\setlength{\fboxsep}{4pt}
\newsavebox{\savefullbox}
\newenvironment{fullbox}{\begin{lrbox}{\savefullbox}\begin{minipage}{\dimexpr\textwidth-2\fboxsep\relax}}{\end{minipage}\end{lrbox}\begin{center}\framebox[\textwidth]{\usebox{\savefullbox}}\end{center}}
\newenvironment{pbox}[1][]{\begin{fullbox}\ifx#1\empty\else\paragraph{#1}\fi}{\end{fullbox}}

% Options
\allowdisplaybreaks
\addtolength{\jot}{4pt}
\theoremstyle{definition}

% Default Commands
\newtheorem{proposition}{Proposition}
\newtheorem{lemma}{Lemma}
\newcommand{\ds}{\displaystyle}
\newcommand{\isp}[1]{\quad\text{#1}\quad}
\newcommand{\N}{\mathbb{N}}
\newcommand{\Z}{\mathbb{Z}}
\newcommand{\Q}{\mathbb{Q}}
\newcommand{\R}{\mathbb{R}}
\newcommand{\C}{\mathbb{C}}
\newcommand{\eps}{\varepsilon}
\renewcommand{\emptyset}{\varnothing}

% Extra Commands
\newcommand{\RR}{\mathcal{R}}
\def\[#1\]{\begin{align*}#1\end{align*}}
\newcommand{\tsum}{\textstyle\sum\limits}


% Document Info
\fancypagestyle{title}{
    \renewcommand{\headrulewidth}{0.4pt}
    \setlength{\headheight}{15pt}
    \fancyhead[R]{Harry Coleman}
    \fancyhead[L]{MATH 118B Homework 5}
    \fancyhead[C]{February 16, 2021}
}

% Begin Document
\begin{document}
\thispagestyle{title}


\begin{pbox}[1]
    We say that a function $\phi:\mathbb{R} \to \mathbb{R}$ is convex if
    \begin{equation}
     \phi(\lambda x + (1-\lambda)y) \le \lambda \phi(x) + (1-\lambda) \phi(y)\ \ \forall \lambda \in [0,1],\ \ \forall x, y \in \mathbb{R}.
    \end{equation}
\end{pbox}

\begin{pbox}[(a)]
    Fix $a, b \in \R$. Show that $\exists\,M>0$ such that 
    \begin{equation}
    |\phi(x)| \le M,\quad \forall x \in [a,b].
    \end{equation}
\end{pbox}

\begin{proof}
    Let $M_0 = \max\{|\phi(a)|, |\phi(b)|\}$. Let $x \in [a, b]$ and define $\lambda = (x - b)/(a - b)$. Then $\lambda \in [0, 1]$ and $x = \lambda a + (1 - \lambda)b$. Using the fact that $\phi$ is convex, we find
    \begin{align*}
        \phi(x)
            &\leq \lambda\phi(a) + (1-\lambda)\phi(b) \\
            &\leq \lambda M_0 + (1-\lambda)M_0 \\
            &= M_0.
    \end{align*}
    Hence, $\phi$ is bounded above on $[a, b]$. Define the point $c = (a+b)/2 \in [a, b]$. Let $x \in (c, b)$ and define $\lambda = (c - a)/(x - a)$. Then $\lambda \in (0, 1)$ and $c = \lambda x + (1-\lambda)a$, so convexity gives us
    \begin{align*}
        \phi(c) &\leq \lambda\phi(x) + (1-\lambda)\phi(a) \\
        \phi(c) &\leq \lambda\phi(x) + M_0 \\
        -\lambda\phi(x) &\leq M_0 - \phi(c) \\
        -\phi(x) &\leq \frac{1}{\lambda}(M_0 - \phi(c)).
    \end{align*}
    Note that $\phi(c) \leq M_0$ so $M_0 - \phi(c) \geq 0$. Because $x \in (a, b)$ and $c = (a+b)/2$ and,
    \[
        \lambda
            = \frac{c-a}{x-a}
            \geq \frac{c-a}{b-a}
            = \frac{1}{2}.
    \]
    So $1/\lambda \leq 2$, which gives us $-\phi(x) \leq 2(M_0 - \phi(c))$. We repeat this process for $x \in (a, c)$, defining $\lambda = (c-b)/(x-b)$. Then $\lambda \in (0, 1)$ and $c = \lambda x + (1 - \lambda)b$, so convexity gives us
    \begin{align*}
        \phi(c) &\leq \lambda(x) + (1-\lambda)\phi(b) \\
        -\phi(x) &\leq \frac{1}{\lambda}(M_0 - \phi(c)).
    \end{align*}
    Once again, we have
    \[
        \lambda
            = \frac{c-b}{x-b}
            \geq \frac{c-b}{a-b}
            = \frac{1}{2}.
    \]
    So $1/\lambda \leq 2$, which gives us $-\phi(x) \leq 2(M_0 - \phi(c))$. Defining
    \[
        M = \max\{M_0,\; |\phi(c)|,\; 2(M_0 - \phi(c))\},
    \]
    we have $|\phi(x)| \leq M$ for all $x \in [a, b]$.
    
    
\end{proof}

\begin{pbox}[(b)]
    Given $a<b$, consider $c<d$ such that $[c,d] \subset (a,b)$.  Show that there exists $L>0$ such that 
    \begin{equation}
    |\phi(x)-\phi(y)|\le L |x-y|,\quad \forall\,x,y\in [c,d].
    \end{equation}
    Conclude that {\bf any} convex function is continuous.
    {\it Hint: Note that if $x= (1-\lambda)y+\lambda w$, then 
    \begin{equation}
    \lambda = \frac{x-y}{w-y}.
    \end{equation}
    }
\end{pbox}

\begin{proof}
    Take $M$ as in part (a), bounding $|\phi|$ on $[a,b]$. Let $x, y \in [c, d]$. Without loss of generality, assume $x < y$. Define $\lambda_1 = (x-y)/(a-y)$, then $\lambda_1 \in [0,1]$ and $x = \lambda_1 a + (1-\lambda_1)y$. Convexity gives us
    \begin{align*}
        \phi(x) &\leq \lambda_1\phi(a) + (1-\lambda_1)\phi(y) \\
        \phi(x) - \phi(y) &\leq (\phi(a) - \phi(y))\lambda_1.
    \end{align*}
    Similarly, define $\lambda_2 = (y-x)/(b-x)$, then $\lambda_2 \in [0, 1]$ and $y = \lambda_2 b + (1-\lambda_2)x$. Convexity gives us
    \begin{align*}
        \phi(y) &\leq \lambda_2\phi(b) + (1-\lambda_2)\phi(x) \\
        \phi(y) - \phi(x) &\leq (\phi(b) - \phi(x))\lambda_2.
    \end{align*}
    Therefore, we have
    \[
        |\phi(x) - \phi(y)| \leq \max\{|(\phi(a) - \phi(y))\lambda_1|, |(\phi(b) - \phi(x))\lambda_2|\}.
    \]
    Since $a < c < y$, then $|a-y| \geq |a-c|$, so
    \[
        |(\phi(a) - \phi(y))\lambda_1|
            = \frac{|\phi(a) - \phi(y)|}{|a-y|} |x-y|
            \leq \frac{2M}{|a-c|} |x-y|.
    \]
    Since $x < d < b$, then $|b-x| \geq |b-d|$, so
    \[
        |(\phi(b) - \phi(b))\lambda_2|
            = \frac{|\phi(b) - \phi(x)|}{|b-x|} |x-y|
            \leq \frac{2M}{|b-d|} |x-y|.
    \]
    Define
    \[
        L = \max\left\{\frac{2M}{|a-c|},\; \frac{2M}{|b-d|}\right\}.
    \]
    Then
    \[
        |\phi(x) - \phi(y)| \leq L|x-y|,
    \]
    which holds for all $x, y \in [c,d]$.
    
\end{proof}

 In other words, $\phi$ is Lipschitz continuous, and therefore continuous, on the interval $[c,d]$. Then for any $x \in \R$, we can choose $a, b, c, d \in \R$ such that $a < c < x < d < b$. In which case $\phi$ is continuous on $[c,d]$, in particular, at $x$. So $\phi$ is continuous on $\R$.




\newpage
\begin{pbox}[(c)]
    Prove that if $\lambda_i \in [0,1]$ for $i=1,2,\dots,n$ are such that $\sum_{i=1}^n \lambda_i = 1$ and $x_1,\dots,x_n \in \mathbb{R}$, then 
    \begin{equation}
     \phi(\lambda_1 x_1 + \dots + \lambda_n x_n) \le \lambda_1 \phi(x_1) + \dots + \lambda_n \phi(x_n).
    \end{equation}
\end{pbox}

\begin{proof}
    We prove this by induction on $n \in \N$. Trivially, if $\lambda = 1$ and $x \in \R$, then
    \[
        \phi(\lambda x) = \lambda \phi(x).
    \]
    Assume that the property holds for some $n \geq 1$. Suppose $\lambda_i \in [0,1]$ and $x_i \in \R$ for $i = 1, \dots, n+1$ such that $\sum_{i=1}^{n+1} \lambda_i = 1$. Without loss of generality, assume $\lambda_{n+1} \ne 1$ and define $\lambda_i' = \lambda_i/(1 - \lambda_{n+1})$ for $i = 1, \dots, n$. Then
    \[
        \lambda_1 + \cdots + \lambda_n + \lambda_{n+1} &= 1, \\[6pt]
        \lambda_1 + \cdots + \lambda_n &= 1 - \lambda_{n+1}, \\[6pt]
        \frac{\lambda_1 + \cdots + \lambda_n}{1 - \lambda_{n+1}} &= \frac{1 - \lambda_{n+1}}{1 - \lambda_{n+1}}, \\[6pt]
        \lambda_1' + \cdots + \lambda_n' &= 1.
    \]
    We can now rewrite the summation to fulfill the necessary conditions to apply the convexity of $\phi$, i.e.,
    \begin{align*}
        \phi\!\left(\tsum_{i=1}^{n+1} \lambda_i x_i\right)
            &= \phi\!\left(\lambda_{n+1}x_{n+1} + \tsum_{i=1}^{n} \lambda_i x_i\right) \\
            &= \phi\!\left(\lambda_{n+1}x_{n+1} + (1 - \lambda_{n+1})\tsum_{i=1}^{n} \lambda_i' x_i\right) \\
            &\leq \lambda_{n+1}\phi(x_{n+1}) + (1 - \lambda_{n+1})\phi\!\left(\tsum_{i=1}^{n} \lambda_i' x_i\right).
    \end{align*}
    And applying the inductive hypothesis to the second term gives us
    \begin{align*}
        \phi\!\left(\textstyle\sum\limits_{i=1}^{n+1} \lambda_i x_i\right)
            &\leq \lambda_{n+1}\phi(x_{n+1}) + (1 - \lambda_{n+1})\sum_{i=1}^{n} \lambda_i'\phi(x_i) \\
            &= \lambda_{n+1}\phi(x_{n+1}) + \sum_{i=1}^{n} \lambda_i\phi(x_i) \\
            &= \sum_{i=1}^{n+1} \lambda_i\phi(x_i).
    \end{align*}
    
\end{proof}


\newpage
\begin{pbox}[(d)]
    Consider $f\in \mathcal{R}([a,b])$, and show that $\phi \circ f: [a,b] \to \mathbb{R}$ is also Riemann integrable. Prove that 
    \begin{equation}
     \phi \left ( \frac{1}{b-a} \int_a^b f(x)\,dx \right ) \le \frac{1}{b-a} \int_a^b \phi(f(x))\,dx.
    \end{equation}
\end{pbox}

\begin{proof}
    Since $f \in \RR([a,b])$ and $\phi$ is continuous on $\R$, in particular on the image $f([a,b])$ (which is a closed, bounded interval as $f$ is continuous), then the composition $\phi \circ f$ is also Riemann integrable on $[a, b]$. For a given $n \in \N$, we define a partition of $[a, b]$ by the equidistributed points $x_{n,k} = a + (b-a)k/n$ for $k = 0, 1, \dots, n$. Since the diameter of these partitions converges to zero (each interval has a width of $(b-a)/n$), then we can rewrite the integral as the limit of Riemann sums to obtain
    \[
        \frac{1}{b-a}\int_{a}^{b} f(x) \dd{x}
            = \lim_{n \to \infty} \sum_{k=1}^{n} \frac{f(x_{n,k})}{n}.
    \]
    Since $\phi$ is continuous, then we can interchange it with the limit, i.e.,
    \[
        \phi\!\left(\lim_{n \to \infty} \sum_{k=1}^{n} \frac{f(x_{n,k})}{n}\right)
         = \lim_{n \to \infty} \phi\!\left(\sum_{k=1}^{n} \frac{f(x_{n,k})}{n}\right).
    \]
    Similar to that of $f$, the integral of $\phi \circ f$ can be rewritten as the limit of Riemann sums
    \[
        \frac{1}{b-a} \int_{a}^{b} \phi(f(x)) \dd{x}
            = \lim_{n \to \infty} \sum_{k=1}^{n} \frac{\phi(f(x_{n,k}))}{n}.
    \]
    A particular case of part (c) is having $n$ points, each with a weight of $1/n$. So the convexity of $\phi$ gives us
    \[
        \phi\!\left(\sum_{k=1}^{n} \frac{f(x_{n,k})}{n}\right)
            \leq \sum_{k=1}^{n} \frac{\phi(f(x_{n,k}))}{n}.
    \]
    And since both limits converge, we obtain the inequality
    \[
        \lim_{n \to \infty} \phi\!\left(\sum_{k=1}^{n} \frac{f(x_{n,k})}{n}\right)
            \leq \lim_{n \to \infty} \sum_{k=1}^{n} \frac{\phi(f(x_{n,k}))}{n}.
    \]
    Substituting both sides with the above expressions, we have
    \[
        \phi\!\left(\frac{1}{b-a}\int_{a}^{b} f(x) \dd{x}\right)
            \leq \frac{1}{b-a} \int_{a}^{b} \phi(f(x)) \dd{x}.
    \]
    
\end{proof}



\newpage
\begin{pbox}[2]
    Find the following limits:
\end{pbox}

\begin{pbox}[(a)]
    \begin{equation}
    \lim _{n\to \infty} \left ( \frac{1}{n+1} +\frac{1}{n+2}+\ldots+\frac{1}{3n}\right ).
    \end{equation}
\end{pbox}

If it exists, we want to find
\[
    \lim_{n \to \infty} \sum_{k=1}^{2n} \frac{1}{n + k}
        = \lim_{n \to \infty} \frac{1}{n} \sum_{k=1}^{2n} \frac{1}{1 + \frac{k}{n}}.
\]
Consider the function
\[
    f(x) = \frac{1}{1 + x},
\]
which is continuous on $[0,2]$, so $f \in \RR([0, 2])$. The integral can be found as the limit of any sequence of Riemann sums with the diameters of the partitions converging to zero. In particular,
\[
    \int_{0}^{2} f(x) \dd{x}
        = \lim_{n \to \infty} \frac{1}{n} \sum_{k=1}^{2n} f\!\left(\tfrac{k}{n}\right)
        = \lim_{n \to \infty} \frac{1}{n} \sum_{k=1}^{2n} \frac{1}{1 + \frac{k}{n}}.
\]
Since the chain rule gives us
\[
    \dv{x} \log(x+1)
        = \frac{1}{x+1}\dv{x}(x+1)
        = f(x),
\]
then from the fundamental theorem of calculus we know that
\[
    \int_{0}^{2} f(x) \dd{x}
        = \log(2 + 1) - \log(0 + 1)
        = \log 3.
\]
Hence,
\[
     \lim_{n \to \infty} \sum_{k=1}^{2n} \frac{1}{n + k} = \log 3.
\]


\newpage
\begin{pbox}[(b)]
    \begin{equation}
    \lim_{n\to \infty} n^4 \int_n^{n+1} \frac{x}{1+x^5} \dd{x}.
    \end{equation}
\end{pbox}

For each $n \in \N$, the mean value theorem for integrals gives us $x_n \in [n, n+1]$ such that
\[
    \int_n^{n+1} \frac{x}{1+x^5} \dd{x}
        = \frac{x_n}{1+x_n^5}.
\]
Then the limit can be rewritten as
\[
    \lim_{n\to \infty} n^4 \int_n^{n+1} \frac{x}{1+x^5} \dd{x}
        = \lim_{n\to \infty} \frac{n^4x_n}{1+x_n^5}
        = \lim_{n\to \infty} \frac{1}{\dfrac{1}{n^4x_n} + \dfrac{x_n^4}{n^4}}.
\]
Now since
\[
    1
        = \frac{n^4}{n^4}
        \leq \frac{x_n^4}{n^4}
        \leq \frac{(n+1)^4}{n^4}
        = \left(1 + \frac{1}{n}\right)^4,
\]
then the squeeze theorem tells us that
\[
    \lim_{n \to \infty} \frac{x_n^4}{n^4} = 1.
\]
Similarly, we have
\[
    0
        \leq \frac{1}{n^4 x_n}
        \leq \frac{1}{n^4},
\]
so
\[
    \lim_{n \to \infty} \frac{1}{n^4 x_n} = 0.
\]
Substituting these values into the original limit, we find
\[
    \lim_{n\to \infty} n^4 \int_n^{n+1} \frac{x}{1+x^5} \dd{x} = 1.
\]




\newpage
\begin{pbox}[3]
    Verify whether the following integrals converge or diverge:
\end{pbox}

\begin{pbox}[(a)]
    \begin{equation}
    \int_2^\infty \frac{\dd{x}}{x\log x}.
    \end{equation}
\end{pbox}

If it exists, the improper integral is given by the limit
\[
    \int_2^\infty \frac{\dd{x}}{x\log x} = \lim_{b \to \infty} \int_{2}^{b} \frac{\dd{x}}{x\log x}.
\]
Assuming $b > 2$, the function
\[
    f(x) = \frac{1}{x \log x}
\]
is continuous on the bounded interval $[2, b]$, so $f \in \RR([2, b])$. Moreover,
\[
    \dv{x} \log(\log x) = \frac{1}{\log x} \dv{x} \log x = f(x),
\]
so the fundamental theorem of calculus tells gives us
\[
    \int_{2}^{b} \frac{1}{x \log x} = \log(\log b) - \log(\log 2).
\]
And since
\[
    \lim_{b \to \infty} \log b = \infty,
\]
then we find that
\[
    \lim_{b \to \infty} \int_{2}^{b} \frac{\dd{x}}{x\log x}
        = \lim_{n \to \infty} \log(\log b) - \log(\log 2)
        = \infty.
\]
Hence, the improper integral diverges to $\infty$.





\newpage
\begin{pbox}[(b)]
    \begin{equation}
    \int_0^\infty \frac{\cos x}{1+x^2} \dd{x}.
    \end{equation}
\end{pbox}

The improper integral
\[
    \int_0^{\infty} \frac{\cos x}{1+x^2} \dd{x}
\]
converges if it is absolutely convergent, i.e., if the improper integral
\[
    \int_0^{\infty} \left|\frac{\cos x}{1+x^2}\right| \dd{x}
\]
converges. For $b > 0$, the functions
\[
    \frac{\cos x}{1 + x^2} \isp{and} \frac{1}{1 + x^2}
\]
are continuous, and therefore Riemann integrable, on the bounded interval $[0, b]$. And since
\[
    \left|\frac{\cos x}{1  + x^2}\right|
        = \frac{|\cos x|}{1 + x^2}
        \leq \frac{1}{1 + x^2},
\]
then
\[
    \int_0^b \left|\frac{\cos x}{1+x^2}\right| \dd{x}
        \leq \int_0^b \frac{1}{1+x^2} \dd{x}.
\]
Moreover, we have that
\[
    \dv{x} \arctan x = \frac{1}{1 + x^2},
\]
so the fundamental theorem of calculus tells us
\[
    \int_0^b \frac{1}{1+x^2} \dd{x}
        = \arctan b - \arctan 0
        = \arctan b,
\]
so
\[
    \lim_{b \to \infty} \int_0^b \left|\frac{\cos x}{1+x^2}\right| \dd{x}
        \leq \lim_{b \to \infty} \arctan b
        = \frac{\pi}{2}.
\]
Since the limit is increasing and bounded above, it is convergent, i.e., the improper integral
\[
    \int_0^{\infty} \left|\frac{\cos x}{1+x^2}\right| \dd{x}
        = \lim_{b \to \infty} \int_0^b \left|\frac{\cos x}{1+x^2}\right| \dd{x}
\]
converges.





\end{document}