\documentclass[12pt]{article}

% packages
\usepackage{kantlipsum}
\usepackage[margin=1in]{geometry}
\usepackage[labelfont=it]{caption}
\usepackage[table]{xcolor}
\usepackage{subcaption,framed,colortbl,multirow,enumitem}
\usepackage{amsmath,amsthm,amssymb,wasysym,mathrsfs,mathtools,babel,accents}
\usepackage{tikz,graphicx,pgf,pgfplots}
\usetikzlibrary{arrows, angles, quotes, decorations.pathreplacing, math, patterns, calc}
\pgfplotsset{compat=1.16}

% Theorems
\newtheorem{theorem}{Theorem}
\newtheorem{lemma}{Lemma}
\newtheorem{proposition}{Proposition}

% Problem Box
\setlength{\fboxsep}{4pt}
\newsavebox{\mybox}
\newenvironment{problem}
    {\begin{lrbox}{\mybox}\begin{minipage}{\textwidth-10pt}}
    {\end{minipage}\end{lrbox}\framebox[6.5in]{\usebox{\mybox}}}

% Environments
\newenvironment{drawing}{\begin{center}\begin{tikzpicture}}{\end{tikzpicture}\end{center}}
\newenvironment{response}{\paragraph{}}{}

% Formatting
\newcommand{\ds}{\displaystyle}
\newcommand{\isp}[1]{\quad\text{#1}\quad}
\newcommand{\seq}[2]{\left\{#1\right\}_{#2=1}^\infty}
\newcommand{\clo}[1]{\overline{#1}}

% Paired Delimiters
\DeclarePairedDelimiter{\ceil}{\lceil}{\rceil}
\DeclarePairedDelimiter\floor{\lfloor}{\rfloor}
\DeclarePairedDelimiter{\ang}{\langle}{\rangle}

% Sets
\newcommand{\N}{\mathbb{N}}
\newcommand{\Z}{\mathbb{Z}}
\newcommand{\I}{\mathbb{I}}
\newcommand{\R}{\mathbb{R}}
\newcommand{\Q}{\mathbb{Q}}
\newcommand{\C}{\mathbb{C}}
\newcommand{\F}{\mathbb{F}}

% Misc Characters
\newcommand{\powerset}{\raisebox{.15\baselineskip}{\Large\ensuremath{\wp}}}
\let\eps\varepsilon
\let\emptyset\varnothing

% Functions
\newcommand{\id}[1]{\mathsf{id}_{#1}}

% Babel Shorthands
\useshorthands*{"}
\defineshorthand{"-}{\setminus}
\defineshorthand{"d}{\partial}

% Probability
\newcommand{\FF}{\mathcal{F}}
\renewcommand{\P}{\mathbb{P}}
\newcommand{\bx}{\ensuremath{\mathbf{x}}} 
\newcommand{\by}{\ensuremath{\mathbf{y}}} 
\newcommand{\bz}{\ensuremath{\mathbf{z}}}

% Complex Analysis
\renewcommand{\Im}{\text{Im }}
\renewcommand{\Re}{\text{Re }}
\newcommand{\Arg}{\text{Arg }}
 
\begin{document}
 
\title{Homework 3\\
    %\large MATH CS 121 Intro to Probability
    %\large MATH CS 122A Complex Analysis I
    \large MATH 118A Intro to Real Analysis
    %\large MATH 111A Intro to Abstract Algebra
}
\author{Harry Coleman}
\date{October 22, 2020}
\maketitle

 \section*{Exercise 2}
\begin{problem}
    Determine whether the following function is a metric in the set $\mathbb{R}^+$ of positive real numbers, and if so, determine what a ball is in this metric: 
    \begin{equation}
    d(x,y) = \Big |\log\left ( \frac{x}{y} \right ) \Big |,\quad x,y>0.
    \end{equation}
\end{problem}

\begin{proposition}
    $d(x,y)$ is a metric on $\R^+$.
\end{proposition}

\begin{proof}
    By definition, $d(x,y)\geq 0$ since it is an absolute value. Additionally, $d(x,y)=0$ if any only if
    \[\log \frac xy = 0,\]
    which is the case if and only if
    \[e^0 = 1 = \frac xy,\]
    which is the case if and only if $x=y$. Secondly, for any $x,y\in\R^+$, we have by the properties of the logarithm
    \begin{align*}
        d(x,y)
            &= \left|\log\frac xy\right| \\
            &= \left|\log x - \log y\right| \\
            &= \left|-(\log y - \log x)\right| \\
            &= \left|-\log \frac yx\right| \\
            &= \left|\log \frac yx\right| \\
            &= d(y,x).
    \end{align*}
    Lastly, for any $x,y,z\in\R^+$, we have by the properties of the logarithm and the triangle inequality of absolute value
    \begin{align*}
        d(x,z)
            &= \left|\log\frac xz\right| \\
            &= \left|\log\frac xy\frac yz\right| \\
            &= \left|\log\frac xy + \log\frac yz\right| \\
            &\leq \left|\log\frac xy\right| + \left|\log\frac yz\right| \\
            &= d(x,y) + d(y,z).
    \end{align*}
    
\end{proof}

Given a center $x\in\R^+$ and a radius $r>0$, a point $y\in\R^+$ is in the ball $B_r(x)$ if and only if
\[d(y,x) = \left|\log\frac yx\right| < r.\]
This condition is equivalent to the following inequalities:
\begin{alignat*}{3}
    -r &<& \log\frac yx &<& r, \\
    e^{-r} &<& \frac yx &<& e^r, \\
    xe^{-r} &<& y &<& xe^r.
\end{alignat*}
Therefore, we have
\[B_r(x) = \{y\in\R^+ : xe^{-r}<y<xe^r\}.\]

\section*{Exercise 3}
\begin{problem}
    Let $(X,d)$ be a metric space, and $A\subseteq X$. Define the distance from $x$ to $A$ by
    \begin{equation}
    d(x,A) = \inf \left \{ d(x,y):\quad  y \in A \right \}.
    \end{equation}
\end{problem}

\newpage
\subsection*{Exercise 3(a)}
\begin{problem}
    Prove that
    \[
    \bigg | d(x,A) - d(y,A) \bigg | \le d(x,y)\ \ \forall x,y \in X.
    \]
\end{problem}

\begin{proof}
    Let $x,y\in X$. For all $z\in A$, we have the triangle inequalities
    \begin{align*}
        d(x,z) &\leq d(x,y) + d(y,z), \\
        d(y,z) &\leq d(x,y) + d(x,z).
    \end{align*}
    By the definition of the distance from a point to $A$, we have
    \begin{align*}
        d(x,A) &\leq d(x,y) + d(y,z), \\
        d(y,A) &\leq d(x,y) + d(x,z),
    \end{align*}
    equivalently,
    \begin{align*}
        d(x,A) - d(x,y) &\leq d(y,z), \\
        d(y,A) - d(x,y) &\leq d(x,z),
    \end{align*}
    Note now that the left-hand sides are lower bounds for $d(y,z)$ and $d(x,z)$ for all $z\in A$. Therefore, by the definition of infimia, we have
    \begin{align*}
        d(x,A) - d(x,y) &\leq d(y,A), \\
        d(y,A) - d(x,y) &\leq d(x,A),
    \end{align*}
    which implies
    \begin{align*}
        d(x,A) - d(y,A) &\leq d(x,y), \\
        d(y,A) - d(x,A) &\leq d(x,y).
    \end{align*}
    Thus, we obtain the desired result
    \[\bigg| d(x,A) - d(y,A) \bigg| \leq d(x,y),\]
    for all $x,y\in X$.
    
\end{proof}

\subsection*{Exercise 3(b)}
\begin{problem}
    Prove that $d(x,A) = 0$ if and only if $x \in \overline{A}$. 
\end{problem}

\begin{proof}
    Suppose $d(x,A)=0$. Then for any $r>0$ there exists some $z\in A$ such that $d(x,z)<r$, that is, $z\in B_r(x)$. So for all $r>0$, $B_r(x)\cap A \ne\emptyset$, therefore $x\in\clo{A}$. Now suppose $x\in\clo{A}$, so for all $r>0$, $B_r(x)\cap A \ne\emptyset$. So for any $r>0$, there exists some $z\in A$ with $d(x,z)<r$, and since $d(x,A)\leq d(x,z)$ for all $z\in A$, we have $d(z,A)\leq r$ for all $r>0$. Therefore $d(x,A)=0$.
    
\end{proof}

\section*{Exercise 4}
\begin{problem}
    Let $A_1$, $A_2$, $A_3$, $\dots$ be subsets of a metric space. 
\end{problem}

\subsection*{Exercise 4(a)}
\begin{problem}
    If $B_n = \cup _{i=1}^n A_i$, prove that $\overline{B_n} = \cup_{i=1}^n \overline{A_i}$.  
\end{problem}

\begin{proof}
    Let $x\in\clo{B_n}$, so $B_r(x)\cap B_n \ne\emptyset$ for all $r>0$. For a given $r>0$, let $z\in B_r(x)\cap B_n$. Since $z\in B_n$ and $B_n = \cup_{i=1}^n A_i$, then $z\in A_i$ for some $i\in\{1,\dots,n\}$. Therefore, for all $r>0$, there is some $i\in\{1,\dots,n\}$ such that $B_r(x)\cap A_i \ne\emptyset$.
    
    Suppose for contradiction that there does not exist a particular $j\in\{1,\dots,n\}$, such that $B_r(x)\cap A_j \ne\emptyset$ for all $r>0$. This implies that for all $i\in\{1,\dots,n\}$, there is some $r_i>0$ such that $B_{r_i}(x)\cap A_i=\emptyset$. Then letting $r=\min\{r_1,\dots,r_n\}$, we have $B_r(x)\cap A_i = \emptyset$ for all $i\in\{1,\dots,n\}$, which contradicts the conclusion of the previous paragraph. Thus for a particular $j\in\{1,\dots,n\}$, we have $B_r(x)\cap A_j\ne\emptyset$ for all $r>0$. So $x\in\clo{A_j}$ which implies $x\in\cup_{i=1}^n \overline{A_i}$.
    
    Now suppose $x\in\cup_{i=1}^n \overline{A_i}$, so $x\in\clo{A_j}$ for some $j\in\{1,\dots,n\}$. Now for a particular $r>0$, let $z\in B_r(x)\cap A_j$. Since $B_n = \cup_{i=1}^n A_i$, we have $z\in B_n$. Thus $B_r(x)\cap B_n\ne\emptyset$ for all $r>0$, giving us $x\in\clo{B_n}$.
    
\end{proof}

\subsection*{Exercise 4(b)}
\begin{problem}
    If $B = \cup _{i=1}^{\infty} A_i$, prove that $\overline{B} \supseteq \cup_{i=1}^{\infty} \overline{A_i}$.  
\end{problem}

\begin{proof}
    Suppose $x\in\cup_{i=1}^\infty \overline{A_i}$, so $x\in\clo{A_j}$ for some $j\in\N$. Now for a particular $r>0$, let $z\in B_r(x)\cap A_j$. Since $B = \cup_{i=1}^\infty A_i$, we have $z\in B$. Thus $B_r(x)\cap B\ne\emptyset$ for all $r>0$, giving us $x\in\clo{B}$.
    
\end{proof}

\subsection*{Exercise 4(c)}
\begin{problem}
    Give an example for which (b) above is strict.  
\end{problem}
\paragraph{}

Let $A_i$ denote the interval $(1/i, 2) \subseteq\R$ for all $i\in\N$. Then $B=\cup _{i=1}^{\infty} A_i = (0,2)$ and $\clo{B}=[0,2]$. However, we have $\clo{A_i}=[1/i,2]$ and $\cup_{i=1}^{\infty} \overline{A_i} = (0,2]$, which is a strict subset of $B$.

\section*{Exercise 5}
\begin{problem}
    Remember that $\accentset{\circ}{E}$ denotes the set of interior points of a set $E$:
\end{problem}

\subsection*{Exercise 5(a)}
\begin{problem}
    Prove that the complement of $\accentset{\circ}{E}$ is the closure of the complement of $E$:
    \begin{equation}
    \left ( \accentset{\circ}{E} \right )^c = \overline{E^c}.
    \end{equation}
\end{problem}

\begin{proof}
    For any $x$ in the space, we have by definition of complement
    \[x\in\left(\accentset{\circ}{E}\right)^C \iff x\notin\accentset{\circ}{E}.\]
    Since the definition of interior tells us that $x\in\accentset{\circ}{E}$ if and only if there exists some $r>0$ such that $B_r(x)\subseteq E$, then
    \[x\notin\accentset{\circ}{E} \iff \forall r>0,\ B_r(x)\not\subseteq E.\]
    And since $B_r(x)\subseteq E$ if and only if $B_r(x)\cap E^C=\emptyset$, then
    \[\forall r>0,\ B_r(x)\not\subseteq E \iff \forall r>0,\ B_r(x)\cap E^C \ne\emptyset.\]
    Finally, by the definition of closure, we have
    \[\forall r>0,\ B_r(x)\cap E^C \ne\emptyset \iff x\in\clo{E^C},\]
    giving us
    \[x\in\left(\accentset{\circ}{E}\right)^C \iff x\in\clo{E^C}.\]
    
\end{proof}

\subsection*{Exercise 5(b)}
\begin{problem}
    Do $E$ and $\overline{E}$ always have the same interiors? Prove or give a counterexample.
\end{problem}
\paragraph{}

Consider the set $E = (-1,0)\cup(0,1)$. Clearly $0\notin\accentset{\circ}{E}$ since $0\notin E$. However, $\clo{E}=[-1,1]$, so $0\in\accentset{\circ}{\clo{E}}$. Thus $\accentset{\circ}{E}\ne\accentset{\circ}{\clo{E}}$


\subsection*{Exercise 5(c)}
\begin{problem}
    Do $E$ and $\accentset{\circ}{E}$ always have the same closures? Prove or give a counterexample.
\end{problem}
\paragraph{}

Consider $E=\{0\}$, then $\clo{E}=\{0\}$. However, $\accentset{\circ}{E}=\emptyset$, so $\clo{\accentset{\circ}{E}}=\emptyset$. Thus $\clo{E}\ne\clo{\accentset{\circ}{E}}$.


\end{document}