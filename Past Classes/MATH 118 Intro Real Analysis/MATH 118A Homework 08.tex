\documentclass[12pt]{article}

% Packages
\usepackage[margin=1in]{geometry} % proper margins
\usepackage{enumitem} % custom numbering for lists
\usepackage{amsmath} % align, cases, eqref, matrices, dots, roots, delimiters, math mode functions, mod, arrows
\usepackage{amsthm} % theorems, proofs
\usepackage{amssymb} % fancy letters, niche relations/negations
\usepackage{mathrsfs} % much more loopy calligraphy font

% Theorems
\newtheorem{theorem}{Theorem}
\newtheorem{lemma}{Lemma}
\newtheorem{proposition}{Proposition}

% Problem Box
\setlength{\fboxsep}{4pt}
\newsavebox{\mybox}
\newenvironment{problem}
    {\begin{lrbox}{\mybox}\begin{minipage}{0.98\textwidth}}
    {\end{minipage}\end{lrbox}\framebox[\textwidth]{\usebox{\mybox}}}

% Formatting
\newcommand{\ds}{\displaystyle}
\newcommand{\isp}[1]{\quad\text{#1}\quad}
\newcommand*{\defeq}{\mathrel{\vcenter{\baselineskip0.5ex \lineskiplimit0pt\hbox{\scriptsize.}\hbox{\scriptsize.}}}=}

% Alternate Characters
\let\eps\varepsilon % double curved, rather than single curve with midline
\let\phi\varphi % single stroke loop, rather than vertical line through circle
\let\emptyset\varnothing % circular, rather than tall

% Named Sets
\newcommand{\N}{\mathbb{N}} % natural numbers
\newcommand{\Z}{\mathbb{Z}} % integers 
\newcommand{\Q}{\mathbb{Q}} % rational numbers
\newcommand{\R}{\mathbb{R}} % real numbers
\newcommand{\C}{\mathbb{C}} % complex numbers

% Fancy Characters
\newcommand{\F}{\mathbb{F}} % arbitrary field
\renewcommand{\P}{\mathbb{P}} % probability measure (apparently, sometimes prime numbers)
\newcommand{\FF}{\mathcal{F}} % sigma algebra
\newcommand{\BB}{\mathcal{B}} % Borel sigma algebra

% Paired Delimiters
\newcommand{\ceil}[1]{\left\lceil #1 \right\rceil} % ceiling
\newcommand{\floor}[1]{\left\lfloor #1 \right\rfloor} % floor
\newcommand{\<}{\left\langle} % left angle bracket
\renewcommand{\>}{\right\rangle} % right angle bracket

% Functions
\renewcommand{\Im}{\operatorname{Im}} % imaginary part of a complex number
\renewcommand{\Re}{\operatorname{Re}} % real part of a complex number
\newcommand{\Arg}{\operatorname{Arg}} % principal argument (angle) of complex number

% Simplified Notation
\newcommand{\id}[1]{\operatorname{id_{\mathnormal{#1}}}} % identity operator
\newcommand{\seq}[2][n]{\left\{#2\right\}_{#1\in\N}} % sequence
\renewcommand{\d}[1]{\operatorname{d}\!#1} % differential operator
\newcommand{\od}[3][1]{\ifnum#1=1{\frac{\d #2}{\d #3}}\else{\frac{\d^{#1}#2}{\d #3^{#1}}}\fi} % ordinary derivative
\newcommand{\pd}[3][1]{\ifnum#1=1{\frac{\partial #2}{\partial#3}}\else{\frac{\partial^{#1}#2}{\partial#3^{#1}}}\fi} % partial derivative
\newcommand{\intr}[1]{\accentset{\circ}{#1}} % interior of a set

% Renaming
\let\sm\setminus % set minus (difference)
\let\clo\overline % closure of a set
\let\conj\overline % conjugate of an object
\let\eqc\overline % equivalence class of an object
\let\teq\trianglelefteq % normal subgroup
\let\iso\cong % isomorphic (groups)

% Notes
% medskip for header-less paragraph
% intertext{} for short text inside big display structure
% dots is dynamic based on surroundings
% dfrac and tfrac to force large or small fractions
% operatorname for new operators instead of text
% consider using nath; it seems to break most formatting and is not compatible with many packages

\begin{document}
 
\title{Homework 8\\
    %\large MATH CS 121 Intro to Probability
    %\large MATH 111A Intro to Abstract Algebra
    %\large MATH CS 122A Complex Analysis I
    \large MATH 118A Intro to Real Analysis
    %\large MATH 104A Intro to Numerical Analysis
}
\author{Harry Coleman}
\date{December 1, 2020}
\maketitle

\section*{Exercise 1}
\begin{problem}
    Does the sequence 
    \[
     \left \{ \sum_{n=1}^k \left ( \frac{1}{\sqrt{k^2+n}} \right )\right \}_{k=1}^\infty
    \]
    converge or diverge? If the sequence converges, find the limit.
    
    \noindent {\it Hint: Use the squeeze theorem.}
\end{problem}

\begin{proposition}
    The sequence converges and its limit is 1.
\end{proposition}

\begin{proof}
    For a fixed $k \in \N$, we consider the term in the sequence
    \[\sum_{n=1}^k \left( \frac{1}{\sqrt{k^2+n}} \right).\]
    Now for every $n \in \{1, \dots, k\}$, we have
    \begin{align*}
        k^2 &\leq k^2 + n, \\
        k &\leq \sqrt{k^2 + n}, \\
        \frac1k &\geq \frac1{\sqrt{k^2 + n}}.
    \end{align*}
    Therefore, we find the following upper bound for the $k$th term in the sequence:
    \[\sum_{n=1}^k \left( \frac{1}{\sqrt{k^2+n}} \right) \leq \sum_{n=1}^k \frac1k = k \cdot \frac1k = 1.\]
    Now for every $n \in \{1, \dots, k\}$, we have
    \begin{align*}
        k^2 + n &\leq k^2 + k, \\
        \sqrt{k^2 + n} &\leq \sqrt{k^2 + k} = k\sqrt{1 + \frac1k}, \\
        \frac1{\sqrt{k^2 + n}} &\geq \frac1{k\sqrt{1 + \frac1k}}.
    \end{align*}
    Therefore, we find the following lower bound for the $k$th term in the sequence:
    \[\sum_{n=1}^k \left( \frac{1}{\sqrt{k^2+n}} \right) \geq \sum_{n=1}^k \left( \frac1{k\sqrt{1 + \frac1k}} \right) = k \cdot \frac1{k\sqrt{1 + \frac1k}} = \frac1{\sqrt{1 + \frac1k}}.\]
    In summary, we have shown that for all $k \in \N$,
    \[\frac1{\sqrt{1 + \frac1k}} \leq \sum_{n=1}^k \left( \frac{1}{\sqrt{k^2+n}} \right) \leq 1.\]
    And since
    \[\lim_{k \to \infty} \frac1{\sqrt{1 + \frac1k}} = \frac1{\sqrt{1 + 0}} = 1,\]
    then by the squeeze theorem, the sequence converges and
    \[\lim_{k \to \infty} \sum_{n=1}^k \left( \frac{1}{\sqrt{k^2+n}} \right) = 1.\]
    
\end{proof}

\newpage
\section*{Exercise 2}
\begin{problem}
    Determine whether the following series converge absolutely, conditionally, or diverge:
\end{problem}

\subsection*{Exercise 2(a)}
\begin{problem}
    \begin{equation}
        \sum_{n=1}^\infty (-1)^n (\sqrt{n+1}-\sqrt{n}).
    \end{equation}
\end{problem}

\begin{proposition}
    The series converges conditionally.
\end{proposition}

\begin{proof}
    We first show that the series does not converge absolutely. For an arbitrary $n \in \N$, we consider the partial sum of absolute terms
    \[S_n = \sum_{k=1}^n \left| (-1)^k (\sqrt{k+1} - \sqrt{k}) \right| = \sum_{k=1}^n \left| \sqrt{k+1} - \sqrt{k} \right|.\]
    For each $k \in \{1, \dots, n\}$, we have
    \begin{align*}
        k &\leq k + 1, \\
        \sqrt{k} &\leq \sqrt{k + 1}, \\
        0 & \leq \sqrt{k + 1} - \sqrt{k}.
    \end{align*}
    Therefore,
    \[\left| \sqrt{k+1} - \sqrt{k} \right| = \sqrt{k+1} - \sqrt{k},\]
    and we have
    \[S_n = \sum_{k=1}^n \left( \sqrt{k+1} - \sqrt{k} \right).\]
    We derive the following expression of $S_n$ by telescoping the middle terms in the summation:
    \begin{align*}
        S_n
            &= \sum_{k=1}^n \sqrt{k+1} - \sum_{k=1}^n \sqrt{k} \\
            &= \sum_{k=2}^{n + 1} \sqrt{k} - \sum_{k=1}^n \sqrt{k} \\
            &= \sqrt{n + 1} + \sum_{k=2}^n \sqrt{k} - \sum_{k=2}^n \sqrt{k} - \sqrt{1} \\
            &= \sqrt{n + 1} - 1.
    \end{align*}
    It is now clear that $S_n$ diverges as $n \to \infty$, so the series is not absolutely convergent. Next, we will show that the alternating series (1), itself, converges by showing that
    \[\lim_{n \to \infty} \left( \sqrt{n+1} - \sqrt{n} \right) = 0.\]
    For an arbitrary $n \in \N$, we derive the following equivalent expression:
    \begin{align*}
        \left( \sqrt{n+1} - \sqrt{n} \right)
            &= \left( \sqrt{n+1} - \sqrt{n} \right) \cdot \frac{\sqrt{n+1} + \sqrt{n}}{\sqrt{n+1} + \sqrt{n}} \\
            &= \frac{n + 1 - n}{\sqrt{n+1} + \sqrt{n}} \\
            &= \frac{1}{\sqrt{n+1} + \sqrt{n}}.
    \end{align*}
    Now since
    \[\lim_{n \to \infty} \sqrt{n+1} + \sqrt{n} = \infty,\]
    then
    \[\lim_{n \to \infty} \left( \sqrt{n+1} - \sqrt{n} \right) = \lim_{n \to \infty} \frac{1}{\sqrt{n+1} + \sqrt{n}} = 0.\]
    Thus, the series (1) converges, but not absolutely. 
    
\end{proof}

\newpage
\subsection*{Exercise 2(b)}
\begin{problem}
    \begin{equation}
        \sum_{n=1}^\infty (-1)^n n \left( \sqrt{4 - \frac{1}{n}} - 2 \right)
    \end{equation}
\end{problem}

\begin{proposition}
    The series diverges.
\end{proposition}

\begin{proof}
    We will show that the series (2) diverges by showing that the sequence of terms does not converge to zero. For an arbitrary $n \in \N$, we derive the following equivalent expression:
    \begin{align*}
        n \left( \sqrt{4 - \frac{1}{n}} - 2 \right)
            &= n \left( \sqrt{4 - \frac{1}{n}} - 4 \right) \cdot \frac{\sqrt{4 - \frac{1}{n}} + 2}{\sqrt{4 - \frac{1}{n}} + 2} \\
            &= n \frac{4 - \frac{1}{n} - 4}{\sqrt{4 - \frac{1}{n}} + 2} \\
            &= \frac{-1}{\sqrt{4 - \frac{1}{n}} + 2}.
    \end{align*}
    Now since
    \[\lim_{n \to \infty} \frac{1}{n} = 0,\]
    then
    \[\lim_{n \to \infty} n \left( \sqrt{4 - \frac{1}{n}} - 2 \right) = \lim_{n \to \infty} \frac{-1}{\sqrt{4 - \frac{1}{n}} + 2} = \frac{-1}{\sqrt{4 - 0} + 2} = \frac{-1}{4}.\]
    Now since the limit of the absolute value of a convergent sequence is the absolute value of the limit,
    \[\lim_{n \to \infty} \left| (-1)^n n \left( \sqrt{4 - \frac{1}{n}} - 2 \right) \right| = \lim_{n \to \infty} \left| n \left( \sqrt{4 - \frac{1}{n}} - 2 \right) \right| = \left| \frac{-1}{4} \right| = \frac{1}{4}.\]
    Thus, the series (2) does not converge because the sequence of its terms does not converge to zero.
    
\end{proof}

\newpage
\subsection*{Exercise 2(c)}
\begin{problem}
    \begin{equation}
        \sum_{n=1}^\infty (-1)^n n\left (\sqrt{4 - \frac{1}{n^2}} -2 \right)
    \end{equation}
\end{problem}

\begin{proposition}
    The series converges conditionally.
\end{proposition}

\begin{proof}
    For an arbitrary $n \in \N$, we derive the following equivalent expression:
    \begin{align*}
        n \left( \sqrt{4 - \frac{1}{n^2}} - 2 \right)
            &= n \left( \sqrt{4 - \frac{1}{n^2}} - 4 \right) \cdot \frac{\sqrt{4 - \frac{1}{n^2}} + 2}{\sqrt{4 - \frac{1}{n^2}} + 2} \\
            &= n \frac{4 - \frac{1}{n^2} - 4}{\sqrt{4 - \frac{1}{n^2}} + 2} \\
            &= \frac{1}{n} \cdot \frac{-1}{\sqrt{4 - \frac{1}{n^2}} + 2}.
    \end{align*}
    Now since
    \[\lim_{n \to \infty} \frac{1}{n} = 0,\]
    then
    \begin{align*}
        \lim_{n \to \infty} n \left( \sqrt{4 - \frac{1}{n}} - 2 \right) 
            &= \lim_{n \to \infty} \frac{1}{n} \cdot \frac{-1}{\sqrt{4 - \frac{1}{n^2}} + 2} \\
            &= 0 \cdot \frac{-1}{\sqrt{4 - 0} + 2} \\
            &= 0 \cdot \frac{-1}{4} \\
            &= 0.
    \end{align*}
    Thus, the alternating series (3) converges. However, for any $n \in \N$, we have
    \begin{align*}
        4 - \frac{1}{n^2} &\leq 4, \\
        \sqrt{4 - \frac{1}{n^2}} &\leq 2, \\
        \sqrt{4 - \frac{1}{n^2}} + 2 &\leq 4, \\
        \frac{1}{\sqrt{4 - \frac{1}{n^2}} + 2} &\geq \frac{1}{4}.
    \end{align*}
    Then looking at the absolute value of the $n$th term in the series, we find
    \begin{align*}
        \left| (-1)^n n\left (\sqrt{4 - \frac{1}{n^2}} -2 \right) \right|
            &= \left| n\left (\sqrt{4 - \frac{1}{n^2}} -2 \right) \right| \\
            &= \left| \frac{1}{n} \cdot \frac{-1}{\sqrt{4 - \frac{1}{n^2}} + 2} \right| \\
            &= \frac{1}{n} \cdot \frac{1}{\sqrt{4 - \frac{1}{n^2}} + 2} \\
            &\geq \frac{1}{n} \cdot \frac{1}{4}.
    \end{align*}
    Therefore, 
    \[\sum_{n=1}^\infty \left| (-1)^n n\left (\sqrt{4 - \frac{1}{n^2}} -2 \right) \right| \geq \sum_{n=1}^\infty \frac{1}{n} \cdot \frac{1}{4} = \frac{1}{4}\sum_{n=1}^\infty \frac{1}{n}.\]
    And since the harmonic series diverges, then so does the series of the absolute values of the terms of the series (3). Thus, the series converges, but not absolutely.
    
\end{proof}

\newpage
\subsection*{Exercise 2(d)}
\begin{problem}
    \begin{equation}
        \sum_{n=1}^\infty (-1)^n n\left (\sqrt{4 - \frac{1}{n^3}} -2 \right)
    \end{equation}
\end{problem}

\begin{proposition}
    The series converges absolutely.
\end{proposition}

\begin{proof}
    For an arbitrary $n \in \N$, we derive the following equivalent expression:
    \begin{align*}
        n \left( \sqrt{4 - \frac{1}{n^2}} - 2 \right)
            &= n \left( \sqrt{4 - \frac{1}{n^3}} - 4 \right) \cdot \frac{\sqrt{4 - \frac{1}{n^3}} + 2}{\sqrt{4 - \frac{1}{n^3}} + 2} \\
            &= n \frac{4 - \frac{1}{n^3} - 4}{\sqrt{4 - \frac{1}{n^3}} + 2} \\
            &= \frac{1}{n^2} \cdot \frac{-1}{\sqrt{4 - \frac{1}{n^3}} + 2}.
    \end{align*}
    We now derive the following inequality:
    \begin{align*}
        \frac{1}{n^3} &\leq 1, \\[1em]
        4 - \frac{1}{n^2} &\geq 3, \\[1em]
        \sqrt{4 - \frac{1}{n^2}} &\geq \sqrt{3}, \\[1em]
        \sqrt{4 - \frac{1}{n^2}} + 2 &\geq \sqrt{3} + 2, \\[1em]
        \frac{1}{\sqrt{4 - \frac{1}{n^2}} + 2} &\leq \frac{1}{\sqrt{3} + 2}.
    \end{align*}
    We now derive the following inequality:
    \begin{align*}
        \left| (-1)^n n\left (\sqrt{4 - \frac{1}{n^3}} -2 \right) \right|
            &= \left| n\left (\sqrt{4 - \frac{1}{n^3}} -2 \right) \right| \\
            &= \left| \frac{1}{n^2} \cdot \frac{-1}{\sqrt{4 - \frac{1}{n^3}} + 2} \right| \\
            &= \frac{1}{n^2} \cdot \frac{1}{\sqrt{4 - \frac{1}{n^3}} + 2} \\
            &\leq \frac{1}{n^2} \cdot \frac{1}{\sqrt{3} + 2}.
    \end{align*}
    Therefore,
    \[\sum_{n=1}^\infty \left| (-1)^n n\left (\sqrt{4 - \frac{1}{n^3}} -2 \right) \right| \leq \sum_{n=1}^\infty \frac{1}{n^2} \cdot \frac{1}{\sqrt{3} + 2} = \frac{1}{\sqrt{3} + 2} \sum_{n=1}^\infty \frac{1}{n^2}.\]
    Now, since the series
    \[\sum_{n=1}^\infty \frac{1}{n^2}\]
    converges, then so does the series of the absolute values of the terms of the series (4). Thus, the series is absolutely convergent.
    
\end{proof}



\end{document}