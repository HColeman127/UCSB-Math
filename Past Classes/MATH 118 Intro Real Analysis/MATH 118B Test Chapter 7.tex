\documentclass[12pt]{article}

% Packages
\usepackage[margin=1in]{geometry}
\usepackage{fancyhdr}
\usepackage{amsmath, amsthm, amssymb, physics}

% Page Style
\fancypagestyle{plain}{
    \fancyhf{}
    \renewcommand{\headrulewidth}{0pt}
    \renewcommand{\footrulewidth}{0pt}
    \fancyfoot[R]{\thepage}
}
\pagestyle{plain}

% Problem Box
\setlength{\fboxsep}{4pt}
\newsavebox{\savefullbox}
\newenvironment{fullbox}{\begin{lrbox}{\savefullbox}\begin{minipage}{\dimexpr\textwidth-2\fboxsep\relax}}{\end{minipage}\end{lrbox}\begin{center}\framebox[\textwidth]{\usebox{\savefullbox}}\end{center}}
\newenvironment{pbox}[1][]{\begin{fullbox}\ifx#1\empty\else\paragraph{#1}\fi}{\end{fullbox}}

% Options
\renewcommand{\thesubsection}{\thesection(\alph{subsection})}
\allowdisplaybreaks
\addtolength{\jot}{4pt}
\theoremstyle{definition}

% Default Commands
\newtheorem{proposition}{Proposition}
\newtheorem{lemma}{Lemma}
\newcommand{\ds}{\displaystyle}
\newcommand{\isp}[1]{\quad\text{#1}\quad}
\newcommand{\N}{\mathbb{N}}
\newcommand{\Z}{\mathbb{Z}}
\newcommand{\Q}{\mathbb{Q}}
\newcommand{\R}{\mathbb{R}}
\newcommand{\C}{\mathbb{C}}
\newcommand{\eps}{\varepsilon}
\renewcommand{\phi}{\varphi}
\renewcommand{\emptyset}{\varnothing}
\newcommand{\pfrac}[2]{\left(\frac{#1}{#2}\right)}

% Extra Commands


% Document Info
\fancypagestyle{title}{
    \renewcommand{\headrulewidth}{0.4pt}
    \setlength{\headheight}{15pt}
    \fancyhead[R]{Harry Coleman}
    \fancyhead[L]{MATH 118B Test Chapter 7}
    \fancyhead[C]{March 13, 2021}
}

% Begin Document
\begin{document}
\thispagestyle{title}


\begin{pbox}[1(a)]
    Prove that $\{x^n\}_{n\in \mathbb{N}}$ does not converge uniformly in $[0,1]$.
\end{pbox}

\begin{proof}
    Let $f_n = x^n$. For any $x \in [0, 1)$, we have $f_n(x) \to 0$ as $n \to \infty$. Additionally, $f_n(1) = 1$ for all $n \in \N$. So $f_n \to f$ pointwise on the interval $[0, 1]$, where $f$ is given by
    \[
        f(x) = \begin{cases}
            0 &\text{if } 0 \leq x < 1 \\
            1 &\text{if } x = 1.
        \end{cases}
    \]
    We will show that this convergence is not uniform. In particular, we take $\eps = 1/2$. Then for any $n \in \N$, consider the point $x = \sqrt[n]{1/2} \in [0, 1)$. We find
    \[
        |f_n(x) - f(x)|
            = \abs{\qty(\sqrt[n]{1/2})^n - 0}
            = \frac{1}{2}
            \geq \eps.
    \]
    Thus, the convergence is not uniform on $[0, 1]$.
    
\end{proof}

\begin{pbox}[1(b)]
    Let $g$ be continuous on $[0,1]$ with $g(1) = 0$. Show that $\{x^ng(x)\}$ converges uniformly in $[0,1]$. 
\end{pbox}

\begin{proof}
    Since $g$ is continuous on the closed interval $[0, 1]$, it is bounded; let $M = \sup_{[0, 1]}|g|$. We will show that $x^ng(x) \to 0$ uniformly on $[0, 1]$. Let $\eps > 0$ be given. By the continuity of $g$ at $1$, and since $g(1) = 0$, let $\delta > 0$ such that $|g(x)| < \eps$ for all $x \in (1 - \delta, 1]$. Then for all $n \in \N$ and $x \in (1 - \delta, 1]$, we have
    \[
        |x^ng(x)| = |x|^n|g(x)| \leq |g(x)| < \eps.
    \]
    Now since $1 - \delta \in [0, 1)$, then $(1 - \delta)^n \to 0$ as $n \to \infty$. Let $N \in \N$ such that $(1- \delta)^n < \eps/M$ for all $n \geq N$. Then if $n \geq N$ and $x \in [0, 1-\delta]$, we have
    \[
        |x^ng(x)| = |x|^n|g(x)| \leq (1-\delta)^nM < \frac{\eps}{M}M = \eps.
    \]
    Combining the two bounds, $|x^ng(x)| < \eps$ for all $n \geq N$ and $x \in [0, 1]$. Thus, the convergence is uniform on the interval $[0, 1]$
    
\end{proof}
    



\newpage
\begin{pbox}[2]
    Let $\{f_n\}$ be a uniformly bounded sequence of functions which are Riemann-integrable on $[a,b]$, and define
    \[
     F_n(x) = \int_a^x f_n(t)\,dt,\ \ \ a \le x \le b.
    \]
Prove that there exists a subsequence $\{F_{n_k}\}$ which converges uniformly on $[a,b]$.
\end{pbox}

\begin{proof}
    By the Arzela-Ascoli theorem, it suffices to prove that $\{F_n\}$ is a pointwise bounded and equicontinuous family of functions on the interval $[a, b]$.
    
    Suppose we have the uniform bound $|f_n(x)| \leq M$ for all $n \in \N$ and $x \in [a, b]$. Then we also have the uniform bound
    \[
        |F_n(x)|
            \leq \int_{a}^{x} |f_n(t)| \dd{t}
            \leq M(b - a).
    \]
    In particular, this is a pointwise bound for $\{F_n\}$ on $[a, b]$. Let $\eps > 0$ be given, and choose $\delta = \eps/M$. For all $n \in \N$ and $x, y \in [a, b]$, if $|x - y| < \delta$ then
    \[
        |F_n(x) - F_n(y)|
            = \abs{\int_{y}^{x} f_n(t) \dd{t}}
            \leq M|x - y|
            < M\frac{\eps}{M}
            = \eps.
    \]
    Thus, $\{F_n\}$ is equicontinuous on $[a, b]$.
    
\end{proof}




\begin{pbox}[3]
    Consider  the sequence $\{f_n\}_{n\in\N}$ defined by 
    \begin{equation}
    f_n(x) = \frac{n^{3/2}x}{1+n^2x^2},\quad x\in [0,1],\ n\in \N.
    \end{equation}
    Show that the sequence converges pointwise and find the limit. Determine whether or not it converges uniformly.
\end{pbox}

\begin{proof}
    First, note that $f_n(0) = 0$ for all $n \in \N$, so $f_n(0) \to 0$ as $n \to \infty$. Additionally, all $f_n$ are nonnegative on $[0, 1]$. Then for $x \in (0, 1]$, we find
    \[
        0 \leq f_n(x)
            = \frac{n^{3/2}x}{1 + n^2x^2}
            \leq \frac{n^{3/2}x}{n^2x^2}
            = \frac{1}{n^{1/2}x^2}.
    \]
    So by the squeeze theorem, $f_n \to 0$ pointwise for all $x \in [0, 1]$. We will show that this convergence is not uniform. In particular, we take $\eps = 1/2$. Then for any $n \in \N$, consider the point $x = 1/n \in (0, 1]$. We find
    \[
        |f_n(x) - 0|
            = \frac{n^{3/2}(1/n)}{1 + n^2(1/n)^2}
            = \frac{n^{1/2}}{1 + 1}
            \geq \frac{1}{2}
            = \eps.
    \]
    Thus, the convergence is not uniform on $[0, 1]$.
    
\end{proof}



\newpage
\begin{pbox}[4]
    Assume that the functions $f_n:[0,1]\to \R$, $n\in \N$, are continuous and that 
    \begin{equation}
    \sum_{n=1}^\infty f_n(x)
    \end{equation}
    converges uniformly on $[0,1)$. Show that the series 
    \begin{equation}
    \sum_{n=1}^\infty f_n(1)
    \end{equation}
    converges.
\end{pbox}

\begin{proof}
    Define the partial sums $F_n(x) = \sum_{k=1}^{n} f_k(x)$ for $x \in [0, 1]$, which converge uniformly to $F(x) = \sum_{n=1}^{\infty} f_n(x)$ on $[0, 1)$. Since each partial sum is continuous and the convergence is uniform, then we know $F$ is continuous on $[0, 1)$. To show that the series in question converges, it is equivalent to show that $\{F_n(1)\}$ is a Cauchy sequence.
    
    Let $\eps > 0$ be given. By the uniform convergence of $F_n \to F$ on $[0, 1)$ let $N \in \N$ such that
    \[
        n, m \geq N,\; x \in [0, 1) \implies |F_n(x) - F_m(x)| < \eps.
    \]
    For $n, m \in \N$, by the continuity of $F_n$ and $F_m$ at $1$, let $\delta_{n,m} > 0$ such that
    \[
        |F_n(x) - F_n(1)| < \eps \isp{and} |F_m(x) - F_m(1)| < \eps,
    \]
    for all $x \in (1 - \delta_{n,m}, 1]$. Then for all $n, m \geq N$, we choose $x_{n,m} \in (1 - \delta_{n,m}, 1)$ and find
    \begin{align*}
        |F_n(1) - F_m(1)|
            &\leq |F_n(1) - F_n(x_{n,m})| + |F_n(x_{n,m}) - F_m(x_{n,m})| + |F_m(x_{n,m}) - F_m(1)| \\
            &< \eps + \eps + \eps \\
            &= 3\eps.
    \end{align*}
    Thus, $\{F_n(1)\}$ is a Cauchy sequence, meaning that the series $\sum_{n=1}^{\infty} f_n(1)$ converges.
    
    
\end{proof}



\newpage
\begin{pbox}[5]
    Let $\{f_n\}_{n\in \N}$ be a sequence of functions $f_n:[a,b]\to \R$, differentiable  on $[a,b]$, and let $f:[a,b]\to \R$ be a function such that $f_n \to f$ pointwise in $[a,b]$. Assume that $\exists\,M>0$ such that 
    \begin{equation}
    |f'_n(x)|\le M,\quad \forall\,x\in [a,b],\ \forall\,n\in \N.
    \end{equation}
    Prove that $\{f_n\}_{n\in \N}$ converges uniformly to $f$ on $[a,b]$.
\end{pbox}

\begin{proof}
    First, for any $n \in \N$ and $x, y \in [a, b]$, the mean value theorem states that there exists some $c_{n,x,y}$ between $x$ and $y$ such that
    \[
        f_n(x) - f_n(y) = f_n'(c_{n,x,y})(x - y).
    \]
    And since $|f_n'| \leq M$, this gives us
    \[
        |f_n(x) - f_n(y)| \leq M|x - y|,
    \]
    for all $n \in \N$ and $x, y \in [a, b]$.
    
    We will prove that $\{f_n\}$ is a uniformly Cauchy sequence of functions. Let $\eps > 0$ be given. Choose $k \in \N$ such that $(b-a)/k < \eps$, and define the points $x_j = a + (b-a)j/k$ for $j = 0, 1, \dots, k$. This gives us a partition of the interval $[a, b]$ into $k$ subintervals $[x_j, x_{j+1}]$ of width less than $\eps$. For each $j$, $\{f_n(x_j)\}$ is a convergent, hence Cauchy, sequence. Let $N_j \in \N$ such that
    \[
        n, m \geq N_j \implies |f_n(x_j) - f_m(x_j)| < \eps,
    \]
    and define $N = \max\{N_0, N_1, \dots, N_k\}$. Let $n, m \geq N$ and $x \in [a, b]$. Assume $x$ is contained in the subinterval $[x_j, x_{j+1}]$, which must be true for some $j$ since the subintervals form a partition of $[a, b]$. Then we find
    \begin{align*}
        |f_n(x) - f_m(x)|
            &\leq |f_n(x) - f_n(x_j)| + |f_n(x_j) - f_m(x_j)| + |f_m(x_j) - f_m(x)| \\
            &< M|x - x_j| + \eps + M|x - x_j| \\
            &< M\eps + \eps + M\eps\\
            &= (2M + 1)\eps.
    \end{align*}
    Since $2M + 1$ is a constant, then this tells us that $\{f_n\}$ is in fact a uniformly Cauchy sequence. Hence, $f_n \to f$ uniformly on $[a, b]$.
    
\end{proof}



\end{document}