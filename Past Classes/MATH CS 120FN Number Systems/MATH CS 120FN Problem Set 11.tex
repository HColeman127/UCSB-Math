\documentclass[12pt]{article}
 
 \usepackage[table,xcdraw]{xcolor}
\usepackage[margin=1in]{geometry} 
\usepackage{amsmath,amsthm,amssymb}
\usepackage{listings}
\usepackage{tikz}
\usepackage{colortbl}
\usepackage{framed}

\lstset{basicstyle=\footnotesize}
\usetikzlibrary{calc}

\newcommand{\N}{\mathbb{N}}
\newcommand{\Z}{\mathbb{Z}}
\newcommand{\I}{\mathbb{I}}
\newcommand{\R}{\mathbb{R}}
\newcommand{\Q}{\mathbb{Q}}

 
\newenvironment{theorem}[2][Theorem]{\begin{trivlist}
\item[\hskip \labelsep {\bfseries #1}\hskip \labelsep {\bfseries #2.}]}{\end{trivlist}}
\newenvironment{lemma}[2][Lemma]{\begin{trivlist}
\item[\hskip \labelsep {\bfseries #1}\hskip \labelsep {\bfseries #2.}]}{\end{trivlist}}
\newenvironment{exercise}[2][Exercise]{\begin{trivlist}
\item[\hskip \labelsep {\bfseries #1}\hskip \labelsep {\bfseries #2.}]}{\end{trivlist}}
\newenvironment{problem}[2][Problem]{\begin{trivlist}
\item[\hskip \labelsep {\bfseries #1}\hskip \labelsep {\bfseries #2.}]}{\end{trivlist}}
\newenvironment{question}[2][Question]{\begin{trivlist}
\item[\hskip \labelsep {\bfseries #1}\hskip \labelsep {\bfseries #2.}]}{\end{trivlist}}
\newenvironment{corollary}[2][Corollary]{\begin{trivlist}
\item[\hskip \labelsep {\bfseries #1}\hskip \labelsep {\bfseries #2.}]}{\end{trivlist}}

\newenvironment{solution}{\begin{proof}[Solution]}{\end{proof}}
 
\begin{document}
 
\title{Problem Set 11\\
    \large MATH CS 120FN Number Systems}
\author{Harry Coleman}
\date{November 14, 2019}

\maketitle

\section*{Problem 1}
\fbox{
    \parbox{\textwidth}{
        Use the fact that $\cos(r)$ is irrational for any nonzero rational $r$ to conclude that $\sin(r)$ and $\tan(r)$ are also irrational for such $r$.
    }
}
\\

For this proof, we need to examine the properties of arithmetic operations on rational and irrational numbers. Let $x$ and $q$ be real numbers such that $x\in\I$ and $q \in\Q$. We want to show
\begin{align*}
    x + q &\in\I \\
    x \cdot q &\in\I \\
    x^{-1} &\in\I \text{ (for $x\ne0$)}\\
    \sqrt{x} &\in\I \text{ (for $x>0$)}
\end{align*}
\begin{center}
    \begin{tabular}{l|l}
        Assume $x+q \notin\I$           & Assume $x\cdot q \notin\I$ \\
        $x+q \in\Q$                     & $x \cdot q \in\Q$ \\
        $x+q = p$ for some $p\in\Q$     & $x \cdot q = p$ for some $p\in\Q$ \\
        $x = p-q$                       & $x = p \div q$ \\
        $x\in\Q$ since $p-q \in\Q$      & $x\in\Q$ since $p\div q \in\Q$ \\
        \hline
        For $x\ne0$                     & For $x>0$ \\
        Assume $x^{-1} \notin\I$        & Assume $\sqrt{x} \notin\I$ \\
        $x^{-1} \in\Q$                  & $\sqrt{x} \in\Q$ \\
        $x \cdot p=1$ for some $p\in\Q$ & $\sqrt{x} = p$ for some $p\in\Q$ \\
        $x = 1\div p$                   & $x = p^2$ \\
        $x \in\Q$ since $1\div p\in\Q$  & $x\in\Q$ since $p^2\in\Q$ \\
    \end{tabular}
\end{center}

Each of these contradict that $x\in\I$, so we have shown all of the above properties. \newpage

We'll let $r$ be some rational number. With the half-angle trig identity,
\[\sin\left(\frac{x}{2}\right) = \sqrt{\frac{1-\cos(x)}{2}}\]
\[\sin(x) = \sqrt{\frac{1-\cos(2x)}{2}}\]
With this and the properties we showed at the beginning,
\begin{align*}
    2r &\in\Q \\
    \cos(2r) &\in\I \\
    1 - \cos(2r) &\in\I \\
    \frac{1-\cos(2r)}{2} &\in\I \\
    \sqrt{\frac{1-\cos(2r)}{2}} &\in\I \\
    \sin(r) &\in\I
\end{align*}

With the Pythagorean identity and the half-angle trig identities,
\[\tan^2(x) + 1 = \sec^2(x)\]
\[\tan(x) = \sqrt{\frac{1}{\cos^2(x)} - 1}\]
\[\tan(x) = \sqrt{\frac{2}{1 + \cos(2x)} - 1}\]
With this and the properties we showed at the beginning,
\begin{align*}
    2r &\in\Q \\
    \cos(2r) &\in\I \\
    1 + \cos(2r) &\in\I \\
    \frac{2}{1 + \cos(2r)} &\in\I \\
    \frac{2}{1 + \cos(2r)} - 1 &\in\I \\
    \sqrt{\frac{2}{1 + \cos(2r)} - 1} &\in\I \\
    \tan(r) &\in\I
\end{align*}




        
\end{document}