\documentclass[12pt]{article}
 
\usepackage[margin=1in]{geometry} 
\usepackage{amsmath,amsthm,amssymb}
\usepackage{listings}

\lstset{basicstyle=\footnotesize}

 
\newcommand{\N}{\mathbb{N}}
\newcommand{\Z}{\mathbb{Z}}

 
\newenvironment{theorem}[2][Theorem]{\begin{trivlist}
\item[\hskip \labelsep {\bfseries #1}\hskip \labelsep {\bfseries #2.}]}{\end{trivlist}}
\newenvironment{lemma}[2][Lemma]{\begin{trivlist}
\item[\hskip \labelsep {\bfseries #1}\hskip \labelsep {\bfseries #2.}]}{\end{trivlist}}
\newenvironment{exercise}[2][Exercise]{\begin{trivlist}
\item[\hskip \labelsep {\bfseries #1}\hskip \labelsep {\bfseries #2.}]}{\end{trivlist}}
\newenvironment{problem}[2][Problem]{\begin{trivlist}
\item[\hskip \labelsep {\bfseries #1}\hskip \labelsep {\bfseries #2.}]}{\end{trivlist}}
\newenvironment{question}[2][Question]{\begin{trivlist}
\item[\hskip \labelsep {\bfseries #1}\hskip \labelsep {\bfseries #2.}]}{\end{trivlist}}
\newenvironment{corollary}[2][Corollary]{\begin{trivlist}
\item[\hskip \labelsep {\bfseries #1}\hskip \labelsep {\bfseries #2.}]}{\end{trivlist}}

\newenvironment{solution}{\begin{proof}[Solution]}{\end{proof}}
 
\begin{document}
 
% --------------------------------------------------------------
%                         Start here
% --------------------------------------------------------------
 
\title{CS120FN Problem Set 1}
\author{Harry Coleman}
\date{October 1, 2019}

\maketitle

\section{Set of Pythagorean Triples with the Same $c$ Value}

\begin{lstlisting}[language=Python, frame=single, basicstyle=\footnotesize]
import math

HIGH = 10000
triples_dict = {}

for s in range(1, HIGH, 2):
	for t in range(1, s, 2):
		if math.gcd(s, t) == 1:
			a = s*t
			b = int((s**2 - t**2)/2)
			c = int((t**2 + s**2)/2)

			if c in triples_dict:
				triples_dict[c].append((a, b, c))
			else:
				triples_dict[c] = [(a, b, c)]

triples = list(triples_dict.values())
best = sorted(triples, key=lambda k: len(k), reverse=True)[0]

print(len(best))
for triple in best:
	print(triple)
\end{lstlisting}

The above program iterates through all odd, coprime values for $s$ and $t$ to generate primitive Pythagorean triples(PPT) of the form:

\begin{equation}
    (a, b, c) = \left(st, \frac{s^2 - t^2}{2}, \frac{s^2 + t^2}{2}\right)
\end{equation}

Every triple generated is added to a dictionary, indexed by the value of $c$. After all triples are generated, the largest collection of triples with the same $c$ value is displayed. Running the above program yields the following collection of 32 PPT's, all with $c = 48612265$.
\newline\newline
$
(1547863, 48587616, 48612265),
(4206377, 48429936, 48612265),
(5031817, 48351144, 48612265),\\
(6365833, 48193656, 48612265),
(10400983, 47486544, 48612265),
(14399273, 46430736, 48612265),\\
(15368407, 46119024, 48612265),
(16162697, 45846696, 48612265),
(19412183, 44568144, 48612265),\\
(21173033, 43759056, 48612265),
(24544343, 41961024, 48612265),
(26394487, 40822584, 48612265),\\
(28590473, 39315864, 48612265),
(29737897, 38455296, 48612265),
(32898647, 35788704, 48612265),\\
(33410167, 35311656, 48612265),
(33640873, 35091936, 48612265),
(37350007, 31114776, 48612265),\\
(37875287, 30473184, 48612265),
(39487273, 28353264, 48612265),
(39859337, 27827784, 48612265),\\
(42674807, 23281176, 48612265),
(43107703, 22469496, 48612265),
(43568777, 21561864, 48612265),\\
(45314953, 17598504, 48612265),
(46580137, 13908384, 48612265),
(47077513, 12118584, 48612265),\\
(47826007, 8707776, 48612265),
(47937143, 8073576, 48612265),
(48048343, 7383024, 48612265),\\
(48577417, 1840344, 48612265),
(48605303, 822696, 48612265)
$

\section{Patterns in $a$, $b$, and $c$}
\subsection{Which odd numbers $a$ can appear in a PPT}

Using the same PPT formula as the previous section, we know that a valid value for $a$ can be found using the equation $a=st$ where $s$ and $t$ are both odd, coprime integers. If we take $s$ to be one, then $t$ can be any odd integer, since no number will share any factors with 1, other than 1 itself. Since $t$ can be any odd integer when $s=1$, $a=st$ can be any odd integer.

\subsection{Which even numbers $b$ can appear in a PPT}
From observation, $b$ is always a multiple of 4, and covers all multiples of 4.

\subsection{Which numbers $c$ can appear in a PPT}
From observation, $c$ is always 1 more than a perfect square, and covers all such numbers.

\section{PPT's for which $c=a+2$}
In order to generalize all PT in which $c=a+2$, we first substitute into the Pythagorean equation:
\[
    a^2+b^2=(a+2)^2
\]
We solve this equation for $a$.
\begin{equation}
a = \frac{b^2 - 4}{4}
\end{equation}
Keeping this equation in mind, we can show for the specific instance pf $c=a+2$ that $b$ must be a multiple of 4. First we will assume that b is some even integer that is not divisible by 4.
\begin{equation}
    b=4h+2
\end{equation}
We will substitute this into the previous equation solved for $a$.
\[
    a=\frac{(4h+2)^2-4}{4}
\]
Simplified.
\begin{equation}
    a=4(h^2+h)
\end{equation}
Equation (4) here shows that $a$ is a multiple of 4, and therefore even. However, this contradicts 2.1, in which $a$ is shown to be an odd integer. Therefore, our initial assumption that $b$ is an even integer not divisible by 4 is false, and $b$ must be divisible by 4.
\begin{equation}
    b=4h
\end{equation}
Where $h$ is some positive integer. We rewrite equation (2) using this substitution.
\[
    a = \frac{(4h)^2 - 4}{4}
\]
Simplified.
\begin{equation}
a = 4h^2 - 1
\end{equation}
We substitute equation (6) into $c=a+2$ to solve for $c$:
\[
c=(4h^2 - 1) +2
\]
\begin{equation}
    c=4h^2+1
\end{equation}
Taking equations (3), (6), and (6) we know that all PT‘s such that $c=a+2$ are of the form:
\[
(4h^2-1, 4h, 4h^2+1)
\]

\section{$2c - 2a$ for values of $a$ and $c$ from PPT's}

Given that the values for $a$ and $c$ are from a PPT, we can use the following:

\begin{equation}
a = st
\end{equation}

\begin{equation}
c = \frac{s^2 + t^2}{2}
\end{equation}
We substitute these into $2c - 2a$:
\begin{equation}
    2\left(\frac{s^2 + t^2}{2}\right) - 2(st)
\end{equation}
\[
s^2 + t^2 - 2st
\]
\[
s^2 - 2st + t^2
\]
\[
(s - t)^2
\] 

Because we know $s$ and $t$ are integers, the difference must also be an integer. This value is squared, resulting in a perfect square. Thus, for any given PPT, the quantity $2c-2a$ is a perfect square. 

\end{document}
