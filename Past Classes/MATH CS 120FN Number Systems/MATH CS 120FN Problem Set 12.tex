\documentclass[12pt]{article}

% packages
\usepackage[margin=1in]{geometry}
\usepackage{framed}
\usepackage[table]{xcolor}
\usepackage{colortbl, multirow}
\usepackage{amsmath,amsthm,amssymb,wasysym}
\usepackage{mathrsfs, mathtools}
\usepackage{tikz,pgf,pgfplots}
\usetikzlibrary{arrows, angles, quotes, decorations.pathreplacing, math, patterns, calc}
\usepackage{graphicx}

% custom commands
\newcommand{\N}{\mathbb{N}}
\newcommand{\Z}{\mathbb{Z}}
\newcommand{\I}{\mathbb{I}}
\newcommand{\R}{\mathbb{R}}
\newcommand{\Q}{\mathbb{Q}}
\newcommand{\p}{^{\prime}}
\DeclarePairedDelimiter{\ceil}{\lceil}{\rceil}
\DeclarePairedDelimiter\floor{\lfloor}{\rfloor}

 
\begin{document}
 
\title{Problem Set 12\\
    \large MATH CS 120FN Number Systems}
\author{Harry Coleman}
\date{December 5, 2019}

\maketitle

\section*{Problem 1}
\fbox{
    \parbox{\textwidth} {
        If $A$ is a Dedekind cut, recall that we attempt to define its additive inverse by
        \[A' = \{q\in\Q \mid \exists s\in\Q \text{ with $s>q$ and $a+s<0$ for all $a\in A$.}\}\]
        Complete the proof that $A'$ is a Dedekind cut and that in the proof that $0 \subseteq A + A'$, that the $q$ we use is indeed in $A'$. 
    }
}
\\

\subsection*{1 Show $A'$ is a Dedekind Cut}
\subsubsection*{1.1 Show $A' \ne \emptyset$}
Since $A$ is a dedekind cut, $A\ne\Q$. So there exists some rational number, say $b$, such that $b\notin A$. Since $A$ is closed downward, we know that $\forall a\in A(a<b)$, equivalently, $\forall a\in A(a+(-b)<0)$. We now consider the values $(-b-1)\in\Q$ and $(-b)\in\Q$; clearly, $(-b-1)<(-b)$. So we have
\begin{align*}
    &(-b-1)\in\Q, \\
    &(-b)\in\Q, \\
    &(-b-1)<(-b), \\
    &\text{and } \forall a\in A(a+(-b)<0).
\end{align*}
Therefore, $(-b-1)\in A'$, so $A'\ne\emptyset$.

\subsubsection*{1.2 Show $A' \ne \Q$}
Let $b\in A$, so $(-b)\in\Q$. We'll assume that $A'=\Q$, so $(-b)\in A'$. Now, we'll let $s\in\Q$ with $(-b)<s$ and $\forall a\in A(a+s<0)$. So $0<b+s$ and since $b\in A$, $b+s<0$. This is a contradiction, therefore $A'\ne\Q$.

\subsubsection*{1.3 Show $A'$ Closed Downward}
Let $a'\in A'$ and $q\in\Q$ with $q<a'$. We'll now let $s\in\Q$ with $a'<s$ and $\forall a\in A(a+s<0)$. We have that $q<a'<s$, therefore $q<s$ so $q\in A'$. This shows that
\[\forall a'\in A', q\in\Q(q<a' \implies q\in A').\]
So $A'$ is closed downward.

\subsubsection*{1.4 Show $A'$ Has No Largest Element}
Let $a'\in A'$, and $s\in\Q$ with $a'<s$ and $\forall a\in A(a+s<0)$. We now want let
\[c=\frac{a'+s}{2} \in\Q\]
With $a'<s$, we find
\[a'+a'<a'+s.\]
\[a'+s<s+s,\]
\[2a' < a'+s < 2s,\]
\[\text{and } a' < \frac{a'+s}{2} < s.\]
\[c<s\]
Therefore, $c\in A'$. This shows that
\[\forall a'\in A',\exists q\in A'(a'<q).\]
So $A'$ has no largest element.

\subsubsection*{1.5 Conclusion}
We have shown that for the given definition of $A'$,
\begin{enumerate}
    \item $A'\ne \emptyset, A'\ne\Q$,
    \item $A'$ is closed downward, and
    \item $A'$ has no largest element.
\end{enumerate}

Therefore, $A'$ is a Dedekind Cut.

\subsection*{2 Show $A+A'=0$}
\subsubsection*{2.1 Show $A+A'\subseteq 0$}
Let $a+a'\in A+A'$ where $a\in A$ and $a'\in A'$. Now let $s\in\Q$ with $a'<s$ and $\forall x\in A(x+s<0)$. Which tells us $a+s<0$ and $a+a'<a+s$. So $a+a'<0$, therefore $a+a'\in 0$ (referring to the Dedekind cut of 0). So $A+A'\subseteq 0$.

\subsubsection*{2.2 Show $0 \subseteq A+A'$}
I did not budget my time well enough to type my solution to this section. The idea is that we would pick some $t<0$, then pick some $\alpha$ in $A$. Then add multiples of $\frac{-t}{2}$ to $\alpha$, $\alpha_n = n\frac{-t}{2}$ until we found a $k$ with $a_k\in A$ and $a_{k+2}\notin A$. Then we would have $-a_{k+2}\in A'$. So we would have an $a+a'=t$, which would give us $t\in A+A'$.

\subsubsection{2.3 Conclusion}
Having shown that $A+A'\subseteq 0$, and if we had shown $0 \subseteq A+A'$, we would have that $A+A'=0$. And therefore, our definition of $A'$ would be the Dedekind cut additive inverse of $A$.





\end{document}