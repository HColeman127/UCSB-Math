\documentclass[12pt]{article}
 
 \usepackage[table,xcdraw]{xcolor}
\usepackage[margin=1in]{geometry} 
\usepackage{amsmath,amsthm,amssymb}
\usepackage{listings}
\usepackage{tikz}
\usepackage{colortbl}

\lstset{basicstyle=\footnotesize}
\usetikzlibrary{calc}

\newcommand{\N}{\mathbb{N}}
\newcommand{\Z}{\mathbb{Z}}
\newcommand{\I}{\mathbb{I}}
\newcommand{\R}{\mathbb{R}}
\newcommand{\Q}{\mathbb{Q}}

 
\newenvironment{theorem}[2][Theorem]{\begin{trivlist}
\item[\hskip \labelsep {\bfseries #1}\hskip \labelsep {\bfseries #2.}]}{\end{trivlist}}
\newenvironment{lemma}[2][Lemma]{\begin{trivlist}
\item[\hskip \labelsep {\bfseries #1}\hskip \labelsep {\bfseries #2.}]}{\end{trivlist}}
\newenvironment{exercise}[2][Exercise]{\begin{trivlist}
\item[\hskip \labelsep {\bfseries #1}\hskip \labelsep {\bfseries #2.}]}{\end{trivlist}}
\newenvironment{problem}[2][Problem]{\begin{trivlist}
\item[\hskip \labelsep {\bfseries #1}\hskip \labelsep {\bfseries #2.}]}{\end{trivlist}}
\newenvironment{question}[2][Question]{\begin{trivlist}
\item[\hskip \labelsep {\bfseries #1}\hskip \labelsep {\bfseries #2.}]}{\end{trivlist}}
\newenvironment{corollary}[2][Corollary]{\begin{trivlist}
\item[\hskip \labelsep {\bfseries #1}\hskip \labelsep {\bfseries #2.}]}{\end{trivlist}}

\newenvironment{solution}{\begin{proof}[Solution]}{\end{proof}}
 
\begin{document}
 
\title{Problem Set 7,8\\
    \large CS120FN Number Systems}
\author{Harry Coleman}
\date{October 31, 2019}

\maketitle

\section*{Problem 7.1}
\fbox{
    \parbox{\textwidth}{
        Let $p$ be an odd prime and let $g$ be a primitive root modulo $p$. Prove that $a$ has a square root modulo $p$ if and only if its discrete logarithm log$_g(a)$ modulo $(p - 1)$ is even.
    }
}
\\

Since we need to prove a biconditional, we must do this proof in two parts. We'll first assume that $a$ has a square root modulo $p$, and show that the second condition follows. So we have
\[a \equiv n^2 \pmod{p}\]

for some integers $a$ and $n$ and an odd prime $p$. Since $g$ is a primitive root, then for some integer $k \in \{1,2,\dots,p-1\}$, we can have
\[n \equiv g^k \pmod{p}\]
from which we can get
\[a \equiv (g^k)^2 \equiv g^{2k} \pmod{p}\]
and apply the discrete logarithm.

\[\log_g a = 2k \pmod{p-1}\]

Since $p-1$ is even, and $2k$ is even, then $2k$ modulo $(p-1)$ is even. This proves the conditional in one direction. For the other, we'll first assume that 
\[\log_g a \equiv 2n \pmod{p-1}\]
for some integers $a$ and $n$. By the definition of modulus,
\[\log_g a = (p-1)x + 2n\]
for some integer $x$. So
\begin{align*}
    a &= g^{(p-1)x + 2n} \\
      &= g^{(p-1)x}g^{2n} \\
      &= (g^{p-1})^x(g^n)^2
\end{align*}

If we look at this in modulo $p$, we get
\[a \equiv (g^{p-1})^x(g^n)^2 \pmod{p}\]
and since gcd$(p,g) = 1$, we can use Fermat's Little Theorem to get
\begin{align*}
    a &\equiv 1^x(g^n)^2 \pmod{p} \\
    a &\equiv (g^n)^2 \pmod{p}
\end{align*}
So $a$ is a perfect square modulo $p$. This proves both directions of the biconditional.


\section*{Problem 7.2}
\fbox{
    \parbox{\textwidth}{
        Suppose that the message $m$ is encrypted using the RSA cryposystem using three public keys, (391, 3), (55, 3) and (87, 3). The resulting ciphertexts are 208, 38 and 32 respectively. Determine $m$ (without using the easy-to-find prime factors of 391, 55 and 87.)
    }
}
\\

We have
\begin{align*}
    m^3 &\equiv 208 \pmod{391} \\
    m^3 &\equiv 38 \pmod{55} \\
    m^3 &\equiv 32 \pmod{87}
\end{align*}
If we can find integers $a,b,c$ such that
\begin{align*}
    a &\equiv 208 & b &\equiv 0 & c &\equiv 0 & \pmod{391} \\
    a &\equiv 0 & b &\equiv 38 & c &\equiv 0 & \pmod{55} \\ 
    a &\equiv 0 & b &\equiv 0 & c &\equiv 32 & \pmod{87} 
\end{align*}
then we could have
\begin{align*}
    a+b+c &\equiv 208 \pmod{391} \\
    a+b+c &\equiv 38 \pmod{55} \\
    a+b+c &\equiv 32 \pmod{87}
\end{align*}

To find such values, we must find integer solutions to
\begin{alignat*}{3}
    a &= 55\cdot87 x_a &&= 391y_a + 208 \\
    b &= 391\cdot87 x_b &&= 55y_b + 38 \\
    c &= 391\cdot55 x_c &&= 87y_c + 32
\end{alignat*}

Using the Euclidean Algorithm, we can find
\begin{align*}{3}
    a = 775170 && b = 1156578 && c = 43010
\end{align*}
\[a+b+c = 1974758\]

So we know that
\[m^3 = 391\cdot55\cdot87z + 1974758\]
for some integer $z$. Just guessing for integer solutions, we find $z=-1$ gives us $m=47$.




\section*{Problem 7.3}
\fbox{
    \parbox{\textwidth}{
        Suppose Alice wants to use the RSA cryptosystem to receive message $m$ from both Bob 1 and Bob 2. She picks a modulus $n$, and suppose she gives Bob 1 the encryption exponent $e_1$ and Bob 2 the encryption exponent $e_2$. Moreover, suppose gcd$(e_1, e_2) = 1$. If Bob 1 then sends Alice $c_1$ and Bob 2 sends Alice $c_2$, show how Eve can use the values of $c_1, c_2, e_1, e_2$ and $n$ to figure out $m$.
    }
}
\\

We have
\begin{align*}
    c_1 &\equiv m^{e_1} \pmod{n} \\
    c_2 &\equiv m^{e_2} \pmod{n}
\end{align*}

Since gcd$(e_1, e_2)=1$, we can find integers $x,y$ such that
\[e_1x + e_2y = 1\]
using the Euclidean Algorithm and back substitution. With these values we can find
\begin{alignat*}{3}
    c_1^xc_2^y &\equiv (m^{e_1})^x(m^{e_2})^y &\pmod{n} \\
               &\equiv m^{e_1x}m^{e_2y} &\pmod{n} \\
               &\equiv m^{e_1x + e_2y} &\pmod{n} \\
               &\equiv m^1 &\pmod{n} \\
               &\equiv m &\pmod{n}
\end{alignat*}

Since $m \in \{1,2,3,\dots,n-1\}$, this value we find is the exact value for $m$.

\newpage
\section*{Problem 8.1}
\fbox{
    \parbox{\textwidth}{
        Find an irreducible polynomial $p(x)$ of degree $3$ in $\mathbb{Z}_2[x]$ and use it to construct a finite field with $8$ elements. (That is, provide the addition and multiplication tables for this field).
    }
}
\\


We'll let
\[g(x) = x^3 + x + 1\]
and use it to construct the finite field
\[\Z_2[x]/g(x)\]
with elements
\[\{(0),(1),(x),(x+1),(x^2),(x^2+1),(x^2+x),(x^2+x+1)\}.\]

\begin{scriptsize}
    \begin{center}
        \begin{tabular}{
            >{\columncolor[HTML]{D6D6D6}}c|c
            >{\columncolor[HTML]{D6D6D6}}c c
            >{\columncolor[HTML]{D6D6D6}}c c
            >{\columncolor[HTML]{D6D6D6}}c c
            >{\columncolor[HTML]{D6D6D6}}c }
            addition                        & \cellcolor[HTML]{EFEFEF}$0$       & $1$                               & \cellcolor[HTML]{EFEFEF}$x$       & $x+1$                             & \cellcolor[HTML]{EFEFEF}$x^2$     & $x^2+1$                           & \cellcolor[HTML]{EFEFEF}$x^2+x$   & $x^2+x+1$                         \\
            \hline
            \cellcolor[HTML]{EFEFEF}$0$     & $0$                               & \cellcolor[HTML]{EFEFEF}$1$       & $x$                               & \cellcolor[HTML]{EFEFEF}$x+1$     & $x^2$                             & \cellcolor[HTML]{EFEFEF}$x^2+1$   & $x^2+x$                           & \cellcolor[HTML]{EFEFEF}$x^2+x+1$ \\
            $1$                             & \cellcolor[HTML]{EFEFEF}$1$       & $0$                               & \cellcolor[HTML]{EFEFEF}$x+1$     & $x$                               & \cellcolor[HTML]{EFEFEF}$x^2+1$   & $x^2$                             & \cellcolor[HTML]{EFEFEF}$x^2+x+1$ & $x^2+x$                           \\
            \cellcolor[HTML]{EFEFEF}$x$     & $x$                               & \cellcolor[HTML]{EFEFEF}$x+1$     & $0$                               & \cellcolor[HTML]{EFEFEF}$1$       & $x^2+x$                           & \cellcolor[HTML]{EFEFEF}$x^2+x+1$ & $x^2$                             & \cellcolor[HTML]{EFEFEF}$x^2+1$   \\
            $x+1$                           & \cellcolor[HTML]{EFEFEF}$x+1$     & $x$                               & \cellcolor[HTML]{EFEFEF}$1$       & $0$                               & \cellcolor[HTML]{EFEFEF}$x^2+x+1$ & $x^2+x$                           & \cellcolor[HTML]{EFEFEF}$x^2+1$   & $x^2$                             \\
            \cellcolor[HTML]{EFEFEF}$x^2$   & $x^2$                             & \cellcolor[HTML]{EFEFEF}$x^2+1$   & $x^2+x$                           & \cellcolor[HTML]{EFEFEF}$x^2+x+1$ & $0$                               & \cellcolor[HTML]{EFEFEF}$1$       & $x$                               & \cellcolor[HTML]{EFEFEF}$x+1$     \\
            $x^2+1$                         & \cellcolor[HTML]{EFEFEF}$x^2+1$   & $x^2$                             & \cellcolor[HTML]{EFEFEF}$x^2+x+1$ & $x^2+x$                           & \cellcolor[HTML]{EFEFEF}$1$       & $0$                               & \cellcolor[HTML]{EFEFEF}$x+1$     & $x$                               \\
            \cellcolor[HTML]{EFEFEF}$x^2+x$ & $x^2+x$                           & \cellcolor[HTML]{EFEFEF}$x^2+x+1$ & $x^2$                             & \cellcolor[HTML]{EFEFEF}$x^2+1$   & $x$                               & \cellcolor[HTML]{EFEFEF}$x+1$     & $0$                               & \cellcolor[HTML]{EFEFEF}$1$       \\
            $x^2+x+1$                       & \cellcolor[HTML]{EFEFEF}$x^2+x+1$ & $x^2+x$                           & \cellcolor[HTML]{EFEFEF}$x^2+1$   & $x^2$                             & \cellcolor[HTML]{EFEFEF}$x+1$     & $x$                               & \cellcolor[HTML]{EFEFEF}$1$       & $0$                              
        \end{tabular}
    \end{center}
\end{scriptsize}


\begin{scriptsize}
    \begin{center}
        \begin{tabular}{
            >{\columncolor[HTML]{D6D6D6}}c|c
            >{\columncolor[HTML]{D6D6D6}}c c
            >{\columncolor[HTML]{D6D6D6}}c c
            >{\columncolor[HTML]{D6D6D6}}c c
            >{\columncolor[HTML]{D6D6D6}}c }
            multiplication                   & \cellcolor[HTML]{EFEFEF}$0$ & $1$                             & \cellcolor[HTML]{EFEFEF}$x$       & $x+1$                             & \cellcolor[HTML]{EFEFEF}$x^2$   & $x^2+1$                           & \cellcolor[HTML]{EFEFEF}$x^2+x$   & $x^2+x+1$                       \\
            \hline
            \cellcolor[HTML]{EFEFEF}$0$     & $0$                         & \cellcolor[HTML]{EFEFEF}$0$     & $0$                               & \cellcolor[HTML]{EFEFEF}$0$       & $0$                             & \cellcolor[HTML]{EFEFEF}$0$       & $0$                               & \cellcolor[HTML]{EFEFEF}$0$     \\
            $1$                             & \cellcolor[HTML]{EFEFEF}$0$ & $1$                             & \cellcolor[HTML]{EFEFEF}$x$       & $x+1$                             & \cellcolor[HTML]{EFEFEF}$x^2$   & $x^2+1$                           & \cellcolor[HTML]{EFEFEF}$x^2+x$   & $x^2+x+1$                       \\
            \cellcolor[HTML]{EFEFEF}$x$     & $0$                         & \cellcolor[HTML]{EFEFEF}$x$     & $x^2$                             & \cellcolor[HTML]{EFEFEF}$x^2+x$   & $x^2+1$                         & \cellcolor[HTML]{EFEFEF}$x^2+x+1$ & $1$                               & \cellcolor[HTML]{EFEFEF}$x+1$   \\
            $x+1$                           & \cellcolor[HTML]{EFEFEF}$0$ & $x+1$                           & \cellcolor[HTML]{EFEFEF}$x^2+x$   & $x^2+1$                           & \cellcolor[HTML]{EFEFEF}$1$     & $x$                               & \cellcolor[HTML]{EFEFEF}$x^2+x+1$ & $x^2$                           \\
            \cellcolor[HTML]{EFEFEF}$x^2$   & $0$                         & \cellcolor[HTML]{EFEFEF}$x^2$   & $x^2+1$                           & \cellcolor[HTML]{EFEFEF}$1$       & $x^2+x+1$                       & \cellcolor[HTML]{EFEFEF}$x+1$     & $x$                               & \cellcolor[HTML]{EFEFEF}$x^2+x$ \\
            $x^2+1$                         & \cellcolor[HTML]{EFEFEF}$0$ & $x^2+1$                         & \cellcolor[HTML]{EFEFEF}$x^2+x+1$ & $x$                               & \cellcolor[HTML]{EFEFEF}$x+1$   & $x^2+x$                           & \cellcolor[HTML]{EFEFEF}$x^2$     & $1$                             \\
            \cellcolor[HTML]{EFEFEF}$x^2+x$ & $0$                         & \cellcolor[HTML]{EFEFEF}$x^2+x$ & $1$                               & \cellcolor[HTML]{EFEFEF}$x^2+x+1$ & $x$                             & \cellcolor[HTML]{EFEFEF}$x^2$     & $x+1$                             & \cellcolor[HTML]{EFEFEF}$x^2+1$ \\
            $x^2+x+1$                       & \cellcolor[HTML]{EFEFEF}$0$ & $x^2+x+1$                       & \cellcolor[HTML]{EFEFEF}$x+1$     & $x^2$                             & \cellcolor[HTML]{EFEFEF}$x^2+x$ & $1$                               & \cellcolor[HTML]{EFEFEF}$x^2+1$   & $x$                            
            \end{tabular}
    \end{center}
\end{scriptsize}



\section*{Problem 8.2}
\fbox{
    \parbox{\textwidth}{
        For the field you constructed above, find an element $g$ such that $(g,g^2,g^3,\ldots,g^8)$ is a list of all elements in the field.
    }
}
\\
We cannot obtain 0 by any multiplication other than by 0, so we can find a value which obtains all elements but 0 through exponentiation.;
\begin{align*}
    x^1 &\equiv x \\
    x^2 &\equiv x^2 \\
    x^3 &\equiv x^2+1 \\
    x^4 &\equiv x^2 + x + 1 \\
    x^5 &\equiv x + 1 \\
    x^6 &\equiv x^2 + x \\
    x^7 &\equiv 1 \\
\end{align*}


\end{document}

