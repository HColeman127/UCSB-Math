\documentclass[12pt]{article}

% packages
\usepackage{kantlipsum}
\usepackage[margin=1in]{geometry}
\usepackage[labelfont=it]{caption}
\usepackage[table]{xcolor}
\usepackage{subcaption,framed,colortbl,multirow}
\usepackage{amsmath,amsthm,amssymb,wasysym,mathrsfs,mathtools}
\usepackage{tikz,graphicx,pgf,pgfplots}
\usetikzlibrary{arrows, angles, quotes, decorations.pathreplacing, math, patterns, calc}
\pgfplotsset{compat=1.16}

% custom commands
\newcommand{\N}{\mathbb{N}}
\newcommand{\Z}{\mathbb{Z}}
\newcommand{\I}{\mathbb{I}}
\newcommand{\R}{\mathbb{R}}
\newcommand{\Q}{\mathbb{Q}}
\newcommand{\C}{\mathbb{C}}
\newcommand{\F}{\mathbb{F}}
\newcommand{\p}{^{\prime}}
\newcommand{\powerset}{\raisebox{.15\baselineskip}{\Large\ensuremath{\wp}}}
\DeclarePairedDelimiter{\ceil}{\lceil}{\rceil}
\DeclarePairedDelimiter\floor{\lfloor}{\rfloor}

\newcommand{\exercise}[2]{\section*{Exercise #1}\framebox{\begin{minipage}{\textwidth}#2\end{minipage}}\par\vspace{1em}}
\newcommand{\problem}[2]{\section*{Problem #1}\framebox{\begin{minipage}{\textwidth}#2\end{minipage}}\par\vspace{1em}}

 
\begin{document}
 
\title{Homework 6\\
    \large MATH CS 108A Linear Algebra I}
\author{Harry Coleman}
\date{January 22, 2020}

\maketitle


\exercise{1}{
    Is
    \[A =
        \begin{bmatrix}
            2 & 1 & 2 & 1  \\
            1 & 0 & 1 & 2  \\
            3 & 2 & 1 & -1 \\
        \end{bmatrix}
    \]
    row equivalent to a matrix in echelon form?
}

We will apply a series of elementary row operations to $A$ in order to obtain a matrix in echelon form. That matrix will, by definition, be row-equivalent to $A$.

\[e_{12}(A) =
    \begin{bmatrix}
        1 & 0 & 1 & 2  \\
        2 & 1 & 2 & 1  \\
        3 & 2 & 1 & -1 \\
    \end{bmatrix}
\]

\[e_{21(-2)}(e_{12}(A)) =
    \begin{bmatrix}
        1 & 0 & 1 & 2  \\
        0 & 1 & 0 & -3  \\
        3 & 2 & 1 & -1 \\
    \end{bmatrix}
\]

\[e_{31(-3)}(e_{21(-2)}(e_{12}(A))) =
    \begin{bmatrix}
        1 & 0 & 1 & 2  \\
        0 & 1 & 0 & -3  \\
        0 & 2 & -2 & -5 \\
    \end{bmatrix}
\]

\[A' = e_{32(-2)}(e_{31(-3)}(e_{21(-2)}(e_{12}(A)))) =
    \begin{bmatrix}
        1 & 0 & 1 & 2  \\
        0 & 1 & 0 & -3  \\
        0 & 0 & -2 & 1 \\
    \end{bmatrix}
\]

Since $A'$, which is a matrix in echelon form, is obtained by a series of elementary row operations on $A$, we know that $A$ and $A'$ are row-equivalent.

\newpage
\exercise{2}{
    Find a matrix in row-echelon form that is row-equivalent to $A$.
    \[ A = 
    \begin{bmatrix}
        1 & -i \\
        2 & 2 \\
        i & 1+i
    \end{bmatrix}
    \in M_{3\times 2}(\C).\]
}

We will apply a series of elementary row operations to $A$ in order to obtain a matrix in echelon form. That matrix will, by definition, be row-equivalent to $A$.
\[e_{21(-2)}(A) = 
    \begin{bmatrix}
        1 & -i \\
        0 & 2+2i \\
        i & 1+i
    \end{bmatrix}
\]

\[e_{31(-i)}(e_{21(-2)}(A)) = 
    \begin{bmatrix}
        1 & -i \\
        0 & 2+2i \\
        0 & i
    \end{bmatrix}
\]

\[e_{23(2i-2)}(e_{31(-i)}(e_{21(-2)}(A))) = 
    \begin{bmatrix}
        1 & -i \\
        0 & 0 \\
        0 & i
    \end{bmatrix}
\]

\[A' = e_{23}(e_{23(2i-2)}(e_{31(-i)}(e_{21(-2)}(A)))) = 
    \begin{bmatrix}
        1 & -i \\
        0 & i \\
        0 & 0
    \end{bmatrix}
\]

Since $A'$, which is a matrix in echelon form, is obtained by a series of elementary row operations on $A$, we know that $A$ and $A'$ are row-equivalent.


\newpage
\exercise{3}{
    We say that two matrices $A$ and $B$ are column-equivalent if there is a sequence of column operations that transform $A$ into $B$. Show that this is an equivalence relation.
}

For this to be an equivalent relation, it must be reflexive, symmetric, and transitive. Any matrix $A$ is column-equivalent to itself, since we can find a series of column operations on $A$ to obtain $A$. For instance, multiplying the first column by 1. This means that column-equivalence is a reflexive relation.

Suppose we have two matrices $A$ and $B$ with $B \sim A$\textemdash that is, there is a sequence of column operations $(e_1 \circ \cdots \circ e_n)$ such that $(e_1 \circ \cdots \circ e_n)(A) = B$. We can take the inverse of each of these column operations to find $(e_n^{-1} \circ \cdots \circ e_1^{-1})(B) = A$. Which means that $A \sim B$. Therefore, column-equivalence is a symmetric relation.

Now suppose we have matrices $A$, $B$, and $C$ where $A\sim B$ and $B\sim C$. This tells us we can find series of row operations $(e_i \circ \cdots \circ e_n)$ and $(f_1 \circ \cdots \circ f_m)$ where
\[(e_i \circ \cdots \circ e_n)(B) = A \text{ and } (f_1 \circ \cdots \circ f_m)(C) = B.\]
Substituting for $B$, we find
\[(e_i \circ \cdots \circ e_n)((f_1 \circ \cdots \circ f_m)(C)) = A,\]
\[(e_i \circ \cdots \circ e_n \circ f_1 \circ \cdots \circ f_m)(C) = A.\]
Since there is a sequence of column operations from $C$ to $A$, we know $A\sim C$. Therefore column-equivalence is a transitive relation. Since it is reflexive, symmetric, and transitive, it is an equivalence relation.



\newpage
\exercise{6}{
    Show that every $n \times m$ matrix is row-equivalent to a matrix in echelon form.
}

A matrix is row-equivalent to some matrix in echelon form if and only if there exists a series of row operations which, when applied to the matrix, produce a matrix in echelon form. We will show that from any matrix, we can obtain a matrix in echelon form using only row operations. To show this is possible for a matrix $A\in\F^{n\times m}$, we will perform induction on the number of rows $n$.

The base case of $n=1$ is clearly true since for $A\in\F^{1\times m}$, we have
\[A = [a_{11} \quad a_{12} \quad a_{13} \quad \cdots \quad a_{1n}],\]
which is already a matrix in echelon form, so we need not apply any row operations.

For the inductive step, we assume that from every matrix of size $n\times m$, we can obtain a matrix in echelon form using only row operations. Let $A\in\F^{(n+1)\times m}$ where
\[A =
    \begin{bmatrix}
        a_{11} & a_{12} & a_{13} & \cdots & a_{1m} \\
        a_{21} & a_{22} & a_{23} & \cdots & a_{2m} \\
        a_{31} & a_{32} & a_{33} & \cdots & a_{3m} \\
        \vdots & \vdots & \vdots & \ddots & \vdots \\
        a_{n1} & a_{n2} & a_{n3} & \cdots & a_{nm} \\
        a_{(n+1)1} & a_{(n+1)2} & a_{(n+1)3} & \cdots & a_{(n+1)m} \\
    \end{bmatrix}
.\]

We now consider the matrix $A'$, which is the first $n$ rows of $A$, and the matrix $B$, which is just the $(n+1)$th row of $A$. That is,
\[A' =
    \begin{bmatrix}
        a_{11} & a_{12} & a_{13} & \cdots & a_{1m} \\
        a_{21} & a_{22} & a_{23} & \cdots & a_{2m} \\
        a_{31} & a_{32} & a_{33} & \cdots & a_{3m} \\
        \vdots & \vdots & \vdots & \ddots & \vdots \\
        a_{n1} & a_{n2} & a_{n3} & \cdots & a_{nm} \\
    \end{bmatrix}
,\]
\[B = [a_{(n+1)1} \quad a_{(n+1)2} \quad a_{(n+1)3} \quad \cdots \quad a_{(n+1)m}].\]

Since $A'$ is an $n\times m$ matrix, we can find a series of row operations $e_1,\dots,e_k$ such that $A_1' = (e_1 \circ\dots\circ e_k)(A')$ is a matrix in echelon form. We now take $A_1 = (e_1 \circ\dots\circ e_k)(A)$, to obtain a matrix with the same entries as $A_1'$ in the first $n$ rows, and the same entries as $B$ in the $(n+1)$th row.

We define $k_i$ to be the column of the first nonzero element in the $i$th row of $A_1$. Since $A_1'$ is a matrix in echelon form, and has the same values as the first $n$ rows of $A_1$, then for the first $r\leq n$ nonzero rows in $A_1$, we find that $k_1<\dots<k_r$.

If row $n+1$ is all zeros, then $A_1$ is already in echelon form. Otherwise, we define $c_1$ to be the column of the first nonzero element in the $(n+1)$th row of $A_1$. If $k_r<c_1$, then we swap rows $r+1$ and $n+1$, which produces a matrix in echelon form. Otherwise, we consider the smallest $i$ for which $c_1<k_i$.

If $k_{i-1}<c_1<k_i$, then we swap rows $i$ and $n+1$, then swap rows $i+1$ and $n+1$, and so on until we swap rows $r+1$ and $n+1$. At which point, we have obtained a matrix in echelon form. Otherwise, it is the case that $k_{i-1}=c_1$, so we add a multiple of row $i-1$ to row $n+1$ which results in the entry of row $n+1$ at column $c_1$ to be zero. Then we find $c_2$, which is the column of the first nonzero entry in row $n+1$. Applying this operation to $A_1$ gives us $A_2$, which has the same entries as $A_1$ in the first $n$ rows, but different in the $(n+1)$th row. Notably, $c_1<c_2$, in other words, the first nonzero element in the $(n+1)$th row occurs at a later column in $A_2$ than in $A_1$. 

If $k_r<c_2$, then we swap rows $r+1$ and $n+1$, which produces a matrix in echelon form. Otherwise, we consider the smallest $i$ for which $c_2<k_i$, and repeat the process in the previous two paragraphs with $A_2,A_3,\dots,A_l$, until a matrix $A_l$ in echelon form in produced. This process will eventually complete, since each $c_1<\cdots<c_l<m$, for a finite fixed $m$.

So, there is a sequence of row operations which can be applied to $A$ to produce a matrix in echelon form. Which completes the inductive step.

Therefore, for any $n\times m$ matrix, we can apply a series of row operations to obtain a matrix in row echelon for. Which tells us that every $n\times m$ matrix is row-equivalent to a matrix in echelon form.







\end{document}