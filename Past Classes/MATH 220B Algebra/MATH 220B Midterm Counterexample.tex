\documentclass[12pt]{article}

% Packages
\usepackage[margin=1in]{geometry}
\usepackage{fancyhdr, parskip}
\usepackage{amsmath, amsthm, amssymb}
\usepackage{tikz, tikz-cd}

% Page Style
\makeatletter
\fancypagestyle{title}{
    \renewcommand{\headrulewidth}{0.4pt}
    \setlength{\headheight}{15pt}
    \fancyhead[R]{\@author}
    \fancyhead[L]{\@title}
    \fancyhead[C]{\@date}
}
\makeatother
\renewcommand{\maketitle}{\thispagestyle{title}}
\fancypagestyle{plain}{
    \fancyhf{}
    \renewcommand{\headrulewidth}{0pt}
    \renewcommand{\footrulewidth}{0pt}
    \fancyfoot[R]{\thepage}
}
\pagestyle{plain}

% Problem Box
\setlength{\fboxsep}{4pt}
\newlength{\myparskip}
\setlength{\myparskip}{\parskip}
\newsavebox{\savefullbox}
\newenvironment{fullbox}{\begin{lrbox}{\savefullbox}\begin{minipage}{\dimexpr\textwidth-2\fboxsep\relax}\setlength{\parskip}{\myparskip}}{\end{minipage}\end{lrbox}\framebox[\textwidth]{\usebox{\savefullbox}}}
\newenvironment{pbox}[1][]{\begin{fullbox}\ifx#1\empty\else\paragraph{#1}\phantom{}\fi}{\end{fullbox}}

% Theorem Environments
\theoremstyle{definition}
\newtheorem{lemma}{Lemma}

% Tikz Environments
\newenvironment{drawing}{\begin{center}\begin{tikzpicture}}{\end{tikzpicture}\end{center}}
% \tikzcdset{row sep/normal=0pt}
\newenvironment{cd}{\begin{center}\begin{tikzcd}}{\end{tikzcd}\end{center}}

% Default Commands
\newcommand{\isp}[1]{\quad\text{#1}\quad}
\newcommand{\N}{\mathbb{N}} 
\newcommand{\Z}{\mathbb{Z}}
\newcommand{\Q}{\mathbb{Q}}
\newcommand{\R}{\mathbb{R}}
\newcommand{\C}{\mathbb{C}}
\newcommand{\A}{\mathbb{A}}
\renewcommand{\P}{\mathbb{P}}
\newcommand{\eps}{\varepsilon}
\renewcommand{\phi}{\varphi}
\renewcommand{\emptyset}{\varnothing}
\newcommand{\<}{\langle}
\renewcommand{\>}{\rangle}
\newcommand{\isom}{\cong}
\newcommand{\eqc}{\overline}
\newcommand{\clo}{\overline}
\newcommand{\seq}{\subseteq}
\newcommand{\teq}{\trianglelefteq}
\DeclareMathOperator{\id}{id}
\DeclareMathOperator{\im}{im}

% Extra Commands


% Document
\begin{document}
\title{MATH 220B Counterexample}
\author{Harry Coleman}
\date{February 6, 2022}
\maketitle

Take $R = \Z$ and consider the $\Z$-modules $A = \Z$ and $B = \bigoplus_{n \in \N} \Z/p\Z$ for a prime $p \in \Z$.

Take $M = A \oplus B$ with maps
\begin{align*}
    f : \Z &\longrightarrow \Z \oplus B \\
        a &\longmapsto pa \oplus 0
\end{align*}
and
\begin{align*}
    g : \Z \oplus \bigoplus_{n \in \N} \Z/p\Z &\longrightarrow \bigoplus_{n \in \N} \Z/p\Z \\
        a \oplus b &\longmapsto (a + p\Z) \oplus b.
\end{align*}

In other words, $f$ is multiplication by $p$ composed with inclusion into the direct sum, and $g$ is the map which quotients $A$ onto the first component and shifts all the components in $B$ to the right by one index.

Then $f$ is inject and $g$ is surjective with $\ker g = p\Z \oplus 0 = \im f$.
Hence we have the following short exact sequence:
\begin{cd}
    0 \rar & A \rar["f"] & A \oplus B \rar["g"] & B \rar & 0.
\end{cd}

However, if there were a $\Z$-module homomorphism $q : B \to A \oplus B$ such that $g \circ q = \id_B$, it can be seen that $q$ would need to map the first component of $B$ to some part of $A$.
In particular, given $(n + p\Z) \oplus 0 \in B$, we must be able to choose some $q((n + p\Z) \oplus 0) = a \oplus b \in A \oplus B$ such that
\[
    (n + p\Z) \oplus 0 = g(a \oplus b) = (a + p\Z) \oplus b,
\]
which implies $b = 0$ and $a \in \Z$ is an integer with $a + p\Z = n + p\Z$.
This means restricting $q$ to the first component of $B$ induces a $\Z$-module homomorphism $\Z/p\Z \to \Z$ where $n \mapsto a$.
However, the only $\Z$-module homomorphism $\Z/p\Z \to \Z$ is the zero map.
But in order to have $g \circ p = \id_B$, we must have $q$ injective, but with the first component of $B$ nonzero, this is not possible.

I believe that the formulation of (i) should also require that $f$ and $g$ act as inclusion and surjection.
That is, not only do we need $M \isom A \oplus B$, but there must be an isomorphism $\phi : M \to A \oplus B$ such that $\phi \circ f : A \to A \oplus B$ is the natural inclusion and $g \circ \phi^{-1} : A \oplus B \to B$ is the natural projection, i.e., the following diagram commutes with exact rows:
\begin{cd}
    0 \rar & A \dar["\id_A"] \rar["f"] & M \dar["\phi"] \rar["g"] & B \dar["\id_B"] \rar & 0 \\
    0 \rar & A \rar[hookrightarrow] & A \oplus B \rar[twoheadrightarrow] & B \rar & 0
\end{cd}
(Where the bottom row uses the natural inclusion and projection.)
The above example fails because the isomorphism $M \isom A \oplus B$ does not make the right square in this diagram commute.





\end{document}