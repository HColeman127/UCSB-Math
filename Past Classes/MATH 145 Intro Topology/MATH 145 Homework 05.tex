\documentclass[12pt]{article}

% packages
\usepackage{kantlipsum}
\usepackage[margin=1in]{geometry}
\usepackage[labelfont=it]{caption}
\usepackage[table]{xcolor}
\usepackage{subcaption,framed,colortbl,multirow,enumitem}
\usepackage{amsmath,amsthm,amssymb,wasysym,mathrsfs,mathtools,babel}
\usepackage{tikz,graphicx,pgf,pgfplots}
\usetikzlibrary{arrows, angles, quotes, decorations.pathreplacing, math, patterns, calc}
\pgfplotsset{compat=1.16}

% Theorems
\newtheorem{theorem}{Theorem}
\newtheorem{lemma}{Lemma}
\newtheorem{proposition}{Proposition}

% Problem Box
\setlength{\fboxsep}{4pt}
\newsavebox{\mybox}
\newenvironment{problem}
    {\begin{lrbox}{\mybox}\begin{minipage}{\textwidth-10pt}}
    {\end{minipage}\end{lrbox}\framebox[6.5in]{\usebox{\mybox}}}

% Environments
\newenvironment{drawing}{\begin{center}\begin{tikzpicture}}{\end{tikzpicture}\end{center}}
\newenvironment{response}{\paragraph{}}{}

% Formatting
\newcommand{\ds}{\displaystyle}
\newcommand{\isp}[1]{\quad\text{#1}\quad}
\newcommand{\seq}[2]{\left\{#1\right\}_{#2=1}^\infty}
\newcommand{\clo}[1]{\overline{#1}}

% Paired Delimiters
\DeclarePairedDelimiter{\ceil}{\lceil}{\rceil}
\DeclarePairedDelimiter\floor{\lfloor}{\rfloor}
\DeclarePairedDelimiter{\ang}{\langle}{\rangle}

% Sets
\newcommand{\N}{\mathbb{N}}
\newcommand{\Z}{\mathbb{Z}}
\newcommand{\I}{\mathbb{I}}
\newcommand{\R}{\mathbb{R}}
\newcommand{\Q}{\mathbb{Q}}
\newcommand{\C}{\mathbb{C}}
\newcommand{\F}{\mathbb{F}}

% Misc Characters
\newcommand{\powerset}{\raisebox{.15\baselineskip}{\Large\ensuremath{\wp}}}
\let\eps\varepsilon
\let\emptyset\varnothing

% Functions
\newcommand{\id}[1]{\mathsf{id}_{#1}}

% Babel Shorthands
\useshorthands*{"}
\defineshorthand{"-}{\setminus}
\defineshorthand{"d}{\partial}

% Topology
\newcommand{\T}{\mathscr{T}}
\renewcommand{\S}{\mathscr{S}}
\newcommand{\B}{\mathscr{B}}
\renewcommand{\int}{\mathsf{int}}
\newcommand{\diam}{\text{diam}}
 
\begin{document}
 
\title{Homework 5\\
    \large MATH 145 Intro to Topology
}
\author{Harry Coleman}
\date{July 27, 2020}
\maketitle

\section*{Exercise 2.12.2}
\begin{problem}
    Prove that each projection $\pi_\beta$ of $\Pi X_\alpha$ onto a coordinate space $X_\beta$ is an open map.
\end{problem}

\begin{proof}
    Let $U\subseteq X = \Pi X_\alpha$ be an open subset. Since $X$ is generated by the set
    \[\{\pi_\alpha^{-1}(U) : \alpha\in I, U_\alpha\subseteq X_\alpha \text{ is open}\},\]
    then we $U$ takes the form
    \[\prod_{i=1}^n U_{\alpha_i} \times \prod_{\alpha\ne\alpha_i, i=1,\dots,n}X_\alpha,\]
    where $U_{\alpha_i}\subseteq X_{\alpha_i}$ is open. Therefore, $\pi_\beta(U) = X_\beta$ or $\pi_\beta(U)=U_\beta$ where $U_\beta\subseteq X_\beta$ is open. In either case, $\pi_\beta(U)$ is open, so $\pi_\beta$ is an open map.
\end{proof}

\section*{Exercise 2.12.3}
\begin{problem}
    Prove that the restriction of the projection $\pi_\beta$ to the coordinate slice $X_\beta \times \{y\}$ is a homeomorphism from $X_\beta \times \{y\}$ to $X_\beta$, $X_\beta\times\{y\}$ being given the relative topology as a subset of $\Pi X_\alpha$. 
\end{problem}

\begin{proof}
    Firstly, note that $\pi_\beta : X_\beta\times\{y\} \to X_\beta$ is continuous since it is the restriction of a continuous function. Now let $x\in X_\beta$. Then $\{x\}\times\{y\}\in X_\beta\times\{y\}$ and $\pi_\beta(\{x\}\times\{y\}) = x$. Therefore the restriction of $\pi_\beta$ is surjective on $X_\beta$. Now suppose $a,b\in X_\beta\times\{y\}$ such that $\pi_\beta(a)=\pi_\beta(b)$. Now since $a=\{x_1\}\times\{y\}$ and $b=\{x_2\}\times\{y\}$ with $x_1,x_2\in X_\beta$ and $x_1=\pi_\beta(a)=\pi_\beta(b)=x_2$, then in fact $a=b$. Therefore, the restriction of $\pi_\beta$ is a bijective, continuous, and, from Exercise 2.12.2, open map, so it is a homeomorphism.
    
\end{proof}

\section*{Exercise 2.12.4}
\begin{problem}
    Prove that the product of Hausdorff spaces is Hausdorff.
\end{problem}

\begin{proof}
    Suppose $X_\alpha$ is a Hausdorff space for all $\alpha\in I$ and consider the product space $X=\Pi X_\alpha$. Let $x,y\in X$ such that $x\ne y$, i.e., for some $\beta\in I$, $x_\beta\ne y_\beta$. Now since $X_\beta$ is Hausdorff and $x_\beta,y_\beta\in X_\beta$ such that $x_\beta\ne y_\beta$, then there exist open subsets $U_\beta,V_\beta\subseteq X_\beta$ such that $x_\beta\in U_\beta$, $y_\beta\in V_\beta$, and $U\cap V=\emptyset$. Then we define
    \[U=\pi_\beta^{-1}(U_\beta) = U_\beta \times\prod_{\alpha\in I"-\{\beta\}}X_\alpha,\]
    \[V=\pi_\beta^{-1}(V_\beta) = V_\beta \times\prod_{\alpha\in I"-\{\beta\}}X_\alpha.\]
    By this definition, we see that $U,V\subseteq X$ are open such that $x\in U$, $y\in V$, and $U\cap V=\emptyset$. Therefore $X$ is Hausdorff.
    
\end{proof}

\section*{Exercise 2.13.1}
\begin{problem}
    Let $X/\sim$ be the quotient space determined by equivalence relation $\sim$ on a topological space $X$. Prove the following assertions:
\end{problem}

\subsection*{Exercise 2.13.1(a)}
\begin{problem}
    If $X$ is compact, then $X/\sim$ is compact.
\end{problem}

\begin{proof}
    Suppose $X$ is compact and let $\{U_\alpha : \alpha\in I\}$ be an open cover of $X/\sim$. Then by definition of the quotient topology, $\{\pi^{-1}(U_\alpha) : \alpha\in I\}$ is an open cover for $X$. Since $X$ is compact, there exists a finite subcover $\{\pi^{-1}(U_{\alpha_i}) : i=1,\dots,n\}$. And since $\pi$ is surjective on $X/\sim$, then for all $y\in X/\sim$, there exists some $x\in X$ such that $\pi(x)=y$. And since each $\pi(x)\in U_{\alpha_i}$ for some $i\in\{1,\dots,n\}$, then $\{\pi^{-1}(U_{\alpha_i}) : i=1,\dots,n\}$ is an open cover for $X/\sim$. And since this is a finite subcover of the original open cover, $X/\sim$ is compact.
    
\end{proof}

\subsection*{Exercise 2.13.1(b)}
\begin{problem}
    If $X$ is connected, then $X/\sim$ is connected.
\end{problem}

\begin{proof}
     Suppose $X$ is connected and let $U\subseteq X/\sim$ be a nonempty closed and open subset. Then by definition of the quotient topology, $\pi^{-1}(U)$ is nonempty and open, and $\pi^{-1}(U^C)$ is open. Now since $\pi^{-1}(U) = \pi^{-1}(U^C)^C$, then $\pi^{-1}(U)$ is also closed. Since $X$ is connected, $\pi^{-1}(U)= X$, so $U=X/\sim$. Therefore $X/\sim$ is connected. 
    
\end{proof}

\subsection*{Exercise 2.13.1(c)}
\begin{problem}
    If $X$ is path-connected, then $X/\sim$ is path-connected.
\end{problem}

\begin{proof}
     Suppose $X$ is path-connected and let $x,y\in X/\sim$. For some $x',y'\in X$, $\pi(x')=x$ and $\pi(y')=y$. Now since $X$ is path connected, there exists some continuous function $f:[0,1]\to X$ such that $f(0)=x'$ and $f(1)=y'$. Now since $\pi$ is continuous, the composition $\pi\circ f: [0,1] \to X/\sim$ is continuous with $f(0)=x$ and $f(1)=y$. Therefore, $X/\sim$ is path-connected.
    
\end{proof}

\section*{Exercise 2.13.3}
\begin{problem}
    Define an equivalence relation on $X=[0,1]\times[0,1]$ by declaring $(s_0,t_0)\sim (s_1,t_1)$ if and only if $t_0=t_1>0$. Describe the quotient space $X/\sim$ and show that it is not a Hausdorff space.
\end{problem}

\begin{response}
    The underlying set of the quotient space can be described as the union of the two intervals $(0,1)$ and $[0,1]$. All the points in the subspace $[0,1]\times(0,1]$ are mapped by $\pi$ to the interval $(0,1]$; in particular, $\pi((s,t))=t$ for all $(s,t)\in(0,1]\times[0,1]$ with $t>0$. And all the points in the subspace $[0,1]\times\{0\}$ are mapped to $[0,1]$; in particular, $\pi((s,0))=s$ for all $s\in[0,1]$.
    
\end{response}

\begin{proposition}
    $X/\sim$ is not Hausdorff.
\end{proposition}
    
\begin{proof}
    Consider the points $x=(0,0),y=(1,0)\in X$. By definition of the equivalence relation, $\pi(x)\ne\pi(y)$. Now suppose $U,V\subseteq X/\sim$ are open neighborhoods of $x,y$, respectively. Then by definition of the quotient topology, $\pi^{-1}(U)$ and $\pi^{-1}(V)$ are open subsets of $X$. So there exists some $\eps_1,\eps_2>0$ such that $B(\eps_1,x)\subseteq\pi^{-1}(U)$ and $B(\eps_2,y)\subseteq\pi^{-1}(V)$. Then for some $z\in(0,\min\{\eps_1,\eps_2)\}$, we have $(0,z)\in\pi^{-1}(U)$ and $(1,z)\in\pi^{-1}(V)$. Therefore $\pi^{-1}(U)\cap\pi^{-1}(V) \ne\emptyset$ so $U\cap V\ne \emptyset$. Therefore, $X/\sim$ is not Hausdorff.
    
\end{proof}

\section*{Exercise 2.13.6}
\begin{problem}
    Let $B^n$ be the closed unit ball in $\R^n$. Prove that the quotient space obtained from $B^n$ by identifying its boundary $S^{n-1}$ to a point is homeomorphic to the $n$-sphere $S^n$.
\end{problem}

\begin{proof}
    We define a function $f:B^n\to\R^n\cup\{\infty\}$ such that
    \[f(x) = 
        \begin{cases}
            -x\log(1-\|x\|) &, x\in\int B^n, \\
            \infty &, x\in"dB^n.
        \end{cases}
    \]
    Note that the interior of $B^n$ is the set of all $x\in B^n$ for which $\|x\|<1$ and the boundary of $B^n$ is the set of all $x\in B^n$ for which $\|x\|=1$. Taking the one-point compactification topology on $\R^n\cup\{\infty\}$, we claim that $f$ is a continuous map. Let $U\subseteq\R^n\cup\{\infty\}$ be open in the one-point compactification topology. If $\infty\notin U$, then $f^{-1}(U)\subseteq\int B^n$, and since $f$ is continuous on the interior of $B^n$, then $f^{-1}(U)$ is open. If $\infty\in U$, then $U^C\subseteq\R^n$ is compact, i.e., closed and bounded. Additionally, since $\infty\notin U^C$, then $f^{-1}(U)\subseteq\int B^n$, and again from the continuity of $f$ on the interior of $B^n$, we have $f^{-1}(U^C)$ closed. Therefore $f^{-1}(U)=f^{-1}(U^C)^C$ is open, and we have that $f$ is continuous. 
    
    Now since $B^n$ is a closed and bounded subset of $\R^n$, it is compact and Hausdorff, and by definition, $\R\cup\{infty\}$ with the one-point compactification topology is compact and Hausdorff. Therefore, we can define an equivalence relation $\sim$ on $B^n$ such that for all $x,y\in B^n$, $x\sim y$ if and only if $f(x)=f(y)$, and $B^n/\sim \simeq \R\cup\{\infty\}$. Note that by our definition of $f$, $B^n/\sim$ is precisely the quotient space obtained from $B^n$ by identifying its boundary to a point. And since $\R^n\cup\{\infty\}$ with the one-point compactification topology is homeomorphic to the $n$-sphere, $S^n$, we have the quotient topology $B^n/\sim$ homeomorphic to $S^n$.
    
\end{proof}

\section*{Exercise 2.13.8}
\begin{problem}
    For $n\geq 1$, define $P^n=S^n/\sim$, where the equivalence relation is defined by declaring $x\sim y$ if and only if $x=y$ or $x=-y$. In other words, $P^n$ is obtained from $S^n$ by identifying pairs of antipodal points. The space $P^n$ is called \emph{real projective space} of dimension $n$, and it can be regarded as the set of lines in $\R^{n+1}$ which pass through the origin. Establish the following assertions:
\end{problem}

\subsection*{Exercise 2.13.8(a)}
\begin{problem}
    $P^n$ is a compact Hausdorff space.
\end{problem}
\begin{proof}
    Since $S^n$ is compact, $P^n=S^n/\sim$ is also compact. Let $\{x,-x\},\{y,-y\}\in P^n$ such that $\{x,-x\}\ne \{y,-y\}$. Now since $S^n$ is a subset of $\R^n$, $S^n$ is a $T_4$-space, so $\{x,-x\},\{y,-y\}\subseteq S^n$ are closed subsets and there exist open subsets $U,V\subseteq S^n$ such that $\{x,-x\}\subseteq U$, $\{y,-y\}\subseteq V$, and $U\cap V = \emptyset$. We now define the sets
    \[-U = \{-z : z\in U\} \isp{and} -V=\{z:-z\in V\},\]
    and $U'=U\cap-U$ and $V'=V\cap-V$. Consider the subsets $\pi(U')$ and $\pi(V')$ of $P^n$. Since for all $z\in U'$, we have $-z\in U'$, then $U'=\pi^{-1}(\pi(U'))$. This implies $\pi(U')$ is open, and similarly for $\pi(V')$. We now claim $\pi(U')\cap\pi(V')=\emptyset$. Suppose otherwise, then there exists some $\{z,-z\}\in\pi(U')\cap\pi(V')$. Without loss of generality, suppose $z\in U'$, which then gives us $-z\in U'$. However, since $U'\cap V' = \emptyset$, then $z,-z\notin V'$, which is a contradiction. Therefore, we have two open subsets $\pi(U'),\pi(V')\subseteq P^n$ such that $\{x,-x\}\subseteq\pi(U')$, $\{y,-y\}\subseteq\pi(V')$, and $\pi(U')\cap\pi(V')$. Therefore, $P^n$ is compact and Hausdorff.
    
\end{proof}

\subsection*{Exercise 2.13.8(b)}
\begin{problem}
    The projection $\pi:S^n \to P^n$ is a local homeomorphism, that is, each $x\in S^n$ has an open neighborhood that is mapped homeomorphically by $\pi$ onto an open neighborhood of $\pi(x)$.
\end{problem}

\begin{proof}
    Let $x_0\in S^n$ and define the open neighborhood $U=\{x\in S^n : \|x-x_0\| < 1/2\}$. Note that this definition gives us that for all $x,y\in U$, $\|x-y\|<2$. This means that since $\|x-(-x)\|=2$ for all $x\in S^n$, then $x\in U$ implies $-x\notin U$. Consider now the restriction of the projection $\pi: U\to \pi(U)$. This map is continuous and surjective on $\pi(U)$. To show it is injective, suppose $x,y\in U$ with $\pi(x)=\pi(y)$. This implies that either $x=y$ or $x=-y$. Since $x\in U$ implies $-x\notin U$, and $y\in U$, then we know $x=y$. Therefore $\pi:U\to\pi(U)$ is injective, and thus a homeomorphism. Therefore $U$ is homeomorphic to $\pi(U)$ and $\pi:S^n\to P^n$ is a local homeomorphism.
    
\end{proof}

\subsection*{Exercise 2.13.8(c)}
\begin{problem}
    $P^1$ is homeomorphic to the circle $S^1$.
\end{problem}

\begin{proof}
    Define the map $f:S^1\to S^1$ by
    \[f(x,y) = (\cos(2\theta), \sin(2\theta))\]
    where
    \[\theta =
        \begin{cases}
            \tan^{-1}(y/x), &x\ne0, \\
            \pi/2, &(x,y)=(0,1), \\
            3\pi/2, &(x,y)=(0,-1),
        \end{cases}
    \]
    for all $(x,y)\in S^1$. This map takes each point of $S^1$ and maps it to to point with twice the angle relative to the $x$-axis. This map is continuous and surjective, and we claim that $f$ is constant on the equivalence classes of $\sim$. Suppose $x\sim y$, then either $x=y$ or $x=-y$. If $x=y$, then clearly $f(x)=f(y)$. In the second case, we have $\theta_x=\theta_y\pm\pi$, that is,
    \begin{align*}
        f(x)
            &= (\cos(2\theta_x),\sin(2\theta_x)) \\
            &= (\cos(2(\theta_y\pm\pi)),\sin(2(\theta_y\pm\pi))) \\
            &= (\cos(2\theta_y\pm2\pi),\sin(2\theta_y\pm2\pi)) \\
            &= (\cos(2\theta_y),\sin(2\theta_y)) \\
            &= f(y).
    \end{align*}
    Now since $S^1$ is compact and Hausdorff, and $f$ is surjective, continuous, and constant on the equivalence classes of $\sim$, then we have $S^1/\sim=P^1\simeq S^1$.
    
\end{proof}

\subsection*{Exercise 2.13.8(d)}
\begin{problem}
    $P^n$ is homeomorphic to the quotient space obtained from the closed unit ball $B^n$ in $\R^n$ by identifying antipodal points of its boundary $S^{n-1}$.
\end{problem}

\begin{proof}
    Define the function $f:B^n\to P^n$ by
    \[f(x) = \{x',-x'\}\]
    where
    \[x' = (x_1,\dots,x_n,\sqrt{1-\|x\|^2})\]
    for all $x\in B^n$. This function is continuous, as each of its component functions are continuous. It is also surjective, since for any $\{z,-z\}\in P^n$, we have the point $z'=(z_1,\dots,z_n)\in B^n$. Assuming $z_{n+1}$ is nonnegative, $f(z')=\{z,-z\}$ because $\sqrt{1-\|z'\|^2}$ is simply the unique nonnegative value which gives $\|f(z')\|=1$, and that is precisely $z_{n+1}$. If $z_{n+1}$ is negative, then $f(-z')=\{z,-z\}$.
    
    Suppose now that $x,y\in "dB^n$ are antipodal, i.e., $x=-y$. Since $x,y\in"dB^n$, we have $\|x\|=\|y\|=1$. Therefore,
    \[f(x) = \{(x_1,\dots,x_n,0),(-x_1,\dots,-x_n,0)\} = \{(-y_1,\dots,-y_n,0),(y_1,\dots,y_n,0)\} = f(y).\]
    Now suppose $x,y\in B^n$ such that $f(x)=f(y)$, which implies $x=y$ or $x=-y$ and $\|x\|=\|y\|=1$. In the latter case, this means that $x,y\in"dB^n$ and are antipodal. Therefore, $f(x)=f(y)$ if and only if $x=y$ or $x,y\in"dB^n$ are antipodal. Now since $B^n$ and $P^n$ are compact and Hausdorff and $f:B^n\to P^n$ is a surjective continuous map, then the quotient space obtained from $B^n$ by identifying antipodal points of its boundary is homeomorphic to $P^n$.
    
\end{proof}

\section*{Exercise 2.13.9}
\begin{problem}
    Let $n\geq 1$ and let $X=\C^{n+1}"-\{0\}$. Let $\sim$ be the equivalence relation  in $X$ obtained by declaring $x\sim y$ if $x=\lambda y$ for some complex number $\lambda$. Then $CP^n = X/\sim$ is called \emph{complex projective space} of dimension $n$. It can be regarded as the set of one-dimensional subspaces of the complex vector space $\C^{n+1}$. 
\end{problem}

\section*{Exercise 2.13.9(a)}
\begin{problem}
    Prove that $CP^n$ is a compact Hausdorff space.
\end{problem}


\section*{Exercise 2.13.9(b)}
\begin{problem}
    Prove that $CP^1$ is homeomorphic to $S^2$.
\end{problem}

\section*{Exercise 2.13.9(c)}
\begin{problem}
    Regard $S^{2n+1}$ as the set of vectors $z = (z_0,\dots,z_n)\in\R^{n+1}$ such that $|z_0|^2 + \cdots + |z_n|^2 = 1$. Show that the natural projection $\pi:S^{2n+1}\to CP^n$ is onto and that $\pi^{-1}(x)$ is homeomorphic to $S^1$ for all $x\in CP^n$. Show that each $x\in CP^n$ has and open neighborhood $U$ such that $\pi^{-1}(U)$ is homeomorphic to the product space $U\times S^1$.
\end{problem}




\end{document}