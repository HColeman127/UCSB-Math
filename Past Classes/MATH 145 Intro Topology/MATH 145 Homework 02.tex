\documentclass[12pt]{article}

% packages
\usepackage{kantlipsum}
\usepackage[margin=1in]{geometry}
\usepackage[labelfont=it]{caption}
\usepackage[table]{xcolor}
\usepackage{subcaption,framed,colortbl,multirow,enumitem}
\usepackage{amsmath,amsthm,amssymb,wasysym,mathrsfs,mathtools,babel}
\usepackage{tikz,graphicx,pgf,pgfplots}
\usetikzlibrary{arrows, angles, quotes, decorations.pathreplacing, math, patterns, calc}
\pgfplotsset{compat=1.16}

% Environments
\newenvironment{drawing}{\begin{center}\begin{tikzpicture}}{\end{tikzpicture}\end{center}}

% Theorems
\newtheorem{theorem}{Theorem}
\newtheorem{lemma}{Lemma}
\newtheorem{proposition}{Proposition}

% Problem Box
\setlength{\fboxsep}{4pt}
\newsavebox{\mybox}
\newenvironment{problem}
    {\begin{lrbox}{\mybox}\begin{minipage}{\textwidth-10pt}}
    {\end{minipage}\end{lrbox}\framebox[6.5in]{\usebox{\mybox}}\\}

% Formatting
\newcommand{\ds}{\displaystyle}
\newcommand{\isp}[1]{\quad\text{#1}\quad}
\newcommand{\seq}[2]{\left\{#1\right\}_{#2=1}^\infty}
\newcommand{\clo}[1]{\overline{#1}}

% Paired Delimiters
\DeclarePairedDelimiter{\ceil}{\lceil}{\rceil}
\DeclarePairedDelimiter\floor{\lfloor}{\rfloor}
\DeclarePairedDelimiter{\ang}{\langle}{\rangle}

% Sets
\newcommand{\N}{\mathbb{N}}
\newcommand{\Z}{\mathbb{Z}}
\newcommand{\I}{\mathbb{I}}
\newcommand{\R}{\mathbb{R}}
\newcommand{\Q}{\mathbb{Q}}
\newcommand{\C}{\mathbb{C}}
\newcommand{\F}{\mathbb{F}}

% Misc Characters
\newcommand{\powerset}{\raisebox{.15\baselineskip}{\Large\ensuremath{\wp}}}
\let\eps\varepsilon
\let\emptyset\varnothing

% Functions
\newcommand{\id}[1]{\mathsf{id}_{#1}}

% Babel Shorthands
\useshorthands*{"}
\defineshorthand{"-}{\setminus}
\defineshorthand{"d}{\partial}

% Topology
\newcommand{\T}{\mathscr{T}}
\renewcommand{\S}{\mathscr{S}}
\newcommand{\B}{\mathscr{B}}
\renewcommand{\int}{\text{int}}
 
\begin{document}
 
\title{Homework 2\\
    \large MATH 145 Intro to Topology
}
\author{Harry Coleman}
\date{July 6, 2020}
\maketitle

\section*{Exercise 2.1.2}
\begin{problem}
    Let $X$ be a set and let $\T$ be the family of subsets $U$ of $X$ such that $X"-U$ is finite, together with the empty set $\emptyset$. Show that $\T$ is a topology. ($T$ is called the \emph{cofinite topology} of $X$.)
\end{problem}

\begin{proof}
    By definition, $\emptyset\in\T$ and since $\emptyset$ is finite, $X=X"-\emptyset\in\T$. Let $\{U_\alpha \in \T: \alpha\in I\}$ be an arbitrary collection of open subsets of $X$. We want to prove that the union of these subsets is also open, that is, we want to show that
    \[X"-\bigcup_{\alpha\in I}U_\alpha = \bigcap_{\alpha\in I}(X"-U_\alpha)\]
    is finite. We see above that is the case since each $X"-U_\alpha$ is finite, their intersection must also be finite. We now let $\{U_i:i=1,\dots,n\}$ be a finite collection of open sets in $X$ and aim to prove that their intersection is also open. That is, we want to show that
    \[X"-\bigcap_{i=1}^n U_i = \bigcup_{i=1}^n(X"-U_i)\]
    is finite, which is clearly the case as each $X"-U_i$ is finite, so their finite union must also be finite. Therefore $(X,\T)$ is a topological space as it satisfies all three conditions.
    
\end{proof}
 
\section*{Exercise 2.1.5}
\begin{problem}
    Let $S$ be a subset of a topological space $X$ in which sets consisting of one point are closed. A point $x\in X$ is a \emph{limit point} of $S$ if every neighborhood of $x$ contains a point if $S$ other than $x$ itself. A point $s\in S$ is an \emph{isolated point} of $S$ if there is a neighborhood $U$ of $s$ such that $U\cap S=\{s\}$. Show that the set of limit points of $S$ is closed. Show that $\clo{S}$ is the disjoint union of the set of limit points of $S$ and the isolated points of $S$.
\end{problem}

\begin{proposition}
    The set of limit points of $S$ is closed.
\end{proposition}

\begin{proof}
    Let $S'$ denote the set of limit points of $S$. Let $x\in X"-S'$, then there exists an open neighborhood $U$ of $x$ such that $(U\cap S)"-\{x\} = \emptyset$. To prove $S'$ is closed we must now show that $U$ is a open neighborhood of $x$ which does not intersect $S'$. Suppose for contradiction that $U\cap S'\ne\emptyset$, then there exists a point $y\in U$ which is a limit point of $S$. Since the singleton $\{x\}$ is closed, $(X"-\{x\})$ is open so the finite intersection $U\cap(X"-\{x\})=U"-\{x\}$ is open. Let $U'=U"-\{x\}$. Now since $y\in U'$, there exists a open neighborhood $V$ of $y$ such that $V\subseteq U'$. Since $y$ is a limit point of $S$ and $V$ is an open neighborhood of $y$, we know that $(V\cap S)"-\{y\}\ne\emptyset$. However, this implies that $U'\cap S = (U\cap S)"-\{x\}\ne\emptyset$, which is a contradiction.
    
\end{proof}

\begin{proposition}
    $\clo{S}$ is the disjoint union of the set of limit points of $S$ and the isolated points of $S$.
\end{proposition}

\begin{proof}
    Let $S'$ denote the set of limit points of $S$ and let $S^\bullet$ denote the set of isolated points. Let $x\in \clo{S}$, then for every neighborhood $U$ of $x$, $U\cap S\ne \emptyset$. In particular, it is either the case that there exist a neighborhood $U$ of $x$ with $U\cap S=\{x\}$ or for every neighborhood $U$ of $x$ such that there is a point $y\in U\cap S$ with $y\ne x$. If the former is true, this is precisely the definition of an isolated point so $x\in S^\bullet$. If the latter is true, then $x$ satisfies the definition of a limit point of $S$, i.e. for every neighborhood $U$ of $x$, $(U\cap S)"-\{x\}\ne\emptyset$, so $x\in S'$. In either case, $x\in S'\cup S^\bullet$, so $\clo{S}\subseteq S'\cup S^\bullet$.
    
    Now let $x\in S' \cup S^\bullet$. Suppose $x\in S'$, and let $U$ be a neighborhood of $x$. Since $x$ is a limit point of $S$, $(U\cap S)"-\{x\}\ne\emptyset$, which implies $U\cap S\ne\emptyset$, so $x\in\clo{S}$. Now suppose $x\in S^\bullet$. Since $x\in S$, then for any neighborhood $U$ of $x$, $x\in U\cap S \ne\emptyset$. So in either clase, $x\in\clo{S}$. Therefore $S'\cup S^\bullet \subseteq \clo{S}$, and in fact we have equality.
    
\end{proof}

\section*{Exercise 2.1.6}

\subsection*{Exercise 2.1.6(a)}
\begin{problem}
    Show that if $X$ is a metrizable topological  space and if $p$ and $q$ are distinct points of $X$, then there are open sets $U$ and $V$ such that $p\in U, q\in V$, and $U\cap V = \emptyset$.
\end{problem}

\begin{proof}
    Since $X$ is metrizable and $p\ne q$, then $\eps=d(p,q)>0$. We claim that the open balls $U=B(\eps/2, p)$ and $V=B(\eps/2, q)$ are the desired open sets. Clearly, $p\in U$ and $q\in V$. Suppose for contradiction that $U\cap V \ne \emptyset$ and that $x\in U\cap V$. Then $d(x,p)<\eps/2$ and $d(x,q)<\eps/2$. However, by the triangle inequality, this means that
    \begin{align*}
        d(p,q) &\leq d(x,p) + d(x,q), \\
        \eps < \eps/2 + \eps/2, \\
        \eps < \eps,
    \end{align*}
    which is a contradiction. Therefore $U$ and $V$ are indeed the desired open sets.
\end{proof}

\subsection*{Exercise 2.1.6(b)}
\begin{problem}
    Let $x$ be an infinite set and let $\T$ be the cofinite topology on $X$. Prove that the property described in part (a) does not hold for the open sets in $X$ and hence that $X$ with the cofinite topology is not metrizable.
\end{problem}

\begin{proof}
    Let $p,q\in X$ and let $U$ and $V$ be arbitrary neighborhoods of $p$ and $q$, respectively, which are open in $\T$. Since $U^C$ and $V^C$ are finite while $X$ is infinite, there exists some point $x\in X$ such that $x\notin U^C$ and $x\notin V^C$. However, this implies that $x\in U\cap V \ne \emptyset$, which is not the case as both $U$ and $V$ contain all but finitely many points in $X$. Therefore any pair of open sets around $p$ and $q$, respectively, will have a nonempty intersection. Thus the cofinite topology on $X$ is not metrizable.
    
\end{proof}

\section*{Exercise 2.1.9}
\begin{problem}
    Show that if $S$ is a subset of a topological space $X$, then
\end{problem}

\subsection*{Exercise 2.1.9(a)}
\begin{problem}
    $\clo{X"-S} = X"-\int(S)$,
\end{problem}

\begin{proof}
    Firstly,
    \begin{align*}
        \clo{X"-S} 
            &= \int(X"-S) \cup "d(X"-S) \\
            &= \int(X"-S) \cup "dS \\
            &= \int(X"-S) \cup (\clo{S}"-\int S) \\
            &\subseteq (X"-S) \cup (X"-\int S) \\
            &\subseteq (X"-\int S) \cup (X"-\int S) \\
            &= X"-\int S.
    \end{align*}
    Secondly,
    \begin{align*}
        X"-\int S
            &= X"-(\clo{S}"-"dS)\\
            &= (X\cap"dS) \cup (X"-\clo{S}) \\
            &= "dS \cup (X"-\clo{S}) \\
            &= "d(X"-S) \cup (X"-\clo{S}) \\
            &\subseteq "d(X"-S) \cup (X"-S) \\
            &\subseteq \clo{X"-S} \cup \clo{X"-S} \\
            &= \clo{X"-S}.
    \end{align*}
    Therefore $\clo{X"-S} = X"-\int(S)$.
    
\end{proof}

\subsection*{Exercise 2.1.9(b)}
\begin{problem}
    $\int(X"-S) = X"-\clo{S}$.
\end{problem}

\begin{proof}
    \begin{align*}
        \int(X"-S)
            &= \clo{X"-S}"-"d(X"-S) \\
            &= (X"-\int S)"-"dS \\
            &= X"-(\int S \cup "dS) \\
            &= X"-\clo{S}.
    \end{align*}
    
\end{proof}

\section*{Exercise 2.2.2}
\begin{problem}
    Prove that if $A$ and $S$ are subsets of a topological space $X$, then the closure of $A\cap S$ in $S$ in the relative topology for $S$ is a subset of the intersection $\clo{A}\cap S$, where $\clo{A}$ is the closure of $A$ in $X$. Give an example where the relative closure of $A\cap S$ is a proper subset of $\clo{A}\cap S$.
\end{problem}

Let $\clo{\,\cdot\,}^S$ denote the closure of a set in the relative topology for $S$.

\begin{proposition}
    The closure of $A\cap S$ in $S$ in the relative topology for $S$ is a subset of the intersection $\clo{A}\cap S$, where $\clo{A}$ is the closure of $A$ in $X$.
\end{proposition}

\begin{proof}
    Suppose $x\in\clo{A\cap S}$, then for every neighborhood $U$ of $x$, $U\cap(A\cap S)\ne\emptyset$, which implies that $U\cap A\ne\emptyset$. So we also have $x\in\clo{A}$, therefore $\clo{A\cap S}\subseteq \clo{A}$. Therefore
    \[\clo{A\cap S}^S = \clo{A\cap S} \cap S \subseteq \clo{A}\cap S.\]
    
\end{proof}

\begin{proposition}
    If $X=\R$, $S=\{0\}$, $A=(0,1)$, then $\clo{A\cap S}^S\subset \clo{A}\cap S$.
\end{proposition}

\begin{proof}
    \[\clo{A\cap S}^S = \clo{(0,1)\cap \{0\}} \cap \{0\} = \clo{\emptyset}\cap\{0\} = \emptyset\cap\{0\}=\emptyset,\]
    and
    \[\clo{A}\cap S = \clo{(0,1)}\cap \{0\} = [0,1]\cap\{0\} = \{0\}.\]
    Therefore $\clo{A\cap S}^S\subset \clo{A}\cap S$.
\end{proof}

\newpage
\section*{Exercise 2.2.6}
\begin{problem}
    Let $X$ be a topological space, let $S$ and $T$ be subsets of $X$, and let $A$ be a subset of $S\cap T$ that is relatively open in $S$ and in $T$. Is $A$ relatively open in $S\cup T$? Justify your answer.
\end{problem}

\begin{proposition}
    $A$ is relatively open in $S\cup T$.
\end{proposition}

\begin{proof}
    Since $A$ is relatively open in both $S$ and $T$, there exists sets $U,V$ which are open in $X$ such that $U\cap S = A = V \cap T$. Then since $A\subseteq S\cap T$, we also have $A\subseteq U$ and $A\subseteq V$. Therefore $A\subseteq U\cap V$, noting that $U\cap V$ is also open in $X$. Then
    \begin{align*}
        (U\cap V) \cap (S\cup T)
            &=  [(U\cap V) \cap S] \cup [(U\cap V) \cap T] \\
            &= (A\cap V) \cup (A\cap U) \\
            &= A\cup A \\
            &= A,
    \end{align*}
    so $A$ is open in $S\cup T$.
    
\end{proof}

\section*{Exercise 2.3.2}
\begin{problem}
    Show that A function $f:X\to Y$ is continuous if and only if $f^{-1}(E)$ is a closed subset of $X$ for every closed subset $E$ of $Y$.
\end{problem}

\begin{lemma}
    $f^{-1}(E^C)^C = f^{-1}(E)$ for all subsets $E$ of $Y$.
\end{lemma}

\begin{proof}
    \begin{align*}
        f^{-1}(E^C)^C
            &= \{x\in X : x\notin f^{-1}(E^C)\} \\
            &= \{x\in X : f(x) \notin E^C\} \\
            &= \{x\in X: f(x) \in E\} \\
            &= f^{-1}(E).
    \end{align*}
\end{proof}

\begin{proposition}
    $f:X\to Y$ is continuous if and only if $f^{-1}(E)$ is a closed subset of $X$ for every closed subset $E$ of $Y$.
\end{proposition}

\begin{proof}
    Suppose $f$ is continuous. Then for any closed subset $E$ of $Y$, $f^{-1}(E) = f^{-1}(E^C)^C$ is closed. Conversely, suppose $f^{-1}(E)$ is closed for every closed subset $E$ of $Y$. Then for any open subset $E$ of $Y$, $f^{-1}(E) = f^{-1}(E^C)^C$ is open, so $f$ is continuous.
    
\end{proof}

\section*{Exercise 2.3.5}
\begin{problem}
    Prove that all semiopen intervals in $\R$ (finite or semi-infinite) are homeomorphic.
\end{problem}

\begin{lemma}\label{145-01 both finite}
    Any pair of finite semiopen intervals, $[a,b), [c,d)$, in $\R$ are homeomorphic.
\end{lemma}

\begin{proof}
    Let $f:[a,b) \to [c,d)$ be defined for any $x\in[a,b)$ by
    \[f(x) = \left(\frac{x-a}{b-a}\right)(d-c) + c.\]
    We see that $f$ is a linear function on $\R$ and is therefore continuous on $[a,b)$ with the continuous inverse
    \[f^{-1}(y) = \left(\frac{y-c}{d-c}\right)(b-a) + a, \quad\text{for all } y\in[c,d).\]
    Thus $f$ is a homeomorphism.
\end{proof}

\begin{lemma}
    Any finite semiopen interval, $[a,b)\subseteq\R$, and semi-infinite semiopen interval $[c,\infty)\subseteq\R$, are homeomorphic.
\end{lemma}

\begin{proof}
    Let $f:[a,b) \to [c,\infty)$ be defined for any $x\in[a,b)$ by
    \[f(x) = \frac{a-b}{x-b} - 1 + c.\]
    We see that $f$ is a reciprocal function on $\R$ and is therefore continuous on $[a,b)$ with continuous inverse
    \[f^{-1}(y) = \frac{a-b}{y-c+1} + b, \quad\text{for all } y\in[c,d),\]
    Thus $f$ is a homeomorphism.
\end{proof}

\begin{lemma} \label{145-01 flip direction}
    The finite semiopen intervals, $[0,1), (-1,0]$, in $\R$ are homeomorphic.
\end{lemma}

\begin{proof}
    Let $f:[0,1) \to (-1,0]$ be defined for any $x\in[0,1)$ by
    \[f(x) = -x.\]
    We see that $f$ is a linear function on $\R$ and is therefore continuous on $[0,1)$ with continuous inverse
    \[f^{-1}(y) = -y, \quad\text{for all } y\in[c,d),\]
    Thus $f$ is a homeomorphism.
\end{proof}

\begin{proposition}
    All semiopen intervals in $\R$ (finite or semi-infinite) are homeomorphic.
\end{proposition}

\begin{proof}
    Let $X,Y\subseteq$ be semiopen intervals in $\R$. Each $X$ and $Y$ is either finite or semi-infinite and opens to the left or the right. In any case, by applying a combination of lemmas \ref{145-01 both finite}-\ref{145-01 flip direction}, we obtain $X\simeq [0,1) \simeq Y$.
    
\end{proof}

\section*{Exercise 2.3.6}
\begin{problem}
    Show that the unit ball $\{(x,y)\in\R^2 : x^2+y^2 < 1\}$ in $\R^2$ is homeomorphic to the open square $\{(x,y) : 0<x<1, 0<y<1\}$.
\end{problem}

\begin{lemma}
    The map $f:\{(x,y) : 0<x<1, 0<y<1\} \to \R^2$ defined by
    \[f(x,y) = \left(\ln\frac{x}{1-x}, \ln\frac{y}{1-y}\right)\]
    is a homeomorphism.
\end{lemma}

\begin{proof}
    The component functions of $f$ are well-defined and continuous on the open square, and therefore, so is $f$. The inverse of $f$ is given by
    \[f^{-1}(x,y) = \left(\frac{e^x}{e^x+1}, \frac{e^y}{e^y+1}\right).\]
    The component functions of $f^{-1}$ are continuous and well-defined on $\R^2$, and therefore, so it $f^{-1}$. Thus, $f$ is a homeomorphism.
    
\end{proof}

\begin{lemma}
    The map $f:\{(x,y) : x^2+y^2 < 1\} \to \R^2$ defined by
    \[f(x,y) = \left(\frac{x}{1-\sqrt{x^2+y^2}}, \frac{y}{1-\sqrt{x^2+y^2}}\right)\]
    is a homeomorphism.
\end{lemma}

\begin{proof}
    The component functions of $f$ are well-defined and continuous on the unit ball, and therefore, so is $f$. The inverse of $f$ is given by
    \[f^{-1}(x,y) = \left(\frac{x}{1+\sqrt{x^2+y^2}}, \frac{y}{1+\sqrt{x^2+y^2}}\right).\]
    The component functions of $f^{-1}$ are continuous and well-defined on $\R^2$, and therefore, so it $f^{-1}$. Thus, $f$ is a homeomorphism.
    
\end{proof}

\begin{proposition}
    The unit ball $\{(x,y)\in\R^2 : x^2+y^2 < 1\}$ in $\R^2$ is homeomorphic to the open square $\{(x,y) : 0<x<1, 0<y<1\}$.
\end{proposition}

\begin{proof}
    Since $\{(x,y) : 0<x<1, 0<y<1\}\simeq \R^2$ and $\R^2\simeq\{(x,y)\in\R^2 : x^2+y^2 < 1\}$, we have $\{(x,y) : 0<x<1, 0<y<1\}\simeq \{(x,y)\in\R^2 : x^2+y^2 < 1\}$.
\end{proof}

\section*{Exercise 2.3.7}
\begin{problem}
    Show that the punctured plane $\R^2"-\{(0,0)\}$ is homeomorphic to the exterior of the closed unit ball $\R^2"-\{(x,y) : x^2 + y^2 \leq 1\}$.
\end{problem}

\begin{proposition}
    Let $P=\R^2"-\{(0,0)\}$ and $Q=\R^2"-\{(x,y) : x^2 + y^2 \leq 1\}$. Then the map $f:P\to Q$ defined by
    \[f(x,y) = \left(x\cdot\frac{\sqrt{x^2+y^2}+1}{\sqrt{x^2+y^2}}, y\cdot \frac{\sqrt{x^2+y^2}+1}{\sqrt{x^2+y^2}}\right)\]
    is a homeomorphism. (Intuitively, $f$ maps a point in $P$ to a point in $Q$ the same direction from the origin, but one unit farther away from the origin.)
\end{proposition}

\begin{proof}
    Since $(x,y)\ne(0,0)$ for all $(x,y)\in P$, then $f$ is well-defined on $P$. And since each of its component functions are continuous on $P$, f itself is continuous on $P$. The inverse of $f$, $f^{-1}:Q\to P$, can be defined in the following way for any $(x,y)\in Q$:
    \[f^{-1}(x,y) = \left(x\cdot\frac{\sqrt{x^2+y^2}-1}{\sqrt{x^2+y^2}}, y\cdot \frac{\sqrt{x^2+y^2}-1}{\sqrt{x^2+y^2}}\right).\]
    As $f$ ``moves'' points one unite farther from the origin, $f^{-1}$ ``moves'' points one unit closer to the origin. Since $(x,y)\ne(0,0)$ for all $(x,y)\in Q$, then $f^{-1}$ is well-defined on $Q$. And since each of its component functions are continuous on $Q$, $f^{-1}$ itself is continuous on $Q$. Therefore, $f$ is a homeomorphism between $P$ and $Q$.
    
\end{proof}

\section*{Exercise 2.3.14}
\begin{problem}
    Let $X$ be a topological space and let $BC(X)$ denote the set of bounded continuous real-valued functions on $X$, with the metric
    \[d(f,g) = \sup\{|f(x)-g(x)| : x\in X\}.\]
    Show that $BC(X)$ is a complete metric space.
\end{problem}

\begin{proof}
    Let $\seq{f_n}{n}$ be a Cauchy sequence in $BC(X)$. For any point $x\in X$, $\seq{f_n(x)}{n}$ is a Cauchy sequence in $\R$, and therefore converges. We define the function $f:X\to\R$ such that $f_n(x)\to f(x)$, which, as noted, is well-defined for all $x\in X$.
    
    Let $\eps>0$ be given. Since $\seq{f_n}{n}$ is Cauchy, there exists some $N\in\N$ such that
    \[n,m\geq N \implies d(f_n,f_m)<\eps.\]
    That is, for all $x\in X$,
    \[n,m\geq N \implies |f_n(x)-f_m(x)|<\eps,\]
    and furthermore that
    \begin{align*}
        |f_n(x)-f(x)|
            &\leq |f_n(x) - f_m(x)| + |f_m(x)-f(x)| \\
            &< \eps + |f_m(x)-f(x)|.
    \end{align*}
    We now take the limit as $m\to\infty$ and obtain that for all $x\in X$,
    \[n\geq N \implies |f_n(x)-f(x)| \leq \eps + |f(x)-f(x)| = \eps.\]
    That is, for all $x\in X$, $n\geq N$ implies that $|f_n(x)-f(x)|\leq\eps$, giving us uniform convergence, and therefore convergence with respect to the metric, of $\seq{f_n}{n}$ to $f$. And since $f$ is the limit of a uniformly convergent sequence of continuous functions from a topological space to a metric space, we have that $f$ is continuous. By the uniform convergence, we let $N\in\N$ such that
    \[n\geq N \implies |f(x)-f_n(x)|<1\]
    for all $x\in X$. And let $M\in\R$ such that $|f_N(x)|<M$ for all $x\in X$. Then
    \begin{align*}
        |f(x)|
            &= |f(x)-f_N(x)+f_N(x)| \\
            &\leq |f(x)-f_N(x)| + |f_N(x)| \\
            & < 1 + M,
    \end{align*}
    for all $x\in X$, so $f$ is bounded. Therefore, $f\in BC(X)$  and $BC(X)$ is a complete metric space.
    
\end{proof}

\section*{Exercise 2.4.5}
\begin{problem}
Let $X$ be a set and let $\S$ be a family of subsets of $X$.
\end{problem}

\subsection*{Exercise 2.4.5(a)}
\begin{problem}
    Show that there exists a unique smallest topology $\T$ on $X$ such that $\S\subseteq\T$. The topology $\T$ is called the \emph{topology generated by} $\S$, and $\S$ is called a subbase for the topology $\T$.
\end{problem}

\begin{proof}
    Let $\B$ be the family of subsets of $X$ consisting of $X,\emptyset$, and all finite intersections of sets in $\S$. That is,
    \[\B = \{\emptyset,X\}\cup\left\{\bigcap_{i=1}^nU_i : \{U_i : i=1,\dots,n\} \subseteq \S\right\}.\]
    We then define
    \[\T = \left\{\bigcup_{\alpha\in I}U_\alpha : \{U_\alpha : \alpha\in I\} \subseteq \B\right\},\]
    which we claim to be the smallest topology on $X$ containing $\S$. By definition, we have $\emptyset, X \in\T$. Suppose $\{\mathcal{U}_\beta: \beta\in J\}$ is a collection of open subsets in $\T$, that is,
    \[\mathcal{U}_\beta = \bigcup_{\alpha\in I_\beta}U_\alpha,\]
    where $\{U_\alpha : \alpha\in I_\beta\}$ is a subset of $\B$. Then defining
    \[I = \bigcup_{\beta \in J}I_\beta,\]
    we obtain
    \[\bigcup_{\beta \in J}\mathcal{U}_\beta = \bigcup_{\beta \in J}\bigcup_{\alpha\in I_\beta}U_\alpha = \bigcup_{\alpha\in I}U_\alpha.\]
    And since $\{U_\alpha : \alpha \in I\}$ is a subset of $\B$, then we have
    \[\bigcup_{\beta \in J}\mathcal{U}_\beta \in \T.\]
    Now suppose $\{\mathcal{U}_i: i=1,\dots,n\}$ is a collection of open subsets in $\T$, that is,
    \[\mathcal{U}_i = \bigcup_{\alpha\in I_i}U_\alpha,\]
    where $\{U_\alpha : \alpha\in I_i\}$ is a subset of $\B$. Then defining
    \[I = \bigcap_{i=1}^nI_i,\]
    we obtain
    \[\bigcap_{i=1}^n\mathcal{U}_i = \bigcap_{i=1}^n\bigcup_{\alpha\in I_i}U_\alpha = \bigcup_{\alpha\in I}U_\alpha.\]
    And since $\{U_\alpha : \alpha \in I\}$ is a subset of $\B$, then we have
    \[\bigcap_{i=1}^n\mathcal{U}_i \in \T,\]
    so $\T$ is a topological space. Now suppose $\T'$ is another topology on $X$ which contains $\S$ and let $U\in\T$. Then by definition of $\T$, $U$ is an arbitrary intersection of sets in $\B$, where $\B$ is the collection of all finite unions of the sets of $\S$, plus the empty set and the whole set. And since $\S\subseteq\T'$, then we also have $\B\subseteq\T'$, by definition of $\T'$ as a topology on $X$. And since $U$ is an arbitrary union of sets in $\T'$, then we have $U\in\T'$. Therefore $\T\subseteq\T'$, so $\T$ is the unique smallest topology on $X$ containing $\S$.
    
\end{proof}

\subsection*{Exercise 2.4.5(b)}
\begin{problem}
    Let $\B$ be the family of subsets of $X$ consisting of $X,\emptyset$, and all finite intersections of sets in $\S$. Show that $\B$ is a base of open sets for the topology generated by $\S$.
\end{problem}

\begin{proof}
    Since $\T$ is defined as the collection of all arbitrary unions of sets in $\B$, then by definition, $\B$ is a base for $\T$.
    
\end{proof}

\subsection*{Exercise 2.4.5(c)}
\begin{problem}
    Let $X$ have the topology generated by $\S$, let $Y$ be a topological space, and let $f:Y\to X$ be a function. Show that $f$ is continuous if and only if $f^{-1}(S)$ is an open subset of $Y$ for every set $S\in\S$. 
\end{problem}

\begin{proof}
    Suppose $f$ is continuous, then for every open subset $S$ in $X$, the inverse image $f^{-1}(S)$ is open in $Y$. In particular, for any open subset $S\in\S\subseteq\T$, the inverse image $f^{-1}(S)$ is open in $Y$.
    
    Now suppose that $f^{-1}(S)$ is an open subset of $Y$ for every set $S\in\S$. Let $\mathcal{U}\in\B$. If $\mathcal{U}$ is $\emptyset$ or $X$, then immediately, $f^{-1}(U)$ would be $\emptyset$ or $Y$, respectively, and therefore be open. Otherwise,
    \[\mathcal{U} = \bigcap_{i=1}^nU_i,\]
    where $\{U_i : i=1,\dots,n\} \subseteq \S$. Then
    \begin{align*}
        f^{-1}(\mathcal{U})
            &= f^{-1}\left(\bigcap_{i=1}^nU_i\right) \\
            &= \left\{x\in X : f(x) \in \bigcap_{i=1}^nU_i \right\} \\
            &= \bigcap_{i=1}^n\{x\in X : f(x) \in U_i \} \\
            &= \bigcap_{i=1}^nf^{-1}(U_i).
    \end{align*}
    And since each $f^{-1}(U_i)$ is open in $Y$, then their finite intersection $\mathcal{U}$ is also open in $Y$. Therefore, $f^{-1}(\mathcal{U})$ is open in $Y$ for any $\mathcal{U}\in\B$. Now suppose $\mathcal{U}\in\T$, that is, $\mathcal{U}$ is an arbitrary open set in $X$. Then by definition of $\T$, we have
    \[\mathcal{U} = \bigcup_{\alpha\in I}U_\alpha,\]
    where $\{U_\alpha : \alpha\in I\} \subseteq \B$. Then
    \begin{align*}
        f^{-1}(\mathcal{U})
            &= f^{-1}\left(\bigcup_{\alpha\in I}U_\alpha\right) \\
            &= \left\{x\in X : f(x) \in \bigcup_{\alpha\in I}U_\alpha \right\} \\
            &= \bigcup_{\alpha\in I}\{x\in X : f(x) \in U_\alpha \} \\
            &= \bigcup_{\alpha\in I}f^{-1}(U_\alpha).
    \end{align*}
    And since each $f^{-1}(U_\alpha)$ is open in $Y$, then their union $\mathcal{U}$ is also open in $Y$. Therefore $f^{-1}(\mathcal{U})$ is open for every open subset $\mathcal{U}$ in $X$, giving us that $f$ is continuous.
    
\end{proof}

\section*{Exercise 2.4.6}
\begin{problem}
    Let $\B$ be the family of subsets of $\R$ of the form $[a,b)$, where
    \[-\infty < a < b <\infty.\]
\end{problem}

\subsection*{Exercise 2.4.6(a)}
\begin{problem}
    Show that $\B$ is a base of open sets for a topology $\T$ of $\R$. The topology determined by $\B$ is the \emph{half-open interval topology}.
\end{problem}

\begin{proof}
    Let $x\in\R$, then $x\in[x,x+1)\in\B$. Now suppose $[a,b),[c,d)\in\B$ and $x\in[a,b)\cap[c,d)$. Then if we define $e=\max\{a,c\}$ and $f=\min\{b,d\}$, then $x\in[e,f)\subseteq[a,b)\cap[c,d)$. Therefore, $\B$ is a base for a topology $\T$ on $\R$, which we can define as follows:
    \[\T = \left\{\bigcup_{\alpha\in I}U_\alpha : \{U_\alpha : \alpha\in I\} \subseteq \B\right\}.\]
    
\end{proof}

\subsection*{Exercise 2.4.6(b)}
\begin{problem}
    Show that every open subset of $\R$ (in the metric topology) is $\T$-open.
\end{problem}

\begin{proof}
    Let $(a,b)\in\R$. If we now choose $N\in\N$ such that $\frac1N < b-a$, then
    \[(a,b) = \bigcup_{n=N}^\infty\left[a+\frac1n, b\right) \in \T.\]
    Now let $U$ be an open subset of $\R$, then
    \[U = \bigcup_{\alpha\in I} (a_\alpha, b_\alpha).\]
    And since each $(a_\alpha, b_\alpha)\in\T$, then $U\in\T$ as well.
    
\end{proof}

\subsection*{Exercise 2.4.6(c)}
\begin{problem}
    Show that each interval $[a,b)$ is $\T$-closed.
\end{problem}

\begin{proof}
    Let $[a,b)\subseteq\R$ and consider its complement $(-\infty,a)\cup[b,\infty)$. Then since $(-\infty,a)$ is open in the metric topology on $\R$, it is also open in $\T$. And since
    \[[b,\infty) = \bigcup_{n=1}^\infty[b,b+n),\]
    we also have $[b,\infty)$ open in $\T$. Therefore the union $(-\infty,a)\cup[b,\infty)$ is open in $\T$, giving us its complement $[a,b)$ closed in $\T$.
    
\end{proof}

\subsection*{Exercise 2.4.6(d)}
\begin{problem}
    Show that a point $t\in\R$ lies in the $\T$-closure of a subset $S$ of $\R$ if and only if there is a sequence $\seq{t_n}{n}$ in $S$ such that $t_n\geq t$ and $|t_n-t| \to 0$.
\end{problem}

\begin{proof}
    Let $S$ be a subset of $\R$ and suppose that $t$ is in the $\T$-closure of $S$. Then for all $n\in\N$ the interval $[t,t+\frac1n)$ is an open neighborhood of $t$, and therefore $[t,t+\frac1n)\cap S \ne\emptyset$. We construct the sequence $\seq{t_n}{n}$ in $S$ such that $t_n\in[t,t+\frac1n)\cap S$ for all $n\in\N$. Then $t_n\geq t$ and $|t_n-t|<\frac1n$ for all $n\in\N$, so $|t_n-t|\to0$.
    
    Now suppose that $\seq{t_n}{n}$ is a sequence in $S$ such that $t_n\geq t$ and $|t_n-t|\to0$. Let $U$ be a neighborhood of $t$, then because $\B$ is a base for $\T$, there exists some $[a,b)\in\B$ such that $t\in[a,b)\subseteq U$. Then letting $\eps=b-t$, we consider the interval $[t,t+\eps)\subseteq[a,b)$. Now since $|t_n-t|\to0$, there exists some $N\in\N$ such that $n\geq N$ implies $|t_n-t|<\eps$. Then $t_N\geq t$ and $|t_N-t|<\eps$, which implies that $t_N\in[t,t+\eps)\subseteq U$. And since $t_N\in S$, then $U\cap S\ne\emptyset$. Therefore $t$ is an adherent point to $S$ and is in the $\T$-closure of $S$.
    
\end{proof}

\subsection*{Exercise 2.4.6(e)}
\begin{problem}
    Show that a function $f$ from $(\R,\T)$ to $\R$ is continuous if and only if $f$ is continuous from the right at each $t\in\R$, that is, if and only if
    \[\lim_{\eps>0, \eps\to0}f(t+\eps) = f(t), \quad t\in\R.\]
\end{problem}

\begin{proof}
    Suppose $f:(\R,\T)\to\R$ is continuous and let $t\in\R$. Then for each $\eps>0$, there exists a neighborhood $U$ of $t$ such that $s\in U$ implies $|f(s)-f(t)|<\eps$. In particular, for each such neighborhood $U$, we can find $[a,b)\in\B$ such that $t\in[a,b)\subseteq U$. Furthermore, we can then choose the open interval $[t,b)\subseteq[a,b)$. This give us that for each $\eps>0$, $s\in[t,b)$ implies $|f(s)-f(t)|<\eps$, so $f$ is continuous from the right at $t$.
    
    Suppose now that $f$ is continuous from the right, ie,
    \[\lim_{\eps>0, \eps\to0}f(t+\eps) = f(t), \quad t\in\R.\]
    Let $t\in\R$ and let $\eps>0$. Then there exists some $\delta>0$ such that $s\in[t,t+\delta)$ implies $|f(s)-f(t)|<\eps$. And since $[t,t+\delta)$ is a neighborhood of $t$, then $f$ is continuous at $t$.
    
\end{proof}

\section*{Exercise 2.5.1}
\begin{problem}
    Show that a sequence in a Hausdorff space cannot converge to more than one point.
\end{problem}

\begin{proof}
    Suppose $X$ is a Hausdorff space and suppose $\seq{x_n}{n}$ is a sequence in $X$ converging to $x\in X$. Let $y\in X$ such that $y\ne x$. Then there exist open subsets $U$ and $V$ of $X$ such that $x\in U$, $y\in V$, and $U\cap V=\emptyset$. Now since $x_n\to x$. Then the neighborhood $U$ of $x$ contains all but finitely many terms of $\seq{x_n}{n}$. However, since $U\cap V=\emptyset$, then $V$ contains at most finitely many terms of $\seq{x_n}{n}$. Therefore, $\seq{x_n}{n}$ does not converge to $y$.
    
\end{proof}

\section*{Exercise 2.5.5}
\begin{problem}
    Let $X$ be the subspace of $\R^2$ consisting of the closed upper half-plane minus the origin:
    \[X=\{(x,y) : y\geq 0, (x,y)\ne(0,0)\}.\]
    Show that $E=\{(x,0) : x<0\}$ and $F=\{(x,0) : x>0\}$ are disjoint closed subsets of $X$. Find a continuous function $f:X\to[0,1]$ such that $f=0$ on $E$ and $f=1$ on $F$.
\end{problem}

\begin{proposition}
    $E=\{(x,0) : x<0\}$ and $F=\{(x,0) : x>0\}$ are disjoint closed subsets of $X$.
\end{proposition}

\begin{proof}
    Consider the complement of $E$ in $X$,
    \[X"-E = \{(x,y)\in X : x>0 \text{ or } y>0\},\]
    and the open subset of $\R^2$
    \[G = \{(x,y)\in\R^2 : x>0 \text{ or } y>0\}.\]
    Now since $X"-E = X\cap G$, then $X"-E$ is open in $X$, so $E$ is closed in $X$. And similarly, as $X"-F=X\cap H$ where $H$ is the open subset of $\R^2$
    \[H = \{(x,y)\in\R^2 : x<0 \text{ or } y>0\},\]
    then we have $F$ closed in $X$. And since for any $(x,0)\in E$, we have $x<0$, then $(x,0)\notin F$, so $E$ and $F$ are disjoint.
    
\end{proof}

\begin{proposition}
    The function $f:X\to[0,1]$ defined by
    \[f(x,y) = 1-\frac{\cos^{-1}\left(\frac{x}{\sqrt{x^2+y^2}}\right)}{\pi}\]
    for all $(x,y)\in X$ is continuous with $f=0$ on $E$ and $f=1$ on $F$.
\end{proposition}

\begin{proof}
    We first note that $x^2+y^2$ being continuous implies $\frac{x}{\sqrt{x^2+y^2}}$ is continuous on $X$, since $(0,0)\notin X$. And since $\frac{x}{\sqrt{x^2+y^2}}\in[-1,1]$ for all $(x,y)\in X$, then $\cos^{-1}\left(\frac{x}{\sqrt{x^2+y^2}}\right)$ is well-defined and continuous for all $(x,y)\in X$. Therefore, $f$ is continuous on $X$. And since $\cos^{-1}:[-1,1]\to[0,\pi]$, then $f:X\to[0,1]$. 
    
    Suppose $(x,y)\in E$, that is, $x<0$ and $y=0$. Then
    \begin{align*}
        f(x,y)
            &= 1-\frac{\cos^{-1}\left(\frac{x}{\sqrt{x^2+y^2}}\right)}{\pi} \\
            &= 1-\frac{\cos^{-1}\left(\frac{x}{\sqrt{x^2}}\right)}{\pi} \\
            &= 1-\frac{\cos^{-1}\left(-1)\right)}{\pi} \\
            &= 1-\frac{\pi}{\pi} \\
            &= 0.
    \end{align*}
    And if $(x,y)\in F$, that is, $x>0$ and $y=0$, then
    \begin{align*}
        f(x,y)
            &= 1-\frac{\cos^{-1}\left(\frac{x}{\sqrt{x^2}}\right)}{\pi} \\
            &= 1-\frac{\cos^{-1}\left(1)\right)}{\pi} \\
            &= 1-\frac{0}{\pi} \\
            &= 1.
    \end{align*}
 
\end{proof}


\section*{Exercise 2.5.8}
\begin{problem}
    Let $\T$ be the half-open interval topology for $\R$, defined in Exercise 2.4.6. Show that $(\R,\T)$ is a $T_4$-space.
\end{problem}

\begin{proof}
    Let $\{x\}\subseteq\R$, and consider it's complement $(-\infty,x)\cup(x,\infty)$. Since $(-\infty,x)$ and $(x,\infty)$ are open in the metric topology, they are also open in the half-open interval topology, as is their union. Therefore $\{x\}$ is closed so the half-open interval topology on $\R$ is a $T_1$-space.
    
    Now suppose $E$ and $F$ are disjoint subsets of $\R$ which are closed in the half-open interval topology. Then each $E$ and $F$ are constructed of intervals of the following forms:
    \[[a,b), [a,\infty), (-\infty,b), [a,b], (-\infty,b].\]
    Any subset of $\R$ which is closed in the half-open interval topology is a union of some disjoint collection of the above intervals. (Note that the singleton $\{x\}\subset\R$, which is closed in the half-open interval topology is obtained by the interval $[x,x]$). We will now construct the $\T$-open subsets $U$ and $V$, which will contain $E$ and $F$, respectively, as well as be disjoint, i.e, $U\cap V=\emptyset$. We will first construct $U$ for $E$, and the construction for $V$ for $F$ is identical.
    
    For each interval $[a,b)\subseteq E$, then since $[a,b)$ is also $\T$-open, $[a,b)\subseteq U$. Similarly, for any intervals $[a,\infty)\subseteq E$ or $(-\infty, b)\subseteq E$, we take $[a,\infty)\subseteq U$ or $(-\infty, b)\subseteq U$, respectively, as they are also $\T$-open. 
    
    For each interval $[a,b]\subseteq E$, we let $c=\min(\{b+1\}\cup\{x\in F : b<x\})$. This minimum is well-defined as the set is bounded below by $b$ and every interval which is bounded below and $\T$-closed is closed on the left. Additionally, if no points in $F$ fall to the right of $b$, then $c=b+1$. Now since this definition of $c$ gives us $[a,b]\subseteq[a,c)$ and $[a,c)\cap F=\emptyset$, we take $[a,c)\subseteq U$. And for each interval $(-\infty,b]\subseteq E$, we define $c$ in the same way and take $(-\infty,c)\subseteq U$.
    
    Following the same construction for $V$ with $F$, we obtain the $\T$-open subsets $U$ and $V$ such that $E\subseteq U$ and $F\subseteq V$. Note that for every interval $e\subseteq E$, the corresponding interval $u\subseteq U$ was defined such that $\min e = \min u$. This ensures that for every interval $[a,b]\subseteq E$, the construction of the corresponding interval $[a,c)\subseteq U$ defines $c$ to exclude not only the following interval of $F$, but also the corresponding interval of $V$. And the same reasoning for $V$ gives us $U\cap V = \emptyset$. So $\T$ is normal, and therefore, a $T_4$-space.
    
\end{proof}

\section*{Exercise 2.5.9}
\begin{problem}
    Let $\B$ be the collection of subsets of $\R$ of the form $(a,b)$, together with those of the form $(a,b)\cap\R_0$, where $-\infty<a<b<\infty$ and $\R_0$ is the set of rational numbers. Prove the following:
\end{problem}

\subsection*{Exercise 2.5.9(a)}
\begin{problem}
    $\B$ is a base of the open sets for a topology $\T$ for $\R$.
\end{problem}

\begin{proof}
    Let $x\in\R$, then $x\in(x-1,x+1)\in\B$. Now suppose $U,V\in|B$ with $x\in U\cap V$. By definition of $\B$, $U$ is of the form $(a,b)$ or $(a,b)\cap\R_0$, and similarly, $V$ is of the form $(c,d)$ or $(c,d)\cap\R_0$. We define $e=\max\{a,c\}$ and $f=\min\{b,d\}$. If both $U,V$ are real intervals, then we have $(e,f)\in\B$ and $x\in(e,f) \subseteq U\cap V$. Otherwise if $U$ or $V$ is a rational interval, then $x\in\R_0$, and we have $(e,f)\cap\R_0\in\B$ and $x\in(e,f)\cap\R_0 \subseteq U\cap V$. Therefore, $\B$ is a base for some topology $\T$ on $\R$. 
    
\end{proof}

\subsection*{Exercise 2.5.9(b)}
\begin{problem}
    $(\R,\T)$ is a Hausdorff space.
\end{problem}

\begin{proof}
    Suppose $x,y\in\R$ such that $x\ne y$. Then letting $z=(x+y)/2$, we have the $\T$-open subsets $(x-1,z),(z,y+1)$ such that $x\in(x-1,z)$, $y\in(z,y+1)$, and $(x-1,z)\cap(z,y+1)=\emptyset$. 
    
\end{proof}

\subsection*{Exercise 2.5.9(c)}
\begin{problem}
    $\R"-\R_0$ is $\T$-closed.
\end{problem}

\begin{proof}
    Let $x\in\R_0$, then we have $(x-1,x+1)\cap\R_0\in\B$ such that $x\in(x-1,x+1)\cap\R_0\in\B$. Therefore $\R_0$ is $\T$-open, so its complement $\R"-\R_0$ is $\T$-closed.
    
\end{proof}

\subsection*{Exercise 2.5.9(d)}
\begin{problem}
    If $f:(\R,\T)\to\R$ is a continuous function such that $f=0$ on $\R"-\R_0$, then $f=0$ everywhere.
\end{problem}

\begin{proof}
    Suppose for contradiction that $x\in\R_0$ with $f(x)\ne0$. Then by the continuity of $f$, there exists an open neighborhood $U$ of $x$ such that $f(U)\subseteq (a,b)$, where $0<a<f(x)<b$. By the definition of $\T$, there exists some $(x-\eps,x+\eps)\cap\R_0 \in \B$ such that $(x-\eps,x+\eps)\cap\R_0 \subseteq U$. By the density of the irrationals in the reals, we pick $y\in\R"-\R_0$ such that $y\in(x-\eps,x+\eps)$. Then $f(y)=0$ and by the continuity of $f$, there exists some open neighborhood $V$ of $y$ such that $f(V)\subseteq (-a,a)$. Then there exists some $(y-\delta,y+\delta)\in\B$ such that $(y-\delta,y+\delta)\subseteq V$. Then there must exists some $z\in(x-\eps,x+\eps)\cap\R_0 \cap(y-\delta,y+\delta)$, which implies that $f(z)\in(a,b)$ and $f(z)\in(-a,a)$, which is a contradiction.
\end{proof}

\subsection*{Exercise 2.5.9(e)}
\begin{problem}
    $(\R,\T)$ is not regular.
\end{problem}

\begin{proof}
    Suppose for contradiction that $(\R,\T)$ is regular, and consider the closed set $\R"-\R_0$ and the point $0\in\R$. Then there exist open subsets $U,v\subseteq\R$ such that $x\in U$, $\R"-\R_0\subseteq V$, and $U\cap V = \emptyset$. Then there exists some open subset $(a,b)\cap\R_0\subseteq U$ such that $x\in(a,b)\cap\R_0$. We now choose $y\in(a,b)$ such that $y\in\R"-\R$. Then there exists some open subset $(c,d)\subseteq V$ such that $y\in(c,d)$. And since $(a,b)\cap\R_0\cap(c,d)\ne\emptyset$, we have $U\cap V\ne \emptyset$, which is a contradiction.
    
\end{proof}


\end{document}